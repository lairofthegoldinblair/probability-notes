\chapter{Independence}
\begin{defn}Given a measure space $(\Omega, \mathcal{A}, P)$, a set $T$ and a
  collection of $\sigma$-algebras $\mathcal{F}_t$ for $t \in T$, we
  say that the $\mathcal{F}_t$ are \emph{k-ary independent} if for every
  finite subset $t_1,\dots, t_n \in T$ with $n \leq k$ and every $A_{t_i} \in
  \mathcal{F}_{t_i}$ we have $\probability{A_{t_1} \cap \cdots \cap A_{t_n}} =
  \probability{A_{t_1}}\cdots\probability{A_{t_n}}$.  We say that
  $\mathcal{F}_t$ are \emph{independent} if the
  $\mathcal{F}_t$ are k-ary independent for every $k>0$.  It is common
  to refer to independent events as \emph{jointly independent}
  or \emph{mutually independent}
  events when it is desirable to provide emphasis that we are not
  considering k-ary independence for some particular value of k.  Furthermore, 2-ary independent events are often referred to
  as \emph{pairwise independent} events.
\end{defn}
\begin{defn}Given a probability space $(\Omega, \mathcal{A}, P)$, a set $T$ and a
  collection of random elements $\xi_t : (\Omega, \mathcal{A}) \to
  (S_t, \mathcal{S}_t)$ for $t \in T$, we
  say that the $\xi_t$ are \emph{independent} if the $\sigma$-algebras
  $\sigma(\xi_t)$ are independent.
\end{defn}
\begin{examp}Given two sets $A,B \in \mathcal{A}$ it is easy to see that $\sigma(A)$
  and $\sigma(B)$ are independent if and only if $\probability{A \cap B} =
  \probability{A}\cdot \probability{B}$ thus the notion of
  independence of $\sigma$-algebras generalizes the
  simple notion of independence from elementary probability.
\end{examp}
\begin{examp}Consider the space of triples
  $\{ (0,0,0),(0,1,1),(1,0,1),(1,1,0) \}$ with a uniform distribution.  Let
  $\xi_1,\xi_2,\xi_3$ be the coordinate functions.  Note that each of
  $\xi_i$ is uniformly distributed and that each joint distribution
  $(\xi_i,\xi_j)$ for $i \neq j$ is uniformly distributed as well.
  This shows that the $\xi_i$ are pairwise independent.  On the other
  hand, note that joint distribution $(\xi_1,\xi_2,\xi_3)$ is also
  uniformly distributed hence does not equal the product of the
  marginal distributions hence the $\xi_i$ are not jointly
  independent.  Intuitively the source of the dependence is clear; we
  have arranged the sample space so that specifying two coordinate
  values determines the value of the third coordinate.  Note this
  example can also be framed in a more elementary way in terms of
  events.  Consider the events $A_1 = \{(0,0,0),(0,1,1)\}$, $A_2 =
  \{(0,0,0),(1,0,1)\}$ and $A_3 = \{(0,1,1),(1,0,1)\}$.  Note that the
  events are pairwise independent but not independent.
\end{examp}

\begin{lem}\label{IndependenceProductMeasures}Suppose we are given a finite collection of
  random elements $\xi_1, \dots, \xi_n$ in measurable spaces
  $S_1, \dots, S_n$ with distributions $\mu_1, \dots, \mu_n$.  The
  $\xi_i$ are independent if and only if the distribution of $(\mu_1,
  \dots, \mu_n)$ on $S_1 \times \cdots \times S_n$ is $\mu_1 \otimes
  \cdots \otimes \mu_n$.
\end{lem}
\begin{proof}If we assume that joint distribution of $\xi_i$ is $\mu_1 \otimes
  \cdots \otimes \mu_n$ the clearly $\xi_i$ are independent since 
\begin{align*}
\probability{\xi_1^{-1}(B_1) \cap \cdots \cap \xi_n^{-1}(B_n)} &=
\probability{(\xi_1, \dots, \xi_n)^{-1}(B_1 \times \cdots \times B_n)}
  \\
&=\probability{\xi_1^{-1}(B_1)} \cdots \probability{\xi_n^{-1}(B_n)}
\end{align*}

On the other hand, if we assume that the $\xi_i$ are independent the
above calculation shows that $(\pushforward{(\xi_1, \dots, \xi_n)}{P}) =
\mu_1 \otimes \cdots \otimes \mu_n$ on cylinder sets which together
with the finiteness of probability measures shows that
they are equal everywhere by the uniqueness of product measure proved
in Theorem \ref{Fubini}.
\end{proof}

Having proven that the joint distribution of independent random
elements is a product measure we can apply Fubini's Theorem to compute
expectations of functions of independent random elements as iterated
integrals.  We make that statement precise in the following result.
Note that there is an important generalization of this fact that
eliminates the assumption of independence but requires the development
of the notion of a conditional distribution (see Theorem
\ref{Disintegration}).  The result is much simpler than the notation
required to state it.

\begin{lem}\label{DisintegrationIndependentLaws}Let $\xi$ and $\eta$ be
  independent random elements in measurable spaces $(S, \mathcal{S})$
  and $(T, \mathcal{T})$ respectively.  Let $f : S \times T \to
  \reals$ be a measurable function and define 
  $g(s) = \expectation{f(s, \eta)}$ and 
  $h(s) = \expectation{\abs{f(s, \eta)}}$. Suppose that either $f$ is
  non-negative or $h(\xi)$ is integrable, then
$\expectation{f(\xi, \eta)} = \expectation{g(\xi)} =
\expectation{\expectation{f(s, \eta)}\mid_{s=\xi}} $.
\end{lem}
\begin{proof}
Let $\mu$ be the distribution of $\xi$ and $\nu$ be the distribution
of $\eta$; by Lemma \ref{IndependenceProductMeasures} we know that the
joint distribution of $(\xi, \eta)$ is $\mu \otimes \nu$.  Suppose
that $f$ is non-negative and use the Expectation Rule (Lemma \ref{ExpectationRule}) and
Tonelli's Theorem \ref{Fubini} to calculate
\begin{align*}
\expectation{f(\xi, \eta)} &= \int f(x,y) \, d(\mu \otimes \nu)(x,y)
\\
&=
\int \left [ \int f(x, y) \, d\nu(y) \right ] \, d\mu(x) = \int g(x)
\, d\mu(x) = \expectation{g(\xi)}
\end{align*}

If instead assuming $f \geq 0$, we assume that $h(\xi)$ is integrable.
Applying the result just proven for the non-negative case to $\abs{f}$
shows that in fact $\expectation{ \abs{f(\xi, \eta)}} < \infty$ so we
may replay the same argument for $f$ without the absolute value using Fubini's Theorem
\ref{Fubini} in place of Tonelli's Theorem.
\end{proof}

The fact that the joint distribution of indpendent random variables
only depends on the distribution of the underlying random variables
has the important consequence that the distribution of \emph{sums} of
independent random variables also only depends on the distribution of
the underlying random variables.  However we can actually be a bit
more precise than that.

\begin{defn}A \emph{measurable group} is a group $G$ with a
  $\sigma$-algebra $\mathcal{G}$ such that the group inverse is
$\mathcal{G}$-measurable and the group operation is $\mathcal{G}
\otimes \mathcal{G} / \mathcal{G}$-measurable.
\end{defn}
\begin{defn}Given two $\sigma$-finite measures $\mu$ and $\nu$ on a measurable group
  $(G, \mathcal{G})$, the \emph{convolution} $\mu * \nu$ is the measure on $G$ defined
  by taking the pushforward of $\mu \otimes \nu$ under the group operation.
\end{defn}
\begin{lem}\label{Convolution}Convolution of measures on a measurable group $(G,
  \mathcal{G})$ is associative.  Furthermore, if $G$ is Abelian, then
  convolution of measures is commutative and we have the formula
\begin{align*}
\mu * \nu (B) &= \int \mu(B - g) \, d \nu(g) = \int \nu(B - g) \,  d \mu(g)
\end{align*}
\end{lem}
\begin{proof}First we derive the formula for the convolution of two
  measures as integrals.
Suppose we are given
  $\sigma$-finite measures $\mu, \nu$ and  a measurable
  $A \in \mathcal{G}$.  Define $A^2 = \lbrace(g,h) \mid gh \in
  A\rbrace$ and then the definition of the pushforward of a
  measure,the construction of product measure and Tonelli's Theorem we get
\begin{align*}
\left(\mu*\nu \right)(A) &= \left(\mu \otimes \nu\right) (A^2) \\
&= \int
\int \characteristic{A^2}(g,h) \, d(\mu\otimes \nu)(g,h) \\
&= \int \left [
\int \characteristic{A^2}(g,h) \, d \mu(g) \right ]
d\nu(h) \\
&= \int \left [
\int \characteristic{A^2}(g,h) \, d\nu(h) \right ]
d \mu(g) \\
\end{align*}
Now consider the inner integral for a fixed $h \in G$ and define for
each such fixed $h$ the right translation $Ah^{-1}$ and note that as a
function of $g$ alone, $\characteristic{A^2}(g,h) =
\characteristic{Ah^{-1}}(g)$.  Similarly, for fixed $g$ we introduce
the left translation $g^{-1}A$ and have  $\characteristic{A^2}(g,h) =
\characteristic{g^{-1}A}(h)$. Substituting into the integrals above, 
\begin{align*}
\left(\mu*\nu \right)(A) &= \int \mu(A\cdot g^{-1}) \, d \nu(g) = \int
\nu(g^{-1}\cdot A) \, d \mu(g)
\end{align*}
In particular, if $G$ is Abelian then $g^{-1}\cdot A = A\cdot g^{-1}$ and we have
the formula above.

To see the associativity is an application of Tonelli's
  Theorem with a bit of messy notation.  Suppose we are given
  $\sigma$-finite measures $\mu_1, \mu_2, \mu_3$ and  a measurable
  $A \in \mathcal{G}$.  Define $A^3 = \lbrace(g,h,k) \mid ghk \in
  A\rbrace$ and note that for fixed $h,k$ we have
  $\characteristic{A^3}(g,h,k) = \characteristic{Ak^{-1}h^{-1}}(g)$
  and for fixed $g,h$ we have $\characteristic{A^3}(g,h,k) =
  \characteristic{k^{-1}g^{-1}A}(k)$ 
Now applying this observation and the integral formula above
\begin{align*}
\left((\mu_1 * \mu_2)*\mu_3 \right)(A) &= \int (\mu_1 *
\mu_2)(Ak^{-1}) \, d\mu_3(k) \\
&= \int \int \mu_1(Ak^{-1}h^{-1}) \, d\mu_2(h) d\mu_3(k) \\
&= \int \int \int \characteristic{A^3}(g,h,k) \, d \mu_1(g) d\mu_2(h) d\mu_3(k) \\
&= \int \int \int \characteristic{A^3}(g,h,k) \, d\mu_3(k) d\mu_2(h) d \mu_1(g) \\
&= \int \int \mu_3(h^{-1} g^{-1} A) \, d\mu_2(h) d\mu_1(g) \\
&= \int  (\mu_2 *\mu_3)(g^{-1} A) \, d\mu_1(g) \\
&= \left(\mu_1 *  (\mu_2 *\mu_3) \right )(A)
\end{align*}
\end{proof}
\begin{defn}A measure $\mu$ on a measurable group
  $(G, \mathcal{G})$ is said to be \emph{left invariant} if for every
  $g \in G$ and $A \in \mathcal{G}$, $\mu(g\cdot A) = \mu(A)$.  A
  measure is said to be \emph{right invariant} if for every
  $g \in G$ and $A \in \mathcal{G}$, $\mu(A\cdot g) = \mu(A)$.  A
  measure that is both right invariant and left invariant is said to
  be \emph{invariant}.
\end{defn}
\begin{lem}\label{ConvolutionDensity}Let $\lambda$ be an invariant measure on a measurable Abelian group
  $(G, \mathcal{G})$ and let $\mu = f \cdot \lambda$ and $\nu = g
  \cdot \lambda$ be measures which have densities with respect to
  $\lambda$.  Then $\mu * \nu$ has the $\lambda$-density
\begin{align*}
(f * g)(x) &= \int f(x-y) g(y) \, d\lambda(y)
\end{align*}
\end{lem}
\begin{proof}
By the integral formula for convolution, given $A \in \mathcal{G}$,
\begin{align*}
(\mu * \nu)(A) &= \int \mu(A - y) \, d \nu(y) \\
&= \int \int\characteristic{A - y}(x) f(x) g(y) \, d\lambda(x)
d\lambda(y) \\
&= \int \int\characteristic{A}(x+y) f(x) g(y) \, d\lambda(x)
d\lambda(y) \\
&= \int \int\characteristic{A}(x) f(x-y) g(y) \, d\lambda(x)
d\lambda(y) \\
&= \int \characteristic{A}(x) \left [ \int f(x-y) g(y) \, d\lambda(y)
  \right ] d\lambda(x) \\
&= ((f*g) \cdot \lambda)(A)
\end{align*}
\end{proof}
\begin{examp}Let $\xi$ and $\eta$ be independent $N(0,1)$ random
  variables.  Then $\xi + \eta$ is an $N(0,2)$ random variable.  From
  Corollary \ref{ConvolutionDensity}, we
  know $\xi + \eta$ has density given by the convolution of Gaussian
  densities.
\begin{align*}
\frac{1}{2\pi} \int e^\frac{-(x-y)^2}{2} e^\frac{-y^2}{2} dy &=
\frac{1}{2\pi} \int e^{-(y^2 -xy + \frac{1}{2}x^2)} dy =
\frac{1}{2\pi} e^\frac{-x^2}{4} \int e^{-(y - \frac{x}{2})^2} dy =
\frac{1}{\sqrt{4\pi}} e^\frac{-x^2}{4}
\end{align*}
\end{examp}

\begin{lem}\label{IndependencePiSystem}Suppose we are given two
  $\pi$-systems $\mathcal{S}$ and $\mathcal{T}$ in a probability space
  $(\Omega, \mathcal{A}, P)$ such that
  $\probability{A \cap B} = \probability{A} \probability{B}$ for all
  $A \in \mathcal{S}$ and $B \in \mathcal{T}$.  Then
  $\sigma(\mathcal{S})$ and $\sigma(\mathcal{T})$ are independent.
\end{lem}
\begin{proof}
This is simply a pair of monotone class arguments.  First pick
arbitrary element $A
\in \mathcal{A}$.  We define $\mathcal{C} = \lbrace B \in
  \mathcal{A} \mid  \probability{A \cap B} =  \probability{A}
  \probability{B} \rbrace$.  We claim that $\mathcal{C}$ is a
  $\lambda$-system.  First it is clear that $\Omega \in \mathcal{C}$.
  Next assume that $B,C \in \mathcal{C}$ with $C \supset B$.  Then $C
\setminus B \in \mathcal{C}$ because
\begin{align*}
\probability{A\cap ( C \setminus B)} &= \probability{ (A \cap C)
  \setminus (A \cap B)} \\
&=\probability{ A \cap C}
  - \probability{A \cap B} \\
&=\probability{ A} \probability{C}
  - \probability{A} \probability{B} \\
&=\probability{A} \left(\probability{C}
  - \probability{B} \right) =  \probability{ A}
\probability{C\setminus B}\\
\end{align*}
Next assume that $B_1 \subset B_2 \subset \cdots$ with $B_i \in
\mathcal{C}$. We have $\cup_{n=1}^\infty B_n \in \mathcal{C}$ by the calculation
\begin{align*}
\probability{A\cap\cup_{n=1}^\infty B_n} &=
\probability{\cup_{n=1}^\infty A \cap B_n} & & \text{by DeMorgan's
  Law} \\
&=\lim_{n \to \infty} \probability{ A \cap B_n} & & \text{by
  Continuity of Measure} \\
&=\lim_{n \to \infty} \probability{ A} \probability{B_n} & &
\text{since $B_n \in \mathcal{C}$}\\
&=\probability{ A} \lim_{n \to \infty} \probability{B_n} \\
&=\probability{A} \probability{\cup_{n=1}^\infty B_n} & & \text{by
  Continuity of Measure}\\
\end{align*}

Our assumption is that if we pick $A \in \mathcal{S}$, then $\mathcal{T} \subset \mathcal{C}$ so the
$\pi$-$\lambda$ Theorem (Theorem \ref{MonotoneClassTheorem}) shows
that $\sigma(\mathcal{T}) \subset \mathcal{C}$.  Since our choice of
$A \in \mathcal{S}$ can be arbitrary, we know for every for every $A\in \mathcal{S}$ and
every $B \in \sigma(\mathcal{T})$ we have $\probability{A \cap B} =
\probability{A} \probability{B}$.  

It remains to extend $\mathcal{S}$ to $\sigma(\mathcal{S})$.  This is
done in exactly the same way.   Pick a $B \in \sigma(\mathcal{T})$ and define $\mathcal{D} \lbrace A \in
  \mathcal{A} \mid  \probability{A \cap B} =  \probability{A}
  \probability{B} \rbrace$.  We have shown that  $\mathcal{D}$ is a
  $\lambda$-system and that $\mathcal{S} \subset \mathcal{D}$ hence
  the $\pi$-$\lambda$ Theorem gives us $\mathcal{D} \supset
  \sigma(\mathcal{S})$.  Since $B\in \sigma(\mathcal{T})$ was
  arbitrary we have shown independence of $\sigma(\mathcal{S})$ and $\sigma(\mathcal{T})$.
\end{proof}

\begin{lem}\label{IndependenceGrouping}Let $\mathcal{A}_t$ for $t \in
  T$ be an independent family of $\sigma$-algebras on $\Omega$.  The
  for any disjoint partition $\mathcal{T}$ of $T$ we have
  $\sigma(\bigcup_{s \in S} \mathcal{A}_s)$ are independent where $S
  \in \mathcal{T}$.
\end{lem}
\begin{proof}
For $S$ and element of the partition of $T$, let $\mathcal{C}_S$ be
the set of all finite intersections of elements from $\cup_{s \in S}
\mathcal{A}_s$.  Clearly each $\mathcal{C}_S$ is a $\pi$-system that
generates $\sigma (\bigcup_{s \in S} \mathcal{A}_s)$.  Moreover,
the independence of the $\mathcal{A}_t$ for all $t \in T$ shows that
the $\mathcal{C}_S$ are independent $\pi$-systems by
associativity of finite intersection of sets and multiplication in $\reals$.
Thus Lemma \ref{IndependencePiSystem} shows the result.
\end{proof}

In order to prove independence of a countable collection of
$\sigma$-algebras it can be useful to reduce the task to showing a
sequence of pairwise independent relationships as in the following
Lemma.
\begin{lem}\label{IndependenceByPairwiseIndependence}Let
  $\mathcal{A}_1, \mathcal{A}_2, \dotsc$ be $\sigma$-algebras, then
  they are independent if and only if $\bigvee_{k=1}^n \mathcal{A}_k$
  is independent of $\mathcal{A}_{n+1}$ for all $n \geq 1$.
\end{lem}
\begin{proof}
The only if direction is an application of Lemma
\ref{IndependenceGrouping}.  The if direction will be shown by
induction.  To set notation, suppose that $A_{k_1} \in
\mathcal{A}_{k_1}, \dotsc, A_{k_m} \in \mathcal{A}_{k_m}$ are chosen
and we must show $\probability{A_{k_1} \cap \dotsb \cap A_{k_m}} = 
\probability{A_{k_1} } \dotsb \probability{ A_{k_m}} $ where without
log of generality we assume $1 \geq k_1 < \dotsb < k_m$.  If we let $n
= k_1 \vee \dotsb \vee k_m=k_m$ the induction variable is $n$.  The case
of $n=1$ is trivial as there is nothing to prove, so suppose the
result is true for $n-1$ and we are given $A_{k_1} \in
\mathcal{A}_{k_1}, \dotsc, A_{k_m} \in \mathcal{A}_{k_m}$ with $k_m =
n$.  Using the hypothesis, the fact that $k_{m-1} < n$ and induction
hypothesis we know that 
\begin{align*}
\probability{A_{k_1} \cap \dotsb \cap A_{k_m}} &= \probability{A_{k_1}
  \cap \dotsb \cap A_{k_{m-1}}} \probability{A_{k_m}} \\
&=\probability{A_{k_1}} \dotsb \probability{A_{k_m}}
\end{align*}
and the result is proven.
\end{proof}

Note that the previous lemma can be taken as demonstrating that
independence of sets cannot be destroyed by applying the operations of
complementation, countable union and countable intersection.  The
property of independence is also very robust in the sense that it
cannot be destroyed by composition with any measurable mapping.
\begin{lem}\label{IndependenceComposition}A
  finite collection of random elements $\xi_1, \dots, \xi_n$ in
  measurable spaces $(S_1,\mathcal{S}_1), \cdots, (S_n,
  \mathcal{S}_n)$ is independent if and only $f_1 \circ \xi_1, \cdots,
  f_n \circ \xi_n$ is independent for every measurable $f_1, \cdots, f_n$.
\end{lem}
\begin{proof}The reverse implication is clear because the identity on
  every $(S_i, \mathcal{S}_i)$ is measurable.

Now if $\xi_i$ are independent then by definition $\sigma(\xi_i)$ are
independent $\sigma$-algebras.  But for any measurable $f_i$, $\sigma(f_i \circ \xi_i) \subset
\sigma(\xi_i)$ and therefore the $f_1 \circ \xi_1, \cdots,
  f_n \circ \xi_n$ are independent.
\end{proof}

Implicit in a few of the above proofs is the fact that independence
among groups of independent objects can be reduced to checking
independence of finite subsets within the groups.  Here is a
codification of this fact stated in the simple case of checking
pairwise independence.
\begin{lem}\label{IndependenceFinitary}Let $\mathcal{F}_t$ and
  $\mathcal{G}_s$ be sets of $\sigma$-algebras. Then
  $\sigma(\bigcup_{t \in T} \mathcal{F}_t)$ is independent of
  $\sigma(\bigcup_{s \in S} \mathcal{G}_s)$ if an only if for every
  finite subset $T^\prime \subset T$ and $S^\prime \subset S$, we have $\sigma(\bigcup_{t \in T^\prime} \mathcal{F}_t)$ is independent of
  $\sigma(\bigcup_{s \in S^\prime} \mathcal{G}_s)$
\end{lem}
\begin{proof}
One direction of this is trivial.  For the other direction suppose we
have independence over each of the finite subsets.  To prove the
result note that set of finite intersections of elements of
$\bigcup_{t \in T} \mathcal{F}_t$
is a $\pi$-system that generates $\sigma(\bigcup_{t \in T}
\mathcal{F}_t)$ (and similarly with $S$).  Our assumption tells us
that these $\pi$-systems are independent hence we appeal to Lemma \ref{IndependencePiSystem}.
\end{proof}

\begin{lem}\label{IndependenceExpectations}A
  finite collection of random elements $\xi_1, \dots, \xi_n$ in
  measurable spaces $(S_1,\mathcal{S}_1), \cdots, (S_n,
  \mathcal{S}_n)$ is
  independent if and only if 
\begin{align*}
\expectation{f_1(\xi_1) \cdots f_n(\xi_n)} =
  \expectation{f_1(\xi_1)} \cdots \expectation{f_n(\xi_n)}
\end{align*}
for all $f_i : S_n \to \reals$ that are either bounded
measurable or positive measurable.
\end{lem}
\begin{proof}
Note that for the special case $f_i = \characteristic{A_i}$ for 
Borel sets $A_i \in \mathcal{B}(\reals)$, $f_i(\xi_i) =
\characteristic{f_i^{-1}(A_i)}$ and therefore the claim is equivalent to
the definition of independence as we can see by the following
calculation
\begin{align*}
\expectation{f_1(\xi_1) \cdots f_n(\xi_n)} &=
\expectation{\characteristic{f_1^{-1}(A_1)} \cdots
  \characteristic{f_n^{-1}(A_n)}} \\
&= \probability{f_1^{-1}(A_1) \cap \cdots \cap f_n^{-1}(A_n)} \\
&= \probability{f_1^{-1}(A_1) } \cdots \probability{f_n^{-1}(A_n)} \\
&=\expectation{f_1(\xi_1)} \cdots \expectation{f_n(\xi_n)}
\end{align*}
Therefore if we assume the result for all positive or bound measurable
$f$ then we certainly have independence.  

On the other hand if we assume independence of the $\xi_i$ then we
know that the desired result holds for $f_i$ that are indicator
functions.  It remains to apply the standard machinery to derive the
result for more general $f_i$.

For $f_i$ simple functions we simply use linearity of expectation.  If
we write $f_i = c_{1,i} \characteristic{A _{1_i,i}} + \cdots + c_{m_i,i}
\characteristic{A _{m_i, i}}$ then 
\begin{align*}
\expectation{f_1(\xi_1) \cdots f_n(\xi_n)} &= \sum_{k_1 = 1}^{m_1}
\cdots \sum_{k_n = 1}^{m_n} c_{k_1, 1} \cdots c_{k_n, n}
\expectation{\characteristic{A_{k_1,1}}(\xi_1)
  \cdots\characteristic{A_{k_n,n}}(\xi_n)} \\
&= \sum_{k_1 = 1}^{m_1}
\cdots \sum_{k_n = 1}^{m_n} c_{k_1, 1} \cdots c_{k_n, n}
\expectation{\characteristic{A_{k_1,1}}(\xi_1)} \cdots
\expectation{\characteristic{A_{k_n,n}}(\xi_n)} \\
&= \sum_{k_1 = 1}^{m_1} c_{k_1, 1} \expectation{\characteristic{A_{k_1,1}}(\xi_1)} 
\cdots \sum_{k_n = 1}^{m_n} c_{k_n, n}
\expectation{\characteristic{A_{k_n,n}}(\xi_n)} \\
&=\expectation{f_1(\xi_1)} \cdots \expectation{f_n(\xi_n)}
\end{align*}

To show the result for positive $f$, first start by assuming that
$f_1$ is positive and $f_2, \cdots, f_n$ are simple.  Pick $f_{i,1}$
increasing simple functions such that $f_{i,1} \uparrow f_1$.  Then we
have $f_{i,1} f_2 \cdots f_n \uparrow f_1 f_2 \cdots f_n$ we have
\begin{align*}
\expectation{f_1(\xi_1) \cdots f_n(\xi_n)} &= \lim_{i\to
  \infty}\expectation{f_{i,1}(\xi_1) \cdots f_n(\xi_n)} & & \text{by
  Monotone Convergence} \\
&=\lim_{i\to
  \infty}\expectation{f_{i,1}(\xi_1)} \cdots \expectation{f_n(\xi_n)}
& & \text{result for simple functions} \\
&=\expectation{f_1(\xi_1)} \cdots \expectation{f_n(\xi_n)} & &\text{by
  Monotone Convergence}
\end{align*}
Having shown the result for $f_1$ positive and $f_2, \cdots, f_n$
simple just iterate with Monotone Convergence as above to see the result for all
$f_1, \cdots, f_n$ positive.

For $f_i$ bounded, first write $f_1 = f_1^+ - f_1^-$ with $f_1^\pm
\geq 0$ and bounded and assume that $f_2, \dots, f_n$ are positive and
bounded.  Note that
$f_1^\pm \circ \xi$ is integrable by the boundedness of $f_1^\pm$.
Therefore by linearity of expectation and the fact that we have proven
the result for positive $f_i$
\begin{align*}
\expectation{f_1(\xi_1) f_2(\xi_2) \cdots f_n(\xi_n)} &=
\expectation{f_1^+(\xi_1) f_2(\xi_2) \cdots f_n(\xi_n)} -
\expectation{f_1^-(\xi_1) f_2(\xi_2) \cdots f_n(\xi_n)} \\
&=\expectation{f_1^+(\xi_1)}\expectation{f_2(\xi_2)} \cdots
\expectation{f_n(\xi_n)} \\ 
&-\expectation{f_1^-(\xi_1)}\expectation{f_2(\xi_2)} \cdots
\expectation{f_n(\xi_n)} \\
&=\expectation{f_1(\xi_1)}\expectation{f_2(\xi_2)} \cdots
\expectation{f_n(\xi_n)} 
\end{align*}
Now perform induction on $i$ to get the final result.
\end{proof}

\begin{examp} TODO:  Find an example where this fails for integrable
  $f$.  I'm pretty sure the crux is to find $f$ that is integrable for which $f \circ
  \xi$ is not.  In any case if one finds such a pair, then the result
  doesn't really even make sense since not all of the expectations are defined.
\end{examp}

\begin{cor}Suppose $f,g$ are independent integrable random variables
 then $fg$ is integrable and $\expectation{fg} = \expectation{f}\expectation{g}$.
\end{cor}
\begin{proof}By Lemma
  \ref{IndependenceExpectations}, independence of $f,g$ and positivity and measurability of $\abs{x}$, we see that 
\begin{align*}\expectation{\abs{fg}} &= \expectation{\abs{f}\cdot \abs{g}} =
  \expectation{\abs{f}}\expectation{\abs{g}} < \infty
\end{align*}
showing integrability of $f g$.

This argument also shows that $\expectation{fg} =
\expectation{f}\expectation{g}$ for positive $f,g$.  To extend to to
integrable $f,g$ write $f = f_+ - f_-$ and $g = g+ - g_-$ and use
linearity of expectation
\begin{align*}
\expectation{fg} &= \expectation{f_+ g_+} - \expectation{f_+ g_-} -
\expectation{f_- g_-} + \expectation{f_- g_-} \\
&= \expectation{f_+}\expectation{g_+} - \expectation{f_+}\expectation{ g_-} -
\expectation{f_-}\expectation{ g_-} + \expectation{f_-}\expectation{
  g_-} \\
&= \left( \expectation{f_+} - \expectation{f_-}\right) \left( \expectation{g_+} - \expectation{g_-}\right)\\
&= \expectation{f}\expectation{g}
\end{align*}
\end{proof}

\begin{examp}\label{UncorrelatedNotIndependent}This is an example of random variables $\xi$ and
  $\eta$ such that $\expectation{\xi \eta} =
  \expectation{\xi} \cdot \expectation{\eta}$ (are
  \emph{uncorrelated}) but $\xi$ and $\eta$ are
  not independent.

Consider the sample space $\Omega = \{1,2,3\}$ with uniform distribution.  A
random variable $\xi:\Omega \to \reals$ is just a vector in $\reals^3$.  Let $\xi = (1,-1,0)$
and let $\eta=(-1,-1,2)$.  Note that
$\expectation{\xi}=\expectation{\eta} = \expectation{\xi \eta} = 0$
and therefore $\xi$ and $\eta$ are uncorrelated.  On the other hand
$\xi$ and $\eta$ are not independent; for example 
\begin{align*}
0=\probability{\xi=1
  \wedge \eta = 2} \neq
\probability{\xi=1}\probability{\eta=2}=\frac{1}{9}
\end{align*}
\end{examp}

\begin{defn}Given a sequence of events $A_n$ the event that $A_n$
  occurs \emph{infinitely often} is the set $\bigcap_{n=1}^\infty
  \bigcup_{k=n}^\infty A_k = \limsup_{n \to \infty} A_n$.  The
  probability that $A_n$ occurs infinitely often is often written $\probability{A_n \text{ i.o.}}$.
\end{defn}

\begin{thm}\label{BorelCantelli}[Borel Cantelli Theorem]Let
  $(\Omega,\mathcal{A},P)$ be a probability space and let $A_1, A_2,
  \dots \in \mathcal{A}$.
 \begin{itemize}
\item[(i)] If $\sum_{i=1}^\infty \probability{A_i} < \infty$
  then $\probability{A_i \text{ i.o.}} = 0$.  
\item[(ii)]If the $A_i$ are
  independent and $\probability{A_i \text{ i.o.}} = 0$, then we have
  $\sum_{i=1}^\infty \probability{A_i} < \infty$.  More precisely, if
  $\sum_{i=1}^\infty \probability{A_i} = \infty$ then $\probability{A_i \text{ i.o.}} = 1$.
\end{itemize}
\end{thm}
\begin{proof}
To prove (i) we observe that the convergence of $\sum_{i=1}^\infty
\probability{A_i}$ implies that the partial sums converge to zero,
$\lim_{n\to \infty} \sum_{i=n}^\infty
\probability{A_i} = 0$.  Now we apply a union bound (subadditivity of
measure) and use continuity of measure to see that
\begin{align*}
\probability{A_n \text{ i.o.}} &= \lim_{n\to \infty}
\probability{\bigcup_{k=n}^\infty A_k}  \leq  \lim_{n\to \infty} \sum_{k=n}^\infty \probability{A_k} = 0
\end{align*}

To see (ii), first observe the simple calculation
\begin{align*}
\probability{\bigcap_{n=1}^\infty \bigcup_{k=n}^\infty A_k} &=
\lim_{n\to \infty} \probability{\bigcup_{k=n}^\infty A_k}  & &\text{ by continuity of measure}\\
&= \lim_{n\to \infty} \lim_{m\to \infty} \probability{\bigcup_{k=n}^m
  A_k} & &\text{ by continuity of measure}\\
&= \lim_{n\to \infty} \lim_{m\to \infty} \left ( 1 - \probability{\bigcap_{k=n}^m
  A_k^c} \right ) & &\text{ by DeMorgan's law} \\
&= \lim_{n\to \infty} \lim_{m\to \infty}\left ( 1 - \prod_{k=n}^m\probability{
  A_k^c} \right )& &\text{by independence} \\
&=  1 - \lim_{n\to \infty} \lim_{m\to \infty}\left (\prod_{k=n}^m\left
    (1 - \probability{
  A_k} \right) \right ) \\
\end{align*}

Now we recall the 
elementary bound $1+x \leq e^x$ for $x \in \reals$ from Lemma
\ref{BasicExponentialInequalities} and assume that
$\sum_{n=1}^\infty \probability{A_n} = \infty$.  By the calculation
above we have
\begin{align*}
\probability{\bigcap_{n=1}^\infty \bigcup_{k=n}^\infty A_k} &= 1 - \lim_{n\to \infty} \lim_{m\to \infty}\left (\prod_{k=n}^m\left
    (1 - \probability{
  A_k} \right) \right ) \\
&\geq 1 - \lim_{n\to \infty} \lim_{m\to \infty}\left (\prod_{k=n}^m
   e^{- \probability{
  A_k}} \right ) \\
&= 1 - \lim_{n\to \infty} \lim_{m\to \infty}
   e^{- \sum_{k=n}^m\probability{
  A_k}} \\
&= 1
\end{align*}
But of course we know that $\probability{\bigcap_{n=1}^\infty
  \bigcup_{k=n}^\infty A_k} \leq 1$ so in fact we have shown that
$\probability {A_n \text{ i.o.}} = 1$.
\end{proof}

\begin{examp}Here is a somewhat synthetic example that shows when
  $A_n$ are dependent it is possible to have $\probability{A_n \text{
      i.o.}} = 0$ while $\sum_{n=1}^\infty \probability{A_n} =
  \infty$.  Take $([0,1], \mathcal{B}([0,1]), \lambda$ as the measure
  space.  Take the intervals $[0, \frac{1}{n}]$ in a sequence such
  that $[0, \frac{1}{n}]$ occurs $n$ times (e.g. $[0,1],
  [0,\frac{1}{2}], [0,\frac{1}{2}], [0,\frac{1}{3}],[0,\frac{1}{3}],
  [0,\frac{1}{3}], \dots$). Clearly $\lbrace A_n \text{ i.o.} \rbrace
  = \lbrace 0 \rbrace$.  On the other hand it is clear that
  $\sum_{n=1}^\infty \probability{A_n} = \infty$.
\end{examp}

\begin{examp}This is a more probabilistic example.  Consider a game in
  which there is a $n$-sided die for each $n=2, 3, \dots$.  In the
  $n^{th}$ round of the game, one rolls the $n$-sided die.  If one
  gets a $1$ then one stops the game else one continues to play.  Let
  $A_n$ be the event that the player is still alive at round $n$.  It
  is clear that player has a probability of $\frac{1}{2} \cdots
  \frac{n-1}{n} = \frac{1}{n}$ of being alive at round $n$.  It is also clear that
  the probability the player never loses is bounded by $\frac{1}{n}$
  for all $n$ hence is 0.  The probability the player never loses is
  the same as $\probability{A_n \text{ i.o.}}$ on the other hand,
  $\sum_{n=1}^\infty \probability{A_n} = \sum_{n=1}^\infty \frac{1}{n}
  = \infty$.
\end{examp}

The Borel Cantelli Theorem tells us
that $\probability{A_n \text{ i.o.}}$ can only take the values $0$ and
$1$ when the $A_n$ are independent events (and in fact gives us a test
for determining which alternative holds).  The $0/1$ dichotomy is a
general feature of sequences of independent events and describing the
nature this dichotomy motivates the following definitions.

\begin{defn}Let $\mathcal{A}_n$ be a sequence of $\sigma$-algebras on
  a space $\Omega$.  The \emph{tail $\sigma$-algebra} $\mathcal{T}_\infty$ is
  defined to be 
\begin{align*}
\mathcal{T}_\infty = \bigcap_{n=1}^\infty \sigma\left(\bigcup_{k=n}^\infty \mathcal{A}_k\right)
\end{align*}
\end{defn}

\begin{thm}[Kolmogorov's $0-1$ Law]\label{Kolmogorov01Law}Let
  $\mathcal{A}_n$ be a sequence of independent $\sigma$-algebras on
  a probability space $(\Omega, \mathcal{A}, P)$ such that $\mathcal{A}_n
  \subset \mathcal{A}$ for all $n>0$.  Then for every $T \in \mathcal{T}_\infty$ we have
  $\probability{T}=0$ or $\probability{T} = 1$.
\end{thm}
\begin{proof}Let $\mathcal{T}_n = \sigma\left(\bigcup_{k=n}^\infty
    \mathcal{A}_k\right)$ and $\mathcal{S}_n = \sigma\left(\bigcup_{k=1}^{n-1}
    \mathcal{A}_k\right)$.  Then by Lemma \ref{IndependenceGrouping}
  we see that $\mathcal{T}_n$ and $\mathcal{S}_n$ are independent.
  Therefore for $A \in \mathcal{T}_n$ and $B \in \mathcal{S}_n$ we
  have $\probability{A \cap B} = \probability{A} \probability{B}$.

Now pick $A \in \mathcal{T}_\infty$, then by the above observation we
have $\probability{A \cap B} = \probability{A} \probability{B}$ for $B
\in \bigcup_{n=1}^\infty \mathcal{S}_n$.  Since $\mathcal{S}_1 \subset
\mathcal{S}_1 \subset \cdots$, we can easily see that
$\bigcup_{n=1}^\infty \mathcal{S}_n$ is a $\pi$-system.  Given $B_1,
B_2 \in \bigcup_{n=1}^\infty \mathcal{S}_n$ there exist $n_1,n_2$ such
that $B_i \in \mathcal{S}_{n_i}$ for $i=1,2$.  Then define $n =
\max(n_1,n_2)$ and $B_i \in \mathcal{S}_n$ for $i=1,2$ and therefore
$B_1 \cap B_2 \in \mathcal{S}_n \subset \bigcup_{n=1}^\infty
\mathcal{S}_n$.  Applying Lemma \ref{IndependencePiSystem} we conclude
that $\mathcal{T}_\infty$ and $\sigma(\bigcup_{n=1}^\infty
\mathcal{S}_n)$ are independent.  Note that for every $n>0$, $\mathcal{T}_n
\subset \sigma(\bigcup_{n=1}^\infty \mathcal{S}_n)$ hence the same is
true of their intersection $\mathcal{T}_\infty$.  We may conclude
that for any $A \in \mathcal{T}_\infty$ we have 
\begin{align*}
\probability{A} &= \probability{A \cap A} =
\probability{A}\probability{A}
\end{align*}
which shows that $\probability{A} = 0$ or $\probability{A} = 1$.
\end{proof}
Tail algebras arise naturally in various limiting processes involving
random variables.  In the case in which the random variables are
independent, the limits have various kinds of almost sure properties
that can be derived from Kolmogorov's $0-1$ Law.  Here are a few examples.
\begin{cor}Let $(S,d)$ be a complete metric space and let $\xi_n$ be a sequence of independent random elements in
  S.  Then either $\xi_n$ converges almost surely or diverges almost surely.
\end{cor}
\begin{proof}
Let $\mathcal{T}_n = \sigma\left( \cup_{k\geq n} \sigma(\xi_k)
\right)$ and let $\mathcal{T} = \cap_{n=1}^\infty \mathcal{T}_n $ be
the tail $\sigma$-algebra.  By Kolmogorov's $0-1$ Law it suffices to show that the event that
$\xi_n$ converges is $\mathcal{T}$-measurable. 
Since $S$ is complete, we know that $\xi_n$ converges if and only if
for every $\epsilon > 0$ there exists $N>0$ such that $d(\xi_m, \xi_n)
< \epsilon$.  With that in mind, for every $m>0$, $n>0$ and $\epsilon>0$ define
\begin{align*}
A_{n,m,\epsilon} = \lbrace d(\xi_m, \xi_n) < \epsilon \rbrace
\end{align*}
which is $\sigma \left(\sigma(\xi_m) \cup \sigma(\xi_n)
\right)$-measurable.

To prove convergence it suffices to demonstrate it for any sequence of
$\epsilon_k \to 0$.  So in particular if we choose $\epsilon_k =
\frac{1}{k}$ we see that the event that $\xi_n$ converges is
\begin{align*}
\bigcap_{k=1}^\infty \bigcup_{N=1}^\infty \bigcap_{m,n\geq N} A_{m,n,\frac{1}{k}}
\end{align*}
Note that each $\bigcap_{m,n\geq N} A_{m,n,\frac{1}{k}}$ is
$\mathcal{T}_N$-measurable and $A_{N+1} \subset A_N$ hence
$\bigcup_{N=1}^\infty \bigcap_{m,n\geq N} A_{m,n,\frac{1}{k}}$ is
$\mathcal{T}$-measurable.  Taking the countable union of
$\mathcal{T}$-measurable  sets we see the event that $\xi_n$ converges
is $\mathcal{T}$-measurable.  
\end{proof}

\begin{cor}\label{ConstantLimitOfIndependent}Let $\xi_n$ be a sequence of independent random variables.
  Then $\limsup_{n \to \infty} \xi_n$ and $\liminf_{n \to \infty} \xi_n$ are almost surely constant.
\end{cor}
\begin{proof}Because $\liminf_n \xi_n= -\limsup_n -\xi_n$ if suffices to
  show the result for $\limsup_n \xi_n$.   Let $\mathcal{T}$ be the tail $\sigma$-algebra of
  $\sigma(\xi_n)$ and let $\mathcal{T}_n = \sigma\left(\cup_{k\geq n}
  \sigma(\xi_k) \right)$.  By Kolmogorov's 0-1 Law, it suffices to show that $\limsup_{n \to \infty}
  \xi_n$ is $\mathcal{T}$-measurable.

By definition, $\limsup_{n \to \infty} \xi_n = \lim_{n \to \infty}
\sup_{k \geq n} \xi_k$.  The term $\sup_{k \geq n} \xi_k$ is
$\mathcal{T}_n$-measurable by \ref{LimitsOfMeasurable} and when taking the limit of the sequence
we can ignore any finite prefix of the sequence.  Therefore we can
express the limit as a limit of $\mathcal{T}_n$-measurable functions
for $n>0$ arbitrary.  This shows that $\limsup_{n \to \infty} \xi_n$
is $\mathcal{T}_n$-measurable for all $n>0$ hence $\mathcal{T}$-measurable.
\end{proof}

\begin{cor}\label{AlmostSureAverages}Let $\xi_n$ be a sequence of independent random variables.
  Then $\lim_{n \to \infty} \frac{1}{n}\sum_{k=1}^\infty \xi_k$ almost
  surely diverges or almost sure converges.  If it converges then the
  limit is  almost surely constant.
\end{cor}
\begin{proof}
Note that $\lim_{n \to \infty} \frac{1}{n}\sum_{k=1}^n \xi_k =
\lim_{n \to \infty} \frac{1}{n}\sum_{k=m}^n \xi_k$ for any $m >
0$.  Pick such an $m > 0$ and note that every finite partial sum
$\frac{1}{n}\sum_{k=m}^n \xi_k$ is $\mathcal{T}_m$-measurable hence so
is the limit $\lim_{n\to \infty} \frac{1}{n}\sum_{k=1}^n \xi_k$.
Since $m >0$ was arbitrary we know that $\lim_{n\to \infty}
\frac{1}{n}\sum_{k=1}^n \xi_k$ is $\mathcal{T}$-measurable.  
\end{proof}

The Borel Cantelli Theorem is a very useful technique in demonstrating
the almost sure convergence of sequences of random variables.  The
following simple version of the Strong Law of Large Numbers
illustrates the technique with a minimum of distractions.
\begin{lem}\label{SLLNL4}Let $\xi, \xi_1, \xi_2, \dots$ be independent identically
  distributed random variables with $\expectation{\xi^4} < \infty$,
  then $\lim_{n \to \infty} \frac{1}{n} \sum_{k=1}^n \xi_k =
  \expectation{\xi}$ a.s.
\end{lem}
\begin{proof}
First note that it suffices to show the result when $\expectation{\xi}
= 0$ since we can just compute
\begin{align*}
0 = \lim_{n \to \infty} \frac{1}{n} \sum_{k=1}^n (\xi_k -
\expectation{\xi}) = 
\lim_{n \to \infty} (\frac{1}{n} \sum_{k=1}^n \xi_k) -
\expectation{\xi}
\end{align*}
Furthermore by Corollary \ref{IncreasingMoments} the finite $4^{th}$ moment of $\xi$ implies finiteness of
the first four moments, hence $\expectation{\left (\xi -
    \expectation{\xi}\right )^4} <
\infty$.

Now assuming that $\xi_k$ have mean zero we fix $\epsilon > 0$ and apply Markov
bounding to see
\begin{align*}
\probability{\abs{\sum_{k=1}^n \xi_k} > n \epsilon} &=
\probability{\left ( \sum_{k=1}^n \xi_k \right ) ^4  > n^4 \epsilon^4} \\
&\leq \frac{\expectation{\left ( \sum_{k=1}^n \xi_k \right )^4}}{ n^4 \epsilon^4}
& & \text{by Markov's inequality}
\\
&= \frac{\sum_{k=1}^n\expectation{\xi_k^4}+6\sum_{k=1}^n
 \sum_{l=k+1}^n\expectation{\xi_k^2 \xi_l^2}}{ n^4 \epsilon^4} & &
\text{by independence and zero mean} \\
&= \frac{\sum_{k=1}^n\expectation{\xi_k^4}+6\sum_{k=1}^n
 \sum_{l=k+1}^n\sqrt{\expectation{\xi_k^4} \expectation{\xi_l^4}}}{ n^4 \epsilon^4} & &
\text{by Cauchy Schwartz} \\
&= \frac{\expectation{\xi^4}(n + 3(n^2 -n))}{ n^4 \epsilon^4} \leq \frac{3\expectation{\xi^4}}{ n^2 \epsilon^4} \\
\end{align*}
And therefore $\sum_{n=1}^\infty \probability{\abs{\sum_{k=1}^n \xi_k}
  > n \epsilon}  < \infty$.  Now we can apply Borel Cantelli to see
that $\probability{\frac{1}{n}\abs{\sum_{k=1}^n \xi_k
    } > \epsilon \text{ i.o.}} = 0$.

By the above argument, for every $m \in \naturals$ we get an event
$A_m$ with $\probability{A_m} = 0$ such that for every $\omega \notin
A_m$ there is $N_{\omega,m}$such that $\frac{1}{n}\abs{\sum_{k=1}^n
  \xi_k(\omega)} \leq \frac{1}{m}$ for $n > N_{\omega,m}$.  Let $A =
\cup{m=1}^\infty A_m$ and note that by countable subadditivity
$\probability{A} = 0$.  Furthermore, for every $\epsilon > 0$, $\omega
\in A$ we pick $m > \frac{1}{\epsilon}$ and then for $n > N_{\omega,
  m}$ we have $\frac{1}{n}\abs{\sum_{k=1}^n
  \xi_k(\omega)} \leq \frac{1}{m} < \epsilon$ for $n > N_{\omega,m}$
giving the result.
\end{proof}
The proof above demonstrates a general pattern in applications of
Borel Cantelli in which one applies it a countably infinite number of
times and still derive an almost sure result.  We'll prove more
refined versions of the Strong Law of Large Numbers later and those
will also use Borel Cantelli but with more complications.


It will prove to be important to be able to construct random variables
with prescribed distributions.  In particular, we will soon need to be
able to construct independent random variables with prescribed
distributions.  The standard way of constructing them is to use
product spaces, however we have only developed product spaces of
finitely many factors.  Rather than developing the full fledged theory
of infinitary products, we provide a mechanism which suffices for the
construction of countably many random variables with prescribed
distributions; in fact we show that it is possible to do so on the
probability space $([0,1], \mathcal{B}([0,1]), \lambda)$.  First
proceed by noticing that there is ready source of independence waiting
for us to harvest.  Given $x \in [0,1]$ we can take the unique binary
expansion $x = 0.\xi_1\xi_2 \cdots$ which has the property that
$\sum_{n=1}^\infty \xi_n = \infty$ (here we are resolving the ambiguity
between expansions that have a tail of $1$'s and those with a tail of
$0$'s).
\begin{lem}\label{BernoulliSequence}Let $\xi_n : [0,1] \to [0,1]$ be defined by taking the
  $n^{th}$ digit of the binary expansion of $x \in [0,1]$.  Then
  $\xi_n$ is a measurable function.  Let $\vartheta :  [0,1] \to
  [0,1]$, then $\vartheta$ has a uniform distribution if and only if
  $\xi_n \circ \vartheta$ comprise an independent sequence of
  Bernoulli random variables with probability $\frac{1}{2}$.
\end{lem}
\begin{proof}To see the measurability of $\xi_n$ we first define the
  \emph{floor function} to be $\lfloor x \rfloor = \sup \lbrace n \in
  \integers \mid n \leq x\rbrace$.  Then define
 \begin{align*}
\xi(x) &= \begin{cases}
0 & \text{if $x - \lfloor x \rfloor \in [0,\frac{1}{2})$} \\
1 & \text{if $x - \lfloor x \rfloor \in [\frac{1}{2},1)$} \\
\end{cases}
\end{align*}
It is clear that $\xi$ is a measurable function since $\xi^{-1}(0) =
\cup_n [n, n+\frac{1}{2})$ and  $\xi^{-1}(1) =
\cup_n [n+\frac{1}{2}, n+1)$.  Now define 
\begin{align*}
\xi_n(x) &= \xi(2^{n-1} x) & &\text{for $n \in \naturals$ and  $x \in \reals$}
\end{align*}
and notice that $\xi_n$ give the binary expansion of $x \in \reals$.
By measurability of $\xi$ we see that $\xi_n$ are also measurable.

Now suppose that $\vartheta$ is a $U(0,1)$ random variable on
$[0,1]$ and consider $\xi_n \circ \vartheta$.  For every $(k_1, \dots,
k_n) \in \lbrace 0,1\rbrace^n$, let $q = \sum_{j=1}^n \frac{k_j}{2^j}$ we clearly have 
\begin{align*}
\probability{\cap_{j \leq n} \lbrace \xi_j(\vartheta(x)) = k_j \rbrace}
  &= \probability{\vartheta(x) \in [q, q+\frac{1}{2^{n}})
    \rbrace} = \frac{1}{2^n}
\end{align*}
and summing over $(k_1, \dots, k_{n-1})$ we see 
\begin{align*}
\probability{\xi_n(\vartheta(x)) = k_n} &=\sum_{(k_1, \dots, k_{n-1})
  \in \lbrace 0,1\rbrace^{n-1}}\probability{\cap_{j \leq n} \lbrace \xi_j(\vartheta(x)) = k_j \rbrace}
  = \frac{1}{2}
\end{align*}
which shows that each $\xi_n \circ \vartheta$ is a Bernoulli random
variable with probability $\frac{1}{2}$.

In a similar vein, given $n_1, \dots, n_m$ and $k_{n_j} \in \lbrace 0,1 \rbrace$ , let $n = \sup(n_1, \dots,
n_m)$ for $j=1,\dots, m$ and  $A_n = \lbrace (l_1, \dots, l_n) \mid l_{n_j} = k_{n_j} \text{
  for } j=1, \dots , m\rbrace$ and we have 
\begin{align*}
\probability{ \cap_{j=1}^m \lbrace \xi_{n_j} (\vartheta(x)) =
  k_{n_j} \rbrace } &=\sum_{(k_1, \dots, k_{n})
  \in A_n}\probability{\cap_{j \leq n} \lbrace
  \xi_j(\vartheta(x)) = k_j \rbrace} \\
  &= 2^{n -m} \frac{1}{2^n} = \frac{1}{2^m}
\end{align*}
which shows that $\xi_{n_j} \circ \vartheta$ are independent.

Next, suppose that we know $\xi_n \circ \vartheta$ is an independent Bernoulli sequence with
probability $\frac{1}{2}$.  Let $\tilde{\vartheta}$ be a $U(0,1)$
random variable (e.g. $\tilde{\vartheta}(x) = x$) and then we know
from the first part of the Lemma
that $\xi_n \circ \tilde{\vartheta}$ is also a Bernoulli sequence with
probability $\frac{1}{2}$.

Because of the
independence of each the sequences and the fact that the elementwise
the two sequences have the same distribution we know that the
distribution of the sums is just the convolution of the distributions
of the terms in the sequence, hence $\sum \xi_n \circ \vartheta
\eqdist \sum \xi_n \circ \tilde{\vartheta}$.  Thus we have shown that
$\sum \xi_n \circ \vartheta$ is also $U(0,1)$.
\end{proof}

\begin{lem}\label{ReproductionOfUniform}There exist measurable functions $f_1, f_2, \dots$ on
  $[0,1]$ such
  that whenever $\vartheta$ is a $U(0,1)$ random variable, the
  sequence
  $f_n \circ \vartheta$ is a family of independent $U(0,1)$  random variables.
\end{lem}
\begin{proof}
Let $\xi_n \circ \vartheta$ denote the binary expansion of $\vartheta$
from Lemma \ref{BernoulliSequence}.  By the result of that Lemma, we
know that the $\xi_n \circ \vartheta$ are an i.i.d. sequence of
Bernoulli random variables with probability $\frac{1}{2}$.  Now choose
any bijection between $\naturals$ and $\naturals^2$ (e.g. the diagonal
mapping).  With this relabeling of of the constructed family we now
have a sequence $\xi_{n,m} \circ \vartheta$ of i.i.d. Bernoulli random
variables.  Define $f_n(x) = \sum_{m=1}^\infty
\frac{\xi_{n,m}(x)}{2^m}$ and apply Lemma \ref{BernoulliSequence} a
second time to see that each $f_n \circ \vartheta$ is a $U(0,1)$
random variable.  Furthermore, $f_n\circ \vartheta$ is $\sigma(\cup_m
\sigma(\xi_{n,m} \circ \vartheta))$-measurable so by Lemma \ref{IndependenceGrouping} we
see that the $f_n \circ \vartheta$ are independent.
\end{proof}

\begin{thm}\label{ExistenceCountableIndependentRandomVariables}For any probability measures $\mu_1, \mu_2, \dots$ on
  $(\reals, \mathcal{B}(\reals))$ there exist independent random
  variables $f_1, f_2, \dots$ on $([0,1], \mathcal{B}([0,1]),
  \lambda)$ such that $\mathcal{L}(f_n) = \mu_n$.
\end{thm}
\begin{proof}
Define $\vartheta(x) = x$ which is clearly a $U(0,1)$-random variable
on $[0,1]$ and use Lemma \ref{ReproductionOfUniform} to construct
$\vartheta_n$, a
sequence of independent $U(0,1)$ random variables.  Let $F_n$ be the
distribution function of the probability measure $\mu_n$ and let
$G_n(y) = \sup \lbrace x \in \reals \mid F(x) \geq y \rbrace$ be the generalized
inverse of $F_n$.  By the proof of Theorem \ref
{LebesgueStieltjesMeasure}, we know that $\mathcal{L}(G_n\circ
  \vartheta_n) = \mu_n$ and by Lemma \ref{IndependenceComposition} we
  know that $G_n\circ  \vartheta_n$ are still independent.
\end{proof}
