\chapter{Empirical Processes}

\section{Weak Convergence in Non Separable Spaces}

\subsection{Outer Expectations and Measurable Cover}

\begin{defn}Let $(\Omega, \mathcal{A}, P)$ be a probability space and let $f : \Omega \to [-\infty, \infty]$ (not assumed to be measurable).  Then consider the two conditions
\begin{itemize}
\item [(i)] for every measurable $g : \Omega \to [-\infty, \infty]$ with $g \geq f$ everywhere the integral $\int g \, dP$ is defined (i.e. $\int g_+ \, dP < \infty$ or $\int g_- \, dP < \infty$) 
\item [(ii)] or there exists a measurable $g : \Omega \to [-\infty, \infty]$  with $g \geq f$ everywhere (? or a.s.) such that $\int g \, dP = -\infty$ 
\end{itemize}
When (i) holds we define the \emph{outer expectation}
\begin{align*}
\oexpectation{f} &= \int^* f \, dP = \inf \left \lbrace \int g  \, dP \mid \text{ $g$ measurable, } g \geq f \right \rbrace
\end{align*}
and when (ii) holds we define $\int^* f \, dP = -\infty$.
If neither (i) nor (ii) hold then we say that the outer expectation $\int^* f \, dP$ is not defined.  Given a subset $A \subset \Omega$ we define the outer probability to be
\begin{align*}
\oprobability{A} &= \inf \lbrace \probability{B} \mid B \supset A, B \in \mathcal{A} \rbrace
\end{align*}
\end{defn}

We show how to calculate outer expectations.
\begin{defn}Let $\mathcal{J}$ be a subset of $\mathcal{L}^0(\Omega, \mathcal{A}, P, [-\infty, \infty])$ the we say that $f \in \mathcal{L}^0$ is an \emph{essential infimum of $\mathcal{J}$} if 
for every $j \in \mathcal{J}$ we have $f \leq j$ a.s. and moreover for any $g \in \mathcal{L}^0$ such that $g \leq j$ a.s. for every $j \in \mathcal{J}$ we also have $g \leq f$ a.s.
\end{defn}

\begin{prop}\label{AlmostSureUniquenessEssentialInfimum}If $f$ and $g$ are both essential infima of $\mathcal{J}$ then $f = g$ a.s.
\end{prop}
\begin{proof}
By the definition of an essential infimum we have both $f \leq g$ a.s. and $g \leq f$ a.s.
\end{proof}

\begin{thm}\label{ExistenceMeasurableCover}For any $\mathcal{J} \subset \mathcal{L}^0(\Omega, \mathcal{A}, P, [-\infty, \infty])$ and essential infimum of $\mathcal{J}$ exists.  Given $f : \Omega \to (-\infty, \infty)$ let
$\mathcal{J}_f = \lbrace j \in \mathcal{L}^0 \mid j \geq f \text{ everywhere} \rbrace$ then there is an essential infimum $f^*$ of $\mathcal{J}_f$ such that $f^* \geq f$ everywhere.  $\int f^* \, dP$ is 
defined if and only if $\int^* f \, dP$ is defined and when they are defined we have
\begin{align*}
\int f^* \, dP &= \int^* f \, dP
\end{align*}
If $f = \characteristic{A}$ for some $A \subset \Omega$ (not necessarily measurable) then $f^*$ may be chosen to be $\characteristic{A^*}$ for a measurable set $A^* \supset A$.  Moreover \begin{align*}
\oprobability{A} &= \int^* \characteristic{A} dP = \probability{A^*}
\end{align*}
\end{thm}
\begin{proof}
Let 
\begin{align*}
\mathcal{J}_1 &= \lbrace j_1 \minop \dotsb \minop j_n \mid n \in \naturals \text{ and } j_1, \dotsc, j_n \in \mathcal{J} \rbrace
\end{align*}
Clearly if $j,k \in \mathcal{J}_1$ the also $j \minop k \in \mathcal{J}_1$.  
\begin{clm} $g$ is an essential infimum of $\mathcal{J}$ if and only if $g$ is an essential infimum of $\mathcal{J}_1$.
\end{clm}
Assume $g$ is an essential infimum of $\mathcal{J}$.  The for any $j_1 \minop \dotsb \minop j_n  \in \mathcal{J}_1$ we know that $g \leq j_i$ a.s. for $i=1, \dotsc, n$ so by taking the interesection of the events $\lbrace g \leq j_1 \rbrace \cap \dotsb \cap \lbrace g \leq j_n \rbrace$ we know that $g \leq j_1 \minop \dotsb \minop j_n$ a.s.  Given $h$ such that $h \leq j_1 \minop \dotsb \minop j_n$ a.s. for every subset $\lbrace j_1, \dotsc, j_n \rbrace \subset \mathcal{J}$ in particular we see that $h \leq j$ a.s. for every $j \in \mathcal{J}$ and therefore $h \leq g$ a.s.  In the other direction assume that $g$ is an essential infimum of $\mathcal{J}_1$.  As above it follows immediately that $g \leq j$ a.s. for every $j \in \mathcal{J}$.  If $h \leq j$ a.s. for every $j \in \mathcal{J}$ then taking an interesection of almost sure events we see that $h \leq j_1 \minop \dotsb \minop j_n$ a.s. for every $\lbrace j_1, \dotsc, j_n \rbrace \subset \mathcal{J}$ and it follows that $h \leq g$ a.s.

By the previous claim we may replace $\mathcal{J}$ with $\mathcal{J}_1$ and therefore assume from here that $j,k \in \mathcal{J}$ implies $j \minop k \in \mathcal{J}$.

Note that $\tan : (-\pi/2, \pi/2) \to (-\infty, \infty)$ is a homeomorphism and since $\lim_{x \to \pm \pi/2} \tan x = \pm \infty$ we may extend $\tan$ to a homeomorphism between $[-\pi/2, \pi/2]$ and $[-\infty, \infty]$.  Therefore for every $j \in \mathcal{J}$ we can consider $\tan^{-1} j : \Omega \to [-\pi/2, \pi/2]$.  We now pick a sequence $j_1, j_2, \dotsc$ such that
\begin{align*}
\lim_{n \to \infty} \int \tan^{-1} j_n \, dP = \inf_{j \in \mathcal{J}} \int \tan^{-1} j \, dP
\end{align*}
By assumption on $\mathcal{J}$ we know that $j_1 \minop \dotsb \minop j_n \in \mathcal{J}$ and by construction the sequence $j_1 \minop \dotsb \minop j_n$ is decreasing and it follows that 
$\lim_{n \to \infty} j_1 \minop \dotsb \minop j_n = g$ where $g = \inf_n j_1 \minop \dotsb \minop j_n \in \mathcal{L}^0$ by Lemma \ref{LimitsOfMeasurable}.   

\begin{clm} $\int \tan^{-1} g \, dP =\inf_{j \in \mathcal{J}} \int \tan^{-1} j \, dP$
\end{clm}
By the definition of infimum we know that
$\int \tan^{-1}  j_1 \minop \dotsb \minop j_n \, dP \geq \inf_{j \in \mathcal{J}} \int \tan^{-1} j \, dP$ and by monotone convergence we know that 
\begin{align*}
\int \tan^{-1} g \, dP &= \lim_{n \to \infty} \int \tan^{-1}  j_1 \minop \dotsb \minop j_n \, dP \geq \inf_{j \in \mathcal{J}} \int \tan^{-1} j \, dP
\end{align*}
On the other hand 
\begin{align*}
\int \tan^{-1} g \, dP &= \lim_{n \to \infty} \int \tan^{-1}  j_1 \minop \dotsb \minop j_n \, dP \leq \lim_{n \to \infty} \int \tan^{-1}  j_n \, dP 
= \inf_{j \in \mathcal{J}} \int \tan^{-1} j \, dP
\end{align*}
and the claim follows.

\begin{clm} $g$ is an essential infimum for $\mathcal{J}$.
\end{clm}
Given any $k \in \mathcal{J}$ because $k \minop g \leq g$ everywhere we know that $\int \tan^{-1} k \minop g \, dP \leq \int  \tan^{-1} g \, dP = \inf_{j \in \mathcal{J}} \int \tan^{-1} j \, dP$ hence in fact
 $\int (\tan^{-1} g - \tan^{-1} k \minop g) \, dP = 0$ which since $\tan^{-1} g \geq \tan^{-1} k \minop g$ implies $\tan^{-1} g = \tan^{-1} k \minop g$ a.s. (equivalently $g \leq k$ a.s.).  Lastly if $h \leq j$ a.s. for every $j \in \mathcal{J}$ then $h \leq j_1 \minop \dotsb \minop j_n$ a.s. for all $n \in \naturals$ and
\begin{align*}
h &\leq \lim_{n \to \infty} j_1 \minop \dotsb \minop j_n = g \text{ a.s. }
\end{align*}

In the special case of $\mathcal{J}_f$ it is easy to see that $\mathcal{J}_f$ is closed under the minimum and therefore we can construct an essential infimum as $g=\lim_{n \to \infty} j_1 \minop \dotsb \minop j_n$ for a sequence $j_n \geq f$ everywhere.  It follows that $g \geq f$ everywhere.

Now we turn to considering the expectations.  First suppose that $\int f^* \, dP$ is defined (i.e. at most one of $\int f^*_\pm \, dP = \infty$).  Suppose there exists a measurable $g \geq f$ such
that $\int g_\pm \, dP = \infty$.  By definition of the essential infimum we know that $f^* \leq g$ a.s. which implies that $\int f^*_- \, dP \geq \int g_- \, dP = \infty$ so we may conclude $\int f^*_+ \, dP < \infty$ and $\int f^* \, dP = -\infty$.  This shows us that $\int^* f \, dP$ is defined (and equal to $-\infty$).    On the other hand suppose that $\int^* f \, dP$ is defined. Suppose that $\int f_\pm^* \, dP = \infty$ then since $f^*$ is measurable and $f^* \geq f$  everywhere there must exist a measurable $g$ with $g \geq f$, $\int g \, dP = -\infty$.  By definition of essential infimum we see that $f^* \leq g$ a.s. hence $\int f^*_+ \, dP \leq \int g_+ \, dP < \infty$ which is contradiction.  When $\int^* f \, dP$ and $\int f^* \, dP$ exist, on the one hand $f^* \geq f$ implies
\begin{align*}
\int f^* \, dP &\leq \int^* f \, dP
\end{align*}
and on the other hand for all $j \geq f$ everywhere we have $f^* \leq j$ a.s. hence
\begin{align*}
\int f* \, dP &\leq \inf_{j \geq f} \int j \, dP = \int^* f \, dP
\end{align*}

If we consider $\characteristic{A}$ then it is clear from the definition that $\int^* \characteristic{A} \, dP \leq \oprobability{A}$.  On the other hand, we claim that we can find measurable sets $\dotsb \supset A_n \supset A_{n+1} \supset \dotsb$ such that $\left( \characteristic{A} \right)^* = \lim_{n \to \infty} \characteristic{A_n}$.  First note that if $f^*$ is a measurable cover of $\characteristic{A}$ then $\characteristic{f^* \geq 1}$ is also a measurable cover of $\characteristic{A}$.  This follows since $f^* \geq f \geq 0$ therefore $f^* \geq \characteristic{f^* \geq 1} \geq \characteristic{A}$ everywhere.  By the construction above we know that there are measurable $j_1, j_2, \dotsc$ with $j_n \geq \characteristic{A}$ and $j_1 \minop \dotsb \minop j_n  \downarrow f^*$.  Define $A_n = \lbrace j_1 \minop \dotsb \minop j_n \geq 1 \rbrace$.  Since $f^* =  \characteristic{f^* \geq 1}$ a.s. we know that $\lim_{n\to \infty} j_1 \minop \dotsb \minop j_n \in \lbrace 0, 1 \rbrace$ a.s.  Since
$j_1 \minop \dotsb \minop j_n$ is a decreasing sequence we know that $ j_1 \minop \dotsb \minop j_n \downarrow 0$ if and only if $j_1 \minop \dotsb \minop j_n < 1$ eventually if and only if $\characteristic{A_n} = 0$ eventually if and only if $\lim_{n \to \infty} \characteristic{A_n} = 0$.  Similarly $ j_1 \minop \dotsb \minop j_n \downarrow 1$ if and only if $j_1 \minop \dotsb \minop j_n \geq 1$ for all $n \in \naturals$ if and only if $\characteristic{A_n} = 1$ for all $n \in \naturals$ if and only if $\lim_{n \to \infty} \characteristic{A_n} = 1$.  It follows that $f^* = \lim_{n \to \infty} \characteristic{A_n}$ a.s. hence
\begin{align*}
\oprobability{A} &\leq \lim_{n \to \infty} \probability{A_n} = \int f^* \, dP = \int^* f \, dP
\end{align*}
\end{proof}

\begin{prop}\label{BasicPropertiesOfMeasurableCover} Let $(\Omega, \mathcal{A}, P)$  be a probability space and let $f,g : \Omega \to (-\infty, \infty]$ then $(f+g)^* \leq f^* + g^*$ a.s. and
$(f - g)^* \geq f^* - g^*$ a.s. whenever both sides are defined a.s.  If $A$ is measurable, $g$ is measurable and $f \leq g$ a.s. on $A$ we have $f^* \leq g$ a.s. on $A$.  If $A$ is measurable, $g$ is measurable and $f = g$ a.s. on $A$ we have $f^* = g$ a.s. on $A$
\end{prop}
\begin{proof}
By Theorem \ref{ExistenceMeasurableCover} we know that $f^* \geq f > -\infty$ and $g^* \geq g > -\infty$ everywhere thus $f^* + g^*$ is well defined everywhere (possibly $+\infty$).  The function
$f^* + g^*$ is measurable and $f+g \leq f^* + g^*$ therefore it follows that $(f+g)^* \leq f^* + g^*$.

By assumption $f^* - g^*$ is defined a.s. therefore $f^*$ is finite a.s. on $\lbrace g^* = \infty \rbrace$.  It follows that $f^* - g^* = -\infty$ a.s. on $\lbrace g^* = \infty \rbrace$ and therefore it is trivial that $(f-g)^* \geq f^* - g^*$ a.s. on $\lbrace g^* = \infty \rbrace$.  On $\lbrace -\infty < g^* < \infty \rbrace$ we know that $g \leq g^* < \infty$ hence we may write $f = (f - g) + g$ and by the first part of this result we get $f^* \leq (f -g)^*+ g^*$ a.s. which implies $f^* - g^* \leq (f - g)^*$.

Now assume that $A$ is measurable, $g$ is measurable and $f \leq g$ a.s. on $A$.  Consider $h = g \cdot \characteristic{A} + f^* \cdot \characteristic{A^c}$.  Since $f^* \leq f$ everywhere we have
$\lbrace h \geq f \rbrace \subset \lbrace g  \geq f \rbrace$ hence is a null set.  Therefore $f^* \leq h$ a.s and therefore $f^* \leq g$ a.s. on $A$.

If $A$ is measurable, $g$ is measurable and $f = g$ a.s. on $A$ then by the previous fact we have $f^* \leq g$ a.s on $A$.  On the other hand, $g = f \leq f^*$ a.s so it follows that $f^* = g$ a.s on $A$.
\end{proof}

\begin{lem}Let $(\Omega, \mathcal{A}, P)$  be a probability space  and $V$ be a vector space with seminorm $\norm{\cdot}$ then given $f,g : \Omega \to V$ we have
\begin{align*}
\norm{f+g}^* \leq (\norm{f}+\norm{g})^* \leq \norm{f}^* + \norm{g}^* \text{ a.s.}
\end{align*}
and
\begin{align*}
\norm{c f}^* &= \abs{c} \norm{f}^* \text{ a.s. for every $-\infty < c < \infty$}
\end{align*}
\end{lem}
\begin{proof}
By the triangle inequality we know $\norm{f+g} \leq \norm{f}+\norm{g}$ hence $\norm{f+g}^* \leq (\norm{f}+\norm{g})^*$ a.s.  By Proposition \ref{BasicPropertiesOfMeasurableCover} we have $ (\norm{f}+\norm{g})^* \leq \norm{f}^* + \norm{g}^*$ a.s.

By measurability of constant functions it is clear that $\norm{0}^* = 0$ hence we may assume that $c \neq 0$.  Suppose that $g$ is measurable and $\norm{cf} \leq g$.  This equivalent to $\norm{f} \leq g/\abs{c}$ and it follows that $\abs{c} \norm{f}^* \leq g$ a.s.  Suppose $h$ is measurable and for every measurable $g$ with $g \geq \norm{cf}$ everywhere we have $g \geq h$ a.s.   If $g$ is measurable such that $g \geq \norm{f}$ everywhere then $\abs{c} g \geq \norm{c f}$ everywhere hence we conclude $g \geq h/\abs{c}$ a.s. Therefore by the maximality of the measurable cover $f^*$ we have $\norm{f}^* \geq h/\abs{c}$ a.s. equivalently $\abs{c}\norm{f}^* \geq h$ a.s.
\end{proof}

Under an assumption of independence the measurable cover of a sum is the sum of measurable covers.
\begin{prop}\label{MeasurableCoverSumOfIndependent}For $i=1, \dotsc, n$ let $(\Omega_i, \mathcal{A}_i, P_i)$  be probability spaces and $f_i : \Omega_i \to (-\infty, \infty]$ be functions.  If we define $g : \Omega_1 \times \dotsb \times \Omega_n \to (-\infty, \infty]$ by
\begin{align*}
g(\omega_1, \dotsc, \omega_n) &= f_1(\omega_1) + \dotsb + f_n(\omega_n)
\end{align*}
then we have $g^*(\omega_1, \dotsc, \omega_n) = f^*_1(\omega_1) + \dotsb + f^*_n(\omega_n)$.
\end{prop}
\begin{proof}
By an induction argument it is clear that we may assume $n=2$.  Let $P = P_1 \otimes P_2$.   Since for $i=1,2$ we have $f_i(\omega_i) \leq f_i^*(\omega_i)$ everywhere and $f_i^*$ measurable we have $g(\omega_1, \omega_2) \leq f_1^*(\omega_1) + f_2*(\omega_2)$ everywhere and therefore $g^*(\omega_1, \omega_2) \leq f_1^*(\omega_1) + f_2*(\omega_2)$ $P$-a.s.  If $\probability{g^* < f_1^* + f_2^*}>0$ then since
\begin{align*}
\lbrace g^* < f_1^* + f_2^* \rbrace &= \cup_{\substack{t,q,r \in \rationals \\ q+r > t }} \lbrace g^* < t < f_1^* + f_2^*,  q < f_1^*, r < f_2^* \rbrace
\end{align*}
by subadditivity there exists $t,q,r \in \rationals$ with $q+r > t$ and 
\begin{align*}
\probability{g^* < t < f_1^* + f_2^*,  q < f_1^*, r < f_2^*} > 0
\end{align*}  
Define $A = \lbrace g^* < t < f_1^* + f_2^*,  q < f_1^*, r < f_2^* \rbrace$ and for each $\omega_1 \in \Omega_1$ define the section $A_{\omega_1} = \lbrace \omega_2 \mid g^*(\omega_1, \omega_2) < t < f_1^*(\omega_1) + f_2^*(\omega_2),  q < f_1^*(\omega_1), r < f_2^*(\omega_2) \rbrace$.  By Lemma \ref{MeasurableSections} each $A_{\omega_1}$ is measurable and by Tonelli's Theorem \ref{Fubini} we have
\begin{align*}
0 &< \probability{A} = \iint \sprobability{A_{\omega_1}}{2} \, P_1(d\omega_1)
\end{align*}
hence $\sprobability{\sprobability{A_{\omega_1}}{2} > 0}{1} > 0$.  

\begin{clm}There exists $\omega_1 \in \Omega_1$ such that $\sprobability{A_{\omega_1}}{2} > 0$ and $f_1(\omega_1) > q$.
\end{clm}
We establish the claim by contradiction.  If $f_1 \leq q$ everywhere on $\lbrace \sprobability{A_{\omega_1}}{2} > 0 \rbrace$ then by Proposition \ref{BasicPropertiesOfMeasurableCover} we have $f_1^* \leq q$ $P_1$-a.s. on $\lbrace \sprobability{A_{\omega_1}}{2} > 0 \rbrace$.  In particular because $\sprobability{\sprobability{A_{\omega_1}}{2} > 0}{1} > 0$ it follows that there exists $\omega_1$ with $f^*_1(\omega_1) \leq q$ and $\sprobability{A_{\omega_1}}{2} > 0$.  On the other hand if $\sprobability{A_{\omega_1}}{2} > 0$ then $A_{\omega_1} \neq \emptyset$ and it follows that $f_1^*(\omega_1) > q$ which is the desired contradiction.

By the previous claim pick $\omega_1$ with $\sprobability{A_{\omega_1}}{2} > 0$ and $f_1(\omega_1) > q$.  For any $\omega_2 \in A_{\omega_1}$, by choice of $\omega_1$ and definition of $A$ we have
\begin{align*}
q + f_2(\omega_2) \leq f_1(\omega_1) + f_2(\omega_2) \leq g^*(\omega_1, \omega_2)
\end{align*}
hence $f_2 \leq g^*(\omega_1, \cdot) - q$ on $A_{\omega_1}$.  By measurability of $A_{\omega_1}$, $g^*(\omega_1, \cdot) - q$ we may apply Proposition \ref{BasicPropertiesOfMeasurableCover}  to conclude that $f^*_2 \leq g^*(\omega_1, \cdot) - q$ $P_2$-a.s. on $A_{\omega_1}$.  Thus $A_{\omega_1} \cap \lbrace f^*_2 \leq g^*(\omega_1, \cdot) - q \rbrace \neq \emptyset$ and we may pick $\omega_2$ such that $(\omega_1, \omega_2) \in A$ and 
\begin{align*}
q + f_2^*(\omega_2) &\leq g^*(\omega_1, \omega_2) < t < q+r
\end{align*}
which implies $f_2^*(\omega_2) < r$ yielding a contradiction.
\end{proof}

Under an assumption of independence and nonnegativity the measurable cover of a product is the product of measurable covers.  In addition, if a function on a product does not depend one of the factors of the product, the measurable cover may be chosen so that it also does not depend on that factor of the product.
\begin{prop}\label{MeasurableCoverProductOfIndependent}For $i=1, \dotsc, n$ let $(\Omega_i, \mathcal{A}_i, P_i)$  be probability spaces and $f_i : \Omega_i \to [-\infty, \infty]$ be functions.  If either $f_i \geq 0$ for $i=1, \dotsc, n$ or $n=2$ and $f_1 \equiv 1$ and we define $g : \Omega_1 \times \dotsb \times \Omega_n \to (-\infty, \infty]$ by
\begin{align*}
g(\omega_1, \dotsc, \omega_n) &= f_1(\omega_1) \dotsb f_n(\omega_n)
\end{align*}
then we have $g^*(\omega_1, \dotsc, \omega_n) = f^*_1(\omega_1) \dotsb f^*_n(\omega_n)$.
\end{prop}
\begin{proof}
In the case that $f_i \geq 0$ for $i=1, \dotsc, n$  an induction argument shows that it suffices to consider the case $n=2$.  Let $P = P_1 \otimes P_2$.  If $0 \leq f_i \leq f_i^*$ for $i=1, 2$ then $g(\omega_1, \omega_2) \leq f_1^*(\omega_1) f_2^*(\omega_2)$.  In the case that $f_1 \equiv 1$ it trivially follows that 
\begin{align*}
g(\omega_1, \omega_2) &= f_2(\omega_2)  \leq  f_2^*(\omega_2) = f_1^*(\omega_1) f_2^*(\omega_2)
\end{align*}  
In both cases we see that $g^*(\omega_1,\omega_2) \leq f_1^*(\omega_1) f_2^*(\omega_2)$ $P$-a.s.  
We argue by contradiction, so assume that $\probability{g^*(\omega_1,\omega_2) < f_1^*(\omega_1) f_2^*(\omega_2)}>0$.

First consider the case in which $f_1 \equiv 1$ so that $f_1^* \equiv 1$ as well.  Writing
\begin{align*}
\lbrace g^*(\omega_1,\omega_2) < f_1^*(\omega_1) f_2^*(\omega_2)\rbrace =\cup_{t \in \rationals} \lbrace g^*(\omega_1,\omega_2) < t < f_1^*(\omega_1) f_2^*(\omega_2)\rbrace
\end{align*}
and using subadditivity there exists $t \in \rationals$ so that if we we have
$\probability{g^*(\omega_1,\omega_2) < t < f_1^*(\omega_1) f_2^*(\omega_2)} > 0$.  
 By Tonelli's Theorem \ref{Fubini} we
have
\begin{align*}
0 < \probability {g^*(\omega_1,\omega_2) < t < f_1^*(\omega_1) f_2^*(\omega_2)} = \int \sprobability{g^*(\omega_1,\omega_2) < t < f_2^*(\omega_2)}{2} \, P_1(d\omega_1)
\end{align*}
so there exists $\omega_1$ such that $\sprobability{g^*(\omega_1,\omega_2) < t < f_2^*(\omega_2)}{2}>0$.  
Since $f_2 = g \leq g^*$ everywhere, by Proposition \ref{BasicPropertiesOfMeasurableCover}  we conclude $f_2^* \leq t$ a.s. on $\lbrace g^*(\omega_1,\omega_2) < t < f_2^*(\omega_2) \rbrace$ which is a contradiction.

Now consider the case in which $f_1,f_2 \geq 0$.  Via a similar argument as in Proposition \ref{MeasurableCoverProductOfIndependent} there exists $t, q, r \in \rationals$ with $t>0$,$q>0$, $r>0$, $qr>t$ so that if we define 
\begin{align*}
A &= \lbrace g^*(\omega_1,\omega_2) < t < f_1^*(\omega_1) f_2^*(\omega_2); f_1^*(\omega_1) > q ; f_2^*(\omega_2) > r \rbrace
\end{align*}
then $\probability{A}>0$.  Also as in that Proposition consider the sections $A_{\omega_1}$. 

\begin{clm} There exists $\omega_1$ such that $\sprobability{A_{\omega_1}}{2}>0$ and $f_1(\omega_1) > q$.
\end{clm}
We argue by contradiction.   By Tonelli's Theorem \ref{Fubini} we have
\begin{align*}
0 < \probability{A} = \int \sprobability{A_{\omega_1}}{2} \, P_1(d\omega_1)
\end{align*}
and conclude $\sprobability{\sprobability{A_{\omega_1}}{2}>0}{1}>0$.
 If $f_1 \leq q$ everywhere on $\lbrace \sprobability{A_{\omega_1}}{2}>0 \rbrace$ then by Proposition \ref{MeasurableCoverProductOfIndependent} we have $f_1^* \leq q$ $P_1$-a.s. on $\lbrace \sprobability{A_{\omega_1}}{2}>0 \rbrace$, i.e. $\sprobability{\sprobability{A_{\omega_1}}{2}>0; f_1^* \leq q}{1}=\sprobability{\sprobability{A_{\omega_1}}{2}}{1}>0$.
In particular, there exists $\omega_1$ with $f_1^*(\omega_1) \leq q$ and $A_{\omega_1} \neq \emptyset$.  The latter statement implies that $f_1^*(\omega_1) > q$ which is a contradiction.

By the claim we select $\omega_1$ with $\sprobability{A_{\omega_1}}{2}>0$ and $f_1(\omega_1) > q$.  Pick $\omega_2 \in A_{\omega_1}$ and then note
\begin{align*}
f_2(\omega_2) &= \frac{g(\omega_1, \omega_2)}{f_1(\omega_1)} < \frac{g(\omega_1, \omega_2)}{q} \leq \frac{g^*(\omega_1, \omega_2)}{q} 
\end{align*}
hence by Proposition \ref{BasicPropertiesOfMeasurableCover} we have $f_2^* \leq \frac{g^*(\omega_1, \cdot)}{q}$ $P_2$-a.s. on $A_{\omega_1}$.  Thus $\sprobability{f_2^* \leq \frac{g^*(\omega_1, \cdot)}{q}; A_{\omega_1}} {2} = \sprobability{A_{\omega_1}} {2} >0$ and in particular there exists $\omega_2 \in \lbrace f_2^* \leq \frac{g^*(\omega_1, \cdot)}{q} \rbrace \cap A_{\omega_1}$.  For such an
$\omega_2$ we have 
\begin{align*}
q f_2^*(\omega_2) \leq g^*(\omega_1, \omega_2) < t < qr
\end{align*}
hence $f_2^*(\omega_2)<r$ which is a contradicts the fact that $(\omega_1,\omega_2) \in A$ implies $f_2^*(\omega_2)>r$.
\end{proof}

