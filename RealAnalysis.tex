\chapter{Real Analysis}
For purposes of our discussion of measure theory, we often make little
use of the structure of the reals.  In many cases it is with little
effort that we can state results much more generally.  Sometimes the
results will be true of arbitrary sets but in other cases we need the
most basic notions of metric spaces.
\begin{defn}A metric space is a set $S$ together with a function
  $d:SxS \to \reals$ satisfying
\begin{itemize}
\item[(i)]$d(x,y) = 0$ if and only if $x=y$.
\item[(ii)]For all $x,y \in S$, $d(x,y) = d(y,x)$.
\item[(iii)]For all $x,y,z \in S$, $d(x,z) \leq d(x,y) + d(y,z)$.
\end{itemize}
\end{defn}
\begin{lem} Given a metric space $(S,d)$, we have $d(x,y) \geq 0$ for all
  $x,y \in S$.
\end{lem}
\begin{proof}
Let $x,y \in S$ and observe 
\begin{align*}
d(x,y) &= \frac{1}{2} (d(x,y) + d(y,x)) \textrm { by symmetry} \\
&\geq \frac{1}{2} d(x,x) \textrm{ by triangle inequality} \\
&= 0
\end{align*}
\end{proof}
 It's pretty easy to see that standard notions of limits and continuity
extend to the case of metric spaces.
\begin{defn}A sequence of elements $x_n \in S$ converges to $x \in S$
  if for every $\epsilon > 0$, there exists $N > 0$ such that
  $d(x_n,x) < \epsilon$ for all $n > N$.
\end{defn}
\begin{defn}A function between metric spaces $f : (S,d) \to (S', d')$
  is continuous at $x \in S$ if for every $\epsilon>0$, there exists
  $\delta > 0$ such that for $y\in S$ such that $d(x,y)<\delta$ we
  have $d'(f(x),f(y)) < \epsilon$.  A function $f$ that is continuous
  at all points $x \in S$ is said to be continuous.
\end{defn}
\begin{lem}$f : (S,d) \to (S', d')$ is continuous at $x \in S$ if and only if for
  every $x_n \to x$ we have $f(x_n) \to f(x)$.
\end{lem}
\begin{proof}
Suppose $f$ is continuous and let $\epsilon > 0$ be given.  By
continuity, we can pick $\delta > 0$ such that for all $y \in S$
with $d(x,y) < \delta$ we have $d'(f(x), f(y)) < \epsilon$.  Now by
convergence of the sequence $x_n$, we can find $N$ such that for all
$n>N$, we have $d(x_n,x) < \delta$.  Hence for all $n > N$, we have
$d'(f(x), f(x_n)) < \epsilon$.

Now suppose that for every $x_n \to x$ we have $f(x_n) \to f(x)$.  We
argue by contradiction.  Suppose $f$ is not continuous at $x$.  There
exists $\epsilon > 0$ such that we can find $x_n \in S$ such that
$d(x,x_n) < 2 ^ {-n}$ and $d'(f(x_n), f(x)) \geq \epsilon$.  Note that
the sequence $x_n \to x$ but $f(x_n)$ doesn't converge to $f(x)$.
\end{proof}
\begin{defn}For $x\in S$ and $r \geq 0$, the open ball at $x$ or
  radius $r$ is the set
\begin{align*}
B(x;r) = \{y\in S | d(x,y) < r \}
\end{align*}
\end{defn}
\begin{defn}A set $U \subset S$ is open if for every $x \in U$ there
  exists $r>0$ such that $B(x;r) \subset U$.  The complement of an
  open set is called a closed set.
\end{defn}
\begin{lem}A set $A \subset S$ is closed if and only if for every $x_n
  \to x$ with $x_n \in A$, we have $x \in A$.
\end{lem}
\begin{proof}
Suppose $A$ is closed.  Then $A^c$ is open.  Let $x_n \in A$ converge
to $x$.  If $x \notin A$, then $x \in A^c$ and we can find an open
ball $B(x;\epsilon) \subset A^c$.  Pick $N>0$ such that $d(x_n, x) <
\epsilon$ for all $n > N$.  Then $x_n \notin A$ for all $n>N$ which is
a contradiction.

Now suppose $A$ contains all of its limit points.  We show that $A^c$
is open.  Let $x\in A^c$ and suppose the balls $B(x;2^{-n}) \bigcap A
\ne \emptyset$.  Then we can construct a sequence $x_n \in A$ such
that $x_n ->x$.  This is a contradiction, hence for some $n$, we have 
$B(x;2^{-n}) \bigcap A = \emptyset$ and therefore $A^c$ is open.
\end{proof}

As it turns out continuity of a function can be expressed entirely in
terms of open sets.
\begin{lem}A function between metric spaces $f : (S,d) -> (T,d^\prime)$ is continuous
  if and only if for every open subset $U \subset T$, we have $f^{-1}(U)$ is an
  open subset of $S$.
\end{lem}
\begin{proof}For the only if direction, let $U \subset T$ be an open set and pick $x \in
  f^{-1}(U)$.  Now, $f(x) \in U$ and by openness of $U$ we can find $\epsilon
  > 0$ such that $B(f(x); \epsilon) \subset U$.  By continutity of $f$
  we can find a $\delta > 0$ such that for all $y \in S$ with $d(x,y) <
  \delta$ we  have $d^\prime(f(x),f(y)) < \epsilon$.  This is just
  another way of saying $B(x; \delta) \subset f^{-1}(U)$ which shows
  that $f^{-1}(U)$ is open.

For the if direction, pick $x \in S$ and suppose we are given
$\epsilon > 0$.  The ball $B(f(x); \epsilon)$ is an open set in $T$.  
By assumption we know that $f^{-1}(B(f(x); \epsilon))$ is
an open set in $S$ containing $x$.   By definition of openness, we can pick a $\delta > 0$, such that 
$B(x;\delta) \subset f^{-1}(B(f(x); \epsilon))$.  Unwinding this
statement shows that for all $y \in S$ with $d(x,y) < \delta$, we have
$d^\prime(f(x),f(y)) < \epsilon$ and we have show that $f$ is
continuous at $x$.  Since $x \in S$ was arbitrary we have shown $f$ is
continuous on all of $S$.
\end{proof}

\begin{defn}A sequence of elements $x_n \in S$ is said to be a
  \emph{Cauchy sequence}
  if for every $\epsilon > 0$, there exists $N > 0$ such that
  $d(x_n,x_m) < \epsilon$ for all $n,m > N$.
\end{defn}
Note that any convergent sequence is Cauchy.
\begin{lem}If a sequence of elements $x_n \in S$ converges to $x \in
  S$ then it is a Cauchy sequence.
\end{lem}
\begin{proof}Pick $\epsilon > 0$ and then pick $N>0$ so that $d(x_n,x)
  < \frac{\epsilon}{2}$ for all $n > N$.  Then by the triangle
  inequality, $d(x_n, x_m) \leq d(x_n, x) + d(x, x_m) < \epsilon$ for
  $n,m > N$.
\end{proof}
It is also easy to construct examples of Cauchy sequences that do not
converge by looking at spaces with \emph{holes}.
\begin{examp}Consider the sequence $\frac{1}{n}$ on $\reals \setminus
  \lbrace 0 \rbrace$.  It is Cauchy but does not converge.
\end{examp}
The existence of non-convergent Cauchy sequences is in some sense the
definition of what it means for a general metric space to have holes.
This motivates the following definition.
\begin{defn}A metric space $(S,d)$ is said to be \emph{complete} if
  every Cauchy sequence is convergent.
\end{defn}
\begin{defn}The real line $\reals$ is complete.
\end{defn}
\begin{proof}Suppose we are given a Cauchy sequence $x_n$.  Let $a =
  \liminf_{n \to \infty} x_n$ and $b = \limsup_{n \to \infty} x_n $.
  We proceed by contradiction and suppose
  that $a < b$ (note
  that the \emph{completeness axiom} of the reals is used in the
  definition of $\liminf$ and $\limsup$).  Let $M =b-a$ then for any
  $0 < \epsilon < M$, $N>0$
  we can find $k,m > N$ such that $\abs{a - x_k} < \frac{M -
    \epsilon}{2}$ and $\abs{b - x_m} < \frac{M -
    \epsilon}{2}$ thus showing $\abs{x_k - x_m} \geq \epsilon$ and
  contradicting the assumption that $x_n$ was a Cauchy sequence.
\end{proof}

The following is a simple fact about $\reals$.
\begin{lem}\label{IncreasingSequenceWithConvergentSubsequence}Let $x_n$ be a nondecreasing sequence in $\reals$.  Suppose
  there is an infinite subsequence $x_{n_k}$ such that $\lim_{k \to
    \infty} x_{n_k} = x$ , then $\lim_{n \to \infty} x_n = x$.
\end{lem}
\begin{proof}
TODO:  This is actually pretty much obvious.
\end{proof}
In our treatment of measure theory we'll want to have a detailed
understanding of the structure of the topology of the real line.  It
can be described quite simply.
\begin{lem}\label{OpenSetsOfReals}The open sets in $\reals$ are precisely the countable
  unions of disjoint open intervals.
\end{lem}
\begin{proof}Pick an open set $U \subset \reals$.  Define an
  equivalence relation on $U$ such that $a \equiv b$ if and only if
  $[a,b] \subset U$ or $[b,a] \subset U$.  It is easy to see this is an equivalence
  relation.  Reflexivity and symmetry are entirely obvious.
  Transitivity follows from taking a union of intervals (carefully
  taking order into consideration).  

Now, consider the equivalence classes of the relation. As equivalence
classes these sets are disjoint and their union is $U$.  Call the
family of equivalence classes $U_\alpha$.

We have to show that
the equivalence classes are open intervals.  Consider $x \in U_\alpha
\subset U$.  Openness of $U_\alpha$ follows from using openness of $U$ to find a small ball (open interval) around $x \in U$ and noting that every
point of the ball is $\equiv$-related to $x$.  Therefore the same open
ball demonstrates the openness of $U_\alpha$.

To see that equivalence classes are intervals, pick an equivalence
class $U_\alpha$ and consider the open interval $(\inf U_\alpha, \sup
U_\alpha)$.  Since $U_\alpha$ is nonempty and open, $\inf U_\alpha \neq \sup
U_\alpha $ and this interval is non-empty.  By definition of $\inf$ and $\sup$ and the openness of
$U_\alpha$ we can see that $U_\alpha \subset (\inf U_\alpha, \sup
U_\alpha)$ (otherwise we could find an element of $U_\alpha$ bigger
than $\sup$ or less than $\inf$).  On the other hand, suppose we are
given $x \in (\inf U_\alpha, \sup U_\alpha)$.  We can find elements
$y,z \in U_\alpha$ such that $\inf U_\alpha  < y < x < z <\sup  U_\alpha$.
By definition of the equivalence relation, this shows $[y,z] \subset
U_\alpha$ and therefore $x \in U_\alpha$.  Therefore we have shown
that $U_\alpha =  (\inf U_\alpha, \sup U_\alpha)$ is an open interval.

The fact that there are
at most countably many equivalence classes follows from the density
and countability of $\rationals$.
\end{proof} 

\begin{lem}\label{ComplementOfCountableSetDense}Let $A \subset \reals$ be a countable set.  Then $A^c$ is
  dense in $\reals$.
\end{lem}
\begin{proof}
Pick an $x \in \reals$ and consider an interval $I_n = (x - \frac{1}{n}, x +
\frac{1}{n})$ for $n > 0$.  Then if $A^c \cap I_n = \emptyset$ we have
$I_n \subset A$ which implies that $I_n$ is countable.  This is
clearly false (since otherwise we could write the reals as a countable
union of countable sets which would imply the reals themselves are countable).
\end{proof}

Just as an aside at this point, we note that notions of open and
closed set are really all that is needed to make sense of the notions
of convergence and continuity.
\begin{defn}A topological space is a set $S$ together with a
  collection of subsets $\tau$ satisfying
\begin{itemize}
\item[(i)]$\tau$ contains $\emptyset$ and $S$.
\item[(ii)]$\tau$ is closed under arbitrary union.
\item[(iii)]$\tau$ is closed under finite intersection.
\end{itemize}
The collection $\tau$ is called a topology on $S$.  The elements of
$\tau$ are called the open sets of $S$ and the complement of the open
sets are called closed sets.  As we have shown above, if one defines
continuity of a function between topological spaces as inverse images
of open sets being open we have a definition that is a compatible
generalization of the $\epsilon/\delta$ definition of calculus.
\end{defn}

\begin{thm}[Taylor's Theorem]\label{TaylorsTheorem}Let $f: \reals\to
  \reals$ be a function which is $m$-times continuously differentiable.  Then for all
  $0 \leq n < m$,
\begin{align*}
f(b) &= \sum_{k=0}^n \frac{(b-a)^k}{k!} f^{(k)}(a) + R_n(b)
\end{align*}
where the \emph{remainder term} is of the form
\begin{align*}
R_n(b) &= \int_a^b
\frac{(b-x)^n}{n!} f^{(n+1)}(x) \, dx
\end{align*}
\end{thm}
\begin{proof}
We proceed by induction.  Note that for $n=1$, then Taylor's Formula simply says $f(b) = f(a) +
\int_a^b f^\prime(x) \, dx$ which is just the Fundamental Theorem of
Calculus.  For the induction step, we integrate the remainder term by parts.  Consider
the integral $\int_a^b \frac{(b-x)^{(n-1)}}{(n-1)!} f^{(n)}(x) \, dx$ and let
$u = f^{(n)}(x)$ and $dv = \frac{(b-x)^{(n-1)}}{(n-1)!} dx$.  Then $du =
f^{(n+1)}(x) dx$ and $v = -\frac{(b-x)^n}{n!}$, so 
\begin{align*}
\int_a^b \frac{(b-x)^{(n-1)}}{(n-1)!} f^{(n)}(x) \, dx &=
-\frac{(b-x)^n}{n!} f^{(n)}(x) \mid_a^b + \int_a^b \frac{(b-x)^n}{n!}
f^{(n+1)}(x) \, dx \\
&= \frac{(b-a)^n}{n!} f^{(n)}(a) + \int_a^b \frac{(b-x)^n}{n!}
f^{(n+1)}(x) \, dx
\end{align*}
which proves the result.
\end{proof}
The version of Taylor's Formula above expresses the ``integral form''
of the remainder term.  It is often useful to transform the remainder term in Taylor's Formula
into the \emph{Lagrange form}.
\begin{lem}\label{LagrangeFormRemainder}There is a number $c \in
  (a,b)$ such that $\int_a^b
R_n(b)= f^{(n+1)}(c) \frac{(b-a)^{n+1}}{(n+1)!}$
\end{lem}
\begin{proof}
If $f^{(n+1)}(x)$ is constant on the interval $[a,b]$ then by explicit
integration we have the result for any $a < c < b$, so let us assume
that $f^{(n+1)}(x)$ is not constant on $[a,b]$.
By continuity of $f^{(n+1)}(x)$ and compactness of $[a,b]$ we know
that there exist $m, M \in \reals$ such that $m = \min_{x \in [a,b]}
f^{(n+1)}(x)$ and $M = \max_{x \in [a,b]}
f^{(n+1)}(x)$.
From this fact and the fact that $(b -x)^n$ is strictly positive on
$[a,b)$ we have bounds
\begin{align*}
m\frac{(b-a)^{n+1}}{(n+1)!} &= m \int_a^b
\frac{(b-x)^n}{n!} \, dx \\
&< \int_a^b
\frac{(b-x)^n}{n!} f^{(n+1)}(x) \, dx \\
&< M \int_a^b
\frac{(b-x)^n}{n!} \, dx = M\frac{(b-a)^{n+1}}{(n+1)!} 
\end{align*}
hence 
\begin{align*}
m < \frac {(n+1)!} {(b-a)^{n+1}}\int_a^b
\frac{(b-x)^n}{n!} f^{(n+1)}(x) \, dx < M
\end{align*}
By continuity of $f^{(n+1)}(x)$ and the Intermediate Value Theorem, we know that $f^{(n+1)}(x)$
takes every value in $[m,M]$ and therefore there exists $c \in [a,b]$
such that $f^{(n+1)}(c) = \frac {(n+1)!} {(b-a)^{(n+1)}}\int_a^b
\frac{(b-x)^n}{n!} f^{(n+1)}(x) \, dx$.  Because the inequalities are
strict and because $(b-x)^n$ is positive, it follows that in fact $c
\in (a,b)$.
\end{proof}

In addition to the integral form and the Lagrange form of the
remainder it can also be useful to have an estimate on the remainder
in hand.
\begin{cor}\label{TaylorsTheoremRemainderEstimate}Let $f: \reals\to
  \reals$ be a function which is $m$-times continuously differentiable.  Then for all
  $1 \leq n \leq m$,
\begin{align*}
f(b) &= \sum_{k=0}^n \frac{(b-a)^k}{k!} f^{(k)}(a) + r_n(b)
\end{align*}
where the remainder term satisfies
\begin{align*}
\abs{r_n(b)} &\leq 
\frac{\sup_{a \leq x \leq b}\abs{f^{(n)}(x) - f^{(n)}(a)}}{n!} \abs{b-a}^n
\end{align*}
in particular we have $\lim_{b \to a} \frac{r_n(b)}{(b-a)^n} = 0$.
\end{cor}
\begin{proof}
By Taylor's Theorem we have
\begin{align*}
f(b) &= \sum_{k=0}^{n-1} \frac{(b-a)^k}{k!} f^{(k)}(a) + R_{n-1}(b)
\end{align*}
with
\begin{align*}
R_{n-1}(b) &= \int_a^b \frac{(b-x)^{n-1}}{(n-1)!} f^{(n)}(x) \, dx \\
&=\int_a^b \frac{(b-x)^{n-1}}{(n-1)!} (f^{(n)}(x) - f^{(n)}(a)) \, dx +
 f^{(n)}(a) \int_a^b \frac{(b-x)^{n-1}}{(n-1)!} \, dx \\
&=\int_a^b \frac{(b-x)^{n-1}}{(n-1)!} (f^{(n)}(x) - f^{(n)}(a)) \, dx +
 f^{(n)}(a) \frac{(b-a)^{n}}{n!}  \\
\end{align*}
so that 
\begin{align*}
r_n(b) &= \int_a^b \frac{(b-x)^{n-1}}{(n-1)!} (f^{(n)}(x) - f^{(n)}(a)) \, dx
\end{align*}
and therefore
\begin{align*}
\abs{r_n(b)} &\leq \int_a^b \frac{\abs{b-x}^{n-1}}{(n-1)!} \abs{f^{(n)}(x) -
               f^{(n)}(a)} \, dx
\leq \sup_{a \leq x \leq b} \abs{f^{(n)}(x) - f^{(n)}(a)} \frac{\abs{b-a}^{n}}{n!} 
\end{align*}
The last statement follows from the continuity of $f^{(n)}(x)$.
\end{proof}

\begin{lem} Let $X$ be a real normed vector space with a subspace $Y$
  of codimension 1.  Then any bounded linear functional $\lambda$ on
  $Y$ extends to a bounded linear functional on $X$ with the same
  operator norm.
\end{lem}
\begin {proof}
We first assume that $\lambda$ has operator norm $1$.
Let $v$ be any vector that is not in $Y$.  Then every element of $X$
is of the form $y + tv$, hence by linearity all we really have to
choose is the value of $\lambda(v)$ so that the operator norm doesn't
increase.  
First, note that it suffices to show $|\lambda(y+v)| \leq
\|y+v\|$ for all $y$.  For it that if that is true then
\begin{align*}
|\lambda(y + tv)| &= |t\lambda(y/t + v)| \\
&\leq |t| \|y/t + v\| \\
&= \|y + tv\| 
\end{align*}
We rewrite the constraint $|\lambda(y+v)| \leq
\|y+v\|$ for all $y$ as
\begin{align*}
-\lambda(y) - \|y+v\| \leq \lambda(v) \leq  \|y+v\| - \lambda(y)
\end{align*}
To see that it is possible to satisfy the constraint derived above, we
use the triangle inequality (subadditivity) of the operator norm.  For
all $y_1,y_2 \in Y$,
\begin{align*}
\lambda(y_1) - \lambda(y_2) &\leq|\lambda(y_1 - y_2)| \\
&\leq \|y_1 - y_2\| \\
&= \|y_1 + v - v - y_2 \| \\
&\leq \|y_1 + v\| + \|y_2 + v\| 
\end{align*}
From which we conclude by rearranging terms 
\begin{align*}
\sup_{y_2 \in Y} -\lambda(y_2) - \|y_2 + v\| \leq \inf_{y_1 \in Y} \|y_1 + v\| - \lambda(y_1)
\end{align*} 
Picking any value between the two terms of the above inequality
results in a valid extension.
To handle the case of operator norm not equal to $1$, notice that the
extension is trivial if the operator norm is $0$ (i.e. $\lambda=0$), otherwise define the
extension by $\|\lambda\|$ times the extension of $\lambda/\|\lambda\|$.
\end {proof}
\begin{thm}[Hahn-Banach Theorem (Real case)]
Let $X$ be a real normed vector space with a subspace $Y$.  Then any bounded linear functional $\lambda$ on  $Y$ extends to a bounded linear functional on $X$ with the same operator norm.
\end{thm}
\begin{proof}We proceed by using the codimension 1 case proved above
  and then applying Zorn's Lemma.  We define a partial extension of
  $\lambda$ to be a pair $(Y^\prime, \lambda^\prime)$ such that $Y
  \subset Y^\prime \subset X$ and $\lambda^\prime$ is an extension of
  $\lambda$ with the same operator norm.  Put a partial order on the
  set of extensions by declaring $(Y^\prime, \lambda^\prime) \leq
  (Y^{\prime\prime}, \lambda^{\prime\prime})$ if and only if $Y^\prime
  \subset Y^{\prime\prime}$ and $\lambda^{\prime\prime}\mid_{Y^\prime}
  = \lambda^\prime$.

To apply Zorn's Lemma, we need to show that every chain has an upper
bound.  If we are given a chain $(Y_\alpha, \lambda_\alpha)$ then we
define $Z = \cup_\alpha Y_\alpha$ and for any $z \in Z$ we define
$\tilde{\lambda}(z) = \lambda_\alpha(z)$ for any $\alpha$ such that $z \in
Y_\alpha$.  It is immediate that this well defined.  It is easy to
show linearity and to show that $\norm{\tilde{\lambda}} =
\norm{\lambda}$ (TODO: do this).

Now we can apply Zorn's Lemma to conclude that there is a maximal
element $(Y^\prime, \lambda^\prime)$.  The codimension one case show
us that $Y^\prime = X$ for otherwise we can construct an extension
that shows $(Y^\prime, \lambda^\prime)$ is not maximal.
\end{proof}
Note that the use of Zorn's Lemma here is not accidental; the Hahn
Banach Theorem cannot be proven in set theory without the Axiom of
Choice (though according to Tao it can be proven without the full
power of the Axiom of Choice using what is know as the Ultrafilter Lemma).
\section{Compactness}

\begin{defn}Let $(S,d)$ be a metric space, then we say $K \subset S$
  is \emph{sequentially compact} if and only if for every sequence $x_1, x_2,
  \dots \in K$ there exists a convergent subsequence $x_{n_j}$ such
  that $\lim_{j \to \infty} x_{n_j} \in K$.
\end{defn}

\begin{defn}Let $(S,d)$ be a metric space, then we say $S$
  is \emph{compact} if and only if for every collection $U_\alpha$ of
  open sets such that $\bigcup_\alpha U_\alpha \supset S$ there
  exists a finite subcollection $U_1, \dots, U_n$ such that
  $\bigcup_{j=1}^n U_j \supset S$.
\end{defn}

\begin{defn}Let $(S,d)$ be a metric space, then we say $S$
  is \emph{totally bounded} if and only if for every $\epsilon >0$
  there exists a finite set of points $F \subset S$ such that for
  every $x\in S$ there is a $y \in F$ such that $d(x,y) < \epsilon$.
\end{defn}

\begin{defn}Let $(S,d)$ be a metric space, then we say $x \in S$
  is \emph{limit point} of a set $A \subset S$ if and only if for
  every open set $U$ containing $x$, $A \cap (U \setminus \lbrace x
  \rbrace) \neq \emptyset$.
\end{defn}

\begin{thm}\label{CompactnessInMetricSpaces}In a metric space $(S,d)$ the following are equivalent
\begin{itemize}
\item[(i)]$S$ is compact
\item[(ii)]$S$ is complete and totally bounded
\item[(iii)]Every infinite subset of $S$ has a limit point
\item[(iv)]$S$ is sequentially compact
\end{itemize}
\end{thm}
\begin{proof}
First we show that (i) implies (ii).  Given $\epsilon > 0$ note that
we have a covering by open balls $\cup_{x \in S} B(x, \epsilon)$.  By
compactness we have a finite set $x_1, \dots, x_m$ such that
$\cup_{i=1}^m B(x_i, \epsilon) = S$.  Thus given $y \in S$, we know
there is an $x_j$ such that $y \in B(x_j, \epsilon)$ and we have shown
total boundedness.  To show completeness, let $x_1, x_2, \dots$ be a
Cauchy sequence in $S$.  For every $m > 0$ we know there exists $N_m$
such that $d(x_{N_m}, x_n) < \frac{1}{m}$ for every $n > N_m$.  Now
define $U_m = \lbrace x \in S \mid d(x_{N_m}, x) > \frac{1}{m}\rbrace$
and note that $U_m$ is open. Furthermore we know that $x_n \notin U_m$
for all $n > N_m$.  By virtue of this latter fact we can see that
there is no finite subset of $U_m$ that covers $S$; for given $U_1,
\dots, U_m$ then $x_n \notin \cup_{k=1}^m U_k$ for any $n > \max(N_1,
\dots, N_m)$.  By compactness of $S$ we know that the $U_m$ do not
cover $S$ and therefore there is an $x \in S \setminus
\cup_{m=1}^\infty U_m$.  For such an $x$, by definition of $U_m$ we
know that $d(x_{N_m}, x) \leq \frac{1}{m}$ for all $m > 0$.  By the
triangle inequality we then get that $d(x_n, x) \leq \frac{2}{m}$ for
all $n > N_m$ and $m > 0$ which shows that $x_n$ converges to $x$.
Thus $S$ is complete.

Next we show that (ii) implies (iii).  Suppose $A \subset S$ is an
infinite set.  By the assumption of total boundedness, for each $n >
0$, we can find a finite set $F_n$ such that for every $y \in S$ there
exists $x \in F_n$ such that $d(x,y) < \frac{1}{n}$.  Since the finite
sets $B(y, 1)$ for $y \in F_1$ cover $S$ there is an $y_1 \in F_1$
such that $A \cap B(y_1, 1)$ is infinite.  Then arguing inductively we
construct for every $n>0$ a $y_n \in F_n$ such that $A \cap B(y_1,1)
\cap \cdots \cap B(y_n, \frac{1}{n})$ is infinite.  Note that for $n > m
>0$, by the triangle inequality using any of the infinite number of
elements in $B(y_n, \frac{1}{n}) \cap B(y_m, \frac{1}{m})$, we have $d(y_n, y_m) < \frac{1}{m} +
\frac{1}{n} < \frac{2}{m}$.  This shows that $y_n$ is a Cauchy
sequence and by assumption we know that this converges to some $y \in
S$ and by the above estimate on $d(y_n, y_m)$, we know that for every
$m > 0$, $d(y, y_m) < \frac{2}{m}$.  Therefore we have the inclusion
$B(y_m, \frac{1}{m}) \subset B(y, \frac{3}{m})$ and therefore $A \cap
B(y, \frac{3}{m})$ is also infinite which shows $y$ is a limit point
of $A$.

Next we show that (iii) implies (iv).  Let $x_1, x_2, \dots$ be an
infinite sequence with an infinite range and by (iii) we can get a limit point $x \in S$.
Thus we can find a subsequence $x_{n_1}, x_{n_2}, \dots$ such that
$x_{n_k} \in B(x, \frac{1}{k})$ which shows that the subsequence
converges.  If the sequence has a finite range then it is eventually
constant and converges.

Lastly let's show that (iv) implies (i).  Pick an open cover
$\mathcal{U}_\alpha$ of $S$.  Our first subtask is to show that there exists a
radius $r > 0$ such that for every $x \in S$, the ball $B(x,r)$ is contained in some
element of $\mathcal{U}_\alpha$.   To that end, for every $x \in S$ let 
\begin{align*}
f(x) &= \sup \lbrace r \mid B(x,r) \subset U_\alpha \text{ for some }
\alpha \rbrace
\end{align*}
We claim that $\inf \lbrace f(x) \mid x \in S \rbrace > 0$.  To verify
the claim, we argue by contradiction and assume we can find a sequence
$x_n$ with $f(x_n) < \frac{1}{n}$ (i.e. the ball $B(x_n, \frac{1}{n})$
is not contained in any $U_\alpha$). 
By sequential compactness we have a convergent subsequence $x_{n_k}$
that converges to $x \in S$.  Because $\mathcal{U}_\alpha$ is a open cover there we
can find an $r > 0$ and $U_\alpha$ such that $B(x, r) \subset
U_\alpha$.  Pick $N_1 > \frac{2}{r}$.  By convergence of $x_{n_k}$ we can find $N_2 > 0$ such that
for $n_k > N_2$ we have $d(x, x_{n_k}) < \frac{r}{2}$.  For $n_k >
\max(N_1, N_2)$, by the triangle inequality we have $B(x_{n_k},
\frac{1}{n_k}) \subset B(x,r) \subset U_\alpha$, so we have
a contradiction.

With the claim verified we return to the problem of proving
compactness.  Pick an arbitrary $x_1 \in S$ and let $c = 2 \wedge
\inf_{x \in S} f(x)$.  We define $x_n$ inductively by the following
algorithm: if there is exists $x_n$ such that $d(x_n, x_j) >
\frac{c}{2}$ for all $j=1, \dots, n-1$ then pick it otherwise stop.
We claim that the algorithm terminates after a finite number of
steps.  If it didn't then we'd have constructed an infinite sequence
$x_n$ such that for all $m,n > 0$ we have $d(x_n,x_m) > \frac{c}{2}$
which implies there is no Cauchy subsequence hence has no convergent
subsequence contradicting sequential compactness.  Therefore there is an $n>0$ such that $S = \cup_{k=1}^n
B(x_k, \frac{c}{2})$; however by construction we know that for every
$x_k$ there is a $U_k$ such that $B(x_k, \frac{c}{2}) \subset U_k$.
Then $U_1, \dots, U_n$ is a finite subcover of $S$ and we are done.
\end{proof}
It is worth noting that the equivalence of the finite subcover
property and sequential compactness does not hold in general
topological spaces.  In general sequential compactness is equivalent
to the weaker property that \emph{countable} open covers have finite
subcovers (sometime this property is refered to as countable
compactness).  It turns out that in these circumstances that the full
power of the finite subcover property is generally needed.

\begin{cor}\label{ClosedSubsetsCompact}Every closed subset of a compact set is compact.
\end{cor} 
\begin{proof}Let $B$ be a compact set and let $A \subset B$ be
  closed.  Let $U_\alpha$ be an open cover of $A$, the we may append
  $A^c$ to get an open cover of $B$.  By compactness of $B$ we may
  extract a finite subcover $U_{\alpha_1}, \dotsc, U_{\alpha_n}, A^c$
  (there is no loss in generality in assuming that $A^c$ is in the
  finite subcover).  Clearly, $U_{\alpha_1}, \dotsc, U_{\alpha_n}$ is
  a finite subcover of $A$.
\end{proof}

\begin{thm}\label{ContinuousImageOfCompact}Let $f : (S, d) \to
  (S^\prime, d^\prime)$ be continuous.  If $S$ is compact then $f(S)$
  is compact.
\end{thm}
\begin{proof}Let $U_\alpha$ be an open cover of $f(S)$.  By continuity
  of $f$, $f^{-1}(U_\alpha)$ is an open cover of $S$ and therefore has
  a finite subcover $f^{-1}(U_1), \dots, f^{-1}(U_n)$.  It is easy to
  see that $U_1, \dots, U_n$ is a finite subcover of $f(S)$:  if $y \in
  f(S)$, we can write $y = f(x)$ for $x \in S$; picking $i$ so that $x \in
  f^{-1}(U_i)$, we see that $y \in U_i$.
\end{proof}

The following is a characterization of compact sets in $\reals^n$.
\begin{thm}\label{HeineBorel}[Heine-Borel Theorem]A subset $A \subset
  \reals^n$ is closed and
  bounded if and only if it is compact.
\end{thm}
TODO:  I don't think it is worth doing the proof from scratch; this is
a simple corollary of the result.
\begin{proof}By Lemma \ref{CompactnessInMetricSpaces} it suffices to
  show that a closed and bounded set in $\reals^n$ is complete and
  totally bounded.  Completeness is simple as any Cauchy sequence in
  $A$ converges in $\reals^n$ by completeness of $\reals^n$ but then
  the limit is in $A$ because $A$ is closed.  To see total
  boundedness, pick an $\epsilon > 0$ and then pick $N >
  \frac{\sqrt{n}}{\epsilon}$.  Since $A$ is bounded, there exists $M >
  0$ such that $A \subset [-M, M] \times \cdots \times [-M,M]$.  It
  suffices to show that the latter set is totally bounded.  Pick
  the finite set of points $\lbrace (x_1/N, \dots , x_n/N) \mid -MN \leq x_j
    \leq MN \rbrace$ and note that 
\begin{align*} [-M, M] \times \cdots \times
  [-M,M] \subset \bigcup B((x_1/N, \dots , x_n/N), \epsilon)
\end{align*}
\end{proof}

Before we begin the proof we need a Lemma.
\begin{lem}Suppose $C_0 \supset C_1 \supset \cdots$ is a nested
  sequence of closed and bounded sets $C_k \subset \reals^n$.  Then
  $\cap_k C_k$ is non empty.
\end{lem}
\begin{proof}Here is the proof for $n=1$.  TODO: Generalize.

Let $a_k = \inf C_k$; because $C_k$ is closed we know that $a_k \in
C_k$.  By the nestedness and boundedness of $C_k$, we know that $a_k$
is a non-decreasing bounded sequence and therefore has a limit $a$.
For any fixed $k$, the sequence $a_n \in C_k$ for all $n \geq k$ and
thus $a=\lim_{n \to \infty} a_n \in C_k$.  Since $k$ was arbitrary we
have $a \in \cap_k C_k$ and we're done.
\end{proof}
With the Lemma in hand we can proceed to the proof of Heine-Borel.
\begin{proof}
Suppose $A$ is closed and bounded.  By boundedness there exists $N>0$
such that $A \subset [-N,N] \times \cdots \times [-N,N]$ and by
Corollary \ref{ClosedSubsetsCompact} it suffices to show that $ [-N,N]
\times \cdots \times [-N,N]$ is compact.

Now suppose that we are given an infinite open covering of $ [-N,N]
\times \cdots \times [-N,N]$ by sets $A_\alpha$ such that there is no
finite subcover.  Now bisect each side of the cube so that we can
write it as a union of $2^n$ cubes each of side $N$.  $A_\alpha$
covers each of the subcubes; if all of the subcubes had a finite
subcover of $A_\alpha$ then by taking the union we'd have constructed
a finite subcover of $ [-N,N]
\times \cdots \times [-N,N]$.  Since we've assumed that this isn't
true at least one of the subcubes has no finite subcover.   Pick that
cube, call it $C_1$ and now iterate the construction to create a
nested sequence of cubes $C_k$ where $C_k$ has side of length
$N/2^k$.  Since the $C_k$ are closed and bounded by the previous Lemma
we know that the intersection $\cap_k C_k \neq \emptyset$ and
therefore we can pick $x \in \cap_k C_k$.  Since $A_\alpha$ is a
cover, there exists an $A$ such that $x \in A$.  Because $A$ is open
we can in fact find a ball $B(x,r) \subset A$ for some $r > 0$.  Then
for sufficiently large $k$, $C_k \subset B(x,r) \subset A$ which means
that we have constructed a finite subcover for $C_k$ which is a contradiction.
\end{proof}

\begin{defn}Let $(S,d)$ and $(T, d^\prime)$ be metric spaces, a
  function $f : S \to T$ is said to be \emph{uniformly continuous} if
  for every $\epsilon > 0$ there exists a $\delta > 0$ such that
  $d(x,y) < \delta$ implies $d^\prime(f(x), f(y)) < \epsilon$.
\end{defn}

\begin{thm}\label{UniformContinuityOnCompactSets} Let $f : (S,d) \to (T, d^\prime)$ be a continuous function, if $S$ is
  compact then $f$ is uniformly continuous.
\end{thm}
\begin{proof}
The proof is by contradiction.  Suppose that $f$ is not uniformly
continuous.  Fix an $\epsilon > 0$, for every $n > 0$ we can find
$x_n$ and $y_n$ such that $d(x_n, y_n) < \frac{1}{n}$ but
$d^\prime(f(x_n), f(y_n)) \geq \epsilon$.  Now by compactness and Theorem
\ref{CompactnessInMetricSpaces} we can find a
common convergence subsequence of both $x_n$ and $y_n$.  Let's say
$\lim_{j \to \infty} x_{n_j} = x$ and $\lim_{j \to \infty} y_{n_j} =
y$.  Note that for every $j>0$, 
\begin{align*}
d(x,y) = \lim_{j \to \infty } d(x,y)\leq \lim_{j \to \infty } d(x,
x_{n_j}) + d(x_{n_j}, y_{n_j}) + d(y_{n_j}, y) = 0
\end{align*}
therefore $x=y$ and $f(x)=f(y)$.  

Again using the triangle inequality we see
\begin{align*}
\lim_{j \to \infty} d^\prime(f(x_{n_j}), f(y_{n_j})) \leq \lim_{j \to \infty}
d^\prime(f(x_{n_j}), f(x)) + d^\prime(f(x), f(y)) + d^\prime(f(y), f(y_{n_j})) = 0
\end{align*}
which is the desired contradiction.
\end{proof}

\begin{lem}\label{IntersectionOfNestedCompactSets}Let $K_1 \supset K_2
  \supset \cdots$ be a nested collection of non-empty compact sets,
  then $\cap_{n=1}^\infty K_n$ is nonempty.
\end{lem}
\begin{proof}Pick $x_n \in K_n$ and note that by compactness there is
  a convergent subsequence.  Let $x$ be the limit of that convergent
  subsequence.  By nestedness and closedness of each $K_n$ we conclude
  that $x \in K_n$ for every $n$.
\end{proof}

\begin{thm}Let $f : S \to \reals^n$ be a continuous function, if $S$ is
  compact then $f$ is bounded.
\end{thm}
\begin{proof}
By the Heine-Borel Theorem and Theorem \ref{ContinuousImageOfCompact}, we know that $f(S)$ is a closed bounded set.
\end{proof}

A related notion is that of uniform convergence of functions.
\begin{defn}Let $f, f_n : S \to (S, d^\prime)$ be a sequence
  of functions.  The we way that $f_n$ converges to $f$
  \emph{uniformly} if and only if for every $\epsilon > 0$ there
  exists a $N > 0$ such that for all $x \in S$, and $n > N$, $d^\prime(f_n(x), f(x)) < \epsilon$.
\end{defn}

One of the most important points about uniform convergence is that a
uniform limit of continuous functions is continuous.
\begin{lem}\label{UniformLimitContinuousFunctionsIsContinuous}Let $f,
  f_n : (S,d) \to (S^\prime, d^\prime)$ be a sequence
  of functions where $f_n$ are continuous.  If the $f_n$ converge
  to $f$ uniformly then $f$ is continuous.
\end{lem}
\begin{proof}
Suppose we are given an $\epsilon > 0$ and let $x \in S$.  By uniform
convergence of $f_n$ we may find an $N > 0$ such that
$d^\prime(f_n(y), f(y)) < \frac{\epsilon}{3}$ for all $n \geq N$ and
$y \in S$.  In particular, consider $f_N$.  Since this function is
continuous we may find $\delta > 0$ so that $d(x,y) < \delta$ implies
$d^\prime(f_N(x), f_N(y)) < \frac{\epsilon}{3}$.  So by the triangle
inequality, we have 
\begin{align*}
d^\prime(f(x), f(y)) < d^\prime(f(x), f_N(x)) + d^\prime(f_N(x),
f_N(y)) + d^\prime(f_N(y), f(y)) < \epsilon
\end{align*}
\end{proof}

\begin{prop}\label{ExtensionOfUniformlyContinuousMapCompleteRange}Let $(S,d)$ be a metric space and $(T,d^\prime)$ a
  complete metric space.  Suppose that $A \subset S$ and that $f: A
  \to T$ is a uniformly continuous function, then $f$ has a unique
continuous  extension $\overline{f} : \overline{A} \to T$ to the closure of $A
  \subset S$.  
\end{prop}
\begin{proof}
Let $x \in \overline{A}$, pick a sequence $x_n$
in $A$ such that $\lim_{n \to \infty} x_n = x$ and observe that by
uniform continuity of $f(x)$, for
any $\epsilon > 0$ there exists a $\delta > 0$ such that $d(x,y)
< \delta$ implies $d^\prime(f(x) , f(y)) < \epsilon$.  If we pick $N > 0$
such that $d(x_n,y) < \delta/2$ for $n \geq N$ then
$d(x_n, x_m) < \delta$ for all $n,m \geq N$ and thus $d^\prime(f(x_n), f(x_m)) < \epsilon$ for all
$n,m \geq N$.  This shows that the sequence $f(x_n)$ is Cauchy and by
completeness of $T$ we can take the limit; we define $f(x) =
\lim_{n\to infty} f(x_n)$.  We claim that this definition is
independent of the sequence chosen.  Indeed, let $y_n$ be another
sequence from $A$ such that $\lim_{n \to \infty} y_n = x$.  Pick an
$\epsilon > 0$ and by uniform continuity of $f(x)$ let $\delta$ be
chosen such that $d^\prime(f(x),f(y)) < \epsilon/2$ whenever $d(x,y) < \delta$.
There exists $N_1 > 0$ such that $d(y_n, x_n) < \delta$ for every
$n > N_1$ and there exists $N_2 > 0$ such that $d^\prime(f(x_n),f(x)) <
\epsilon/2$ for all $n \geq N_2$.  Then we have for all $n \geq N_1 \vee
N_2$ by the triangle inequality $d^\prime(f(y_n), f(x)) < \epsilon$.
Note that this also shows that the extension $f(x)$ to $\overline{A}$ is
continuous at $x \in \overline{A}$; since is was continuous at all points
of $A$ we know the extension is continuous.  
\end{proof}

\section{Stone Weierstrass Theorem}

\begin{lem}\label{ApproximationOfLatticeInfimum}Let $L$ be a lattice of continuous functions on a compact Hausdorff
  space $X$ and suppose that the pointwise infimum $g(x) = \inf_{f \in
    L} f(x)$ is continuous.  Then for every $\epsilon > 0$ there
  exists $f \in L$ such that $0 \leq \sup \lbrace x \in X \mid f(x) - g(x)
  \rbrace < \epsilon$.
\end{lem}
\begin{proof}
For every $x \in X$ we can find an $f_x \in L$ such that $f_x(x) - g(x) <
\epsilon/3$.  By continuity of $f_x$ and $g$ we can find an open
neighborhood  $U_x$ of $x$ such that $\abs{f_x(x) - f_x(y)} < \epsilon/3$
and $\abs{g(x) - g(y)} < \epsilon/3$.  By the triangle inequality it
follows that $f_x(y) - g(y) < \epsilon$ for all $y \in U_x$.  The
$U_x$ are an open cover of $X$ so by compactness we may take a finite
subcover $U_{x_1}, \dotsc, U_{x_n}$.  Let $f = f_{x_1} \wedge \cdots
\wedge f_{x_n}$ then for every $x \in X$ we have $x \in U_{x_j}$ for
some $x_j$ and 
\begin{align*}
f(x) - g(x) &\leq f_{x_j}(x) - g(x) < \epsilon
\end{align*}
\end{proof}

\begin{lem}Let $L$ be a lattice of continuous functions on a compact
  Hausdorff space $X$ such that 
\begin{itemize}
\item[(i)]$L$ separates points (i.e. for every $x \neq y \in X$ there
  exists $f \in L$ such that $f(x) \neq f(y)$)
\item[(ii)]If $f \in L$ then for every $c \in \reals$ we have $cf \in
  L$ and $f + c \in L$.
\end{itemize}
Then for every continuous function $g$ on $X$ and $\epsilon > 0$ there
exists $f \in L$ such $0 \leq \sup \lbrace x \in X \mid f(x) - g(x)
\rbrace < \epsilon$.
\end{lem}
\begin{proof}
The first thing is to observe that for the lattice $L$ we have
complete control over the values of the function that separates points.

Claim 1: Suppose $x \neq y \in X$ and $a \neq b \in \reals$ then
there exists $f \in L$ such that $f(x) = a$ and $f(y) = b$.

To see the claim because $L$ separates points we have an $h \in L$
such that $h(x) \neq h(y)$.  Now it suffices to define
\begin{align*}
f(z) &= \frac{a -b}{h(x) - h(y)} h(z) + \frac{b h(x) -a h(y) }{h(x) - h(y)} 
\end{align*}
and note that by (ii) we have $f \in L$.

Claim 2: For any closed set $F \subset X$, $y \notin F$ and $a \leq b
\in \reals$ we can find $f \in L$ such that $f \geq a$, $f(y) = a$ and
$f(x) > b$ for all $x \in F$.

Pick an $x \in F$ then by Claim 1, we can find $f_x$ such that $f_x(x)
= b+1$ and $f_x(y) = a$.  By continuity of $f_x$ we have an open
neighborhood $U_x$ of $x$ such that $f(y) > b$ for all $y \in U_x$.
Clearly the $U_x$ form an open cover of $F$.
Since $F$ is closed and $X$ is compact Hausdorff we know that $F$ is
also compact hence we can extract a finite open cover $U_{x_1},
\dotsc,U_{x_n}$ of $F$.  Define
\begin{align*}
f &= (f_{x_1} \vee \cdots \vee f_{x_n}) \wedge a
\end{align*}
and observe that $f \in L$ since $L$ is a lattice and by (ii) $L$
contains the constant functions.

Now we can prove the Lemma proper.  With $g$ selected, let $L_g =
\lbrace f \in L \mid g \leq f \rbrace$.  Clearly $L_g$ is a lattice so
the result follows from Lemma \ref{ApproximationOfLatticeInfimum} if
we can show $g = \inf_{f \in L_g} f$.  Pick an $\delta > 0$ and a $y
\in X$, we try
to find $f \in L_g$ such that $f(y) - g(y) 
\leq \delta$.  First we find
such and $f \in L$ and then show that in fact $f \in L_g$.  Let $F =
\lbrace x\in X \mid g(x) + \delta \rbrace$ which is closed by
continuity of $g$.  By compactness of $X$ and continuity of $g$ we
know that $g$ has a maximum value $M$.  Using Claim 2 we know that we
can find $f \in L$ such that $g(y) +\delta \geq f(x)$ for all $x \in
X$, $g(y) + \delta = f(y)$ and $f(x) > M$ for all $x \in F$.  To see
that $f \in L_g$, note that by definition of $F$ and construction of $f$  for all $x \in X
\setminus F$ we have 
$g(x) < g(y) + \delta \leq f(x)$ and for all $x \in F$ we have $g(x)
\leq M < f(x)$.
\end{proof}

The Stone Weierstrass Theorem concerns the approximation properties of
subalgebras of $C(X)$ but we have been describing the approximation
properties of lattices of continuous functions.  The connection will
rely on the fact that we can uniformly approximate the absolute value
function by a polynomial on a compact interval.  We record that fact
as the following 
\begin{lem}For every $\epsilon > 0$ there exists a polynomial $p(x)$
  such that 
\begin{align*}
\sup \lbrace x \in [-1,1] \mid \abs{p(x) - \abs{x}}
  \rbrace < \epsilon
\end{align*}
\end{lem}
\begin{proof}
TODO:
\end{proof}

\begin{thm}[Stone Weierstrass Theorem]\label{StoneWeierstrassApproximation}Let $X$ be a compact Hausdorff space and let $A \subset
  C(X;\reals)$ be a subalgebra which contains a non-zero constant
  function.  The $A$ is dense in $C(X;\reals)$ if and only if $A$
  separates points.
\end{thm}
\begin{proof}
Let $\overline{A}$ be the uniform closure of $A$ (that is to say the
set of $f$ such that for every $\epsilon > 0$ there exists $g \in A$
such that $\sup \lbrace x \in X \mid \abs{g(x) - f(x)} \rbrace <
\epsilon$.  By Lemma \ref{UniformLimitContinuousFunctionsIsContinuous}
any such limit is continuous hence $\overline{A} \subset C(X)$. (TODO: The referenced result is stated
for a metric space domain however the proof clearly works for a domain
that is a general topological space).

TODO: Finish
\end{proof}

\begin{cor}[Fourier Series Approximation]\label{FourierSeries}For
  every continuous $f : \reals^n \to \reals$ such that $f(x + v) =
  f(x)$ for all $x \in \reals$, and $v \in \integers^n$, for every
$\epsilon > 0$  there exists constants $c_{j,k}$ and
$d_{j,k}$ such that 
\begin{align*}
\sup_x \abs{\sum_{j=0}^n \sum_{k=0}^N (c_{j,k} \sin(2k\pi x_j) + d_{j,k}
\cos(2k\pi x_j)) - f(x)} < \epsilon
\end{align*}
\end{cor}
\begin{proof}First we observe that there is a bijection between
  periodic function as in the hypothesis and functions on the
  topological space $T^n = S^1 \times \cdots \times S^1$ (the $n$-torus).
Observe that if one has a uniform approximation to a function
viewed as having a domain $T^n$ then the uniform approximation applies
equally well when considered as a periodic function on $\reals^n$.

It remains to observe that $T^n$ is compact Hausdorff, the functions $\sin(2k\pi x_j)$ an $\cos(2k
\pi x_j)$ separate points and contain the constants so the Stone Weierstrass
Theorem applies.

An alternative approach is a more constructive one using the Fejer kernel.
\end{proof}

\begin{cor}[Weierstrass Approximation
  Theorem]\label{WeierstrassApproximation}For every continuous
  function $f : [0,T] \to \reals$ there exists a sequence of
  polynomials $p_n(x)$ such that $\lim_{n \to \infty} \sup_{0 \leq x \leq T} \abs{f(x) -
    p_n(x)} = 0$.
\end{cor}

