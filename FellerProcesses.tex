\chapter{Feller Processes}

We now specialize to the case of time homogeneous Markov processes and
develop an approach that allow one to bring powerful tools of
functional analysis to bear on the theory of Markov processes and
ultimately elucidates a deep connection between Markov processes and
partial differential equations.  Any treatment of this topic must make several pedagogical
decisions.  The functional analysis tools we will use are part of the theory of operator
semigroups and one could simply assume the reader has been exposed to them and quote the
results with appropriate references.  We deem such an approach an undue burden on the reader
as any treatment of semigroups is likely deeply embedded in a textbook in which the core results
and difficult to extract efficiently (much of semigroup theory is motivated by differential equations and
not probability theory).  Thus we have the choice of how to present the required functional
analysis.  One choice is to present the results in a separate chapter or appendix and the other is to 
present the results on an as needed basis.  While we have relegated the basic theory of Banach spaces to
an appendix, we have chosen the second path for the theory of operator semigroups hoping that the 
probabilistic development can provide motivation for the functional analysis and make it
easier to digest.  The disadvantage in doing things this way is that functional analysis results become spread
out thinly through the text and that a reader looking for a particular result cannot find it without knowing or guessing
the places that it might be used.  We accept this disadvantage in hopes that the spirit of the interaction between
the fields can be better appreciated.

TODO:
\begin{itemize}
\item Chapman Kolmogorov relation is equivalent to semigroup property
\item Feller process defined in terms of Feller semigroup properties
\item Feller semigroup generators
\item Feller semigroup strongly continuous (seen via Yosida approximation)
\item Kolmogorov forward/backward equations follow from strong continuity
\item Generators of strongly continuous semigroups and Feller semigroups characterized (Hille-Yosida)
\item Convergence/Approximation of semigroups in terms of generators
\item Convergence of Markov processes in terms of convergence of semigroups/generators
\item Every Feller semigroup has an associated cadlag Feller process (approximation by pure jump-type processes or by Kinney regularization)
\item Approximation of Feller semigroup by Markov chains
\item Continuous sample paths and elliptic generators
\end{itemize}

\section{Semigroups and Generators}

The first step is to change the point of view on transition kernels
slightly.  In the case of a time homogeneous Markov process, the
family of transition kernels is a single parameter family of kernels
$\mu_t$.  Note that in the discrete time case it is clear that the
entire family of kernels is generated by the single time unit kernel
$\mu = \mu_1$ via kernel multiplication $\mu_n = \mu^{n}$ (in
the case of discrete time Markov chains this is just matrix multiplication).  The
first question that we will pursue is whether there is an analogy in
the continuous time case.  The Chapman Kolmogorov relation gives us a
hint on how to proceed.  In the time homogeneous case the Chapman
Kolmogorov relation says that $\mu_s \mu_t = \mu_{s+t}$ which is the
\emph{semigroup property} and suggests that we may be able to write
$\mu_s$ as $exp(s A)$ for some appropriately defined $A$.  With
some additional assumptions this may be done, but first we want to
recast the transition kernels in a different light in which these
questions may be more naturally resolved.  Let $f$ be a measurable
function on $S$ that is either non-negative or bounded.  For any
probability kernel $\mu : S \to \mathcal{P}(S)$, by
Lemma \ref{KernelTensorProductMeasurability} we know that $\int f(t)
\, \mu(s,dt)$ is a itself a measurable function of $s$ that is non-negative
or bounded when $f$ is.  Thus if we are given the transition kernels
of a time homogeneous Markov process we may define an operator 
\begin{align*}
T_tf (s) &= \int f(u) \, \mu_t(s, du) 
\end{align*} 
on an appropriate space of measurable
functions to itself (say the space of bounded measurable functions).  The first thing to observe is that the
Chapman-Kolmogorov relations are equivalent to the semigroup property
for these operators.

\begin{prop}\label{SemigroupsAndChapmanKolmogorov}Let $\mu_t$ for $t \geq 0$ be a family of probability kernels on a measurable space $(S, \mathcal{S})$ and define 
$T_t f(x) = \int f(s) \, \mu_t(x, ds)$ for all bounded measurable function $f : S \to \reals$, then $\mu_t$ satisfies the Chapman-Kolmogorov relations if and only
if $T_t T_s = T_{t + s}$ for all $t,s \geq 0$.
\end{prop}
\begin{proof}
Let $A \in \mathcal{S}$ then we have $T_{t+s} \characteristic{B}(x) =  \mu_{t+s} (x, B)$ and
\begin{align*}
T_t T_s \characteristic{B} (x) &= \int T_s \characteristic{B} (y) \, \mu_t(x, dy) =\int \mu_s(y, B) \, \mu_t(x, dy) \\
&= \mu_t \mu_s (B)
\end{align*}
therefore the Chapman-Kolomogorov relations are equivalent to $T_t T_s \characteristic{B} = T_{t+s} \characteristic{B}$ for all $B \in \mathcal{S}$.  Therefore the semigroup property implies the Chapman-Kologorov relations and if the the Chapman-Kolmogorov relations hold then the semigroup property holds for simple functions by linearity.   If $f$ is positive, bounded and measurable then find simple functions $f_n \uparrow f$ then by Monotone Convergence 
\begin{align*}
T_t f(x) &= \int f(s) \, \mu_t(x, ds) = \lim_{n \to \infty} \int f_n(s) \, \mu_t(x, ds) = \lim_{n \to \infty} T_t f_n(x)
\end{align*}
Moreover $T_t f_n (x)$ is increasing by positivity of integral and therefore another application of Monotone Convergence shows
\begin{align*}
T_t T_s f(x) &= \lim_{n \to \infty} T_t T_s f_n (x) = \lim_{n \to \infty} T_{t + s} f_n (x) = T_{t+s} f(x)
\end{align*}
The semigroup property extends to arbitrary bounded measurable $f$ by writing $f = f_+ - f_-$ with $f_\pm \geq 0$ and using linearity of $T_t$.
\end{proof}

In the case that the kernels $\mu_t$ are the transition kernels of a Markov process we will call the semigroup $T_t$ the transition semigroup of the Markov process.  It is worth
recording for future use the expression for the transition semigroup in terms of the underlying Markov process.
\begin{prop}\label{TransitionSemigroupAsExpectation}Let $X$ be a homogeneous Markov process with time scale $\reals_+$ and state space $S$ then for every bounded measurable function $f : S \to \reals$ and $s \geq 0$,
\begin{align*} 
T_t f (X_s) &= \cexpectationlong{X_s}{f(X_{t+s})} = \cexpectationlong{\mathcal{F}_s}{f(X_{t+s})} 
\end{align*}
in particular, if $X_t$ is a Markov family (e.g. $X$ is canonical) then
\begin{align*} 
T_t f (x) &= \sexpectation{f(X_t)}{x}
\end{align*}
\end{prop}
\begin{proof}
By definition we know that $\cprobability{X_s}{X_{t+s} \in \cdot} (\omega) = \mu_t(X_s(\omega), \cdot)$ and therefore by distintigration (Theorem \ref{Disintegration}) we have 
\begin{align*}
\cexpectationlong{X_s}{f(X_{t+s})}(\omega) &= \int f(s) \mu_t(X_s(\omega), ds) = T_t f(X_s)
\end{align*}
If $X$ is a Markov family then for $x \in S$ under the measure $P_x$ we have by the Markov property (Theorem \ref{StrongMarkovPropertyMarkovProcessCountableValues}) and the
first part of this result
\begin{align*}
\sexpectation{f(X_t)}{x} &= \cexpectationlong{X_0}{f(X_{t})} = T_t f(X_0) = T_t f(x)
\end{align*}
\end{proof}

\begin{prop}Let $X$ be a pure jump-type Markov process with state space $S$ and a bounded rate kernel $\alpha$.  For every bounded measurable function $f : S \to \reals$ define
\begin{align*}
Af (x) &= \int (f(y) - f(x)) \, \alpha(x, dy)
\end{align*}
then $A$ is a bounded linear operator from $B(S) \to B(S)$ and $T_t = e^{tA}$.
\end{prop}
\begin{proof}
Linearity of $A$ is immediate from the linearity of the integral.  Write $\alpha(x,\cdot) = c(x) \mu(x, \cdot)$ so that $Af(x) = c(x) \int (f(y) - f(x)) \, \mu(x, dy) = c(x) (\int f(y) \, \mu(x, dy) - f(x))$ and the boundedness of $\alpha$ means that $\norm{c}_\infty < \infty$.  Now given $f \in B(S)$, the fact that $\alpha$ is a kernel (Lemma \ref{MeasurabilityRateKernel}) and Lemma \ref{KernelTensorProductMeasurability} imply that $Af$ is a measurable function on $S$.  In addition, we have $\abs{Af(x)} \leq \abs{c(x)} \int \abs{f(y) - f(x)} \mu(x, dy) \leq 2 \norm{c}_\infty \norm{f}_\infty$ which shows that $Af \in B(S)$ and moreover shows upon taking the supremum over $x \in S$ that $\norm{Af} \leq 2 \norm{c}_\infty \norm{f}_\infty$ so that $A$ is a bounded operator with $\norm{A} \leq 2 \norm{c}_\infty$.  

TODO:  Finish.  Kallenberg's proof is very elegant as it reduces to the case in which $c(x)$ is constant and therefore we have a pseudo-Poisson process.  I haven't proven that result as it is tied up a bit in my confusion about how to sort out Markov families.  I should try to come up with a proof that avoids the reduction and just uses the Strong Markov property.
\end{proof}

Given a general operator semigroup $T_t$ we can think a bit about how find an operator $A$ such that $T_t = e^{tA}$.  There are a couple of ways to proceed, but perhaps the simplest is to observe that looking at the equality pointwise in $B(S)$  and formally differentiating at $t=0$ we'd get $\lim_{t \to 0} \frac{T_tf - f}{t} = Af$.  The trick in making this formal is to not assume that the resulting $A$ will be defined for all $B(S)$.
\begin{defn}Let $T_t$ be an operator semigroup on a Banach space $X$ then the \emph{generator} is the operator $A$ defined by $A v = \lim_{t \to 0} \frac{T_tv -v}{t}$ for all $v \in X$ 
for which the limit on the right exists.  
\end{defn}
We reiterate that $A$ is not necessarily defined everywhere and that part of the definition of the generator is its domain of definition.  Unless it is explicity noted otherwise we will use $\domain{A}$ to denote the domain of a generator.

It is trivial to see the following.
\begin{prop}The generator is a linear operator on its domain of definition.
\end{prop}
\begin{proof}
This follows directly from linearity of the $T_t$ and linearity of limits.  If $v,w \in \domain{A}$ and $a \in \reals$ then
\begin{align*}
\lim_{t \to 0} \frac{T_t av - av}{t} &= a \lim_{t \to 0} \frac{T_t v - v}{t} = aAv \\
\lim_{t \to 0} \frac{T_t (v+w) - (v+w)}{t} &= \lim_{t \to 0} \frac{T_t v - v}{t} + \lim_{t \to 0} \frac{T_t w - w}{t}= Av + Aw \\
\end{align*}
\end{proof}

\section{Strongly Continuous and Feller Semigroups}

We begin the process of seeing how the theory of operator semigroups and Markov processes interacts from the semigroup point of view.  We shall enumerate some properties of
semigroups that make them well behaved in important ways and then define an associated class of Markov processes  as those whose semigroup posess these properties.  

\begin{defn}A \emph{semigroup of operators} $T_t$ on a Banach space $X$ is one parameter family of bounded continuous operators $T_t : X \to X$ such that
for all $0 \geq s,t < \infty$ we have $T_s \circ T_t = T_{s+t}$.
\end{defn}

\begin{defn}A semigroup of operators $T_t$ on a Banach space $X$ is said to be \emph{strongly continuous} if $\lim_{t \to 0} T_tv = v$ for all $v \in X$.
\end{defn}

\begin{examp}\label{ExponentialOfBoundedStronglyContinuousSemigroup}Let $A : X \to X$ be a bounded linear operator then $T_t = e^{tA}$ is a strongly continuous semigroup.  Strong continuity follows from the much stronger property that
$e^{tA}$ is a continuous function $\reals$ into $L(X)$.
\end{examp}

In fact paths of strongly continuous semigroups are continuous; to see this we first need the following result.
\begin{lem}\label{StronglyContinuousSemigroupNormBound}Let $T_t$ be a strongly continuous semigroup on a Banach space $X$ then there exists constants $M \geq 1$ and $c > 0$ such that 
$\norm{T_t} \leq M e^{ct}$.
\end{lem}
\begin{proof}
\begin{clm}There exists a $t_0 > 0$ and $M \geq 1$ such that $\norm{T_s} \leq M$ for all $0 \leq s \leq t_0$
\end{clm}
Suppose that the claim is not true then clearly we can find $0 \leq t_1 \leq 1$ such that $\norm{T_{t_1}} > 1$; since $\norm{T_0} = 1$ we actually know that $t_1 > 0$.  If we have $0 \leq t_n \leq \dotsb \leq t_1$ with $t_j \leq 1/j$ and $T_{t_j} \geq j$ for $j=1, \dotsc, n$ then there must also be $0 \leq t_{n+1} \leq t_{n} \wedge 1/(n+1)$ such that $\norm{T_{t_{n+1}}} > n+1$; as before since $\norm{T_0} = 1$ we know that $0 < t_{n+1}$.  In this way,  we find a sequence $t_n \downarrow 0$ for which $\norm{T_{t_n}} \geq n$.  Now by the Principle of Uniform Boundedness Theorem \ref{PrincipleOfUniformBoundedness} we conclude that $T_{t_n}$ are not pointwise bounded so there must exist a $v \in X$ for which $sup_n \norm{T_{t_n} v} = \infty$.  It then follows that $\lim_{n \to \infty} T_{t_n} v \neq v$ which is a contradiction.

Now let $c = t_0^{-1} \ln M$ and for an arbitrary $t \geq 0$ write $t = n t_0 + s$ for $0 \leq s < t_0$ then we have
\begin{align*}
\norm{T_t} &= \norm{T_s \circ T_{t_0}^k} \leq M M^{n} \leq M M^{t/t_0} = M e^{\frac{t}{t_0} \ln M} = M e^{ct}
\end{align*}
\end{proof}

\begin{prop}\label{StronglyContinuousSemigroupContinuousPaths}Let $T_t$ be a strongly continuous semigroup on a Banach space $X$ then for each $v \in X$ the function $T_t v : [0,\infty) \to X$ is a continuous function.
\end{prop}
\begin{proof}
Let $v \in X$ and $t \geq 0$ then 
\begin{align*}
\lim_{h \downarrow 0} \norm{T_{t+h} v - T_t v} &= \lim_{h \downarrow 0} \norm{T_{t}(T_hv - v)} \leq M e^{ct} \lim_{h \downarrow 0}\norm{T_h v - v} = 0
\end{align*}
for $t > 0$ we have
\begin{align*}
\lim_{h \downarrow 0} \norm{T_{t-h} v - T_t v} &= \lim_{h \downarrow 0} \norm{T_{t-h}(v - T_hv)} \leq M e^{c(t-h)} \lim_{h \downarrow 0}\norm{v - T_h v} = 0
\end{align*}
\end{proof}

Note that if $A$ is a bounded generator then paths $e^{tA}v$ are actually differentiable and are solutions of the abstract Cauchy problem $\frac{d}{dt} f(t) = A f(t)$ with $f(0) = v$.   As it turns out, for an arbitrary strongly continuous contraction semigroup, paths $T_t v$ are differentiable solutions of an abstract Cauchy problem so long as $v \in \domain{A}$.  In this context we may also refer to the differential equation $\frac{d}{dt} T_t v = A T_t v$  as the Kologorov forward equation.  
It also turns out that the Kolmogorov backward equation generalizes to arbitrary strongly continuous semigroups.
\begin{prop}\label{StronglyContinuousSemigroupKolomgorovBackwardEquation}Let $T_t$ be a strongly continuous semigroup on a Banach  space $X$ with generator $A$ then 
\begin{itemize}
\item[(i)] if $v \in X$ and $t \geq 0$ then $\int_0^t T_s v \, ds \in \domain{A}$ and moreover $T_t v - v = A \int_0^t T_sv \, ds$
\item[(ii)] if $v \in \domain{A}$ then $T_t v \in \domain{A}$ for all $t \geq 0$ and moreover $\frac{d}{dt} T_t v = A T_t v = T_t A v$
\item[(iii)] for each $v \in \domain{A}$ and $t \geq 0$ we have $T_t v - v = \int_0^t A T_sv \, ds = \int_0^t T_s A v \, ds$
\end{itemize}
\end{prop}
\begin{proof}
To see (i), by strong continuity of $T_t$, Proposition \ref{StronglyContinuousSemigroupContinuousPaths} implies  that $T_tv$ is continuous hence Riemann integrable on all finite intervals.  The same follows for $T_u T_t v = T_{u+t}v$ thus we may calculate using Proposition \ref{ClosedOperatorOfRiemannIntegral}, a change of integration variable and the Fundamental Theorem of Calculus \ref{FundamentalTheoremOfCalculusForBanachSpaceRiemannIntegrals}
\begin{align*}
A \int_0^t T_sv \, ds &= \lim_{h \to 0}  \frac{T_h \int_0^t T_sv \, ds - \int_0^t T_sv \, ds}{h} \\
&= \lim_{h \to 0}  \frac{\int_0^t T_{h+s} v \, ds - \int_0^t T_sv \, ds}{h} \\
&= \lim_{h \to 0}  \frac{\int_h^{t+h} T_{s} v \, ds - \int_0^t T_sv \, ds}{h} \\
&= \lim_{h \to 0}  \frac{\int_t^{t+h} T_{s} v \, ds - \int_0^h T_sv \, ds}{h} = T_t v - v\\
\end{align*}

To see (ii), note that a trivial consequence of the semigroup property is that for all $s,t \geq 0$ we have $T_s \circ T_t = T_{s+t} = T_t \circ T_s$.  Using this fact and
the continuity of $T_t$ we calculate for any $v \in \domain{A}$ 
\begin{align*}
A T_t v &= \lim_{h \to 0} \frac{T_h T_t v - T_t v}{h} =\lim_{h \to 0} \frac{T_t T_h v - T_t v}{h} \\
&= T_t \lim_{h \to 0} \frac{T_h v - v}{h} = T_t A v
\end{align*}
It remains to show that for $t > 0$ we also have $\lim_{h \to 0} \frac{T_{t-h} v - T_t v}{-h} = T_t A v$.  
We compute
\begin{align*}
\lim_{h \to 0} \frac{T_t v - T_{t-h} v}{h} &= \lim_{h \to 0} T_{t-h} \frac{T_h v - v}{h} 
= \lim_{h \to 0} T_{t-h} \left[\frac{T_h v - v}{h} -Av\right] + \lim_{h \to 0} T_{t-h} Av \\
&=\lim_{h \to 0} T_{t-h} \left[\frac{T_h v - v}{h} -Av\right] + T_t Av
\end{align*}
To see that the first limit on the last line is zero we can use Lemma \ref{StronglyContinuousSemigroupNormBound} and the fact that $v \in \domain{A}$ to see for all $0 < h \leq t$,
\begin{align*}
\lim_{h \to 0} \norm{T_{t-h} \left[\frac{T_h v - v}{h} -Av\right]} &\leq M e^{ct} \lim_{h \to 0} \norm{\frac{T_h v - v}{h} -Av} = 0
\end{align*}

To see (iii) we apply the Fundamental Theorem of Calculus \ref{FundamentalTheoremOfCalculusForBanachSpaceRiemannIntegrals} and (ii) to see that
\begin{align*}
\int_0^t A T_s v \, ds &= \int_0^t A T_s v \, ds = \int_0^t \frac{d}{ds} T_s v \, ds = T_t v - v
\end{align*}
\end{proof}
As a consequence of the fact $\frac{d}{dt} T_t v = A T_t v$ for $v \in \domain{A}$ by analogy it is sometime helpful to think of general paths $T_t v$ as being a type of generalized or weak solution to the forward equation.

\begin{cor}\label{StronglyContinuousSemigroupGeneratorClosedDomainDense}Let $A$ be the generator of a strongly continuous semigroup on the Banach space $X$ then $A$ is a closed operator and $\domain{A}$ is dense.
\end{cor}
\begin{proof}
If we let $v \in X$ then by Proposition \ref{StronglyContinuousSemigroupKolomgorovBackwardEquation} we know that $\int_0^t T_s v \, ds \in \domain{A}$ for all $t \geq 0$ so it follows that $t^{-1} \int_0^t T_s v \, ds \in \domain{A}$ for all $t > 0$.  Now observe that $\lim_{t \to 0} t^{-1} \int_0^t T_s v \, ds = v$ and we see that $\domain{A}$ is dense.

Now suppose that $v_n \in \domain{A}$ and suppose that $\lim_{n \to \infty} v_n = v$ and $\lim_{n \to \infty} A v_n  = w$.  Applying continuity of $T_t$ and Proposition \ref{StronglyContinuousSemigroupKolomgorovBackwardEquation}  we see that 
\begin{align*}
T_t v - v &= \lim_{n \to \infty} (T_t v_n - v_n) = \lim_{n \to \infty} \int_0^t T_s A v_n \, ds = \int_0^t T_s w \, ds
\end{align*}
TODO: Justify exchanging the limit and the integral...

From this is follows that 
\begin{align*}
\lim_{t \to 0} \frac{T_t v - v}{t} = \lim_{t \to 0} t^{-1} \int_0^t T_s w \, ds = w
\end{align*}
which shows that $v \in \domain{A}$ and $Av = w$.
\end{proof}

\subsection{The Hille-Yosida Theorem}

Our next goal is to prove a significant theorem that characterizes completely the unbounded operators $A$ that are generators of strongly continuous contraction semigroups.  In particular, we need to be able to take a given operator $A$ and construct from it the corresponding semigroup $T_t$.  Due to the Kolmogorov Forward Equation in Proposition \ref{StronglyContinuousSemigroupKolomgorovBackwardEquation} one useful way of think about the problem (which is in fact the historical motivation for the theory we present) is that we are trying to construct solutions to the differential equation $\frac{d}{dt} f(t) = A f(t)$ given an unbounded operator $A$.  If one chooses to stress the analogy with the case of a bounded operator $A$, another way of thinking about the task is that we are trying characterize those unbounded operators $A$ for which we can define $e^{tA}$ for $t \geq 0$.  In fact the problem that we are posing is a bit more restricted than we've indicated.  Since we are considering contraction semigroups that means that we are looking for \emph{bounded} solutions to the aforementioned problems.  

TODO : Motivate the study of the resolvent by showing what bad thing would happen if we have a singular value in $(0,\infty)$.  Note that a bounded operator can arbitrary spectrum since the exponential function is entire however if there is positive real spectrum then the contraction property is not obeyed.  We really don't need spectral theory in what follows; essentially we just need a proper definition of a value that \emph{isn't} in the spectrum of an unbounded operator.

\begin{defn}Let $A : X \to X$ be a closed linear operator on a Banach
  space $X$ then the \emph{resolvent set} $\resolventset{A}$ is the set of $\lambda \in \reals$ such that
\begin{itemize}
\item[(i)] $\lambda - A$ is injective on $\domain{A}$.
\item[(ii)] $\range{\lambda - A} = X$.
\item[(iii)] $(\lambda -A)^{-1}$ is a bounded linear operator
\end{itemize}
The operator $R_\lambda = (\lambda -A)^{-1}$ the \emph{resolvent} of $A$.
\end{defn}
Though we don't make use of the concept, the complement of the resolvent set (actually extended to the complex numbers case) is called the spectrum of $A$.

The resolvent operator of the generator of a strongly continuous contraction semigroup is also the Laplace transform of the semigroup.
\begin{prop}\label{SCCSResolventAsLaplaceTransform}Let $T_t$ be a strongly continuous contraction semigroup on $X$ with generator $A$ then $(0,\infty) \subset \resolventset{A}$ and 
for all $0 < \lambda < \infty$ and $v \in X$ we have
\begin{align*}
R_\lambda v &= (\lambda - A)^{-1}v = \int_0^\infty e^{-\lambda t} T_tv \, dt
\end{align*}
and $\norm{R_\lambda} \leq \lambda^{-1}$. 
\end{prop}
\begin{proof}
Let $0 < \lambda < \infty$.  We know that $e^{-\lambda t} T_t v$ is a continuous function of $t$ since $T_t$ is strongly continuous (Proposition \ref{StronglyContinuousSemigroupContinuousPaths}).  Moreover,
$\int_0^\infty \norm{e^{-\lambda t} T_t v} \, dt \leq \norm{v} \int_0^\infty e^{-\lambda t} \, dt \leq \norm{v} \lambda^{-1}$ and therefore $\int_0^\infty e^{-\lambda t} T_tv \, dt$ is well defined
and in fact defines a bounded linear operator $U_\lambda$ with $\norm{U_\lambda} \leq \lambda^{-1}$. 

\begin{clm}For every $0 < \lambda < \infty$ and $v \in X$ we have $\int_0^\infty e^{-\lambda t} T_t v \, dt \in \domain{A}$ and $(\lambda - A) \cdot \int_0^\infty e^{-\lambda t} T_t v \, dt = v$.
\end{clm}
Applying Proposition \ref{ClosedOperatorOfRiemannIntegral}, a change of integration variable, L'Hopital's Rule, the Fundamental Theorem of Calculus (Theorem \ref{FundamentalTheoremOfCalculusForBanachSpaceRiemannIntegrals}) and strong continuity of $T_t$
\begin{align*}
&\lim_{t \to 0} \frac{T_t \cdot \int_0^\infty e^{-\lambda s} T_s v \, ds - \int_0^\infty e^{-\lambda s} T_s v \, ds}{t} \\
&=\lim_{t \to 0} \frac{\int_0^\infty e^{-\lambda s} T_{t+s} v \, ds - \int_0^\infty e^{-\lambda s} T_s v \, ds}{t} \\
&=\lim_{t \to 0} \frac{\int_t^\infty e^{-\lambda (s-t)} T_{s} v \, ds - \int_0^\infty e^{-\lambda s} T_s v \, ds}{t} \\
&=\lim_{t \to 0} \frac{e^{\lambda t} \int_0^\infty e^{-\lambda s} T_{s} v \, ds - \int_0^\infty e^{-\lambda s} T_s v \, ds - e^{\lambda t} \int_0^t e^{-\lambda s} T_{s} v \, ds}{t} \\
&=\lim_{t \to 0} \frac{e^{\lambda t} - 1}{t} \int_0^\infty e^{-\lambda s} T_{s} v \, ds - \lim_{t \to 0} \frac{e^{\lambda t}}{t} \int_0^t e^{-\lambda s} T_{s} v \, ds\\
&= \lambda \int_0^\infty e^{-\lambda s} T_{s} v \, ds - \lim_{t \to 0} \lambda e^{\lambda t} \int_0^t e^{-\lambda s} T_{s} v \, ds - \lim_{t \to 0} e^{\lambda t}  e^{-\lambda t} T_{t} v\\
&= \lambda \int_0^\infty e^{-\lambda s} T_{s} v \, ds - v\\
\end{align*}
which shows the claim.

From the claim, it follows that $(\lambda -A)$ is surjective.  To see that $(\lambda -A)$ is injective on $\domain{A}$ we let $v \in \domain{A}$ and the apply Proposition \ref{StronglyContinuousSemigroupKolomgorovBackwardEquation} and Proposition \ref{ClosedOperatorOfRiemannIntegral} to see
\begin{align*}
\int_0^\infty e^{-\lambda s} T_s A v \, ds &= \int_0^\infty A T_s v \, ds = A \int_0^\infty T_s v \, ds 
\end{align*}
By this fact and the claim, if $(\lambda - A)v = 0$ then
\begin{align*}
0 &= \int_0^\infty e^{-\lambda s} T_s (\lambda - A) v \, ds = (\lambda - A) \int_0^\infty e^{-\lambda s} T_s v \, ds = v
\end{align*}
thus $(\lambda - A)$ is injective.

We know know that $(\lambda -A)^{-1}$ is a well defined linear operator and the claim shows that $(\lambda -A)^{-1} = U_\lambda$.  The rest of the result follows from the properties already proven of $U_\lambda$.
\end{proof}

We now want to derive some simple properties of resolvents.
\begin{prop}\label{SimplePropertiesOfResolvents}Let $A$ be a closed linear operator then
\begin{itemize}
\item[(i)]$(\mu - A)^{-1}(\lambda -A)^{-1} = (\lambda - A)^{-1}(\mu -A)^{-1} = (\lambda - \mu)^{-1}\left( (\mu - A)^{-1} - (\lambda -A)^{-1}\right)$ for all $\mu, \lambda \in \resolventset{A}$. 
\item[(ii)] $(\lambda -A)^{-1} A = A (\lambda -A)^{-1}$ on $\domain{A}$ for all $\lambda \in \resolventset{A}$.
\end{itemize}
\end{prop}
\begin{proof}
To see (i) we first write $(\lambda - \mu) = ((\lambda - A) - (\mu - A))$ and note that the right hand size has domain $\domain{A}$.  Since $\range{(\lambda - A)^{-1}} = \domain{A}$ for $\lambda \in \resolventset{A}$ we can compute
\begin{align*}
(\lambda - \mu) (\mu - A)^{-1}(\lambda -A)^{-1} &= (\mu - A)^{-1}((\lambda - A) - (\mu - A)) (\lambda -A)^{-1} \\
&= (\mu - A)^{-1} - (\lambda - A)^{-1}
\end{align*}

To see (ii) note that $\range{(\lambda -A)^{-1}} = \domain{A}$ so both sides of the equality are well defined operators with the left hand size having with domain $\domain{A}$ and right hand size domain of $X$.  Now
for $v \in \domain{A}$, 
\begin{align*}
(\lambda -A)^{-1} A v &= - (\lambda -A)^{-1} (\lambda - A) v + \lambda  (\lambda -A)^{-1} v \\
&= -v + \lambda  (\lambda -A)^{-1} v \\
&=- (\lambda - A) (\lambda -A)^{-1} v + \lambda  (\lambda -A)^{-1} v = A (\lambda -A)^{-1} v
\end{align*}
\end{proof}

\begin{prop}\label{SCCSResolventSetOpen}Let $T_t$ be a strongly continuous contraction semigroup on $X$ with generator $A$ then $\resolventset{A}$ is open.  For any $\lambda \in \resolventset{A}$ and 
$\abs{\mu - \lambda} < \norm{R_\lambda}^{-1}$ we have $\mu \in \resolventset{A}$ and 
\begin{align*}
R_\mu &= \sum_{n=0}^\infty (\lambda - \mu)^n R_\lambda^{n+1}
\end{align*}
\end{prop}
\begin{proof}
First note that for $\abs{\mu - \lambda} < \norm{R_\lambda}^{-1}$ we have
\begin{align*}
\sum_{n=0}^\infty \abs{\lambda - \mu}^n \norm{R_\lambda}^{n+1} 
&\leq \norm{R_\lambda} \sum_{n=0}^\infty \abs{\lambda - \mu}^n \norm{R_\lambda}^{n} 
= \frac{\norm{R_\lambda}}{1 - \abs{\lambda - \mu} \norm{R_\lambda}}
< \infty
\end{align*} 
and therefore $\sum_{n=0}^\infty (\lambda - \mu)^n R_\lambda^{n+1}$ defines a bounded linear operator.  Let $v \in X$ then for every $m \in \naturals$
$\sum_{n=0}^m  (\lambda - \mu)^n R_\lambda^{n+1} v \in \domain{A}$ therefore we can compute
\begin{align*}
(\lambda - A) \sum_{n=0}^m  (\lambda - \mu)^n R_\lambda^{n+1} v
&= \sum_{n=0}^m  (\lambda - \mu)^n R_\lambda^{n} v
\end{align*}
and 
\begin{align*}
\sum_{n=0}^\infty \abs{\lambda - \mu}^n \norm{R_\lambda}^{n} v &= \frac{\norm{v}}{1 - \abs{\lambda - \mu} \norm{R_\lambda}} < \infty
\end{align*}
Thus $(\lambda - A) \sum_{n=0}^m  (\lambda - \mu)^n R_\lambda^{n+1} v$ converges absolutely and since $\lambda -A$ is closed
(Corollary \ref{StronglyContinuousSemigroupGeneratorClosedDomainDense}) it follows that $\sum_{n=0}^\infty  (\lambda - \mu)^n R_\lambda^{n+1} v \in \domain{A}$ and
\begin{align*}
(\lambda -A) \sum_{n=0}^\infty  (\lambda - \mu)^n R_\lambda^{n+1}v  &= \sum_{n=0}^\infty  (\lambda - \mu)^n R_\lambda^{n} v
\end{align*}
Now we compute for any $v \in X$,
\begin{align*}
(\mu - A) \sum_{n=0}^\infty  (\lambda - \mu)^n R_\lambda^{n+1} v
&= (\mu - \lambda) \sum_{n=0}^\infty  (\lambda - \mu)^n R_\lambda^{n+1} v + \sum_{n=0}^\infty  (\lambda - \mu)^n R_\lambda^{n} v\\
&= -\sum_{n=1}^\infty  (\lambda - \mu)^n R_\lambda^{n} v + \sum_{n=0}^\infty  (\lambda - \mu)^n R_\lambda^{n} v  = v\\
\end{align*}
from which it follows that $\range{\mu - A} = X$.  To see that $\mu - A$ is injective on $\domain{A}$, assume that $v \in \domain{A}$ and  $(\mu - A) v = 0$.   We
apply $\sum_{n=0}^\infty  (\lambda - \mu)^n R_\lambda^{n+1}$ and use the fact that $A$ and $R_\lambda$ commute (Proposition \ref{SimplePropertiesOfResolvents})
\begin{align*}
0 &=\sum_{n=0}^\infty  (\lambda - \mu)^n R_\lambda^{n+1}  (\mu - A)  v =  (\mu - A)  \sum_{n=0}^\infty  (\lambda - \mu)^n R_\lambda^{n+1} v = v
\end{align*}
and this computation also shows that 
\begin{align*}
R_\mu &= (\mu-A)^{-1} = \sum_{n=0}^\infty  (\lambda - \mu)^n R_\lambda^{n+1}
\end{align*}
which we have already shown to be a bounded operator.  Thus $\mu \in \resolventset{A}$.
\end{proof}

\begin{defn}A linear operator $A : X \to Y$ is \emph{dissipative} if $\norm{\lambda v - A v} \geq \lambda \norm{v}$ for every $v \in \domain{A}$ and $\lambda>0$.
\end{defn}

The reason this definition is of interest is the following.
\begin{examp}\label{SCCSGeneratorDissipative}The generator of any strongly continuous contraction semigroup is dissipative. This follows from Proposition \ref{SCCSResolventAsLaplaceTransform} since by that result for any $v \in \domain{A}$ and $\lambda > 0$ 
\begin{align*}
\norm{v} &= \norm{(\lambda - A)^{-1} (\lambda -A) v} \leq \lambda^{-1} \norm{(\lambda -A) v} 
\end{align*}
\end{examp}

We also record the following example of a class of bounded dissipative operators.
\begin{examp}\label{BoundedTranslationDissipative}Let $A : X \to X$ be a bounded linear operator then for any $c \geq \norm{A}$ the operator $A - c$ is dissipative.  This follows from the triangle inequality
\begin{align*}
\norm{(\lambda - (A - c))v} &= \norm{(\lambda + c) v} - \norm{A v} \geq \norm{(\lambda + c) v} - \norm{A} \norm{v}  \geq \lambda \norm{v}
\end{align*}
\end{examp}

It is also useful to note that the set of dissipative operators is a cone.
\begin{examp}\label{DissipativeOperatorsCone}
If $A$ is dissipative and $c > 0$ then $c A$ is dissipative since 
\begin{align*}
\norm{(\lambda - c A) v} &= \norm{(\lambda/c - A) c v} \geq \lambda/c \norm{cv} = \lambda \norm{v}
\end{align*}
\end{examp}

Now we turn to less trivial matters.  It turns out that the property of being dissipative is a key part of the characterization of generators of strongly continuous semigroups.  As a step in the direction of proving such a result we first study the resolvent set properties of dissipative operators.

\begin{prop}\label{DissipativeAndClosed}Let $A$ be dissipative and $\lambda > 0$ then $A$ is closed if and only if $\range{\lambda - A}$ is closed.  Thus if $A$ is dissipative and $\range{\lambda - A}$ is closed for a single value of $\lambda > 0$ it is closed for all $\lambda > 0$.
\end{prop}
\begin{proof}
Let $A$ be closed.  Suppose $\lambda>0$ and $(\lambda -A) v_n$ converges, then it is Cauchy.  Let $\epsilon > 0$ be given.  We may find $N > 0$ such that $\norm{(\lambda -A)(v_n-v_m)} \leq \epsilon$ for all $n,m \geq N$.  Applying the fact that $A$ is dissipative we have $\norm{v_n - v_m} \leq \lambda^{-1} \epsilon$ which shows that $v_n$ is Cauchy in $X$ and therefore convergent; let $v$ be the limit of $v_n$.  Since $\lambda v_n$ converges to $\lambda v$ it follows that $A v_n = lambda v_n - (\lambda -A) v_n$ converges and since $A$
is assumed closed we know that $v \in \domain{A}$ and $\lim_{n \to \infty} A v_n = A v$.  It follows that $\lim_{n \to \infty} (\lambda -A)v_n  = (\lambda -A)v$ hence $\range{\lambda - A}$ is closed.

Now assume $\range{\lambda - A}$ is closed for all $\lambda > 0$.  Suppose $\lim_{n \to \infty} v_n = v$ and $\lim_{n \to \infty} A v_n = w$.  It follows that $\lim_{n \to \infty} (\lambda - A) v_n = \lambda v - w$.  Since $\range{\lambda - A}$ is closed there exists $u \in \domain{A}$ such that $(\lambda - A) u = \lambda v - w$.  Applying the fact that $A$ is dissipative,
\begin{align*}
\lim_{n \to \infty} \norm{v_n - u} &\leq \lambda^{-1}\lim_{n \to \infty} \norm{(\lambda - A)(v_n - u)} = \lambda^{-1}\lim_{n \to \infty} \norm{(\lambda - A)v_n - \lambda v + w} = 0
\end{align*}
Therefore $v = u$ hence $v \in \domain{A}$ and 
\begin{align*}
\lim_{n \to \infty} A v_n &= \lim_{n \to \infty} \lambda v_n - \lim_{n \to \infty} (\lambda -A) v_n = \lambda v - (\lambda v - A v) = Av
\end{align*}
hence $A$ is closed.

If $A$ is dissipative and $\range{\lambda - A}$ for a single $\lambda>0$ then $A$ is closed and dissipative and it follows that $\range{\lambda - A}$ is closed for all $\lambda > 0$.
\end{proof}


\begin{lem}\label{ClosedDissipativeResolventSet}Let $A$ be a closed dissipative operator on $X$, if there exists $\lambda > 0$ such that $\lambda \in \resolventset{A}$ then $\lambda \in \resolventset{A}$ for all $\lambda > 0$.
\end{lem}
\begin{proof}
We need a simple property of the topology of the real line.
\begin{clm}If $A \subset (0, \infty)$ is non-empty, closed and open then $A = (0,\infty)$.
\end{clm}
We know that all open sets of $(0,\infty)$ are countable unions of disjoint open intervals (Lemma \ref{OpenSetsOfReals}).  Suppose that there is a open interval $(a,b) \subset A$ with either $a\neq 0$ or $b \neq \infty$ then either $a$ or $b$ is a limit point of $A$ that is not contained in $A$ which contradicts the fact that $A$ is closed.

By the claim, the fact that $(0, \infty) \cap \resolventset{A}$ is assumed to be non-empty and the fact that $(0, \infty) \cap \resolventset{A}$ is open (Proposition \ref{SCCSResolventSetOpen}), it suffices to show that $(0, \infty) \cap \resolventset{A}$ is closed.  Thus suppose that we have a sequence 
$\lambda_n \in (0,\infty) \cap \resolventset{A}$ such that $\lim_{n\to \infty} \lambda_n = \lambda > 0$.  
Let $v \in X$ and define $v_n = (\lambda - A) R_{\lambda_n} v$ which is well defined since $\range{R_{\lambda_n}} = \domain{A}$.  Applying the norm bound $\norm{R_{\lambda_n}} \leq \lambda_n^{-1}$ (Proposition \ref{SCCSResolventAsLaplaceTransform}),
\begin{align*}
\lim_{n \to \infty} \norm{v_n - v} &= \lim_{n \to \infty} \norm{(\lambda -A) R_{\lambda_n}  v - (\lambda_n - A) R_{\lambda_n} v} \\
&=\lim_{n \to \infty} \norm{(\lambda -\lambda_n) R_{\lambda_n}  v}  \leq \lim_{n \to \infty} \frac{\abs{\lambda - \lambda_n}}{\lambda_n} \norm{v} = 0\\
\end{align*}
Therefore $\lim_{n \to \infty} v_n = v$; in particular $\range{\lambda - A}$ is dense in $X$.  However, since $A$ is dissipative and closed we can apply Proposition \ref{DissipativeAndClosed} to conclude that $\range{\lambda - A}$ is closed so in fact we have $\range{\lambda - A}=X$.  To see that $\lambda - A$ is injective, suppose
$(\lambda - A) v = 0$ then using the fact that $A$ is dissipative, $\norm{v} \leq \lambda^{-1} \norm{(\lambda - A) v} = 0$.  Similarly using the fact that $A$ is dissipative
we know that 
\begin{align*}
\norm{(\lambda - A)^{-1} v} &\leq \lambda^{-1} \norm{(\lambda-A)(\lambda - A)^{-1} v} = \lambda^{-1} \norm{v}
\end{align*}
which shows that $(\lambda - A)^{-1}$ is a bounded operator hence $\lambda \in \resolventset{A}$.
\end{proof}

\begin{defn}Let $A : X \to X$ be a closed linear operator and suppose $\lambda \in \resolventset{A}$ the we define the \emph{Yosida approximation} be be
\begin{align*}
A_\lambda &= \lambda A R_{\lambda}
\end{align*}
\end{defn}
To see why we refer to $A_\lambda$ as an approximation consider the case in which $A$ is bounded.  In that case for $\lambda > \norm{A}$ we can write
\begin{align*}
A_\lambda &= A (1 - \lambda^{-1} A)^{-1} = A + \lambda^{-1} A^2 + \lambda^{-2} A^3 + \cdots
\end{align*}
which shows that $\lim_{\lambda \to \infty} A_\lambda = A$.
The next result shows how this idea can be applied with unbounded $A$ : closed dissipative operators can be approximated by bounded operators using the Yosida approximation.
\begin{lem}[Yosida approximation]\label{ClosedDissipativeYosidaApproximation}Let $A$ be a closed dissipative operator with $\domain{A}$ dense and $(0,\infty) \subset \resolventset{A}$, then $A_\lambda$ satisfies
\begin{itemize}
\item[(i)] $A_\lambda$ is a a bounded linear operator and $e^{t A_\lambda}$ is a strongly continuous contraction semigroup.
\item[(ii)] $A_\lambda A_\mu = A_\mu A_\lambda$ for all $\lambda, \mu > 0$
\item[(iii)] $\lim_{\lambda \to \infty} A_\lambda v = A v$ for all $v \in \domain{A}$.
\end{itemize}
\end{lem}
\begin{proof}
We begin by showing (i).  Let $\lambda > 0$.  Then since $\lambda \in \resolventset{A}$ we have $R_\lambda$ is defined and $\range{R_\lambda} = \domain{A}$.   Because $A$ is dissipative we have for every $v \in X$,
\begin{align}\label{YosidaNormBound}
\norm{R_\lambda v} &\leq \lambda^{-1} \norm{(\lambda - A)R_\lambda v}  = \lambda^{-1} \norm{v}
\end{align}
hence $\norm{R_\lambda} \leq \lambda^{-1}$.

Using $(\lambda - A)R_\lambda = \IdentityMatrix$ on $X$ and $R_\lambda (\lambda -A) = \IdentityMatrix$ on $\domain{A}$ we get 
\begin{align}\label{YosidaAlternative}
A_\lambda &= \lambda A R_\lambda = \lambda (\lambda R_\lambda - \IdentityMatrix) = \lambda^2 R_\lambda - \lambda \text{ on $X$} 
\end{align}
and it follows that $A_\lambda$ is a bounded linear operator.  Therefore $e^{t A_\lambda}$ is a strongly continuous semigroup (Example \ref{ExponentialOfBoundedStronglyContinuousSemigroup}).  Furthermore by
\eqref{YosidaNormBound}
\begin{align*}
\norm{e^{t A_\lambda}} &= \norm{e^{t \lambda^2 R_\lambda} e^{-t \lambda}} \leq e^{t \lambda^2 \norm{R_\lambda}} e^{-t\lambda} \leq e^{t \lambda} e^{-t\lambda}  = 1
\end{align*}
so $e^{tA_\lambda}$ is contractive.  Thus (i) is shown.

Now (ii) follows from \eqref{YosidaAlternative} and the commutativity of resolvents (Proposition \ref{SimplePropertiesOfResolvents})
\begin{align*}
A_\lambda A_\mu &= (\lambda^2 R_\lambda - \lambda) (\mu^2 R_\mu - \mu) \\
&= \lambda^2 \mu^2 R_\lambda R_\mu - \lambda^2 \mu R_\lambda - \mu^2 \lambda R_\mu +\lambda \mu ) \\
&= \lambda^2 \mu^2 R_\mu R_\lambda - \lambda^2 R_\lambda - \mu^2 R_\mu + \mu \lambda)  \\
&= A_\mu A_\lambda
\end{align*}

To see (iii) first we have
\begin{clm}$\lim_{\lambda \to \infty} \lambda R_\lambda v = v$ for all $v \in X$.
\end{clm}
For $v \in \domain{A}$ we have using Proposition \ref{SimplePropertiesOfResolvents} and Proposition \ref{SCCSResolventAsLaplaceTransform}
\begin{align*}
\norm{\lambda R_\lambda v - v} &= \norm{\lambda R_\lambda v - (\lambda  -  A) R_\lambda v} \\
&= \norm{A R_\lambda v} = \norm{R_\lambda A v} \leq \lambda^{-1} \norm{Av}
\end{align*}
Now take the limit $\lambda \to \infty$ to see the claim for $v \in \domain{A}$.  By assumption $\domain{A}$ is dense in $X$ so for general $v \in X$ let $v_n \in \domain{A}$ such
that $\lim_{n \to \infty} v_n = v$.  Then for all $n \in \naturals$
\begin{align*}
\norm{\lambda R_\lambda v - v} &\leq \norm{(\lambda R_\lambda - \IdentityMatrix)(v - v_n)} + \norm{\lambda R_\lambda v_n - v_n} \\
&\leq 2 \norm{v-v_n} + \norm{\lambda R_\lambda v_n - v_n}
\end{align*}
so take the limit as $\lambda \to \infty$ and then as $n \to \infty$.  

To finish (iii) we use the claim and Proposition \ref{SimplePropertiesOfResolvents} to see that for all $v \in \domain{A}$,
\begin{align*}
\lim_{\lambda \to \infty} A_\lambda v &= \lim_{\lambda \to \infty} \lambda R_\lambda A v = A v
\end{align*}
\end{proof}

\begin{lem}\label{SemigroupBoundInTermsOfBoundedGenerator}Suppose $A$ and $B$ are communting bounded linear operators on a Banach space $X$ such that $\norm{e^{tA}} \leq 1$ and $\norm{e^{tB}} \leq 1$ for all $t \geq 0$. It follows that
\begin{align*}
\norm{e^{tA} - e^{tB}} &\leq t \norm{A-B}
\end{align*}
\end{lem}
\begin{proof}
Let $v \in X$ and $t \geq 0$ then by the Fundamental Theorem of Calculus (Theorem \ref{FundamentalTheoremOfCalculusForBanachSpaceRiemannIntegrals}) and the fact that $A$ and $B$ commute we get
\begin{align*}
e^{tA}v  - e^{tB} v &= \int_0^t \frac{d}{ds} e^{sA} e^{(t-s)B} v \, ds = \int_0^t e^{sA} e^{(t-s)B} (A-B) v \, ds
\end{align*}
Now if we take norms over both sides and use Proposition \ref{NormRiemannIntegralBanachSpace} and the hypotheses that $\norm{e^{sA}}, \norm{e^{(t-s)B}} \leq 1$ the result follows.
\end{proof}

\begin{thm}[Hille-Yosida Theorem]\label{HilleYosidaTheorem}Let $X$ be a Banach space, then a linear operator $A : X \to X$ is the generator of a strongly continuous contraction semigroup if and only if 
\begin{itemize}
\item[(i)] $ \domain{A}$ is dense in $X$
\item[(ii)] $A$ is dissipative
\item[(iii)] $\range{\lambda_0 -A} = X$ for some $\lambda_0 > 0$.
\end{itemize}
\end{thm}
\begin{proof}
If $A$ is the generator of a strongly continuous contraction semigroup then Corollary \ref{StronglyContinuousSemigroupGeneratorClosedDomainDense} implies (i), Example \ref{SCCSGeneratorDissipative} implies (ii) and Proposition \ref{SCCSResolventAsLaplaceTransform} implies that $\range{\lambda -A} = X$ for all $\lambda > 0$, so in particular (iii) holds.

Let us assume that (i), (ii) and (iii) hold.  

\begin{clm} $A$ is a closed operator and $\lambda_0 \in \resolventset{A}$.
\end{clm}
By (iii) we know $\range{\lambda_0 -A} = X$.  This implies $\range{\lambda_0 -A}$ is closed and since $A$ is dissipative Proposition \ref{DissipativeAndClosed} implies that $A$ is closed.  The fact that $A$ is dissipative implies that $\lambda_0 -A$ is injective (in fact for all $\mu > 0$, if $(\mu -A)v=0$ then $\norm{v} \leq \mu^{-1} \norm{(\mu -A)v} = 0$).  The fact that $A$ is dissipative implies that for all $v \in X$, $\norm{R_{\lambda_0} v} \leq \lambda_0^{-1} \norm{(\lambda_0 -A)R_{\lambda_0} v} = \lambda_0^{-1} v$ hence $R_{\lambda_0}$ is a bounded linear operator.  Therefore we have shown that $\lambda_0 \in \resolventset{A}$.

From the claim and Lemma \ref{ClosedDissipativeResolventSet} we know that $(0,\infty) \subset \resolventset{A}$.  This fact, the fact that $A$ is closed and dissipative and (i) means that we can apply the Yosida approximation Lemma \ref{ClosedDissipativeYosidaApproximation}.  For $\lambda>0$ define $A_\lambda =\lambda A R_\lambda$ and $T^\lambda_t = e^{tA_\lambda}$ so that $T^\lambda_t$ is a strongly continuous contraction semigroup.  By application of Lemma \ref {SemigroupBoundInTermsOfBoundedGenerator} we get the inequality
\begin{align}\label{HilleYosidaBoundSemigroupByGenerator}
\norm{T^\lambda_tv  - T^\mu_tv} &\leq t\norm{A_\lambda v - A_\mu v} \text{ for all $t\geq 0$ and $v \in X$}
\end{align}
Lemma \ref{ClosedDissipativeYosidaApproximation} also tells us that $\lim_{\lambda \to \infty} A_\lambda v = Av$ for all $v \in \domain{A}$.  Therefore the sequence $A_n v$ is Cauchy
hence \eqref{HilleYosidaBoundSemigroupByGenerator} tells us that $T^n_t v$ is uniformly Cauchy on every bounded interval of $[0, \infty)$ and therefore converges uniformly on bounded intervals to a continuous function $T_t v$ (recall that $T^n_t v$ is continuous by Proposition \ref{StronglyContinuousSemigroupContinuousPaths} and Lemma \ref{UniformConvergenceOnCompacts}).

\begin{clm} For every fixed $v \in X$, $T^n_t v$ converges in $C([0,\infty); X)$.
\end{clm}
By (i) we can find $v_m \in \domain{A}$ such that $\lim_{m \to \infty} v_m = v$.  
Now by the triangle inequality, \eqref{HilleYosidaBoundSemigroupByGenerator} and the fact that $T^\lambda_t$ is a contraction semigroup for all $\lambda > 0$, for all $m \in \naturals$
\begin{align*}
\norm{T^\lambda_t v - T^\mu_t v} &\leq  \norm{T^\lambda_t v - T^\lambda_t v_m}   + \norm{T^\lambda_t v_m - T^\mu_t v_m} + \norm{T^\mu_t v_m - T^\mu_t v} \\
&\leq 2 \norm{v -v_m} + t \norm{A_\lambda v_m - A_\mu v_m}
\end{align*}
The fact that $v_m \in \domain{A}$ and Lemma \ref{ClosedDissipativeYosidaApproximation} tell us that $\lim_{\lambda \to \infty} A_\lambda v_m = Av_m$  for all $m \in \naturals$. Thus for every $T \geq 0$ and $\epsilon > 0$ we can find $m \in \naturals$ such that $\norm{v -v_m} \leq \epsilon/4$ and a $N \geq 0$ such that $\norm{A_\lambda v_m - A_\mu v_m} \leq \epsilon/2T$ for $\lambda, \mu \geq N$ which shows
\begin{align*}
\sup_{0 \leq t \leq T} \norm{T^\lambda_t v - T^\mu_t v} \leq 2 \norm{v -v_m} + T \sup_{0 \leq t \leq T} \norm{A_\lambda v_m - A_\mu v_m} \leq 2 \norm{v -v_m} \leq \epsilon
\end{align*}
which shows that $T^\lambda_t v$ is Cauchy in $C([0,T];X)$ and therefore converges in $C([0,T];X)$.  It follows from Lemma \ref{UniformConvergenceOnCompacts} that $T^\lambda_t v$ converges in $C([0,\infty); X)$.

Now by the claim, for every $v \in X$ we have a $T_t v \in C([0,\infty); X)$ such that $T^\lambda_t v \to T_t v$ in $C([0,\infty); X)$.  
\begin{clm} $T_t$ is a strongly continuous contraction semigroup.
\end{clm}
To see the semigroup property we let $v \in X$ we use the fact that $T^\lambda_t v \to T_t v$ for every $0 \leq t < \infty$ and the fact that $\norm{T^\lambda_t} \leq 1$ to see
\begin{align*}
\norm{T_{s+t} v - T_s T_t v} &\leq \lim_{\lambda \to \infty} \left [\norm{T_{s+t} v - T^\lambda_{s+t} v} + \norm{T^\lambda_s T^\lambda_t v - T_s T_t v}  \right] \\
&\leq \lim_{\lambda \to \infty} \left[ \norm{T_{s+t} v - T^\lambda_{s+t} v } + \norm{T^\lambda_s (T^\lambda_t v - T_t v)} + \norm{T^\lambda_s T_t v - T_s T_t v} \right] \\
&\leq \lim_{\lambda \to \infty} \left[ \norm{T_{s+t} v - T^\lambda_{s+t} v } + \norm{T^\lambda_t v - T_t v} + \norm{T^\lambda_s T_t v - T_s T_t v} \right] \\
&=0
\end{align*}
The contraction property of $T_t$ follows easily from the contraction property of $T^\lambda_t$,
\begin{align*}
\norm{T_t v} &\leq \lim_{\lambda \to \infty} \left[\norm{T^\lambda_t v} + \norm{T_tv - T^\lambda_t v} \right] leq \norm{v} + \lim_{\lambda \to \infty} \norm{T_tv - T^\lambda_t v} = \norm{v}
\end{align*}
For strong continuity we need to use the full power of the fact that $T^\lambda_t v \to T_t v$ in $C([0,\infty); X)$:
\begin{align*}
\lim_{t \to 0} \norm{T_t v - v} &\leq \lim_{\lambda \to \infty} \lim_{t \to 0} \left[ \norm{T^\lambda_t v - v} + \norm{T_t v - T^\lambda_t v} \right] \\
&\leq \lim_{\lambda \to \infty} \lim_{t \to 0} \norm{T^\lambda_t v - v} + \lim_{\lambda \to \infty} \sup_{0 \leq t \leq 1} \norm{T_t v - T^\lambda_t v} \\
&= 0
\end{align*}

\begin{clm} $A$ is the generator of $T_t$.
\end{clm}
By the Kolomogorov backward equation (Proposition \ref{StronglyContinuousSemigroupKolomgorovBackwardEquation}) and the fact that $\domain{A_\lambda} = X$ we have for all $v \in X$,
\begin{align*}
T^\lambda_t v - v &= \int_0^t T^\lambda_s A_\lambda v \, ds
\end{align*}
If we assume that $v \in \domain{A}$ and $T > 0$ then using the fact that $\norm{T^\lambda_t} \leq 1$, the Yosida approximation and the definition of $T_t$,
\begin{align*}
\lim_{\lambda \to \infty} \sup_{0 \leq t \leq T} \norm{T^\lambda_t A_\lambda v - T_t A v} 
&\leq \lim_{\lambda \to \infty} \sup_{0 \leq t \leq T} \left[ \norm{T^\lambda_t A_\lambda v - T^\lambda_t A v} + \norm{T^\lambda_t A v - T_t A v} \right]\\
&\leq \lim_{\lambda \to \infty} \norm{A_\lambda v - A v} + \lim_{\lambda \to \infty} \sup_{0 \leq t \leq T} \norm{T^\lambda_t A v - T_t A v}  \\
&=0
\end{align*}
From the uniform convergence of $T^\lambda_t A_\lambda v $ on compacts we conclude 
\begin{align*}
\lim_{\lambda \to \infty} \int_0^t T^\lambda_s A_\lambda v \, ds &= \int_0^t T_s A v \, ds
\end{align*}
and therefore the Kolmogorov backward equation is established for $T_t$ and $A$:
\begin{align*}
T_t v - v &= \int_0^t T_s A v \, ds \text{ for $v \in \domain{A}$}
\end{align*}
The fact that $\lim_{t \to 0} t^{-1} (T_t v -v) =Av$ for all $v \in \domain{A}$ follows from the Fundamental Theorem of Calculus and the fact that $T_0 = \IdentityMatrix$.  If $B$ is the 
generator of $T_t$ we know that $A$ and $B$ agree on $\domain{A}$ thus it remains to show that $\domain{A}=\domain{B}$.  Suppose there exists $v \in \domain{B} \setminus \domain{A}$.  By (iii) we know that there is $\lambda > 0$ and $w \in \domain{A}$ such that $(\lambda - A) w = (\lambda - B) v$.  On the other hand $(\lambda - A) w =(\lambda - B) w$ and by (ii) we know that $\lambda - B$ is injective which is a contradiction.
\end{proof}

TODO:  Show this prior to proving Hille-Yosida and incorporate the next corollary into the statement of Hille-Yosida.  
The next important fact that we want to demonstrate is that the generator uniquely identifies a strongly continuous contraction semigroup.  That fact (as well as a few others) is a corollary of the following technical lemma.
\begin{lem}\label{DissipativeContractivePath}Let $A$ be a dissipative linear operator on $X$, let $u : [0,\infty) \to X$ be a continuous path in $X$ such that $u(t) \in \domain{A}$ for all $t > 0$, $A u : (0,\infty) \to X$ is continuous and 
\begin{align}\label{EpsilonBackwardEquation}
u(t) &= u(\epsilon) + \int_\epsilon^t A u(s) \, ds \text{ for all $0 < \epsilon < t$}
\end{align}
Then $\norm{u(t)} \leq \norm{u(0)}$.
\end{lem}
\begin{proof}
Since $Au(s)$ is uniformly continuous on $[\epsilon, t]$ given any $\delta > 0$ we may choose partition $0< \epsilon=t_0 < t_1 < \dotsb < t_n =t$ such that $\max_{1 \leq i \leq n} \sup_{t_{i-1} \leq s \leq t_i} \norm{A u(s) - A(t_i)} \leq \delta$.  Then writing a telescoping sum, using eqref \ref{EpsilonBackwardEquation}, the dissipative property (in the form $\norm{v} \leq \norm{v - \lambda^{-1}A v}$) and the triangle inequality 
\begin{align*}
\norm{u(t)} &= \norm{u(\epsilon)} + \sum_{i=1}^n \left [ \norm{u(t_i)} - \norm{u(t_{i-1})} \right] \\
&=\norm{u(\epsilon)} + \sum_{i=1}^n \left [ \norm{u(t_i)} - \norm{u(t_i) - \int_{t_{i-1}}^{t_i} A u(s) \, ds} \right] \\
&\leq \norm{u(\epsilon)} + \sum_{i=1}^n \left [ \norm{u(t_i) - (t_i - t_{i-1}) A u(t_i) } - \norm{u(t_i) - \int_{t_{i-1}}^{t_i} A u(s) \, ds} \right] \\
&\leq \norm{u(\epsilon)} + \sum_{i=1}^n  \norm{\int_{t_{i-1}}^{t_i} A u(s) \, ds - (t_i - t_{i-1}) A(t_i)} \\
&= \norm{u(\epsilon)} + \sum_{i=1}^n\norm{\int_{t_{i-1}}^{t_i} (A u(s) - A(t_i)) \, ds} \\
&\leq \norm{u(\epsilon)} + \sum_{i=1}^n\int_{t_{i-1}}^{t_i} \norm{A u(s) - A(t_i)} \, ds \\
&\leq \norm{u(\epsilon)} + (t - \epsilon) \max_{1 \leq i \leq n} \sup_{t_{i-1} \leq s \leq t_i} \norm{A u(s) - A(t_i)}  \\
&\leq \norm{u(\epsilon)} + (t - \epsilon) \delta \\
\end{align*}
As $\delta>0$ was arbitrary we may take the limit as $\delta \to 0$ conclude that $\norm{u(t)} \leq \norm{u(\epsilon)}$.  We may also take the limit as $\epsilon \to 0$ and use the continuity of $u$ at $0$ to conclude $\norm{u(t)} \leq \norm{u(0)}$.
\end{proof}

\begin{cor}\label{SCCSGeneratorDeterminesSemigroup}Let $T_t$ and $S_t$ be strongly continuous contraction semigroups both with generator $A$ then it follows that $T_t=S_t$ for all $t \geq 0$.
\end{cor}
\begin{proof}
By Example \ref{SCCSGeneratorDissipative} we know that $A$ is dissipative.  Let $v \in \domain{A}$.  Since $\domain{A}$ is dense (Corollary \ref{StronglyContinuousSemigroupGeneratorClosedDomainDense}) and $T_t$ and $S_t$ are bounded operators it suffices to show that $T_t v = S_t v$ for all $t \geq 0$.  By Proposition \ref{StronglyContinuousSemigroupContinuousPaths} we know that $T_tv$ and $S_tv$ are continuous.  By Proposition \ref{StronglyContinuousSemigroupKolomgorovBackwardEquation} we know that $A T_t v = T_t Av$ and $A S_t v = S_t Av$ hence $A T_t v$ and $A S_t v$ are continuous, that $T_tv, S_tv \in \domain{A}$ for all $t \geq 0$ and that $T_tv - S_tv = \int_0^t A (T_s v - S_s v) \, ds$ for all $t \geq 0$.  Thus applying Lemma \ref{DissipativeContractivePath} to the path
$T_t v - S_t v$ we see that $\norm{T_t v - S_t v} \leq \norm{T_0 v - S_o v}  = 0$ for all $t \geq 0$.  
\end{proof}

Now that we know that the generator determines the semigroup we know that a strongly continuous contraction semigroup is equal to the limit of the exponential of the Yosida approximations of its generators.  Moreover 
\begin{cor}\label{YosidaApproximationContinuity}Let $T_t$ be a strongly continuous contraction semigroup on $X$ with generator $A$  and let $A_\lambda$ be the Yosida approximation of $A$, then for all $v \in \domain{A}$ and $\lambda > 0$ 
\begin{align*}
\norm{e^{tA_\lambda} - T_t v} \leq t \norm{A_\lambda v - Av}
\end{align*}
and therefore for all $v \in X$ and $0 \leq t < \infty$, $\lim_{\lambda \to \infty} e^{t A_\lambda} = T_t v$ uniformly on bounded intervals.
\end{cor}
\begin{proof}
By Corollary \ref{SCCSGeneratorDeterminesSemigroup} and the proof of the Hille-Yosida theorem we know that $T_t v = \lim_{\mu \to \infty} e^{t A_\mu} v$ for all $v \in X$ and $0 \leq t < \infty$ as noted in the proof that convergence is in $C([0,\infty); S)$ for fixed $v \in X$.  From \eqref{HilleYosidaBoundSemigroupByGenerator} in the proof of Hille-Yosida theorem we know that for all $t\geq 0$ and $v \in X$
\begin{align*}
\norm{e^{\lambda t}v  - T_tv} &\leq \norm{e^{\lambda t}v  - e^{\mu t} v} + \norm{e^{\mu t}v  - T_tv} \\
&\leq t\norm{A_\lambda v - A_\mu v} + \norm{e^{\mu t}v  - T_tv} \\
&\leq t\norm{A_\lambda v - A v} + t\norm{A v - A_\mu v} + \norm{e^{\mu t}v  - T_tv} \\
\end{align*}
now let $\mu \to \infty$ and use the facts that $\lim_{\mu \to \infty} A_\mu v = Av$ and $\lim_{\mu \to \infty} e^{\mu t}v  = T_tv$.  
\end{proof}

The following corollary is a technical device that allows one to show that a semigroup preserves a subset by considering the behavior of its
generator.  In particular, we will be able to use this to show that a semigroup comprises positive operators on function spaces (i.e. the semigroup preserves
the positive cone of functions $f \geq 0$) by showing positivity of the resolvents.
\begin{cor}\label{SCCSInvarianceFromResolventInvariance}Let $T_t$ be a strongly continuous contraction semigroup on $X$ with generator $A$, let $Y \subset X$ and 
\begin{align*}
\Lambda_Y &= \lbrace \lambda > 0 \mid \lambda (\lambda - A)^{-1} (Y) \subset Y \rbrace
\end{align*}
If either
\begin{itemize}
\item[(i)] $Y$ is a closed convex subset of $X$ and $\Lambda_Y$ is unbounded
\item[(ii)] $Y$ is a closed subspace of $X$ and $\Lambda_Y$ is nonempty
\end{itemize}
then $T_t (Y) \subset Y$ for all $0 \leq t < \infty$.
\end{cor}
\begin{proof}
TODO:
\end{proof}

Sometimes it is inconvenient to deal with the full domain of a generator.  In particular, $\domain{A}$ might be too big in the sense that one lacks a clean characterization of the elements it contains.  In these situations it is often the case that there there is a subset of $\domain{A}$ which is easy to identify and which is big enough to capture all of the
information of $A$ in the sense the pair of $(A, \domain{A})$ is a limit of the restriction to the subset.  
\begin{defn}A linear operator is said to be \emph{closable} if it has a closed linear extension.  The smallest closed linear extension of $A$ is called the \emph{closure} of $A$.  Given a closable operator $A$ the closure of $A$ is denoted $\overline{A}$.
\end{defn}

\begin{prop}Let $A$ be a closable linear operator then the closure of the graph of $A$ defines a single valued closed linear operator $\overline{A}$.  Any closed extension of $A$ is an extension of $\overline{A}$.
\end{prop}
\begin{proof}
Suppose $B$ is a closed linear extension of $A$.  If we have a sequence $(v_n, A v_n)$ with $v_n \in \domain{A}$ that converges in $X \times X$ to $(v,w)$ then since $B$ is a closed extension of $A$ we know that $v \in \domain{B}$ and $w = B v$.  Thus if $(v,w)$ and $(v,u)$ are in the closure of the graph of $A$ we have $w = u = B v$; hence the closure of the graph of $A$ defines a function $\overline{A} : X \to X$.  The fact that $\overline{A}$ is linear follows from the linearity of limits and the linearity of $A$; pick $(v_n, A v_n) \to (v, Av)$ and $(w_n, A w_n) \to (w, A w)$ then 
\begin{align*}
(a v + b w, \overline{A}(a v + b w)) 
&=\lim_{n \to \infty} (a v_n + b w_n, A(a v_n + b w_n)) 
= \lim_{n \to \infty} (a v_n + b w_n, a A v_n + b A w_n) 
= (a v + b w, a \overline{A} v + b \overline{A} w) 
\end{align*}

If we let $C$ be any closed extension of $A$ and $v \in \domain{\overline{A}}$ then we may pick a sequence $v_n \in \domain{A}$ such that $\lim_{n \to \infty} (v_n, A v_n) = (v, \overline{A} v)$; but since $C$ is an extension of $A$, $(v_n, A v_n) = (v_n, C v_n)$ converges in $X \times X$ and therefore since $C$ is closed $(v, \overline{A} v) =  (v, C v)$ which shows that $C$ is an extension of $\overline{A}$.
\end{proof}

Our goal is to provide an alternative statement of the Hille-Yosida theorem in terms of closable operators rather than closed operators. We need to do a small bit of work to examine 
the interactions of some of our existing concepts with the new concept of the closure of a closable operator.

\begin{lem}\label{ClosableDissipativeClosure}Suppose $A$ is a closable linear operator then $A$ is dissipative if and only if $\overline{A}$ is dissipative.
\end{lem}
\begin{proof}
First suppose $\overline{A}$ is dissipative.  Then using the fact that $\overline{A}$ is an extension of $A$, for any $\lambda > 0$ and $v \in \domain{A}$
\begin{align*}
\norm{(\lambda - A)v} &= \norm{(\lambda - \overline{A})v} \geq \lambda \norm{v}
\end{align*}
On the other hand if $A$ is dissipative then for any $v \in \domain{\overline{A}}$ we may pick a sequence $v_n \in \domain{A}$ such that $\lim_{n \to \infty} (v_n,A v_n) = (v, \overline{A}v)$ and therefore for every $\lambda > 0$,
\begin{align*}
\norm{ (\lambda - \overline{A}) v} &= \lim_{n \to \infty} \norm{ (\lambda - A) v_n} \geq \lambda \lim_{n \to \infty} \norm{ v_n} = \lambda \norm{v} 
\end{align*}
\end{proof}

\begin{prop}\label{DissipativeDenseDomainImpliesClosable}Let $A$ be dissipative with $\domain{A}$ dense then $A$ is closable and $\overline{\range{\lambda -A}} = \range{\lambda - \overline{A}}$ for all $\lambda > 0$.
\end{prop}
\begin{proof}
\begin{clm} $A$ is closable
\end{clm}
Pick a sequence $v_n \in \domain{A}$ such that $\lim_{n \to \infty} v_n = 0$ and $\lim_{n \to \infty} A v_n = w$.  Since $\domain{A}$ is dense we may pick a sequence $w_n \in \domain{A}$ such that $\lim_{n \to \infty} w_n = w$.  Therefore for all $m \in \naturals$ and $\lambda > 0$ using the continuity of norms and the fact that $A$ is dissipative,
\begin{align*}
\norm{(\lambda - A) w_m - \lambda w} &= \lim_{n \to \infty} \norm{(\lambda - A) w_m - (\lambda - A) \lambda v_n} \\
&\geq \lambda \lim_{n \to \infty} \norm{w_m - \lambda v_n} = \lambda \norm{w_m}
\end{align*}
so after dividing by $\lambda$, and taking limits
\begin{align*}
\norm{w} &= \lim_{m \to \infty} \norm{w_m} \leq \lim_{m \to \infty} \lim_{\lambda \to \infty}\norm{ (\IdentityMatrix - \lambda^{-1} A) w_m - w} \\
&\leq \lim_{m \to \infty} \lim_{\lambda \to \infty} \left[ \norm{w_m - w} + \lambda^{-1} \norm{A w_m} \right] = 0
\end{align*}
and therefore $w = 0$.  

Now to see that $A$ is closable suppose that $(v_n, A v_n)$ and  $(u_n, A u_n)$ are convergent sequences in $X \times X$ with $v_n, u_n \in \domain{A}$ and $\lim_{n \to \infty} v_n = \lim_{n \to \infty} u_n$.  Then if follows that 
$\lim_{n \to \infty} (v_n - u_n) = 0$ and therefore by the preceeding argument $\lim_{n \to \infty} A(v_n - u_n) = \lim_{n \to \infty} Av_n - \lim_{n \to \infty}  A u_n = 0$; which is to say that $A$ is closable.

\begin{clm} $\range{\lambda - \overline{A}} \subset\overline{\range{\lambda -A}}$
\end{clm}
Now suppose that $w \in \range{\lambda - \overline{A}}$ and pick $v \in \domain{\overline{A}}$ with $(\lambda - \overline{A})v = w$.  By definition of $\overline{A}$ we may pick
a sequence $v_n \in \domain{A}$ such that $\lim_{n \to \infty} v_n$ and $\lim_{n \to \infty} A v_n = \overline{A} v = \lambda v - w$.  Therefore $w = \lim_{n \to \infty} (\lambda v_n - A v_n)$
which shows that $w \in \overline{\range{\lambda -A}}$.  

\begin{clm} $\overline{\range{\lambda -A}} \subset\range{\lambda - \overline{A}} $
\end{clm}
By Lemma \ref{ClosableDissipativeClosure} we know that $\overline{A}$ is dissipative and closed and 
by Proposition \ref{DissipativeAndClosed} we know this implies that $\range{\lambda - \overline{A}}$ is closed.  Since $\overline{A}$ is an
extension of $A$ we know that $\range{\lambda - A} \subset \range{\lambda - \overline{A}}$.  The claim follows by combining these two observations and taking closures.
\end{proof}

We can have enough to state a prove an alternative formulation of the Hille-Yosida theorem in terms of closable operators.
\begin{thm}[Hille-Yosida Theorem for Closable Operators]\label{HilleYosidaTheoremClosable}Let $X$ be a Banach space, then a linear operator $A : X \to X$ is closable with $\overline{A}$ the generator of a strongly continuous contraction semigroup if and only if 
\begin{itemize}
\item[(i)] $ \domain{A}$ is dense in $X$
\item[(ii)] $A$ is dissipative
\item[(iii)] $\range{\lambda_0 -A}$ is dense in $X$ for some $\lambda_0 > 0$.
\end{itemize}
\end{thm}
\begin{proof}
Suppose $A$ is closable and $\overline{A}$ is generator of a strongly continuous contraction semigroup.  Then by the Hille-Yosida Theorem \ref{HilleYosidaTheorem} we know that $\domain{\overline{A}}$ is dense in $X$, $\overline{A}$ is dissipative and $\range{\lambda_0 - \overline{A}} = X$ for some $\lambda_0 > 0$.  By Lemma \ref{ClosableDissipativeClosure} we know that $A$ is dissipative.   To see that $\domain{A}$ is dense,  just note that for any $\epsilon > 0$ and $v \in X$ we may find a $w \in \domain{\overline{A}}$ such that $\norm{v - w} \leq \epsilon/2$ but also that we may find a $u \in \domain{A}$ such that $\norm{w - u} \leq \epsilon/2$ (and $\norm{\overline{A}w - Au} \leq \epsilon/2$ although we don't need this fact).  Now that we know $\domain{A}$ is dense it follow from Proposition \ref{DissipativeDenseDomainImpliesClosable} that $\overline{\range{\lambda_0 - A}} = \range{\lambda_0 - \overline{A}}=X$.

On the other hand, suppose that $A$ satisfies (i), (ii) and (iii).  By Proposition \ref{DissipativeDenseDomainImpliesClosable} we know that $A$ is closable.  Then by Lemma \ref{ClosableDissipativeClosure} we know that $\overline{A}$ is dissipative.  It is immediate from (i) and the fact that $\domain{A} \subset \domain{\overline{A}}$ that $\domain{\overline{A}}$ is dense.  Lastly by (i), (ii) and (iii) we may apply Proposition \ref{DissipativeDenseDomainImpliesClosable} implies that $\range{\lambda_0 - \overline{A}} = \overline{\range{\lambda_0 - A}} =X$.  Thus we may apply the Hille-Yosida Theorem \ref{HilleYosidaTheorem} to conclude that $\overline{A}$ is the generator of a strongly continuous contraction semigroup.
\end{proof}

\subsection{Feller Semigroups}

TODO: Define Feller processes and use the Yosida approximation to show that they are strongly continuous.

Having presented the theory of semigroups in a generic Banach space setting we now specialize back to the case of interest for Markov processes.  In fact we specialize the state space and assume that $S$ is a locally compact separable metric space and consider the action on the Banach space of continuous functions vanishing at infinity (see Proposition 
\ref{BanachSpaceOfFunctionsVanishingAtInfinity}).  Recall that semigroups We want to apply the Hille-Yosida theorem  in the special case and as a new property makes an appearance.

\begin{defn}Let $S$ be a topological space then a semigroup $T_t : C_0(S) \to C_0(S)$ is said to be \emph{positive} if for every $f \geq 0$ we have $T_t f \geq 0$ for all $t \geq 0$. 
A positive strongly continuous contraction semigroup on $C_0(S)$ is called a \emph{Feller semigroup}.   An unbounded operator $A : C_0(S) \to C_0(S)$ is said to satisfy the \emph{positive maximum principle} if given $f \in \domain{A}$ and $x_0 \in S$ satisfying $\sup_{x \in S} f(x) = f(x_0) \geq 0$ we have $Af(x_0) \leq 0$.
\end{defn}


\begin{lem}\label{PositiveMaximumPrincipleDissipative}Let $S$ be locally compact Hausdorff and suppose that $A$ satisfies the positive maximum principle then $A$ is dissipative.
\end{lem}
\begin{proof}
Let $f \in \domain{A}$ and suppose $\lambda > 0$.  Using Corollary \ref{VanishingAtInfinityLocallyCompactAttainsNormInfSup} pick $x_0 \in S$  such that $\abs{f(x_0)} = \sup_{x \in S} \abs{f(x)}$.  First suppose that $f(x_0) \geq 0$.  It follows that $f(x_0) \geq f(x)$ for all $x \in S$ and from the positive maximum principle that $Af(x_0) \leq 0$, thus
\begin{align*}
\norm{\lambda f - A f} &\geq \abs{\lambda f(x_0) - Af(x_0)} = \lambda f(x_0) - Af(x_0) \geq \lambda f(x_0) = \lambda \norm{f}
\end{align*}
If $f(x_0) < 0$ then we can apply the same argument to $-f$.  It follows that $A$ is dissipative.
\end{proof}

\begin{thm}\label{HilleYosidaTheoremFellerSemigroup} Let $S$ be a locally compact separable metric space then $A: C_0(S) \to C_0(S)$ is closable and generates a Feller semigroup if and only if 
\begin{itemize}
\item[(i)] $ \domain{A}$ is dense in $X$
\item[(ii)] $A$ satisfies the positive maximum principle
\item[(iii)] $\range{\lambda_0 -A}$ is dense in $X$ for some $\lambda_0 > 0$.
\end{itemize}
\end{thm}
\begin{proof}
If $\overline{A}$ generates a Feller semigroup then we can apply the Hille-Yosida Theorem \ref{HilleYosidaTheoremClosable} to conclude (i) and (iii).  Moreover we know that $A$ is dissipative.  Suppose $f \in \domain{A}$  and $f(x_0) = \sup_{x \in S} f(x) \geq 0$.  Using positivity and the contraction property we get for all $t \geq 0$,
\begin{align*}
T_t f (x_0) &\leq T_t f_+(x_0) \leq \norm{f_+} = f(x_0)
\end{align*}
and therefore 
\begin{align*}
Af(x_0) &= \lim_{t \to 0} \frac{T_t f(x_0) - f(x_0)}{t} \leq 0
\end{align*}
and it follows that $A$ satisfies (ii).

On the other hand, suppose $A$ satisfies (i), (ii) and (iii) by Lemma \ref{PositiveMaximumPrincipleDissipative} we know that $A$ is dissipative and therefore we can apply 
Hille-Yosida Theorem \ref{HilleYosidaTheoremClosable} to conclude that $A$ is closable and $\overline{A}$ generates a strongly contractive semigroup $T_t$.  It remains
to prove that $T_t$ is positive.  

\begin{clm}\label{HilleYosidaFeller:ResolventPositivity}For all $f \in \domain{\overline{A}}$ and $\lambda > 0$ if $(\lambda - \overline{A}) f \geq 0$ then $f \geq 0$.
\end{clm}

We argue by contradiction.  Pick $f \in \domain{\overline{A}}$, $\lambda > 0$ and suppose $\inf_{x \in S} f(x) < 0$.  Pick $f_n \in \domain{A}$ such that 
$\lim_{n \to \infty} f_n = f$ and $\lim_{n \to \infty} A f_n = \overline{A} f$.  It follows that $\lim_{n \to \infty} (\lambda - A) f_n = (\lambda - \overline{A}) f$.  Using Corollary \ref{VanishingAtInfinityLocallyCompactAttainsNormInfSup}  for each $n \in \naturals$ we select
$x_n \in S$ such that $f_n(x_n) = \inf_{x \in S} f_n(x)$ and select $x_0 \in S$ such that $f(x_0) = \inf_{x \in S} f(x) < 0$.

We need to piece together some simple facts.
\begin{clm}$\lim_{n \to \infty} f_n(x_n) = f(x_0)$
\end{clm}
From the definition of $x_n$ and $\lim_{n \to \infty} f_n = f$ we see that $\lim_{n \to \infty} f_n(x_n) \leq \lim_{n \to \infty} f_n(x_0) = f(x_0)$.  On the other hand for every $\epsilon > 0$
there exists $N$ such that $\sup_{x \in S} \abs{f_n(x) - f(x)} < \epsilon$ for $n \geq N$ and therefore $f(x_0) - \epsilon \leq f(x_n) - \epsilon \leq f_n(x_n)$ for $n \geq N$.  This implies $f(x_0) - \epsilon \leq \lim_{n \to \infty} f_n(x_n)$ and since $\epsilon >0$ was arbitrary we get $f(x_0) \leq \lim_{n \to \infty} f_n(x_n)$.

Note that exactly the same argument shows that $\lim_{n \to \infty} \inf_{x \in S} (\lambda - A) f_n(x) =  \inf_{x \in S} (\lambda - \overline{A}) f(x)$.

\begin{clm}For $n$ sufficiently large $A f_n (x_n) \geq 0$.
\end{clm}
From the previous claim and the fact that $f(x_0) < 0$ we know that for sufficiently large $n$ we must have $f_n(x_n) \leq 0$ and thus $-f_n(x_n) = \sup_{x \in S} (-f_n(x)) 
\geq 0$ for sufficiently large $n$; by the positive maximum principle applied to $-f_n$ we see that $A f_n (x_n) \geq 0$.  

By the previous claim we get for every $n \in \naturals$,
\begin{align*}
\inf_{x \in S} (\lambda - A) f_n(x) &\leq (\lambda - A) f_n(x_n) \leq \lambda f_n(x_n)
\end{align*}
and therefore taking the limit as $n \to \infty$ we have
\begin{align*}
\inf_{x \in S} (\lambda - \overline{A}) f(x)  &= \lim_{n \to \infty} \inf_{x \in S} (\lambda - A) f_n(x) \leq \lim_{n \to \infty} \lambda f_n(x_n) = \lambda f(x_0) < 0
\end{align*}
and Claim \ref{HilleYosidaFeller:ResolventPositivity} is shown.  

To finish, note that the positive cone $\lbrace f \in C_0(S) \mid f \geq 0 \rbrace$ is closed and convex.  Claim \ref{HilleYosidaFeller:ResolventPositivity} shows that for all $\lambda > 0$ if $f \geq 0$ then $(\lambda - A)^{-1} f \geq 0$ so by Corollary \ref{SCCSInvarianceFromResolventInvariance} we conclude that $T_t$ is positive for all $t \geq 0$.
\end{proof}

As it turns out the strong continuity required in the definition of a Feller semigroup may be derived from a weaker pointwise continuity property.  
\begin{prop}Let $S$ be locally compact Hausdorff and suppose that $T_t : C_0(S) \to C_0(S)$ is a positive contraction semigroup such that
\begin{align*}
\lim_{t \to 0} T_tf (x) = f(x) \text{ for all $f \in C_0(S)$ and $x \in S$}
\end{align*}
then $T_t$ is a Feller semigroup.
\end{prop}
\begin{proof}
TODO: The proof follows by defining the resolvent for $\lambda > 0$ explicitly using the Laplace transform and then using the Yosida approximation.  
\end{proof}

\subsection{Cores}
\begin{defn}Let $A$ be a closed operator then $D \subset \domain{A}$ is a \emph{core} if $\overline{A \mid_D} = A$.  
\end{defn}

\begin{prop}\label{SCCSCoreViaDenseRange}Let $A$ be the generator of a strongly continuous contraction semigroup then $D$ is a core for $A$ if and only if $(\lambda_0 -A)(D)$ is dense for some $\lambda_0 > 0$.  In either case $(\lambda -A)(D)$ is dense for all $\lambda > 0$.
\end{prop}
\begin{proof}
Suppose $D$ is a core then let $v \in X$ then $R_\lambda v \in \domain{A}$ and therefore we can find $w_n \in D$ such that $\lim_{n \to \infty} w_n = R_\lambda v$ and $\lim_{n \to \infty} A w_n = A R_\lambda v$.  Therefore $\lim_{n \to \infty} (\lambda - A) w_n = (\lambda - A) R_\lambda v = v$.  

Suppose $(\lambda_0 -A)(D)$ is dense for some $\lambda_0 > 0$.  Let $v \in \domain{A}$ and choose $v_n \in D$ such that $\lim_{n \to \infty} (\lambda_0 - A) v_n = (\lambda_0 - A) v$.  From Proposition \ref{SCCSResolventAsLaplaceTransform} we know that $R_{\lambda_0}$ is bounded and therefore 
\begin{align*}
\lim_{n \to \infty} v_n &= R_{\lambda_0} \lim_{n \to \infty} (\lambda_0 - A) v_n = R_{\lambda_0}  (\lambda_0 - A) v = v
\end{align*}
from which we get 
\begin{align*}
\lim_{n \to \infty} A v_n &= \lim_{n \to \infty} \lambda_0 v_n - \lim_{n \to \infty} (\lambda_0 - A) v_n = \lambda_0 v - (\lambda_0 - A) v = Av
\end{align*}

TODO: In EK they add the assumption that $D$ is dense; do we need that or is it a consequence of $(\lambda_0 -A)(D)$ is dense?
\end{proof}

\begin{prop}\label{SCCSCoreViaInvariance}Let $A$ be the generator of a strongly continuous contraction semigroup and let $D_0 \subset D \subset \domain{A}$ be dense subspaces such that $T_t D_0 \subset D$ for all $t \geq 0$ then $D$ is a core.
\end{prop}
\begin{proof}
Let $v \in D_0$, $\lambda > 0$ and define
\begin{align*}
v_n &= \frac{1}{n} \sum_{k=0}^{n^2} e^{-\lambda k/n} T_{k/n} v \in D
\end{align*}
for all $n \in \naturals$.
\begin{align*}
\lim_{n \to \infty} (\lambda - A) v_n &= \lim_{n \to \infty} \frac{1}{n} \sum_{k=0}^{n^2} e^{-\lambda k/n} T_{k/n}(\lambda - A)  v \\
&=\int_0^\infty e^{-\lambda t} T_t (\lambda -A) v \, dt = R_\lambda (\lambda -A) v = v
\end{align*}
which shows us that $D_0 \in \overline{(\lambda - A)(D)}$.  Now by density of $D_0$ we see that $(\lambda - A)(D)$ is dense and by Proposition \ref{SCCSCoreViaDenseRange} we see that $D$ is a core.
\end{proof}

\section{Existence of Feller Processes}

Our next goal is to show that there is a cadlag Feller process associated with any Feller semigroup.  We begin with some motivational comments.  It isn't too hard to come up with a rough sketch for how to proceed:  
\begin{itemize}
\item since the semigroup is on the Banach space $C_0(S)$ for each $x \in S$ and $t \geq 0$ we get a positive linear functional $\lambda_{t,x} f = T_t f (x)$
\item by the Riesz Representation Theorem we therefore get transition measures $\mu_{t}(s, \cdot)$ each of which is a finite Radon measure
\item the semigroup property implies the Chapman Kolmogorov relations thus we can assert the existence of a Markov process with transition measures $\mu_t(x, \cdot)$.
\end{itemize}
A couple of things need to be sorted out.  We have not come up with an obvious plan for how to prove that there is a cadlag modification of the constructed Markov process.  Indeed
this requires some real ingenuity and work.  The second issue is much simpler; in the second step of our plan we mentioned that the Riesz Representation Theorem will only guarantee that we have a finite Radon measure but not necessarily a probability measure.  Indeed the contraction property of $T_t$ shows that the total mass of the measure is less than or equal to one (it is equal to $\norm{\lambda_{t,x}}$); however, in our definition of a Feller semigroup we have not introduced the condition that is necessary to guarantee that the constructed transition measures are probability measures.  From a Markov process point of view, we have not introduced the condition that prevents explosion.  We do that now.

\begin{defn}A Feller semigroup is \emph{conservative} if and only if $\sup_{f \leq 1} T_t f(x) = 1$ for all $x \in S$ and all $t \geq 0$.
\end{defn}
Morally we want the property of being conservative to simply say that $T_t 1 = 1$ for all $t \geq 0$ but we have chosen to define a Feller semigroup as being defined only on $C_0(S)$ and $1$ does not vanish at infinity unless $S$ is compact.  As we will see it is in fact possible to extend every Feller semigroup to the entire space $B_b(S)$ in which case the property $T_t 1 = 1$ for all $t \geq 0$ is a valid (and equivalent) definition of conservativeness.  We don't simply proceed by tacking on the hypothesis that $T_t$ is conservative, rather we show that every Feller semigroup can naturally be extended to a conservative Feller semigroup.  From a transition measure point of view we add a new state to $S$ and put the missing mass there so that every transition measure becomes a probability measure.  From a Markov process point of view we are adding an absorbing state to which explosions can transition.

We append a state $\Delta$ to the state space $S$.  The manner in which this is done depends on whether $S$ is compact or not.  If $S$ is compact then we
make $\Delta$ an isolated point of $S^{\Delta} = S \cup \lbrace \Delta \rbrace$ otherwise we let $\Delta$ be the point at infinity in the one point compactification of $S$ (Definition \ref{OnePointCompactificationDefinition}).  In the latter case, recall from Theorem \ref{OnePointCompactification} that $S^\Delta$ is compact and from Proposition \ref{IsometricEmbeddingVanishingAtInfinityIntoCompact} that we can regard every element of $f \in C_0(S)$ isometrically as an element of $C(S^\Delta)$ by defining $f(\Delta) = 0$.  In the former case we leave it to the reader to verify both properties (they follow immediately from the fact that $\lbrace \Delta \rbrace$ is an open set).  It what follows we will freely identify $C_0(S)$ with the subspace of $f \in C(S^\Delta)$ for which $f(\Delta) = 0$.

\begin{prop}\label{FellerSemigroupCompactification}Let $S$ be locally compact separable metric space and let $T_t$ be a Feller semigroup on $S$.  Define $T^\Delta_t$ on $C(S^\Delta)$ by
\begin{align*}
T^\Delta_t f &= f(\Delta) + T_t (f - f(\Delta))
\end{align*}
then $T^\Delta_t$ is a conservative Feller semigroup on $S^\Delta$.
\end{prop}
\begin{proof}
First note that for any $f \in C(S^\Delta)$, $T^\Delta_t f (\Delta) = f(\Delta) + T_t (f - f(\Delta)) (\Delta) = f(\Delta)$ since $T_t (f - f(\Delta)) \in C_0(S)$.
\begin{clm} $T^\Delta_t$ is a semigroup
\end{clm}
To see the semigroup property of $T^\Delta_t$ use the semigroup property of $T_t$ and the fact that $T^\Delta_s f(\Delta) = f(\Delta)$,
\begin{align*}
T^\Delta_{t + s} f &= f(\Delta) + T_{t + s} (f - f(\Delta)) = f(\Delta) + T_{t} T_{s} (f - f(\Delta)) \\
&=f(\Delta) + T_t (T^\Delta_{s} f - f(\Delta)) = T^\Delta_s f(\Delta) + T_{t} (T^\Delta_{s} f - T^\Delta_s f(\Delta))) = T^\Delta_t T^\Delta_s f
\end{align*}
and also
\begin{align*}
T^\Delta_0 f &= f(\Delta) + T_0 (f - f(\Delta)) = f(\Delta) + f - f(\Delta) = f
\end{align*}

Strong continuity of $T^\Delta_t$ follows easily from strong continuity of $T_t$,
\begin{align*}
\lim_{t \to 0} T^\Delta_t f &= f(\Delta) + \lim_{t \to 0} T_t (f - f(\Delta)) = f(\Delta) + f - f(\Delta) =f
\end{align*}

\begin{clm}$T^\Delta_t$ is positive
\end{clm}
Let $f \in C_0(S)$ and pick $\alpha \in \reals$ such that $\alpha + f \geq 0$.  Note that $\alpha + f \in C(S^\Delta)$ and by definition $T^\Delta_t (\alpha + f) = \alpha + T_t f$; also note that every $g \in C(S^\Delta)$ may be written in the form $\alpha + f$ for $f \in C_0(S)$.
Write $f = f_+ - f_-$ with $f_\pm \geq 0$ so that $T_t f_\pm \geq 0$.  Therefore $f_- + f = f_- + (f_+ - f_-) = f_+$ and positivity of $T_t$ implies $-T_t f \leq T_t f_-$.  Therefore
\begin{align*}
(T_t f)_- &= (-T_t f \vee 0) \leq (T_t f_- \vee 0) = T_t f_-
\end{align*}
Since $T_t$ is a contraction, $\norm{T_t f_-} \leq \norm{f_-} \leq \alpha$ and therefore $(T_tf)_- \leq \alpha$ which imples $\alpha + T_t f \geq 0$.

To see that $T^\Delta_t$ is a contraction, note that since $\norm{f} \pm f \geq 0$, by positivity $\norm{T^\Delta_t f} \leq T^\Delta_t \norm{f} = \norm{f}$ and therefore $\norm{T_t} \leq 1$.

To see that $T^\Delta_t$ is conservative from $T^\Delta_t 1 = 1$ and the positivity of $T^\Delta_t$,
\begin{align*}
\sup_{f \leq 1} T^\Delta_t f \leq T^\Delta_t 1 = 1
\end{align*}
\end{proof}

Now we can show that a unique transition kernel can be associated with
any Feller semigroup.  By the application Daniell-Kolmogorov Theorem an associated
Markov process exists as well.

\begin{prop}\label{FellerSemigroupToTransitionKernel}Let $S$ be locally compact separable metric space and let $T_t$ be a Feller semigroup on $S$, then there exists
a unique homogeneous transition kernel $\mu_t$ on $S^\Delta$ such that
\begin{align*}
T_t f(x) &= \int f(s) \mu_t(x, ds) \text{ for all $f \in C_0(S)$ and $x \in S$}
\end{align*}
and such that $\mu_t(\Delta, \lbrace \Delta \rbrace) =1$ for all $t \geq 0$.
\end{prop}
Moreover there exists a homogeneous Markov process $X$ with transition kernel $\mu_t$ and semigroup $T_t$.
\begin{proof}
Let $T^\Delta_t$ be the conservative Feller semigroup on $C(S^\Delta)$ constructed in Proposition \ref{FellerSemigroupCompactification}.  For fixed $x \in S^\Delta$ and $t \geq 0$, $T_t f(x)$ defines a positive linear functional on $C(S^\Delta)$.  By the Riesz-Markov Theorem \ref{RieszMarkov} we know that there exists a Radon measure $\mu_t(x, \cdot)$ such that $T_t f(x) = \int f(s) \, \mu_t(x, ds)$.  Since $T^\Delta_t$ is conservative we have $1 = T_t 1 = \mu_t(x, S)$ and therefore $\mu_t(x, \cdot)$ is a probability measure for all $x \in S^\Delta$ and $t \geq 0$.

To see that $\mu_t$ is a probability kernel first note that by Proposition \ref{StronglyContinuousSemigroupContinuousPaths} we know that $T_t f(x) = \int f(s) \, \mu_t(x, ds)$ is a continuous function of $x$ for every $f \in C(S^\Delta)$ and $t \geq 0$.  Given a open set $U \subset S^\Delta$ we pick a metric $d$ on $S$ and approximate $\characteristic{U}(s) = \lim_{n \to \infty} n d(s, U^c) \wedge 1$ use Dominated Convergence to see that $\mu_t(x, U) = \lim_{n \to \infty} \int (n d(s, U^c) \wedge 1) \, \mu(x, ds)$ and therefore $\mu_t(x, U)$ is measurable in $x$ (Lemma \ref{LimitsOfMeasurable}).  Finally let $t \geq 0$ be fixed and define $\mathcal{C} = \lbrace A \in \mathcal{B}(S) \mid \mu_t(x, A) \text{ is measurable}\rbrace$.  If $A, B \in \mathcal{C}$ and $A \subset B$ then $\mu_t(x, B \setminus A) = \mu_t(x, B) - \mu_t(x, A)$ is measurable hence $B \setminus A \in \mathcal{C}$ and if $A_1 \subset A_2 \subset \cdots$ with $A_n \in \mathcal{C}$ then $\mu_t(x, \cup_n A_n) = \lim_{n \to \infty} \mu_t(x, A_n)$ by Lemma \ref{ContinuityOfMeasure} hence $\cup_n A \in \mathcal{C}$.  It follows that $\mathcal{C}$ is a $\lambda$-system and it is clear that the set of open sets is a $\pi$-system therefore $\mathcal{C} = \mathcal{B}(S)$ by the $\pi$-$\lambda$ Thereom \ref{MonotoneClassTheorem}.

The Chapman-Kolmogorov relations follow from Proposition \ref{SemigroupsAndChapmanKolmogorov}.  To see $\mu_t(\Delta, \lbrace \Delta \rbrace) =1$ note that for every $f \in C_0(S)$ we have
\begin{align*}
\int f(s) \, \mu_t(\Delta, ds) &= T^\Delta_t f(\Delta) = f(\Delta) + T_t (f - f(\Delta))(\Delta) = 0
\end{align*}
so just approximate $\characteristic{S}$ by an sequence of nonnegative $f \in C_0(S)$ and use Fatou's Lemma (TODO: Explicity create such a sequence).

Since a compact metric space $S^\Delta$ is complete (Theorem \ref{CompactnessInMetricSpaces}) it is Polish and therefore Borel (Theorem \ref{PolishImpliesBorel}).  Applying Theorem \ref{ExistenceMarkovProcess} we conclude that there is a Markov process with homogeneous transition kernels $\mu_t$.
\end{proof}

As mentioned we have to be a bit more clever to show that we may find a cadlag version of Markov process $X$ constructed in the last result.  The basic idea is that we show that
many functions of $X$ are in fact supermartingales and therefore have cadlag versions; once enough functions $X$ are known to have
a cadlag version then we may approximate the identity and conclude that $X$ itself has a cadlag version.  Before stating the Theorem we need two preliminary lemmas.  The first constructs the supermartingales in question and the latter gives a probabilistic interpretation of the strong continuity of the Feller semigroup; the probabilistic continuity will
be used in the limiting process of showing $X$ is cadlag.

\begin{lem}\label{FellerResolventsAreSupermartingales}Let $S$ be locally compact separable metric space, let $T_t$ be a Feller semigroup on $S$, $X$ be the Markov process 
on $S^\Delta$ defined by $T_t$ and initial distribution $\nu$ and $\mathcal{F}$ be the filtration induced by $X$.  For every $f \in C_0(S)$ such that $f \geq 0$ define
\begin{align*}
Y_t &= e^{-t} R_1 f (X_t) = e^{-t} \int_0^\infty e^{-s} T_s f(X_t) \, ds
\end{align*}
then $Y_t$ is a $\mathcal{F}$-supermartinagle.
\end{lem}
\begin{proof}
Using the Markov property, the definitions of $T_t$ and $R_1$, a change of integration variables and the non-negativity of $f$,
\begin{align*}
\cexpectationlong{\mathcal{F}_{s}}{Y_t} 
&=\cexpectationlong{\mathcal{F}_{s}}{e^{-t} R_1 f(X_t)} \\
&=\sexpectation{e^{-t} R_1 f(X_{t-s})}{X_s} \\
&=T_{t-s} e^{-t} R_1 f(X_s) \\
&=T_{t-s} e^{-t} \int_0^\infty e^{-u} T_uf(X_s) \, du \\
&=e^{-t} \int_0^\infty e^{-u} T_{t+u-s} f(X_s) \, du \\
&=e^{-s} \int_0^\infty e^{-(t+u-s)} T_{t+u-s} f(X_s) \, du \\
&=e^{-s}\int_{t-s}^\infty e^{-u} T_{u} f(X_s) \, du \\
&\leq e^{-s} \int_{0}^\infty e^{-u} T_{u} f(X_s) \, du \\
&=e^{-s} R_1 f(X_s) = Y_s
\end{align*}
TODO:  Make sure everything here is sensible with respect to working on $S$ versus on $S^\Delta$.  The main point is that one can compute the resolvent of $f$ using either $T_t$ or the extension $T^\Delta_t$; however since $f(\Delta)=0$ we know that $T_t f = T^\Delta_t f$.

TODO: What about the use of $\mathcal{F}^X_+$ in EK?
\end{proof}

\begin{lem}\label{FellerSemigroupProbabilisticStrongContinuity}Let $(S, \rho)$ be compact separable metric space, let $X^x$ be a Markov process on $S$ starting at $x \in S$ and with a Feller transition semigroup $T_t$, then $\lim_{h \to 0^+} \sup_{x} \sexpectation{\rho(X^x_{t+h}, X^x_t) \wedge 1}{x} = 0$ for all $t \geq 0$.  In particular for every initial distribution $\mu$,  $X^\nu_{t+h} \toprob X^\nu_t$ as $h \downarrow 0$ for all $t \geq 0$.
\end{lem}
\begin{proof}
Since $S$ is compact and separable it follows that $C(S)$ is separable (Lemma \ref{SeparabilityOfBoundedUniformlyContinuous}).  Pick a countable dense set $f_1, f_2, \dotsc \in C(S)$.  

\begin{clm}Let $x_n$ be a sequence in $S$ then $\lim_{n \to \infty} x_n = x$ if and only if $\lim_{n \to \infty} f_m(x_n) = f_m(x)$ for all $m \in \naturals$.
\end{clm}
Clearly if $x_n \to x$ then $f_m(x_n) \to f_m(x)$ for all $f_m$ since $f_m$ is continuous.  On the other hand, suppose that $x_n$ does not converge.  Then by compactness there exists $x \in S$ and a subsequence $N \subset \naturals$ such that $x_n$ converges to $x$ along $N$.  Since $x_n$ does not have a limit there exists an open neighborhood of $x$ such that $x_n \notin U$ infinitely often; again by compactness we may pass to a convergent subsequence $N^\prime \subset \naturals$ and by construction $x_n$ converges to $y$ along $N^\prime$ and $x \neq y$.  The function $\rho(x, \cdot)$ is continuous and $\rho(x,x) \neq \rho(x,y)$.  Pick $f_m$ such that $\sup_{s \in S} \abs{\rho(x,s) - f_m(s)} < \rho(x,y)/2$ in particular by the triangle inequality $\abs{f_m(x) - f_m(y)} \geq \rho(x,y) - \abs{\rho(x,y) - f_m(y)} - \abs{f_m(x)} > 0$ and $f_m(x) \neq f_m(y)$.   By continuity of $f_m$, $f_m(x_n) \to f_m(x)$ along $N$ and $f_m(x_n) \to f_m(y)$ along $N^\prime$ and therefore $f_m(x_n)$ does not converge.

By the claim, it follows that $\rho$ is topologically equivalent to the metric 
\begin{align*}
\rho^{\prime}(x,y) &= \sum_{m=1}^\infty 2^{-m} (\abs{f_m(x) - f_m(y)} \wedge 1)
\end{align*}

Now suppose that $f \in C(S)$, $x \in S$  and $t,h \geq 0$ we compute
\begin{align*}
&\sexpectation{(f(X_t) - f(X_{t+h}))^2}{x}
=\sexpectation{f^2(X_t) - 2 f(X_t) f(X_{t+h}) + f ^2 (X_{t+h})} {x}\\
&=\sexpectation{f^2(X_t) - 2 f(X_t) \cexpectationlong{\mathcal{F}_t}{f(X_{t+h})} + \cexpectationlong{\mathcal{F}_t} {f ^2 (X_{t+h})}}{x} \\
&=\sexpectation{f^2(X_t) - 2 f(X_t) T_hf(X_{t}) + T_h f ^2 (X_{t})}{x} \\
&\leq \sup_{x \in S} \abs{f^2(x) - 2 f(x) T_hf(x) + T_h f ^2 (x)} \\
&\leq \sup_{x \in S} \abs{2f^2(x) - 2 f(x) T_hf(x)}  + \sup_{x \in S} \abs{T_h f ^2 (x) - f^2(x)} \\
&\leq 2 \sup_{x \in S}\abs{f(x)} \sup_{x \in S} \abs{f(x) - T_hf(x)}  + \sup_{x \in S} \abs{T_h f ^2 (x) - f^2(x)} \\
\end{align*}
From the strong continuity of $T_t$ we conclude $\lim_{h \to 0} \sup_{x \in S} \sexpectation{(f(X_t) - f(X_{t+h}))^2}{x} = 0$ for each fixed $f \in C(S)$.  
By Cauchy-Schwartz or Jensen's Inequality we know that $\sexpectation{\abs{f(X_t) - f(X_{t+h})}}{x}^2 \leq \sexpectation{(f(X_t) - f(X_{t+h}))^2}{x}$ and therefore we also 
have $\lim_{h \to 0} \sup_{x \in S} \sexpectation{\abs{f(X_t) - f(X_{t+h})}}{x} = 0$ for each fixed $f \in C(S)$.
In particular this is true for each
of the $f_m$ and therefore by Tonelli's Theorem (Corollary \ref{TonelliIntegralSum}) and Dominated Convergence we get
\begin{align*}
\lim_{h \to 0^+} \sup_{x} \sexpectation{\rho^\prime(X^x_{t+h}, X^x_t)}{x} 
&=\lim_{h \to 0^+} \sup_{x} \sum_{m=1}^\infty 2^{-m} \sexpectation{\abs{f_m(X^x_{t+h}) - f_m(X^x_t)} \wedge 1}{x} \\
&\leq \lim_{h \to 0^+} \sum_{m=1}^\infty 2^{-m} \sup_{x} \sexpectation{\abs{f_m(X^x_{t+h}) - f_m(X^x_t)} \wedge 1}{x} \\
&= \sum_{m=1}^\infty 2^{-m} \lim_{h \to 0^+} \sup_{x} \sexpectation{\abs{f_m(X^x_{t+h}) - f_m(X^x_t)} \wedge 1}{x} = 0\\
\end{align*}

TODO: Now argue that this implies the result for $\rho$ not just $\rho^\prime$.

To see the last statement by Lemma \ref{MarkovMixtures} and Dominated Convergence we get
\begin{align*}
\lim_{h \to 0^+} \sexpectation{\rho(X_{t+h}, X_t) \wedge 1}{\nu} &= \lim_{h \to 0^+} \int \sexpectation{\rho(X_{t+h}, X_t) \wedge 1}{x} \, \nu(dx) \\
&= \int \lim_{h \to 0^+} \sexpectation{\rho(X_{t+h}, X_t) \wedge 1}{x} \, \nu(dx) = 0
\end{align*}
so we use Lemma \ref{ConvergenceInProbabilityAsConvergenceInExpectation}.
\end{proof}

\begin{thm}\label{CadlagModificationFellerProcess}Let $S$ be locally compact separable metric space, let $T_t$ be a Feller semigroup on $S$ and let $X$ be the Markov process 
on $S^\Delta$ defined by $T_t$ and initial distribution $\nu$.  Then $X$ has a cadlag version $\tilde{X}$ such that $\Delta$ is an absorbing state for $\tilde{X}$.  If $T_t$ is conservative then $X$ has a cadlag version $\tilde{X}$ with values in $S$.
\end{thm}
\begin{proof}
By Lemma \ref{FellerResolventsAreSupermartingales} we know that $e^{-t} R_1 f(X_t)$ is a supermartingale for any $f \in C_0(S)$ with $f \geq 0$.  Thus applying Theorem \ref{CadlagModificationContinuousMartingale} we know that there is a $P_\nu$-null set in $N_f \in \mathcal{F}^X_\infty$ such that the restriction of $e^{-t} R_1 f(X_t)$ to $\rationals_+$ has left and right limits for all $t \geq 0$ outside of $N_f$; multiplying by the continuous function $e_t$, the same statement holds for $R_1 f(X_t)$.  Writing an arbitrary $f \in C_0(S)$ as $f = f_+ - f_-$ with $f_\pm \in C_0(S)$ and $f_\pm \geq 0$ and using the linearity of $R_1$ we see that $R_1 f(X_t)$ almost surely has all left and right limits along $\rationals_+$ for arbitrary $f \in C_0(S)$.  By definition of the resolvent and the Hille-Yosida theorem we know that $\range{R_1 f}$ is dense in $C_0(S)$ (or is it $C(S^\Delta)$?).
Thus given an arbitrary $f \in C_0(S)$ we may find $f_n \in C_0(S)$ such that $\lim_{n \to \infty} \sup_{x \in S} \norm{R_1 f_n(x) - f(x)}$.  For every $t \geq 0$, $q \in \rationals_+$ and $n \in \naturals$ we write
\begin{align*}
\abs{f(X_q) - f(X_t)} &\leq \abs{f(X_q) - R_1f_n(X_q)} + \abs{R_1f_n(X_q) - R_1f_n(X_t)} + \abs{R_1 f_n(X_t) - f(X_t)} \\
&\leq 2 \sup_{x \in S} \norm{R_1 f_n(x) - f(x)} + \abs{R_1f_n(X_q) - R_1f_n(X_t)}
\end{align*}
Each $R_1 f_n$ has left and right limits along $\rationals_+$ outside a null set $N_{n}$.  Letting either $q \uparrow t$ or $q \downarrow t$ and then taking $n \to \infty$ we see that $f(X_t)$ has left and right limits along $\rationals_+$ outside the null set $\cup_n N_n$.  Note also that $C_0(S)$ is separable (Corollary \ref{VanishingAtInfinityLocallyCompactSeparable}) so we may find a countable dense subset $f_1, f_2,\dotsc \in C_0(S)$ and letting $N = \cup_n N_{f_n}$ it follows by repeating the same limiting argument as above that for all $f \in C_0(S)$, $R_1 f$ has left and right limits along $\rationals$ outside of $N$ (where $N$ is now independent of $f$). 

\begin{clm} $X$ has left and right limits along $\rationals_+$ outside of the null set $N$.
\end{clm}
By a slight modification of the argument in Lemma \ref{FellerSemigroupProbabilisticStrongContinuity} one sees that the compactness of $S^\Delta$ implies that if we are given
a sequence $x_1, x_2, \dotsc \in S^\Delta$ such that $f(x_n)$ converges for every $f \in C_0(S)$ then it follows that $x_n$ converges in $S^\Delta$ (observe that if there are two convergent subsequences which converge to different elements of $S^\Delta$ then we can separate the two points with an element $f \in C_0(S)$).  From this fact, for every $\omega \notin N$ and we know that all increasing or decreasing sequences $q_n \in \rationals_+$ we have $\lim_{n \to \infty} f(X_{q_n}(\omega))$ exists and thus $\lim_{n \to \infty} X_{q_n}(\omega)$ exists.

Now we can define $\tilde{X}_t \equiv \Delta$ on $N$ and $\tilde{X}_t = \lim_{\substack{q \to t^+ \\ q \in \rationals}} X_q$ off of $N$ and by the proof of Theorem \ref{CadlagModificationContinuousMartingale} we know that $\tilde{X}_t$ is cadlag.  
\begin{clm} $\tilde{X}$ is a version of $X$.
\end{clm}
The claim means simply that for every $t \geq 0$ we have $\lim_{\substack{q \to t^+ \\ q \in \rationals}} X_q =X_t$ a.s. We know that the limit in question exists almost surely (i.e. off of $N$) so it suffices to show that $\lim_{\substack{q \to t^+ \\ q \in \rationals}} X_q =X_t$ a.s. along a subsequence.  Lemma \ref{FellerSemigroupProbabilisticStrongContinuity} we know that $X_q \toprob X_t$ and the convergence of the subsequence follows by Lemma \ref{ConvergenceInProbabilityAlmostSureSubsequence}.

Now let $f \in C_0(S)$ with $f > 0$ on $S$ (e.g. $f(x) = \rho(\Delta, x)$).  Note that $R_1 f > 0$; by strong continuity for each $x \in S$ we pick $\delta>0$ such that $\norm{T_t f - f} < f(x)/2$ for $0 \leq t \leq \delta$ and if follows that $\abs{T_t f(x)} \geq  f(x) - \abs{T_tf(x) - f(x)}  \geq f(x)/2 > 0$ for all $0  \leq t \leq \delta$ and therefore by positivity of $T_t$ we have $R_1 f (x) \geq \int_0^\delta e^{-u} T_u f(x) \, du \geq f(x)(1-e^{-\delta})/2 > 0$.  It is also clear that $f(\Delta) = 0$ implies $R_1 f(\Delta) = 0$.  Therefore $Y_t = e^{-t} R_1 f(\tilde{X}_t)$ is a non-negative cadlag supermartingale such that $Y_t = 0$ is equivalent to $\tilde{X}_t = \Delta$.  If we apply Lemma \ref{PositiveSupermartingaleAbsorption} to $Y_t$ and translate the conclusion in terms of $\tilde{X}_t$ we see that if we define $\tau = \inf \lbrace t \mid \tilde{X}_{t-} \wedge \tilde{X}_t =\Delta \rbrace$ then $\tilde{X}_t \equiv \Delta$ a.s. on $[\tau, \infty)$.  Setting $\tilde{X}$ to be identically $\Delta$ on the null set where the conclusion fails we that by a further modification of $\tilde{X}$ we can assume $\Delta$ is absorbing everywhere.

If $T_t$ is conservative on $C_0(S)$ and we assume that $\nu(\lbrace \Delta \rbrace) = 0$ then it follows that $\tilde{X}_t \in S$ a.s. for every $t \geq 0$ (TODO: Why?)  Therefore we must have $\tau > t$ a.s. for all $t \geq 0$ which implies that $\tau = \infty$ a.s.  Pick an arbitrary $x \in S$ and make an additional modification of $\tilde{X}$ to set $\tilde{X}_t \equiv x$ on $\tau < \infty$ and therefore $\tau = \infty$ everywhere.  This implies that $\tilde{X}_t$ and $\tilde{X}_{t-}$ take values in $S$.
\end{proof}

TODO:  Show that we get a homogeneous cadlag Markov family out of this construction.  At a minimum we can see that we get probability measures $\sprobabilityop{x}$ for each $x \in S^\Delta$ and this is a kernel from $S$ to $S^{[0,\infty)}$ (by Lemma \ref{MarkovMixtures}).

\begin{thm}\label{StrongMarkovFellerProcess} Let $X$ be a Feller family, $A \in (\mathcal{S}^\Delta)^{[0,\infty)}$  and let $\tau$ be an $\mathcal{F}^X_+$-optional time, then
\begin{align*}
\cprobability{\mathcal{F}^X_{\tau+}}{\theta_{\tau} X \in A} &= \sprobability{A}{X_\tau} \text{ a.s. on $\tau < \infty$}
\end{align*}
Let $X$ be the canonical Feller process, $\nu$ be an initial distribution, $\tau$ an $\mathcal{F}^X_+$-optional time, $\xi$ a non-negative random variable then
\begin{align*}
\csexpectationlong{\mathcal{F}^X_{\tau+}}{\xi \circ \theta_{\tau} }{\nu} &= \sexpectation{\xi}{X_\tau} \text{ $\sprobabilityop{\nu}$ a.s. on $\tau < \infty$}
\end{align*}
\end{thm}
\begin{proof}
By Proposition \ref{StrongMarkovFromStrongMarkovFiniteOptional} it suffices to assume that $\tau<\infty$ almost surely.  Let $A \in (\mathcal{S}^\Delta)^{[0,\infty)}$.  Define $\tau_n = 2^{-n}\floor{2^n \tau + 1}$ so that by Lemma \ref{DiscreteApproximationOptionalTimes} we know that $\tau_n$ are $\mathcal{F}^X$-optional times and $\tau_n \downarrow \tau$.  Since $\tau < \tau_n$ we also know that $\mathcal{F}^X_{\tau+} \subset \mathcal{F}^X_{\tau_n}$ for all $n \in \naturals$.  Since $\tau_n$ is countably valued $X$ is strong Markov at $\tau_n$ (Theorem \ref{StrongMarkovPropertyMarkovProcessCountableValues}) and
\begin{align*}
\cprobability{\mathcal{F}^X_{\tau_n}}{\theta_{\tau_n} X \in A} &=
\sprobability{A}{X_{\tau_n}} \text{ a.s. on $\tau_n < \infty$}
\end{align*}

It is useful to avoid appeal to Theorem \ref{StrongMarkovPropertyMarkovProcessCountableValues} so that the role of the Feller properties can be appreciated even in the case of countably valued $\tau$ (not sure I agree because the same continuity argument seems to appear when approximating by countably valued).  So let $\tau$ be $\mathcal{F}^X_+$-optional and let $T$ be the countable range of $\tau$, $f \in C(S^\Delta)$, $s \geq 0$ and $A \in \mathcal{F}^X_{t+}$.  Thus,
for every $t \geq 0$, $\epsilon > 0$ we have $A \cap \lbrace \tau=t \rbrace \in \mathcal{F}^X_{t+} \subset \mathcal{F}^X_{t+\epsilon}$.  Let $0 \leq \epsilon \leq s$ and use
the tower property of conditional expectation and Proposition \ref{TransitionSemigroupAsExpectation} to get
\begin{align*}
\expectation{f(X_{\tau + s}) ; A} &= \sum_{t \in T} \expectation{f(X_{t + s}) ; A \cap \lbrace \tau=t \rbrace } \\
&= \sum_{t \in T} \expectation{\cexpectationlong{\mathcal{F}^X_{t+\epsilon}}{f(X_{t + s})} ; A \cap \lbrace \tau=t \rbrace} \\
&= \sum_{t \in T} \expectation{T_{s-\epsilon} f (X_{t + \epsilon}) ; A \cap \lbrace \tau=t \rbrace} \\
&= \expectation{T_{s-\epsilon} f (X_{\tau + \epsilon}) ; A} \\
\end{align*}
By strong continuity of $T_tf$ and right continuity of $X_t$ we know that $\lim_{\epsilon \to 0} T_{s-\epsilon} f (X_{\tau + \epsilon}) = T_s f(X_\tau)$ (TODO: Show this in more detail; actually we don't need this since we only need countably valued $\mathcal{F}^X$-optional times) and therefore by Dominated Convergence
we get $\expectation{f(X_{\tau + s}) ; A} = \expectation{T_{s} f (X_{\tau }) ; A} $.   Since $T_s f (X_\tau)$ is $\mathcal{F}^X_{\tau+}$-measurable we conclude 
\begin{align*}
\cexpectationlong{\mathcal{F}^X_{\tau+}}{f(X_{\tau + s})} = T_s f (X_\tau)
\end{align*}
To extend this fact to arbitrary $\tau$ with $\tau < \infty$,


In what follows we are using a claim that $\sigma < \tau$ implies $\mathcal{F}_{\sigma+} \subset \mathcal{F}_\tau$.  TODO: Prove it or disprove it (if it turns out not to be true then we use the $\epsilon$ argument above taken from EK).

Let  $\tau_n = 2^{-n}\floor{2^n \tau + 1}$ so that $\tau_n$ are countably valued $\mathcal{F}^X$-optional times
such that  $\tau < \tau_n$  and $\tau_n \downarrow \tau$(Lemma \ref{DiscreteApproximationOptionalTimes}).  Now we use continuity of $T_sf$, Dominated Convergence for conditional expectations and the $\mathcal{F}^X_{\tau+}$-measurability of $T_s f (X_{\tau})$ (since $X$ is cadlag, it is progressive by Lemma \ref{ContinuityAndProgressiveMeasurability} thus $X_\tau$ is $\mathcal{F}^X_{\tau+}$-measurable by Lemma \ref{StoppedProgressivelyMeasurableProcess}) to see
\begin{align*}
&\cexpectationlong{\mathcal{F}^X_{\tau+}}{f(X_{\tau + s})} \\
&= \cexpectationlong{\mathcal{F}^X_{\tau+}}{\lim_{n \to \infty}  f(X_{\tau_n + s})} && \text{right continuity of $X$, continuity of $f$}\\
&= \lim_{n \to \infty}  \cexpectationlong{\mathcal{F}^X_{\tau+}}{f(X_{\tau_n + s})} && \text{Dominated Convergence}\\
&= \lim_{n \to \infty} \cexpectationlong{\mathcal{F}^X_{\tau+}}{\cexpectationlong{\mathcal{F}^X_{\tau_n}}{f(X_{\tau_n + s})}} && \text{chain rule of conditional expectations}\\
&=\lim_{n \to \infty} \cexpectationlong{\mathcal{F}^X_{\tau+}}{T_s f (X_{\tau_n})} &&  \text{Theorem \ref{StrongMarkovPropertyMarkovProcessCountableValues} and Proposition \ref{TransitionSemigroupAsExpectation}}\\
&=\cexpectationlong{\mathcal{F}^X_{\tau+}}{\lim_{n \to \infty} T_s f (X_{\tau_n})}  && \text{Dominated Convergence}\\
&=\cexpectationlong{\mathcal{F}^X_{\tau+}}{T_s f (X_{\tau})} && \text{right continuity of $X$, continuity of $T_sf$}\\
&=T_s f (X_{\tau}) && \text{$\mathcal{F}^X_{\tau+}$-measurability of $T_s f (X_{\tau})$}\\
&=\int f(u) \, \mu_s (X_{\tau}, du) \\
\end{align*}
Now we may apply Proposition \ref{StrongMarkovFromOneDimensionalDistribution} to complete the proof.

TODO: Show the canonical case...
\end{proof}

We first need to introduce the formalize notation and concepts surrounding bounded pointwise limits and closure.

\begin{defn}Let $S$ be a metric space and let $B_b(S)$ be the space of bounded measurable functions then given $f, f_1, f_2, \dotsc \in B_b(S)$ we say that $f = \bplim_{n \to \infty} f_n$ if and only if $sup_n \sup_{x \in S} \abs{f(x)} < \infty$ and $\lim_{n \to \infty} f_n(x) = f(x)$ for all $x \in S$.  We say that a set $M \subset S$ is \emph{bp-closed} if $f_1, f_2, \dotsc \in M$ and $f = \bplim_{n \to \infty} f_n$ implies $f \in M$.  
\end{defn}

TODO:  On $C_0(S)$ we know (or should prove) that the dual $C^*_0(S)$ is the space of finite signed Radon measures.  It turns out that a sequence $f = \bplim f_n$ if and only if $f_n$ converges weakly to $f$ (i.e. $\int f_n \, d \mu \to \int f  \, d\mu$ for all finite signed Radon measures $\mu$).  This is not true for arbitrary nets however.

\begin{defn}A multivalued operator $A \subset B_b(S) \times B_b(S)$ is \emph{conservative} if and only if $(1,0)$ is contained in the bp-closure of $A$.
\end{defn}

TODO: Understand the relationship between this definition of conservative and the more elementary statement that $T_t 1 = 1$ for all $t \geq 0$.  Understand the relationship between this definition and the Kallenberg definition of conservativeness of a the semigroup (NOT the generator) that says $\sup_{f \leq 1} T_t f (x) = 1$ for all $x \in S$.  Understand the relationship between convervativeness and the statement that a sub-Markov transition semigroup is Markov (i.e. the property that every measure has total mass 1).  Yet another defintion of conservative that is specific to Feller semigroups: if $f_n \uparrow 1$, $f_n \in C_0(S)$ then $T_t f_n \uparrow 1$ (this is essentially Kallenberg's definition).  I believe that the following is true: $(1,0) \in bp-closure(A)$ implies $T_t 1 = 1$ for all $t \geq 0$.  Also if $S$ is compact then $T_t 1 = 1$ for all $t \geq 0$ if and only if $1 \in \domain{A}$ and $A 1 = 0$.  Note also the fact that $1 \notin C_0(S)$ if $S$ is not compact so the definition in terms of $T_t 1 = 1$ not correct; it can be rescued by proving another theorem (which is a consequence of Riesz representation) that says any semigroup of continuous operators on $C_0(S)$ can be extended to a semigroup of continuous operators on $B_b(S)$.

\begin{lem}Let $A$ be the generator of a strongly continuous contraction semigroup on a subspace of $X \subset B_b(S)$ then if $A$ is conservative then
$T_t 1 = 1$ for all $t \geq 0$ (TODO: This statement requires the extension of $T_t$ to $C_b(S)$ or at least a space that contains $1$; this should follow from the construction of the transition semigroup constructed from the Feller semigroup).
\end{lem}
\begin{proof}
By the Kolmogorov backward equation Proposition \ref{StronglyContinuousSemigroupKolomgorovBackwardEquation} we know that 
\begin{align*}
\lbrace (f, Af) \mid f \in \domain{A} \rbrace &\subset \lbrace (f,g) \in B_b(S) \times B_b(S) \mid T_t f- f = \int_0^t T_s g \, ds \text{ for all $t\geq 0$} \rbrace
\end{align*}
\begin{clm}The right hand set is bp-closed
\end{clm}
Suppose $(f_n, g_n)$ is a sequence in the right hand set such that $\sup_n \sup_{x\in S} \abs{f_n(x)} < \infty$, $\sup_n \sup_{x\in S} \abs{g_n(x)} < \infty$, $\lim_{n \to \infty} (f_n(x), g_n(x)) = (f(x), g(x))$  for all $x \in S$.  In particular it is the case that $\lim_{n \to \infty} f_n = f$ and $\lim_{n \to \infty} g_n = g$ in $X$ (TODO: WRONG we only have pointwise limits not uniform limits; how do we know that $T_t f_n(x) \to T_t f(x)$ and $T_t g_n(x) \to T_t g(x)$?  I believe the answer is that for contraction semigroups norm continuity and weak continuity coincide) and therefore $\lim_{n \to \infty} T_t f_n = T_t f$
and $\lim_{n \to \infty} T_t g_n = T_t g$ for all $t \geq 0$.  Since $T_t$ is a contraction operator we also have $\sup_n \sup_{x \in S} \abs{T_t g_n(x)} < \infty$ and therefore by Dominated Convergence we also have $\lim_{n \to \infty} \int_0^t T_s g_n \, ds = \int_0^t T_s g \, ds$.  

Now since $A$ is conservative we conclude that $(1,0)$ is in the right hand set and lemma follows.
\end{proof}

\section{Approximation of Feller Processes}

In this section we begin considering the weak convergence theory of
Feller Processes.  Semigroup theory plays a central role and we begin by proving a convergence
theorem for strongly continuous contraction semigroups that will be applied in the Feller case.  We begin with 
two lemmas. The first extends Lemma \ref{SemigroupBoundInTermsOfBoundedGenerator}.

\begin{lem}\label{SemigroupBoundInTermsOfBoundedGenerator2}Let $X$ and $Y$ be Banach spaces, let $T_t$ be a strongly continuous contraction semigroup on $X$ with generator $A$,
let $S_t$ be a strongly continuous contraction semigroup on $Y$ with generator $B$ and let $\pi : X \to Y$ be a bounded linear operator.  Let
$v \in \domain{A}$ assume that $\pi T_t v \in \domain{B}$ for all $t \geq 0$ and that $B \pi T_t v : [0,\infty) \to Y$ is continuous then
\begin{align*}
S_t \pi v - \pi T_t v &= \int_0^t S_{t-s} (B \pi - \pi A) T_s v \, ds
\end{align*}
for all $t \geq 0$ and in particular, 
\begin{align*}
\norm{S_t \pi v - \pi T_t v} &= \int_0^t \norm{B \pi - \pi A) T_s v} \, ds
\end{align*}
\end{lem}
\begin{proof}
Let $v \in \domain{A}$ and consider the term $S_{t - s} \pi T_s v$ for $0 \leq s \leq t$.  Since $v \in \domain{A}$ we know that $T_s v$ is a differentiable function of $s$ and 
$\frac{d}{ds} T_s v = A T_s v$ (Proposition \ref{StronglyContinuousSemigroupKolomgorovBackwardEquation}).   Since $\pi$ is a bounded linear map we know from the chain rule (Proposition \ref{ChainRuleBanachSpaces}) that $\pi T_s v$ is also differentiable and $\frac{d}{ds} \pi T_s v = \pi A T_s v$.  Also for fixed $w \in \domain{B}$ and $0 \leq s \leq t$ we know from Proposition \ref{StronglyContinuousSemigroupKolomgorovBackwardEquation} that $S_{t -s} w$ is differentiable with respect to $s$ and $\frac{d}{ds} S_{t -s} w = - S_{t-s} B w$.  We claim that we have a product rule for differentiation that shows
\begin{align*}
\frac{d}{ds} S_{t - s} \pi T_s v = S_{t-s} \pi A T_s v - S_{t-s} B \pi T_s v = -S_{t -s} (B \pi - \pi A) T_s v
\end{align*}
TODO: Show this

Now we can just apply the Fundamental Theorem of Calculus \ref{FundamentalTheoremOfCalculusForBanachSpaceRiemannIntegrals} to see that
\begin{align*}
\int_0^t S_{t -s} (B \pi - \pi A) T_s v \, ds &= -S_{t - s} \pi T_s v \mid_{0}^t = S_t \pi v - \pi T_t v
\end{align*}
Also we can just apply Proposition \ref{NormRiemannIntegralBanachSpace} and the fact that $S_t$ is a contraction to see
\begin{align*}
\norm{S_t \pi v - \pi T_t v} &\leq \int_0^t \norm{S_{t -s} (B \pi - \pi A) T_s v} \, ds \leq \int_0^t \norm{(B \pi - \pi A) T_s v} \, ds
\end{align*}
\end{proof}

The second required lemma is a continuity property of the Yosida approximation.
\begin{lem}\label{SCCSYosidaContinuity}Let $X$ and $X_1, X_2, \dotsc$ be Banach spaces, $T_{n,t}$ be strongly continuous contraction semigroups on $X_n$ with generator $A_n$ and let $T_t$ be a strongly continuous contraction semigroup on $X$ with generator $A$.  Let $\pi_n : X \to X_n$ be bounded linear operators and assume that $\sup_{n} \norm{\pi_n} < \infty$.  Let $D$ be a
core for $A$ and suppose that for every $v \in D$ there exists $v_n \in \domain{A_n}$ such that $\lim_{n \to \infty} \norm{v_n - \pi_n v} = 0$ and $\lim_{n \to \infty} \norm{A_n v_n - \pi_n A v} = 0$.  Suppose that $A^\lambda_n$ and $A^\lambda$ denote the Yosida approximations of $A_n$ and $A$ then for all $v \in X$ and $\lambda > 0$ we have
\begin{align*}
\lim_{n \to \infty} \norm{A^\lambda_n \pi_n v - \pi_n A^\lambda v} &= 0
\end{align*}
\end{lem}
\begin{proof}
Let $\lambda > 0$ and $v \in D$.  Set $w = (\lambda - A) v$ and pick $v_n \in \domain{A_n}$ such that $\lim_{n \to \infty} {v_n - \pi_n v} = \lim_{n \to \infty} {A_nv_n - \pi_n Av} = 0$.  Note that by the triangle inequality $\lim_{n \to \infty} \norm{(\lambda - A_n) v_n - \pi_n w} = 0$.  Recall that since 
\begin{align*}
A^\lambda &= \lambda A R_\lambda = \lambda (\lambda - (\lambda - A)) R_\lambda = \lambda^2 R_\lambda - \lambda \IdentityMatrix
\end{align*}
and from $\norm{R_\lambda} \leq \lambda^{-1}$ shows that $\norm{A^\lambda} \leq 2 \lambda$.  The same holds for $A^\lambda_n$ where we use the notation $R_{n, \lambda}$ for the resolvent of $T_{n,t}$.

Using the above identity we compute
\begin{align*}
\norm{A^\lambda_n \pi_n w  - \pi_n A^\lambda w} 
&= \norm{(\lambda^2 R_{n,\lambda} - \lambda \IdentityMatrix) \pi_n w  - \pi_n (\lambda^2 R_{\lambda} - \lambda \IdentityMatrix)  w} \\
&=\lambda^2 \norm{(R_{n,\lambda} \pi_n - \pi_n R_{\lambda}) w}  \\
&\leq \lambda^2 \norm{R_{n,\lambda} \pi_n w - v_n}  + \lambda^2 \norm{v_n - \pi_n R_{\lambda} w}  \\
&= \lambda^2 \norm{R_{n,\lambda} [\pi_n w - (\lambda - A_n) v_n]}  + \lambda^2 \norm{v_n - \pi_n v}  \\
&= \lambda \norm{\pi_n w - (\lambda - A_n) v_n}  + \lambda^2 \norm{v_n - \pi_n v}  \\
\end{align*}
and therefore we see that $\lim_{n \to \infty} \norm{A^\lambda_n \pi_n w  - \pi_n A^\lambda w} = 0$ for $w \in (\lambda -A)(D)$.

On the other hand $(\lambda - A)(D)$ is dense in $X$ (Proposition \ref{SCCSCoreViaDenseRange}) and 
\begin{align*}
\norm{A^\lambda_n \pi_n - \pi_n A^\lambda} &\leq (\norm{A^\lambda_n} + \norm{A^\lambda}) \norm{\pi_n} \leq 4 \lambda \sup_{n} \norm{\pi_n} < \infty
\end{align*}
and therefore $\lim_{n \to \infty} \norm{A^\lambda_n \pi_n w  - \pi_n A^\lambda w} = 0$ for $w \in X$ (let $w_m \to w$ with $w_m \in  (\lambda -A)(D)$, write
$\norm{A^\lambda_n \pi_n w  - \pi_n A^\lambda w} \leq \norm{A^\lambda_n \pi_n w_m  - \pi_n A^\lambda w_m} + \sup_n \norm{A^\lambda_n \pi_n - \pi_n A^\lambda} \norm{w_m -w}$
and then let $n \to \infty$ followed by $m \to \infty$).
\end{proof}

Now we can present the semigroup convergence theorem itself.

\begin{thm}\label{KurtzSovaSemigroupConvergence}Let $X$ and $X_1, X_2, \dotsc$ be Banach spaces, $T_{n,t}$ be strongly continuous contraction semigroups on $X_n$ with generator $A_n$ and let $T_t$ be a strongly continuous contraction semigroup on $X$ with generator $A$.  Let $\pi_n : X \to X_n$ be bounded linear operators and assume that $\sup_{n} \norm{\pi_n} < \infty$.  Let $D$ be a
core for $A$ then the following are equivalent
\begin{itemize}
\item[(i)]For every $v \in X$ and every $t \geq 0$,  $\lim_{n \to \infty} \sup_{0 \leq s \leq t} \norm{T_{n,s} \pi_n v - \pi_n T_s v} = 0$.
\item[(ii)]For every $v \in X$ and every $t \geq 0$,  $\lim_{n \to \infty} \norm{T_{n,t} \pi_n v - \pi_n T_t v} = 0$.
\item[(iii)] For every $v \in D$ there exists $v_n \in \domain{A_n}$ such that $\lim_{n \to \infty} \norm{v_n - \pi_n v} = 0$ and $\lim_{n \to \infty} \norm{A_n v_n - \pi_n A v} = 0$.
\end{itemize}
\end{thm}
\begin{proof}
The implication (i) implies (ii) is immediate.

To see that (ii) implies (iii), let $\lambda > 0$, $v \in \domain{A}$ and set $w = (\lambda - A) v$.  By Proposition \ref{SCCSResolventAsLaplaceTransform} we have
\begin{align*}
v &= R_\lambda w = \int_0^\infty e^{- \lambda t} T_t w \, dt
\end{align*}
Now define
\begin{align*}
v_n &= \int_0^\infty e^{- \lambda t} T_{n,t} \pi_n w \, dt = R_{n,\lambda} \pi_n w
\end{align*}
and note that by (ii) we have $\lim_{n \to \infty} \norm{T_{n,t} \pi_n w - \pi_n T_t w} = 0$ and since $T_{n,t}$ and $T_t$ are contractions we have $\sup_n \norm{T_{n,t} \pi_n w - \pi_n T_t w} \leq 2 \sup_n \norm{\pi_n} \norm{w} < \infty$. Now apply Proposition \ref{ClosedOperatorOfRiemannIntegral}, Proposition \ref{NormRiemannIntegralBanachSpace} and Dominated Convergence
\begin{align*}
\lim_{n \to \infty} \norm{v_n - \pi_n v} &= \lim_{n \to \infty} \norm{\int_0^\infty e^{- \lambda t} T_{n,t} \pi_n w \, dt - \pi_n \int_0^\infty e^{- \lambda t} T_t w \, dt} \\
&= \lim_{n \to \infty} \norm{\int_0^\infty e^{- \lambda t} T_{n,t} \pi_n w \, dt - \int_0^\infty e^{- \lambda t} \pi_n  T_t w \, dt} \\
&\leq \lim_{n \to \infty} \int_0^\infty e^{- \lambda t} \norm{T_{n,t} \pi_n w \, dt - \pi_n  T_t w} \, dt \\
&= \int_0^\infty e^{- \lambda t} \lim_{n \to \infty} \norm{T_{n,t} \pi_n w \, dt - \pi_n  T_t w} \, dt = 0\\
\end{align*}
Also note that
\begin{align*}
\norm{A_n v_n - \pi_n A v} &= \norm{\lambda v_n - (\lambda - A_n) v_n - \lambda \pi_n v + \pi_n w} \\
&\leq \norm{\lambda v_n - \lambda \pi_n v + \pi_n (\lambda - A) v} + \norm{\pi_n w - (\lambda - A_n) v_n} \\
&= \norm{\lambda v_n - \lambda \pi_n v + \pi_n (\lambda - A) v} + \norm{\pi_n w - (\lambda - A_n) R_{n,\lambda} \pi_n w} \\
&= \norm{\lambda v_n - \lambda \pi_n v + \pi_n (\lambda - A) v} \\
\end{align*}
and therefore (iii) is proven.

To see that (iii) implies (i) let $A^\lambda_n$ and $A^\lambda$ be the Yosida approximations of $A_n$ and $A$ respectively; let $T^\lambda_{n,t} = e^{t A^\lambda_n}$ and $T^\lambda_t = e^{tA^\lambda}$ be the corresponding strongly continuous contraction semigroups.  Let $v \in D$ and choose $v_n \in \domain{A_n}$ such that $\lim_{n \to \infty} \norm{v_n - \pi_n v}=0$ and $\lim_{n \to \infty} \norm{A_n v_n - \pi_nA v}=0$.  Use the triangle inequality to write
\begin{align*}
\norm{T_{n,t} \pi_n v - \pi_n T_t v} &\leq \norm{T_{n,t}(\pi_n v - v_n)} + \norm{T_{n,t}v_n - T^\lambda_{n,t}v_n } + \norm{T^\lambda_{n,t} (v_n - \pi_n v)}  \\
&+\norm{T^\lambda_{n,t} \pi_n v - \pi_n T^\lambda_{t} v } + \norm{\pi_n ( T^\lambda_{t} v - T_t v)}
\end{align*}
Now we consider estimates of each of the five terms on the right hand side.  Let $t \geq 0$ be fixed.  For the first term note that since $T_{n,t}$ is a contraction
\begin{align*}
\limsup_{n \to \infty} \sup_{0 \leq s \leq t} \norm{T_{n,s}(\pi_n v - v_n)}  &\leq \lim_{n \to \infty} \norm{\pi_n v - v_n}  = 0
\end{align*}
and the same argument works for the third term as well.  For the second term, apply Corollary \ref{YosidaApproximationContinuity}, the triangle inequality, Lemma \ref{SCCSYosidaContinuity} and the fact that $\norm{A^\lambda_n} \leq 2 \lambda$ to see that
\begin{align*}
&\limsup_{n \to \infty} \sup_{0 \leq s \leq t} \norm{T_{n,s}v_n - T^\lambda_{n,s}v_n } 
\leq t \limsup_{n \to \infty} \norm{A_n v_n - A^\lambda_n v_n} \\
&\leq t \limsup_{n \to \infty} \left [ \norm{A_n v_n -\pi_n A v} + \norm{\pi_n A v - \pi_n A^\lambda v} + \norm{\pi_n A^\lambda v - A^\lambda_n \pi_n v} + \norm{A^\lambda_n \pi_n v - A^\lambda_n v_n} \right]\\
&\leq t \sup_n \norm{\pi_n}  \norm{A v - A^\lambda v}
\end{align*}

For the fourth term we apply Lemma \ref{SemigroupBoundInTermsOfBoundedGenerator2} to see that 
\begin{align*}
\sup_{0 \leq s \leq t} \norm{T^\lambda_{n,s} \pi_n v - \pi_n T^\lambda_{s} v } 
&\leq \int_0^t \norm{(A^\lambda_n \pi_n - \pi_n A^\lambda) T^\lambda_s v} \, ds
\end{align*}
By Lemma \ref{SCCSYosidaContinuity} we know that $\lim_{n \to \infty} \norm{(A^\lambda_n \pi_n - \pi_n A^\lambda) T^\lambda_s v} = 0$ and moreover $\norm{(A^\lambda_n \pi_n - \pi_n A^\lambda) T^\lambda_s v} \leq 4 \lambda \sup_n \norm{\pi_n} \norm{v}$ so that by Dominated Convergence we get  
\begin{align*}
\limsup_{n \to \infty} \sup_{0 \leq s \leq t} \norm{T^\lambda_{n,s} \pi_n v - \pi_n T^\lambda_{s} v } = 0
\end{align*}
For the fifth and final term we apply Corollary \ref{YosidaApproximationContinuity} to get
\begin{align*}
\limsup_{n \to \infty} \sup_{0 \leq s \leq t} \norm{\pi_n ( T^\lambda_{s} v - T_s v)} &\leq \sup_{n} \norm{\pi_n} \sup_{0 \leq s \leq t} \norm{T^\lambda_{s} v - T_s v} \\
&\leq t \sup_{n} \norm{\pi_n} \norm{A^\lambda v - A v}
\end{align*}

Putting all of the estimates together we see that
\begin{align*}
\limsup_{n \to \infty} \sup_{0 \leq s \leq t} \norm{T_{n,t} \pi_n v - \pi_n T_t v}  &\leq 2 t \sup_{n} \norm{\pi_n} \norm{A^\lambda v - A v}
\end{align*}
Letting $\lambda \to \infty$ and using Lemma \ref{ClosedDissipativeYosidaApproximation} we see that $\lim_{n \to \infty} \sup_{0 \leq s \leq t} \norm{T_{n,t} \pi_n v - \pi_n T_t v} =0$ for 
$v \in D$.  Since $D$ is dense by the Hille-Yosida Theorem \ref{HilleYosidaTheoremClosable} for $v \in X$ we may take $v_m \in D$ such that $\lim_{m \to \infty} v_m = v$ and then
\begin{align*}
&\limsup_{n \to \infty} \sup_{0 \leq s \leq t} \norm{T_{n,t} \pi_n v - \pi_n T_t v} \\
&\leq \limsup_{n \to \infty} \sup_{0 \leq s \leq t} \left[\norm{T_{n,t} \pi_n v- T_{n,t} \pi_n v_m} + \norm{T_{n,t} \pi_n v_m - \pi_n T_t v_m} + \norm{\pi_n T_t v - \pi_n T_t v_m}  \right] \\
&\leq 2 \sup_n \norm{\pi_n} \norm{v - v_m} + \lim_{n \to \infty} \sup_{0 \leq s \leq t} \norm{T_{n,t} \pi_n v_m - \pi_n T_t v_m} = 2 \sup_n \norm{\pi_n} \norm{v - v_m} 
\end{align*}
Now let $m \to \infty$ and (i) follows.
\end{proof}

TODO: Compare Kallenberg and EK accounts of the following theorem and understand the differences and similarities.  The main difference is that the Theorem in Kallenberg is really
a combination of two theorems in EK.  The first theorem is Theorem \ref{KurtzSovaSemigroupConvergence} and simply addresses equivalent conditions for convergence of sccs.  The second theorem is what we have as Theorem \ref{KurtzMackevicius} below and shows that Feller semigroup convergence implies weak convergence of Markov processes (using the standard FDD and tightness approach).  EK and Kallenberg use the same argument to show FDD convergence whereas Kallenberg uses the Aldous criterion directly but EK use a more sophisticated argument for tightness using the martingale property (which may ultimately boil down to the Aldous criterion).  Also EK are careful to prove the Theorem in such a way that the existence of cadlag $X$ is not assumed but rather it is derived whereas Kallenberg has already proved that a cadlag $X$ may be created from $T_t$ and thus may appeal to the strong Markov property in showing the Aldous criterion is satisfied.

\begin{thm}\label{KurtzMackevicius}Let $(S,r)$ be a locally compact and separable metric space and let $X^n$ for $n \in \naturals$ be
cadlag Feller processes in $S$ with semigroup $T_{n,t}$ on $C_0(S)$ respectively.  Suppose that $T_t$ is a Feller semigroup on $C_0(S)$ and
\begin{align*}
\lim_{n \to \infty} T_{n,t} f = T_t f \text{ for all $f \in C_0(S)$ and $t \geq 0$}
\end{align*}
If the distributions of $X^n(0)$ converges to a limit $\nu \in \mathcal{P}(S)$ then there exists a cadlag Feller process $X$ with semigroup
$T_t$ and initial distribution $\nu$ and $X^n \todist X$.
\end{thm} 
\begin{proof}
For $n \in \naturals$ let $A_n$ be the generator of $T_{n,t}$.

First we assume that $S$ is compact and that the $T_{n,t}$ and $T_t$ are conservative.  TODO: Remove this assumption.

\begin{clm} For all $m \in \integers_+$, $0=t_0 < t_1 < \dotsb < t_m$ and $f_i \in C(S)$ for $i=0, \dotsc, m$ we have
\begin{align*}
\lim_{n \to \infty} \expectation{f_0(X^n_{t_0}) \dotsb f_m(X^n_{t_m})} = \expectation{f_0(X_{t_0}) \dotsb f_{m}(X_{t_{m}})}
\end{align*}
\end{clm}
To see the claim, note that the case $m=0$ is simply the statement that $X^n_0 \todist X_0$ which is an assumption.  For $m > 0$, 
we apply Proposition \ref{TransitionSemigroupAsExpectation}
\begin{align*}
\lim_{n \to \infty} \expectation{f_0(X^n_{t_0}) \dotsb f_m(X^n_{t_m})} 
&=\lim_{n \to \infty} \expectation{f_0(X^n_{t_0}) \dotsb f_{m-1}(X^n_{t_{m-1}}) \cexpectationlong{X^n_{t_{m-1}}}{f_m(X^n_{t_m})}} \\
&=\lim_{n \to \infty} \expectation{f_0(X^n_{t_0}) \dotsb f_{m-1}(X^n_{t_{m-1}}) T_{n,t_m-t_{m-1}} f_m(X^n_{t_{m-1}})}\\
&=\lim_{n \to \infty} \expectation{f_0(X^n_{t_0}) \dotsb f_{m-1}(X^n_{t_{m-1}}) T_{t_m-t_{m-1}} f_m(X^n_{t_{m-1}})}\\
&=\expectation{f_0(X_{t_0}) \dotsb f_{m-1}(X_{t_{m-1}}) T_{t_m-t_{m-1}} f_m(X_{t_{m-1}})}\\
&=\expectation{f_0(X_{t_0}) \dotsb f_{m-1}(X_{t_{m-1}}) \cexpectationlong{X_{t_{m-1}}}{f_m(X_{t_{m}})}}\\
&=\expectation{f_0(X_{t_0}) \dotsb f_{m-1}(X_{t_{m-1}}) f_m(X_{t_{m}})}\\
\end{align*}
where in the third line we have the fact that $T_{n,t} f$ converges to $T_t f$  for every $f \in C(S)$ to see
\begin{align*}
&\lim_{n \to \infty} \abs{\expectation{f_0(X^n_{t_0}) \dotsb f_{m-1}(X^n_{t_{m-1}}) (T_{n,t_m - t_{m-1}} - T_{t_m-t_{m-1}}) f_m(X^n_{t_{m-1}})}}  \\
&\leq \norm{f_0} \dotsb \norm{f_{m-1}} \lim_{n \to \infty} \norm{(T_{n,t_m - t_{m-1}} - T_{t_m-t_{m-1}}) f_m} = 0
\end{align*}
and in the fourth line applied the induction hypothesis to the continuous functions $f_0, \dotsc, f_{m-2}, f_{m-1} T_{t_m-t_{m-1}} f_m$.

Because $S$ is separable we know that the finite dimensional distributions of $X^n$ converge to those of $X$.  (TODO: prove this; basically we use induction and the standard approximation of indicators by continuous functions to show that $\lim_{n \to \infty} \probability{(X^n_{t_0}, \dotsc, X^n_{t_m}) \in A_1 \times \dotsb \times A_m}$ and then use separability of $S$ to conclude that we have a generating $\pi$-system and monotone classes apply).

Now we need to show tightness of the family $X^n$.  We appeal to the
Aldous criterion so we must show that for every bounded sequence of
optional times $\tau_n$ and deterministic sequence $\delta_n$ such
that $\lim_{n \to \infty} \delta_n = 0$ we have $r(X^n_{\tau_n},
X^n_{\tau_n + \delta_n}) \toprob 0$.  It suffices to show that every subsequence $N^\prime \subset \naturals$ has a futher
subsequence $N^{\prime \prime} \subset N^{\prime}$ such that $r(X^n_{\tau_n},X^n_{\tau_n + \delta_n}) \toprob 0$ along $N^{\prime \prime}$ (for then
we can use Lemma \ref{ConvergenceInProbabilityAlmostSureSubsequence} to find yet another subsequence $N^{\prime \prime \prime} \subset N^{\prime \prime}$ such
that $r(X^n_{\tau_n},X^n_{\tau_n + \delta_n}) \toas 0$ along $N^{\prime \prime \prime}$ and using Lemma \ref{ConvergenceInProbabilityAlmostSureSubsequence} in the
opposite direction we conclude that $r(X^n_{\tau_n},X^n_{\tau_n + \delta_n}) \toprob 0$; alternatively just note that convergence in probability is convergence with respect to the
Ky Fan metric).

Let $\nu_n = \mathcal{L}(X^n_{\tau_n})$ for $n \in \naturals$.  Then
by compactness of $S$ the family $\nu_n$ is automatically tight and therefore relatively compact by Prohorov's Theorem \ref{Prohorov}.  Given any 
subsequence $N^\prime$ then there is a further subsequence $N^{\prime \prime}$ and a probability measure $\nu$ such that $\nu_n \todist \nu$.  In what follows we
implicit working along the subsequence $N^{\prime \prime}$ and mention it no further.

Let $f$ and $g$ be continuous functions on $S$ then since $\delta_n$ converges it is bounded and by Theorem \ref{KurtzSovaSemigroupConvergence} and strong continuity of $T$ we have
\begin{align*}
\lim_{n \to \infty} \norm{T_{n, \delta_n} g - g} &\leq \lim_{n \to \infty} \left[ \norm{T_{n, \delta_n} g - T_{\delta_n} g} + \norm{T_{\delta_n} g - g} \right ] \\ 
&\leq \lim_{n \to \infty} \sup_{0 \leq h \leq \sup_n \delta_n} \norm{T_{n,h} g - T_{h} g} + \lim_{n \to \infty}\norm{T_{\delta_n} g - g} = 0
\end{align*}
and by the Strong Markov Property Theorem \ref{StrongMarkovFellerProcess} and Proposition \ref{TransitionSemigroupAsExpectation}
\begin{align*}
\expectation{f(X^n_{\tau_n}) g(X^n_{\tau_n+\delta_n})} &= \expectation{\cexpectationlong{\mathcal{F}_{\tau_n}}{f(X^n_{\tau_n}) g(X^n_{\tau_n+\delta_n})}} \\
&= \expectation{\sexpectation{f(X^n_{0}) g(X^n_{\delta_n})}{X^n_{\tau_n}}} \\
&= \expectation{\sexpectation{f(X^n_{0}) \cexpectationlong{X^n_0}{g(X^n_{\delta_n})}}{X^n_{\tau_n}}} \\
&= \expectation{\sexpectation{f(X^n_{0}) T_{n,\delta_n}g(X^n_{0})}{X^n_{\tau_n}}} \\
&= \expectation{\cexpectationlong{\mathcal{F}_{\tau_n}}{f(X^n_{\tau_n}) T_{n,\delta_n}g(X^n_{\tau_n})}} \\
&= \expectation{f(X^n_{\tau_n}) T_{n,\delta_n}g(X^n_{\tau_n})} \\
\end{align*}
Putting these two facts together with the expectation rule Lemma \ref{ExpectationRule} and the prior claim we see that
\begin{align*}
\lim_{n \to \infty} \expectation{f(X^n_{\tau_n}) g(X^n_{\tau_n+\delta_n})} &= \lim_{n \to \infty} \expectation{f(X^n_{\tau_n}) T_{n,\delta_n}g(X^n_{\tau_n})}\\
&= \lim_{n \to \infty} \expectation{f(X^n_{\tau_n}) T_{n,\delta_n}g(X^n_{\tau_n}) - f(X^n_{\tau_n}) g(X^n_{\tau_n})} + \lim_{n \to \infty} \expectation{f(X^n_{\tau_n}) g(X^n_{\tau_n}) } \\
&= \lim_{n \to \infty} \int f(s) g(s) \, \nu_n(ds) \\
&= \int f(s) g(s) \, \nu(ds) \\
\end{align*}
Thus by the same argument we used in concluding that fdd's converged we have $\mathcal{L}(X^n_{\tau_n}, X^n_{\tau_n+\delta_n}) \toweak \nu$ where we are regarding $\nu$ as the measure on $S \times S$ concentrated on the diagonal.
By the Continuous Mapping Theorem \ref{ContinuousMappingTheorem} and the fact that $\nu$ is concentrated on the diagonal,
\begin{align*}
r(X^n_{\tau_n}, X^n_{\tau_n+\delta_n}) \todist 0
\end{align*}
and we conclude that $r(X^n_{\tau_n}, X^n_{\tau_n+\delta_n}) \toprob 0$ by Lemma \ref{ConvergeInDistributionToConstant} . Now apply Theorem \ref{SkorohodInfiniteAldousCriterion}.
\end{proof}

\subsection{Approximating Feller Processes by Markov Chains}

Now we turn to the consideration of approximating Feller processes by Markov chains.  Underlying the result is a theorem about the approximation of strongly continuous contraction semigroups by powers of contraction operators.   Before stating the theorem we call out an estimate that will be required in the proof.

\begin{lem}\label{KurtzDiscreteTimeSemigroupConvergenceEstimate}Let $B$ be a contraction operator on a Banach space $X$ then for all $v \in X$ and $n \in \integers_+$ we have
\begin{align*}
\norm{B^n v - e^{n(B - \IdentityMatrix)} v} &\leq \sqrt{n} \norm{Bv - v}
\end{align*}
\end{lem}
\begin{proof}
First let $n \geq m$ and note that
\begin{align*}
\norm{B^n v - B^m v} &= \norm{\sum_{k=0}^{n-m-1} B^{m+k} (B v - v)} \leq \sum_{k=0}^{n-m-1} \norm {B^{m+k}} \norm{B v - v} \leq (n-m) \norm{Bv-v}
\end{align*}
Expanding in power series and using this estimate and Cauchy Schwartz we get
\begin{align*}
\norm{B^n v - e^{n(B - \IdentityMatrix)} v} &= e^{-n} \norm{e^n B^n v - e^{nB} v} 
= e^{-n} \norm{\sum_{k=0}^\infty \frac{n^k}{k!} (B^n v - B^k v)} \\
&= e^{-n} \sum_{k=0}^\infty \frac{n^k}{k!} \abs{n-k} \norm{B v - v} \\
&\leq e^{-n} \norm{B v - v} \left( \frac{n^k}{k!} (n-k)^2\right)^{1/2} \left( \frac{n^k}{k!} \right)^{1/2} \\
&=\norm{B v - v} \left( e^{-n} \frac{n^k}{k!} (n-k)^2\right)^{1/2} = \sqrt{n} \norm{Bv-v}\\
\end{align*}
where in the last line we have used the formula for the variance of a Poisson random variable of rate $n$ (Proposition \ref{MomentsPoissonDistribution})
\end{proof}

\begin{thm}\label{KurtzDiscreteTimeSemigroupConvergence}Let $X$ and $X_1, X_2, \dotsc$ be Banach spaces, let $T_{n}$ be a contraction operator on $X_n$, $\epsilon_n \geq 0 $ such that $\lim_{n\to \infty} \epsilon_n = 0$ and let $T_t$ be a strongly continuous contraction semigroup on $X$ with generator $A$.  Let $\pi_n : X \to X_n$ be bounded linear operators and assume that $\sup_{n} \norm{\pi_n} < \infty$.  Let $D$ be a
core for $A$ and define $A_n = \epsilon_n^{-1}(T_n-\IdentityMatrix)$ then the following are equivalent
\begin{itemize}
\item[(i)]For every $v \in X$ and every $t \geq 0$,  $\lim_{n \to \infty} \sup_{0 \leq s \leq t} \norm{T^{\floor{t/\epsilon_n}}_{n} \pi_n v - \pi_n T_s v} = 0$.
\item[(ii)]For every $v \in X$ and every $t \geq 0$,  $\lim_{n \to \infty} \norm{T^{\floor{t/\epsilon_n}}_{n} \pi_n v - \pi_n T_t v} = 0$.
\item[(iii)] For every $v \in D$ there exists $v_n \in \domain{A_n}$ such that $\lim_{n \to \infty} \norm{v_n - \pi_n v} = 0$ and $\lim_{n \to \infty} \norm{A_n v_n - \pi_n A v} = 0$.
\end{itemize}
\end{thm}
\begin{proof}
Before we proceed first note that $e^{t A_n}$ is in fact a contraction operator for all $t \geq 0$,
\begin{align*}
\norm{e^{t A_n}} &= \norm{e^{t \epsilon_n^{-1}(T_n - \IdentityMatrix)}} = e^{-t \epsilon_n^{-1}} \norm{e^{t \epsilon_n^{-1} T_n}} \leq e^{-t \epsilon_n^{-1}} e^{t \epsilon_n^{-1} \norm{T_n}} \leq 1
\end{align*}
Clearly (i) implies (ii).  To see that (ii) implies (iii) let $\lambda > 0$ be given, $v \in \domain{A}$ and set $w = (\lambda - A) v$ (so that $v = \int_0^\infty e^{-\lambda t} T_t w \, dt$).  For $n \in \naturals$ define
\begin{align*}
v_n &= \epsilon_n \sum_{k=0}^\infty e^{-\lambda k \epsilon_n} T_n^k \pi_n w \in X_n
\end{align*}
TODO: We need to show that $\norm{v_n}$ is bounded (or maybe we use Tonneli to push the limit inside the sum and don't worry about it); there may be something to this.
From (ii) and Dominated Convergence we have
\begin{align*}
\lim_{n \to \infty} \norm{v_n - \pi_n v} &= \lim_{n \to \infty} \norm{v_n - \int_0^\infty e^{-\lambda t} \pi_n T_t w \, dt} \\
&\leq  \lim_{n \to \infty} \norm{v_n - \int_0^\infty e^{-\lambda t} T^{\floor{t/\epsilon_n}}_{n} \pi_n v \, dt} \\
&= \lim_{n \to \infty} \norm{v_n - \sum_{k=0}^\infty \int_{k\epsilon_n}^{(k+1)\epsilon_n} e^{-\lambda t} T^{k}_{n} \pi_n v \, dt} \\
&= \lim_{n \to \infty} \norm{v_n - () \sum_{k=0}^\infty e^{-\lambda k \epsilon_n} T^{k}_{n} \pi_n v }\\
&= \lim_{n \to \infty} (1 - \frac{1-e^{-\lambda \epsilon_n}}{\lambda})\norm{v_n}  = 0\\
\end{align*}
We also get
\begin{align*}
(\lambda - A_n) v_n &= (\lambda - \epsilon_n^{-1}(T_n - \IdentityMatrix)) \epsilon_n \sum_{k=0}^\infty e^{-\lambda k \epsilon_n} T_n^k \pi_n w \\
&=\lambda \epsilon_n \sum_{k=0}^\infty e^{-\lambda k \epsilon_n} T_n^k \pi_n w 
- \sum_{k=0}^\infty e^{-\lambda k \epsilon_n} T_n^{k+1} \pi_n w +  
\sum_{k=0}^\infty e^{-\lambda k \epsilon_n} T_n^{k} \pi_n w \\
&= \lambda \epsilon_n \pi_n w 
+ e^{-\lambda \epsilon_n} \lambda \epsilon_n \sum_{k=0}^\infty e^{-\lambda k \epsilon_n} T_n^{k+1} \pi_n w 
- \sum_{k=0}^\infty e^{-\lambda k \epsilon_n} T_n^{k+1} \pi_n w +  \\
&\pi_n w  +
e^{-\lambda \epsilon_n}  \sum_{k=0}^\infty e^{-\lambda k \epsilon_n} T_n^{k+1} \pi_n w \\
&= \lambda \epsilon_n \pi_n w 
+ \pi_n w  + 
(e^{-\lambda \epsilon_n} \lambda \epsilon_n -1 + e^{-\lambda \epsilon_n} ) \sum_{k=0}^\infty e^{-\lambda k \epsilon_n} T_n^{k+1} \pi_n w \\
\end{align*}
Now take a difference, a limit and apply Tonelli's Theorem 
\begin{align*}
\lim_{n \to \infty} \norm{(\lambda - A_n) v_n - \pi_n w} &\leq 
\lim_{n \to \infty} \norm{\epsilon_n \lambda \pi_n w} + \lim_{n \to \infty} \sum_{k=0}^\infty \abs{e^{-\lambda \epsilon_n} \lambda \epsilon_n -1 + e^{-\lambda \epsilon_n} }
e^{-\lambda k \epsilon_n} \norm{T_n^{k+1} \pi_n w} \\
&\leq \lambda \sup_n \norm{\pi_n} \norm{w} \lim_{n \to \infty} \epsilon_n + \sup_n \norm{\pi_n} \norm{w} \sum_{k=0}^\infty \lim_{n \to \infty} \abs{e^{-\lambda \epsilon_n} \lambda \epsilon_n -1 + e^{-\lambda \epsilon_n} }
e^{-\lambda k \epsilon_n}  \\
&= 0
\end{align*}
Therefore 
\begin{align*}
\lim_{n \to \infty} \norm{A_n v_n - \pi_n A v} &\leq \lim_{n \to \infty} \left( \norm{(\lambda - A_n) v_n - \pi_n (\lambda - A) v} + \lambda \norm{v_n - \pi_n v} \right) = 0
\end{align*}

To see that (iii) implies (i), let $v \in X$ and pick $v_n \in X_n$ such that $\lim_{n \to \infty} \norm{v_n - \pi_n v} = 0$ and $\lim_{n \to \infty} \norm{A_n v_n - \pi_n A v} = 0$.  For
$n \in \naturals$ adding and subtracting terms and using the triangle equality we get
\begin{align*}
\norm{T^{\floor{t/\epsilon_n}}_n \pi_n v - \pi_n T_t v} 
&\leq \norm{T^{\floor{t/\epsilon_n}}_n \pi_n v - T^{\floor{t/\epsilon_n}}_n v_n}  + 
\norm{T^{\floor{t/\epsilon_n}}_n v_n - e^{\epsilon_n \floor{t/\epsilon_n} A_n} v_n} + \\
&\norm{e^{\epsilon_n \floor{t/\epsilon_n} A_n} v_n - e^{\epsilon_n \floor{t/\epsilon_n} A_n} \pi_n v} + 
\norm{e^{\epsilon_n \floor{t/\epsilon_n} A_n} \pi_n v - e^{t A_n} \pi_n v} + \\
&\norm{e^{t A_n} \pi_n v - \pi_n T_t v}
\end{align*}
We consider each of the terms on the right hand side in sequence.  For the first term simply note that since $T_n$ is a contraction operator
\begin{align*}
\limsup_{n \to \infty} \sup_{0 \leq s \leq t} \norm{T^{\floor{s/\epsilon_n}}_n \pi_n v - T^{\floor{t/\epsilon_n}}_n v_n} &\leq \lim_{n \to \infty} \norm{\pi_n v - v} = 0
\end{align*}
For the second term we use Lemma \ref{KurtzDiscreteTimeSemigroupConvergenceEstimate} to see 
\begin{align*}
\limsup_{n \to \infty} \sup_{0 \leq s \leq t} \norm{T^{\floor{s/\epsilon_n}}_n v_n - e^{\epsilon_n \floor{s/\epsilon_n} A_n} v_n} 
&=\limsup_{n \to \infty} \sup_{0 \leq s \leq t} \norm{T^{\floor{s/\epsilon_n}}_n v_n - e^{\floor{s/\epsilon_n} (T_n- \IdentityMatrix)} v_n}  \\
&\leq \limsup_{n \to \infty} \sup_{0 \leq s \leq t} \sqrt{\floor{s/\epsilon_n}} \norm{(T_n- \IdentityMatrix) v_n} \\
&= \limsup_{n \to \infty} \sqrt{\floor{t/\epsilon_n}} \epsilon_n \norm{A_n v_n} \\
&\leq \limsup_{n \to \infty} \sqrt{\floor{t/\epsilon_n}} \epsilon_n (\sup_n \norm{\pi_n} \norm{A v}  + \norm{A_n v_n - \pi_n A v}) \\
&= 0
\end{align*}
For the third term using the fact that $e^{tA_n}$ is a contraction operator 
\begin{align*}
\limsup_{n \to \infty} \sup_{0 \leq s \leq t}\norm{e^{\epsilon_n \floor{s/\epsilon_n} A_n} v_n - e^{\epsilon_n \floor{s/\epsilon_n} A_n} \pi_n v}
&\leq \limsup_{n \to \infty} \norm{v_n -\pi_n v} =0
\end{align*}
For the fourth term we use the Fundamental Theorem of Calculus, the fact that $e^{tA_n}$ is a contraction operator and the fact that $\norm{\pi_n A v - A_n \pi_n v} \to 0$ to see
\begin{align*}
\limsup_{n \to \infty} \sup_{0 \leq s \leq t} \norm{e^{\epsilon_n \floor{s/\epsilon_n} A_n} \pi_n v - e^{s A_n} \pi_n v} 
&=\limsup_{n \to \infty} \sup_{0 \leq s \leq t} \norm{\int^s_{\epsilon_n \floor{s/\epsilon_n}} \frac{d}{du} e^{uA_n} \pi_n v \, du} \\
&=\limsup_{n \to \infty} \sup_{0 \leq s \leq t} \norm{\int^s_{\epsilon_n \floor{s/\epsilon_n}} e^{uA_n} A_n \pi_n v \, du} \\
&\leq \limsup_{n \to \infty} \epsilon_n \norm{e^{uA_n}} \norm{ A_n \pi_n v }\\
&\leq \limsup_{n \to \infty} \epsilon_n (\sup_n \norm{\pi_n} \norm{Av} +  \norm{\pi_n A v - A_n \pi_n v }) = 0\\
\end{align*}
Lastly for the fifth term we see that 
\begin{align*}
\limsup_{n \to \infty} \sup_{0 \leq s \leq t} \norm{e^{s A_n} \pi_n v - \pi_n T_s v} = 0
\end{align*}
from Theorem \ref{KurtzSovaSemigroupConvergence} applied to the semigroups $e^{t A_n}$ and $T_t$.
\end{proof}

TODO: Trotter product formula and Chernoff product formula.  The Trotter product formula is inspired by the Lie product formula for matrix groups.

Possible exercise: According to Bratelli (who attributes to Hille) the Trotter product formula implies the Weierstrass approximation theorem (I think Goldstein has this as well).  Let $X$ be Banach space of bounded uniformly continuous functions and let $T_tf(x) = f(x+t)$ be the translation semigroup with generator $A=\frac{d}{dx}$.  Then let $A_n = (T_{1/n} -1)/(1/n)$ and the Trotter product formula then shows 
\begin{align*}
f(t) &= \lim_{n \to \infty} \sum_{m=1}^\infty t^m (A^m_n f)(0)/m!
\end{align*} 
where the convergence is uniform on compacts.  Apparently this implies the result.  Also chase down and understand how Goldstein uses the Chernoff product formula to show the central limit theorem.