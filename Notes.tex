\documentclass{amsbook}

\usepackage{dsfont}
\usepackage{tikz}

\DeclareMathOperator*{\argmin}{arg\,min}
\DeclareMathOperator*{\argmax}{arg\,max}
\DeclareMathOperator*{\card}{card}
\DeclareMathOperator*{\RePart}{Re}
\DeclareMathOperator*{\ImPart}{Im}
\DeclareMathOperator*{\median}{Med}
\DeclareMathOperator*{\diag}{Diag}
\DeclareMathOperator*{\interior}{int}
\DeclareMathOperator*{\diam}{diam}
\DeclareMathOperator*{\sgn}{sgn}
\DeclareMathOperator*{\IdentityMatrix}{Id}
\DeclareMathOperator*{\esssup}{ess\,sup}
\DeclareMathOperator*{\essinf}{ess\,inf}
\DeclareMathOperator*{\bplim}{bp-lim}

\newtheorem{thm}{Theorem}[chapter]
\newtheorem{cor}[thm]{Corollary}
\newtheorem{lem}[thm]{Lemma}
\newtheorem{prop}[thm]{Proposition}
\newtheorem{alg}[thm]{Algorithm}

\theoremstyle{definition}
\newtheorem{defn}[thm]{Definition}
\newtheorem{ex}[thm]{Exercise}
\newtheorem{xca}{Exercise}
\newtheorem{examp}[thm]{Example}

\theoremstyle{remark}
\newtheorem{rem}[thm]{Remark}
\newtheorem{clm}{Claim}[thm]

\newcommand{\Independent}{\perp \! \! \! \perp}
\newcommand{\cindependent}[3]{#1 \Independent_{#3} #2}
\newcommand{\expectation}[1]{\textbf{E}\left[#1\right]} 
\newcommand{\expectationop}{\textbf{E}}
\newcommand{\sexpectation}[2]{\textbf{E}_{#2}[#1]} 
\newcommand{\cexpectationop}[1]{\textbf{E}^#1} 
\newcommand{\cexpectation}[2]{\textbf{E}^{#1} #2} 
\newcommand{\cexpectationlong}[2]{\textbf{E}\left[ #2 \mid #1 \right]} 
\newcommand{\csexpectationlong}[3]{\textbf{E}_{#3}\left[ #2 \mid #1 \right]} 
\newcommand{\variance}[1]{\textbf{Var} \left (#1 \right )} 
\newcommand{\covariance}[1]{\textbf{Cov} \left (#1 \right )} 
\newcommand{\scovariance}[2]{\textbf{Cov} \left (#1, #2 \right )} 
\newcommand{\probabilityop}{\textbf{P}} 
\newcommand{\probability}[1]{\textbf{P}\{#1\}} 
\newcommand{\sprobability}[2]{\textbf{P}_{#2}\{#1\}} 
\newcommand{\sprobabilityop}[1]{\textbf{P}_{#1}}
\newcommand{\cprobability}[2]{\textbf{P}\{#2 \mid #1\}} 
\newcommand{\csprobability}[3]{\textbf{P}_{#3}\{#2 \mid #1\}} 
\newcommand{\characteristic}[1]{\textbf{1}_{#1}} 
\newcommand{\pushforward}[2]{#2 \circ #1^{-1}} 
\newcommand{\complexes}{\mathbb{C}} 
\newcommand{\reals}{\mathbb{R}} 
\newcommand{\rationals}{\mathbb{Q}} 
\newcommand{\naturals}{\mathbb{N}} 
\newcommand{\integers}{\mathbb{Z}} 
\newcommand{\distribution}[1]{\textbf{P}^{#1}}
\newcommand{\borel}[1]{\mathcal{B}(#1)} 
\newcommand{\abs}[1]{\left \vert #1 \right \vert}
\newcommand{\ceil}[1]{\lceil #1 \rceil}
\newcommand{\floor}[1]{\lfloor #1 \rfloor}
\newcommand{\norm}[1]{\lVert #1 \rVert}
\newcommand{\toprob}{\overset{P}\to}
\newcommand{\toas}{\overset{a.s.}\to}
\newcommand{\todist}{\overset{d}\to}
\newcommand{\toweak}{\overset{w}\to}
\newcommand{\tolp}[1]{\overset{L^#1}\to}
\newcommand{\eqfdd}{\overset{f.d.d.}=}
\newcommand{\eqdist}{\overset{d}=}
\newcommand{\kldiv}[2]{D\left( #1 \mid\mid #2 \right)}
\newcommand{\sample}[1]{\boldsymbol{#1}}
\newcommand{\range}[1]{\Re(#1)}
\newcommand{\domain}[1]{\mathcal{D}(#1)}
\newcommand{\resolventset}[1]{\rho(#1)}
\newcommand{\law}[1]{\mathcal{L}(#1)}
\newcommand{\andop}{\wedge}
\newcommand{\orop}{\vee}

\begin{document}


\frontmatter
\title{Probability Theory}
\author{David Blair}
\email{dblair@akamai.com}
\maketitle

\mainmatter

\tableofcontents

\chapter{Real Analysis}
For purposes of our discussion of measure theory, we often make little
use of the structure of the reals.  In many cases it is with little
effort that we can state results much more generally.  Sometimes the
results will be true of arbitrary sets but in other cases we need the
most basic notions of metric spaces.
\begin{defn}A metric space is a set $S$ together with a function
  $d:SxS \to \reals$ satisfying
\begin{itemize}
\item[(i)]$d(x,y) = 0$ if and only if $x=y$.
\item[(ii)]For all $x,y \in S$, $d(x,y) = d(y,x)$.
\item[(iii)]For all $x,y,z \in S$, $d(x,z) \leq d(x,y) + d(y,z)$.
\end{itemize}
\end{defn}
\begin{lem} Given a metric space $(S,d)$, we have $d(x,y) \geq 0$ for all
  $x,y \in S$.
\end{lem}
\begin{proof}
Let $x,y \in S$ and observe 
\begin{align*}
d(x,y) &= \frac{1}{2} (d(x,y) + d(y,x)) \textrm { by symmetry} \\
&\geq \frac{1}{2} d(x,x) \textrm{ by triangle inequality} \\
&= 0
\end{align*}
\end{proof}
 It's pretty easy to see that standard notions of limits and continuity
extend to the case of metric spaces.
\begin{defn}A sequence of elements $x_n \in S$ converges to $x \in S$
  if for every $\epsilon > 0$, there exists $N > 0$ such that
  $d(x_n,x) < \epsilon$ for all $n > N$.
\end{defn}
\begin{defn}A function between metric spaces $f : (S,d) \to (S', d')$
  is continuous at $x \in S$ if for every $\epsilon>0$, there exists
  $\delta > 0$ such that for $y\in S$ such that $d(x,y)<\delta$ we
  have $d'(f(x),f(y)) < \epsilon$.  A function $f$ that is continuous
  at all points $x \in S$ is said to be continuous.
\end{defn}
\begin{lem}$f : (S,d) \to (S', d')$ is continuous at $x \in S$ if and only if for
  every $x_n \to x$ we have $f(x_n) \to f(x)$.
\end{lem}
\begin{proof}
Suppose $f$ is continuous and let $\epsilon > 0$ be given.  By
continuity, we can pick $\delta > 0$ such that for all $y \in S$
with $d(x,y) < \delta$ we have $d'(f(x), f(y)) < \epsilon$.  Now by
convergence of the sequence $x_n$, we can find $N$ such that for all
$n>N$, we have $d(x_n,x) < \delta$.  Hence for all $n > N$, we have
$d'(f(x), f(x_n)) < \epsilon$.

Now suppose that for every $x_n \to x$ we have $f(x_n) \to f(x)$.  We
argue by contradiction.  Suppose $f$ is not continuous at $x$.  There
exists $\epsilon > 0$ such that we can find $x_n \in S$ such that
$d(x,x_n) < 2 ^ {-n}$ and $d'(f(x_n), f(x)) \geq \epsilon$.  Note that
the sequence $x_n \to x$ but $f(x_n)$ doesn't converge to $f(x)$.
\end{proof}
\begin{defn}For $x\in S$ and $r \geq 0$, the open ball at $x$ or
  radius $r$ is the set
\begin{align*}
B(x;r) = \{y\in S | d(x,y) < r \}
\end{align*}
\end{defn}
\begin{defn}A set $U \subset S$ is open if for every $x \in U$ there
  exists $r>0$ such that $B(x;r) \subset U$.  The complement of an
  open set is called a closed set.
\end{defn}
\begin{lem}A set $A \subset S$ is closed if and only if for every $x_n
  \to x$ with $x_n \in A$, we have $x \in A$.
\end{lem}
\begin{proof}
Suppose $A$ is closed.  Then $A^c$ is open.  Let $x_n \in A$ converge
to $x$.  If $x \notin A$, then $x \in A^c$ and we can find an open
ball $B(x;\epsilon) \subset A^c$.  Pick $N>0$ such that $d(x_n, x) <
\epsilon$ for all $n > N$.  Then $x_n \notin A$ for all $n>N$ which is
a contradiction.

Now suppose $A$ contains all of its limit points.  We show that $A^c$
is open.  Let $x\in A^c$ and suppose the balls $B(x;2^{-n}) \bigcap A
\ne \emptyset$.  Then we can construct a sequence $x_n \in A$ such
that $x_n ->x$.  This is a contradiction, hence for some $n$, we have 
$B(x;2^{-n}) \bigcap A = \emptyset$ and therefore $A^c$ is open.
\end{proof}

As it turns out continuity of a function can be expressed entirely in
terms of open sets.
\begin{lem}A function between metric spaces $f : (S,d) -> (T,d^\prime)$ is continuous
  if and only if for every open subset $U \subset T$, we have $f^{-1}(U)$ is an
  open subset of $S$.
\end{lem}
\begin{proof}For the only if direction, let $U \subset T$ be an open set and pick $x \in
  f^{-1}(U)$.  Now, $f(x) \in U$ and by openness of $U$ we can find $\epsilon
  > 0$ such that $B(f(x); \epsilon) \subset U$.  By continutity of $f$
  we can find a $\delta > 0$ such that for all $y \in S$ with $d(x,y) <
  \delta$ we  have $d^\prime(f(x),f(y)) < \epsilon$.  This is just
  another way of saying $B(x; \delta) \subset f^{-1}(U)$ which shows
  that $f^{-1}(U)$ is open.

For the if direction, pick $x \in S$ and suppose we are given
$\epsilon > 0$.  The ball $B(f(x); \epsilon)$ is an open set in $T$.  
By assumption we know that $f^{-1}(B(f(x); \epsilon))$ is
an open set in $S$ containing $x$.   By definition of openness, we can pick a $\delta > 0$, such that 
$B(x;\delta) \subset f^{-1}(B(f(x); \epsilon))$.  Unwinding this
statement shows that for all $y \in S$ with $d(x,y) < \delta$, we have
$d^\prime(f(x),f(y)) < \epsilon$ and we have show that $f$ is
continuous at $x$.  Since $x \in S$ was arbitrary we have shown $f$ is
continuous on all of $S$.
\end{proof}

\begin{defn}A sequence of elements $x_n \in S$ is said to be a
  \emph{Cauchy sequence}
  if for every $\epsilon > 0$, there exists $N > 0$ such that
  $d(x_n,x_m) < \epsilon$ for all $n,m > N$.
\end{defn}
Note that any convergent sequence is Cauchy.
\begin{lem}If a sequence of elements $x_n \in S$ converges to $x \in
  S$ then it is a Cauchy sequence.
\end{lem}
\begin{proof}Pick $\epsilon > 0$ and then pick $N>0$ so that $d(x_n,x)
  < \frac{\epsilon}{2}$ for all $n > N$.  Then by the triangle
  inequality, $d(x_n, x_m) \leq d(x_n, x) + d(x, x_m) < \epsilon$ for
  $n,m > N$.
\end{proof}
It is also easy to construct examples of Cauchy sequences that do not
converge by looking at spaces with \emph{holes}.
\begin{examp}Consider the sequence $\frac{1}{n}$ on $\reals \setminus
  \lbrace 0 \rbrace$.  It is Cauchy but does not converge.
\end{examp}
The existence of non-convergent Cauchy sequences is in some sense the
definition of what it means for a general metric space to have holes.
This motivates the following definition.
\begin{defn}A metric space $(S,d)$ is said to be \emph{complete} if
  every Cauchy sequence is convergent.
\end{defn}
\begin{defn}The real line $\reals$ is complete.
\end{defn}
\begin{proof}Suppose we are given a Cauchy sequence $x_n$.  Let $a =
  \liminf_{n \to \infty} x_n$ and $b = \limsup_{n \to \infty} x_n $.
  We proceed by contradiction and suppose
  that $a < b$ (note
  that the \emph{completeness axiom} of the reals is used in the
  definition of $\liminf$ and $\limsup$).  Let $M =b-a$ then for any
  $0 < \epsilon < M$, $N>0$
  we can find $k,m > N$ such that $\abs{a - x_k} < \frac{M -
    \epsilon}{2}$ and $\abs{b - x_m} < \frac{M -
    \epsilon}{2}$ thus showing $\abs{x_k - x_m} \geq \epsilon$ and
  contradicting the assumption that $x_n$ was a Cauchy sequence.
\end{proof}

The following is a simple fact about $\reals$.
\begin{lem}\label{IncreasingSequenceWithConvergentSubsequence}Let $x_n$ be a nondecreasing sequence in $\reals$.  Suppose
  there is an infinite subsequence $x_{n_k}$ such that $\lim_{k \to
    \infty} x_{n_k} = x$ , then $\lim_{n \to \infty} x_n = x$.
\end{lem}
\begin{proof}
TODO:  This is actually pretty much obvious.
\end{proof}
In our treatment of measure theory we'll want to have a detailed
understanding of the structure of the topology of the real line.  It
can be described quite simply.
\begin{lem}\label{OpenSetsOfReals}The open sets in $\reals$ are precisely the countable
  unions of disjoint open intervals.
\end{lem}
\begin{proof}Pick an open set $U \subset \reals$.  Define an
  equivalence relation on $U$ such that $a \equiv b$ if and only if
  $[a,b] \subset U$ or $[b,a] \subset U$.  It is easy to see this is an equivalence
  relation.  Reflexivity and symmetry are entirely obvious.
  Transitivity follows from taking a union of intervals (carefully
  taking order into consideration).  

Now, consider the equivalence classes of the relation. As equivalence
classes these sets are disjoint and their union is $U$.  Call the
family of equivalence classes $U_\alpha$.

We have to show that
the equivalence classes are open intervals.  Consider $x \in U_\alpha
\subset U$.  Openness of $U_\alpha$ follows from using openness of $U$ to find a small ball (open interval) around $x \in U$ and noting that every
point of the ball is $\equiv$-related to $x$.  Therefore the same open
ball demonstrates the openness of $U_\alpha$.

To see that equivalence classes are intervals, pick an equivalence
class $U_\alpha$ and consider the open interval $(\inf U_\alpha, \sup
U_\alpha)$.  Since $U_\alpha$ is nonempty and open, $\inf U_\alpha \neq \sup
U_\alpha $ and this interval is non-empty.  By definition of $\inf$ and $\sup$ and the openness of
$U_\alpha$ we can see that $U_\alpha \subset (\inf U_\alpha, \sup
U_\alpha)$ (otherwise we could find an element of $U_\alpha$ bigger
than $\sup$ or less than $\inf$).  On the other hand, suppose we are
given $x \in (\inf U_\alpha, \sup U_\alpha)$.  We can find elements
$y,z \in U_\alpha$ such that $\inf U_\alpha  < y < x < z <\sup  U_\alpha$.
By definition of the equivalence relation, this shows $[y,z] \subset
U_\alpha$ and therefore $x \in U_\alpha$.  Therefore we have shown
that $U_\alpha =  (\inf U_\alpha, \sup U_\alpha)$ is an open interval.

The fact that there are
at most countably many equivalence classes follows from the density
and countability of $\rationals$.
\end{proof} 

\begin{lem}\label{ComplementOfCountableSetDense}Let $A \subset \reals$ be a countable set.  Then $A^c$ is
  dense in $\reals$.
\end{lem}
\begin{proof}
Pick an $x \in \reals$ and consider an interval $I_n = (x - \frac{1}{n}, x +
\frac{1}{n})$ for $n > 0$.  Then if $A^c \cap I_n = \emptyset$ we have
$I_n \subset A$ which implies that $I_n$ is countable.  This is
clearly false (since otherwise we could write the reals as a countable
union of countable sets which would imply the reals themselves are countable).
\end{proof}

Just as an aside at this point, we note that notions of open and
closed set are really all that is needed to make sense of the notions
of convergence and continuity.
\begin{defn}A topological space is a set $S$ together with a
  collection of subsets $\tau$ satisfying
\begin{itemize}
\item[(i)]$\tau$ contains $\emptyset$ and $S$.
\item[(ii)]$\tau$ is closed under arbitrary union.
\item[(iii)]$\tau$ is closed under finite intersection.
\end{itemize}
The collection $\tau$ is called a topology on $S$.  The elements of
$\tau$ are called the open sets of $S$ and the complement of the open
sets are called closed sets.  As we have shown above, if one defines
continuity of a function between topological spaces as inverse images
of open sets being open we have a definition that is a compatible
generalization of the $\epsilon/\delta$ definition of calculus.
\end{defn}

\begin{thm}[Taylor's Theorem]\label{TaylorsTheorem}Let $f: \reals\to
  \reals$ be a function which is $m$-times continuously differentiable.  Then for all
  $0 \leq n < m$,
\begin{align*}
f(b) &= \sum_{k=0}^n \frac{(b-a)^k}{k!} f^{(k)}(a) + R_n(b)
\end{align*}
where the \emph{remainder term} is of the form
\begin{align*}
R_n(b) &= \int_a^b
\frac{(b-x)^n}{n!} f^{(n+1)}(x) \, dx
\end{align*}
\end{thm}
\begin{proof}
We proceed by induction.  Note that for $n=1$, then Taylor's Formula simply says $f(b) = f(a) +
\int_a^b f^\prime(x) \, dx$ which is just the Fundamental Theorem of
Calculus.  For the induction step, we integrate the remainder term by parts.  Consider
the integral $\int_a^b \frac{(b-x)^{(n-1)}}{(n-1)!} f^{(n)}(x) \, dx$ and let
$u = f^{(n)}(x)$ and $dv = \frac{(b-x)^{(n-1)}}{(n-1)!} dx$.  Then $du =
f^{(n+1)}(x) dx$ and $v = -\frac{(b-x)^n}{n!}$, so 
\begin{align*}
\int_a^b \frac{(b-x)^{(n-1)}}{(n-1)!} f^{(n)}(x) \, dx &=
-\frac{(b-x)^n}{n!} f^{(n)}(x) \mid_a^b + \int_a^b \frac{(b-x)^n}{n!}
f^{(n+1)}(x) \, dx \\
&= \frac{(b-a)^n}{n!} f^{(n)}(a) + \int_a^b \frac{(b-x)^n}{n!}
f^{(n+1)}(x) \, dx
\end{align*}
which proves the result.
\end{proof}
The version of Taylor's Formula above expresses the ``integral form''
of the remainder term.  It is often useful to transform the remainder term in Taylor's Formula
into the \emph{Lagrange form}.
\begin{lem}\label{LagrangeFormRemainder}There is a number $c \in
  (a,b)$ such that $\int_a^b
R_n(b)= f^{(n+1)}(c) \frac{(b-a)^{n+1}}{(n+1)!}$
\end{lem}
\begin{proof}
If $f^{(n+1)}(x)$ is constant on the interval $[a,b]$ then by explicit
integration we have the result for any $a < c < b$, so let us assume
that $f^{(n+1)}(x)$ is not constant on $[a,b]$.
By continuity of $f^{(n+1)}(x)$ and compactness of $[a,b]$ we know
that there exist $m, M \in \reals$ such that $m = \min_{x \in [a,b]}
f^{(n+1)}(x)$ and $M = \max_{x \in [a,b]}
f^{(n+1)}(x)$.
From this fact and the fact that $(b -x)^n$ is strictly positive on
$[a,b)$ we have bounds
\begin{align*}
m\frac{(b-a)^{n+1}}{(n+1)!} &= m \int_a^b
\frac{(b-x)^n}{n!} \, dx \\
&< \int_a^b
\frac{(b-x)^n}{n!} f^{(n+1)}(x) \, dx \\
&< M \int_a^b
\frac{(b-x)^n}{n!} \, dx = M\frac{(b-a)^{n+1}}{(n+1)!} 
\end{align*}
hence 
\begin{align*}
m < \frac {(n+1)!} {(b-a)^{n+1}}\int_a^b
\frac{(b-x)^n}{n!} f^{(n+1)}(x) \, dx < M
\end{align*}
By continuity of $f^{(n+1)}(x)$ and the Intermediate Value Theorem, we know that $f^{(n+1)}(x)$
takes every value in $[m,M]$ and therefore there exists $c \in [a,b]$
such that $f^{(n+1)}(c) = \frac {(n+1)!} {(b-a)^{(n+1)}}\int_a^b
\frac{(b-x)^n}{n!} f^{(n+1)}(x) \, dx$.  Because the inequalities are
strict and because $(b-x)^n$ is positive, it follows that in fact $c
\in (a,b)$.
\end{proof}

In addition to the integral form and the Lagrange form of the
remainder it can also be useful to have an estimate on the remainder
in hand.
\begin{cor}\label{TaylorsTheoremRemainderEstimate}Let $f: \reals\to
  \reals$ be a function which is $m$-times continuously differentiable.  Then for all
  $1 \leq n \leq m$,
\begin{align*}
f(b) &= \sum_{k=0}^n \frac{(b-a)^k}{k!} f^{(k)}(a) + r_n(b)
\end{align*}
where the remainder term satisfies
\begin{align*}
\abs{r_n(b)} &\leq 
\frac{\sup_{a \leq x \leq b}\abs{f^{(n)}(x) - f^{(n)}(a)}}{n!} \abs{b-a}^n
\end{align*}
in particular we have $\lim_{b \to a} \frac{r_n(b)}{(b-a)^n} = 0$.
\end{cor}
\begin{proof}
By Taylor's Theorem we have
\begin{align*}
f(b) &= \sum_{k=0}^{n-1} \frac{(b-a)^k}{k!} f^{(k)}(a) + R_{n-1}(b)
\end{align*}
with
\begin{align*}
R_{n-1}(b) &= \int_a^b \frac{(b-x)^{n-1}}{(n-1)!} f^{(n)}(x) \, dx \\
&=\int_a^b \frac{(b-x)^{n-1}}{(n-1)!} (f^{(n)}(x) - f^{(n)}(a)) \, dx +
 f^{(n)}(a) \int_a^b \frac{(b-x)^{n-1}}{(n-1)!} \, dx \\
&=\int_a^b \frac{(b-x)^{n-1}}{(n-1)!} (f^{(n)}(x) - f^{(n)}(a)) \, dx +
 f^{(n)}(a) \frac{(b-a)^{n}}{n!}  \\
\end{align*}
so that 
\begin{align*}
r_n(b) &= \int_a^b \frac{(b-x)^{n-1}}{(n-1)!} (f^{(n)}(x) - f^{(n)}(a)) \, dx
\end{align*}
and therefore
\begin{align*}
\abs{r_n(b)} &\leq \int_a^b \frac{\abs{b-x}^{n-1}}{(n-1)!} \abs{f^{(n)}(x) -
               f^{(n)}(a)} \, dx
\leq \sup_{a \leq x \leq b} \abs{f^{(n)}(x) - f^{(n)}(a)} \frac{\abs{b-a}^{n}}{n!} 
\end{align*}
The last statement follows from the continuity of $f^{(n)}(x)$.
\end{proof}

\begin{lem} Let $X$ be a real normed vector space with a subspace $Y$
  of codimension 1.  Then any bounded linear functional $\lambda$ on
  $Y$ extends to a bounded linear functional on $X$ with the same
  operator norm.
\end{lem}
\begin {proof}
We first assume that $\lambda$ has operator norm $1$.
Let $v$ be any vector that is not in $Y$.  Then every element of $X$
is of the form $y + tv$, hence by linearity all we really have to
choose is the value of $\lambda(v)$ so that the operator norm doesn't
increase.  
First, note that it suffices to show $|\lambda(y+v)| \leq
\|y+v\|$ for all $y$.  For it that if that is true then
\begin{align*}
|\lambda(y + tv)| &= |t\lambda(y/t + v)| \\
&\leq |t| \|y/t + v\| \\
&= \|y + tv\| 
\end{align*}
We rewrite the constraint $|\lambda(y+v)| \leq
\|y+v\|$ for all $y$ as
\begin{align*}
-\lambda(y) - \|y+v\| \leq \lambda(v) \leq  \|y+v\| - \lambda(y)
\end{align*}
To see that it is possible to satisfy the constraint derived above, we
use the triangle inequality (subadditivity) of the operator norm.  For
all $y_1,y_2 \in Y$,
\begin{align*}
\lambda(y_1) - \lambda(y_2) &\leq|\lambda(y_1 - y_2)| \\
&\leq \|y_1 - y_2\| \\
&= \|y_1 + v - v - y_2 \| \\
&\leq \|y_1 + v\| + \|y_2 + v\| 
\end{align*}
From which we conclude by rearranging terms 
\begin{align*}
\sup_{y_2 \in Y} -\lambda(y_2) - \|y_2 + v\| \leq \inf_{y_1 \in Y} \|y_1 + v\| - \lambda(y_1)
\end{align*} 
Picking any value between the two terms of the above inequality
results in a valid extension.
To handle the case of operator norm not equal to $1$, notice that the
extension is trivial if the operator norm is $0$ (i.e. $\lambda=0$), otherwise define the
extension by $\|\lambda\|$ times the extension of $\lambda/\|\lambda\|$.
\end {proof}
\begin{thm}[Hahn-Banach Theorem (Real case)]
Let $X$ be a real normed vector space with a subspace $Y$.  Then any bounded linear functional $\lambda$ on  $Y$ extends to a bounded linear functional on $X$ with the same operator norm.
\end{thm}
\begin{proof}We proceed by using the codimension 1 case proved above
  and then applying Zorn's Lemma.  We define a partial extension of
  $\lambda$ to be a pair $(Y^\prime, \lambda^\prime)$ such that $Y
  \subset Y^\prime \subset X$ and $\lambda^\prime$ is an extension of
  $\lambda$ with the same operator norm.  Put a partial order on the
  set of extensions by declaring $(Y^\prime, \lambda^\prime) \leq
  (Y^{\prime\prime}, \lambda^{\prime\prime})$ if and only if $Y^\prime
  \subset Y^{\prime\prime}$ and $\lambda^{\prime\prime}\mid_{Y^\prime}
  = \lambda^\prime$.

To apply Zorn's Lemma, we need to show that every chain has an upper
bound.  If we are given a chain $(Y_\alpha, \lambda_\alpha)$ then we
define $Z = \cup_\alpha Y_\alpha$ and for any $z \in Z$ we define
$\tilde{\lambda}(z) = \lambda_\alpha(z)$ for any $\alpha$ such that $z \in
Y_\alpha$.  It is immediate that this well defined.  It is easy to
show linearity and to show that $\norm{\tilde{\lambda}} =
\norm{\lambda}$ (TODO: do this).

Now we can apply Zorn's Lemma to conclude that there is a maximal
element $(Y^\prime, \lambda^\prime)$.  The codimension one case show
us that $Y^\prime = X$ for otherwise we can construct an extension
that shows $(Y^\prime, \lambda^\prime)$ is not maximal.
\end{proof}
Note that the use of Zorn's Lemma here is not accidental; the Hahn
Banach Theorem cannot be proven in set theory without the Axiom of
Choice (though according to Tao it can be proven without the full
power of the Axiom of Choice using what is know as the Ultrafilter Lemma).
\section{Compactness}

\begin{defn}Let $(S,d)$ be a metric space, then we say $K \subset S$
  is \emph{sequentially compact} if and only if for every sequence $x_1, x_2,
  \dots \in K$ there exists a convergent subsequence $x_{n_j}$ such
  that $\lim_{j \to \infty} x_{n_j} \in K$.
\end{defn}

\begin{defn}Let $(S,d)$ be a metric space, then we say $S$
  is \emph{compact} if and only if for every collection $U_\alpha$ of
  open sets such that $\bigcup_\alpha U_\alpha \supset S$ there
  exists a finite subcollection $U_1, \dots, U_n$ such that
  $\bigcup_{j=1}^n U_j \supset S$.
\end{defn}

\begin{defn}Let $(S,d)$ be a metric space, then we say $S$
  is \emph{totally bounded} if and only if for every $\epsilon >0$
  there exists a finite set of points $F \subset S$ such that for
  every $x\in S$ there is a $y \in F$ such that $d(x,y) < \epsilon$.
\end{defn}

\begin{defn}Let $(S,d)$ be a metric space, then we say $x \in S$
  is \emph{limit point} of a set $A \subset S$ if and only if for
  every open set $U$ containing $x$, $A \cap (U \setminus \lbrace x
  \rbrace) \neq \emptyset$.
\end{defn}

\begin{thm}\label{CompactnessInMetricSpaces}In a metric space $(S,d)$ the following are equivalent
\begin{itemize}
\item[(i)]$S$ is compact
\item[(ii)]$S$ is complete and totally bounded
\item[(iii)]Every infinite subset of $S$ has a limit point
\item[(iv)]$S$ is sequentially compact
\end{itemize}
\end{thm}
\begin{proof}
First we show that (i) implies (ii).  Given $\epsilon > 0$ note that
we have a covering by open balls $\cup_{x \in S} B(x, \epsilon)$.  By
compactness we have a finite set $x_1, \dots, x_m$ such that
$\cup_{i=1}^m B(x_i, \epsilon) = S$.  Thus given $y \in S$, we know
there is an $x_j$ such that $y \in B(x_j, \epsilon)$ and we have shown
total boundedness.  To show completeness, let $x_1, x_2, \dots$ be a
Cauchy sequence in $S$.  For every $m > 0$ we know there exists $N_m$
such that $d(x_{N_m}, x_n) < \frac{1}{m}$ for every $n > N_m$.  Now
define $U_m = \lbrace x \in S \mid d(x_{N_m}, x) > \frac{1}{m}\rbrace$
and note that $U_m$ is open. Furthermore we know that $x_n \notin U_m$
for all $n > N_m$.  By virtue of this latter fact we can see that
there is no finite subset of $U_m$ that covers $S$; for given $U_1,
\dots, U_m$ then $x_n \notin \cup_{k=1}^m U_k$ for any $n > \max(N_1,
\dots, N_m)$.  By compactness of $S$ we know that the $U_m$ do not
cover $S$ and therefore there is an $x \in S \setminus
\cup_{m=1}^\infty U_m$.  For such an $x$, by definition of $U_m$ we
know that $d(x_{N_m}, x) \leq \frac{1}{m}$ for all $m > 0$.  By the
triangle inequality we then get that $d(x_n, x) \leq \frac{2}{m}$ for
all $n > N_m$ and $m > 0$ which shows that $x_n$ converges to $x$.
Thus $S$ is complete.

Next we show that (ii) implies (iii).  Suppose $A \subset S$ is an
infinite set.  By the assumption of total boundedness, for each $n >
0$, we can find a finite set $F_n$ such that for every $y \in S$ there
exists $x \in F_n$ such that $d(x,y) < \frac{1}{n}$.  Since the finite
sets $B(y, 1)$ for $y \in F_1$ cover $S$ there is an $y_1 \in F_1$
such that $A \cap B(y_1, 1)$ is infinite.  Then arguing inductively we
construct for every $n>0$ a $y_n \in F_n$ such that $A \cap B(y_1,1)
\cap \cdots \cap B(y_n, \frac{1}{n})$ is infinite.  Note that for $n > m
>0$, by the triangle inequality using any of the infinite number of
elements in $B(y_n, \frac{1}{n}) \cap B(y_m, \frac{1}{m})$, we have $d(y_n, y_m) < \frac{1}{m} +
\frac{1}{n} < \frac{2}{m}$.  This shows that $y_n$ is a Cauchy
sequence and by assumption we know that this converges to some $y \in
S$ and by the above estimate on $d(y_n, y_m)$, we know that for every
$m > 0$, $d(y, y_m) < \frac{2}{m}$.  Therefore we have the inclusion
$B(y_m, \frac{1}{m}) \subset B(y, \frac{3}{m})$ and therefore $A \cap
B(y, \frac{3}{m})$ is also infinite which shows $y$ is a limit point
of $A$.

Next we show that (iii) implies (iv).  Let $x_1, x_2, \dots$ be an
infinite sequence with an infinite range and by (iii) we can get a limit point $x \in S$.
Thus we can find a subsequence $x_{n_1}, x_{n_2}, \dots$ such that
$x_{n_k} \in B(x, \frac{1}{k})$ which shows that the subsequence
converges.  If the sequence has a finite range then it is eventually
constant and converges.

Lastly let's show that (iv) implies (i).  Pick an open cover
$\mathcal{U}_\alpha$ of $S$.  Our first subtask is to show that there exists a
radius $r > 0$ such that for every $x \in S$, the ball $B(x,r)$ is contained in some
element of $\mathcal{U}_\alpha$.   To that end, for every $x \in S$ let 
\begin{align*}
f(x) &= \sup \lbrace r \mid B(x,r) \subset U_\alpha \text{ for some }
\alpha \rbrace
\end{align*}
We claim that $\inf \lbrace f(x) \mid x \in S \rbrace > 0$.  To verify
the claim, we argue by contradiction and assume we can find a sequence
$x_n$ with $f(x_n) < \frac{1}{n}$ (i.e. the ball $B(x_n, \frac{1}{n})$
is not contained in any $U_\alpha$). 
By sequential compactness we have a convergent subsequence $x_{n_k}$
that converges to $x \in S$.  Because $\mathcal{U}_\alpha$ is a open cover there we
can find an $r > 0$ and $U_\alpha$ such that $B(x, r) \subset
U_\alpha$.  Pick $N_1 > \frac{2}{r}$.  By convergence of $x_{n_k}$ we can find $N_2 > 0$ such that
for $n_k > N_2$ we have $d(x, x_{n_k}) < \frac{r}{2}$.  For $n_k >
\max(N_1, N_2)$, by the triangle inequality we have $B(x_{n_k},
\frac{1}{n_k}) \subset B(x,r) \subset U_\alpha$, so we have
a contradiction.

With the claim verified we return to the problem of proving
compactness.  Pick an arbitrary $x_1 \in S$ and let $c = 2 \wedge
\inf_{x \in S} f(x)$.  We define $x_n$ inductively by the following
algorithm: if there is exists $x_n$ such that $d(x_n, x_j) >
\frac{c}{2}$ for all $j=1, \dots, n-1$ then pick it otherwise stop.
We claim that the algorithm terminates after a finite number of
steps.  If it didn't then we'd have constructed an infinite sequence
$x_n$ such that for all $m,n > 0$ we have $d(x_n,x_m) > \frac{c}{2}$
which implies there is no Cauchy subsequence hence has no convergent
subsequence contradicting sequential compactness.  Therefore there is an $n>0$ such that $S = \cup_{k=1}^n
B(x_k, \frac{c}{2})$; however by construction we know that for every
$x_k$ there is a $U_k$ such that $B(x_k, \frac{c}{2}) \subset U_k$.
Then $U_1, \dots, U_n$ is a finite subcover of $S$ and we are done.
\end{proof}
It is worth noting that the equivalence of the finite subcover
property and sequential compactness does not hold in general
topological spaces.  In general sequential compactness is equivalent
to the weaker property that \emph{countable} open covers have finite
subcovers (sometime this property is refered to as countable
compactness).  It turns out that in these circumstances that the full
power of the finite subcover property is generally needed.

\begin{cor}\label{ClosedSubsetsCompact}Every closed subset of a compact set is compact.
\end{cor} 
\begin{proof}Let $B$ be a compact set and let $A \subset B$ be
  closed.  Let $U_\alpha$ be an open cover of $A$, the we may append
  $A^c$ to get an open cover of $B$.  By compactness of $B$ we may
  extract a finite subcover $U_{\alpha_1}, \dotsc, U_{\alpha_n}, A^c$
  (there is no loss in generality in assuming that $A^c$ is in the
  finite subcover).  Clearly, $U_{\alpha_1}, \dotsc, U_{\alpha_n}$ is
  a finite subcover of $A$.
\end{proof}

\begin{thm}\label{ContinuousImageOfCompact}Let $f : (S, d) \to
  (S^\prime, d^\prime)$ be continuous.  If $S$ is compact then $f(S)$
  is compact.
\end{thm}
\begin{proof}Let $U_\alpha$ be an open cover of $f(S)$.  By continuity
  of $f$, $f^{-1}(U_\alpha)$ is an open cover of $S$ and therefore has
  a finite subcover $f^{-1}(U_1), \dots, f^{-1}(U_n)$.  It is easy to
  see that $U_1, \dots, U_n$ is a finite subcover of $f(S)$:  if $y \in
  f(S)$, we can write $y = f(x)$ for $x \in S$; picking $i$ so that $x \in
  f^{-1}(U_i)$, we see that $y \in U_i$.
\end{proof}

The following is a characterization of compact sets in $\reals^n$.
\begin{thm}\label{HeineBorel}[Heine-Borel Theorem]A subset $A \subset
  \reals^n$ is closed and
  bounded if and only if it is compact.
\end{thm}
TODO:  I don't think it is worth doing the proof from scratch; this is
a simple corollary of the result.
\begin{proof}By Lemma \ref{CompactnessInMetricSpaces} it suffices to
  show that a closed and bounded set in $\reals^n$ is complete and
  totally bounded.  Completeness is simple as any Cauchy sequence in
  $A$ converges in $\reals^n$ by completeness of $\reals^n$ but then
  the limit is in $A$ because $A$ is closed.  To see total
  boundedness, pick an $\epsilon > 0$ and then pick $N >
  \frac{\sqrt{n}}{\epsilon}$.  Since $A$ is bounded, there exists $M >
  0$ such that $A \subset [-M, M] \times \cdots \times [-M,M]$.  It
  suffices to show that the latter set is totally bounded.  Pick
  the finite set of points $\lbrace (x_1/N, \dots , x_n/N) \mid -MN \leq x_j
    \leq MN \rbrace$ and note that 
\begin{align*} [-M, M] \times \cdots \times
  [-M,M] \subset \bigcup B((x_1/N, \dots , x_n/N), \epsilon)
\end{align*}
\end{proof}

Before we begin the proof we need a Lemma.
\begin{lem}Suppose $C_0 \supset C_1 \supset \cdots$ is a nested
  sequence of closed and bounded sets $C_k \subset \reals^n$.  Then
  $\cap_k C_k$ is non empty.
\end{lem}
\begin{proof}Here is the proof for $n=1$.  TODO: Generalize.

Let $a_k = \inf C_k$; because $C_k$ is closed we know that $a_k \in
C_k$.  By the nestedness and boundedness of $C_k$, we know that $a_k$
is a non-decreasing bounded sequence and therefore has a limit $a$.
For any fixed $k$, the sequence $a_n \in C_k$ for all $n \geq k$ and
thus $a=\lim_{n \to \infty} a_n \in C_k$.  Since $k$ was arbitrary we
have $a \in \cap_k C_k$ and we're done.
\end{proof}
With the Lemma in hand we can proceed to the proof of Heine-Borel.
\begin{proof}
Suppose $A$ is closed and bounded.  By boundedness there exists $N>0$
such that $A \subset [-N,N] \times \cdots \times [-N,N]$ and by
Corollary \ref{ClosedSubsetsCompact} it suffices to show that $ [-N,N]
\times \cdots \times [-N,N]$ is compact.

Now suppose that we are given an infinite open covering of $ [-N,N]
\times \cdots \times [-N,N]$ by sets $A_\alpha$ such that there is no
finite subcover.  Now bisect each side of the cube so that we can
write it as a union of $2^n$ cubes each of side $N$.  $A_\alpha$
covers each of the subcubes; if all of the subcubes had a finite
subcover of $A_\alpha$ then by taking the union we'd have constructed
a finite subcover of $ [-N,N]
\times \cdots \times [-N,N]$.  Since we've assumed that this isn't
true at least one of the subcubes has no finite subcover.   Pick that
cube, call it $C_1$ and now iterate the construction to create a
nested sequence of cubes $C_k$ where $C_k$ has side of length
$N/2^k$.  Since the $C_k$ are closed and bounded by the previous Lemma
we know that the intersection $\cap_k C_k \neq \emptyset$ and
therefore we can pick $x \in \cap_k C_k$.  Since $A_\alpha$ is a
cover, there exists an $A$ such that $x \in A$.  Because $A$ is open
we can in fact find a ball $B(x,r) \subset A$ for some $r > 0$.  Then
for sufficiently large $k$, $C_k \subset B(x,r) \subset A$ which means
that we have constructed a finite subcover for $C_k$ which is a contradiction.
\end{proof}

\begin{defn}Let $(S,d)$ and $(T, d^\prime)$ be metric spaces, a
  function $f : S \to T$ is said to be \emph{uniformly continuous} if
  for every $\epsilon > 0$ there exists a $\delta > 0$ such that
  $d(x,y) < \delta$ implies $d^\prime(f(x), f(y)) < \epsilon$.
\end{defn}

\begin{thm}\label{UniformContinuityOnCompactSets} Let $f : (S,d) \to (T, d^\prime)$ be a continuous function, if $S$ is
  compact then $f$ is uniformly continuous.
\end{thm}
\begin{proof}
The proof is by contradiction.  Suppose that $f$ is not uniformly
continuous.  Fix an $\epsilon > 0$, for every $n > 0$ we can find
$x_n$ and $y_n$ such that $d(x_n, y_n) < \frac{1}{n}$ but
$d^\prime(f(x_n), f(y_n)) \geq \epsilon$.  Now by compactness and Theorem
\ref{CompactnessInMetricSpaces} we can find a
common convergence subsequence of both $x_n$ and $y_n$.  Let's say
$\lim_{j \to \infty} x_{n_j} = x$ and $\lim_{j \to \infty} y_{n_j} =
y$.  Note that for every $j>0$, 
\begin{align*}
d(x,y) = \lim_{j \to \infty } d(x,y)\leq \lim_{j \to \infty } d(x,
x_{n_j}) + d(x_{n_j}, y_{n_j}) + d(y_{n_j}, y) = 0
\end{align*}
therefore $x=y$ and $f(x)=f(y)$.  

Again using the triangle inequality we see
\begin{align*}
\lim_{j \to \infty} d^\prime(f(x_{n_j}), f(y_{n_j})) \leq \lim_{j \to \infty}
d^\prime(f(x_{n_j}), f(x)) + d^\prime(f(x), f(y)) + d^\prime(f(y), f(y_{n_j})) = 0
\end{align*}
which is the desired contradiction.
\end{proof}

\begin{lem}\label{IntersectionOfNestedCompactSets}Let $K_1 \supset K_2
  \supset \cdots$ be a nested collection of non-empty compact sets,
  then $\cap_{n=1}^\infty K_n$ is nonempty.
\end{lem}
\begin{proof}Pick $x_n \in K_n$ and note that by compactness there is
  a convergent subsequence.  Let $x$ be the limit of that convergent
  subsequence.  By nestedness and closedness of each $K_n$ we conclude
  that $x \in K_n$ for every $n$.
\end{proof}

\begin{thm}Let $f : S \to \reals^n$ be a continuous function, if $S$ is
  compact then $f$ is bounded.
\end{thm}
\begin{proof}
By the Heine-Borel Theorem and Theorem \ref{ContinuousImageOfCompact}, we know that $f(S)$ is a closed bounded set.
\end{proof}

A related notion is that of uniform convergence of functions.
\begin{defn}Let $f, f_n : S \to (S, d^\prime)$ be a sequence
  of functions.  The we way that $f_n$ converges to $f$
  \emph{uniformly} if and only if for every $\epsilon > 0$ there
  exists a $N > 0$ such that for all $x \in S$, and $n > N$, $d^\prime(f_n(x), f(x)) < \epsilon$.
\end{defn}

One of the most important points about uniform convergence is that a
uniform limit of continuous functions is continuous.
\begin{lem}\label{UniformLimitContinuousFunctionsIsContinuous}Let $f,
  f_n : (S,d) \to (S^\prime, d^\prime)$ be a sequence
  of functions where $f_n$ are continuous.  If the $f_n$ converge
  to $f$ uniformly then $f$ is continuous.
\end{lem}
\begin{proof}
Suppose we are given an $\epsilon > 0$ and let $x \in S$.  By uniform
convergence of $f_n$ we may find an $N > 0$ such that
$d^\prime(f_n(y), f(y)) < \frac{\epsilon}{3}$ for all $n \geq N$ and
$y \in S$.  In particular, consider $f_N$.  Since this function is
continuous we may find $\delta > 0$ so that $d(x,y) < \delta$ implies
$d^\prime(f_N(x), f_N(y)) < \frac{\epsilon}{3}$.  So by the triangle
inequality, we have 
\begin{align*}
d^\prime(f(x), f(y)) < d^\prime(f(x), f_N(x)) + d^\prime(f_N(x),
f_N(y)) + d^\prime(f_N(y), f(y)) < \epsilon
\end{align*}
\end{proof}

\begin{prop}\label{ExtensionOfUniformlyContinuousMapCompleteRange}Let $(S,d)$ be a metric space and $(T,d^\prime)$ a
  complete metric space.  Suppose that $A \subset S$ and that $f: A
  \to T$ is a uniformly continuous function, then $f$ has a unique
continuous  extension $\overline{f} : \overline{A} \to T$ to the closure of $A
  \subset S$.  
\end{prop}
\begin{proof}
Let $x \in \overline{A}$, pick a sequence $x_n$
in $A$ such that $\lim_{n \to \infty} x_n = x$ and observe that by
uniform continuity of $f(x)$, for
any $\epsilon > 0$ there exists a $\delta > 0$ such that $d(z,w)
< \delta$ implies $d^\prime(f(z) , f(w)) < \epsilon$.  If we pick $N > 0$
such that $d(x_n,x) < \delta/2$ for $n \geq N$ then
$d(x_n, x_m) < \delta$ for all $n,m \geq N$ and thus $d^\prime(f(x_n), f(x_m)) < \epsilon$ for all
$n,m \geq N$.  This shows that the sequence $f(x_n)$ is Cauchy and by
completeness of $T$ we can take the limit; we define $f(x) =
\lim_{n\to \infty} f(x_n)$.  We claim that this definition is
independent of the sequence chosen.  Indeed, let $y_n$ be another
sequence from $A$ such that $\lim_{n \to \infty} y_n = x$.  Pick an
$\epsilon > 0$ and by uniform continuity of $f(x)$ let $\delta$ be
chosen such that $d^\prime(f(z),f(w)) < \epsilon/2$ whenever $d(z,w) < \delta$.
There exists $N_1 > 0$ such that $d(y_n, x_n) < \delta$ for every
$n > N_1$ and there exists $N_2 > 0$ such that $d^\prime(f(x_n),f(x)) <
\epsilon/2$ for all $n \geq N_2$.  Then we have for all $n \geq N_1 \vee
N_2$ by the triangle inequality $d^\prime(f(y_n), f(x)) < \epsilon$.
Note that this also shows that the extension $f(x)$ to $\overline{A}$ is
continuous at $x \in \overline{A}$; since is was continuous at all points
of $A$ we know the extension is continuous.  
\end{proof}

\section{Stone Weierstrass Theorem}

\begin{lem}\label{ApproximationOfLatticeInfimum}Let $L$ be a lattice of continuous functions on a compact Hausdorff
  space $X$ and suppose that the pointwise infimum $g(x) = \inf_{f \in
    L} f(x)$ is continuous.  Then for every $\epsilon > 0$ there
  exists $f \in L$ such that $0 \leq \sup \lbrace x \in X \mid f(x) - g(x)
  \rbrace < \epsilon$.
\end{lem}
\begin{proof}
For every $x \in X$ we can find an $f_x \in L$ such that $f_x(x) - g(x) <
\epsilon/3$.  By continuity of $f_x$ and $g$ we can find an open
neighborhood  $U_x$ of $x$ such that $\abs{f_x(x) - f_x(y)} < \epsilon/3$
and $\abs{g(x) - g(y)} < \epsilon/3$.  By the triangle inequality it
follows that $f_x(y) - g(y) < \epsilon$ for all $y \in U_x$.  The
$U_x$ are an open cover of $X$ so by compactness we may take a finite
subcover $U_{x_1}, \dotsc, U_{x_n}$.  Let $f = f_{x_1} \wedge \cdots
\wedge f_{x_n}$ then for every $x \in X$ we have $x \in U_{x_j}$ for
some $x_j$ and 
\begin{align*}
f(x) - g(x) &\leq f_{x_j}(x) - g(x) < \epsilon
\end{align*}
\end{proof}

\begin{lem}Let $L$ be a lattice of continuous functions on a compact
  Hausdorff space $X$ such that 
\begin{itemize}
\item[(i)]$L$ separates points (i.e. for every $x \neq y \in X$ there
  exists $f \in L$ such that $f(x) \neq f(y)$)
\item[(ii)]If $f \in L$ then for every $c \in \reals$ we have $cf \in
  L$ and $f + c \in L$.
\end{itemize}
Then for every continuous function $g$ on $X$ and $\epsilon > 0$ there
exists $f \in L$ such $0 \leq \sup \lbrace x \in X \mid f(x) - g(x)
\rbrace < \epsilon$.
\end{lem}
\begin{proof}
The first thing is to observe that for the lattice $L$ we have
complete control over the values of the function that separates points.

Claim 1: Suppose $x \neq y \in X$ and $a \neq b \in \reals$ then
there exists $f \in L$ such that $f(x) = a$ and $f(y) = b$.

To see the claim because $L$ separates points we have an $h \in L$
such that $h(x) \neq h(y)$.  Now it suffices to define
\begin{align*}
f(z) &= \frac{a -b}{h(x) - h(y)} h(z) + \frac{b h(x) -a h(y) }{h(x) - h(y)} 
\end{align*}
and note that by (ii) we have $f \in L$.

Claim 2: For any closed set $F \subset X$, $y \notin F$ and $a \leq b
\in \reals$ we can find $f \in L$ such that $f \geq a$, $f(y) = a$ and
$f(x) > b$ for all $x \in F$.

Pick an $x \in F$ then by Claim 1, we can find $f_x$ such that $f_x(x)
= b+1$ and $f_x(y) = a$.  By continuity of $f_x$ we have an open
neighborhood $U_x$ of $x$ such that $f(y) > b$ for all $y \in U_x$.
Clearly the $U_x$ form an open cover of $F$.
Since $F$ is closed and $X$ is compact Hausdorff we know that $F$ is
also compact hence we can extract a finite open cover $U_{x_1},
\dotsc,U_{x_n}$ of $F$.  Define
\begin{align*}
f &= (f_{x_1} \vee \cdots \vee f_{x_n}) \wedge a
\end{align*}
and observe that $f \in L$ since $L$ is a lattice and by (ii) $L$
contains the constant functions.

Now we can prove the Lemma proper.  With $g$ selected, let $L_g =
\lbrace f \in L \mid g \leq f \rbrace$.  Clearly $L_g$ is a lattice so
the result follows from Lemma \ref{ApproximationOfLatticeInfimum} if
we can show $g = \inf_{f \in L_g} f$.  Pick an $\delta > 0$ and a $y
\in X$, we try
to find $f \in L_g$ such that $f(y) - g(y) 
\leq \delta$.  First we find
such and $f \in L$ and then show that in fact $f \in L_g$.  Let $F =
\lbrace x\in X \mid g(x) + \delta \rbrace$ which is closed by
continuity of $g$.  By compactness of $X$ and continuity of $g$ we
know that $g$ has a maximum value $M$.  Using Claim 2 we know that we
can find $f \in L$ such that $g(y) +\delta \geq f(x)$ for all $x \in
X$, $g(y) + \delta = f(y)$ and $f(x) > M$ for all $x \in F$.  To see
that $f \in L_g$, note that by definition of $F$ and construction of $f$  for all $x \in X
\setminus F$ we have 
$g(x) < g(y) + \delta \leq f(x)$ and for all $x \in F$ we have $g(x)
\leq M < f(x)$.
\end{proof}

The Stone Weierstrass Theorem concerns the approximation properties of
subalgebras of $C(X)$ but we have been describing the approximation
properties of lattices of continuous functions.  The connection will
rely on the fact that we can uniformly approximate the absolute value
function by a polynomial on a compact interval.  We record that fact
as the following 
\begin{lem}For every $\epsilon > 0$ there exists a polynomial $p(x)$
  such that 
\begin{align*}
\sup \lbrace x \in [-1,1] \mid \abs{p(x) - \abs{x}}
  \rbrace < \epsilon
\end{align*}
\end{lem}
\begin{proof}
TODO:
\end{proof}

\begin{thm}[Stone Weierstrass Theorem]\label{StoneWeierstrassApproximation}Let $X$ be a compact Hausdorff space and let $A \subset
  C(X;\reals)$ be a subalgebra which contains a non-zero constant
  function.  The $A$ is dense in $C(X;\reals)$ if and only if $A$
  separates points.
\end{thm}
\begin{proof}
Let $\overline{A}$ be the uniform closure of $A$ (that is to say the
set of $f$ such that for every $\epsilon > 0$ there exists $g \in A$
such that $\sup \lbrace x \in X \mid \abs{g(x) - f(x)} \rbrace <
\epsilon$.  By Lemma \ref{UniformLimitContinuousFunctionsIsContinuous}
any such limit is continuous hence $\overline{A} \subset C(X)$. (TODO: The referenced result is stated
for a metric space domain however the proof clearly works for a domain
that is a general topological space).

TODO: Finish
\end{proof}

\begin{cor}[Fourier Series Approximation]\label{FourierSeries}For
  every continuous $f : \reals^n \to \reals$ such that $f(x + v) =
  f(x)$ for all $x \in \reals$, and $v \in \integers^n$, for every
$\epsilon > 0$  there exists constants $c_{j,k}$ and
$d_{j,k}$ such that 
\begin{align*}
\sup_x \abs{\sum_{j=0}^n \sum_{k=0}^N (c_{j,k} \sin(2k\pi x_j) + d_{j,k}
\cos(2k\pi x_j)) - f(x)} < \epsilon
\end{align*}
\end{cor}
\begin{proof}First we observe that there is a bijection between
  periodic function as in the hypothesis and functions on the
  topological space $T^n = S^1 \times \cdots \times S^1$ (the $n$-torus).
Observe that if one has a uniform approximation to a function
viewed as having a domain $T^n$ then the uniform approximation applies
equally well when considered as a periodic function on $\reals^n$.

It remains to observe that $T^n$ is compact Hausdorff, the functions $\sin(2k\pi x_j)$ an $\cos(2k
\pi x_j)$ separate points and contain the constants so the Stone Weierstrass
Theorem applies.

An alternative approach is a more constructive one using the Fejer kernel.
\end{proof}

\begin{cor}[Weierstrass Approximation
  Theorem]\label{WeierstrassApproximation}For every continuous
  function $f : [0,T] \to \reals$ there exists a sequence of
  polynomials $p_n(x)$ such that $\lim_{n \to \infty} \sup_{0 \leq x \leq T} \abs{f(x) -
    p_n(x)} = 0$.
\end{cor}



\chapter{Measure Theory}
Measure theory is concerned with the theory of integration.  Thinking
intuitively for a moment, we know that we want to compute
expressions of the form $\int_A f$ in which $A$ is a set and $f$ is a
real valued function on the set $A$.  If we take functions $f$ that
are equal to $1$ on the set $A$, then it is clear from our intuition
from elementary calculus that $\int_A 1$ should correspond the the
size of $A$ in some appropriate sense.  Therefore, even if we set out
to create a theory of integration we will get as a by product a theory
of set measure.  In fact, the development of the theory starts from
the notion of set measures and develops the theory of integration
using that.

Before setting out the definitions, it is worth mentioning that set
theory is a weird and wild territory.  Over the years, mathematicians
have come up with some truly astounding constructs with sets that defy
intuition. The first trivial example is to note the cardinality of $Z$
and $Z^2$ is the same.  A second much deeper example is the
Banach-Tarski Paradox which says in effect that there is a
decomposition of the unit ball in $\reals^3$ into a finite number of
pieces such that the pieces can be rearranged by only translations and
rotations into two copies of the unit ball.  We won't prove the
Banach-Tarski paradox here, but it suffices to say that it shows you
can't have all of the following in a definition of volume;
\begin{itemize}
\item[(i)]Translations are volume preserving.
\item[(ii)]Rotations are volume preserving.
\item[(iii)]All sets are measurable.
\end{itemize}

By now, the time honored approach to these matters is to give up on
the naive idea that all sets can be measured.  Thus the definition of a measure theory comprises a definition of
which sets are measurable, a means of measuring those sets and a
theory of integrating suitable functions using that measure.



\section{Measurable Spaces}
\begin{defn}A non-empty collection $\mathcal{A}$ of subsets of a set
  $\Omega$ is called a $\sigma$-algebra if given $A, A_1, A_2, \dots
  \in  \mathcal{A}$ we have
\begin{itemize}
\item[(i)]$A^c \in \mathcal{A}$
\item[(ii)]$\bigcup_n A_n \in \mathcal{A}$
\item[(iii)]$\bigcap_n A_n \in \mathcal{A}$
\end{itemize}
\end{defn}
Note that this definition makes a lot of sense.  Whatever our
definition of the class of measurable sets is, we want to be able to
perform meaningful constructions with those sets.  Thus we want the
set of allowable operations to be as large as possible.  On the other
hand, we know that we can't go beyond countable unions. For the
reals once one allows points to be measurable, allowing uncountable unions would
mean that every set is measurable and we already know we can't have that.
\begin{lem}Let $\sigma$-algebra $\mathcal{A}$ in $\Omega$, and $A_1,
  A_2, \dots \in \mathcal{A}$,
\begin{itemize}
\item[(i)] $\Omega \in \mathcal{A}$
\item[(ii)] $\emptyset \in \mathcal{A}$
\end{itemize}
\end{lem}
\begin{proof}
Since $\mathcal{A}$ is non empty, we can find $A \in \mathcal{A}$.
Thus $\Omega = A \bigcup A^c \in \mathcal{A}$.  Then taking
complements shows $\emptyset \in \mathcal{A}$.
\end{proof}
Note that in many accounts of measure theory, the result of the above lemma is assumed as part of
the definition of a $\sigma$-algebra.

\begin{lem}Given a class $\mathcal{C}$ of $\sigma$-algebras on
  $\Omega$, the intersection is also a $\sigma$-algebra.
\end{lem}
\begin{proof}
Because we have shown that every $\sigma$-algebra contains $\Omega$,
we know that the intersection is non-empty.  Now let $A, A_1, A_2, \dots$
be in every $\sigma$-algebra.  Clearly every $\sigma$-algebra in the
class contains $\bigcap_n A_n$, hence so does the intersection.
Similarly with $\bigcup_n A_n$ and $A^c$.
\end{proof}
Note that a union of $\sigma$-algebras is not necessarily a
$\sigma$-algebra.  However, a union of $\sigma$-algebras generates a
$\sigma$-algebra in an appropriate sense.
\begin{defn}Given a collection $\mathcal{C}$ of subsets of $\Omega$,
  we let $\sigma(\mathcal{C})$ be the smallest $\sigma$-algebra
  containing $\mathcal{C}$.
\end{defn}
Note that the definition makes sense since the set of all subsets of
$\Omega$ is a $\sigma$-algebra.  Therefore, the class of
$\sigma$-algbras containing $\mathcal{C}$ is non-empty and
$\sigma(\mathcal{C})$ is the intersection of of the class by the
previous lemma.

For metric spaces (and general topological spaces) there is an
important $\sigma$-algebra that is associated with the topology.
\begin{defn}Given a metric space $S$, the Borel $\sigma$-algebra
  $\borel{S}$ is the $\sigma$-algebra generated by the open sets on $S$.
\end{defn}

\begin{lem}\label{IntervalsGenerateBorel}The Borel $\sigma$-algebra of $\reals$ is generated by intervals
  of the form $(-\infty, x]$ for $x \in \rationals$.
\end{lem}
\begin{proof}Let $\mathcal{C}$ be the collection of all open
  intervals.
We know that the open sets of $\reals$ are countable
  unions of open intervals.  Therefore, the Borel $\sigma$-algebra is
  generated by the set of open intervals.  Now let $\mathcal{D}$ be
  the set of closed intervals of the form $(-\infty,x]$ for $x \in
  \rationals$.  Pick an open interval
  $(a,b)$ and pick a descreasing sequence or rationals $a_n \downarrow
  a$ and an increasing sequence of rationals $b_n \uparrow b$.  Then
  we have 
\begin{align*}(a,b) &= \bigcup_{n=1}^\infty (a_n,b_n] \\
&= \bigcup_{n=1}^\infty \left ( (-\infty,b_n] \cap (-\infty,a_n] \right )
\end{align*}
which shows that $\mathcal{C} \subset \sigma(\mathcal{D})$ hence 
$\sigma(\mathcal{C}) \subset \sigma(\mathcal{D})$.  However, since the
elements of $\mathcal{D}$ are closed sets and $\sigma$-algebras are
closed under set complement, we have $\mathcal{D} \subset
\sigma{\mathcal{C}}$ and therefore
\begin{align*}
\mathcal{B} = \sigma(\mathcal{C}) \subset \sigma(\mathcal{D}) \subset
\sigma(\mathcal{C}) = \mathcal{B}
\end{align*}
and we have $\sigma(\mathcal{D}) = \mathcal{B}$.
\end{proof}

Next we consider how $\sigma$-algebras behave in the presence of
functions.  Given a function $f:S \to T$ we have the induced map on
sets $f^{-1}: 2^T \to 2^S$ defined by 
\begin{align*}
f^{-1}(B) = \left \{x \in S; f(x) \in B \right \}
\end{align*}
\begin{lem}\label{SetOperationsUnderPullback}For $A, B,B_1,B_2,\dots
  \subset T$, then 
\begin{itemize}
\item[(i)] $f^{-1}(B^c) = \left[
    f^{-1}(B) \right ]^c$
\item[(ii)] $f^{-1} \bigcap_n B_n = \bigcap_n f^{-1}
  B_n$
\item[(iii)] $f^{-1} \bigcup_n B_n = \bigcup_n f^{-1}
  B_n$
\item[(iv)]$f^{-1}(B \setminus A) = f^{-1}(B) \setminus f^{-1}(A)$
\end{itemize}
\end{lem}
\begin{proof}
(i)\begin{align*}
f^{-1}(B^c) &= \left \{x \in S; f(x) \notin B \right \} \\
&= \left \{x \in S; f(x) \in B \right \}^c = \left[ f^{-1}(B) \right ]^c
\end{align*}
(ii)\begin{align*}
f^{-1} \bigcap_n B_n &= f^{-1} \left \{x \in T ; \forall n x \in B_n
\right \} \\
& = \left \{x \in S; \forall n f(x) \in B_n \right \} = \bigcap_n f^{-1}  B_n
\end{align*}
(iii)\begin{align*}
f^{-1} \bigcup_n B_n &= f^{-1} \left \{x \in T ; \exists n x \in B_n
\right \} \\
& = \left \{x \in S; \exists n f(x) \in B_n \right \} = \bigcup_n f^{-1}  B_n
\end{align*}

(iv) follows from (i) and (ii) by writing $B \setminus A = B \cap A^c$.
\end{proof}
\begin{lem}\label{SigmaAlgebraPullback}Given an arbitrary function $f$ between measurable spaces
  $(S,\mathcal{S})$ and $(T,\mathcal{T})$, then
\begin{itemize}
\item[(i)] $\mathcal{S}^\prime = f^{-1} \mathcal{T}$ is a
  $\sigma$-algebra on $S$.
\item[(ii)] $\mathcal{T}^\prime = \left \{A \subset T ; f^{-1}(A) \in
      \mathcal{S} \right \}$ is a $\sigma$-algebra on $T$.
\end{itemize}
The $\sigma$-algebra denoted $\mathcal{T}^\prime$ is often denoted
$f_* \mathcal{S}$.
\end{lem}
\begin{proof}
To show (i), let $A,A_1,A_2,\dots \in \mathcal{S}^\prime$.  Since
$\mathcal{S}^\prime = f^{-1}\mathcal{T}$, there exist $B,B_1,B_2,\dots
\in \mathcal{T}$ such that $A = f^{-1}(B)$ and $A_i = f^{-1}(B_i)$ for
$i=1,2,\dots$.  Now since $\mathcal{T}$ is a $\sigma$-algebra, we know
that $B^c$, $\bigcup_n B_n$ and $\bigcap_n B_n$ are all in
$\mathcal{T}$.  Now using the previous lemma,
\begin{align*}
A^c &= \left[f^{-1}(B)\right]^c &&= f^{-1}(B^c) \in
\mathcal{S}^\prime\\
\bigcap_n A_n &= \bigcap_n f^{-1}B_n &&= f^{-1} \bigcap_n B_n \in
\mathcal{S}^\prime \\
\bigcup_n A_n &= \bigcup_n f^{-1}B_n &&= f^{-1} \bigcup_n B_n \in
\mathcal{S}^\prime \\
\end{align*}
Now to see (ii), first note that $\mathcal{T}^\prime$ is non-empty
since $f^{-1} (\emptyset) = \emptyset \in \mathcal{S}$.  Next, pick $B,B_1,B_2,\dots \in \mathcal{T}^\prime$ so that
$f^{-1}B,f^{-1}B_1,f^{-1}B_2 \in \mathcal{S}$.  Again use the previous
lemma to see
\begin{align*}
f^{-1} B^c &= \left[f^{-1}(B)\right]^c \in
\mathcal{S}\\
f^{-1} \bigcap_n B_n &= \bigcap_n f^{-1}B_n \in
\mathcal{S} \\
f^{-1} \bigcup_n B_n &= \bigcup_n f^{-1}B_n \in
\mathcal{S} \\
\end{align*}
and this shows that $B^c,f^{-1} \bigcap_n B_n, f^{-1} \bigcap_n B_n
\in \mathcal{T}^\prime$.
\end{proof}

\begin{lem}\label{BijectivityOfInducedSetMap}Let $f : S \to T$ be a set function and $f^{-1} : 2^T \to
  2^S$ be the induced function on sets.
\begin{itemize}
\item[(i)]$f^{-1}$ is surjective if and only if $f$ is injective
\item[(ii)]$f^{-1}$ is injective if and only if $f$ is surjective
\item[(iii)]$f^{-1}$ is a bijection if and only if $f$ is a bijection
\end{itemize}
\end{lem}
\begin{proof}
Suppose $f$ is surjective and pick $A, B \subset T$ with $A \neq B$.
Then, possibly switching the names of $A$ and $B$, we have $t \in A
\setminus B$.  By surjectivity we know there exists an $s \in S$ such
that $f(s) = t$ and therefore $s \in f^{-1}(A) \setminus f^{-1}(B)$
showing $ f^{-1}(A) \neq f^{-1}(B)$.  Now if $f$ is not surjective
then there exists $t \in T$ such that there is no $s \in S$ with $f(s)
= t$.  In this case we see that $f^{-1}(T) = S = f^{-1}(T \setminus
\lbrace t \rbrace)$ showing $f^{-1}$ is not injective.

Suppose $f$ is injective and let $B \subset S$ and we claim $B =
f^{-1}(f(B))$.  Clearly $A \subset f^{-1}(f(B))$ and if they are not
equal then there exists $s \in S \setminus B$ such that $f(s) = f(b)$
for some $b \in B$ contradicting injectivity.  If $f$ is not injective
then there exists $s,t \in S$ with $s \neq t$ and $f(s) = f(t)$ and
clearly there can be no $A \subset T$ such that $f^{-1}(A) = \lbrace s \rbrace$.

The statement of (iii) is an immediate consequence of (i) and (ii).
\end{proof}

The definition given for $\sigma(\mathcal{C})$ for a set $\mathcal{C}
\subset 2^\Omega$ as the smallest $\sigma$-algebra containing
$\mathcal{C}$ may lack appeal because of the fact that it is
non-constructive.  It is possible to give a constructive definition of
$\sigma(\mathcal{C})$ by making a transfinite recursive definition.
The following makes use of the theory of ordinal numbers.
\begin{lem}Let $\mathcal{C} \subset 2^\Omega$, and let $\omega_1$ be
  the first uncountable ordinal and define for each countable ordinal
\begin{itemize}
\item[(i)]$\mathcal{C}_{\omega_0} = \mathcal{C}$
\item[(ii)]For a successor ordinal $\alpha$, $\mathcal{C}_\alpha$ is
  the set of countable unions of elements of $\mathcal{C}_{\alpha -
    1}$ and complements of such unions.
\item[(iii)]For a limit ordinal $\alpha$, define $\mathcal{C}_\alpha =
  \bigcup_{\beta < \alpha} \mathcal{C}_\beta$.
\end{itemize}
Then $\bigcup_{\alpha < \omega_1} \mathcal{C}_\alpha = \sigma(\mathcal{C})$.
\end{lem}
\begin{proof}
First we show $\bigcup_{\alpha < \omega_1} \mathcal{C}_\alpha \supset
\sigma(\mathcal{C})$.  Since we know that $\mathcal{C} \subset
\bigcup_{\alpha < \omega_1} \mathcal{C}_\alpha$, it suffices to show
that $\bigcup_{\alpha < \omega_1} \mathcal{C}_\alpha$ is a
$\sigma$-algebra.

It is explicit in the definition for successor
ordinals, that given any $A \in \mathcal{C}_\alpha$, we have $A^c \in  \mathcal{C}_{\alpha+1}$.

To show closure under set union, we suppose that we are
given $A_1, A_2, \dots$ where $A_i \in \mathcal{C}_{\alpha_i}$.  We now
use the fact that given a countable set of countable ordinals, there
is a countable ordinal that bounds them (TODO: Prove this somewhere or
find a good reference).  Thus we may pick a countable ordinal $\hat
\alpha$ such that $\alpha_i < \hat {\alpha}$ for every $i=1,2,\dots$.  Since
$\mathcal{C}_\alpha \subset \mathcal{C}_{\alpha+1}$, we know that $A_i
\in \mathcal{C}_{\hat \alpha}$ for all $i$.  Now simply apply the definition
of $\mathcal{C}_{\hat \alpha + 1}$ to see
$\bigcup_{i=1}^\infty A_i \in \mathcal{C}_{\hat \alpha + 1}$.  Having
proven closure under complement and countable union, use De
Morgan's Law to derive the countable intersection property and we are done.

Now we need to show that $\bigcup_{\alpha < \omega_1} \mathcal{C}_\alpha \subset
\sigma(\mathcal{C})$.  This is an easy transfinite induction on
$\alpha$ using the properties of the $\sigma$-algebra
$\sigma(\mathcal{C})$.
TODO: Write this out.
\end{proof}

\section{Measurable Functions}
We've seen that arbitrary set functions can be used to create
$\sigma$-algebras but when we consider functions between
measurable spaces the $\sigma$-algebras are given and it makes sense
to restrict our attention to a class of functions that are compatible
with those $\sigma$-algebras.
\begin{defn}A function $f : (S,\mathcal{S}) \to (T,\mathcal{T})$ is
  called measurable if for every $B \in \mathcal{T}$, we have
  $f^{-1}(B) \in \mathcal{S}$.  When we want to emphasize that the
  measurability is with repsect to particular $\sigma$-algebras we may
  say that $f$ is $\mathcal{S}/\mathcal{T}$-measurable.
\end{defn}
\begin{lem}\label{MeasurableByGeneratingSet}Suppose we are given a function $f : (S,\mathcal{S}) \to
  (T,\mathcal{T})$ and a class of subsets $\mathcal{C} \subset 2^T$
  such that $\sigma(\mathcal{C}) = \mathcal{T}$.  The $f$ is
  measurable if and only if $f^{-1} \mathcal{C} \subset \mathcal{S}$.
\end{lem}
\begin{proof}The only if direction is trivial.  So suppose $f^{-1}  \mathcal{C} \subset \mathcal{S}$.
Now consider $\mathcal{T}^\prime = \left \{ B \subset T; f^{-1} B \in
  S \right \}$.  By our assumption, we have $\mathcal{C} \subset
\mathcal{T}^\prime$.  Furthermore we know from
Lemma \ref{SigmaAlgebraPullback} that $\mathcal{T}^\prime$ is
a $\sigma$-algebra, thus $\sigma(\mathcal{C}) \subset
\mathcal{T}^\prime$ and this shows that $f$ is
$\mathcal{S}/\mathcal{T}$ measurable.
\end{proof}
\begin{lem}\label{CompositionOfMeasurable}Let $f:(S,\mathcal{S}) \to (T,\mathcal{T})$
  and $g:(T,\mathcal{T}) \to (U,\mathcal{U})$ be measurable.  Then $g
  \circ f :(S,\mathcal{S}) \to (U,\mathcal{U})$ is measurable.
\end{lem}
\begin{proof}This follows simply from the fact that $(g \circ
  f)^{-1}(B) = f^{-1}(g^{-1}(B))$ and the measurability of $f$ and $g$.
\end{proof}
Note, from this point forward, when we refer to $\reals$ as a
measurable space, it should be assumed that we are referring to
$\reals$ with the Borel $\sigma$-algebra.  Note that a function
$f:(\Omega, \mathcal{A}) \to \reals$ is measurable if and only if
$\left \{\omega \in \Omega ; f(\omega) \leq x \right \} \in \mathcal{A}$ for all $x \in \reals$ (in fact
it suffices to consider $x \in \rationals$).  It is also very common to consider
extensions of $\reals$ such as $\overline{\reals} = [-\infty,\infty]$
and $\overline{\reals}_+ = [0,\infty]$
obtained by appending points at infinity.  For these spaces we take the $\sigma$-algebra
generated by $\left \{\omega \in \Omega ; f(\omega) \leq x \right \}$
for $x \in \overline{\reals}$ respectively.  It can be shown that
there are natural topologies on each of these compactifications and
the $\sigma$-algebras defined are the Borel $\sigma$-algebras of these topologies.

We will often talk about the
convergence of sequences of measurable functions.  Unless we say
otherwise, it should be understood that this convergence is taken pointwise.
\begin{lem}\label{LimitsOfMeasurable}Let $f_1, f_2, \dots$ be measurable functions from
  $(\Omega, \mathcal{A})$ to $\overline{\reals}$.  Then $\sup_n f_n$,
  $\inf_n f_n$, $\limsup_n f_n$, $\liminf_n f_n$ are all measurable.
\end{lem}
\begin{proof}
To see measurability of $\sup_n f_n$ we suppose that $\omega \in
\Omega$ is such that $\sup_n f_n(\omega) \leq x$, then $x$ is an upper
bound we have $f_n(\omega) \leq x$ for all $n$.  On the other hand, if
we assume that $\omega \in \Omega$ is such that $f_n(\omega) \leq x$
for all $n$ then  $\sup_n f_n(\omega) \leq x$ so we have
\begin{align*}
\left \{\omega ; \sup_n f_n(\omega) \leq x \right \} = \bigcap_n \left
  \{\omega ; f_n(\omega) \leq x \right \} \in \mathcal{A}
\end{align*}
To see that $\inf_n f_n$ is measurable we use the identity $\inf_n f_n
= -\sup_n (-f_n)$.

We also have the definitions
\begin{align*}
\limsup_{n\to \infty} f_n = \inf_n \sup_{k \geq n} f_k \textrm{,} \liminf_{n\to \infty} f_n = \sup_n \inf_{k \geq n} f_k
\end{align*}
and the measurability of $\sup$ and $\inf$ already shown implies the
measurability of $\liminf$ and $\limsup$.
\end{proof}

From the measurability of limits of real valued functions we can also
generalize to measurability of limits in arbitrary metric spaces.
\begin{lem} \label{LimitsOfMeasurableMetricSpace}Let $(S,d)$ be a metric space and let
  $f_1, f_2, \dotsc$ be
  measurable functions  $(\Omega, \mathcal{A})$ to $(S,
  \mathcal{B}(S)$, then $\lim_{n \to \infty} f_n$ is measurable if it
  exists.
\end{lem}
\begin{proof}
Let $g : S \to \reals$ be an arbitrary continuous function.  Then $g$
is Borel measurable and therefore $g \circ f_n$ are Borel measurable
real valued functions.  Moreover by continuity of $g$ we know that
$\lim_{n \to \infty} g \circ f_n  = g \circ f$.  Therefore by Lemma
\ref{LimitsOfMeasurable} we can conclude that $g \circ f$ is Borel
measurable for all continuous $g : S \to \reals$.  

Now let $U \subset S$ be an open set and define $g_n(s) = n d(s,U^c)
\wedge 1$ so that $g_n$ are continuous functions such that $g_n
\uparrow \characteristic{U}$.  We know that $g_n \circ f$ are Borel
measurable hence if follows that $\characteristic{U} \circ f$ is Borel
measurable by another application of Lemma
\ref{LimitsOfMeasurable} which shows that $\lbrace f \in U \rbrace$ is
measurable.  Measurability of $f$ follows from the fact that open sets generate the Borel
$\sigma$-algebra and application of Lemma \ref{MeasurableByGeneratingSet}.
\end{proof}

We now introduce an extremely important class of measurable
functions.  Simple measurable functions will be used to approximate
arbitrary measurable functions and in particular, will serve as the
analogue of Riemann sums when we start to consider integration.

\begin{defn}Given a set $\Omega$ and a set $A
  \subset \Omega$, the \emph{indicator function} $\characteristic{A}$ is
  equal to $1$ on $A$ and $0$ on $A^c$.  A linear combination $c_1
  \characteristic{A_1} + \cdots + c_n \characteristic{A_n}$ is called
  a \emph{simple function}.
\end{defn}

\begin{prop}\label{prop:SimpleFunctions}A function $f : \Omega \to \reals$ is simple if and only it
  takes a finite number of values.  A simple function on a measurable space $(\Omega, \mathcal{A})$ is measurable if
  an only if $f^{-1}(c_j)$ is measurable for each of its distinct
  values $c_j \in \reals$.
\end{prop}
\begin{proof}
If $f = c_1  \characteristic{A_1} + \cdots + c_n \characteristic{A_n}$
is simple, then since indicator functions take only the value ${0,1}$
it is clear that $f$ can have at most $2^n$ values.

On the other hand, if $f : \Omega \to \reals$ only takes the finite
number of distinct values $c_1, \dots, c_n$ then clearly we may write 
$f = c_1  \characteristic{A_1} + \cdots + c_n \characteristic{A_n}$
where $A_j = f^{-1}(c_j)$.

As regards measurability, first notice that $\characteristic{A}$ is
measurable if and only if $A \in  \mathcal{A}$.  This follows from
that fact that there are only four possible preimages under
$\characteristic{A}$: $A, A^c, \Omega, \emptyset$ and each of these
preimages is the preimage of a measurable subset of $\reals$.

Similarly, if a simple function $f$ has the distinct values $c_1, \dots,
c_n$ (including $0$ if necessary) then clearly for $f$ to be
measurable it is necessary $f^{-1}(c_j)$ is measurable since points
are measurable in $\reals$.   On the hand, there are $2^n$ possible preimages under $f$ and they are all
constructed from unions of the preimages $f^{-1}(c_j)$ so if know that
$f^{-1}(c_j)$ are measurable then so is every $f^{-1}(A)$ for $A
\subset \reals$ (a stronger condition than measurability).
\end{proof}

It is useful to note that the measurability of indicator functions and sets are equivalent notions.
\begin{cor}\label{MeasurabilityOfCharacteristicFunctions}Let $(\Omega, \mathcal{A})$ be a measurable space and let $A  \subset \Omega$
then $\characteristic{A}$ is a measurable function if and only if $A$ is measurable.
\end{cor}
\begin{proof}
An exercise left to the reader.
\end{proof}

Note that the representation of a simple function as a linear
combination of indicator functions is not unique.  However, we
have
just shown that a simple function is equally well characterized as a
function that takes a finite number of values.  The canonical
representation of a simple function is a representation such that the
$c_i$ are distinct and non-zero and the $A_i$ are pairwise disjoint; the canonical
representation is unique.
\begin{lem}\label{PointwiseApproximationBySimple}For any positive measurable function $f : (\Omega,
  \mathcal{A}) \to \overline{\reals}_+$ there exist a sequence of simple measurable
  functions $f_1, f_2, \dots$ such that $0 \leq f_n \uparrow f$.
\end{lem}
\begin{proof}
Define
\begin{align*}
f_n(\omega) = 
\begin{cases}k2^{-n} & \text{if $k2^{-n} \leq f(\omega) < (k+1)2^{-n}$
    and $0 \leq k \leq n2^n -1$.} \\
n & \text{if $f(\omega) \geq n$.}
\end{cases}
\end{align*}
Note that $f_n$ is simple since it has at most $2^n + 1$ values $0,
\frac{1}{2^n}, \dots, n$.  $f_n$ is measurable since
$f_n^{-1}(k2^{-n}) = f^{-1}[k2^{-n},(k+1)2^{-n})$ is measurable by measurability of $f$.  Similarly
with $f_n^{-1}(n) = f^{-1} [n,\infty)$ and Proposition \ref{prop:SimpleFunctions}.
\end{proof}

As an application of approximation by simple functions, 
\begin{lem}\label{ArithmeticCombinationsOfMeasurableFunctions}
Let $f,g : (\Omega, \mathcal{A}) \to \reals$ be measurable functions
and let $a,b \in \reals$.  Then $af + bg$ and $fg$ are measurable and
$f/g$ is measurable when $g \neq 0$ on $\Omega$.
\end{lem}
\begin{proof}As $f$ and $g$ are measurable, we can apply the previous
  lemma to $f_\pm = \pm((\pm f) \wedge 0)$ and $g_{\pm} = \pm((\pm g) \wedge 0)$
  to get measurable simple functions $f_n$ and $g_n$ such that $\lim_{n \to
    \infty} f_n = f$ and $\lim_{n \to
    \infty} g_n = g$.  Basic properties of limits show that $\lim_{n
    \to \infty} (a f_n + b g_n) = a f + b g$, $\lim_{n \to \infty} f_n
  g_n = f g$ and $\lim_{n \to \infty} \frac{f_n}{  g_n} =
  \frac{f}{g}$.  Thus by Lemma \ref{LimitsOfMeasurable} we are done if
  we can show that each of $a f_n + b g_n$, $f_n g_n$ and
  $\frac{f_n}{g_n}$ is measurable.  In fact we will show that each of
  these is simple measurable.

It is easy to see that $a f_n + b g_n$ are also
  measurable simple as are $f_n g_n$.  Let $f_n$ take the values $c_1,
  \dots, c_s$ and let $g_n$ take the values $d_1, \dots, d_t$.  
Clearly the functions  $a f_n + b g_n$, $f_n g_n$ and
  $\frac{f_n}{g_n}$ are simple as each takes at most the values  $a c_i + b d_j$, $c_i d_j$ and
  $\frac{c_i}{d_j}$ for $i=1,\dots,s$ and $j=1,\dots, t$.  
Measurability follows from noting that each
  possible value of the linear combination is created from a finite
  set of combinations of the values of the $f_n$ and $g_n$; hence
  $(af_n + bg_n)^{-1}(c_j)$ is a finite union of intersections of the
  form $f_n^{-1}(x) \cap g_n^{-1}(y)$ where $x,y \in \reals$ are
  values of $f_n$ and $g_n$ respectively.
\end{proof}

\begin{defn}Given two measurable functions $f,g$ on the same measurable space
  $(\Omega, \mathcal{A})$, we say that $f$ is $g$-measurable if
  $\sigma(f) \subset \sigma(g)$.
\end{defn}

TODO: Where is the right place to introduce this concept?
While the basic results of measure theory can be formulated in terms
of general measurable spaces certain more advanced results require
topological assumptions that prevent the wildness of set theory from
taking over.  For the results of this nature in which we are
interested what is required is that the measure space look
sufficiently like the Borel algebra on the $\reals$.  Somewhat
surprisingly such a constraint isn't too severe (as we will show
later) and for the purposes of these notes (and following the lead
of Kallenberg) we will settle on the following definitions to capture
these restrictions.

\begin{defn}Two measure spaces $(S, \mathcal{S})$ and $(T,
  \mathcal{T})$ are said to be \emph{Borel isomorphic} if there exists
  a bijection $f : S \to T$ such that both $f$ and $f^{-1}$ are measurable.
\end{defn}

\begin{defn}A measurable space $(S, \mathcal{S})$ is said to be a
  \emph{Borel space} if it is Borel isomorphic to a Borel subset of $[0,1]$.
\end{defn}

The following lemma is extremely useful both conceptually and
practically.  In addition it's proof is a paradigmatic example of a
common measure theoretic argument and gives us a chance to show how
results may carry over from $\reals$ to general Borel spaces.
\begin{lem}\label{FunctionalRepresentation}Let $(S, \mathcal{S})$ be a
  Borel space and let $f : (\Omega, \mathcal{A}) \to S$ and $g :
  (\Omega, \mathcal{A}) \to (T,\mathcal{T})$ be measurable.  Then $f$
  is $g$-measurable if and only if there exists measurable $h :
  T  \to S$ such that $f = h \circ g$.
\end{lem}
\begin{proof}
For the if direction, assume $f = h \circ g$.  Then for $B \in
\mathcal{B}([0,1])$, we have $f^{-1}(B) = g^{-1}(h^{-1}(B))$.  Now we know
that $h^{-1}(B) \in \mathcal{T}$ and therefore, $f^{-1}(B) \in
\sigma(g)$.

For the only if direction, we first assume that $(S, \mathcal{S})
=([0,1], \mathcal{B}([0,1]))$.  
Assume $f$ is an indicator
function $\characteristic{A}$.  Our assumption of $g$-measurability
means that there exists $B \in \mathcal{T}$ such that $A =
g^{-1}(B)$.  If we define $h = \characteristic{B}$, then we have $f =
h \circ g$.  Now let us suppose that $f$ is a simple function and take
its canonical representation $f = c_1 \characteristic{A_1} + \cdots + c_n
\characteristic{A_n}$ with $A_i$ disjoint and $c_i$ distinct.  Since $f$ is
$g$-measurable, we know that there exist $B_i \in \mathcal{T}$ such
that $A_i = g^{-1}(B_i)$.  If we define $h=c_1 \characteristic{B_1} + \cdots + c_n
\characteristic{B_n}$, then $f = h \circ g$.

Now if we assume $f \geq 0$, then we know that we can find a sequence
of $g$-measurable simple functions such that $f_n \uparrow f$.  We
have shown that there are $h_n$ such that $f_n = h_n \circ g$.  Define
$h = \limsup_n h_n$ and then note $h$ is $g$-measurable and that 
\begin{align*}
h(g(\omega)) = \limsup_n h_n(g(\omega)) = \limsup_n f_n(\omega) =
\lim_{n \to \infty} f_n(\omega) = f(\omega)
\end{align*}
Lastly, for arbitrary $f$, we write $f = f_+ - f_{-}$ where $f_{\pm}
\geq 0$ and are both $g$-measurable (e.g. $f_{\pm} = (\pm f) \wedge 0$).
We find $h_{\pm}$ such that $f_{\pm} = h_{\pm} \circ g$ and define $h
= h_+ - h_{-}$.

Now let us assume that $S$ is a Borel subset of $[0,1]$ and note that
every measurable subset of $S$ is over the form $A \cap S$ for a Borel
subset $A \subset [0,1]$ and thus $f$ is also
$g$-measurable when considered as a function from $\Omega$ to $[0,1]$.  
By what we have just proven applied to
$f$, we get $\tilde{h} : T \to [0,1]$ such that $\tilde{h}
\circ g = f$.  Because of the latter identity, we know that
$\tilde{h}(g(\Omega)) \subset S$ however it is not necessarily the
case that $tilde{h}(T) \subset S$.  Since $S$ is a Borel subset of
$[0,1]$, we know that $\tilde{h}^{-1}(S)$ is $\mathcal{T}$-measurable
and therefore we can pick an arbitrary point $s_0 \in S$ and define
\begin{align*}
h(t) &= \begin{cases}
\tilde{h}(t) & \text{if $t \in \tilde{h}^{-1}(S)$} \\
s_0 & \text{otherwise}
\end{cases}
\end{align*}
and note that we now have $h : T \to S$ and $f = h \circ g$.

It remains to extend the argument to general Borel spaces $S$.  Assume
that $j : S \to A \subset [0,1]$ is a Borel isomorphism to a Borel
subset $A$.  We can define $\tilde{h}: T \to A$ such that $j \circ f =
h \circ g$ by the above argument.  Now let $h = j^{-1} \circ
\tilde{h}$ so we have $h : T \to S$ and $f = h \circ g$.
\end{proof}

The following definitions and lemma may seem merely technical, but in fact are
an important part of the most common methodology for proving measure
theoretic results.  
\begin{defn}A class $\mathcal{C}$ of subsets of a set $\Omega$ is called a $\lambda$-system if
\begin{itemize}
\item[(i)] $\Omega \in \mathcal{C} $.
\item[(ii)] for all $A,B \in \mathcal{C}$ if $A \subset B$, then
  $B \setminus A \in \mathcal{C}$.
\item[(iii)] for all $A_n \in
  \mathcal{C}$ if $A_1 \subset A_2 \subset \cdots$ and $A_n \uparrow A$, then $A \in \mathcal{C}$.
\end{itemize}
\end{defn}
\begin{defn}A class $\mathcal{C}$ of subsets of a set $\Omega$ is
  called a $\pi$-system if it is closed under finite intersections.
\end{defn}
The first observation is that the concepts of $\pi$-system and
$\lambda$-system factor the conditions for being a $\sigma$-algebra.
\begin{lem}\label{PiLambdaSigma}If a class $\mathcal{C} \subset 2^\Omega$ is both a
  $\pi$-system and a $\lambda$-system, then it is a $\sigma$-algebra.
\end{lem}
\begin{proof}
First we show closure under set complement.  Let $A \in \mathcal{C}$.
Then since $\Omega \in \mathcal{C}$, we know that $A^c = \Omega
\setminus A \in \mathcal{C}$.  Now note that having closure under set complement
together with closure under finite intersection gives closure under
finite union by De Morgan's law $\left \{ \bigcup_{i=1}^n A_i \right \} ^
c =  \bigcap_{i=1}^n A_i^c $.

Let $A_1, A_2, \dots \in \mathcal{C}$.  Next we show closure under
countable union.  Defining $B_n = \bigcup_{i=1}^n A_i$, we know that
$B_n \in \mathcal{C}$ and clearly $B_n \uparrow \bigcup_{i=1}^\infty
A_i$ and therefore $\bigcup_{i=1}^\infty A_i \in \mathcal{C}$.
Closure under countable intersections follows from closure under
countable unions and the infinite version of De Morgan's Law.
\end{proof}
\begin{thm}[$\pi$-$\lambda$ Theorem]\label{MonotoneClassTheorem}Suppose $\mathcal{C}$ is a $\pi$-system, $\mathcal{D}$ is a
  $\lambda$-system such that $\mathcal{C} \subset \mathcal{D}$.  Then
  $\sigma(\mathcal{C}) \subset \mathcal{D}$.
\end{thm}
\begin{proof}The first thing to note is that the intersection of a
  collection of $\lambda$-systems is also a $\lambda$-system and that
   $2^\Omega$ is a $\lambda$-system.
  Therefore, in a way entirely analogous to $\sigma$-algebras we may
  define the $\lambda$-system generated by a collection of sets as the
  intersection of all $\lambda$-systems containing the collection.

The theorem is proved for general $\mathcal{D}$ if we prove it for the
special case $\mathcal{D} = \lambda(\mathcal{C})$.  To see this
special case, by \ref{PiLambdaSigma} it suffices to show that
$\lambda(\mathcal{C})$ is a $\pi$-system.  A trivial induction
argument shows it suffices to show closure under pairwise
intersection:  for every $A,B \in \lambda(\mathcal{C})$ we have $A
\cap B \in \lambda(\mathcal{C})$.

By definition of $\pi$-algebra, we have closure when $A,B
\in \mathcal{C}$.  Now fix $C \in \mathcal{C}$ and let $\mathcal{A}_C
= \left \{A \subset \Omega ; A \cap C \in \lambda(\mathcal{C}) \right
\}$.  We claim that $\mathcal{A}_C$ is a $\lambda$-system.

To see that $\Omega \in \mathcal{A}_C$ is trivial: $C \cap \Omega
= C \in \mathcal{C} \subset \lambda(\mathcal{C})$.
Suppose $A \supset B$ where $A,B \in \mathcal{A}_C$, then $C
\cap (A \setminus B) = (C \cap A) \setminus (C \cap B) \in
\lambda(\mathcal{C})$.  Suppose $A_1 \subset A_2 \subset \cdots$ with
$A_i \in \mathcal{A}_C$.  $C \cap \bigcup_{i=1}^\infty A_i =
\bigcup_{i=1}^\infty C \cap A_i \in \lambda(\mathcal{C})$ by
distributivity of set intersection over set union and closure of $\lambda$-system under increasing unions.

Now that we know $\mathcal{A}_C$ is a $\lambda$-system containing
$\mathcal{C}$ we know that $\lambda(\mathcal{C}) \subset
\mathcal{A}_C$ and therefore $C \cap A \in \lambda(\mathcal{C})$ for
every $A \in \lambda(\mathcal{C})$ and $C \in \mathcal{C}$.

To finish up the proof, for every $C \in \lambda(\mathcal{C})$, let
$\mathcal{B}_C = \left \{ A \in \Omega ; A \cap C \in
  \lambda(\mathcal{C}) \right \}$.  We have just shown that
$\mathcal{C} \subset \mathcal{B}_C$ and an argument exactly analogous
to the one above shows that $\mathcal{B}_C$ is a $\lambda$-algebra and
therefore $\lambda(\mathcal{C}) \subset \mathcal{B}_C$ proving the result.
\end{proof}

Though we'll see many examples of this along the way, it is worth
making explicit how the Theorem \ref{MonotoneClassTheorem} is applied.
Suppose that one wishes to prove a property holds for a
$\sigma$-algebra $\mathcal{A}$ of sets.  A common sub-case is we'll be trying to show a
property holds for the indicator functions associated with those sets
(those being the most basic building blocks of measurable functions).
The $\pi$-$\lambda$ Theorem allows us to prove the property holds on
$\mathcal{A}$ by showing 
\begin{itemize}
\item[(i)] The collection of all sets satisfying the property is a $\lambda$-system
\item[(ii)] There is a $\pi$-system of sets $\mathcal{P}$ that
  satisfies the property and $\sigma(\mathcal{P}) = \mathcal{A}$.
\end{itemize}
A proof along these lines is referred to as a \emph{monotone class
  argument}.

\section{Measures and Integration}
Armed with a way of describing and transforming measurable sets it is
finally time to measure them.
\begin{defn}A \emph{measure} on a measurable space $(\Omega,
  \mathcal{A})$ is a function $\mu : \mathcal{A} \to
  \overline{\reals}_+$ satisfying
\begin{itemize}
\item[(i)] $\mu(\emptyset) = 0$
\item[(ii)] $\mu(\bigcup_{i=1}^\infty A_i) = \sum_{i=1}^\infty \mu(A_i)$
  for $A_1, A_2, \dots \in \mathcal{A}$ disjoint.
\end{itemize}
A triple $(\Omega, \mathcal{A}, \mu)$ is called a \emph{measure space}.
\end{defn}

An important special case of measure theory occurs when the underlying
space has unit measure.  Many of the concepts we have already
discussed have different names when discussing this special case.
\begin{defn}A \emph{probability space} is a measure space $(\Omega,
  \mathcal{A}, P)$ such that $P(\Omega) = 1$.  The measure $P$ is
  called the \emph{probability measure}.  Measurable sets $A \in
  \mathcal{A}$ are referred to as \emph{events}.  Given a measurable
  space $(S, \mathcal{S})$, a measurable function $\xi : \Omega \to S$
  is called a \emph{random element} in $S$.  For the special case in
  which $(S, \mathcal{S}) = (\reals, \mathcal{B}(\reals)$, we call a
  measurable $\xi : \Omega \to \reals$ a \emph{random variable}.
\end{defn}

\begin{lem}\label{ContinuityOfMeasure}Given a measure space $(\Omega, \mathcal{A}, \mu)$, and sets $A_1, A_2, \dots \in \mathcal{A}$.
\begin{itemize}
\item[(i)] If $A_i \uparrow A$ then $\mu A_i \uparrow \mu A$.
\item[(ii)] If $A_i \downarrow A$ and $\mu A_1 < \infty$ then $\mu A_i \downarrow \mu A$.
\end{itemize}
\end{lem}
\begin{proof}To show (i), define $B_1 = A_1$ and $B_i = A_i \setminus
  A_{i-1}$ for $i > 1$.  Clearly, $B_i$ are disjoint and it is equally
  clear that $\bigcup_{i=1}^n B_i = A_n$ and $\bigcup_{n=1} ^\infty
  B_n = A$.
Therefore
\begin{align*}
\mu A_n = \mu \bigcup_{i=1}^n B_i = \sum_{i=1}^n \mu B_i \uparrow
\sum_{i=1}^\infty \mu B_i = \mu \bigcup_{i=1}^\infty  B_i = \mu A
\end{align*}
where we have used finite and countable additivity of $\mu$ over the
$B_i$.

To see (ii), note that $A_1 \setminus A_n \uparrow A_1
\setminus A$ and then under the finiteness assumption $\mu A_1 <
\infty$, we see 
\begin{align*}
\mu (A_1 \setminus A_n) = \mu A_1 - \mu A_n \uparrow
\mu (A_1 \setminus A) = \mu A_1 - \mu A
\end{align*}
Subtract $\mu A_1$ from both sides multiply by $-1$ to get the result.
\end{proof}

\begin{lem}Given a measure space $(\Omega, \mathcal{A}, \mu)$, $\mu
  \left (\bigcup_{i=1}^\infty A_i \right ) \leq \sum_{i=1}^\infty \mu(A_i)$  for $A_1, A_2, \dots \in \mathcal{A}$.
\end{lem}
\begin{proof}First we prove finite subadditivity by an induction argument.  For $n=2$, we note that we may
  write disjoint unions
\begin{align*}
A &= (A \setminus B) \cup (A \cap B) \\
B  &= (B\setminus A) \cup (A \cap B) \\
A \cup B  &= (A\setminus B)  \cup (B\setminus A) \cup (A \cap B) \\
\end{align*}
By finite additivity of measure and positivity of measure, we see $\mu A \cup B = \mu A + \mu B
- \mu A \cap B \leq \mu A + \mu B$. 

For the induction step, assume $\mu \left (\bigcup_{i=1}^{n-1} A_i \right
) \leq \sum_{i=1}^{n-1} \mu(A_i)$, then use the case $n=2$ and
Lemma \ref{ContinuityOfMeasure} to see
\begin{align*}
\mu \left (\bigcup_{i=1}^{n} A_i \right) &= \mu \left
  (\bigcup_{i=1}^{n-1} A_i  \cup A_n\right) \\
&\leq \mu \left (\bigcup_{i=1}^{n-1} A_i \right ) + \mu A_n \\
&\leq \sum_{i=1}^{n-1} \mu(A_i) + \mu A_n = \sum_{i=1}^n \mu(A_i)
\end{align*}

To extend the result to infinite unions, define $B_n = \bigcup_{i=1}^n
A_i$ and note that $B_n \uparrow \bigcup_{i=1}^\infty
A_i$ and that by finite subadditivity, $\mu B_n \leq \sum_{i=1}^n \mu
A_i$.  Taking limits we see
\begin{align*}
\mu \bigcup_{i=1}^\infty A_i = \lim_{n \to \infty} \mu B_n \leq \lim_{n
  \to \infty} \sum_{i=1}^n \mu A_i = \sum_{i=1}^\infty \mu A_i
\end{align*}
\end{proof}

Next up is the definition of integral of a measurable function on a
measure space.  First we proceed by defining the integral for a simple functions.
\begin{defn}
Given a canonical representation of a simple function $f =
c_1 \characteristic{A_1} + \cdots + c_n \characteristic{A_n}$ we
define the integral of $f$ to be
\begin{align*}
\int f d\mu = \mu f = c_1 \mu A_1 + \cdots + c_n \mu A_n
\end{align*}
\end{defn}
Having the definition of the integral of a simple function in terms of
the canonical representation is inconvenient at times when one is
given a simple function that is not known to be in a canonical
representation.  It turns out that the formula above extends to any
representation of the simple function as a linear combination of
indicator functions.  To see that we proceed in steps.
\begin{lem}Given any representation of a simple function $f =
c_1 \characteristic{A_1} + \cdots + c_n \characteristic{A_n}$ with
$A_i$ pairwise disjoint,
\begin{align*}
\int f d\mu = c_1 \mu A_1 + \cdots + c_n \mu A_n
\end{align*}
\end{lem}
\begin{proof}We have to construct the canonical representation of
  $f$.  It is conceptually simple, but there is a bit of notation to
  deal with.
  Let $d_1, d_2, \dots, d_m$ be the distinct values of $c_1, \dots,
  c_n$.  Furthermore, for each $i=1,\dots,m$, let $B_{i,j}$ 
  $j=1,\dots,k_i$ be the set of $A_n$ for which $c_n = d_i$.  Then the
  canonical representation of $f$ is 
\begin{align*}
f = d_1 \characteristic{\bigcup_{j=1}^{k_1} B_{1,j}} + \cdots + d_m \characteristic{\bigcup_{j=1}^{k_m} B_{m,j}}
\end{align*}
and then 
\begin{align*}
\int f d \mu &= d_1 \mu \bigcup_{j=1}^{k_1} B_{1,j} + \cdots + d_m \mu
\bigcup_{j=1}^{k_m} B_{m,j} \\
&= d_1 \sum_{j=1}^{k_1} \mu B_{1,j} + \cdots + d_m 
\sum_{j=1}^{k_m} \mu B_{m,j} \\
&= c_1 \mu A_1 + \cdots + c_n \mu A_n
\end{align*}
\end{proof}
\begin{lem}\label{LinearityIntegralSimpleFunctions}Given two simple functions $f,g$, for all $a,b \in \reals$,
 \begin{align*}
\int \left (af + bg \right ) d \mu = a \int f d \mu + b \int g d \mu
\end{align*}
If $f \geq g$ $a.e.$ then we have
 \begin{align*}
\int f d \mu \geq \int g d \mu
\end{align*}
\end{lem}
\begin{proof}Take the canonical representation of both $f$ and $g$, $f
  = \sum_{i=1}^n c_i \characteristic{A_i}$ and $f
  = \sum_{i=1}^m d_i \characteristic{B_i}$.   Furthermore define $A_0 =
  \Omega \setminus \bigcup_{i=1}^n A_i$ and $B_0 =
  \Omega \setminus \bigcup_{i=1}^m B_i$.  Now consider all of the
  pairs $A_i \cap B_j$ and write 
\begin{align*}
f &= \sum_{i=0}^n \sum_{j=0}^m c_i
  \characteristic{A_i \cap B_j} \\
g &= \sum_{i=0}^n \sum_{j=0}^m d_j
  \characteristic{A_i \cap B_j}
\end{align*}
 where we have defined $c_0=d_0=0$.  Thus, we have the  representation 
\begin{align*}
af + bg = \sum_{i=0}^n \sum_{j=0}^m (ac_i + bd_j)   \characteristic{A_i \cap
  B_j}
\end{align*}
Since the $A_i  \cap B_j$ are pairwise disjoint, we can write
\begin{align*}
\int af+bg &= \int \sum_{i=0}^n \sum_{j=0}^m (ac_i + bd_j)   \characteristic{A_i \cap B_j} \\
&= \sum_{i=0}^n \sum_{j=0}^m (ac_i + bd_j)   \mu A_i \cap B_j \\
&= a \sum_{i=0}^n \sum_{j=0}^m c_i  \mu A_i \cap B_j + b \sum_{i=0}^n
\sum_{j=0}^m d_j  \mu A_i \cap B_j \\
&= a \int f + b \int g
\end{align*}

Using the same representation as above, we see that if $f \geq g$,
then since the $A_i \cap B_j$ are disjoint, we must have $c_i \geq
d_j$ whenever $A_i \cap B_j \neq \emptyset$.  This shows $\int f \geq
\int g$.
\end{proof}
\begin{cor}Given any representation of a simple function $f =
c_1 \characteristic{A_1} + \cdots + c_n \characteristic{A_n}$,
\begin{align*}
\int f  = c_1 \mu A_1 + \cdots + c_n \mu A_n
\end{align*} 
\end{cor}
The corollary above is used so often that we use it without mentioning
it and essentially treat it as the definition of the integral of a
simple function.

Having defined integrals of simple functions, we leverage the fact
that we can approximate positive measurable functions by increasing
sequences of simple functions to define the integral of a positive
measurable function.
\begin{defn}Given a measurable function $f : (\Omega, \mathcal{A},
  \mu) \to \overline{\reals}_+$, we define 
\begin{align*}\int f = \sup_{0 \leq g \leq f} \int g
\end{align*}
where the supremum is taken over positive simple functions $g$.
\end{defn}
Working with the supremum above is a bit inconvenient and it turns out
that it suffices to work with increasing sequences of positive simple
functions.  To see that we first need a lemma.
\begin{lem}Given a measurable function $f : (\Omega,
  \mathcal{A}, \mu) \to \overline{\reals}_+$, a sequence $0 \leq f_1, f_2, \dots$ of simple measurable
  functions such that $f_n \uparrow f$ and a simple measurable
  function $g$ such that $0 \leq g \leq f$, we have $\lim_{n \to \infty} \int f_n
  \, d\mu
  \geq \int g \, d\mu$.
\end{lem}
\begin{proof}Consider the case where $g = \characteristic{A}$ for $A
  \in \mathcal{A}$.  Pick $\epsilon > 0$, and define
\begin{align*}
A_n = \left \{ \omega \in A; f_n(\omega) \geq 1 - \epsilon \right \}
\end{align*}
Since $f_n$ is increasing, so is $A_n$.  Also it is simple to see that
$A_n \subset A$ since $f \geq f_n$ and $A \subset \bigcup_n A_n$ since
for each $\omega \in A$ convergence of $f_n(\omega) \uparrow
f(\omega)$ tells us that there is $N > 0$ such that for $n > N$, we
have $\abs{f_n(\omega) - f(\omega)} < \epsilon$, hence $A_n \uparrow
A$ and 
$\mu A_n \uparrow \mu A = \int g d \mu$.

Now the definition of $A_n$, the positivity of $f_n$ and the
positivity of integration tells us that
$\int f_n d \mu \geq (1 - \epsilon) \mu A_n$, so taking limits we see
\begin{align*}
\lim_{n \to \infty} \int f_n d \mu \geq (1 - \epsilon) \lim_{n \to
  \infty} \mu A_n = (1-\epsilon) \int g d \mu
\end{align*}
Now let $\epsilon \to 0$ to get the result.

To extend the result to arbitrary positive simple functions,  first consider $g
= c \characteristic{A}$ for $c > 0$.  Note that we can apply the
lemma to $\characteristic{A}$ and the functions $\frac{1}{c}f_n
\uparrow \frac{1}{c}f$, to see that $\lim_{n \to \infty}
\frac{1}{c}f_n \geq \mu A$ and multiply both sides by $c$.

Now consider a positive simple function in canonical form $g = c_1
\characteristic{A_1} + \cdots + c_m \characteristic{A_m}$.  Since $g$
is in the canonical form, $c_i > 0$ for $i=1, \dots, m$.  Also, $A_i \cap A_j = \emptyset$ for $i \neq j$
and therefore $g \characteristic{A_i}  = c_i \characteristic{A_i}$.
Now apply the lemma to each $g \characteristic{A_i} $ and the family
$f_n \characteristic{A_i}  \uparrow f \characteristic{A_i} $ and use
linearity of integral and limits.
\end{proof}
\begin{cor}Given a measurable positive function $f : (\Omega, \mathcal{A},
  \mu) \to \overline{\reals}_+$ and any sequence of positive simple functions $0 \leq f_1,
  f_2, \dots$ such that $f_n \uparrow f$, 
\begin{align*}
\int f d \mu = \lim_{n \to \infty} \int f_n d\mu
\end{align*}
\end{cor}
\begin{proof}As $f_n$ are positive simple functions with $f_n \leq f$
  we know each $\int f_n \leq \int f$ and therefore $\lim_{n \to \infty} \int f_n
  d\mu \leq \int f d \mu$.  

To see the other inequality, pick $\epsilon > 0$, and a positive
simple $0 \leq g \leq f$ such that $\int f d \mu - \epsilon \leq \int
g d \mu$.  Apply the above lemma and we see that $\int f d \mu - \epsilon \leq \int
g d \mu \leq \lim_{n \to \infty} \int f_n d \mu$.  Now let $\epsilon \to 0$ to see $\int f d \mu
\leq \int \lim_{n \to \infty} f_n d \mu$.
\end{proof}

\begin{lem}\label{LinearityAndMonotonictyPositiveIntegral}Given $f,g$ positive measurable and $a,b \geq 0$, 
\begin{align*}
\int \left ( a f + b g \right ) d \mu = a \int f d \mu + b \int g d \mu
\end{align*}
and if $f \geq g$,
\begin{align*}
\int f  d \mu \geq \int g d \mu
\end{align*}
\end{lem}
\begin{proof}Linearity follows by taking $0 \leq f_n \uparrow f$ and
  $0 \leq g_n \uparrow g$ and noting that $0 \leq a f_n + b g_n
  \uparrow a f + b g$.  Now apply linearity of integral of simple
  functions Lemma \ref{LinearityIntegralSimpleFunctions}.

Monotonicity follows immediately from noting that any simple $0 \leq h
\leq g$ also satisfies $0 \leq h \leq f$.
\end{proof}

It is important to note that two functions equal except on a set of measure zero have the same integral.
\begin{prop}\label{AlmostEverywhereEqualIntegralEqual}Given $f,g$ positive measurable functions such that $\mu \lbrace f \neq g \rbrace = 0$ then
\begin{align*}
\int f \, d\mu &= \int g \, d\mu
\end{align*}
\end{prop}
\begin{proof}
Let $\epsilon > 0$ and pick a positive simple function $h = c_1 \characteristic{A_1} + \dotsb + c_n \characteristic{A_n}$ 
such that $h \leq f$ and $\int f \, d\mu \leq \int h \, d\mu + \epsilon = c_1 \mu(A_1) + \dotsb + c_n \mu(A_n) + \epsilon$.  Define $B_j = A_j \setminus \lbrace f \neq g \rbrace$ for $j = 1, \dotsc, n$ and $\tilde{h} = c_1 \characteristic{B_1} + \dotsb + c_n \characteristic{B_n}$.  Because $f=g$ on each $B_j$, we have $0 \leq \tilde{h} \leq g$.  By subadditivity $\mu(B_j) \leq \mu(A_j)$ but also 
\begin{align*}
\mu(A_j) &= \mu(A_j \cap \lbrace f \neq g \rbrace)  + \mu(A_j \cap \lbrace f \neq g \rbrace^c) \leq  \mu\lbrace f \neq g \rbrace + \mu(B_j) = \mu(B_j)
\end{align*}
so in fact $\mu(A_j) = \mu(B_j)$ for all $j=1, \dotsc, n$ and $\int h \, d\mu = \int \tilde{h} \, d\mu$.   We conclude that 
\begin{align*}
\int f \, d\mu &\leq \int h \, d\mu + \epsilon = \int \tilde{h} \, d\mu + \epsilon \leq \int g \, d\mu + \epsilon
\end{align*}
Since $\epsilon>0$ was arbitrary we conclude that $\int f \, d\mu \leq \int g \, d\mu$.  Repeating the above argument with the roles of $f$ and $g$ reversed we get 
$\int g \, d\mu \leq \int f \, d\mu$ as well.
\end{proof}

Perhaps the most important basic theorems of measure theory are those
that describe how limits and integrals behave; in particular what
happens we exchange the order of limits and integrals.  There are three
commonly used variants and we are now ready to state and prove the
first.  Before we do that we illustrate three simple examples of the
things that can go wrong when we exchange the order of limits and
integrals.  All of these examples assume the existence of a measure
$\lambda$ on
$(\reals,\mathcal{B}(\reals))$ such that $\lambda([a,b]) = b -a$.  We
will prove later that such a measure exists (it is the \emph{Lebesgue
  measure} on $\reals$).
\begin{examp}[Escape to horizontal infinity]Consider the sequence of functions 
$f_n = \characteristic{[n,n+1]}$.  Note that $\lim_{n \to \infty}
\int f_n \, d \lambda = 1$ but $\int \lim_{n \to \infty}  f_n \, d
\lambda = 0$.
\end{examp}
\begin{examp}[Escape to vertical infinity]Consider the sequence of functions 
$f_n = n \characteristic{[0,\frac{1}{n}]}$.  Note that $\lim_{n \to \infty}
\int f_n \, d \lambda = 1$ but $\int \lim_{n \to \infty}  f_n \, d
\lambda = 0$.
\end{examp}
\begin{examp}[Escape to width infinity]Consider the sequence of functions 
$f_n = \frac{1}{n}\characteristic{[0,n-1]}$.  Note that $\lim_{n \to \infty}
\int f_n \, d \lambda = 1$ but $\int \lim_{n \to \infty}  f_n \, d
\lambda = 0$.
\end{examp}
 In all cases the integral of the limit is strictly less than the limit
of the integrals and in all cases some amount of \emph{mass} has
\emph{escaped to infinty}.  The limit theorems amount to proving the
fact that mass can only be lost when passing to the limit of a
sequence of measurable functions and establishing generally useful
hypotheses that prevent mass from escaping to infinity.

\begin{thm}[Monotone Convergence Theorem]\label{MCT}Given $f, f_1,
  f_2, \dots$ positive measurable functions from
  $(\Omega, \mathcal{A}, \mu)$ to $\overline{\reals}_+$ such that $0 \leq f_n \uparrow f$, we
  have $\int f_n d \mu \uparrow \int f d \mu$.
\end{thm}
\begin{proof}Choose an approximation of each $f_n$ by an increasing
  sequence of positive simple functions $g_{nk} \uparrow f_n$.  For
  each $n,k>0$, define $h_{nk} = g_{1k} \vee \cdots \vee g_{nk}$.
  Note that $h_{nk}$ is increasing in both of its subscripts.
  Furthermore, note that $h_{nk} \leq f_n$ because $g_{ik} \leq f_{i}
  \leq f_n$ for $i \leq n$ by the monotonicity of $f_n$.

We  claim that $h_{kk} \uparrow f$.  To see this, for every $n>0$,
  $h_{kk} \geq g_{nk}$ for $k \geq n$ and therefore
\begin{align*}
\lim_{k \to \infty} h_{kk} \geq \lim_{k \to \infty} g_{nk} = f_n
\end{align*}
By taking limits we get the inequality
\begin{align*}
\lim_{k \to \infty} h_{kk} \geq \lim_{n \to \infty} f_n = f
\end{align*}
We get the opposite inequality because $f_n$ increases to $f$, we know that for
every $k>0$, $h_{kk} \leq f_k \leq f$ and therefore $\lim_{k \to
  \infty} h_{kk} \leq f$.

We have an approximation of $0 \leq h_{kk} \uparrow f$ by simple
functions, now we can calculate the integral of $f$ using $h_{kk}$
\begin{align*}
\int f \, d\mu = \lim_{k \to \infty} \int h_{kk} \, d\mu \leq \lim_{k
  \to \infty} \int f_k \, d\mu \leq \int f \, d\mu
\end{align*}
where we have used the monotonicity of the integral in both
inequalities.  
\end{proof}
\begin{cor}[Tonelli's Theorem for Integrals and Sums]\label{TonelliIntegralSum}Given $f_1,
  f_2, \dots$ positive measurable functions from
  $(\Omega, \mathcal{A}, \mu)$ to $\overline{\reals}_+$, we
  have 
\begin{align*}
\int \sum_{n=1}^\infty f_n d \mu = \sum_{n=1}^\infty \int f_n \, d \mu
\end{align*}
\end{cor}
\begin{proof}Note that the sequence partial sums $\sum_{i=1}^n f_i$ is
  increasing in $n>0$.  Now use linearity of integral and apply the Montone Convergence Theorem.
\end{proof}

 In some cases, we may have a sequence of positive functions that are not known
to be increasing.  In those cases, limits may not even exists but we
still have a fundamental inequality
\begin{thm}[Fatou's Lemma]\label{Fatou}Given $f_1, f_2, \dots$
  positive measurable functions from
  $(\Omega, \mathcal{A}, \mu)$ to $\overline{\reals}_+$,
  then $\int \liminf_{n \to \infty} f_n d \mu \leq \liminf_{n \to
    \infty} \int f_n d \mu$.
\end{thm}
\begin{proof}The proof uses the Monotone Convergence Theorem.  To find
  an increasing sequence of positive measurable functions one needn't
  look further than the definition $\liminf_{n \to \infty} f_n =
  \lim_{n \to \infty} \inf_{k \geq n} f_k$.  Since $\inf_{k \geq n}
  f_k \uparrow \liminf_{n \to \infty} f_n$, we know by Monotone
  Convergence that $\lim_{n \to \infty} \int \inf_{k \geq n}
  f_k \, d \mu = \int \liminf_{n \to \infty} f_n \, d \mu$.  

However, we have the following calculation
\begin{align*}
\inf_{k \geq n} f_k &\leq f_k  & &\textrm{for all $k\geq n$ by
  definition of infimum} \\
\int \inf_{k \geq n} f_k \, d\mu &\leq \int f_k \,
d \mu & &\textrm{for all $k\geq n$ by
  monotonicity of integral} \\
\int \inf_{k \geq n} f_k \, d\mu &\leq \inf_{k \geq n} \int f_k \,
d \mu & &\textrm{by
  definition of infimum} \\
\lim_{n \to \infty} \int
\inf_{k \geq n} f_k \, d\mu &\leq \liminf_{n \to \infty} \int f_k \,
d \mu & &\textrm{taking limits and the
  definition of $\liminf$} \\
\int \liminf_{n \to \infty} f_n d \mu &= & & \textrm{by Monotone Convergence}
\end{align*}

In prose, by the  definition of the infimum $\inf_{k \geq n} f_k \leq f_k$ for every
  $k \geq n$, therefore monotonicity of the integral yields $\int
  \inf_{k \geq n} f_k d \mu \leq \int f_k d \mu$ for every
  $k \geq n$ and hence $\int
  \inf_{k \geq n} f_k d \mu \leq \inf_{k\geq n} \int f_k d \mu$.  Now
  take the limit as $n \to \infty$.
\end{proof}

Our last task is to eliminate the assumption of positivity in the
definition of the integral.  
\begin{defn}Let $(\Omega,
  \mathcal{A}, \mu)$ be a measure space and let $f : \Omega \to [-\infty, \infty]$ be a measurable function.  We say that the integral
$\int f \, d\mu$ is \emph{defined} if at least one of $\int f_\pm \, d\mu < \infty$; in that case we define  $\int f \, d \mu = \int
  f_+ \, d\mu - \int f_{-} \, d \mu$.  We say that $f$ is \emph{integrable} if $\int \abs{f} \, d\mu <
  \infty$ (equivalently if $\int f \, d\mu$ is defined and $\int f \, d\mu < \infty$).  
\end{defn}

\begin{prop}\label{IntegrableAlmostEverywhereEqualIntegralEqual}Given $f,g : \Omega \to [-\infty, \infty]$ be measurable functions such that $\mu \lbrace f \neq g \rbrace = 0$ then $\int f \, d\mu$ is defined if and only if $\int g \, d\mu$ is defined and in that case
\begin{align*}
\int f \, d\mu &= \int g \, d\mu
\end{align*}
\end{prop}
\begin{proof}
Clearly $\lbrace f_\pm \neq g_\pm \rbrace \subset \lbrace f \neq g \rbrace$.  By subadditivity $\mu \lbrace f_\pm \neq g_\pm \rbrace = 0$ and therefore by Proposition \ref{AlmostEverywhereEqualIntegralEqual} we know that $\int f_\pm \, d\mu = \int g_\pm \, d\mu$.
\end{proof}
 
Note that an integrable function is allowed to take the values $-\infty$ and $\infty$,
however the condition $\int \abs{f} \, d\mu<\infty$ ensures that $\abs{f} < \infty$ $\mu$-almost everywhere.
By virtue of Proposition \ref{IntegrableAlmostEverywhereEqualIntegralEqual} the integral of two functions that are equal almost everywhere is the same and it often
convenient to simply assume that an integrable function is finite everywhere.

We've defined the integral of an integrable function in terms of a
canonical decomposition $f = f_+ - f_-$.  It is occasionally useful to
observe that any decomposition of an integrable function as a
difference of positive measurable functions can be used to calculate
the integral.
\begin{lem}Suppose we are given a measure space $(\Omega, \mathcal{A},
  \mu)$ and an integrable function $f : \Omega \to [-\infty, \infty]$.  Suppose
  $f = f_1 -f_2$ where $f_i  : \Omega \to \reals$ are positive measurable
  with $\int f_i \, d \mu < \infty$. Then $\int f \, d\mu = \int f_1 \,
  d\mu - \int f_2 \, d\mu$.
\end{lem}
\begin{proof}Write $f = f_+ - f_-$ and note that $f_1 \geq f_+$ and
  $f_2 \geq f_-$.  For example either $f_+(\omega) = 0$ or
  $f_+(\omega) = f(\omega)$ and we know that $f_1(\omega) = f(\omega)
  + f_2(\omega) \geq f(\omega)$.  We also know that $f_1 - f_+ = f_2 -
  f_-$ and we can see that $\int (f_1 - f_+) \, d\mu = \int (f_2 - f_-) \, d\mu 
  < \infty$.  Therefore by
  linearity of integral
\begin{align*}
\int f \, d\mu &= \int f_+ \, d\mu - \int  f_- \, d\mu \\
&= \int f_+ \, d\mu + \int (f_1 - f_+) \, d\mu - \int (f_2 - f_-) \,
d\mu - \int  f_- \, d\mu \\
&= \int f_1 \, d\mu - \int  f_2 \, d\mu
\end{align*}
\end{proof}

Also linearity and monotonicity of integrals extend to the integrable
case.  Linearity of the integral subsumes the previous result.
\begin{lem}\label{SignedIntegralLinearity}Suppose we are given a measure space $(\Omega, \mathcal{A},
  \mu)$ and integrable functions $f,g: \Omega \to \reals$.  Then for
  $a,b \in \reals$ we have $\int (af + bg) \, d\mu = a\int f \, d\mu +
  b\int g \, d\mu$.
\end{lem}
\begin{proof}
Write $f = f_+ - f_-$ and $g = g_+ - g_-$.  Define 
\begin{align*}
\hat{f}_\pm &= \begin{cases}
a f_\pm & \text{if $a  \geq 0$} \\
-a f_\mp & \text{if $a < 0$}
\end{cases}
\end{align*} 
It is easy to see that $\hat{f}_\pm \geq 0$,
$\int \hat{f}_\pm \, d\mu < \infty$,  $a f = \hat{f}_+ - \hat{f}_-$
and
\begin{align*}
\int af \, d\mu &= \int \hat{f}_+ \, d\mu - \int \hat{f}_- \, d\mu\\
&= \begin{cases}
\int a f_+ \, d\mu - \int af_- \, d\mu & \text{if $a \geq 0$} \\
\int -a f_- \, d\mu - \int -af_+ \, d\mu & \text{if $a < 0$} \\
\end{cases}\\
&= a \int f_+ \, d\mu - a \int f_- \, d\mu = a \int f\, d\mu
\end{align*}  
The same construction and
observations are true with $g$ and $\hat{g}_\pm$.
Then $a f + b g = (\hat{f}_+ + \hat{g}_+) - (\hat{f}_- + \hat{g}_-)$
and we have
\begin{align*}
\int (a f + b g) \, d\mu &= \int (\hat{f}_+ + \hat{g}_+) \, d\mu -
\int (\hat{f}_- + \hat{g}_-) \, d\mu \\
&= \int \hat{f}_+ \, d\mu - \int \hat{f}_- \, d\mu + \int \hat{g}_+ \, d\mu -
\int \hat{g}_- \, d\mu \\
&= a \int f \, d\mu + b \int g \, d\mu
\end{align*}
\end{proof}

TODO: It is also true that if $f$ is integrable and $g$ non-negative measurable then $\int (f+g) \, d\mu = \int f \, d\mu + \int g \, d\mu$.  Do we need that anywhere?

\begin{prop}\label{SignedIntegralMonotonicity}Let $(\Omega, \mathcal{A}, \mu)$ be a measure space and let $f,g: \Omega \to [-\infty, \infty]$ be measurable with
$f \leq g$ almost everywhere then one of the following is true
\begin{itemize}
\item[(i)]Both $\int f \, d\mu$ and $\int g \, d\mu$ are defined and $\int f \, d\mu \leq \int g \, d\mu$
\item[(ii)]$\int f \, d\mu$ is undefined, $\int g \, d\mu$ is defined and $\int g \, d\mu = \infty$
\item[(iii)]$\int g \, d\mu$ is undefined, $\int f \, d\mu$ is defined and $\int f \, d\mu = -\infty$
\item[(iv)]Both $\int f \, d\mu$ and $\int g \, d\mu$ are undefined
\end{itemize}
\end{prop}
\begin{proof}
First by Proposition \ref{IntegrableAlmostEverywhereEqualIntegralEqual}, the question of whether an integral is defined and it is value isn't changed if the integrand is modified on a null set.  Thus by redefining $g$ to be equal to $f$ on $\lbrace g < f \rbrace$ we may assume that $f \leq g$ everywhere.
For $f \geq 0$ then by Lemma \ref{LinearityAndMonotonictyPositiveIntegral} we know that the only case is (i).  For the general case we know that $f_+ \leq g_+$ and $f_- \geq g_-$.  If $\int f$ is undefined then $\int f_+ = \int f_- = \infty$ and it follows from the positive case that $\int g_+ = \infty$ hence either $\int g$ is undefined or $\int g = \infty$.  Similarly if $\int g$ is undefined it follows by monotonicity in the positive case that $\int f_- = \infty$ hence either $\int f$ is undefined or $\int f = -\infty$.  If both $\int f$ and $\int g$ are defined then from the non-negative case we have $0 \leq \int f_+  \leq \int g_+ < \infty$ and $0 \leq \int g_- \leq \int f_- < \infty$ so no infinities are involved and we compute
\begin{align*}
\int f  &= \int f_+ - \int f_- \leq \int g_+ - \int g_- = \int g
\end{align*}
\end{proof}

We have the following limit theorem for limits of integrable
functions.

\begin{thm}[Monotone Convergence Theorem for Signed Integrals]\label{SignedMCT}Given $f, f_1,
  f_2, \dots$ measurable functions from
  $(\Omega, \mathcal{A}, \mu)$ to $[-\infty, \infty]$ such that $f_n \uparrow f$, $\int f_1 \, d\mu$ exists and $\int f_1 \, d\mu > -\infty$.  Then $f$ is measurable, $\int f \, d\mu$ and $\int f_n \, d\mu$ are defined for all $n \in \naturals$
and we  have $\int f_n d \mu \uparrow \int f d \mu$.
\end{thm}
\begin{proof}
Note measurability of $f$ follows from Lemma \ref{LimitsOfMeasurable} and Proposition \ref{SignedIntegralMonotonicity} together with the hypothesis that $\int f_1 \, d\mu > -\infty$ implies that all the other integrals are defined and not $-\infty$.  

First assume that $f \leq 0$; then for all $n \in \naturals$, $f_n \leq f \leq 0$ hence $-\infty < \int f_n \, d\mu \leq 0 < \infty$ (i.e. $f, f_1, f_2, \dotsc$ are all integrable).  This implies that almost everywhere $-\infty < f_n \leq 0$ for all $n \in \naturals$.  We redefine $f_n$ be $0$ on the union of $\lbrace f_n = -\infty$ so we can assume $-\infty < f_n \leq 0$ everywhere.  By  Proposition \ref{IntegrableAlmostEverywhereEqualIntegralEqual} the values of the integrals remain unchanged.  Observe that $0 \leq f_n - f_1 \uparrow f - f_1$, by Lemma \ref{SignedIntegralLinearity} all the functions are integrable and by the Monotone Convergence Theorem \ref{MCT} we have $\int (f_n - f_1) \, d\mu \uparrow \int (f - f_1) \, d\mu$.  By another application of Lemma \ref{SignedIntegralLinearity} we can add $\int f_1  \, d\mu$ to both sides of the limit and conclude that $\int f_n \, d\mu \uparrow \int f \, d\mu$.

In the general case we write $(f_n)_+ \uparrow f_+$ and $(f_n)_- \downarrow f_-$ and $\int (f_1)_- \, d\mu < \infty$.  By Montone Convergence Theorem \ref{MCT} we have $\int (f_n)_+ \, d\mu \uparrow \int f_+\, d\mu$.  By the previous part of this proof $\int (f_n)_- \, d\mu \downarrow \int f_- \, d\mu$ and $\int f_- \, d\mu < \infty$.  Thus,
\begin{align*}
\int f_n \, d\mu &= \int (f_n)_+ \, d\mu - \int (f_n)_- \, d\mu \uparrow \int f_+ \, d\mu - \int f_- \, d\mu = \int f \, d\mu
\end{align*}
\end{proof}


\begin{thm}[Dominated Convergence Theorem]\label{DCT}Suppose we are given $f, f_1, f_2, \dots$ and $g,g_1,g_2,
  \dots$ measurable functions on $(\Omega,
  \mathcal{A}, \mu)$ such that $\abs{f_n} \leq g_n$, $\lim_{n \to
    \infty} f_n = f$, $\lim_{n \to
    \infty} g_n = g$ and $\lim_{n \to \infty} \int g_n \, d \mu  =
  \int g \, d \mu < \infty$.  Then $\lim_{n \to \infty} \int f_n \,
  d\mu = \int f \, d\mu$.
\end{thm}
\begin{proof}The trick here is to notice that by our assumption, $g_n
  \pm f_n \geq 0$ and we can apply Fatou's Lemma to both sequences.
  Doing so we get
\begin{align*}
\int g \, d\mu \pm \int f \, d\mu &= \int \lim_{n \to \infty} g_n \,
d\mu \pm \int \lim_{n \to \infty}  f_n \, d\mu \\
&=\int \liminf_{n \to \infty} g_n \,
d\mu \pm \int \liminf_{n \to \infty}  f_n \, d\mu \\
&=\int \liminf_{n \to \infty} \left (g_n  \pm  f_n \right ) \, d\mu \\
&\leq \liminf_{n \to \infty} \int \left (g_n  \pm  f_n \right ) \, d\mu \\
&= \liminf_{n \to \infty} \int g_n \, d\mu + \liminf_{n \to \infty} \int \pm  f_n  \, d\mu \\
&= \int g \, d\mu + \liminf_{n \to \infty} \int \pm  f_n  \, d\mu \\
\end{align*}
Now subtract $\int g \, d\mu$ from both sides of the equation and we
get two inequalities $\pm \int f \, d\mu \leq \liminf_{n \to \infty}
\int \pm  f_n  \, d\mu $.  It remains to put these two inequalities
together 
\begin{align*}
\limsup_{n \to \infty} \int f_n \, d\mu &= -\liminf_{n \to \infty}
\int -f_n \, d\mu \\
&\leq \int f \, d\mu \\
&\leq \liminf_{n \to \infty} \int f_n \, d\mu \\
\end{align*}
and the result is proved by the
obvious fact that $\liminf f_n \leq \limsup f_n$.
\end{proof}
Most applications of Dominated Convergence only use the special case in
which the sequence $g_n$ is constant.  We call out this special case
as a corollary of the general theorem.
\begin{cor}Suppose we are given $f, f_1, f_2, \dots$ and $g$ measurable functions on $(\Omega,
  \mathcal{A}, \mu)$ such that $\abs{f_n} \leq g$, $\lim_{n \to
    \infty} f_n = f$ and $\int g \, d \mu < \infty$.  Then $\lim_{n \to \infty} \int f_n \,
  d\mu = \int f \, d\mu$.
\end{cor}
\begin{proof}Let $g_n = g$ for all $n>0$ and use Theorem \ref{DCT}.
\end{proof}

\begin{lem}\label{PushforwardMeasure}Suppose we are given a measure space $(\Omega, \mathcal{A},
  \mu)$, a measurable space $(S, \mathcal{S})$ and measurable
  function $f : \Omega \to S$.  The function $\pushforward{f}{\mu} (A) =
  \mu(f^{-1}(A))$ defines a measure on $(S, \mathcal{S})$.  The
  measure $\pushforward{f}{\mu}$ is called the \emph{push forward} of
  $\mu$ by $f$.
\end{lem}
\begin{proof}Clearly, $\pushforward{f}{\mu} (\emptyset) = \mu(\emptyset) =
  0$.  
If we are given disjoint $A_1, A_2, \dots$ then by and the fact that $\mu$ is a measure,
we know 
\begin{align*}
\pushforward{f}{\mu} \left (\bigcup_{i=1}^\infty A_i \right ) & =
\mu \left (\bigcup_{i=1}^\infty f^{-1}(A_i) \right ) & & \text{by Lemma
\ref{SetOperationsUnderPullback}} \\
& = \sum_{i=1}^\infty \mu (f^{-1}(A_i)) & & \text{by countable
  additivity of measure}\\
&=  \sum_{i=1}^\infty \pushforward{f}{\mu} (A_i) & & \text{by
  definition of push forward}
\end{align*}
\end{proof}

\begin{defn}For a probability space $(\Omega, \mathcal{A}, P)$, a
  measurable space $(S, \mathcal{S})$ and a
  random element $\xi : \Omega \to S$, the measure $\pushforward{\xi}{P}$ is
  called the \emph{distribution} or \emph{law} of $\xi$.  We often
  write $\mathcal{L}(\xi)$ for the law of $\xi$.
\end{defn}
\begin{lem}[Change of Variables]\label{ChangeOfVariables}Suppose we are given a measure space $(\Omega, \mathcal{A},
  \mu)$, a measurable space $(S, \mathcal{S})$, and measurable
  functions $f : \Omega \to S$ and $g : S \to \reals$, then 
\begin{align*}
\int (g \circ f) \, d \mu = \int g \, d (\pushforward{f}{\mu})
\end{align*}
Whenever either side of the equality exists, the other does and they
are equal.
\end{lem}
\begin{proof}
To begin with we assume that $g = \characteristic{A}$ for $A \in
\mathcal{S}$.  The first simple claim is that $\characteristic{A}
\circ f = \characteristic{f^{-1}(A)}$.  This is seen by unfolding
definitions for an $\omega \in \Omega$:
\begin{align*}
(\characteristic{A} \circ f)(\omega) &= \characteristic{A} (f(\omega)) \\
&= \begin{cases}
1 &\text{if $f(\omega) \in A$}\\
0 &\text{if $f(\omega) \notin A$}
\end{cases}\\
&= \begin{cases}
1 &\text{if $\omega \in f^{-1}(A)$}\\
0 &\text{if $\omega \notin f^{-1}(A)$}
\end{cases}\\
&= \characteristic{f^{-1}(A)}(\omega)
\end{align*}
Using this fact the result of the theorem follows for
$\characteristic{A}$ by another simple calculation
\begin{align*}
\int \characteristic{A} \, d (\pushforward{f}{\mu}) &= (\pushforward{f}{\mu})(A)
\\
&= \mu(f^{-1}(A)) \\
&= \int \characteristic{f^{-1}(A)} \, d\mu \\
&= \int (\characteristic{A} \circ f) \, d\mu
\end{align*}

Next we assume that $g = c_1 \characteristic{A_1} + \cdots + c_n
\characteristic{A_n}$ is a simple function. As a general property of
the linearity of composition of functions we can see that 
\begin{align*}
g \circ f
&= c_1 (\characteristic{A_1} \circ f) + \cdots + c_n (\characteristic{A_n} \circ f)
\end{align*}
Coupling this with the result for indicator functions and linearity of
integral we get
\begin{align*}
\int g \, d (\pushforward{f}{\mu}) &= \sum_{i=1}^n c_i \int \characteristic{A_i} \, d (\pushforward{f}{\mu})
\\
&= \sum_{i=1}^n c_i \int (\characteristic{A_i} \circ f) \, d \mu \\
&=  \int \sum_{i=1}^n c_i (\characteristic{A_i} \circ f) \, d \mu \\
&= \int (g \circ f) \, d\mu
\end{align*}

Next we suppose that $g$ is a positive measurable function.  We know
that we can find an increasing sequence of positive simple functions
$g_n \uparrow g$.  Note that $g \circ f$ is positive measurable, $g_n
\circ f$ is positive simple and $g_n \circ f \uparrow g \circ f$.  Now
can use the result proven for simple functions and Monotone
Convergence 
\begin{align*}
\int g \, d(\pushforward{f}{\mu}) &= \lim_{n \to \infty} \int g_n
\, d(\pushforward{f}{\mu}) & &\text{ by Monotone Convergence} \\
&= \lim_{n \to \infty} \int (g_n \circ f)
\, d\mu & &\text{ by result for simple functions} \\
&=\int (g \circ f)
\, d\mu & &\text{ by Monotone Convergence} \\
\end{align*}

The last step is to consider an integrable $g$.  Write it as $g = g_+
- g_-$ for $g_\pm$ positive and use linearity of the integral and the
result just proven for positive functions.
\end{proof}

\begin{defn}Suppose we are given a measure space $(\Omega, \mathcal{A},
  \mu)$ and a positive measurable function $f : \Omega \to \reals_+$.
  We define the measure $f \cdot \mu$ by the formula
\begin{align*}
(f \cdot \mu)(A) = \int \characteristic{A} \cdot f \, d \mu = \int_A f
\, d\mu
\end{align*}
If $\nu$ is a measure of the above form, then we say that $f$ is a
$\mu$\emph{-density} of $\nu$.
\end{defn}
\begin{lem}\label{ChainRuleDensity}Suppose we are given a measure space $(\Omega, \mathcal{A},
  \mu)$, a positive measurable function $f : \Omega \to \reals_+$ and
  and measurable function $g : \Omega \to \reals$, then 
\begin{align*}
\int f g \, d \mu = \int g \, d (f \cdot \mu)
\end{align*}
Whenever either side of the equality exists, the other does and they
are equal.
\end{lem}
\begin{proof}First assume that $g=\characteristic{A}$ is an indicator
  function.  The result is just the definition of the measure $f \cdot
  \mu$:
\begin{align*}
\int \characteristic{A} \, d(f \cdot \mu) = (f \cdot \mu)(A) = \int \characteristic{A}
\cdot f \, d\mu
\end{align*}
Next assume that $g = \sum_{i=1}^n c_i \characteristic{A_i}$ is a
simple function.  Then we can simply apply linearity of the integral
\begin{align*}
\int g \, d(f \cdot \mu) &=\sum_{i=1}^n c_i \int \characteristic{A_i}
\, d(f \cdot \mu)\\
 &=\sum_{i=1}^n c_i \int \characteristic{A_i}\cdot f
\, d\mu  \\
 &=\int g\cdot f
\, d\mu  \\
\end{align*}
For a positive measurable $g$ we pick an increasing approximation by simple
functions $g_n \uparrow g$.  We note that for positive $f$ we have 
 $g_n \cdot f$ positive (not necessarily simple) with $g_n \cdot f \uparrow g \cdot f$.  Thus,
\begin{align*}
\int g \, d(f \cdot \mu) &= \lim_{n \to \infty} \int g_n \, d(f \cdot
\mu) & &\text{definition of integral}\\
&= \lim_{n \to \infty} \int g_n \cdot f \, d \mu & &\text{by result for
  simple functions}\\
&= \int g \cdot f \, d \mu & &\text{by Monotone Convergence}\\
\end{align*}
The last step is to pick an integrable $g = g_+ - g_-$ and use
linearity of integral.  Note also that in this case the two integrals
in question are defined for exactly the same $g$.
\end{proof}

\subsection{Standard Machinery}
We've put together a collection of definitions and tools for
talking about integration and proving theorems about integration.
What is probably not clear at this point is that there are some very
useful patterns for how these defintitions, lemmas and theorems are
used.  One such pattern is so commonplace that I have heard it called
the \emph{standard machinery}.  
Suppose one wants to show a result about general measurable
functions.  A proof of the result using the standard machinery proceeds by 
\begin{itemize}
\item[(i)]Demonstrating the result for indicator functions.
\item[(ii)]Arguing by linearity that the result holds for simple functions.
\item[(iii)]Showing the result holds for non-negative measurable
  functions by approximating by an increasing limit of simple
  functions and using the Monotone Convergence Theorem.
\item[(iv)]Showing the result for arbitrary functions by expressing an
  arbitrary measurable function as a difference of non-negative
  measurable functions.
\end{itemize}
The proof of Lemma \ref{ChangeOfVariables} and Lemma
\ref{ChainRuleDensity} are examples of proofs
using the standard machinery.  It is a good idea to get very
comfortable with such arguments as it is quite common in many texts to
leave any such proof as an exercise for the reader.  An important refinement of the standard machinery involves using a
monotone class argument with the $\pi$-$\lambda$ Theorem to demonstrate the result for all indicator
functions.  Recall that to do that, one shows that the collection of
sets whose indicator functions satisfy the theorem is a
$\lambda$-system and to then prove the result a $\pi$-system of sets
such that the $\pi$-system generates the $\sigma$-algebra of the
measurable space.
\section{Products of Measurable Spaces}

Given a collection of measurable spaces there is a standard
construction that makes the cartesian product of the spaces into a
measurable space.
\begin{defn}Suppose we are given an index set $T$ and for each $t \in
  T$ we have a measurable space $(\Omega_t, \mathcal{A}_t)$.  The
  \emph{product} $\sigma$\emph{-algebra} $\bigotimes_t \mathcal{A}_t$ on the cartesian product
  $\prod_t \Omega_t$ is the $\sigma$-algebra
  generated by all one dimensional \emph{cylinder sets} $A_t \times
  \prod_{s \neq t} \Omega_s$ for $A_t \in \mathcal{A}_t$.
\end{defn}

TODO: Show that this is the smallest $\sigma$-algebra that make the
projections measurable


\begin{prop}\label{GeneratingSetProductSigmaAlgebra}Suppose we are given an index set $T$ and for each $t \in
  T$ we have a measurable space $(\Omega_t, \mathcal{A}_t)$ and a set $\mathcal{C}_t \subset \mathcal{A}_t$ with $\sigma(\mathcal{C}_t) = \mathcal{A}_t$.  
Then $\bigotimes_t \mathcal{A}_t$ is generated by cylinder sets of the form $C_t \times
  \prod_{s \neq t} \Omega_s$ for $C_t \in \mathcal{C}_t$.
\end{prop}
\begin{proof}
It suffices to show that every cylinder set  $\prod_{s \neq t} \Omega_s$ for $A_t \in \mathcal{A}_t$ is in $\sigma$-algebra generated by sets of the form $C_t \times
  \prod_{s \neq t} \Omega_s$ for $C_t \in \mathcal{C}_t$.  This follows from Exercise \ref{PullbackGeneratingSet}.
\end{proof}

\begin{prop}\label{MeasurableFunctionsIntoProductSigmaAlgebra}Suppose we are given an index set $T$ and for each $t \in
  T$ we have a measurable space $(\Omega_t, \mathcal{A}_t)$ and let
  $(\Omega,\mathcal{A}$ be a measurable space.  A function
  $f :\Omega \prod_t \Omega_t $ is $\mathcal{A} / \bigotimes_t
  \mathcal{A}_t$-measurable if and only if $\pi_t \circ f$ is $\mathcal{A} / \mathcal{A}_t$-measurable for every $t \in T$.
\end{prop}
\begin{proof}
By definition each projection $\pi_t$ is measurable so $\pi_t \circ f$ is measurable if $f$ is.  Suppose that every $\pi_t \circ f$ is measurable.  For every $t \in T$ and $A_t \in \mathcal{A}_t$ we know that $f^{-1}(\pi_t^{-1}(A_t)) = (\pi_t \circ f)^{-1}(A_t)$ is measurable.  By definition sets of the form $\pi_t^{-1}(A_t)$ generate $\bigotimes_t  \mathcal{A}_t$ so measurability of $f$ follows from Lemma \ref{MeasurableByGeneratingSet}.
\end{proof}

TODO: Show that the countable product of Borel $\sigma$-algebras is the Borel
$\sigma$-algebra with respect to the product topology in the separable
case.  Note that the non-separable case is more subtle and in fact
turns out to be important (especially in statistics)!

\begin{prop}\label{BorelProductSigmaAlgebrasOnProductSpaces}Let $S_1, S_2, \dotsc$ be topological spaces then
  $\mathcal{B}(S_1) \otimes \mathcal{B}(S_2) \otimes \dotsb \subset
  \mathcal{B}(S_1 \times S_2 \times \dotsb)$.  If every $S_n$ is
  second countable then $\mathcal{B}(S_1) \otimes \mathcal{B}(S_2) \otimes \dotsb =
  \mathcal{B}(S_1 \times S_2 \times \dotsb)$.
\end{prop}
\begin{proof}
$\mathcal{B}(S_1) \otimes \mathcal{B}(S_2) \otimes \dotsb$ is the
$\sigma$-algebra generated by cylinder sets $A_n \times \prod_{m \neq
  n} S_m$ for $A_n \in \mathcal{B}(S_n)$ so it suffices to show that
such cylinder sets are in $\mathcal{B}(S_1 \times S_2 \times
\dotsb)$.  This is clearly true for the case of $A_n$ open since in
this case we have a cylinder set for the product topology.  On the
other hand the set of all $A_n \subset S_n$ for which $A_n \times \prod_{m \neq
  n} S_m \in \mathcal{B}(S_1 \times S_2 \times \dotsb)$ is easily seen
to be a $\lambda$-system so we may apply the $\pi$-$\lambda$ Theorem
\ref{MonotoneClassTheorem}.

On the other hand, assume that each $S_n$ is second countable.  It
follows that $S_1 \times S_2 \times \dotsb$ is second countable and
therefore every open set is a countable union 

TODO: Finish...  I think there is a result that $\mathcal{B}(S \times
S) = \mathcal{B}(S) \otimes \mathcal{B}(S)$ implies that $S$ is second
countable (check Van der Vaart and Wellner).
\end{proof}

An important corollary of this result is that we have countable generating set (in fact a countable generating $\pi$-system) 
for $\mathcal{B}(\reals^d)$.
\begin{cor}\label{MultidimensionalIntervalsGenerateBorel}The Borel $\sigma$-algebra of
  $\reals^d$ is generated by sets of the form
  of the form $(-\infty, x_1] \times \dotsb \times (-\infty, x_d]$ for $x_1, \dotsc, x_d \in \rationals$.
\end{cor}
\begin{proof}
 By Proposition \ref{BorelProductSigmaAlgebrasOnProductSpaces} we know that
$\mathcal{B}(\reals^d)=\mathcal{B}(\reals)^d$ now we can use Proposition \ref{GeneratingSetProductSigmaAlgebra}
and the fact that intervals $(-\infty,x]$ for $x \in \rationals$ generate $\mathcal{B}(\reals)$ (Lemma \ref{IntervalsGenerateBorel}).
\end{proof}

The following is an important scenario that we shall often encounter.
Suppose we have a measurable space $(\Omega, \mathcal{A})$ and a
collection of measurable functions $f_t : \Omega \to (S_t,
\mathcal{S}_t)$.  From a purely set-theoretic point of view this
specification of functions is in fact
equivalent to the specification of a single function $f : \Omega \to
\prod_t S_t$ (i.e. if we let $\pi_s : \prod_t S_t \to S_s$ be the
projections then we define $\pi_s(f(\omega)) = f_s(\omega)$).  

\begin{lem}Given a collection of measurable functions $f_t : \Omega
  \to S_t$ and the equivalent function $f : \Omega \to \prod_t S_t$ we
  have $\sigma(\bigwedge_t \sigma(f_t)) = \sigma(f)$.
\end{lem}
\begin{proof}
To see that $\sigma(\bigwedge_t \sigma(f_t)) \subset \sigma(f)$ it
suffices to show that $\sigma(f_t) \subset \sigma(f)$ for all $t \in
T$.  This follows since for any $A_t \in \mathcal{S}_t$, we have
$f_t^{-1}(A) = f^{-1}(A \times  \prod_{s \neq t} \Omega_s)$.  This
fact also shows that $\sigma(f) \subset \sigma(\bigwedge_t
\sigma(f_t))$ since the cylinder sets $A \times  \prod_{s \neq t}
\Omega_s$ generate $\otimes_t \mathcal{S}_t$ by Lemma \ref{MeasurableByGeneratingSet}.
\end{proof}

TODO: Show that the collection $(f_{t_1}, \dotsc, f_{t_n}) \in A$ is a
$\pi$-system (which is clearly generating by the previous Lemma).  Use
this fact in Lemma \ref{ProcessLawsAndFDDs} and Theorem \ref{ExistenceMarkovProcess}.

\section{Null Sets and Completions of Measures}
\begin{defn}Let $(\Omega,  \mathcal{A}, \mu)$ be a measure space.  We
  say that a property hold \emph{almost everywhere} if the set where
  the property does not hold is contained in a set of measure zero.
\end{defn}
\begin{lem}\label{ZeroIntegralImpliesZeroFunction}Let $f \geq 0$ be a measurable function on $(\Omega,
  \mathcal{A}, \mu)$.  $\int f \, d\mu = 0$ if and only if $f = 0$
  almost everywhere.
\end{lem}
\begin{proof}Clearly this is true by definition for indicator
  functions.  It also is true for simple functions by linearity of integral.  Write $f = c_1 \characteristic{A_1} + \dotsb + c_n \characteristic{A_n}$ with $c_j > 0$ for all $j=1, \dotsc, n$ and the $A_j$ disjoint so that 
$\int f \, d\mu = c_1 \mu(A_1) + \dotsc + c_n \mu(A_n)$.  It is clear that $\int f \, d \mu = 0$ if and only if $\mu(A_j) = 0$ for all $j=1, \dotsc, n$.  Since $\mu \lbrace f \neq 0 \rbrace = \sum_{j=1}^n \mu(A_j)$ the result follows for simple functions.

For arbitrary $f\geq 0$, we take an
increasing  approximating sequence of simple functions $0 \leq f_n \uparrow
f$.  If $\int f \, d\mu = 0$, monotonicity of integral
implies $\int f_n \, d\mu = 0$ for each $n$.  Therefore, $f_n = 0$ almost
everywhere for each $n$ and therefore $f_n = 0$ almost everywhere for
all $n$ by taking a countable union.  Thus $f = \lim_{n \to \infty} f_n = 0$ almost
everywhere. 
If on the other hand we assume that $f = 0$ almost
everywhere, then by the increasing nature of $f_n$, we see that $f_n
=0$ for all $n$ almost everywhere and therefore $\int f_n \, d\mu = 0$
for every $n$.  By Monotone Convergence we see that $\int f \, d\mu = 0$.
\end{proof}
 
\begin{defn}Let $(\Omega, \mathcal{A}, \mu)$ be a measure space then we say $A \subset \Omega$ is a \emph{$\mu$-null set} if
there exists a measurable $B \in \mathcal{A}$ such that $A \subset B$ and $\mu(B) = 0$.  The $\sigma$-algebra generated by $\mathcal{A}$ and 
the $\mu$-null sets is called the \emph{$\mu$-completion of $\mathcal{A}$} and is denoted $\mathcal{A}^\mu$.  
\end{defn}

\begin{lem}\label{CompletionOfSigmaAlgebra}Let $(\Omega, \mathcal{A}, \mu)$ then a set $A \in \mathcal{A}^\mu$ if and only if there exists $B \in \mathcal{A}$ such that $A \Delta B$ is a $\mu$-null set.  Let $(S, \mathcal{S})$ be a Borel space then a function $f : \Omega \to S$ is $\mathcal{A}^\mu$-measurable if and only if there exists a $\mathcal{A}$-measurable $g : \Omega \to S$ such that $f = g$ $\mu$-almost everywhere.
\end{lem}
\begin{proof}
TODO: Finish
\end{proof}

\begin{lem}\label{CompletionOfMeasure}Let $(\Omega, \mathcal{A}, \mu)$ be a measure space, let
  $\mathcal{F}$ be a sub $\sigma$-algebra of $\mathcal{A}$ and let
  $\mathcal{F}^{\mu}$ be the $\mu$-completion of $\mathcal{F}$.  Then
  for every $A \in \mathcal{F}^{\mu}$ there exist $A_-, A_+ \in
  \mathcal{F}$ such that $A_- \subset A \subset A_+$ and $\mu(A_-) = \mu(A_+)$.
Moreover if for every $A \in \mathcal{F}^{\mu}$ we define $\tilde{\mu}(A) = \mu(A_-) = \mu(A_+)$ then
$\tilde{\mu}$ is a measure on $\mathcal{F}^{\mu}$ and is the unique extension
of $\mu$ to a measure on $\mathcal{F}^{\mu}$.  Suppose $f$ is $\mathcal{F}^\mu$-measurable and either non-negative or integrable and $g$ is $\mathcal{F}$-measurable with
$f = g$ $\mu$-almost everywhere, then
\begin{align*}
\int f d \tilde{\mu} &= \int g \mu
\end{align*}
\end{lem}
\begin{proof}
Pick $B, C, D \in \mathcal{F}$  such that $A \setminus B \subset C$,
$\mu(C)=0$, $B \setminus A \subset D$  and $\mu(D) = 0$.  Define
$A_+ = B \cup C$ and $A_- = B \setminus D$  noting that
\begin{align*}
A &= A \cap B \cup A \setminus B  \subset B \cup C
\end{align*}
and
\begin{align*}
A_- &= B \cap D^c \subset B \cap (B \cap A^c)^c  = B \cap A \subset A
\end{align*}
To compute the measures
\begin{align*}
\mu(B) &= \mu(B \cap D^c) + \mu(B \cap D) \leq \mu(B \setminus D) + \mu(D) = \mu(A_-)
\end{align*}
and subadditivity shows $\mu(B) = \mu(A_-)$.  Similarly 
\begin{align*}
\mu(A_+) &\leq \mu(B) + \mu(C) = \mu(B)
\end{align*}
and subadditivity shows $\mu(A_+) = \mu(B)$.

Suppose we are given $A \in \mathcal{F}^\mu$ and we pick $A_\pm, B_\pm \in \mathcal{A}$ satisfying $A_- \subset A \subset A_+$,  $B_- \subset A \subset B_+$, $\mu(A_-) = \mu(A_+)$ and $\mu(B_-) = \mu(B_+)$.  If follows that $A_- \subset B_+$ and therefore by subadditivity we have $\mu(A_-) \leq \mu(B_+)$.  Reversing the roles of $A_\pm$ and $B_\pm$ we get $\mu(B_-) \leq \mu(A_+)$ and therefore $\mu(A_-) = \mu(A_+) = \mu(B_-) = \mu(B_+)$.  Thus the definition of $\tilde{\mu}$ is consistent.
To see that $\tilde{\mu}$ is a measure on $\mathcal{F}^{\mu}$, only countable additivity needs to 
be shown.  By the previous construction and countable additivity of $\mu$ on $\mathcal{F}$, it suffices to show the following claim.
\begin{clm} Let $A_1, A_2, \dotsc \in \mathcal{F}^{\mu}$ be disjoint there exist disjoint $B_1, B_2, \dotsc, \in \mathcal{F}$ such that
$A_n \Delta B_n$ is a $\mu$-null set for every $n \in \naturals$.
\end{clm}
Pick $C_1, C_2, \dotsc \in \mathcal{F}$ such
that $A_n \Delta C_n$ is a $\mu$-null set for every $n \in \naturals$.  Define $B_n = C_n \setminus (C_1 \cup \dotsb \cup C_{n-1})$ and note
that $B_1, B_2, \dotsc \in \mathcal{F}$ and are disjoint.  Also we have using the disjointness of the $A_n$,
\begin{align*}
A_n \cap B_n^c &= A_n \cap (C_n \cap (C_1 \cup \dotsb \cup C_{n-1})^c)^c = A_n \cap (C^c_n \cup (C_1 \cup \dotsb \cup C_{n-1})) \\
&= A_n \setminus C_n \cup (A_n \cap C_1 \cap A_1^c) \cup \dotsb \cup (A_n \cap C_{n-1} \cap A_{n-1}^c)  \\
&\subset \cup_{i=1}^n A_i \Delta C_i
\end{align*}
hence $A_n \setminus B_n$ is a $\mu$-null set.
Also, trivially since $B_n \subset C_n$ it follows that $B_n \setminus A_n \subset C_n \setminus A_n$ is also a $\mu$-null set.

The fact that the extension described is the only such extension is obvious from the existence of sets $A_\pm$ and subadditivity of measure.

Now suppose $g$ is $\mathcal{F}$-measurable, then $g$ is also $\mathcal{F}^\mu$-measurable and $\int g \mu = \int g \tilde{\mu}$.  If $f$ is $\mathcal{F}^\mu$-measurable then we pick $\mathcal{F}$-measurable $g$ with $\tilde{\mu}\lbrace f = g \rbrace = 0$ and it follows from either 
Proposition \ref{AlmostEverywhereEqualIntegralEqual} or Proposition \ref{IntegrableAlmostEverywhereEqualIntegralEqual} that
\begin{align*}
\int f \, d \tilde{\mu} &= \int g \, d \tilde{\mu} = \int g \, d \mu
\end{align*}
\end{proof}

\section{Outer Measures and Lebesgue Measure on the Real Line}
To construct Lebesgue measure on the real line, one proceeds by
demonstrating that one may construct a measure by first constructing a more
primitive object called an outer measure and then proving that outer
measure become measures when restricted to an appropriate collection
of sets.  Having redefined the problem as the construction of outer
measure, one constructs outer measure on real line in a hands on way.

Much of this process that has broader applicability than just the real line,  therefore we state and prove the results
in the more general case.
TODO: Come up with some intuition about outer measure (more
specifically Caratheodory's characterizaiton of sets measurable with
respect to an outer measure; it says in some sense that a measurable
set and its complement have aren't \emph{too} entangled with one another).
\begin{defn}Given a set $\Omega$, an \emph{outer measure} is a
positive  function $\mu : 2^\Omega \to \overline{\reals}_+$ satisfying
\begin{itemize}
\item[(i)] $\mu(\emptyset) = 0$
\item[(ii)] If $A \subset B$, then $\mu(A) \leq \mu(B)$
\item[(iii)] Given $A_1, A_2, \dots \subset \Omega$, then $\mu \left
    (\bigcup_{i=1}^\infty A_i \right ) \leq \sum_{i=1}^\infty \mu(A_i)$.
\end{itemize}
\end{defn}

\begin{defn}Given a set $\Omega$ with outer measure $\mu$, we say a
  set $A \subset \Omega$ is $\mu$\emph{-measurable} if for every $B
  \subset \Omega$,
\begin{align*}
\mu(B) = \mu(A \cap B) + \mu(A^c \cap B)
\end{align*}
\end{defn}

\begin{rem}For every $A,B \subset \Omega$, we have from finite
  subadditivity of outer measure
\begin{align*}
\mu(B) = \mu((A \cap B) \cup (A^c \cap B)) \leq \mu(A \cap B) + \mu(A^c \cap B)
\end{align*}
and therefore to show $\mu$-measurability we only need to show the
reverse inequality.
\end{rem}

\begin{lem}\label{CaratheodoryRestriction}Given a set $\Omega$ with an outer measure $\mu$, let
$\mathcal{A}$ be the collection of $\mu$-measurable sets.  Then
$\mathcal{A}$ is a $\sigma$-algebra and the
  restriction of $\mu$ to $\mathcal{A}$ is a measure.
\end{lem}
\begin{proof}We first note that $A \in \mathcal{A}$ if and only if $A^c
  \in \mathcal{A}$ since the defining condition of $\mathcal{A}$ is
  symmetric in $A$ and $A^c$.

Next we show $\emptyset \in \mathcal{A}$.  To see this,
  take $B \subset \Omega$,
\begin{align*}
\mu(B) &= \mu(\emptyset) + \mu(B) & &\text{since $\mu(\emptyset) = 0$}
\\
& = \mu(\emptyset \cap B) + \mu(B \cap \Omega)
\end{align*}

Next we show that $\mathcal{A}$ is closed under finite intersection.
Pick $A, B \in \mathcal{A}$ and $E \subset \Omega$ and calculate
\begin{align*}
\mu(E) &= \mu(E \cap A) + \mu(E \cap A^c) & &\text{since $A \in
  \mathcal{A}$} \\
&= \mu(E \cap A \cap B) + \mu(E \cap A \cap B^c) +\mu(E \cap A^c) & &\text{since $B \in
  \mathcal{A}$} \\
&\geq \mu(E \cap (A \cap B)) + \mu(E \cap A \cap B^c \cup E \cap A^c)
& & \text{by subadditivity} \\
&\geq \mu(E \cap (A \cap B)) + \mu(E \cap (A \cap B)^c)
& & \text{by monotonicity of $\mu$} \\
\end{align*}
and we have noted that it suffices to show this inequality to show $A
\cap B \in \mathcal{A}$.  Now by De Morgan's Law we conclude that
$\mathcal{A}$ is closed under finite union.

Now we turn to consider the behavior of $\mu$ and show that $\mu$ is
finitely and countably additive over disjoint unions; in fact we show
a bit more.
We let $A,B \in \mathcal{A}$ and let $E \subset \Omega$
be disjoint.
\begin{align*}
\mu(E \cap (A \cup B)) &= \mu(E \cap (A \cup B) \cap A) + \mu(E \cap
(A \cup B) \cap A^c) & & \text{since $A \in \mathcal{A}$} \\
&= \mu(E \cap A) + \mu(E \cap B) & &\text{by set algebra}
\end{align*}
It is easy to see that one can do induction to extend the above result
to all finite disjoint unions.
Now let $A_1, A_2, \dots \in \mathcal{A}$ and $E \subset \Omega$.
Define $U_n = \bigcup_{i=1}^n A_i$ and $U = \bigcup_{i=1}^\infty A_i$.
\begin{align*}
\mu(E \cap U) &\geq \mu(E \cap U_n) & & \text{by monotonicity} \\
&= \sum_{i=1}^n \mu(E \cap A_i) & &\text{by finite additivity and
  disjointness of $A_i$}
\end{align*}
Now take the limit we have $\mu(E \cap U) \geq \sum_{i=1}^\infty \mu(E
\cap A_i)$.  Applying subadditivity of $\mu$ we get the opposite
inequality and we have shown 
\begin{align*}
\mu(E \cap \bigcup_{i=1}^\infty A_i) = \sum_{i=1}^\infty \mu(E
\cap A_i)
\end{align*}
In particular, we can take $E=\Omega$ to show that $\mu$ is countably
additive over disjoint unions.

Having shown how to calculate $\mu$ over countable disjoint unions, we
can show that $U \in \mathcal{A}$. For every $n > 0$,
\begin{align*}
\mu(E) &= \mu(E \cap U_n) + \mu(E \cap U_n^c) \\
&\geq \sum_{i=1}^n \mu(E \cap A_i) + \mu(E \cap U) & & \text{by
  subadditivity and monotonicity} \\
\end{align*}
Take the limit and use the previous claim to see
\begin{align*}
\mu(E) &\geq \sum_{i=1}^\infty \mu(E \cap A_i) + \mu(E \cap U) \\
&= \mu(E \cap U) + \mu(E \cap U^c)
 \end{align*}
thereby showing $U \in \mathcal{A}$.

The last thing to show is that a countable union of elements of
$\mathcal{A}$ are in $\mathcal{A}$.  This follows from what we have
shown about countable
disjoint unions since we have already proven this for complements, finite unions
and intersections and therefore for any $A_1,A_2, \dots$ we can define
$B_n = A_n \setminus \bigcup_{i=1}^{n-1} A_i$ so that
$\bigcup_{i=1}^\infty A_i =\bigcup_{i=1}^\infty B_i$ with the $B_i$ disjoint.
\end{proof}

To define \emph{Lebesgue measure} on $\reals$ we will leverage the
construction above and first define an outer measure by approximating
by intervals.  Given an interval $I \subset \reals$, let $\abs{I}$ be
length of $I$.
\begin{thm}[Existence of Lebesgue Measure]\label{LebesgueMeasure}There exists a
  unique measure $\lambda$ on $(\reals, \mathcal{B}(\reals)$ such that
  $\lambda(I) = \abs{I}$ for all intervals $I \subset \reals$.
\end{thm}
Before we begin the proof of the theorem we need first construct an
outer measure.
\begin{lem}[Lebesgue Outer Measure]\label{LebesgueOuterMeasure}Define the function $\lambda : 2^\reals \to \reals$ defined by 
\begin{align*}
\lambda(A) &= \inf_{\{I_k\}} \sum_k \abs{I_k} 
\end{align*}
where the infimum ranges over countable covers of $A$ by intervals.
Then $\lambda$ is
an outer measure.  In addition, $\lambda(I) = \abs{I}$ for every
interval $I \subset \reals$.
\end{lem}
\begin{proof}It is clear that $\lambda$ is positive and $\lambda(\emptyset) = 0$.  It is also
  clear that $\lambda$ is increasing since for any $A \subset B
  \subset \reals$ any cover of $B$ is also a cover of $A$.  

To see subadditivity, take $A_1, A_2, \dots \subset \reals$.  Pick
$\epsilon > 0$ and then for each $A_n$ we take a countable cover by intervals
$I_{n1}, I_{n2}, \dots$ such that $\lambda(A_n) \geq \sum_{k=1}^\infty
\abs{I_{nk}} - \frac{\epsilon}{2^n}$.  Then, the collection of
intervals $I_{nk}$ for $n,k > 0$ is an countable cover of
$\bigcup_{i=1}^\infty A_i$ and therefore
\begin{align*}
\lambda \left (\bigcup_{i=1}^\infty A_i \right ) &\leq
\sum_{n=1}^\infty \sum_{k=1}^\infty \abs{I_{nk}} \\
&\leq \sum_{n=1}^\infty \left ( \lambda(A_n) + \frac{\epsilon}{2^n}
\right )\\
&= \sum_{n=1}^\infty \lambda(A_n) + \epsilon
\end{align*}
Now let $\epsilon \to 0$ and we have proven subadditivity.

To prove that $\lambda(I) = \abs{I}$, we first consider intervals of
the form $I = [a,b]$ with $a < b$.  The family of intervals $(a -
\epsilon, b+\epsilon)$ for $\epsilon > 0$ shows that $\lambda{I} \leq
\abs{I}$ so we only need to show the opposite inequality.
Suppose we are given a countable cover by open intervals $I_1, I_2,
\dots$.  We need to show that $\abs{I} \leq \sum_{k=1}^\infty \abs{I_k}$.
By the Heine-Borel Theorem (Theorem \ref{HeineBorel}), there
is a finite subcover $I_1, \dots, I_n$ and it suffices to show
that $\abs{I} \leq \sum_{k=1}^n \abs{I_k}$ for the finite subcover.

For finite covers we
can proceed by induction.  To begin, consider a cover by a single
interval.  For any $J \supset I$ we know that
$\abs{J} \geq \abs{I}$.

For the induction step, assume that $\inf_{\{I_k\}} \sum_{k=1}^n
\abs{I_k} = \abs{I}$ where the infimum is over covers by $n$
intervals.  Take a cover of $I$ by $n+1$ intervals $I_1, \dots,
I_{n+1}$.  There exists an $I_k$ such that $b \in I_k$.  If we write
$I_k = (a_k,b_k)$, then the rest of the $I_j$ form a cover of
$[a,a_k]$.
\begin{align*}
\abs{I} &= (b-a_k) + (a_k - a) \\
&\leq \abs{I_k} + \sum_{m \neq k} \abs{I_m} & &\text{by induction
  hypothesis applied to $[a,a_k]$} \\
&=\sum_m \abs{I_m}
\end{align*}
It remains to eliminate the restriction to bounded closed intervals.
Clearly every cover of 
$[a,b]$ by open intervals is a cover of $(a,b)$.  On the other hand, every countable cover of $(a,b)$ can be extended to a countable cover of $[a,b]$ by adding
at most two arbitrarily small intervals of the form
$(a-\epsilon,a+\epsilon)$ and $(b-\epsilon,b+\epsilon)$.  An
\emph{epsilon of room} argument shows that $\lambda (a,b) = \lambda
[a,b]$.  Monotonicity of $\lambda$ shows the same is true for half
open intervals.


TODO: Show that outer measure of infinite intervals is infinite.
\end{proof}

\begin{defn}A subset $A \subset \reals$ is \emph{Lebesgue measurable}
  if $A$ is $\lambda$-measurable with respect to the Lebesgue outer measure.
\end{defn}

\begin{lem}\label{BorelsAreLebesgueMeasurable}Every Borel measurable $A \subset \reals$ is also Lebesgue measurable.
\end{lem}
\begin{proof}Since we know that the collection of Lebesgue measurable
  sets is a $\sigma$-algebra, and we know that the Borel algebra on
  $\reals$ is generated by intervals of the form $(-\infty, x]$, it
  suffices to show that each such interval is Lebesgue measurable.

Take an interval $I=(-\infty, x]$, a set $E \subset \reals$ and
$\epsilon > 0$.  Pick a countable covering $I_1, I_2, \dots$ of $E$ by
open intervals so that
$\lambda(E) + \epsilon \geq \sum_{k=1}^\infty \abs{I_k} $.
\begin{align*}
\lambda(E) + \epsilon &\geq \sum_{k=1}^\infty \abs{I_k} \\
&=\sum_{k=1}^\infty \abs{I_k \cap I}  + \sum_{k=1}^\infty \abs{I_k \cap I^c} \\
&=\sum_{k=1}^\infty \lambda(I_k \cap I) + \sum_{k=1}^\infty
\lambda(I_k \cap I^c) \\
&\geq \lambda \left ( \bigcup_{k=1}^\infty I_k \cap I \right )
+\lambda \left ( \bigcup_{k=1}^\infty I_k \cap I^c \right ) &
&\text{by subadditivity} \\
&\geq \lambda(E \cap I) + \lambda(E \cap I^c)
\end{align*}
where the last line holds because $I_k \cap I$ is a countable cover of
$E \cap I$ and similarly for $E \cap I^c$.  Now let $\epsilon \to 0$
to get the result.

TODO: Actually $I_k \cap I$ are half open intervals.  The proof needs
to be extended 
to handle this fact.  Presumably an $\frac{\epsilon}{2^n}$ argument
works here.  Note most definitions of Lebesgue outer measure do not
restrict to open covers (then you have to pay the cost of the
$\frac{\epsilon}{2^n}$ argument to apply Heine Borel).
\end{proof}

\begin{lem}[Uniqueness of measure]\label{UniquenessOfMeasure}Let $(\Omega, \mathcal{A}, \mu)$ be a measure space with
  $\mu$ a finite measure.  Suppose $\nu$ is a finite
  measure on
  $(\Omega, \mathcal{A})$ such that there is a $\pi$-system
  $\mathcal{C}$ such that $\sigma(\mathcal{C})=\mathcal{A}$, $\Omega \in \mathcal{C}$ and for all $A \in
  \mathcal{C}$ we have $\mu(A) = \nu(A)$, then $\mu=\nu$.

If we assume that $\mu$ a $\sigma$-finite measure and $\nu$ is a 
$\sigma$-finite measure such that there exists a partition $\Omega =
\Omega_1 \cup \Omega_2 \cup \cdots$ with $\mu(\Omega_n) =
\nu(\Omega_n) < \infty$, the result holds as well.
\end{lem}
\begin{proof}
First we assume that $\mu$ (and then by hypothesis $\nu$) is finite.
We apply a monotone class
argument.  Consider the collection $\mathcal{D}$ of $A \in
\mathcal{A}$ such that $\mu(A) = \nu(A)$.  We claim that
this collection is a $\lambda$-system.  Since we have assumed
$\mu(\Omega) = \nu(\Omega)$ we have that $\Omega \in \mathcal{D}$.  Now suppose $A
\subset B \in \mathcal{D}$.  By additivity of measure and finiteness
of $\mu$ and $\nu$,
\begin{align*}
\mu(B \setminus A) = \mu(B) - \mu(A) = \nu(B) -\nu(A) =
\nu(B \setminus A)
\end{align*}
Now we assume $A_1 \subset A_2 \subset \cdots \in \mathcal{D}$.  By
continuity of measure (Lemma \ref{ContinuityOfMeasure}) 
\begin{align*}
\mu \left ( \bigcup_i A_i \right ) = \lim_{n \to \infty} \mu(A_i) = \lim_{n \to \infty} \nu(A_i) =
\nu \left ( \bigcup_i A_i \right )
\end{align*}
Application of the $\pi$-$\lambda$ Theorem (Theorem
\ref{MonotoneClassTheorem}) together with the fact that
$\sigma(\mathcal{C}) = \mathcal{A}$ shows that equality holds on all
of $\mathcal{A}$.

Now we handle to the $\sigma$-finite case.  We a partition
$\Omega=\Omega_1 \cup \Omega_2 \cup \cdots$ such that $\mu(\Omega_n) =
\nu(\Omega_n) < \infty$ for all $n$.  Denote $\mu_n$ and
$\nu_n$ the restriction of $\mu$ and $\nu$ to the set $\Omega_n$.  We
note that $\mu_n$ and $\nu_n$ each satisfy the hypothesis of the lemma
for the finite measure case (e.g. $\mu_n(A) = \mu(\Omega_n \cap A)$).  Therefore
we can conclude that $\mu_n = \nu_n$ on all of $\mathcal{A}$ for all
$n$.  For any
$A \in \mathcal{A}$ define $A_n = \cup_{k=1}^n \Omega_k \cap A$ note
that 
\begin{align*}
\mu (A_n) = \sum_{k=1}^n \mu_k(A) = \sum_{k=1}^n \nu_k(A)  = \nu(A_n)
\end{align*}
and $A_1 \subset A_2 \subset \cdots $ with $\cup_{n=1}^\infty A_n =
A$.  Now apply continuity of measure (Lemma \ref{ContinuityOfMeasure}) to see that $\mu(A)=\nu(A)$.
\end{proof}

TODO: Do we need to assume that there is a partition with
$\mu(\Omega_n) = \nu(\Omega_n)$ or can it be derived from the fact
that $\sigma(\mathcal{C}) = \mathcal{A}$.  Is suspect it can be
derived but the applications we have in mind it is trivial to generate
the partition by hand.

 Now we are ready to prove the existence and uniqueness of Lebesgue
measure (Theorem \ref{LebesgueMeasure}).

\begin{proof}The existence of Lebesgue measure clearly follows from
  Lemma \ref{CaratheodoryRestriction} applied to the outer measure
  constructed in Lemma \ref{LebesgueOuterMeasure}.  The fact that the
  $\sigma$-algebra of the restriction contains the Borel sets follows
  from Lemma \ref{BorelsAreLebesgueMeasurable}.

It remains to show uniqueness.  Now clearly the collection of intervals is closed under finite
intersections hence is a $\pi$-system that generates
$\mathcal{B}(\reals)$.  Furthermore, $\reals =
\cup_{n=-\infty}^\infty (n, n+1]$ so we may apply Lemma
\ref{UniquenessOfMeasure} to get uniqueness.
\end{proof}

TODO: Show that the $\sigma$-algebra of $\lambda$-measurable sets is
the completion of the Borel $\sigma$-algebra.

\begin{defn}A measure space $(\Omega, \mathcal{A}, \mu)$ is
  \emph{$\sigma$-finite} if there exists a countable partition $\Omega
  = \Omega_1 \cup \Omega_2 \cup \cdots$ such that $\mu(\Omega_i) < \infty$.
\end{defn}

\subsection{Abstract Version of Caratheodory Extension}
The construction of Lebesgue measure we have given actually has a
broad generalization which we present here.

\begin{defn}A non-empty collection $\mathcal{A}_0$ of subsets of a set
  $\Omega$ is called a \emph{Boolean  algebra} if given any $A, B
  \in \mathcal{A}_0$ we have
\begin{itemize}
\item[(i)]$A^c \in \mathcal{A}_0$
\item[(ii)]$A \cup B \in \mathcal{A}_0$
\item[(iii)]$A \cap B \in \mathcal{A}_0$
\end{itemize}
\end{defn}

Note that it is trivial induction argument to extend the closure
properties to arbitrary finite unions and intersections.

\begin{defn}A \emph{pre-measure} on a Boolean algebra $(\Omega, \mathcal{A}_0)$ is
  a function $\mu_0 : \mathcal{A}_0 \to \overline{\reals}_+$ such that 
\begin{itemize}
\item[(i)]$\mu_0(\emptyset) = 0$
\item[(ii)]For any $A_1, A_2, \dotsc \in \mathcal{A}_0$ such that the
  $A_n$ are disjoint and 
  $\cup_{n=1}^\infty A_n \in \mathcal{A}_0$, we have
  $\mu_0(\cup_{n=1}^\infty A_n) = \sum_{n=1}^\infty \mu_0(A_n)$.
\end{itemize}
\end{defn}

\begin{lem}A pre-measure is finitely additive and montonic.  That is to say given
  any disjoint $A_1, \dotsc, A_n \in \mathcal{A}_0$ we have
  $\mu_0(\cup_{i=1}^n A_i) = \sum_{i=1}^n \mu_0(A_i)$ and given $A
  \subset B$ with $A, B \in \mathcal{A}_0$, we have $\mu_0(A) \leq \mu_0(B)$.
\end{lem}
\begin{proof}
Finite additivity follows by extending the finite sequence to an infinite sequence by appending
copies of the emptyset and using the fact that $\mu_0(\emptyset)=0$.  Monotonicity follows from finite additivity by writing $B = A \cup
B\setminus A$ so that $\mu_0(B) = \mu_0(A) + \mu_0(B\setminus A) \geq \mu_0(A)$.
\end{proof}

Our goal is to show that any pre-measure on a Boolean algebra
$\mathcal{A}_0$ may be
extended to a measure on a $\sigma$-algebra containing
$\mathcal{A}_0$.  We proceed in four steps
\begin{itemize}
\item[1)] Define an outer measure $\mu^*$ from $\mu_0$
\item[2)] Show that all sets in $\mathcal{A}_0$ are $\mu^*$-measurable.
\item[3)] Show that for all sets $A \in \mathcal{A}_0$, $\mu^*(A) = \mu_0(A)$.
\item[4)] Use the Caratheodory restriction to create a
  $\sigma$-algebra and measure.
\end{itemize}

The construction of an outer measure from a set function requires almost no assumptions.
\begin{lem}\label{PremeasureToOuterMeasure}Let $\mathcal{C}$ be collection of subsets of $\Omega$ such that $\emptyset \in \mathcal{C}$ and
let $\mu_0 : \mathcal{C} \to \overline{\reals}_+$ satisfy $\mu_0(\emptyset) = 0$  then the set function $\mu^* : 2^\Omega \to
  \overline{\reals}_+$ defined by
\begin{align*}
\mu^*(A) &= \inf \lbrace \sum_{n=1}^\infty \mu_0(A_n) \mid A \subset
\cup_{n=1}^\infty A_n \text { and } A_n \in \mathcal{C} \text{ for
  all $n$} \rbrace
\end{align*}
is an outer measure.
\end{lem}
\begin{proof}
Because $\mu_0(\emptyset)$ and $\emptyset \subset \emptyset$ we see
that $\mu^*(\emptyset) = 0$.

Suppose we are given $A \subset B$.  Then if we have a cover $B
\subset \cup_{n=1}^\infty B_n$ where $B_n \in \mathcal{C}$, then
this is also a cover of $A$.  Therefore $\mu^*(A)$ is an infimum over
a larger collection of covers than that used in calculating $\mu^*(B)$
hence $\mu^*(A) \leq \mu^*(B)$ (we could actually pick an $\epsilon$
and an approximating cover as below then let $\epsilon \to 0$).

Now to show subadditivity.  Let $A_1, A_2, \dotsc$ be a sequence of
arbitrary subsets of $\Omega$.  If any $\mu^*(A_n) = \infty$ then we
automatically know $\mu^*(\cup_{n=1}^\infty A_n) \leq \sum_{n=1}^\infty \mu^*(A_n)$, so we
may assume that all $\mu^*(A_n) < \infty$.  Let $\epsilon > 0$ be
given and for each $n$ we pick $B_{1n}, B_{2n}, \dotsc$ such that $A_n
\subset \cup_{m=1}^\infty B_{mn}$ and $\sum_{m=1}^\infty \mu_0(B_{mn})
\leq \mu^*(A_n) + \frac{\epsilon}{2^n}$.  Now, we also have that
$\cup_{n=1}^\infty A_n
\subset  \cup_{n=1}^\infty \cup_{m=1}^\infty B_{mn}$ and therefore we
know that $\mu^*(\cup_{n=1}^\infty A_n) \leq \sum_{n=1}^\infty \sum_{m=1}^\infty
\mu_0(B_{mn}) \leq \sum_{n=1}^\infty \mu^*(A_n) + \epsilon$.  Since
$\epsilon$ was arbitrary, we have $\mu^*(\cup_{n=1}^\infty A_n) \leq \sum_{n=1}^\infty
\mu^*(A_n)$ so subadditivity is proven.
\end{proof}

\begin{lem}\label{PremeasureBooleanAlgebraOuterMeasurable}Let $\mu_0$  be a set function on a Boolean algebra $(\Omega,
  \mathcal{A}_0)$ such that $\mu_0$ is finitely additive on disjoint sets.  If $\mu^*$ is the outer measure  constructed in Lemma
  \ref{PremeasureToOuterMeasure} and $A \in \mathcal{A}_0$ then $A$ is
  $\mu^*$-measurable.
\end{lem}
\begin{proof}
Since $\emptyset \cap \emptyset = \emptyset$ thus by finite additivity $\mu_0(\emptyset) = 2 \mu_0(\emptyset)$ and
therefore $\mu_0(\emptyset) = 0$.   Therefore we can indeed construct an outer measure from $\mu_0$.

Let $A \in \mathcal{A}_0$ and $B \subset \Omega$ and we have to show
$\mu^*(B) \geq \mu^*(A \cap B) + \mu^*(A^c \cap B)$. 
Pick $B_1, B_2, \dotsc$ such that $B_n \in \mathcal{A}_0$ for all $n$
and $\sum_{n=1}^\infty \mu_0(B_n) \leq \mu^*(B) + \epsilon$.  By
finite additivity of $\mu_0$ and the fact that $A, B_n \in \mathcal{A}_0$,
we can write $\mu_0 (B_n) = \mu_0(A \cap B_n) + \mu_0(A^c \cap B_n)$
and therefore $\sum_{n=1}^\infty \mu_0(A \cap B_n) + \sum_{n=1}^\infty
\mu_0(A^c \cap B_n) \leq \mu^*(B) + \epsilon$.  On the other hand, we
know that $A \cap B \subset \cup_{n=1}^\infty A \cap B_n$ so $\mu^*(A
\cap B) \leq \sum_{n=1}^\infty \mu_0( A \cap B_n)$ and similarly with
$A^c$.  Therefore $\mu^*(A \cap B) + \mu^*(A^c \cap B)\leq \mu^*(B) + \epsilon$.
Take the limit as $\epsilon$ goes to zero and we are done.
\end{proof}

\begin{lem}\label{PremeasureOuterMeasureEqual}Given a pre-measure $\mu_0$ on a Boolean algebra $(\Omega,
  \mathcal{A}_0)$ and the outer measure $\mu^*$ constructed in Lemma
  \ref{PremeasureToOuterMeasure}, if $A \in \mathcal{A}_0$ then $\mu^*(A)=\mu_0(A)$.
\end{lem}
\begin{proof}
Suppose we are given $A \in \mathcal{A}_0$.  Since $A$ is a singleton
cover of itself, we know that $\mu^*(A) \leq \mu_0(A)$.  It remains to
show $\mu_0(A) \leq \mu^*(A)$.  If $\mu^*(A) =\infty$ then this is
trivally true so we may assume $\mu^*(A) < \infty$.  Let $\epsilon
>0$ be given and pick $A_1, A_2, \dotsc \in \mathcal{A}_0$ such that
$A \subset \cup_{n=1}^\infty A_n$ and 
$\sum_{n=1}^\infty \mu_0(A_n) \leq \mu^*(A) + \epsilon$.  Our goal now
is to shrink each of the $A_n$ so that we wind up with a partition of
$A$.  Then we will be able to apply the countable additivity of pre-measures.

First, we convert the cover by $A_n$ into a disjoint cover of $A$.  Let
$B_1 = A_1$ and then define $B_n = A_n \setminus (A_1 \cup \cdots \cup
A_{n-1})$ for $n>1$.
By construction, the $B_n$ are disjoint and $\cup_{i=1}^n B_i =
\cup_{i=1}^n A_i$.  Furthermore $B_n \subset A_n$ so by monotonicity
of $\mu_0$ we have $\mu_0(B_n) \leq \mu_0(A_n)$.  Now have $A \subset
\cup_{n=1}^\infty B_n$ with $B_n$ disjoint, $B_n \in \mathcal{A}_0$
for all $n$  and $\sum_{n=1}^\infty
\mu_0(B_n) \leq \mu^*(A) + \epsilon$.  

Lastly we convert the disjoint cover $B_n$ into a partitioning of $A$.  
Consider $C_n = B_n \cap A$.  We still have $C_n \in
\mathcal{A}_0$, $C_n$ disjoint and montonicity implies $\sum_{n=1}^\infty
\mu_0(C_n) \leq \mu^*(A) + \epsilon$.  But now we have
$\cup_{n=1}^\infty C_n = A \in \mathcal{A}_0$ so we may apply
countable additivity of premeasure to conclude $\mu_0(A) = \sum_{n=1}^\infty
\mu_0(C_n) \leq \mu^*(A) + \epsilon$.  Once again, $\epsilon$ was
arbitrary so let it go to zero and we are done.
\end{proof}


TODO: construction that takes us from a semiring to a Boolean algebra.
It is often convenient to start a construction of a measure with a
collection of sets that is so small that it doesn't even form a
Boolean algebra.  For example when constructing Lebesgue measure on
$\reals$ we were really motivated by a desire that the measure of an
interval $(a,b]$ should be $b-a$, yet the set of such intervals on
$\reals$ is not a Boolean algebra.
\begin{defn}A set $\mathcal{D} \subset 2^\Omega$ is called a
  \emph{semiring} if 
\begin{itemize}
\item[(i)]$\emptyset \in \mathcal{D}$
\item[(ii)]if $A, B \in \mathcal{D}$ then $A \cap B \in \mathcal{D}$
\item[(iii)]if $A, B \in \mathcal{D}$ then there exist disjoint $C_1,
  \dotsc, C_n \in \mathcal{D}$ such that $A \setminus B = \cup_{j=1}^n
  C_j$
\end{itemize}
\end{defn}

\begin{examp}The set of intervals $(a,b]$ with $a \leq b$ is a
  semiring.  To be excruciatingly explicit we have the formulae
\begin{align*}
(a,b] \cap (c,d] &= (a\vee c, (b \wedge d) \vee a \vee c]
\intertext{and}
(a,b] \setminus (c,d] &= (a \wedge c, (a \vee c) \wedge b \wedge d]
\cup a \vee c \vee (b \wedge d), c \vee d]
\end{align*}
\end{examp}

TODO: Other constructions of semirings (e.g. products)

\begin{defn}A set $\mathcal{R} \subset 2^\Omega$ is called a
  \emph{ring} if 
\begin{itemize}
\item[(i)]$\emptyset \in \mathcal{R}$
\item[(ii)]if $A, B \in \mathcal{R}$ then $A \cup B \in \mathcal{R}$
\item[(iii)]if $A, B \in \mathcal{R}$ then $A \setminus B \in \mathcal{R}$
\end{itemize}
\end{defn}

\begin{lem}\label{RingFromSemiring}If $\mathcal{D}$ is a semiring then $\mathcal{R} = \lbrace
  \cup_{j=1}^n C_j \mid C_j \in \mathcal{D} \text{ and the $C_j$ are
    disjoint} \rbrace$ is a ring.  Furthermore it is the smallest ring
  containing $\mathcal{D}$.
\end{lem}
\begin{proof}
The fact that $\emptyset \in \mathcal{R}$ is immediate.  Suppose we
are given $\cup_{i=1}^n A_i$ and $\cup_{j=1}^m B_j$ in
$\mathcal{R}$. Then we have
\begin{align}
\left ( \cup_{i=1}^n A_i \right ) \cap \left ( \cup_{j=1}^m B_j \right
) &= \cup_{i=1}^n \cup_{j=1}^m A_i \cap B_j
\end{align}
which is in $\mathcal{R}$ because each $A_i \cap B_j \in \mathcal{D}$
and they are disjoint by the disjointness since each of $A_i$ and
$B_j$ is a disjoint set of sets.  

We also have 
\begin{align}
\left ( \cup_{i=1}^n A_i \right ) \setminus \left ( \cup_{j=1}^m B_j \right
) &=\left ( \cup_{i=1}^n A_i \right ) \cap \left ( \cup_{j=1}^m B_j \right
)^c \\
&=\cup_{i=1}^n \cap_{j=1}^m A_i \cap B_j^c \\
&=\cup_{i=1}^n \cap_{j=1}^m A_i \setminus B_j
\end{align}
and we know that each $A_i \setminus B_j \in \mathcal{D}$ and we know
that $\mathcal{D}$ is closed under finite intersections thus
$\cap_{j=1}^m A_i \setminus B_j \in \mathcal{D}$.  Furthermore by
disjointness of $A_i$ we have that $\cap_{j=1}^m A_i \setminus B_j$
are disjoint and therefore we have shown that $\left ( \cup_{i=1}^n A_i \right ) \setminus \left ( \cup_{j=1}^m B_j \right
) \in \mathcal{R}$.

To see that $\mathcal{R}$ is the smallest ring containing
$\mathcal{D}$ note simply that it is a ring and any ring containing
$\mathcal{D}$ must contain all of the finite disjoint unions of
elements in $\mathcal{D}$.
\end{proof}

\begin{examp}\label{RingOfDisjointUnionHalfOpenIntervals}The set of disjoint unions of intervals $(a,b]$ with $a \leq b$ is a
  ring.  This follows from the general result Lemma
  \ref{RingFromSemiring} but later on we shall have some use for the
  explicit formula
\begin{align*}
&(a,b] \cup (c,d] \\
&= (a \wedge c, (a \vee c) \wedge b \wedge d] \cup (a \vee c, (b
\wedge d) \vee a \vee c]
\cup a \vee c \vee (b \wedge d), c \vee d]
\end{align*}
which decomposes a union of half open intervals into a disjoint union
of half open intervals.
\end{examp}
 
To connect up the concept of rings with that of Boolean algebras we
have the following result.
\begin{lem}Let $\mathcal{R}$ be a ring and define $\mathcal{R}^c =
  \lbrace A^c \mid A \in \mathcal{R}\rbrace$.  Then $\mathcal{A} =
  \mathcal{R} \cup \mathcal{R}^c$ is a Boolean algebra and is the
  Boolean algebra generated by $\mathcal{R}$.  If $\mathcal{R}$ is a
  $\sigma$-ring then $\mathcal{R} \cup \mathcal{R}^c$ is the
  $\sigma$-algebra generated by $\mathcal{R}$.
\end{lem}
\begin{proof}Since Boolean algebras are closed under set complement it
  suffices to show that $\mathcal{A} = \mathcal{R} \cup \mathcal{R}^c$ is a Boolean
  algebra (respectively $\sigma$-algebra).  Closure under set
  complement is immediate from construction.  Closure under set
  intersection follows from handling the three possible cases
\begin{itemize}
\item[(i)]if $A, B \in \mathcal{R}$ then $A\cap B \in \mathcal{R}
  \subset \mathcal{A}$ since $\mathcal{R}$ is a ring.
\item[(ii)]if $A \in \mathcal{R}$ and $B \in \mathcal{R}^c$ then
  $A\cap B = A \cap (B^c)^c = A \setminus B^c \in \mathcal{R}
  \subset \mathcal{A}$ since $B^c \in \mathcal{R}$ and $\mathcal{R}$ is a ring.
\item[(iii)]if $A, B \in \mathcal{R}^c$ then $A \cap B = (A^c \cup
  B^c)^c \in \mathcal{R}^c
  \subset \mathcal{A}$ since $A^c, B^c \in \mathcal{R}$ and $\mathcal{R}$ is a ring.
\end{itemize}
Closure under finite set union follows as usual from De Morgan's Law.

Now if $\mathcal{R}$ is a $\sigma$-ring then 

TODO: Finish
\end{proof}

We have the following result for $\sigma$-rings that is analagous to
Lemma \ref{SigmaAlgebraPullback} proven for $\sigma$-algebras.
\begin{lem}\label{SigmaRingPullback}Given an arbitrary set function $f
  : S \to T$ and $\sigma$-rings $\mathcal{S}$ and $\mathcal{T}$ on
  $S$ and $T$ respectively 
\begin{itemize}
\item[(i)] $\mathcal{S}^\prime = f^{-1} \mathcal{T}$ is a
  $\sigma$-ring on $S$.
\item[(ii)] $\mathcal{T}^\prime = \left \{A \subset T ; f^{-1}(A) \in
      \mathcal{S} \right \}$ is a $\sigma$-ring on $T$.
\end{itemize}
\end{lem}
\begin{proof}
The proof of Lemma \ref{SigmaAlgebraPullback} shows closure under
countable union and intersection.  From these two facts, closure under
set difference follows by writing $B \setminus A = B \cap A^c$.
\end{proof}
 TODO: We have proven abstract Caratheodory construction in the
language of Boolean algebras; fill in a gap that shows that a
countably additive function on a ring actually defines a premeasure as
defined above.

\begin{lem}Let $\mu$ be an additive function on a semiring $\mathcal{D}$.  Let
  $\mu(\cup_{i=1}^n A_i) = \sum_{i=1}^n \mu(A_i)$ for any disjoint
  $A_1, \dotsc, A_n \in \mathcal{D}$.  Then $\mu$ is well defined and
  finitely additive on
  the ring $\mathcal{R}$ generated by $\mathcal{D}$.  If $\mu$ is
  countably additive on $\mathcal{D}$ then $\mu$ is countably additive
  on $\mathcal{R}$ and extends to a measure on $\sigma$-algebra
  generated by $\mathcal{D}$.
\end{lem}
\begin{proof}
TODO
\end{proof}

\subsection{Product Measures and Fubini's Theorem}

Prior to showing how to construct product measures, we need a
technical lemma.
\begin{lem}[Measurability of Sections]\label{MeasurableSections}Let $(S, \mathcal{S}, \mu)$ be a measure space with $\mu$ a
  $\sigma$-finite measure, let $(T, \mathcal{T})$ be a measurable
  space and $f : S \times T \to \reals_+$ be a positive $\mathcal{S} \otimes \mathcal{T} $-measurable
  function.  Then
\begin{itemize}
\item[(i)]$f(s,t)$ is an $\mathcal{S}$-measurable function of $s \in
  S$ for every fixed $t \in T$.
\item[(ii)] $\int f(s,t) \,  d \mu(s)$ is $\mathcal{T}$-measurable for
  as a function of $t \in T$.
\end{itemize}
\end{lem}
\begin{proof}
To see (i) and (ii),  let us first assume that $\mu$ is a bounded measure.  The proof uses
the standard machinery.  First assume that $f(s,t)
= \characteristic{B \times C}$ for $B \in \mathcal{S}$ and $C \in
\mathcal{T}$. Then note that for fixed $t \in T$, $f(s,t) = \characteristic{B}$ if $t
\in C$ and $f(s,t) = 0$ otherwise; in both cases we see that $f$ is
$\mathcal{S}$-measurable.  Also we calculate, $\int \characteristic{B
  \times C} (s,t) \,  d \mu(s) = \characteristic{C} (t)\int
\characteristic{B} (s) \,  d \mu(s) = \mu(B) \characteristic{C} (t) $
which clearly $\mathcal{T}$-measurable since $\mu(B) < \infty$.

Observe that the set of sets $B \times C$ is a $\pi$-system.  Let
\begin{align*}
\mathcal{H} = \lbrace A \in \mathcal{S} \otimes \mathcal{T} \mid
\characteristic{A}(s,t) \text{ is $\mathcal{S}$-measurable for every
  fixed $t \in T$ and $\int \characteristic{A}(s,t) \, d\mu(s)$ is $\mathcal{T}$-measurable } \rbrace
\end{align*}
and we claim that $\mathcal{H}$ is a $\lambda$-system.  Clearly $S
\times T \in \mathcal{H}$ from what we have already shown.  Suppose
next that $A \subset B$ are both in $\mathcal{H}$.  Note that
$\characteristic{B \setminus A} = \characteristic{B} -
\characteristic{A}$ so each section is a difference of
$\mathcal{S}$-measurable functions hence $\mathcal{S}$-measurable.
Similarly, 
\begin{align*}
\int \characteristic{B \setminus A} (s,t) \, d\mu(s) &= \int
\characteristic{B} (s,t) \, d\mu(s) - \int
\characteristic{A} (s,t) \, d\mu(s) 
\end{align*}
is a difference of $\mathcal{T}$-measurable function hence
$\mathcal{T}$-measurable.

Lastly, suppose that $A_1 \subset A_2 \subset \cdots \in
\mathcal{H}$.  Then $\characteristic{A_i} \uparrow \characteristic{\bigcup A_i}$ and this statement is true when considering each
function as a function on $S \times T$ but also for every
section with fixed $t \in T$.  Hence every section is a increasing limit of $\mathcal{S}$-measurable
functions and therefore $\mathcal{S}$-measurable.  Also we can apply
Montone Convergence Theorem to see that 
\begin{align*}
\int \characteristic{\bigcup A_i}(s,t) \, d\mu(s) = \lim_{n \to
  \infty} \int \characteristic{A_i}(s,t) \, d\mu(s)
\end{align*}
which shows $\mathcal{T}$-measurability.
Now the
$\pi$-$\lambda$ Theorem shows that $\mathcal{H} =  \mathcal{S} \otimes
\mathcal{T}$ and we have the result for all indicators.  

Next, linearity of taking sections and integrals shows that all simple functions
also satisfy the theorem.  Lastly for a general positive $f(s,t)$ we
take an increasing sequence of simple functions $f_n \uparrow f$.
Again, the limit is taken pointwise so every section of $f$ is the
limit of the sections of $f_n$ each of which has been shown
$\mathcal{S}$-measurable.  As the limit of $\mathcal{S}$-measurable
functions, we see that every section $f$ is also
$\mathcal{S}$-measurable.  Since for a fixed $t \in T$, $f_n(s,t)$ is
increasing as a function of $s$ alone we apply the Monotone
Convergence Theorem to see that
\begin{align*}
\int f(s,t) \, d\mu(s) = \lim_{n \to \infty} \int f_n(s,t) \, d\mu(s)
\end{align*}
which shows $\mathcal{T}$-measurability of $\int f(s,t) \, d\mu(s)$
since it is a limit of $\mathcal{T}$-measurable functions.

Now let $\mu$ be a $\sigma$-finite measure on $S$.  Then there is a
disjoint partition $S_1, S_2, \dots$ of $S$ such that $\mu S_n <
\infty$.  Thus, $\mu_n (A) = \mu(A \cap S_n)$ defines a bounded
measure and we know from Lemma \ref{ChainRuleDensity} that for any
measurable $g$, $\int g \, d\mu_n = \int g
\characteristic{S_n} \, d\mu$.
Putting these observations together,
\begin{align*}
\int f(s,t) \,  d \mu(s) &= \int f(s,t) \sum_{n=1}^\infty
\characteristic{S_n}(s) \,  d \mu(s) & & \text{since $S_n$ is a partition
  of $S$} \\
&= \sum_{n=1}^\infty \int f(s,t) 
\characteristic{S_n}(s) \,  d \mu(s) & & \text{by Corollary 
  \ref{TonelliIntegralSum} } \\
&= \sum_{n=1}^\infty \int f(s,t) \,  d \mu_n(s) 
\end{align*}
Since each $\mu_n$ is bounded, we have proven that each $\int f(s,t) \,  d \mu_n(s) $ is
$\mathcal{T}$-measurable hence the same is true for the partial sums
by linearity and then the infinite sum by taking a limit.
\end{proof}

TODO: Come up with an example of a non-measurable function for which all sections are measurable.

\begin{thm}[Fubini-Tonelli Theorem]\label{Fubini}Let $(S, \mathcal{S}, \mu)$
  and $(T, \mathcal{T}, \nu)$ be two $\sigma$-finite measure spaces.
  There exists a unique measure $\mu \otimes \nu$ on $(S \times T,
  \mathcal{S} \otimes \mathcal{T})$ satisfying 
\begin{align*}
(\mu \otimes \nu)(B \times C) &= \mu B \cdot \nu C & &\text{ for all
  $B \in \mathcal{S}$, $C \in \mathcal{T}$.}
\end{align*}
In addition if $f : S \times T \to \reals_+$ is a positive measurable
function then 
\begin{align*}
\int f(s,t) \,  d (\mu \otimes \nu) = \int \left [ \int f(s,t) \, d
\nu(t) \right ] d \mu(s)  = \int \left [ \int f(s,t) \, d \mu(s)
\right ] d\nu(t)
\end{align*}
This last sequence of equalities also holds if $f : S \times T \to \reals$
is measurable and integrable with respect to $\mu \otimes \nu$.
\end{thm}
\begin{proof}Note that the class of sets of the form $A \times B$ for
  $A \in \mathcal{S}$ and $B \in \mathcal{T}$ is clearly a
  $\pi$-system and generates $\mathcal{S} \otimes \mathcal{T}$ by
  definition of the product $\sigma$-algebra.  Furthermore by
  $\sigma$-finiteness of both $\mu$ and $\nu$ we can construct a
  disjoint partition $S \times T = \cup_i \cup_j S_i \times T_j$ with
  $\mu(S_i)\nu(T_j) < \infty$.  Therefore we can apply Lemma
  \ref{UniquenessOfMeasure} to see that the property $(\mu \otimes
  \nu)(A \times B) = \mu(A) \nu(B)$ uniquely determines $\mu \otimes
  \nu$.

To show existence of such a measure, define 
\begin{align*}
(\mu \otimes \nu)(A) = \int \left [\int \characteristic{A}(s,t) \, d
  \nu(t) \right ] d \mu(s)
\end{align*}
The fact that the iterated integrals are well defined follows from
Lemma \ref{MeasurableSections}.  
To see that it is a measure, first
note that it is simple to see $(\mu \otimes \nu)(\emptyset) = 0$.

To prove countable additivity, suppose we are given disjoint $A_1,
A_2, \dots \in \mathcal{S}
\otimes \mathcal{T}$.  By disjointness, we know
$\characteristic{\bigcup_{i=1}^\infty A_i} = \sum_{i=1}^\infty \characteristic{A_i}$.
Now because indicator functions and the inner integrals are positive, we can interchange
integrals and sums twice (Corollary \ref{TonelliIntegralSum}) and get
\begin{align*}
(\mu \otimes \nu)(\bigcup_{i=1}^\infty A_i) &= \int \left [ \int
\characteristic{\bigcup_{i=1}^\infty A_i}(s,t) \, d \nu(t) \right ] d \mu(s)\\
&= \int \left [ \int \sum_{i=1}^\infty \characteristic{A_i}(s,t) \, d
  \nu(t) \right ]  d \mu(s)\\
&= \sum_{i=1}^\infty \int \left [ \int \characteristic{A_i}(s,t) \, d
  \nu(t) \right ] d \mu(s)\\
\end{align*}

It is also clear that for $A = B \times C$ with $B \in \mathcal{S}$
and $C \in \mathcal{T}$, 
\begin{align*}
(\mu \otimes \nu)(B \times C) &= \int \left [ \int \characteristic{B}(s)
\characteristic{C}(t) \, d\nu(t) \right ] d \mu(s)\\
&= \int \characteristic{B}(s) \, d \mu(s) \cdot \int 
\characteristic{C}(t) \, d\nu(t) \\
&= \mu B \cdot \nu C
\end{align*}
Therefore we have proven the existence of the product measure.

The argument proving existence of the product measure applies equally well if we reverse the order
of $\mu$ and $\nu$ and shows that 
\begin{align*}
(\mu \otimes \nu)(B \times C) = \int \left [ \int \characteristic{B
    \times C}(s,t) \, d
\nu(t) \right ] d \mu(s) = \int \left [ \int \characteristic{B \times C}(s,t)
\, d \mu(s) \right ] d \nu(t)
\end{align*}
which proves that the integrals are equal for indicator functions of
sets of the form $B \times C$ and therefore for all indicator
functions by the montone class argument we used at the beginning of
the proof.  At this point, the
standard machinery can be deployed.  Linearity of integrals easily
shows that the equality extends to simple functions.  Lastly suppose
we have a positive measurable function $f(s,t) : S \times T \to
\overline{R}_+$ with a sequence of positive simple functions
$f_n(s,t) \uparrow f(s,t)$.  By the Monotone Convergence Theorem and
monotonicity of integral we know that 
\begin{align*}
0 &\leq \int f_n(s,t) \, d\mu(s) \uparrow \int f(s,t) \, d\mu(s) \\
0 &\leq \int f_n(s,t) \, d\nu(t) \uparrow \int f(s,t) \, d\nu(t) \\
\end{align*}
and therefore we have
\begin{align*}
\int f(s,t) \,  d (\mu \otimes \nu) 
&= \lim_{n \to \infty} \int
 f_n(s,t) \,  d (\mu \otimes \nu)  & & \text{by definition of integral
   of $f$}\\
&= \lim_{n \to \infty} \int \left [
\int f_n(s,t) \,  d \mu(s) \right ] d\nu(t) & &\text{by Tonelli
for simple functions}\\
&= \int \left [
\int f(s,t) \,  d \mu(s) \right ] d\nu(t) & &\text{by Monotone Convergence
   on $\int f_n \, d \mu(s)$}\\
\end{align*}

It is worth pointing out explicitly that even if $f(s,t)$ is never
equal to infinity, the integrals may be equal to infinity on all of
$S$ or $T$ and it is critical that we have phrased the theory of
integration for positive functions in terms of functions with values
in $\overline{\reals}_+$.

TODO: Clean up the following argument; it has all right details but is
more than a bit ragged.  Particularly annoying is that this is the
first time we've talked about defining integrals for signed functions
that take infinite values on a set of measure zero (ACTUALLY I HAVE ADRESSED THIS)

Now assume that $f$ is integrable with respect to $\mu \otimes \nu$: 
$\int \abs{f(s,t)} \,  d (\mu \otimes \nu) < \infty$.  We
write $f = f_+ - f_-$ and note that 
\begin{align*}
\int f_\pm(s,t) \,  d (\mu \otimes \nu) &\leq \int \abs{f(s,t)} \,  d (\mu \otimes \nu)< \infty
\end{align*} 
and use Tonelli's Theorem just proven to see that 
\begin{align*}
\int f_\pm(s,t) \,  d (\mu \otimes \nu) &= \int \left [ \int f_\pm(s,t) \, d
\nu(t) \right ] d \mu(s)  = \int \left [ \int f_\pm(s,t) \, d \mu(s)
\right ] d\nu(t) < \infty
\end{align*}
The finiteness of the iterated integrals implies that the integrands
$\int f_\pm \, d\mu(s)$ and $\int f_\pm \, d\nu(t)$ are finite almost everywhere.  
The trick is that being finite almost everywhere isn't good
enough when trying to calculate the iterated integrals of $f$ and we
might run into the awkward situation in which there is a $t \in T$
such that \emph{both} $\int f_+ \, d\mu(s)$ and $\int f_- \, d\mu(s)$
are infinite.  However define
$N_S = \{ s \in S \mid \int \abs{f} \, d\nu(t) = \infty \}$ and $N_T =
\{ t \in T \mid \int \abs{f} \, d\mu(s) = \infty \}$.  We have noted
that $N_S$ is a $\mu$-null set and that $N_T$ is a $\nu$-null set
hence $N_S \times N_T$ is a $(\mu \otimes \nu)$-null set.  We modify
$f$ so that it is zero on $N_S \times N_T$ by defining 
$\tilde{f}(s,t) = (1 - \characteristic{N_S \times N_T}) f(s,t)$.  By Proposition \ref{IntegrableAlmostEverywhereEqualIntegralEqual}
we have the following equalities
\begin{align*}
\int \tilde{f} d(\mu\otimes \nu) &= \int f d(\mu\otimes \nu) \\
\int \tilde{f} d\mu(s) &= \begin{cases}
\int f d\mu(s) & \text{if $t \notin N_T$} \\ 
0 & \text{if $t \in N_T$} \\ 
\end{cases} \\
\int \tilde{f} d\nu(t) &= \begin{cases}
\int f d\nu(t) & \text{if $s \notin N_S$} \\ 
0 & \text{if $s \in N_S$} \\ 
\end{cases}
\end{align*}
Now we can write $\tilde{f} = \tilde{f}_+ - \tilde{f}_-$ and apply
Tonelli's Theorem to see
\begin{align*}
\int \tilde{f} d(\mu\otimes \nu) &= \int \tilde{f}_+ d(\mu\otimes \nu)
- \int \tilde{f}_- d(\mu\otimes \nu) \\
&= \int \left [ \int \tilde{f}_+ d\mu(s)\right ] d \nu(t) - \int \left
  [ \int \tilde{f}_- d\mu(s)\right ] d \nu(t) \\
&= \int \left [  \int \tilde{f}_+ d\mu(s) - \int \tilde{f}_-
d\mu(s) \right ] d \nu(t) \\
&= \int \left [ \int \tilde{f} d\mu(s)\right ] d \nu(t) \\
\end{align*}

But we know $\int \left [ \int \tilde{f} d\mu(s)\right ] d \nu(t) =
\int \left [ \int f d\mu(s)\right ] d \nu(t)$ so we get the result for
$f$ as well.
\end{proof}

TODO: Royden has some exercises that demonstrate how each of these
hypotheses is necessary (e.g. Counterexample to Fubini for
non-integrable f).  Incorporate them.
\begin{examp}Define the measure space $(\naturals, 2^\naturals,
  \mu)$ where $\mu(A) = \card{A}$.  $\mu$ is called the \emph{counting
    measure}.  
Consider the function 
\begin{align*}
f(s,t) &= \begin{cases}
2 - 2^{-s+1} & \text{if $s=t$}\\
-2 + 2^{-s+1} & \text{if $s = t + 1$} \\
0 & \text{otherwise}
\end{cases}
\end{align*}
on $(\naturals \times \naturals, 2^{\naturals \times \naturals}, \mu
\otimes \mu)$.  Since $\mu \otimes \mu$ is the counting measure on
$\naturals \times \naturals$ it is easy to see that 
\begin{align*}
\int \abs{f(s,t)} \, d ( \mu \otimes \mu) &= \sum_{s=1}^\infty
\sum_{t=1}^\infty \abs{f(s,t)} = \infty
\end{align*}
so $f$ is not integrable.  However in this case both of the iterated
integrals are defined.
For fixed $t$, 
\begin{align*}
\int f(s,t) \, d\mu(s) &=
\sum_{s=1}^\infty f(s,t) = 2^{-t} - 2^{-t+1} = -2^{-t}
\end{align*}
hence 
\begin{align*}
\int \left [ \int f(s,t) \, d\mu(s) \right ] d \mu(t) &=
\sum_{t=1}^\infty -2^{-t} = -1
\end{align*}

For fixed $s$, 
\begin{align*}
\int f(s,t) \, d\mu(s) &=
\sum_{t=1}^\infty f(s,t) = \begin{cases}
1 & \text {if $s=1$}\\
0 & \text {otherwise}
\end{cases}
\end{align*}
and therefore 
\begin{align*}
\int \left [ \int f(s,t) \, d\mu(t) \right ] d \mu(s) &= 1
\end{align*}
This example shows that the positivity of $f$ is a necessary condition
in Tonelli's Theorem and that the assumption of integrability is
necessary in Fubini's Theorem.
\end{examp}


TODO
Outer measures, Caratheodory construction, Lesbegue Measure (existence
and uniqueness), Product Measures and Fubini's Theorem, Radon-Nikodym Theorem and
Fundamental Theorem of Calculus, Differential Change of Variables for
Lebesgue Measure on $\reals^n$ (useful for calculations involving
probability densities).

\begin{lem}[Translation Invariance of Lebesgue Measure]\label{LesbegueTranslationInvariance} Suppose $\mu$ is a measure on $\reals^n$ which is
  translation invariant and for which $\mu([0,1]^n) = 1$, then $\mu =
  \lambda^n$.
\end{lem}
\begin{proof}Suppose we are given a translation invariant measure
  $\mu$ such that $\mu([0,1]^n) = 1$.  By writing boxes as a
  union of cubes and using finite
  and countable additivity together with translation invariance it is
  easy to see that
  for any box $\mathcal{I}_1 \times \cdots \times \mathcal{I}_n$ where
  each $\mathcal{I}_k$ has rational endpoints that we have 
\begin{align*}
\mu\left (\mathcal{I}_1 \times \cdots \times
    \mathcal{I}_n \right ) &= \abs{\mathcal{I}_1} \cdots
  \abs{\mathcal{I}_n} \\
 &= \lambda^n \left (\mathcal{I}_1 \times \cdots \times
    \mathcal{I}_n \right )
\end{align*}
Now fix $\mathcal{I}_2, \dots ,\mathcal{I}_n$ and consider $\nu(A) = \frac{1}{ \abs{\mathcal{I}_2} \cdots  \abs{\mathcal{I}_n}}\mu \left
  (A \times \mathcal{I}_2 \times \cdots \times
    \mathcal{I}_n \right ) $ as a function of $A \in
  \mathcal{B}(\reals)$.  It is easy to see that this is a Borel
  measure and we have already seen that $\nu(\mathcal{I}) =
  \abs{\mathcal{I}}$ for all rational intervals (hence all intervals
    by countable additivity).  Therefore $\nu = \lambda$ is Lebesgue
    measure on $\mathcal{B}(\reals)$ and we have for every $B_1 \in
    \mathcal{B}(\reals)$,
\begin{align*}
\mu\left (B_1 \times \mathcal{I}_2 \times\cdots \times
    \mathcal{I}_n \right ) 
 &= \lambda^n \left (B_1 \times \mathcal{I}_2 \times \cdots \times
    \mathcal{I}_n \right )
\end{align*}
Now iterate the argument $2, \cdots, n$ fixing all but the $i^{th}$
argument to extend to all cylinder sets $B_1 \times \cdots \times B_n$
and we apply the uniqueness of product measures.

Now it remains to show that $\lambda^d$ is indeed translation
invariant.
TODO
\end{proof}
\begin{cor}\label{LesbegueRotationInvariance} Lebesgue measure $\lambda^n$ on $\reals^n$ is invariant
  under orthogonal transformations.
\end{cor}
\begin{proof}Suppose we are given an orthogonal transformation $P$.
  We claim that the measure $\lambda^n_P(A) = \lambda^a(P A)$ is
  translation invariant.   To see this, assume we are given $h \in
  \reals^n$ and note that 
\begin{align*}
\lambda^n_P(A + h) &= \lambda^n(P A + Ph)  & &\text{linearity of $P$} \\
&= \lambda^n(PA) & &\text{translation invariance of $\lambda^n$} \\
&= \lambda^n_P(A) & &\text{definition of $\lambda^n_P$}
\end{align*}
Therefore we know that $\lambda^n_P = c \lambda^n$ for some constant
$c>0$.  Take the unit ball $B^n \subset \reals^n$ and notice that $P
B^n = B^n$ to see that in fact $c = 1$.
\end{proof}
\begin{cor}\label{LebesgueLinearChangeOfVariables}[Linear Change of Variables]For an arbitrary linear transformation $T : \reals^n \to
  \reals^n$, $\lambda^n(T A) = \abs{\det{T}} \lambda^n(A)$ for all
  measurable $A$.
\end{cor}
\begin{proof}Note that by the Singular Value Decompostion, we can
  write $T = U D V$ with $U,V$ orthogonal.  By the rotation invariance
  of $\lambda^n$, we are reduced to the case of a diagonal matrix.  In
  that case, the result is easy.
TODO write down the easy stuff too!
\end{proof}

\subsection{Further Properties of Outer Measures}

The results in this section will not be used until much later in the text.  There will be little harm in the reader skipping these results and coming back to them when they are referenced.

One of the fundamental properties of countably additive measures is continuity (Lemma \ref{ContinuityOfMeasure}).  It turns out that
by adding a hypothesis we get a continuity property for outer measures as well.

\begin{defn}An outer measure $\mu^*$ is said to be \emph{regular} if for each set $A$ there exists a $\mu^*$-measurable set $B$ such that
$A \subset B$ and $\mu^*(A) = \mu^*(B)$.  Any such $B$ is called a \emph{measurable cover} of $A$ (TODO: Compare with Van der Vaart and Wellner).
\end{defn}

Examples of regular outer measures aren't too hard to come up with.
Indeed, any outer measure that is constructed from a premeasure is regular.
\begin{prop}\label{InducedOuterMeasuresAreRegular}If an outer measure $\mu^*$ on a set $\Omega$ is induced from a finitely additive non-negative set function $\mu_0$ on a Boolean algebra $\mathcal{A} \subset 2^\Omega$ then $\mu^*$ is regular.  If in addition $\mu_0$ is countably additive and there exists $\Omega_1, \Omega_2, \dotsc \in \mathcal{A}$ with $\Omega = \cup_{n=1}^\infty \Omega_n$ and $\mu_0(\Omega_n) < \infty$ for all $n \in \naturals$ then a set $A$ is $\mu^*$-measurable if and only if there exists $B \in \mathcal{A}_{\delta \sigma}$ and a $\mu^*$-null set $N$ such that $B \cap N = \emptyset$ and $A = B \cup N$.
\end{prop}
\begin{proof}
First assume only that $\mu_0$ is finitely additive on disjoint sets.
We establish a sequence of claims from which the result follows.
\begin{clm}$\mathcal{A}_\sigma$ is closed under finite intersections
\end{clm}
Let $A_1, A_2, \dotsc \in \mathcal{A}$ and $B_1, B_2, \dotsc \in \mathcal{A}$, let $A = \cup_{n=1}^\infty A_n$ and $B=\cup_{m=1}^\infty B_m$ and compute
\begin{align*}
A \cap B  &=\cup_{n=1}^\infty  \left ( A_n  \cap \left ( \cup_{m=1}^\infty B_m  \right ) \right )\\
&=\cup_{n=1}^\infty \cup_{m=1}^\infty A_n  \cap B_m 
\end{align*}
Since $\mathcal{A}$ is a Boolean algebra it follows that $A_n \cap B_m \in \mathcal{A}$ for every $m,n$ and therefore $A \cap B \in \mathcal{A}_\sigma$.  The claim follows by a simple induction.

\begin{clm}For any set $A \subset \Omega$ we have $\mu^*(A) = \inf \lbrace \mu^*(B) \mid B \in \mathcal{A}_\sigma, A \subset B \rbrace$
\end{clm}
If $\mu^*(A) = \infty$ then for any $A \subset B$ we have $\mu^*(B)$ so the result follows.  Suppose that $\mu^*(A) < \infty$ and let $\epsilon > 0$ be given.  By definition of $\mu^*$ there exist $B_1, B_2, \dotsc \in \mathcal{A}$ such that $\mu^*(A) \leq \sum_{n=1}^\infty \mu_0(B_n) < \mu^*(A) + \epsilon$.  Let $B = \cup_{n=1}^\infty B_n$ and note that by subadditivity and monotonicity of $\mu^*$ and the fact that $\mu^* \leq \mu_0$ on $\mathcal{A}$ we have  
\begin{align*}
\mu^*(A) &\leq \mu^*(B) \leq \sum_{n=1}^\infty \mu^*(B_n) = \sum_{n=1}^\infty \mu_0(B_n) < \mu^*(A) + \epsilon
\end{align*}  
Since $\epsilon$ was arbitrary the claim follows.

\begin{clm}For any set $A \subset \Omega$ there is a $B \in \mathcal{A}_{\sigma \delta}$ such that $A \subset B$ and $\mu^*(A) = \mu^*(B)$.
\end{clm}
From the previous claim we may pick $C_1, C_2, \dotsc \in \mathcal{A}_\sigma$ such that $A \subset C_n$ for all $n \in \naturals$ and $\lim_{n \to \infty} \mu^*(C_n) = \mu^*(A)$.  Define $B_n = \cap_{j=1}^n C_j$.  Since $\mathcal{A}_\sigma$ is closed under finite intersections $B_n \in \mathcal{A}_\sigma$ for every $n \in \naturals$.  Also $A \subset B_n \subset C_n$ and therefore 
$\lim_{n \to \infty} \mu^*(B_n) = \mu^*(A)$.  Let $B = \cap_{n =1}^\infty B_n \in \mathcal{A}_{\sigma \delta}$.   Since the $B_n$ are decreasing we have $A \subset B \subset B_n$ for every $n \in \naturals$ and thus by monotonicity and subadditivity of $\mu^*$ we get
\begin{align*}
\mu^*(A) &\leq \mu^*(B) \leq \lim_{n \to \infty}\mu^*(B_n) = \mu^*(A)
\end{align*}
and $\mu^*(A)=\mu^*(B)$ follows.

By Lemma \ref{PremeasureBooleanAlgebraOuterMeasurable} we know that every $A \in \mathcal{A}$ is $\mu^*$-measurable and therefore every $A \in \mathcal{A}_{\sigma \delta}$ is $\mu^*$-measurable.  The regularity of $\mu^*$ follows from the previous claim.

Now assume that $\mu_0$ is countably additive and let $\Omega_1, \Omega_2, \dotsc \in \mathcal{A}$ be chosen so that $\Omega = \cup_{n=1}^\infty \Omega_n$ and $\mu_0(\Omega_n) < \infty$.  By Lemma \ref{PremeasureOuterMeasureEqual} we have $\mu_0 = \mu^*$ on $\mathcal{A}$.  In particular $\mu^*(\Omega_n) = \mu_0(\Omega_n)< \infty$ for all $n \in \naturals$.
Note that by passing to $\Omega_n \setminus (\Omega_1 \cup \dotsb \cup \Omega_{n-1})$ we may assume that the $\Omega_n$ are disjoint.  Let $A$ be a $\mu^*$-measurable set.  For each $n \in \naturals$ we apply the last claim to the measurable set $A^c \cap \Omega_n$ and get a $B^\prime_n \in \mathcal{A}_{\sigma \delta}$ such that $A^c \cap \Omega_n \cup B^\prime_n$ and 
$\mu^*(B^\prime_n) = \mu^*(A^c \cap \Omega_n)$.  Let $B_n = B^\prime_n \cap \Omega_n$ and note that $B_n \in \mathcal{A}_{\sigma \delta}$, $A^c \cap \Omega_n \subset B_n \subset \Omega_n$
and $\mu^*(B_n) = \mu^*(A^c \cap \Omega_n)$.  Let $N_n = B_n \setminus (A^c \cap \Omega_n)$ so that $\mu^*(N_n) = 0$.  Then
\begin{align*}
\Omega_n \cap A &= \Omega_n \setminus A^c = \Omega_n \setminus \left(\Omega_n \cap A^c \right) \\
&=\Omega_n \setminus \left( B_n \setminus N_n \right) = \Omega_n \cap \left( B_n \cap N_n^c \right)^c = \left(\Omega_n \cap B_n^c  \right) \cup \left( \Omega_n \cap N_n \right) \\
&= \Omega_n \setminus B_n \cup N_n
\end{align*}
Since $B_n \in \mathcal{A}_{\sigma \delta}$ it follows that $\Omega_n \setminus B_n \in \mathcal{A}_{\delta \sigma}$.  Now write
\begin{align*}
A &= \cup_{n=1}^\infty \Omega_n \cap A  = \cup_{n=1}^\infty \Omega_n \setminus B_n \cup N_n
\end{align*}
and use the fact that a countable union of sets in $\mathcal{A}_{\delta \sigma}$ is in $\mathcal{A}_{\delta \sigma}$ and a countable union of null sets is null.
\end{proof}

\begin{prop}\label{InducedOuterMeasureFromRegularOuterMeasure}Let $\mu^*$ be an outer measure on a set $\Omega$ and let $\mu^+$ be the outer measure induced by $\mu^*$ and the 
$\sigma$-algebra of $\mu^*$-measurable sets.  Then $\mu^* = \mu^+$ if and only if $\mu^*$ is regular.
\end{prop}
\begin{proof}
First note that for an arbitrary outer measure $\mu^*$ we have $\mu^*(A) \leq \mu^+(A)$, for if $A_1, A_2, \dotsc$ is a countable cover of $A$ then by countable subadditivity of $\mu^*$ we have
$\mu^*(A) \leq \sum_{i=1}^\infty \mu^*(A_j)$.  Now take the infimum of the right hand side over all countable subcovers with $\mu^*$-measurable $A_j$.

Suppose $\mu^*$ is regular. If $A$ is a set and $B$ is its measurable cover we have $\mu^+(A) \leq \mu^*(B) = \mu^*(A)$ as well and it follows that $\mu^*=\mu^+$.

On the other hand suppose that $\mu^* = \mu^+$ and let $A$ be a subset of $\Omega$.  If $\mu^*(A) = \infty$ then $\Omega$ is measurable cover so we may assume that $\mu*(A) < \infty$.  
For each $n \in \naturals$ we may pick $\mu^*$-measurable $C_1, C_2, $ such that $A \subset \cup_{j=1}^\infty C_j$ and $\sum_{j=1}^\infty \mu^*(C_j) < \mu^*(A) + 1/n$.  Since the $\mu^*$-measurable sets are a $\sigma$-algebra the set $B_n = \cup_{j=1}^\infty C_j$ is $\mu^*$-measurable and by subadditivity of $\mu^*$
\begin{align*}
\mu^*(B_n) &\leq \sum_{j=1}^\infty \mu^*(C_j) < \mu^*(A) + 1/n
\end{align*}
Now let $B = \cap_{n=1}^\infty B_n$ which is $\mu^*$-measurable and by continuity of measure $\mu^*(B) \leq \lim_{n \to \infty} \mu^*(A) + 1/n = \mu^*(A)$.  On the other hand $A \subset B$ and
therefore by monotonicity we get $\mu^*(A) = \mu^*(B)$.
\end{proof}

TODO: Don't know if this result will be useful; maybe make an exercise (I got it from Biskup's notes)
\begin{prop}Let $\mu^*$ be a finite regular outer measure on $\Omega$ then $A$ is $\mu^*$-measurable if  and only if 
\begin{align*}
\mu^*(\Omega) &= \mu^*(A) + \mu^*(A^c)
\end{align*}
\end{prop}
\begin{proof}
If $A$ is $\mu^*$-measurable then the result follows from the definition of $\mu^*$-measurability applied to the set $\Omega$ so only the other direction requires proof.  

Let $E$ be a subset of $\Omega$ and let $B$ be a measurable cover of $E$.
By measurability of $B$ we have
\begin{align*}
\mu^*(A) &= \mu^*(A \cap B) + \mu^*(A \cap B^c) \\
\mu^*(A^c) &= \mu^*(A^c \cap B) + \mu^*(A^c \cap B^c) \\
\mu^*(\Omega) &= \mu^*(B) + \mu^*(B^c)\\
\end{align*}
From these three facts we compute using subadditivity and monotonicity of $\mu^*$ to get
\begin{align*}
\mu^*(E) &= \mu^*(B) = \mu^*(\Omega) - \mu^*(B^c) =\mu^*(A) + \mu^*(A^c) - \mu^*(B^c) \\
&=\mu^*(A \cap B) + \mu^*(A \cap B^c)  + \mu^*(A^c \cap B) + \mu^*(A^c \cap B^c) - \mu^*(B^c) \\
&\geq \mu^*(A \cap B) + \mu^*(A^c \cap  B)  \\
&\geq \mu^*(A \cap B) + \mu^*(A^c \setminus E)
\end{align*}
Since the opposite inequality follows by subadditivity, measurability of $A$ follows.
\end{proof}

TODO: Show that an outer measure induced by a premeasure is regular (or more generally an outer measure constructed as in Proposition \ref{PremeasureToOuterMeasure}?)

\begin{prop}\label{ContinuityOfRegularOuterMeasure}Let $\mu^*$ be a regular outer measure on a set $\Omega$ and let $A, A_1,A_2, \dotsc$ be subsets of $\Omega$ then
if $A_i \uparrow A$ then $\mu^* A_i \uparrow \mu^* A$.
\end{prop}
\begin{proof}
Pick $\mu^*$-measurable sets $C_k$ such that $A_k \subset C_k$ and $\mu^* A_k = \mu^* C_k$ and define $B_k = \cap_{j=k}^\infty C_j$.  Clearly, $B_k$ is $\mu^*$-measurable and $A_k \subset B_k$.  Also since $A_k \subset B_k$ we have $\mu^* A_k = \mu^* C_k \leq \mu^* B_k$ and since $B_k \subset C_k$ we have $\mu^* B_k \leq \mu^* C_k$; it follows that $\mu^* A_k = \mu^* B_k$ as well.Since $B_1 \subset B_2 \subset \dotsb$ and $\cup_{k=1}^\infty A_k \subset \cup_{k=1}^\infty B_k$ we can use continuity of measure Lemma \ref{ContinuityOfMeasure} and monotonicity of $\mu^*$ to see
\begin{align*}
\lim_{k \to \infty} \mu^* A_k &= \lim_{k \to \infty} \mu^* B_k = \mu^* \cup_{k=1}^\infty B_k \geq \mu^* \cup_{k=1}^\infty A_k
\end{align*}
On the other hand since $A_k \subset \cup_{j=1}^\infty A_j$ we have
\begin{align*}
\lim_{k \to \infty} \mu^* A_k \leq \mu^* \cup_{k=1}^\infty A_k
\end{align*}
\end{proof}

The following technical Lemma is useful (we'll use it when
discussing Hausdorff outer measures).  If the reader is in a hurry,
no harm will come from skipping over this result and returning to it
when the need arises.  Note that if the user is only interested in
probability theory this result may never come up.

\begin{defn}Let $(S,d)$ be a metric space with an outer measure $\mu^*$ then we say that
$\mu^*$ is a \emph{metric outer measure} if and only if $\mu^*(A \cup B) = \mu^*(A) +
  \mu^*(B)$ for all $A,B$ such that $d(A, B) > 0$.
\end{defn}

TODO: Compare with Biskup's notes
\begin{lem}[Caratheodory Criterion]\label{CaratheodoryCriterion}Let $(S,d)$ be a metric space with an outer measure $\mu^*$.
  Then $\mu^*$ is a Borel outer measure (i.e. all Borel sets are
  $\mu^*$-measurable) if and only if $\mu^*$ is metric.
\end{lem}
\begin{proof}
We begin with the only if direction.  Let $A$ be a closed set in $S$
and let $B \subset S$.  To show $A$ is $\mu^*$-measurable it suffices
to show $\mu^*(B) \geq \mu^*(A \cap B) + \mu^*(A^c \cap B)$.  Since
the inequality is trivially satisfied when $\mu^*(B) = \infty$ we
assume that $\mu^*(B) < \infty$.  For
every $n \in \naturals$, let $A_n =
\lbrace x \in S \mid d(x, A) \leq \frac{1}{n} \rbrace$.  By definition
of $A_n$, we have $d(A,
A_n^c) > \frac{1}{n} > 0$ and therefore $d(A \cap B, A_n^c \cap B)
> \frac{1}{n} > 0$.  Now by our assumption, we can conclude $\mu^*((A
\cap B) \cup (A_n^c \cap B)) = \mu^*(A \cap B) +
\mu^*(A_n^c \cap B)$.

We claim that $\lim_{n \to \infty} \mu^*(A_n^c \cap B) = \mu^*(A^c
\cap B)$.  Note that if we prove the claim the Lemma is proven because then we have
\begin{align*}
\mu^*(B) &\geq \mu^*((A
\cap B) \cup (A_n^c \cap B)) & & \text{by monotonicity}\\
&= \mu^*(A \cap B) +
\mu^*(A_n^c \cap B)
\end{align*}
and taking limits we have 
\begin{align*}
\mu^*(B) \geq \lim_{n\to \infty} \mu^*(A \cap B) +
\mu^*(A_n^c \cap B) &= \mu^*(A \cap B) +
\mu^*(A^c \cap B)
\end{align*}
To prove the claim we observe that monotonicity of outer measure
implies that $\lim_{n \to \infty} \mu^*(A_n^c \cap B) \leq \mu^*(A^c
\cap B)$ so we just need to
work on the opposite inequality.  To see it first define the rings
around $A$
\begin{align*}
R_n &= \lbrace x \mid \frac{1}{n+1} < d(x, A) \leq \frac{1}{n} \rbrace
\end{align*}
and note that because $A$ is closed, for each $n$,
\begin{align*}
A^c &= \lbrace x \in S \mid d(x, A) > 0 \rbrace \\
&=\lbrace x \in S \mid d(x, A) > n \rbrace \cup \bigcup_{m=n}^\infty \lbrace
x \in S \mid \frac{1}{m+1} < d(x, A) \leq \frac{1}{m} \rbrace \\
&=A_n^c \cup \bigcup_{m=n}^\infty R_m
\end{align*}
It follows that
$A^c \cap B = A_n^c \cap B \cup \cup_{m=n}^\infty
R_m \cap B$ and therefore by subadditivity of outer measure 
\begin{align*}
\mu^*(A^c \cap B) \leq \mu^*(A_n^c \cap B) + \sum_{m=n}^\infty
\mu^*(R_m \cap B)
\end{align*}
The claim will follow if we can show $\lim_{n \to \infty} \sum_{m=n}^\infty
\mu^*(R_m \cap B)=0$ which in turn will follow if we can show that $\sum_{m=1}^\infty
\mu^*(R_m \cap B)$ converges.  By construction, $d(R_{2m}, R_{2n})
> 0$ and therefore $d(R_{2m} \cap B, R_{2n} \cap B)
> 0$ for any $m \neq n$.  So if we consider only the even terms of the
series we can use our hypothesis to show that for any $n$
\begin{align*}
\sum_{m=1}^n \mu^*(R_{2m} \cap B) &= \mu^*(\cup_{m=1}^n
R_{2m} \cap B) \leq \mu^*(B) < \infty
\end{align*}
and by taking limits $\sum_{m=1}^\infty \mu^*(R_{2m} \cap B) \leq \mu^*(B)$
The same argument applies to the odd indexed terms and we get
\begin{align*}
\sum_{m=1}^\infty \mu^*(R_{m} \cap B) &\leq 2\mu^*(B) < \infty
\end{align*}
The claim and the Lemma follow.
\end{proof}


\begin{examp}\label{NonMetricOuterMeasure}Let $\mathcal{C}$ be the set of closed squares in $\reals^2$ and let
$\mu_0$ be the length of a side of each square.  Now construct the outer measure $\mu^*$ as in 
Proposition \ref{PremeasureToOuterMeasure}.  Let $0 < \epsilon < 1/4$ be chosen and 
consider the closed intervals $I_1 = [0, 1/2 - \epsilon]$ and $I_2 = [1/2+\epsilon, 1]$.  Define
\begin{align*}
A &= I_1 \times I_1 \cup I_1 \times I_2 \cup I_2 \times I_1 \cup I_2 \times I_2
\end{align*}
Consider covering $A$ by countable numbers of squares.  We claim that if $A$ is not covered 
by a single square then it is covered by at least 3 squares and the sum of the sides of those squares
exceeds 1.  From this the minimal covering of $A$ by squares is the single square $[0,1] \times [0,1]$ and
$\mu^*(A) = 1$.  (TODO: Show this more carefully).

On the other hand we claim $\mu^*(I_i \times I_j) = 1/2 - \epsilon$ for $i,j \in \lbrace 1, 2 \rbrace$
and from this we see that 
\begin{align*}
\mu^*(A) &= 1 < 4(1/2 - \epsilon) = \sum_{i,j} \mu^*(I_i \times I_j)
\end{align*}
and it follows that $\mu^*$ is not metric.
\end{examp}


\section{Radon-Nikodym Theorem and Differentiation}
We have seen the construction of measures by integration of a
density.  A productive line of inquiry is to ask if one can
characterize measures that arise through this construction and those
that cannot arise through this construction.  As it
turns out an precise answer may be given for $\sigma$-finite measures;
this is the content of the Radon-Nikodym Theorem.  If one restricts
attention to $\reals$ and considers the Fundamental Theorem of
Calculus for Riemann integrals
\begin{align*}
\frac{d}{dx} \int_0^x f(y) \, dy = f(x)
\end{align*}
one can surmise that there is a connection between the considerations
of the Radon-Nikodym Theorem and the theory of differentiation of
integrals.  This is indeed the case and we will prove the extension of
the Fundamental Theorem of Calculus to Lebesgue integrals using the
Radon-Nikodym Theorem.  Note that it is probably more traditional to
explore the theory of differention of functions of a real variable
without using the more abstract Radon-Nikodym Theorem but if one
intends to cover both one can save some time by proceeding in the way
we have chosen (stolen unabashedly from Kallenberg).

The first step is to develop a couple of tools that may be used to
compare two measures.  The trick is that if one takes the difference
of two measure, one does not get a measure.  However there is a clever
observation that helps to repair the defect.  
\begin{defn}A \emph{ bounded signed measure} on a measurable space $(\Omega,
  \mathcal{A})$ is a bounded function $\nu : \mathcal{A} \to
  \reals$ such that $\nu(\emptyset) = 0$ and for every disjoint $A_1, A_2, \dots \in
  \mathcal{A}$ such that $\sum_{n=1}^\infty \abs{\nu (A_n)}  < \infty$, we have $\nu(\bigcup_{n=1}^\infty A_n ) =
  \sum_{n=1}^\infty \nu (A_n)$ 
\end{defn}
Note that a bounded signed measure is finitely additive (just take
infinitely many copies of the empty set and use countable additivity).
It is important to note that a bounded signed measure is continuous in
the same way that an ordinary measure is.
\begin{prop}\label{ContinuityOfSignedMeasure}Let $\nu$ be a bounded
  signed measure on the measurable space $(\Omega,  \mathcal{A})$.
If $A, A_1, A_2, \dotsc \in \mathcal{A}$, $\sum_{n=1}^\infty \abs{\nu(A_n \setminus A_{n-1})} < \infty$ and $A_n \uparrow A$ then $\nu(A) = \lim_{n \to \infty} \nu(A_n)$. If $A, A_1, A_2, \dotsc \in \mathcal{A}$, $\sum_{n=1}^\infty \abs{\nu(A_n \setminus A_{n+1})} < \infty$ and $A_n \downarrow A$ then $\nu(A) = \lim_{n \to \infty} \nu(A_n)$.
\end{prop}
\begin{proof}
Continuity follows from the same proof as Lemma \ref{ContinuityOfMeasure}.  Defining $B_1 = A_1$ and $B_n = A_n \setminus A_{n-1}$ for $n > 1$ we see that $B_n$ are disjoint, $A_n = \cup_{j=1}^n B_j$ and $A = \cup_{j=1}^\infty B_j$.  By assumption, $\sum_{j=1}^\infty \abs{\nu(B_j)} < \infty$ and therefore we may apply countable and finite additivity to see
\begin{align*}
\nu(A) = \sum_{j=1}^\infty \nu(B_j) = \lim_{n \to \infty} \sum_{j=1}^n \nu(B_j) = \lim_{n \to \infty} \nu(A_n)
\end{align*}
To see continuity under decreasing sequences of sets 
\begin{align*}
\nu(A_1) - \nu(\cap_{j=1}^\infty A_j) &= \nu(A_1 \setminus \cap_{j=1}^\infty A_j) = \nu(\cup_{j=1}^\infty A_j \setminus A_{j+1}) \\
&= \sum_{j=1}^\infty\nu(A_j \setminus A_{j+1}) = \lim_{n \to \infty} \sum_{j=1}^n\nu(A_j \setminus A_{j+1}) 
\end{align*}
\end{proof}

Equally important to note is that monotonicity and subadditivity fail for signed measures.
\begin{examp}Let $\Omega = \lbrace 1,2,3 \rbrace$ and define $\nu(1) = \nu(2) = 1$ and $\nu(3) = -1$ then $0=\nu(\lbrace 1,3 \rbrace) < \nu(\lbrace 1 \rbrace)=1$ and $1 = \nu(\lbrace 1,3 \rbrace \cup \lbrace 2,3 \rbrace) > \nu(\lbrace 1,3 \rbrace) +  \nu(\lbrace 2,3 \rbrace) = 0$.
\end{examp}

\begin{defn}Two measures $\mu$ and $\nu$ on a measurable space $(\Omega,
  \mathcal{A})$ are said to be \emph{mutually singular} if there
  exists $A \in \mathcal{A}$ such that $\mu A = 0$ and $\nu A^c = 0$.
  We often write $\mu \perp \nu$.
\end{defn}
\begin{examp}Lebesgue measure and any Dirac measure on $\reals$ are
  mutually singular.
\end{examp}
\begin{examp}Let $f,g$ be positive measurable functions on $\reals$
  such that $\int f \wedge g \, d\lambda= 0$.  Then $f \cdot \lambda$ and $g
  \cdot \lambda$ are mutually singular.
\end{examp}
\begin{defn}Given two measures $\mu$ and $\nu$ on a measurable space $(\Omega,
  \mathcal{A})$ we say that $\nu$ is \emph{absolutely continuous} with
  respect to $\mu$ if for every $A \in \mathcal{A}$ such that $\mu A =
  0 $ we also have $\nu A = 0$.
  We often write $\nu \ll \mu$.
\end{defn}
\begin{examp}Let $f$ be a positive measurable function on the measure
  space $(\Omega,
  \mathcal{A}, \mu)$, then $f \cdot \mu$ is absolutely continuous with
  respect to $\mu$.  We shall soon see that this is the only way to
  construct absolutely continuous measures when the dominating measure is $\sigma$-finite.
\end{examp}
\begin{thm}[Hahn Decomposition]\label{HahnDecomposition}Given a
  bounded signed measure $\nu$ on a measurable space $(\Omega,
  \mathcal{A})$ there are unique bounded mutually singular positive
  measures $\nu_+$ and $\nu_-$ such that $\nu = \nu_+ - \nu_-$.
\end{thm}
\begin{proof}Let $c=\sup_{A \in \mathcal{A}} \nu(A)$.  The first claim
  is that there is a $A_+ \in \mathcal{A}$ such that $\nu A_+ = c$.
By continuity of measure Proposition \ref{ContinuityOfSignedMeasure} we expect to
be able to show this by taking a limit over sets with $\nu(A)$ getting arbitrarily close to $c$.  
Reflecting for a moment one realizes there are two reasons that a set $A$ may have measure less than 
$c$.  The first reason is that the set may be missing some positive mass that can be added and the second
reason is that the set may have some negative mass that can be subtracted; of course these two reasons are not 
mutually exclusive.  In the first case taking a union of sets improves the approximation in the second case taking an 
intersection improves the approximation therefore we expect our limiting process to involve both unions and intersections.

A trick is that signed measures are not subadditive hence taking a union does not always increase measure. 
The first thing we need is a simple bound on the damage that
taking a union can do to approximations to the supremum.
\begin{clm}Suppose we  are given $A,A^\prime \in \mathcal{A}$ such that $\nu A \geq c -
  \epsilon$ and $\nu A^\prime \geq c - \epsilon^\prime$ then $\nu(A \cup A^\prime) \geq c - \epsilon - \epsilon^\prime$.
 If $B \subset A \setminus A^\prime$ then $-\epsilon \leq \nu(B) \leq \epsilon + \epsilon^\prime$.
\end{clm}
To see the first part of the claim,
\begin{align*}
\nu (A \cup A^\prime) &= \nu (A \setminus (A \cap A^\prime)) + \nu (A^\prime  \setminus (A \cap A^\prime)) + \nu (A \cap A^\prime) \\
&=\nu (A) + \nu (A^\prime) - \nu (A \cap A^\prime) \\
&\geq \nu A + \nu A^\prime - c & &\text{by bound on $\nu$} \\
&\geq c - \epsilon - \epsilon^\prime & &\text{by bounds on $A,A^\prime$}
\end{align*}
For the second part of the claim, the lower bound actually holds for any subset of $A$; if $B \subset A$ then
by choice of $A$, finite additivity of $\nu$ and the definition of $c$
\begin{align*}
c  - \epsilon &\leq \nu(A) = \nu(A\setminus B) + \nu(B) \leq c + \nu(B)
\end{align*} 
and it follows that $\nu(B) \geq -\epsilon$.  
For the upper bound, if $B \subset A \setminus A^\prime$ then by finite additivity, the definition of $c$
and the two lower bounds already established,
\begin{align*}
\nu(B) &= \nu(A \cup A^\prime) - \nu(A^\prime) - \nu(A \setminus A^\prime \setminus B) \\
&\leq c - (c - \epsilon^\prime) - (-\epsilon) = \epsilon + \epsilon^\prime
\end{align*}

\begin{clm}There exists a measurable set $A_+$ such that $\nu(A_+)=c$.
\end{clm}
Approximate the supremum by taking $A_1, A_2, \dots \in \mathcal{A}$ such that $\nu
  A_n \geq c - 2^{-n}$.  By the first part of the previous claim and a simple induction we have
$\nu(\cup_{j=n+1}^m A_j) \geq c - \sum_{j=n+1}^m 2^{-j}$ for all $n < m$.  Note that
\begin{align*}
\cup_{j=n+1}^{m+1} A_j \setminus \cup_{j=n+1}^{m} A_j &= A_{m+1} \setminus \cup_{j=n+1}^{m} A_j \subset A_{m+1} \setminus A_m
\end{align*}
and therefore by the second part of the prior claim, $\abs{\nu(\cup_{j=n+1}^{m+1} A_j \setminus \cup_{j=n+1}^{m} A_j)} \leq 2^{-m-1} + 2^{-m} < 2^{-m+1}$
so that we may apply continuity of measure (Proposition \ref{ContinuityOfSignedMeasure}) to conclude
\begin{align*}
\nu \bigcup_{i=n+1}^\infty A_i &= \lim_{m \to \infty} \nu \bigcup_{i=n+1}^m A_i
\geq \lim_{m \to \infty} c - \sum_{i=n+1}^m 2^{-i} = c - 2^{-n}
\end{align*}
Let $A_+ = \bigcap_{n=1}^\infty \bigcup_{i=n+1}^\infty A_i$ and by the same argument as above we have
\begin{align*}
\bigcup_{i=n}^\infty A_i \setminus \bigcup_{i=n+1}^\infty A_i \subset A_n \setminus A_{n+1}
\end{align*}
and therefore 
\begin{align*}
\sum_{n=1}^\infty \abs{ \nu \bigcup_{i=n}^\infty A_i \setminus \bigcup_{i=n+1}^\infty A_i} &\leq \sum_{n=1}^\infty 2^{-n+1} < \infty
\end{align*}
and we may apply Proposition \ref{ContinuityOfSignedMeasure} to conclude
\begin{align*} \nu A_+ = \lim_{n \to \infty} \nu
  \bigcup_{i=n+1}^\infty A_i \geq c
\end{align*}
By the definition of $c$ we see that $\nu A_+ = c$.  

\begin{clm}Define $A_- =
A_+^c$ and the restrictions 
\begin{align*}
\nu_+ B &= \nu (A_+ \cap B ) \\
\nu_- B &= -\nu ( A_- \cap B )
\end{align*}
then $\nu_\pm$ are both measures.
\end{clm}
We prove this for $\nu_+$.  Since $\nu(A_+) = \sup_{A \in \mathcal{A}} \nu(A)$ it follows that $\nu_+(B) \geq 0$ for all $B$; if not then $\nu(A_+ \setminus B) = \nu(A_+) - \nu(A_+ \cap B) > \nu(A_+)$ which is a contradiction.  Let $B_1, B_2, \dotsc \in \mathcal{A}$ be disjoint then $\sum_{j=1}^n \nu(A_+ \cap B_j) \leq \nu(A_+)$ and is an increasing sequence thus $\sum_{j=1}^\infty \abs{\nu(A_+ \cap B_j)} = \sum_{j=1}^\infty \nu(A_+ \cap B_j)$ exists and is finite.  There since $\nu$ is a bounded signed measure
\begin{align*}
\sum_{j=1}^\infty \nu_+(B_j) &= \sum_{j=1}^\infty \nu(A_+ \cap B_j) = \nu(A_+ \cap \cup_{j=1}^\infty  B_j) = \nu_+(\cup_{j=1}^\infty  B_j)
\end{align*}
and the claim follows.  To see that $\nu_-$ is also a measure first note that $\nu(A_-) = \inf_{A \in \mathcal{A}} \nu(A)$.  If $\nu(A_-) > \inf_{A \in \mathcal{A}} \nu(A)$ there exists $B \in \mathcal{A}$ with $\nu(B) < \nu(A_-)$ and 
\begin{align*}
\nu(B^c) &= \nu(\Omega) - \nu(B) > \nu(\Omega) - \nu(A_-) = \nu(A_+)
\end{align*}
which is a contradiction.  Thus the fact that $\nu_-$ is a measure follows by using $-\inf_{A \in \mathcal{A}} \nu(A) = \sup_{A \in \mathcal{A}} - \nu(A)$ 
by considering the bounded signed measure $-\nu$.

Since $\nu_+(A_+^c) = \nu(\emptyset) = 0$ and $\nu_-(A_+) = -\nu ( \emptyset ) = 0$ it follows that $\nu_+$ and $\nu_-$ are mutually singular.  Furthermore by finite additivity we know that $\nu(B) = \nu(B \cap A_+) + \nu(B \cap A_-) = \nu_+(B) - \nu_-(B)$.

If $\nu = \mu_+ - \mu_-$ with $\mu_\pm$ bounded mutually singular measures then pick $B$ such that $\mu_+(B^c) = 0$ and $\mu_-(B) = 0$ then we have
\begin{align*}
\nu_+(A_+) &= \nu(A_+) = \mu_+(A_+ \cap B) - \mu_-(A_+ \cap B^c)
\end{align*}
from which it follows that $\mu_-(A_+ \cap B^c)=0$ since otherwise 
\begin{align*}
\mu_+(A_+ \cap B) = \nu(A_+ \cap B) > \nu(A_+) = \sup_{A \in \mathcal{A}} \nu(A)
\end{align*}
therefore $\mu_-(A_+) = \mu_-(A_+ \cap B) + \mu_-(A_+\cap B^c) = 0$.  By a similar argument we conclude that $\mu_+(A_-) = 0$ thus we may assume that $B=A_+$ and it follows that $\mu_\pm = \nu \mid_{A_\pm} = \nu_\pm$.
\end{proof}

\begin{thm}[Radon-Nikodym Theorem]\label{RadonNikodym}Let $\mu, \nu$
  be $\sigma$-finite measures on the measurable space $(\Omega,
  \mathcal{A})$.  There exist unique measures $\nu_a \ll \mu$ and
  $\nu_s \perp \mu$ such that $\nu = \nu_a + \nu_s$.  Furthermore,
  there is a unique positive measurable $f : \Omega \to \reals$ such
  that $\nu_a = f \cdot \mu$.
\end{thm}
\begin{proof}TODO
\end{proof}

In addition to the product measure construction we have just seen
there is another important construction for $\reals$.
\begin{defn}A measure $\mu$ on $(\reals, \mathcal{B}(\reals))$ is called
  \emph{locally finite} if $\mu(I) < \infty$ for every finite interval
  $I \subset \reals$.
\end{defn}
\begin{lem}[Lebesgue-Stieltjes Measure]\label{LebesgueStieltjesMeasure}There is a 1-1
  correspondence between locally finite measures on $(\reals,\mathcal{B}(\reals))$ and
  nondecreasing right continuous functions $F : \reals \to \reals$ such that $F(0)=0$ given by 
\begin{align*}
\mu((a,b]) = F(b) - F(a)
\end{align*}
\end{lem}
\begin{proof}
Suppose we are given a locally finite measure $\mu$ on
$(\reals,\mathcal{B}(\reals))$.  Define
\begin{align*}
F(x) = \begin{cases}
\mu (0,x] & \text{if $x > 0$}\\
-\mu (x, 0] & \text{if $x < 0$}\\
0 & \text{if $x=0$}
\end{cases}
\end{align*}
Local finiteness of $\mu$ implies that $F$ is well defined.
Monotonicity of $\mu$ implies that $F$ is nondecreasing.  Continuity
of measure implies that $F$ is right continuous.  Clearly, 
\begin{align*}
\mu (a,b] = F(b) - F(a)
\end{align*} and furthermore $F$ is the unique function that satisfies
this property.

On the other hand, given an $F$ that is nondecreasing, right
continuous and satisfies $F(0) = 0$ we define a generalized inverse by 
\begin{align*}
G(y) = \inf \lbrace x \in \reals \mid F(x) \geq y \rbrace = \sup
\lbrace x \in \reals \mid F(x) < y \rbrace
\end{align*}
Note that if $y < w$ then $\lbrace x \in \reals \mid F(x) \geq w
\rbrace \subset \lbrace x \in \reals \mid F(x) \geq y \rbrace$ which
shows that $G$ is a nondecreasing function.  The fact that $G$ is
nondecreasing implies that $G^{-1} (-\infty, y] = (-\infty, x]$ for
some $x \in \reals$ and therefore $G$ is a measurable function.  
Furthermore, 
\begin{align*}
G(F(x)) &= \inf\lbrace s \in \reals \mid F(s) \geq F(x)
\rbrace \leq x
\end{align*}
and on the other hand since 
\begin{align*}
G(y) &= \inf\lbrace x \in \reals \mid F(x) \geq y
\rbrace 
\end{align*}
we can find a sequence $x_n \downarrow G(y)$ such that $F(x_n) \geq y$
and therefore by right continuity of $F$ we now that $F(G(y)) =
\lim_{n\to\infty} F(x_n) \geq y$.

Together these two facts show that 
$G(y) \leq c$ if and only if $y \leq F(c)$.  In one direction suppose
$y \leq F(c)$, then applying $G$ to both sides and using the
nondecreasing nature of $G$, we get $G(y) \leq G(F(c)) \leq c$.  In
the other direction, we assume $G(y) \leq c$ and apply $F$ to both
sides and to see
\begin{align*}
F(c) \geq F(G(y)) \geq y
\end{align*}
It follows that we also have the contrapositive assertion $c < G(y)$ if and only if $F(c) < y$.

Now we can finish the proof by 
defining $\mu = (\pushforward{G}{\lambda})$ where $\lambda$ is Lebesgue
measure on $\reals$.  We observe that this is an inverse to the
construction of $F$ given above.  
\begin{align*}
\mu (a,b] &= \lambda \left ( \lbrace y \in \reals \mid a < G(y)  \leq b
  \rbrace \right ) \\
&= \lambda (F(a), F(b)] = F(b) - F(a)
\end{align*}

Uniqueness of measure $\mu$ with this property follows by Lemma
\ref{UniquenessOfMeasure} as local finiteness obviously implies
$\sigma$-finiteness on $\reals$.
\end{proof}

Note the choice of the normalizing condition $F(0) = 0$ is somewhat
arbitrary albeit a natural choice when considering arbitrary locally
finite measures on $\reals$.  We will see later that for finite
measures, and probability
measures in particular, it is more useful to pick a different
normalization $\lim_{x \to -\infty} F(x) = 0$.

By the description of all measures on $\reals$ as
Lebesgue-Stieltjes measures, we have set the stage for the
translation of results about measures into results about
nondecreasing, right continuous functions.  In particular, if we apply
the Radon-Nikodym Theorem to we see that any such $F$ may be written
as $F = F_a + F_s$ which represent the absolutely continuous and
singular parts of the decomposition respectively.  If one unwinds the
defining property of $F_a$ from the Lebesgue-Stieltjes integral, one
sees
 that in the absolutely continuous case, $F_a(x) = \int_0^x f \,
 d\lambda$ for an appropriate density $f$.

\begin{thm}[Fundamental Theorem Of Calculus]\label{FundamentalTheoremOfCalculus}Let any nondecreasing, right continuous function $F(x) = \int_0^x
  f \, d\lambda + F_s(x)$ is differentiable a.e. with derivative $F^\prime = f$.
\end{thm}
Before we give the proof of the Fundamental Theorem we need a couple of lemmas.
\begin{lem}\label{IntervalSelection}Let $\mathcal{I}$ be an arbitrary
  collection of open intervals of $\reals$.  Let $G = \bigcup_{I \in
    \mathcal{I}} I$ and suppose that $\lambda G < \infty$.  Then there
  exists disjoint $I_1, \dots, I_n$ such that $\sum_{i=1}^n \abs{I_i}
  \geq \frac{\lambda G}{4}$.
\end{lem}
\begin{proof}
We begin by finding a compact set to focus attention on.
\begin{clm}There is a compact set $K$ such that $K \subset G$ and $\lambda(K) \geq \frac{3}{4} \lambda (G)$.
\end{clm}
We first note that $G$ is a countable union of compact sets (in fact a
countable union of closed bounded intervals).  For an open interval
this is easy to see by construction and since $G$ is open it is a countable union of disjoint
open intervals (Lemma \ref{OpenSetsOfReals} the claim follows.  Thus we may write $G = \cup_{n=1}^\infty [a_n,b_n]$ and
if we define $K_j = \cup_{n=1}^j [a_n,b_n]$ each $K_j$ is compact and by continuity of measure 
$\lim_{j \to \infty} \lambda(K_j) = \lambda(G)$.  For $j$ sufficiently large we have $\lambda(K_j) \geq \frac{3}{4} \lambda (G)$.

Since $K$ is compact there is a finite set of intervals $J_1, \dotsc, J_k$ with $J_i \in \mathcal{I}$ such that $K \subset J_1 \cup \dotsb \cup J_k$.
Define $I_1$ to be the the largest interval among the $J_i$ and inductively define $I_j$ so that
\begin{align*}
\lambda(I_j) &= \max \lbrace \lambda(J_i) \mid J_i \cap I_l = \emptyset \text{ for $l=1, \dotsc, j-1$} \rbrace
\end{align*}
where we stop the iteration if there is no $J_i$ that is disjoint for each of the $I_l$ for $l=1, \dotsc, j-1$.

Now observe that for every $J_i$ there is an $I_k$ such that $J_i \cap I_k \neq \emptyset$ and $\lambda(I_k) \geq \lambda(J_i)$.  
To see this note that by construction there must be a smallest index $k$ such that $J_i \cap I_k \neq \emptyset$.  If $k=1$ then again by construction
it follows that $\lambda(I_1)=\lambda(I_k) \geq \lambda(J_i)$; if $k>1$ then $J_i \cap I_l = \emptyset$ for $l=1, \dotsc, k-1$ and 
by construction we know $\lambda(I_k) \geq \lambda(J_i)$.  Define $\hat{I}_k$ to be the open interval with the same center as $I_k$ but length
three times as large; it follows from 
$J_i \cap I_k \neq \emptyset$ and $\lambda(I_k) \geq \lambda(J_i)$ that $J_i \subset \hat{I}_k$ and therefore we get 
\begin{align*}
K &\subset J_1 \cup \dotsb \cup J_k \subset \hat{I}_1 \cup \dotsb \cup \hat{I}_n
\end{align*}
By subadditivity we get
\begin{align*}
\frac{3}{4} \lambda{G} &\leq \lambda{K} \leq \sum_{j=1}^n \lambda(\hat{I}_j) = 3 \sum_{j=1}^n \lambda(I_j) 
\end{align*}
\end{proof}
Now we prove the fundamental theorem of calculus in the special case of measures that are mutually singular to
Lebesgue measure.
\begin{lem}\label{DifferentiationOnNullSets}Let $\mu$ be a locally finite measure on $(\reals, \mathcal{B}(\reals))$
  and let $F(x) = \mu (0,x]$.  Let $A \in \mathcal{B}$ be a set with
  $\mu A = 0$, then $F^\prime = 0$ almost everywhere $\lambda$ on $A$.
\end{lem}
\begin{proof}The intuition behind the proof is that the derivative
  $F^\prime(x)$ represents the ratio of $\mu$-measure and
  $\lambda$-measure for arbitrarily small intervals around $x \in
  \reals$.  For $x \in A$, we expect the $\mu$-measure and therefore
  the derivative to be $0$.  Since $A$ may not contain any honest
  intervals, there is some finesse required to make the intuition rigorous.

First pick $\delta > 0$ and and open set $G_\delta \supset A$ such that $\mu
G_\delta < \delta$.  

TODO: Prove that such $G_\delta$ exists; this is a fact for arbitrary
Borel $\sigma$-algebras.

For each $\epsilon > 0$, let 
\begin{align*}
A_\epsilon &= \{ x \in A \mid \limsup_{h \to 0} \frac{F(x + h) - F(x -
  h)}{h} > \epsilon \} \\
\end{align*}
so that for $x \in A_\epsilon$ there exist arbitrarily small $h > 0$
such that 
\begin{align*}
\mu(x-h,x+h] &=F(x + h) - F(x -  h) > \epsilon h =
\frac{1}{2}\epsilon \lambda(x-h,x+h]
\end{align*}
Note that $A_\epsilon$ is measurable since 
\begin{align*}
\limsup_{h \to 0}  \frac{F(x + h) - F(x -
  h)}{h} &= \limsup_{n \to \infty} n \left(F(x + 1/n) - F(x -
  1/n)\right)
\end{align*} 
is measurable (Lemma \ref{LimitsOfMeasurable}).

By openness of $G_\delta$ and by the above remarks, for any $x \in A_\epsilon$ we can pick $h > 0$ small enough so that
$I_x = (x - h, x+h] \subset G_\delta$ and $2\mu(I_x)/\epsilon > \lambda(I_x)$.  Since $A_\epsilon \subset
\bigcup_{x \in A_\epsilon} I_x$, by the previous Lemma
\ref{IntervalSelection} we pick a finite disjoint set $I_{x_1}, \dots,
I_{x_n}$ and note that
\begin{align*}
\lambda A_\epsilon &\leq \lambda 
\bigcup_{x \in A_\epsilon} I_x \leq 4 \sum_{k=1}^n \abs{I_{x_k}} \leq
4 \sum_{k=1}^n \frac{2\mu I_{x_k}}{ \epsilon} = \frac{8}{\epsilon} \mu
\bigcup_{k=1}^n I_{x_k} \leq \frac{8\delta}{\epsilon}
\end{align*}
Now $\delta > 0$ was arbitrary so we see that $\lambda A_\epsilon =
0$.  Since $\epsilon > 0$ was arbitrary and since the set of points in
$A$ where $F^\prime \neq
0$ is a countable union of $A_\epsilon$ (e.g. take $\bigcup_n
A_{\frac{1}{n}}$)  we see that $F^\prime(x) = 0$
almost everywhere on $A$.
\end{proof}

\begin{proof}
From Lemma \ref{DifferentiationOnNullSets} if follows that $\frac{d}{dx} F_s = 0$ a.s. so it suffices to assume that
$F(x) = \int_0^x f \, d\lambda$ for a non-negative locally integrable $f$.   The trick is to reduce this case to  Lemma \ref{DifferentiationOnNullSets} as well.  
Let $q \in \rationals$ be
chosen arbitrarily, let $F_q(x) = \int_0^x (f-q)_+ \, d\lambda$ and consider the locally finite
measure $\mu_q(0,x] = F_q(x)$.  Since $(f-q)_+ = 0$ on $\lbrace f \leq q \rbrace$ we know
\begin{align*}
\mu_q \lbrace f \leq q \rbrace &= \int \characteristic{f \leq q} f \, d\lambda = 0
\end{align*}
Apply Lemma \ref{DifferentiationOnNullSets} to conclude that $F^\prime_q(x) = 0$ a.e. on $\lbrace f \leq q \rbrace$.
On the other hand, $f \leq q + (f-q)_+$ and therefore
\begin{align*}
\limsup_{h \to 0} \frac{F(x+h) - F(x)}{h} &= \limsup_{h \to 0} h^{-1} \int_x^{x+h} f \, d\lambda \\
&\leq q + \limsup_{h \to 0} h^{-1} \int_x^{x+h} (f-q)_+ \, d\lambda = q \text{ a.e. on $\lbrace f \leq q \rbrace$}
\end{align*}

Via a union bound we conclude
\begin{align*}
\lambda \lbrace f < \limsup_{h \to 0} \frac{F(x+h) - F(x)}{h} \rbrace &= \lambda \bigcup_{q \in \rationals} \lbrace  f \leq q < \limsup_{h \to 0} \frac{F(x+h) - F(x)}{h} \rbrace \\
&\leq \sum_{q \in \rationals} \lambda \lbrace  f \leq q < \limsup_{h \to 0} \frac{F(x+h) - F(x)}{h} \rbrace = 0
\end{align*}
which shows $\limsup_{h \to 0} \frac{F(x+h) - F(x)}{h} \leq f$ a.e.

Arguing analogously we see that $\frac{d}{dx} \int_0^x (q - f)_+ \, d\lambda=0$ a.e. on $\lbrace f \geq q \rbrace$ and from $f \geq q - (q - f)_+$ we get
\begin{align*}
\liminf_{h \to 0} \frac{F(x+h) - F(x)}{h} &= q \text{ a.e. on $\lbrace f \geq q \rbrace$}
\end{align*}
Again a union bound yields $\liminf_{h \to 0} \frac{F(x+h) - F(x)}{h} \geq f$ a.e.  and it follows that
\begin{align*}
\liminf_{h \to 0} \frac{F(x+h) - F(x)}{h} &= \limsup_{h \to 0} \frac{F(x+h) - F(x)}{h} = \lim_{h \to 0} \frac{F(x+h) - F(x)}{h} = f \text{ a.e.}
\end{align*}
\end{proof}

\subsection{Functions of Bounded Variation}
Recall that we have define $x_+ = x \vee 0$ and $x_- = \abs{x} -
x_+ = -(x \wedge 0)$.  Given a real valued function $F$ on $[a,b]$ we consider a partition $a=x_0
< x_1 < \dotsb < x_n=b$ and define 
\begin{align*}
p &= \sum_{j=1}^n (F(x_j) - F(x_{j-1}))_+ \\
n &= \sum_{j=1}^n (F(x_j) - F(x_{j-1}))_- \\
v &= \sum_{j=1}^n \abs{F(x_j) - F(x_{j-1})}\\
\end{align*}
and note that $p+n = v$ and $p - n = F(b) - F(a)$.  We define the
\emph{positive}, \emph{negative} and \emph{total variation} of $F$ on
$[a,b]$ to be the supremum of the above over all partitions of
$[a,b]$:
\begin{align*}
P_a^b(F) &= \sup_{\substack{n \geq 1 \\ a=x_0 < x_1 < \dotsb x_n=b}} \sum_{j=1}^n (F(x_j) - F(x_{j-1}))_+ \\
N_a^b(F) &= \sup_{\substack{n \geq 1 \\ a=x_0 < x_1 < \dotsb x_n=b}} \sum_{j=1}^n (F(x_j) - F(x_{j-1}))_- \\
TV_a^b(F) &= \sup_{\substack{n \geq 1 \\ a=x_0 < x_1 < \dotsb x_n=b}} \sum_{j=1}^n \abs{F(x_j) - F(x_{j-1})}\\
\end{align*}


\begin{lem}\label{ElementaryVariationInequalities}For any function $F$ defined on $[a,b]$ we have 
\begin{align*}
P_a^b(F) \vee N_a^b(F) &\leq TV_a^b(F) \\
\intertext{and}
TV_a^b(F) &\leq P_a^b(F) + N_a^b(F)
\end{align*}
\end{lem}
\begin{proof}
For any partition $a=x_0 < x_1 < \dotsb < x_n=b$ we noted above $p + n
\leq v$ and therefore $p \leq v$ which implies by taking the supremum on
the right $p \leq TV_a^b(F)$ and then by taking the supremum on the
left $P_a^b(F) \leq TV_a^b(F)$.  The argument to show $N_a^b(F) \leq
TV_a^b(F)$ is identical.  Similarly from $v = p + n$, we can take two
different suprema on the right to see that $v \leq P_a^b(F) +
N_a^b(F)$ and then taking the supremum on the left we get $TV_a^b(F)
\leq P_a^b(F) \leq TV_a^b(F)$.
\end{proof}

\begin{defn}We say that  function $F$ defined on $[a,b]$ has
  \emph{bounded variation} on $[a,b]$ if $TV_a^b(F) < \infty$.
\end{defn}

\begin{lem}\label{TotalVariationAsSumOfPositiveAndNegativeVariation}If $F$ has bounded variation on $[a,b]$ then 
\begin{align*}
TV_a^b(F) &= P_a^b(F) + N_a^b(F) \\
\intertext{and}
F(b) - F(a) &=  P_a^b(F) - N_a^b(F) 
\end{align*}
\end{lem}
\begin{proof}
From \ref{ElementaryVariationInequalities}, we know that $F$ being of
bounded variation implies that both the positive and negative
variation are finite.  Now with a fixed $a=x_0 < x_1 < \dotsb < x_n=b$
we had $p  = n + F(b) - F(a)$, so taking supremum on the right we get
$p  \leq N_a^b(F) + F(b) - F(a)$
and the taking suspremum on the left we get $P_a^b(F)  \leq N_a^b(F) +
F(b) - F(a)$.  As noted the negative variation is finite and therefore
we conclude $P_a^b(F)  - N_a^b(F) \leq
F(b) - F(a)$. Similarly we get from applying the same steps to $n = p
+ F(a) - F(b)$
that $N_a^b(F) \leq P_a^b(F) + F(a) - F(b)$ which gives us $F(b) -F(a)
\leq P_a^b(F)  - N_a^b(F)$ and therefore we conclude that $F(b) -F(a)
= P_a^b(F)  - N_a^b(F)$.

Now arguing from $p + n = v$ and taking the supremum on the right we
have using $F(b) - F(a) = p -n$,
\begin{align*}
TV_a^b(F) \geq p+n = 2p + F(a) - F(b) = 2p + N_a^b(F) - P_a^b(F)
\end{align*}
which upon taking another supremum gives
\begin{align*}
TV_a^b(F) \geq 2P_a^b(F) + N_a^b(F) - P_a^b(F) = P_a^b(F) + N_a^b(F)
\end{align*}

Note a more hands on way of proving the this result is to note
that we have a triangle inequality $(x+y)_+ leq x_+ + y_+$ and therefore
if we are given a partition $a=x_0 < x_1 < \dotsb < x_n=b$ and
refine the partition by adding a new point then to create a new
partition $a=\tilde{x}_0 < \tilde{x}_1 < \dotsb < \tilde{x}_n=b$ then
we have
\begin{align*}
\sum_{j=1}^n (F(x_j) - F(x_{j-1}))_+ &\leq \sum_{j=1}^{n+1} (F(\tilde{x}_j) - F(\tilde{x}_{j-1}))_+
\end{align*}
and similarly with the negative variation.  Now let $\epsilon > 0$ be
chosen and find partitions $a=x_0 < x_1 < \dotsb < x_n=b$ such that 
\begin{align*}
P_a^b(F) - \epsilon/2 &< \sum_{j=1}^n (F(x_j) - F(x_{j-1}))_+  \leq P_a^b(F) 
\end{align*}
and $a=y_0 < y_1 < \dotsb < y_m=b$ such that
\begin{align*}
N_a^b(F) - \epsilon/2 &< \sum_{j=1}^m (F(y_j) - F(y_{j-1}))_+  \leq N_a^b(F) 
\end{align*}
By the above argument, both inequalities continue to hold if we take
the common refinement of both partitions so we may in fact assume that
$n=m$ and $x_j=y_j$ for $j=0, \dotsc, n$.  Therefore by adding we get
\begin{align*}
P_a^b(F)  + N_a^b(F) - \epsilon &< \sum_{j=1}^n  (F(x_j) -
F(x_{j-1}))_+ +  (F(x_j) - F(x_{j-1}))_- \\
&= \sum_{j=1}^n  \abs{F(x_j) - 
F(x_{j-1})} \leq TV_a^b(F)
\end{align*}
and the result follows by taking the limit as $\epsilon$ goes to $0$.
\end{proof}

\begin{thm}\label{BoundedVariationAsDifferenceOfMonotone}A function on $[a,b]$ is of bounded variation if an only if
  it is the difference of two non-decreasing functions.
\end{thm}
\begin{proof}
First we show that a function of bounded variation is a difference of
monotone functions.  Consider a point $a \leq x \leq b$ and note that since every partition
of $[a,x]$ can be extended to a partiton of $[a,b]$ we have $P_a^x(F)
\leq P_a^b(F) \leq TV_a^b(F) < \infty$ and similarly with $N_a^x(F)$.
The same argument for any pair $a \leq x \leq y \leq b$ shows that
$P_a^x(F) \leq P_a^y(F)$ and similarly $N_a^x(F) \leq N_a^y(F)$.
Therefore $P_a^x(F)$ and $N_a^x(F)$ are both non-decreasing functions
and applying Lemma
\ref{TotalVariationAsSumOfPositiveAndNegativeVariation} on the
interval $[a,x]$ we get $F(x) = P_a^x(F) - N_a^x(F) - F(a)$.  Since
$N_a^x(F) - F(a)$ is also a non-decreasing function we are done with
this direction.

Now if $F(x) = G(x) - H(x)$ with both $G$ and $H$ monotone then for
any partition $a=x_0 < x_1 < \dotsb < x_n=b$ we have
\begin{align*}
\sum_{j=1}^n \abs{F(x_j) - F(x_{j-1})} &= \sum_{j=1}^n \abs{G(x_j) -
  G(x_{j-1}) - H(x_j) + H(x_{j-1})} \\
&\leq \sum_{j=1}^n (G(x_j) -
  G(x_{j-1})) + \sum_{j=1}^n( H(x_{j-1}) + H(x_{j})) = G(b) - G(a) +
  H(a) - H(b)
\end{align*}
\end{proof}

\begin{lem}\label{AdditivityOfTotalVariation}Let $f$ be a function of
  bounded variation on $[a,b]$, then for every $a < x < b$, $TV_a^x(f)
  + TV_x^b(f) = TV_a^b(f)$.
\end{lem}
\begin{proof}Pick partitions $a=x_0 < \dotsb < x_n=x$ of $[a,x]$ and $x=y_0 <
  \dotsb < y_m=b$ of $[x,b]$ and note that $a=x_0 < \dotsb < x_n=y_0 < y_1 < \dotsb <
  y_m=b$ is a partition of $[a,b]$.  Therefore
\begin{align*}
\sum_{j=1}^n \abs{f(x_j) - f(x_{j-1})} + \sum_{j=1}^m \abs{f(y_j) -
  f(y_{j-1})} &\leq TV_a^b(f)
\end{align*}
which upon taking suprema over partitions of $[a,x]$ and $[x,b]$ shows
$TV_a^x(f) + TV_x^b(f) \leq TV_a^b(f)$.  

On the other hand, let $a=x_0 < \dotsb < x_n=b$ be a partition of
$[a,b]$.  First assume that there exists an $0 < m < n$ such that $x_m
= x$.  It then follows that $a=x_0 < \dotsb < x_m=x$ is a partition of
$[a,x]$ and $x = x_m < \dotsb < x_n=b$ is a partition of $[x,b]$ and
therefore
\begin{align*}
\sum_{j=1}^n \abs{f(x_j) - f(x_{j-1})} &= \sum_{j=1}^m \abs{f(x_j) -
  f(x_{j-1})} + \sum_{j=m+1}^n \abs{f(x_j) - f(x_{j-1})} \leq
TV_a^x(f)  + TV_x^b(f)
\end{align*}
On the other hand, if $x$ is not a member of the partition then we may
add it and by the triangle inequality that can only increase the
variation of the partition so the inequality still holds.  Thus we may
take the supremum over all partitions of $[a,b]$ and we get $TV_a^b(f)
\leq TV_a^x(f)  + TV_x^b(f)$ and the result is proven.
\end{proof}

\begin{lem}\label{ContinuityOfTotalVariation}Let $f$ be a left
  continuous function with bounded variation on $[a,b]$, then
  $TV_a^x(f)$ is a left continuous function of $x$.  Similarly if $f$
  is right continuous (resp. continuous) then  $TV_a^x(f)$ is a right
  continuous (resp. continuous).
\end{lem}
\begin{proof}
We first suppose that $f$ is left continuous and show that $TV_a^x(f)$
is left continuous at $x$.  Pick $\epsilon > 0$ and select a partion
$a=x_0 < x_1 < \dotsb < x_n=x$ such that $\sum_{j=1}^n \abs{f(x_j) -
  f(x_{j-1})} > TV_a^x(f) - \epsilon/2$.  By left continuity of $f$ at $x$
we can pick a $\delta>0$ such that $\abs{f(x) - f(y)} < \epsilon/2$
for all $x-\delta < y < x$. Without loss of generality we may also
assume that $\delta < x - x_{n-1}$.  For any such $y$ we
define a new partition by adding the point $y$ to the existing
partition $x_0, \dotsc, x_n$; precisely define 
\begin{align*}
\tilde{x}_j &= \begin{cases}
x_j & \text{for $j=0, \dotsc, n-1$} \\
y & \text{for $j=n$} \\
x & \text{for $j=n+1$}
\end{cases}
\end{align*}
and note that by the triangle inequality, 
\begin{align*}
TV_a^x(f) - \epsilon/2 &< \sum_{j=1}^n \abs{f(x_j) -
  f(x_{j-1})} \leq \sum_{j=1}^{n+1} \abs{f(\tilde{x}_j) -
  f(\tilde{x}_{j-1})} \leq TV_a^x(f)
\end{align*}
If restrict our attention to the partition $a = \tilde{x}_0 < \dotsb <
\tilde{x}_n = y$, by monotonicity of total variation and the choice of $y$ we have
\begin{align*}
TV_a^x(f) &\geq TV_a^y(f) \geq \sum_{j=1}^{n} \abs{f(\tilde{x}_j) -
  f(\tilde{x}_{j-1})} \\
&> TV_a^x(f) - \epsilon/2 - \abs{f(x) - f(y)} > TV_a^x(f) - \epsilon
\end{align*}
which shows left continuity of $TV_a^x(f)$.

One could prove the case of right continuous $f$ by an analogous
argument that shows $TV_x^b(f)$ is a right continuous function of $x$
and then observing $TV_a^x(f) = TV_a^b(f) - TV_x^b(f)$ Lemma
\ref{AdditivityOfTotalVariation} (do this as an exercise!).  Here we take a slightly different
approach and derive the case of right continuity from the case of left
continuity.  Given $f$ a function on $[a,b]$, define the function
$\tilde{f}(x) = f(b+a-x)$ on $[a,b]$.  Note that $f$ is right
continuous if and only if $\tilde{f}$ is left continuous.  Note also
that the transformation $x \mapsto b+a-x$ is a bijection of $[a,y]$
and $[b+a-y,b]$ for every $a \leq y \leq b$
and therefore is a bijection of partitions of $[a,y]$ and $[b+a-y,b]$
for every such $y$.  From this it follows that $TV_a^y(f) =
TV_{b+a-y}^b(\tilde{f})$ for every $a \leq y \leq b$.  In particular
taking $y=b$,
$f$ is of bounded variation on $[a,b]$ if and only if $\tilde{f}$ is.
Stitching all of these observations together, if $f$ is right
continuous, then $\tilde{f}$ is left continuous and therefore by the
first part of the Lemma and Lemma \ref{AdditivityOfTotalVariation} we
know that $TV_{y}^b(\tilde{f}) = TV_a^b(\tilde{f}) -
TV_a^{y} (\tilde{f})$ is a left continuous function of $y$.  From this
it follows that $TV_a^y(f) = TV_{b+a-y}^b(\tilde{f})$ is a right
continuous function of $y$.

The case of $f$ continuous follows immediately as a function is
continuous if and only if it is both right continuous and left continuous.
\end{proof}

As an exercise, one should show that continuity of a function not only
implies the continuity of the total variation but also the positive
and negative variations (all we needed positivity and the triangle
inequality of the absolute value; properties that are shared by the
positive and negative part functions).  (TODO: Can we instead derive the
positive and negative variation cases from right continuity of total
variation and $f$?)
If we assume that we are given a right continuous function $f$ of bounded
variation, then by Lemma \ref{ContinuityOfTotalVariation} we know that
positive and negative variations are right continuous and therefore by
Theorem \ref{BoundedVariationAsDifferenceOfMonotone} we see that $f$
is a difference of monotone right continuous functions.  By the
construction of Lebesgue-Stieltjes measures this allows us to
associate locally finite (signed) measures to $f$.  

TODO: Define all of the measures involved and observe that $dF = dF_+ - dF_-$
is the Jordon decompostiion of the signed measure $dF$ and that $dTV_a^s(F)$ is the
absolute value of the signed measure $dF$.

\begin{lem}\label{AbsoluteValueOfStieltjes}Let $F$ be a function of bounded variation of $[a,b]$ and
  let $g$ be a measurable function then $\abs{ \int g \, dF} \leq \int
  \abs{g} \abs{dF}$.
\end{lem}
\begin{proof}
This is just a computation using the definitions and the triangle inequality
\begin{align*}
\abs{\int g \, dF} &= \abs{\int g \, dF_+ - \int g \, dF_-} \leq
\abs{\int g \, dF_+}  + \abs{ \int g \, dF_-} \\
&\leq\int \abs{ g} \,
dF_++ \int \abs{ g} \, dF_- = \int \abs{g} \, \abs{dF}
\end{align*}
\end{proof}

In addition to functions of bounded variation providing signed
measures via the construction of Stieltjes measures integrals also
provide a source of functions of bounded variation.

\begin{defn}A function $F$ is \emph{absolutely continuous} on an
  interval $[a,b]$ if for every $\epsilon > 0$ there exists $\delta>0$
  such that for every $n > 0$ and every set of disjoint intervals $(a_j, b_j] \subset
  (a,b]$ for $j=1, \dotsc, n$ with $\sum_{j=1}^n (b_j-a_j) < \delta$
  we have $\sum_{j=1}^n \abs{F(b_j)-F(a_j)} < \epsilon$.
\end{defn}

\begin{lem}\label{AbsoluteContinuityImpliesContinuousBoundedVariation}If $F$ is absolutely continuous on $[a,b]$ then $F$ is
 uniformly continuous on $[a,b]$ and has bounded variation on $[a,b]$.
\end{lem}
\begin{proof}
The fact that $F$ is uniformly continuous is immediate by considering a single
subinterval of $[a,b]$.  Seeing that $F$ has bounded variation is
conceptually simple but notationally a little ugly.  The idea is simply
that any sufficiently fine partition of $[a,b]$ can be decomposed into
a union of subpartitions of a subinterval of length less than any
desired $\delta$; this is enough to bound the total variation. To see
the details, pick
$N > 0$ so that  $\sum_{j=1}^n (b_j-a_j) < (b-a)/N$ implies
$\sum_{j=1}^n \abs{F(b_j)-F(a_j)} < 1$.  First assume that we have a
partition $a=x_0 < x_1 < \dotsb < x_n=b$ such that for each $k=0,
\dotsc, N$ there is an $n_k$ with $x_{n_k} = (b-a)*k/N$. Then we
have $\sum_{j=n_{k-1}+1}^{n_k} (x_j - x_{j-1}) < \delta$ for
each $k$ and therefore
\begin{align*}
\sum_{j=1}^n \abs{F(b_j)-F(a_j)} &= \sum_{k=1}^{(b-a)/N}\sum_{j=n_{k-1}+1}^{n_k}
\abs{F(b_j)-F(a_j)} < (b-a) /N < \infty
\end{align*}
The assumption that $x_{n_k} = (b-a)*k/N$ can be arranged for by refining
an arbitrary partition and noting that the total variation can only increase by
doing so.
\end{proof}

To construct a general construction of absolutely continuous functions
from Stieltjes measures we first prove the
following fact about integrals on general measurable spaces.

\begin{lem}\label{LimitOfIntegralAsMeasureGoesToZero}Let $(S, \mathcal{S}, \mu)$ be a measure space and
  integrable function $f:S \to \reals$, then for
  every $\epsilon>0$ there exists a $\delta>0$ such that for all $A
  \in \mathcal{S}$ such that $\mu(A) < \delta$ we have $\abs{\int_A f\,
  d\mu} < \epsilon$.
\end{lem}
\begin{proof}
First assume that $f$ is a positive integrable function.  For each $n
> 0$ define $f_n = f \wedge n$ and note that $f_n \uparrow f$;
moreover $f_n \characteristic{A} \uparrow f \characteristic{A}$ for
every $A \in \mathcal{S}$.  By
Monotone Convergence we know that $\int_A f_n \, d\mu \uparrow \int_A
f \, d\mu$.  Let $\epsilon > 0$ be given and choose $N > 0$ such that
$\int f \, d\mu - \epsilon/2 < \int f_N \, d\mu \leq \int f \, d\mu$.
Choose $\delta = \epsilon/2*N$ and note that if $\mu(A) < \delta$ then
\begin{align*}
\int_A f \, d\mu &= \int_A f_N \, d\mu + \int_A (f-f_N) \, d\mu \leq N
\mu(A) + \int (f-f_N) \, d\mu < \epsilon
\end{align*}

For general integrable $f$ simply note that $\abs{\int_A f \, d\mu}
\leq \int_A \abs{f} \, d\mu$ and apply the result just proved for
positive integrable functions.
\end{proof}

Specializing to the case of locally finite signed measures on $\reals$ we get
\begin{cor}\label{StieltjesIntegralBoundedVariationAndContinuous}Let $F$ be a right continuous function of bounded variation and let $g$ be a
  measurable function that is integrable with respect to $F$ then
  $\int_{-\infty}^t g \, dF$ has bounded variation.  If $F$ is also
  continuous then $\int_{-\infty}^t g \, dF$ is continuous.
\end{cor}
\begin{proof}
First assume that $F$ is
non-decreasing and right continuous.  If $g$ is integrable with respect to $F$ then
$\int_{-\infty}^t g_\pm \, dF$ is non-decreasing by monotonicity of integral and therefore
$\int_{-\infty}^t g \, dF = \int_{-\infty}^t g_+ \, dF -
\int_{-\infty}^t g_- \, dF$ is a difference of non-decreasing
functions and therefore is of bounded variation by Theorem \ref{BoundedVariationAsDifferenceOfMonotone}.  To extend to $F$ of
bounded variation, write 
\begin{align*}
\int_{-\infty}^t g \, dF &= \int_{-\infty}^t g_+ \, dF_+ +
\int_{-\infty}^t g_- dF_- -
\int_{-\infty}^t g_- \, dF_+ - \int_{-\infty}^t g_+ \, dF_-
\end{align*}
and apply Theorem  \ref{BoundedVariationAsDifferenceOfMonotone}.

Now suppose that $F$ is continuous and non-decreasing.  For $\epsilon > 0$, pick $\delta>0$ as in Lemma
\ref{LimitOfIntegralAsMeasureGoesToZero} and then as any union of
intervals is measurable we get $\sum_{j=1}^n (F(b_j)-F(a_j)) < \delta$
implies $\abs{\sum_{j=1}^n \int_{a_j}^{b_j} g \, dF} < \epsilon$.
Let $t$ be given and by continuity of $F$ pick $\rho > 0$ such that
$\abs{s - t} < \rho$ implies $\abs{F(s) - F(t)} < \delta$ and
therefore 
\begin{align*}
\abs{\int_{-\infty}^t g \, dF - \int_{-\infty}^s g \, dF} &=
\abs{\int_s^t g \, dF} < \epsilon
\end{align*}
and continuity at $t$ is proven.
\end{proof}

NOTE:  It is not the case that every continuous function of bounded
variation is absolutely continous.  For that to be true we need to add
the \emph{Lusin N property} that says every the image of every Lebesgue
null set is a null set.  It turns out that absolute continuity is
equivalent to continuity, bounded variation and the Lusin property.

TODO: Where to put this?
\begin{cor}[Integration By Parts]\label{IntegrationByParts}Suppose $f$
  and $g$ are absolutely continuous functions.  Then 
\begin{align*}
\int_a^b f^\prime g d\lambda
  = f(b)g(b) - f(a)g(a) - \int_a^b f g^\prime d \lambda
\end{align*}
\end{cor}

\section{Approximation By Smooth Functions}
In this section we discuss a technique for approximating arbitrary
measurable and integrable functions by smooth functions.  

Here is an alternative description of a smooth bump function.
Let 
\begin{align*}
f_1(x)  &= \begin{cases}
e^{-1/x} & \text{ for $x>0$} \\
0 & \text{for $x \leq 0$}
\end{cases}
\end{align*}
By induction we see that $\frac{d^n f_1}{dx^n} = p_n(1/x) e^{-1/x}$ where
$p_n(x)$ is polynomial of degree equal to $2n$; just calculate
\begin{align*}
\frac{d}{dx} p_n(1/x) e^{-1/x} &= p^\prime_n(1/x) (-1/x^2) e^{-1/x} + p_n(x) e^{-1/x} (1/x^2) = (1/x^2)(p_n(x) - p^\prime_n(1/x)) e^{-1/x}
\end{align*}
In particular, $f_1(x)$ is $C^\infty$ on $(0,\infty)$.  $f_1(x)$ is indentically zero on $(-\infty,0)$ hence it trivially $C^\infty$ there.
By L'Hopital's rule and writing $p_n(1/x) = q_n(x)/x^{2n}$ for a polynomial $q_n$ we have 
\begin{align*}
\lim_{x \to 0^+} \frac{f^{(n)}_1(x)}{x} &= \lim_{x \to 0^+} \frac{q_n(x) e^{-1/x}}{x^{2n+1}} = 0
\end{align*}
and trivially $\lim_{x \to 0^-} \frac{f^{(n)}_1(x)}{x} =0$ which shows us that $f_1$ is $C^\infty$ at $x=0$.

We define
\begin{align*}
f_2(x) = \frac{f_1(x)}{f_1(x) + f_1(1-x)}
\end{align*}
which is well defined since $f_1(x)$ is strictly positive on $(0,\infty)$ and $f_1(1-x)$ is strictly positive on $(-\infty,1)$.   Since $f_1(x)$ is strictly positive on $(0,\infty)$ 
the same is true of $f_2(x)$ and since $f_1(x)$ is zero on $(-\infty,0]$ the same is true of $f_2$.  Moreover since $f_1(x)$ is non-negative it follows that $0 \leq f_2(x) \leq 1$ on $(-\infty, \infty)$
and $f_2(x) = 1$ on $[1,\infty)$.  The fact that $f_2(x)$ is $C^\infty$ follows from induction using the quotient rule of calculus to see that there is a polynomial $r_n$ such that
\begin{align*}
\frac{d^n f_2}{dx^n}(x) &= \frac{r_n(f_1(x), \dotsc, f^{(n)}_1(x), f_1(1-x), \cdots, f^{(n)}_1(1-x))}{(f_1(x) + f_1(1-x))^{n+1}}
\end{align*}

TODO: Finish this...

To start, we establish the existence of an infinitely differentiable
function which is supported on the interval $[-1,1]$.

\begin{lem}\label{ExistenceOfBumpFunction}The function 
\begin{align*}
f(x) = \begin{cases}
e^\frac{-1}{1-x^2} & \abs{x} < 1\\
0 & \abs{x} \geq 1
\end{cases}
\end{align*} is compactly supported on $[-1,1]$ and has continuous
derivatives of all orders.
\end{lem}
\begin{proof}
It is clear from the definition that $f(x)$ is compactly supported on
$[-1,1]$.  To see that it has continuous derivatives of all orders we
use an induction to prove that for every $n\geq 0$, there exists a
polynomial $P_n(x)$ and a nonnegative integer $N_n$ such that 
\begin{align*}
f^{(n)}(x) = \frac{P_n(x)}{(1 - x^2)^{N_n}} e^\frac{-1}{1-x^2}
\end{align*}
Clearly this is true for $n=0$.  Supposing that it is true for $n >0$,
we calculate using the induction hypothesis, the product rule and
chain rule
\begin{align*}
f^{(n+1)}(x) &= \frac{d}{dx}\frac{P_n(x)}{(1 - x^2)^{N_n}}
e^\frac{-1}{1-x^2} \\
&= \frac{(1 - x^2)^{N_n} P_n^\prime(x) - P_n(x) N_n (1- x^2)^{N_n
    -1}}{(1 - x^2)^{2N_n}}e^\frac{-1}{1-x^2} + \frac{P_n(x)}{(1 - x^2)^{N_n}}\frac{-1}{1-x^2} \frac{-2x}{(1-x^2)^2}
e^\frac{-1}{1-x^2} \\
\end{align*}
which shows the result after creating a common denominator.

It is clear that the derivatives are continuous away from ${-1,1}$ so
it remains to show $\lim_{x \to -1^+} f^{(n)}(x) = 0$ and $\lim_{x \to
  1^-} f^{(n)}(x) = 0$.

Take the former limit.  We write $f^{(n)}(x) = \frac{P_n(x)}{(1 -
  x)^{N_n}(1 + x)^{N_n}} e^\frac{-1}{1-x^2}$ and note that


TODO: Show $\lim_{x \to -1} \frac{1}{(1 + x)^M} e^\frac{-1}{1-x^2} = 0$
for all $M \geq 0$.
\end{proof}

TODO: What is $\int f(x)?$

\begin{lem}\label{ApproximationByMollifiers}Let $\rho(x)$ be a positive
  function in $C^\infty_c(\reals)$ such
  that $\rho(x)$ is supported on $[-1,1]$ and $\int_{-\infty}^\infty
  \rho(x) \, dx = \int_{-1}^1
  \rho(x) \, dx= 1$.  Let $f : \reals \to \reals$ be a continuous function.  Define 
\begin{align*}
f_n(x) = n \int_{-n}^n \rho(n(x - y)) f(y) dy
\end{align*}
Then $f_n \in C_c^\infty(\reals)$, $f_n^{(m)}(x) = n \int_{-n}^n
\rho^{(m)}(n(x -y)) f(y) dy$ and $f_n$ converges to $f$ uniformly on
compact sets.  Furthermore, if $f$ is bounded then $\norm{f_n}_\infty \leq
\norm{f}_\infty$.
\end{lem}
\begin{proof}
First note that because $\rho(x)$ and all of its derivatives are
compactly supported, they are also bounded.  In particular, there is
an $M > 0$ such that $\abs{\rho^\prime(x)} \leq M$.  To clean up the notation
a little bit, define $\rho_n(y) = n\rho(ny)$ so we have
\begin{align*}
f_n(x) &= \int_{-n}^{n} \rho_n(x - y) f(y) dy
\end{align*}
Since the support of $\rho_n(x)$ is contained in
$[-\frac{1}{n},\frac{1}{n}]$, if we fix $x \in \reals$ and
view $\rho_n(x - y)$ as a  function of $y$, its support is
contained in $[x-\frac{1}{n},x+\frac{1}{n}]$.  Thus the support of
$f_n(x)$ is contained in  $[-n-\frac{1}{n},n+\frac{1}{n}]$.

To examine the derivative
of $f_n(x)$, pick $h > 0$ and consider the difference
quotient
\begin{align*}
\frac{f_n(x + h) - f_n(x)}{h} &= \frac{1}{h} \int_{-n}^{n}
(\rho_n(x+h - y) - \rho_n(x - y) ) f(y) dy
\end{align*}
Taylor's Theorem tells us that $\frac{1}{h}(\rho_n(x+h - y) - \rho_n(x - y)) =
\rho_n^\prime(c)$ for some $c \in [x+h - y, x - y]$.  Therefore,
$\abs{\frac{1}{h} (\rho_n(x+h - y) - \rho_n(x - y))f(y)} \leq M\abs{f(y)}$ and by
integrability of $f(y)$ on the interval $[-n,n]$ (i.e. the
integrability of $f(y) \cdot \characteristic{[-n,n]}(y)$ which follows
rom the boundedness of $f(y)$ on the compact set $[-n,n]$) we may use Dominated Convergence to conclude
that 
\begin{align*}
f_n^\prime(x) &=\lim_{h \to 0} \frac{f_n(x + h) - f_n(x)}{h} \\
&= \lim_{h \to 0} \frac{1}{h} \int_{-n}^{n}
(\rho_n(x+h - y) - \rho_n(x - y) ) f(y) dy \\
&= \int_{-n}^{n }
\lim_{h \to 0} \frac{1}{h} (\rho_n(x+h - y) - \rho_n(x - y) ) f(y) dy
\\
&= \int_{-n}^{n} \rho^\prime_n(x - y) f(y) dy 
\end{align*} 
Continuity of $f_n^\prime(x)$ follows from the continuity of $f(y)$
and $\rho^\prime_n(x - y) $ and Dominated Convergence as above.  A simple induction extends the result to derivatives of arbitrary
order.

Next we show the convergence.  Pick a compact set $K \subset \reals$
and  $\epsilon > 0$,
Since $f$ is uniformly continuous on $K$,  there is a $\delta >0$ such
that for any $x \in K$ we have $\abs{x - y} \leq \delta$ implies
$\abs{f(x) - f(y)} \leq \epsilon$.  Pick $N_1 > 0$ such that
$\frac{1}{n} < \delta$ for all $n \geq N_1$.  The hypothesis $\int_{-\infty}^\infty
\rho(y) \, dy = \int_{-1}^1
\rho(y) \, dy = 1$ and simple change of variables shows $\int_{-\infty}^\infty
\rho_n(x - y) \, dy = \int_{x - \frac{1}{n}}^{x + \frac{1}{n}}
\rho_n(x - y) \, dy = 1$ for all $x \in \reals$ and $n > 0$.  Pick $N_2>0$ so that for all $n > N_2$,
we have $K \subset [-n + \frac{1}{n}, n - \frac{1}{n}]$.  Therefore we can write $f(x) = \int_{- n}^{n}
\rho_n(x - y) f(x)  \, dy = 1$ for any $x \in K$ and $n > N_2$.  We
have for any $n \geq \max(N_1, N_2)$
\begin{align*}
\abs{f_n(x) - f(x)} &= \abs{\int_{- n}^{n} (\rho_n(x -
  y)f(y) -
\rho_n(x- y) f(x)) \, dy} \\
&= \abs{\int_{x - \frac{1}{n}}^{x + \frac{1}{n}} (\rho_n(x -
  y)f(y) -
\rho_n(x- y) f(x)) \, dy}  & & \text{since $n>N_2$}\\
&\leq \int_{x - \frac{1}{n}}^{x + \frac{1}{n}} \rho_n(x- y) \abs{f(y)
  - f(x)} \, dy\\
&\leq \epsilon \int_{x -\frac{1}{n}}^{x + \frac{1}{n}} \rho_n(x -  y)
\, dy & &\text{since $\frac{1}{n} < \delta$}\\
&\leq \epsilon & & \text{since $\rho_n$ is positive and
  $\int_{-\infty}^\infty \rho_n(x) \, dx = 1$}
\end{align*}

The last thing to prove is the norm inequality in case $f$ is
bounded.  
\begin{align*}
\abs{f_n(x)} &\leq n \int_{-n}^n \rho(n(x - y)) \abs{ f(y)} dy & &
\text{because $\rho$ is positive} \\
&\leq n \norm{f}_\infty \int_{-\infty}^\infty \rho(n(x - y)) dy = \norm{f}_\infty 
\end{align*}
\end{proof}

Here is a more refined version of the above approximation result.

\begin{defn}Let $(X,\tau, \mu)$ be a topological measure space then we say $f : X \to \reals$ is \emph{locally integrable} if 
$f$ is measurable and $\int_K \abs{f} \, d\mu < \infty$ for all compact sets $K \subset X$.
\end{defn}

\begin{defn}Let $U \subset \reals^d$ be an open set and let $f, g$ be locally integrable on $U$ we say that $g$ is the
\emph{weak partial derivative of $f$ with respect to $x_j$} if 
\begin{align*}
\int_U f(x) \frac{\partial \phi}{\partial x_j}(x) \, dx &= -\int_U f(x) g(x) \, dx
\end{align*}
for all $\phi \in C^1_c(U)$ (i.e. $\phi$ is continuously differentiable on $U$ and the closure of $\lbrace \phi > 0 \rbrace$ is a compact subset of
$U$).  When such a $g$ exists we usually write $g = \frac{\partial f}{\partial x_j}$.  If all the weak partial derivatives exist then we write
$Df = (\frac{\partial f}{\partial x_1}, \dotsc, \frac{\partial f}{\partial x_d})$.
\end{defn}

\begin{defn}Let $U \subset \reals^d$ be an open set and $1 \leq p < \infty$ we let $W^{1,p}(U)$ be the set of $f \in L^p(U)$ such that 
the weak partial derivatives $\frac{\partial f}{\partial x_j}$ exist and are in $L^p(U)$.  $W^{1,p}_{\loc}(U)$ is the set of measurable functions $f : U \to \reals$
such that $f \in W^{1,p}(V)$ for all $V \subset \overline{V} \subset U$ with $\overline{V}$ compact.
\end{defn}

TODO: Fix the references to $U_\epsilon$ below as I used the wrong definition (I used thickening not thinning).
\begin{thm}\label{ApproximationByStandardMollifier} Let $\rho(x)$ be a positive
  function in $C^\infty_c(\reals^d)$ such
  that $\rho(x)$ is supported on $\overline{B}(0, 1)$ and $\int
  \rho(x) \, dx = \int_{\overline{B}(0, 1)}
  \rho(x) \, dx= 1$.  
Let $U \subset \reals^d$ be an open set and $f : U \to \reals$ be a locally integrable function.  For $\epsilon > 0$ define
$U_\epsilon = \lbrace x \in U \mid d(x, U^c) > \epsilon \rbrace$, $\rho_\epsilon(x) = \frac{1}{\epsilon^d} \rho(\frac{x}{\epsilon})$ and 
\begin{align*}
f_\epsilon(x) &= \rho_\epsilon \ast f = \frac{1}{\epsilon^d} \int_U \rho(\frac{x - y}{\epsilon}) f(y) \, dy 
\end{align*}
Then 
\begin{itemize}
\item [(i)] $f_\epsilon \in C^\infty(U_\epsilon)$
\item [(ii)] If $f$ is continuous then $f_\epsilon$ converges to $f$ uniformly on
compact subsets of $U$.  
\item [(iii)] If $f \in L^p_{\loc}(U; \reals)$ then $f_\epsilon \to f$ in $L^p_{\loc}(U;\reals)$.
\item [(iv)] If $x \in U$ is a Lebesgue point of $f$ then $f_\epsilon(x) \to f(x)$.  In particular, $f_\epsilon \to f$ $\lambda^d$-a.s.
\item [(v)] If $f \in W^{1,p}_{\loc}(U, \reals)$ for some $1 \leq p < \infty$ then for $i = 1, \dotsc, d$ and all $x \in U_\epsilon$,
\begin{align*}
\frac{\partial f_\epsilon}{\partial x_i} (x) &= \rho_\epsilon \ast \frac{\partial f}{\partial x_i}(x)
\end{align*}
\item [(vi)] If $f \in W^{1,p}_{\loc}(U; \reals)$ for some $1 \leq p < \infty$ then $f_\epsilon \to f$ in $W^{1,p}(U;\reals)$.
\end{itemize}
\end{thm}
\begin{proof}
To see (i) we first prove the following

\begin{clm} Let $\eta_\epsilon \in C^1_c(\reals^d; \reals)$ such that $\eta_\epsilon$ is supported on $B(0,\epsilon)$ then $\eta_\epsilon \ast f$ is differentiable on $U_\epsilon$ and 
\begin{align*}
\frac{\partial \eta_\epsilon \ast f}{\partial x_i} = \frac{\partial \eta_\epsilon}{\partial x_i} \ast f \text{ for all $i=1, \dotsc, d$}
\end{align*}
\end{clm}
Let $x \in U_\epsilon$ then for $0 \leq \abs{h} < \epsilon - d(x,\overline{U})$ we have $x + h e_i \in U_\epsilon$ since for each $y \in \overline{U}$ we have
\begin{align*}
\norm{x+he_i - y} &\leq \norm{x- y} + \norm{he_i} \leq d(x,\overline{U}) + h < \epsilon 
\end{align*}
Form the open set $V = U \cap \cup_{0 \leq \abs{h} \leq \epsilon - d(x,\overline{U}) } B(x+he_i, \epsilon)$ and note that $V$ is bounded hence has compact closure.
We form the difference quotient for $0 < \abs{h} < \epsilon - d(x,\overline{U})$ and use the fact that as a function of $y$, the support of $\rho_\epsilon(x+he_i-y)$ 
is contained in $B(x+he_i,\epsilon) \subset V$,
\begin{align*}
\frac{\eta_\epsilon \ast f(x+he_i) - \eta_\epsilon \ast f(x)}{h} &= \int_U \frac{1}{h} \left (\eta_\epsilon(x+he_i-y)- \eta_\epsilon(x-y) \right ) f(y) \, dy \\
&= \int_{V } \frac{1}{h} \left (\eta_\epsilon(x+he_i-y)- \eta_\epsilon(x-y) \right ) f(y) \, dy \\
\end{align*}
For each fixed $y$ we know that 
\begin{align*}
\lim_{h \to 0} \frac{1}{h} \left (\eta_\epsilon(x+he_i-y)- \eta_\epsilon(x-y) \right ) = \frac{\partial \eta_\epsilon}{\partial x_i}(x-y)
\end{align*}
By the fact that $\eta$ is $C^1$ and has compact support we know that $\frac{\partial \eta}{\partial x_i}$ is continuous and has compact support hence $\norm{\frac{\partial \eta_\epsilon}{\partial x_i}}_\infty = \frac{1}{\epsilon^{d+1}}\norm{\frac{\partial \eta}{\partial x_i}}_\infty < \infty$.  By the Mean Value Theorem \ref{MeanValueTheoremBanachSpaces} (or really just the Fundamental Theorem of Calculus \ref{FundamentalTheoremOfCalculus} in this case)
\begin{align*}
\frac{1}{\abs{h}} \abs{\eta_\epsilon(x+he_i-y)- \eta_\epsilon(x-y)} 
&= \frac{1}{\abs{h}} \abs{\int_0^h \frac{\partial \eta_\epsilon}{\partial x_i}(x + te_i - y) \, dt} 
\leq \norm{\frac{\partial \eta_\epsilon}{\partial x_i}}_\infty 
\end{align*}
By the compactness of the closure of $V$ and the locally integrability of $f$ we know that 
\begin{align*}
\int_{V} \abs{f}(y) \, dy < \infty
\end{align*}
which allows us to use Dominated Convergence to conclude that
\begin{align*}
\int_V \frac{\partial \eta_\epsilon}{\partial x_i}(x-y) f(y) \, dy &= \lim_{h \to 0} \int_{V } \frac{1}{h} \left (\eta_\epsilon(x+he_i-y)- \eta_\epsilon(x-y) \right ) f(y) \, dy \\
&=\lim_{h \to 0} \frac{\eta_\epsilon \ast f(x+he_i) - \eta_\epsilon \ast f(x)}{h} = \frac{\partial \eta_\epsilon \ast f}{\partial x_i}(x)
\end{align*}
Note also that since the support of $\frac{\partial \eta_\epsilon}{\partial x_i}(x-y)$ is contained in $B(x, \epsilon) \subset V$ in fact we have 
\begin{align*}
\frac{\partial \eta_\epsilon \ast f}{\partial x_i}(x) &= \int_V \frac{\partial \eta_\epsilon}{\partial x_i}(x-y) f(y) \, dy \\
&=\int_U \frac{\partial \eta_\epsilon}{\partial x_i}(x-y) f(y) \, dy = \left(\frac{\partial \eta_\epsilon}{\partial x_i} \ast f \right) (x)
\end{align*}

Now to see (i) use the fact that $\rho_\epsilon \in C^\infty(\reals^d; \reals)$ and that for every multiindex $\alpha=(\alpha_1, \dotsc, \alpha_d)$ we know that the support of $\frac{\partial^\alpha \rho_\epsilon}{\partial x^\alpha}$ is contained in $B(0,\epsilon)$ to see that we may apply the previous claim inductively.
TODO: Do we need to restrict to $f$ with support strictly contained in $U$?  I don't think so.

To see (ii) suppose we are given $V \subset \overline{V}  \subset U$ with $\overline{V}$ compact.  Since $\reals^d$ is locally compact and Hausdorff  by Lemma \ref{BoundedNeighborhoodsOfCompactSets}  we may pick an open set $W$ such
that 
\begin{align*}
V &\subset \overline{V} \subset W \subset \subset{W} \subset U
\end{align*}
and $\overline{W}$ compact.
There exists a $\overline{\epsilon}$ such that for all $x \in V$ and $0<\epsilon<\overline{\epsilon}$ we have $B(x, \epsilon) \subset W \subset U$ (if not then by compactness of $\overline{V}$ there exist sequences $\epsilon_n \to 0$ and $x_n \to x$ in $\overline{V}$ with $B(x_n, \epsilon_n) \not \subset W$;  on the other hand since $x \in \overline{V}$ we can find $B(x,\epsilon) \subset W$ which is a contradiction since for $\epsilon_n < \epsilon/2$ and $x_n \in B(x,\epsilon/2)$ we have $B(x_n,\epsilon_n) \subset B(x,\epsilon)$).    By a linear change of variables $y = x-\epsilon z$ (Corollary \ref{LebesgueLinearChangeOfVariables})
\begin{align*}
f_\epsilon(x) &= \frac{1}{\epsilon^d} \int_U \rho(\frac{x-y}{\epsilon}) f(y) \, dy &= \frac{1}{\epsilon^d} \int_{B(x,\epsilon)} \rho(\frac{x-y}{\epsilon}) f(y) \, dy \\
&= \int_{B(0,1)} \rho(z) f(x-\epsilon z) \, dz 
\end{align*}
Now by the fact that $\int_{B(0,1)} \rho(z) \, dz=1$ we get for all $x \in V$ and $0 < \epsilon < \overline{\epsilon}$,
\begin{align*}
\abs{f_\epsilon(x) - f(x)} &\leq \int_{B(0,1)} \rho(z) \abs{f(x-\epsilon z) - f(x)} \, dz \leq \sup_{\substack{z,y \in \overline{W} \\ \norm{z-y} \leq \epsilon}} \abs{f(y) - f(z)}
\end{align*}
Since $f$ is continuous on $U$ and $\overline{W}$ is compact, $f$ is uniformly continuous on $\overline{W}$ (Theorem \ref{UniformContinuityOnCompactSets}) and therefore (ii) follows 
by letting $\epsilon \to 0$.

To see (iii) pick $V \subset \overline{V} \subset U$ with $\overline{V}$ compact and by the above arguments find an open set $W$ with $\overline{W}$ compact and $\overline{\epsilon}>0$ such
that
\begin{align*}
V &\subset \overline{V} \subset V_\epsilon \subset W \subset \overline{W} \subset U \text{ for all $0 < \epsilon < \overline{\epsilon}$}
\end{align*}
Using the change of variables from above and H\"{o}lder's Inequality we get for all $x \in V$, $0 < \epsilon < \overline{\epsilon}$, $1 < p < \infty$ and $\frac{1}{p} + \frac{1}{q} =1$,
\begin{align*}
\abs{f_\epsilon(x)} &\leq \int_{B(0,1)} \eta(z) \abs{f(x-\epsilon z)} \, dz = \int_{B(0,1)} \eta(z)^{1/q} \eta(z)^{1/p} \abs{f(x-\epsilon z)} \, dz \\
&\leq \left( \int_{B(0,1)} \eta(z) \, dz \right)^{1/q} \left(\int_{B(0,1)} \eta(z) \abs{f(x-\epsilon z)}^p \, dz\right)^{1/p} \\
&=\left(\int_{B(0,1)} \eta(z) \abs{f(x-\epsilon z)}^p \, dz \right)^{1/p}
\end{align*}
Therefore for $1 \leq p < \infty$ and $0 < \epsilon < \overline{\epsilon}$ we get by Tonelli's Theorem, the fact that $V_\epsilon \subset W$ and $\int_{B(0,1)} \eta(z) \, dz=1$ to see
\begin{align*}
\int_V \abs{f_\epsilon(x)} \, dx &\leq \int_{B(0,1)} \eta(z) \left( \int_V \abs{f(x-\epsilon z)}^p \, dx \right) \, dz \\
&\leq \int_W \abs{f(x)} \, dx
\end{align*}
There exist a sequence $f_n \in C(W)$ such that $\lim_{n \to \infty} \norm{f - f_n}_{L^p(W)} = 0$ (TODO: This follows from the fact that $\lambda^d(W) < \infty$ and the outer regularity of $\lambda^d$; give the details somewhere).  
\begin{align*}
\norm{f_\epsilon - f}_{L^p(V)} &\leq \norm{f_\epsilon - f_{n,\epsilon}}_{L^p(V)} + \norm{f_n - f_{n,\epsilon}}_{L^p(V)} + \norm{f - f_n}_{L^p(V)} \\
&\leq 2 \norm{f - f_{n}}_{L^p(W)} + \sup_{x \in \overline{V}} \abs{f_n(x) - f_{n,\epsilon}(x)}^p\lambda^d(V)  \\
\end{align*}
Now conclude (iii) by letting $\epsilon \to 0$ and then $n \to \infty$ using (ii) applied to $f_n$.  

To see (iv) suppose that $f$ is locally integrable and let $x \in U$ be a Lebesgue point (i.e. $\lim_{r \to 0} \frac{1}{\lambda^d(B(x,r))} \int_{B(x,r)} f(y) \, dy = f(x)$).  Pick $\overline{\epsilon}>0$ such that $B(x,\overline{\epsilon}) \subset U$.  Then using 
$\frac{1}{\epsilon^d} \int_{B(x,\epsilon)} \rho(\frac{x-y}{\epsilon} \, dy = 1$ we see for all $0 < \epsilon \leq \overline{\epsilon}$,
\begin{align*}
\abs{f_\epsilon(x) - f(x)} &= \frac{1}{\epsilon^d} \abs{\int_{B(x,\epsilon)} \rho (\frac{x-y}{\epsilon}) (f(y) - f(x)) \, dy} \\
&\leq \frac{1}{\epsilon^d} \int_{B(x,\epsilon)} \rho (\frac{x-y}{\epsilon}) \abs{ f(y) - f(x)} \, dy \\
&\leq \lambda^d(B(0,1)) \norm{rho}_\infty \frac{1}{\lambda^d(B(x, \epsilon))} \int_{B(x,\epsilon)} \abs{ f(y) - f(x)} \, dy
\end{align*}
Now let $\epsilon \to 0$ and use the fact that $x$ is a Lebesgue point to see that $\lim_{\epsilon \to 0} \abs{f_\epsilon(x) - f(x)}=0$.

To see (v) let $f \in W^{1,p}_{\loc}(U)$ for some $1 \leq p < \infty$.  Letting $x \in U_\epsilon$ it follows that when viewed as a function of $y$, $\rho_\epsilon(x-y) \in C^1_c(U)$.  Using (i), the chain rule and the definition of the weak derivative of $f$ we get
\begin{align*}
\frac{\partial f_\epsilon}{\partial x_j}(x) &= \int_U \frac{\partial \rho_\epsilon}{\partial x_j}(x-y) f(y) \, dy 
= -\int_U \frac{\partial \rho_\epsilon}{\partial y_j}(x-y) f(y) \, dy \\
&= \int_U \rho_\epsilon(x-y) \frac{\partial f}{\partial y_j}(y) \, dy 
= \rho_\epsilon \ast \frac{\partial f}{\partial y_j}(x)
\end{align*}

To see (vi) simply note that if $f \in W^{1,p}_{\loc}(U)$ and $V \subset \overline{V} \subset U$ with $\overline{V}$ compact then $f$ and $Df$ are in $L^p(V)$ and by (iii) we know that $\lim_{\epsilon \to 0} \norm{\rho_\epsilon \ast f - f}_{L^p(V)} = 0$ and $\lim_{\epsilon \to 0} \norm{\rho_\epsilon \ast Df - Df}_{L^p(V)} = 0$.  On the other hand, by (v) we know that $\rho_\epsilon \ast Df = D (\rho_\epsilon \ast f)$ on $U_\epsilon$  and since $\overline{V} \subset U$ and $\overline{V}$ is compact we know that $d(\overline{V}, U^c) = \inf_{x \in \overline{V}} d(x, U^c) > 0$ (by Theorem \ref{ExistenceGlobalMinimizerContinuous} there exists an $x^* \in \overline{V} \subset U$ such that $\inf_{x \in \overline{V}} d(x, U^c) = d(x^*, U^c)>0$) and for $0<\epsilon <d(\overline{V}, \partial U)$ we also have $V \subset U_\epsilon$.  Hence
\begin{align*}
\lim_{\epsilon \to 0} \norm{f_\epsilon - f}^p_{W^{1,p}(V)} &= \lim_{\epsilon \to 0} (\norm{f_\epsilon - f}^p_{L^{p}(V)} + \norm{Df_\epsilon - Df}^p_{L^{p}(V)}) = 0
\end{align*}
\end{proof}

Approximation by convolution with a compactly supported bump function
is usually sufficient for our purposes, however it is also useful to
replace the bump function with Gaussians.  

We will need the following fact that is a standard exercise from multivariate calculus
\begin{lem}\label{IntegralGaussian}$\int_{-\infty}^\infty e^{-x^2/2}\, dx = \sqrt{2\pi}$.
\end{lem}
\begin{proof}
By Tonelli's Theorem,
\begin{align*}
\int_{-\infty}^\infty \int_{-\infty}^\infty e^{-(x^2 + y^2)/2} \, dxdy
&= \int_{-\infty}^\infty e^{-x^2/2} \,
dx \int_{-\infty}^\infty e^{-y^2/2} \,
dy = \left(int_{-\infty}^\infty e^{-x^2/2} \,
dx \right)^2
\end{align*}
However, if we switch to polar coordinates and Tonelli's Theorem,
\begin{align*}
\int_{-\infty}^\infty \int_{-\infty}^\infty e^{-(x^2 + y^2)/2} \, dxdy
&= \int_{0}^{2\pi} \int_{0}^\infty e^{-r^2/2} r \, dr d\theta =
\int_{0}^{2\pi} d\theta = 2\pi
\end{align*}
and we are done.
\end{proof}

Now we can see that we may uniformly approximate compactly supported continuous
functions by convolution with Gaussians.
\begin{lem}\label{UniformApproximationByGaussians}Define $\rho(x) =
  \frac{1}{\sqrt{2 \pi} }e^{-x^2/2}$ and let $\rho_n(x) = n\rho(nx)$.
Let $f \in C_c(\reals)$ then define $f_n(x) = (f *
  \rho_n)(x)$.  Then $f_n(x) \in C_c^\infty(\reals)$ and $f_n$
  converges to $f$ uniformly.
\end{lem}
\begin{proof}
The proof is rather similar to that in the preceeding Lemma
\ref{ApproximationByMollifiers}.  By simple change of variables and
Lemma \ref{IntegralGaussian} we see
that $\int_{-\infty}^{\infty} \rho_n(y) \, dy =
\int_{-\infty}^{\infty} \rho_n(x - y) \, dy = 1$ and therefore we
have the trivial identity $f(x) = \int_{-\infty}^{\infty} f(x)
\rho_n(x - y) \, dy$.  Because $f$ has compact support, we know that
$f$ is uniformly continuous, so given $\epsilon  > 0$ we can find
$\delta > 0$ such that $\abs{x -y} < \delta$ implies $\abs{f(x) -f(y)}
< \epsilon$.  Similarly, by compact support $f$ is bounded and we may
assume $f(x) < M$ for some $M > 0$.  Assume we are given
$\epsilon > 0$ then take $\delta>0$ as above and for any $n>0$ we have
Therefore
\begin{align*}
\abs{f*\rho_n(x) - f(x)} &= \abs{\int_{-\infty}^{\infty} \rho_n(x-y)
  (f(y) - f(x)) \, dy} \\
&\leq \int_{\abs{x-y} < \delta} \rho_n(x-y)
  \abs{  f(y) - f(x) } \, dy + \int_{\abs{x-y} \geq \delta} \rho_n(x-y)
  \abs{ f(y) - f(x) } \, dy \\
&\leq \epsilon + 2M \int_{\abs{x-y} \geq \delta} \rho_n(x-y) \, dy
\end{align*}
Now we consider the last term and change integration variables
\begin{align*}
\int_{\abs{x-y} \geq \delta} \rho_n(x-y) \, dy &= \int_{\abs{y} \geq
  \delta} \rho_n(y) \, dy \\
&= \frac{1}{\sqrt{2 \pi}} \int_{\abs{y} \geq
  n\delta} e^{-y^2/2} \, dy \\
&\leq \frac{2}{\sqrt{2 \pi}} \int_{n\delta}^\infty \frac{y}{n\delta}
e^{-y^2/2} \, dy \\
&= \frac{2}{n \delta \sqrt{2 \pi}} e^{-n^2\delta^2/2}
\end{align*}
One point here is the elementary fact that $\lim_{n \to \infty} \frac{2}{n \delta
  \sqrt{2 \pi}} e^{-n^2\delta^2/2} = 0$ but the second point is that this limit does
not depend on $x$.   Thus we may pick $N > 0$ independent of $x$, such
that $\int_{\abs{x-y} \geq \delta} \rho_n(x-y) \, dy  <
\frac{\epsilon}{2M}$ for $n > N$ and therefore
\begin{align*}
\abs{f*\rho_n(x) - f(x)} < 2 \epsilon
\end{align*}
which proves the uniform convergence of $f*\rho_n$.
\end{proof}

\section{Daniell-Stone Integrals}

We record some required facts about $\sigma$-rings that are completely
analogous to corresponding facts about $\sigma$-algebras.

\begin{lem}Let $X$ be a topological space and let $\mathcal{B}(X)$ be
  the Borel $\sigma$-algebra on $X$.  If $A$ is a Borel set then
  $\lbrace B \in \mathcal{B}(X) \mid B \subset A \rbrace$ is a
  $\sigma$-ring of sets in $X$.
\end{lem}
\begin{proof}
Clearly it contains the empty set and is closed under countable
union.  To see that it is closed under set difference simply note $B
\setminus C = B \cap C^c \subset B \subset A$ and is clearly a Borel
set of $X$.  
\end{proof}
Note that in fact the set of sets in the previous Lemma is the Borel
$\sigma$-algebra of $A$ with the induced topology.
By virtue of the above Lemma we will refer to the $\sigma$-ring of
Borel sets on $\reals$ that do not contain $0$ as $\mathcal{B}(\reals
\setminus \lbrace 0 \rbrace)$.  This is at the risk of potential
confusion about whether we are considering this a $\sigma$-ring of
subsets of $\reals$ or a $\sigma$-algebra of subsets of $\reals
\setminus \lbrace 0 \rbrace$; pretty much we always have the former
interpretation in mind.  Our first order of business is to establish a
simple generating set for Borel $\sigma$-ring on $\reals$.

\begin{lem}\label{IntervalsGenerateBorelPunctured}The $\sigma$-ring of Borel sets of $\reals$ that do not
  contain $0$ is generated by intervals $(-\infty, -c)$ and $(c,
  \infty)$ with $c > 0$.
\end{lem}
\begin{proof}
As noted above the $\sigma$-ring in the statement of the Lemma is the
$\sigma$-algebra of $\reals \setminus \lbrace 0 \rbrace$ in the
induced topology.  We know that open sets of $\reals$ are precisely
countable disjoint unions of open intervals (Lemma
\ref{OpenSetsOfReals}).  For any open interval $(a,b)$ we either have
$(a,b) \subset \reals \setminus \lbrace 0 \rbrace$ or $a < 0 < b$
hence $(a,b) \cap \reals \setminus \lbrace 0 \rbrace = (a,0) \cup
(0,b)$.  We conclude that the open sets of $\reals \setminus \lbrace 0
\rbrace$ are countable disjoint unions of open intervals none of which
contains $0$.  Now one can adapt the proof of Lemma
\ref{IntervalsGenerateBorel} to get the result.
\end{proof}

One of the most often used facts from measure theory is the fact that
measurable functions may be approximated by simple functions (Lemma
\ref{PointwiseApproximationBySimple}).  We need a small refinement of
that Lemma that applies with $\sigma$-rings.
\begin{lem}\label{PointwiseApproximationBySimpleSigmaRing}For any
  function $f : \Omega \to \overline{\reals}_+$ 
measurable with respect to a $\sigma$-ring $\mathcal{R}$, there exist a sequence of simple 
  functions $f_1, f_2, \dots$ measurable with respect to $\mathcal{R}$
  such that $0 \leq f_n \uparrow f$.
\end{lem}
\begin{proof}
Recalling the proof of \ref{PointwiseApproximationBySimple}, define
\begin{align*}
f_n(\omega) = 
\begin{cases}k2^{-n} & \text{if $k2^{-n} \leq f(\omega) < (k+1)2^{-n}$
    and $0 \leq k \leq n2^n -1$.} \\
n & \text{if $f(\omega) \geq n$.}
\end{cases}
\end{align*}
and we know that $f_n$ are simple functions $f_n \uparrow f$.  The
only thing to prove is that the $f_n$ are $\mathcal{R}$-measurable;
this follows because each preimage of $f_n$ is either of the form
$f^{-1}([k2^{-n}, (k+1)2^{-n}))$, for $k =0, \dotsc, n2^n -1$ or
$f^{-1}([n, \infty))$ and $f_n=0$ precisely on $f^{-1}([0,1/2^n))$.
Therefore every preimage of a set in $\mathcal{B}(\reals \setminus
\lbrace 0 \rbrace)$ is a union of sets $f^{-1}([k2^{-n}, (k+1)2^{-n}))$, for $k =1, \dotsc, n2^n -1$ or
$f^{-1}([n, \infty))$ and is therefore in $\mathcal{R}$ by the
$\mathcal{R}$-measurability of $f$.
\end{proof}

TODO:  Introduce notation for the $\sigma$-ring generated by a
set of sets.

\begin{lem}\label{SigmaRingPullbackGenerators}Let $f : S \to T$ be a set mapping and let $\mathcal{C}
  \subset 2^T$, then the $\sigma$-ring generated by
  $f^{-1}(\mathcal{C})$ is the same as the pullback of the
  $\sigma$-ring generated by $\mathcal{C}$.
\end{lem}
\begin{proof}
It is clear that the $\sigma$-ring generated by $f^{-1}(\mathcal{C})$
is contained in the pullback of the $\sigma$-ring generated by
$\mathcal{C}$.  To see the reverse conclusion, pushforward the
$\sigma$-ring generated by $f^{-1}(\mathcal{C})$; this is equal to
$\lbrace A \subset T \mid f^{-1}(A) \text{ is in the $\sigma$-ring
  generated by } f^{-1}(\mathcal{C})\rbrace$ and is itself a $\sigma$-ring
(Lemma \ref{SigmaRingPullback}).  It clearly contains $\mathcal{C}$ and therefore
the $\sigma$-ring generated by $\mathcal{C}$ as well.  Therefore the
pullback of the $\sigma$-ring generated by $\mathcal{C}$ is contained
in $\sigma$-generated by $f^{-1}(\mathcal{C})$ and we are done.
\end{proof}

It turns out that having a countably additive set function on a
$\sigma$-ring is almost the same thing as having a measure on the
generated $\sigma$-algebra.  This fact is made precise by the
following result.
\begin{lem}\label{ExtendingMeasuresFromSigmaRing}Let $\mathcal{R}$ be a $\sigma$-ring on a set $S$ and let
  $\mu : \mathcal{R} \to \overline{\reals}_+$ be a function that is
  countably additive on disjoint sets.  Let $\mu_*(E) = \sup \lbrace
  \mu(A) \mid A
  \subset E \text{ and } A \in \mathcal{R} \rbrace$ be the inner measure
  defined by $\mu$ on all of $2^S$.  Let $\mathcal{A} = \mathcal{R}
  \cup \mathcal{R}^c$ be the $\sigma$-algebra generated by
  $\mathcal{R}$.
\begin{itemize}
\item[(i)]If we define $\tilde{\mu}(A) = \mu(A)$ for $A \in
  \mathcal{R}$ and $\tilde{\mu}(A) = \infty$ for $A \in \mathcal{R}^c$
  then $\tilde{\mu}$ is a measure on $\mathcal{A}$.
\item[(ii)]For any $b \in \overline{\reals}_+$ if we define
  $\tilde{\mu}(A) = \mu(A)$ for $A \in
  \mathcal{R}$ and $\tilde{\mu}(A) = \mu_*(A) + b$ for $A \in \mathcal{R}^c$
  then $\tilde{\mu}$ is a measure on $\mathcal{A}$.
\item[(iii)]Every measure on $\mathcal{A}$ that extends $\mu$ on
  $\mathcal{R}$ is of the above form.
\item[(iv)]$\mu$ has a unique extension to $\mathcal{A}$ if and only
  if $\mathcal{R} = \mathcal{A}$ or $\mu_*(A) = \infty$ for every $A \in \mathcal{R}^c$.
\end{itemize}
\end{lem}
\begin{proof}
There is nothing to prove if $\mathcal{R} =
\mathcal{A}$ so we assume otherwise.  Note that in this case there are no
disjoint sets in $\mathcal{R}^c$ (if $A,B \in \mathcal{R}^c$ satisfy
$A \cap B = \emptyset$ then taking complements $A^c \cup B^c = S$
which shows $S \in \mathcal{R}$ which implies $\mathcal{R} =
\mathcal{A}$).

To prove the that the proposed set functions are measures we only need
to show countable additivity over all of $\mathcal{A}$.  By the above
comment we can assume that we have $A_1, A_2, \dots \in \mathcal{R}$
and $A_0 \in \mathcal{R}^c$ which are all disjoint.  Recall that $\cup_{i=0}^\infty
A_i \in \mathcal{R}^c$.  For (i) we have
\begin{align*}
\infty &= \tilde{\mu}(\cup_{i=0}^\infty A_i ) & &\text{by
  definition of $\tilde{\mu}$ on $\mathcal{R}^c$} \\
&=\sum_{i=0}^\infty \mu(A_i) & & \text{since $\tilde{\mu}(A_0)=\infty$}
\end{align*}

For (ii) things are a little more complicated.  First we handle the
case of $b=0$.  Since for any $A \in \mathcal{R}$ we have $\mu_*(A) =
\mu(A)$ we simplify notation and let the extension be denoted by $\mu_*$.  Note that 
for any $\epsilon > 0$ we can find $B_0 \in \mathcal{R}$ such that
$B_0 \subset A_0$ and $\mu(B_0) \geq \mu_*(A_0) -\epsilon$.  Then if
we define $B_i = A_i$ for $i=1,2,\dotsc$ we have the $B_i$ are all disjoint sets in
$\mathcal{R}$ and $\cup_{i=0}^\infty B_i \subset \cup_{i=0}^\infty A_i $.  Therefore
\begin{align*}
\mu_*(\cup_{i=0}^\infty A_i) &= \sup \lbrace \mu(C) \mid C \subset
\cup_{i=0}^\infty A_i \text{ and } C \in \mathcal{R}\rbrace \\
&\geq \mu(\cup_{i=0}^\infty B_i) \\
&= \sum_{i=0}^\infty \mu(B_i) \\
&\geq \sum_{i=0}^\infty \mu_*(A_i) - \epsilon \\
\end{align*}
Since $\epsilon$ was arbitrary we conclude
$\mu_*(\cup_{i=0}^\infty A_i) \geq \sum_{i=0}^\infty \mu_*(A_i)$.

To see the other inequality, for any $\epsilon >0$ we can pick $C \in
\mathcal{R}$ such that $C \subset \cup_{i=0}^\infty A_i$ and $\mu(C)
\geq \mu_*(\cup_{i=0}^\infty A_i) - \epsilon$.  Since $A_0 \in
\mathcal{R}^c$ there is a $B_0 \in \mathcal{R}$ such that $A_0 =
B_0^c$ and therefore $C \cap A_0 = C \cap B_0^c = C\setminus B_0 \in
\mathcal{R}$.  Because $A_i \in \mathcal{R}$ for $i=1,2,\dotsc$ we
know that $A_i \cap C \in \mathcal{R}$ for $i=1,2, \dotsc$.  Putting
these two observations together we know can write $C =
\cup_{i=0}^\infty C_i$ where each $C_i = C \cap A_i \in \mathcal{R}$,
$C_i \subset A_i$ and $C_i$ are disjoint.  Now applying the definition
of $\mu_*$ and countable additivity and monotonicity of $\mu$ we see
\begin{align*}
\mu_*(\cup_{i=0}^\infty A_i) - \epsilon &\leq \mu(C) = \sum_{i=0}^\infty \mu(C_i) \leq \sum_{i=0}^\infty \mu_*(A_i)
\end{align*}
Since $\epsilon > 0$ was arbitrary we conclude
$\mu_*(\cup_{i=0}^\infty A_i)  \leq \sum_{i=0}^\infty
\mu_*(A_i)$ and therefore we have proven $\mu_*(\cup_{i=0}^\infty A_i)  = \sum_{i=0}^\infty
\mu_*(A_i)$.

Now we extend the argument to see that defining $\tilde{\mu}(A) =
\mu_*(A) + b$ on $\mathcal{R}^c$ also defines a measure.  Once again
only countable additivity needs to be shown.  As noted
$\cup_{i=0}^\infty A_i \in \mathcal{R}^c$ so using what we have just
proven for $\mu_*$,
\begin{align*}
\tilde{\mu}(\cup_{i=0}^\infty A_i) &= \mu_*(\cup_{i=0}^\infty A_i ) +
b = \mu_*(A_0) + \sum_{i=1}^\infty \mu(A_i) + b = \sum_{i=0}^\infty \tilde{\mu}(A_i)
\end{align*}

To see (iii) we must show that every extension of $\mu$ to
$\mathcal{A}$ has the form $\mu_* + b$ on $\mathcal{R}^c$ for a particular $b \in
\overline{\reals}_+$.  Let $\tilde{\mu}$ be an extension of $\mu$ to
$\mathcal{A}$.  Suppose we have $A_1, A_2 \in
\mathcal{R}^c$.  From monotonicity we know that
$\mu_*(A) \leq \tilde{\mu}(A)$ for every $A\in \mathcal{R}^c$.  So
there exists constants $b_1, b_2 \in \overline{\reals}_+$ such that
$\tilde{\mu}(A_i) = \mu_*(A_i) + b_i$ for $i=1,2$ and we need to show
that $b_1 = b_2$.   In addition since $A_1 \cup A_2 \in \mathcal{R}^c$, there is a $b$ such that
$\tilde{\mu}(A_1 \cup A_2) = \mu_*(A_1 \cup A_2) + b$.  Note that $A_2 \setminus A_1 = A_2 \cap A_1^c = A_1^c
\setminus A_2^c \in \mathcal{R}$ therefore
\begin{align*}
\mu_*(A_1 \cup A_2) + b &= \tilde{\mu}(A_1 \cup A_2) =
\tilde{\mu}(A_1) + \tilde{\mu}(A_2 \setminus A_1) = \mu_*(A_1) + b_1 +
\mu_*(A_2 \setminus A_1)
\end{align*}
which implies $b = b_1$ since $\mu_*$ is a measure.  The same argument
shows that $b = b_2$ hence we see that $b_1 = b_2$ and we are done.

The claim in (iv) is direct consequence of what we have shown.  If
$\mu_*(A) \neq \infty$ for some $A \in \mathcal{R}^c$ then we have
constructed a uncountably infinite number of distinct extension of
$\mu$ given by $\mu_* + b$ on $\mathcal{R}^c$.  On the other hand if
$\mu_*(A) = \infty$ for all $A \in \mathcal{R}^c$ then we know any
extension must be of the form $\mu_* + b$ on $\mathcal{R}^c$ but these
are all equal to $\infty$ so the uniqueness of the extension is established.
\end{proof}

\begin{examp}It is instructive to consider the scenario of the
  previous Lemma in the context of the specific example of the
  $\sigma$-ring generated by taking the set of Borel sets on $\reals$
  that do not contain $0$ and Lebesgue measure.  We are clearly in the
  non-unique case with this example and the different extensions
  correspond to putting point masses with different weights at $0$.
\end{examp}
We have developed tools that enable us to define measures based on
more primitive set functions and this has allowed us to create very
important measures such as Lebesgue measure on $\reals$.  There is
another broad class of results that exist that allow one to construct
measures.  The basic observation is that a measure begets an integral
that is a linear function from measurable functions to the extended
reals hence it makes sense to pose the question of when a linear
functional on some set of measurable functions arises from a measure.
Being in possession of such results we are in a position to construct
measures by constructing linear functionals instead.  In all cases the
results in the space make some assumptions about the space of
measurable functions on which the functional is defined.  In this
section we consider the first result in this class; one that is
distinguished by the fact that it works on general spaces that do not
possess any topological structure.

\begin{defn}Let $\mathcal{L}$ be a real vector space of real valued
  functions on a set $\Omega$.  We say $\mathcal{L}$ is a \emph{vector
    lattice} if given any $f,g \in \mathcal{L}$ we have $f \vee g \in
  \mathcal{L}$ and $f \wedge g \in \mathcal{L}$.
\end{defn}

\begin{prop}If $\mathcal{L}$ is a real vector space of real valued
  functions on a set $\Omega$ such that for any $f,g \in \mathcal{L}$
  we have $f \vee g \in \mathcal{L}$ then $\mathcal{L}$ is a vector lattice.
\end{prop}
\begin{proof}
Simply note that $f \wedge g = -(-f \vee -g)$.  
\end{proof}

\begin{defn}Given a set $\Omega$ and a vector lattice $\mathcal{L}$ of
  real functions on $\Omega$ a \emph{pre-integral} is a linear
  function $I : \mathcal{L} \to \reals$ such that 
\begin{itemize}
\item[(i)]if $f \in \mathcal{L}$ and $f \geq 0$ then $I(f) \geq 0$
\item[(ii)]if $f_1, f_2, \dots \in \mathcal{L}$ such that $f_n
  \downarrow 0$ then $I(f_n) \downarrow 0$.
\end{itemize}
\end{defn}

To construct a measure that corresponds to a pre-integral we make an
intermediate step using the interpretation of an integral as the area
under a curve.  This will provide us with a measure on the product
space $\Omega \times \reals$ and then we will show how we restrict
this measure in an appropriate way to construct the measure that
generates an integral equivalent to $I$.

\begin{thm}\label{Zaanen}Let $\mathcal{L}$ be a vector lattice of
  functions on a set $S$ with a pre-integral $I$.  For any $f, g \in \mathcal{L}$ such
  that $f \leq g$ we define
\begin{align*}
[f, g) &= \lbrace (s, t) \in S \times \reals \mid f(s)
\leq t < g(s) \rbrace
\end{align*}
, $\mathcal{D} = \lbrace [f,g) \mid f,g \in \mathcal{L}
\text{ such that } f\leq g\rbrace$ and $\nu([f,g)) = I(g - f)$.  Then
$\nu$ is countably additive on $\mathcal{D}$ and extends to a measure on the
$\sigma$-algebra generated by $\mathcal{D}$.

For every $c > 0$, we let $M_c : S \times \reals \to S \times
\reals$ be the mapping $M_c(s,t) = (s, ct)$.  Then $M_c^{-1} : 2^{S
  \times \reals} \to 2^{S \times \reals}$ restricts to a bijection on
the $\sigma$-algebra generated by $\mathcal{D}$ and furthermore for
every set $A \in \sigma(\mathcal{D})$ and $c >0$ we have $c
\nu(M_c^{-1} A) = \nu(A)$.
\end{thm}
\begin{proof}
The proof proceeds by showing that $\mathcal{D}$ is a semiring, that
$\nu$ is countably additive on $\mathcal{D}$ and by applying Lemma (TODO:).

Let $c > 0$ and consider the mapping $M_c(s,t) = (s, ct)$.  

Claim 1: $M_c^{-1}(\sigma(\mathcal{D})) = \sigma(\mathcal{D})$.

Since $M_c$ is a bijection it follows that $M_c^{-1} : 2^{S \times \reals}
\to 2^{S \times \reals}$ is also a bijection.  Furthermore if we
consider a set of the form $[f,g)$ then 
\begin{align*}
M_c^{-1}([f,g)) &= \lbrace (s,t) \in S \times \reals \mid f(s) \leq ct
< g(s) \rbrace \\
&= \lbrace (s,t) \in S \times \reals \mid (f/c)(s) \leq t
< (g/c)(s) \rbrace = [f/c, g/c)
\end{align*}
So if $[f,g) \in \mathcal{D}$ then it follows from the fact that
$\mathcal{L}$ is a vector space that $M_c^{-1}$ is bijection of
$\mathcal{D}$
to itself.  In particular, we know that $M_c^{-1}
(\sigma(\mathcal{D}))$ is a $\sigma$-algebra containing $\mathcal{D}$
  and therefore $M_c^{-1}
(\sigma(\mathcal{D})) \supset \sigma(\mathcal{D})$.  On the other
  hand, $(M_c)_*(\sigma(\mathcal{D})) = \lbrace A \subset S \times
  \reals \mid M_c^{-1}(A) \in \sigma(\mathcal{D}) \rbrace$ is also
  $\sigma$-algebra (Lemma \ref{SigmaAlgebraPullback}) containing
  $\mathcal{D}$; hence $\sigma(\mathcal{D}) \subset
  (M_c)_*(\sigma(\mathcal{D}))$ which implies
  $M_c^{-1}(\sigma(\mathcal{D})) \subset \sigma(\mathcal{D})$.

Claim 2: For any $A \in \sigma(\mathcal{D})$ and $c > 0$ we have $c
\nu(M_c^{-1}(A)) = \nu(A)$.

We start with considering $[f,g) \in \mathcal{D}$.  We have already
seen that $M_c^{-1}([f,g)) = [f/c, g/c)$ so we can just apply the
definition to see the claim holds.
\begin{align*}
c \nu(M_c^{-1}([f,g))) &= c \nu([f/c,g/c)) = c I(f/c-g/c) = I(f-g) =
\nu([f,g))
\end{align*}
To extend to the ring $\mathcal{R}$ generated by $\mathcal{D}$ we note that every
element of the ring is a disjoint union of elements in $\mathcal{D}$.
Furthermore $M_c^{-1}$ preserves the Boolean algebra structure on
$2^{S \times \reals}$ (Lemma \ref{SetOperationsUnderPullback})
therefore 
we have
\begin{align*}
c \nu(M_c^{-1}(\cup_{i=1}^n [f_i, g_i))) &= c \nu(\cup_{i=1}^n
M_c^{-1}([f_i, g_i))) = c \sum_{i=1}^n \nu(M_c^{-1}([f_i,g_i))) \\
&= \sum_{i=1}^n \nu([f_i,g_i)) = \nu(\cup_{i=1}^n [f_i, g_i))
\end{align*}
To extend to all of $\sigma(\mathcal{D})$ we use the fact that $\nu$
is defined as its associated outer measure $\nu(A) = \inf \lbrace
\nu(B) \mid B \supset A \text{ and } B \in \mathcal{R}\rbrace$.
Consider $A \in \sigma(\mathcal{D})$ and let $\epsilon > 0$.  By
definition we can find $B \in \mathcal{R}$ such that $B \supset A$ and
$\nu(B) \leq \nu(A) + \epsilon$.  Again applying Lemma
\ref{SetOperationsUnderPullback} we see that $M_c^{-1} (B) \in
\mathcal{R}$ and $M_c^{-1}(B) \supset M_c^{-1}(A)$ and therefore
\begin{align*}
c \nu(M_c^{-1}(A)) & c \leq M_c^{-1}(B) = \nu(B) \leq \nu(A) +\epsilon
\end{align*}
Since $\epsilon>0$ was arbitrary we conclude that $c \nu(M_c^{-1}(A))
\leq \nu(A)$.  In the opposite direction for every $\epsilon > 0$ we
can find $M_c^{-1}(B) \supset M_c^{-1}(A)$ such that $\nu(M_c^{-1}(B)
\leq \nu(M_c^{-1}(A)) + \epsilon$.  We know that $B \supset A$ and
therefore
\begin{align*}
\nu(A) &\leq \nu(B) = c \nu(M_c^{-1}(B) \leq c \nu(M_c^{-1}(A)) + \epsilon
\end{align*}
so letting $\epsilon$ go to zero we conclude $\nu(A) \leq c
\nu(M_c^{-1}(A))$ and we are done.

TODO: In the proof we use the fact that $M_c^{-1}$ is a bijection on
$\mathcal{R}$ which is a simple consequence of the fact $M_c^{-1}$ is
a bijection on $\mathcal{D}$ and Lemma \ref{SetOperationsUnderPullback};
find the correct place to note this fact explicitly.
\end{proof}

\begin{thm}\label{DaniellStoneTheorem}Let $I$ be a pre-integral on a Stone vector lattice
  $\mathcal{L}$.  Then on the $\sigma$-algebra generated by the
  lattice $\mathcal{L}$ there is a measure $\mu$ such that $I(f) =
  \int f \, d\mu$ for all $f \in \mathcal{L}$.  Futhermore the measure
  $\mu$ is uniquely determined on the $\sigma$-ring generated by $\mathcal{L}$.
\end{thm}
\begin{proof}
We proceed by first defining our measure on the $\sigma$-ring
$\mathcal{R}$ generated by the functions $\mathcal{L}$.  This can be extended (not
necessarily uniquely) to a measure on the $\sigma$-algebra using Lemma \ref{ExtendingMeasuresFromSigmaRing}.
Because we have arranged for all of the functions in $\mathcal{L}$ to
be $\mathcal{R}$ measurable their integrals will not depend on the
extension of $\mu$ to a full $\sigma$-algebra and their integrals will
be determined by the values of $\mu$ on $\mathcal{R}$ alone.

Claim 1: $\mathcal{R}$ is generated by sets of the form $f^{-1}(1,
\infty)$ for $f \in \mathcal{L}$.

Note that for $c > 0$, 
\begin{align*}
f^{-1}(c, \infty) &= 
\lbrace \omega \in \Omega \mid f(\omega) \geq c \rbrace = 
\lbrace \omega \in \Omega \mid \left(f/c\right)(\omega) \geq 1 \rbrace =
\left(f/c\right)^{-1}(1, \infty)
\end{align*}
and since $\mathcal{L}$ is a Stone lattice (a fortiori a real vector
space) we know that $f/c \in \mathcal{L}$.  A similar argument
shows that for $c > 0$, $f^{-1}(-\infty, -c) = (-f/c)^{-1}(1,
\infty)$.  We know that intervals $(-\infty, -c)$ and $(c, \infty)$
generate the $\sigma$-ring on $\reals \setminus \lbrace 0 \rbrace$,
therefore for any $f \in \mathcal{L}$, we have
$f^{-1}(\mathcal{B}(\reals\setminus \lbrace 0 \rbrace)$ is the
$\sigma$-ring generated by sets $f^{-1}(c, \infty)$ and
$f^{-1}(-\infty, -c)$ for $c > 0$
(Lemma \ref{SigmaRingPullbackGenerators}) which are the same as the sets
$(f/c)^{-1}(1, \infty)$ for $c \neq 0$. Thus the $\sigma$-ring generated by $\cup_{f
  \in \mathcal{L}} f^{-1}(\mathcal{B}(\reals\setminus \lbrace 0
\rbrace)$ is contained in the $\sigma$-ring generated by $\cup_{f \in
  \mathcal{L}} f^{-1}(1, \infty)$.

Claim 2: We can define a measure $\mu$ on the $\sigma$-algebra
generated by $\mathcal{L}$.

If suffices to define a countably additive set function on the $\sigma$-ring $\mathcal{R}$
(Lemma \ref{ExtendingMeasuresFromSigmaRing}).  We define the measure by embedding $\mathcal{R}$ as
sub-$\sigma$-ring in $\sigma$-algbra $\mathcal{A}$ constructed in
Theorem \ref{Zaanen}.  To see this, suppose that we have a set $A =
f^{-1}(1, \infty)$ with $f \in \mathcal{L}$ and $f \geq 0$.  For arbitrary $c > 0$,
we define
\begin{align*}
f_n(\omega) &= n(f(\omega) - f(\omega) \wedge 1) \wedge c
= \begin{cases}
0 &  \text{if $\omega \notin A$} \\
n(f(\omega) - 1) \wedge c & \text{if $\omega \in A$}
\end{cases}
\end{align*}
and observe that $f_n \in \mathcal{L}$ and $f_n \uparrow c\characteristic{A}$.  Applying this
observation to graphs of $f_n$ in $\Omega \times \reals$ we see that
$A \times [0,c) = [0, c\characteristic{A}) = \cup_{n=1}^\infty [0, f_n)$ which shows that $A
\times [0,c) \in \mathcal{A}$ for all $c > 0$.  From this it follows
that $A \times [0,c) \in \mathcal{A}$ for all $A \in \mathcal{R}$.  To
see this note that for a fixed $c >0$, the set $\mathcal{R}_c = \lbrace A \times [0,c)
\mid A \in \mathcal{R} \rbrace$ is a $\sigma$-ring and the set
$\lbrace A \subset \Omega \mid A \times [0,c) \in \mathcal{R}_c
\rbrace$ is a $\sigma$-ring (it can be constructed as a pushforward
under an appropriately constructed map or one can see it directly)
that contains sets of the form $f^{-1}(1, \infty)$.  Thus,
$\mathcal{R} \subset \lbrace A \subset \Omega \mid A \times [0,c) \in \mathcal{R}_c
\rbrace$.

Having shown that $\mathcal{R}_c$ is a $\sigma$-ring in $\mathcal{A}$,
we take $c=1$ and define $\mu(A) = \nu(A \times [0,1)$.  That this
is countably additive follows from the fact that $\nu$ is a measure,
so we can extend $\mu$ to the $\sigma$-algebra $\mathcal{R} \cup
\mathcal{R}^c$ in any way we chose.

Now we show how to compute integrals of functions $f \in \mathcal{L}$
with respect to $\mu$ and show that they agree with the pre-integral
$I$.
Claim 3: For every $\mathcal{R}/\mathcal{B}(\reals \setminus \lbrace 0
\rbrace)$ simple function $f \geq 0$ of the form $f = \sum_{i=1}^n c_i
\characteristic{A_i}$ we have $\int f \, d\mu = \nu([0,f))$.

To see this we know by Theorem \ref{Zaanen} that for every $c > 0$ and
$B \in \mathcal{A}$ we have $c \nu(M_c^{-1}(B) = \nu(B)$.  We have
shown that for every $A \in \mathcal{R}$, we have $A \times [0,c) \in
\mathcal{A}$ and by definition $M_c^{-1}(A \times [0,c)) = A \times
[0,1)$; therefore $\nu(A \times [0,c)) = c \nu(A \times [0,1)) = c
\mu(A)$. It is also easy to see that $A \times [0,c) = [0,
c\characteristic{A})$, so we have for scalar multiples of
characteristic functions $\int f \, d\mu = \nu([0,f))$.  As for simple
functions, each can be expressed as a sum $f = \sum_{i=1}^n c_i
\characteristic{A_i}$ with $A_i \in \mathcal{R}$ and the $A_i$
disjoint.  Once again by definition we can see that $[0,f) =
\cup_{i=1}^n [0, c \characteristic{A_i})$ where the disjointness of
the $A_i$ implies that the sets $ [0, c \characteristic{A_i})$ are
disjoint.  Now by definition of the integral for a simple function and
the additivity of the measure $\nu$ we get
\begin{align*}
\int f \, d \mu &= \sum_{i=1}^n c_i \mu(A_i) = \sum_{i=1}^n \nu([0,c_i
\characteristic{A_i})) = \nu([0,f))
\end{align*}

Claim 4: For every $\mathcal{R}/\mathcal{B}(\reals \setminus \lbrace 0
\rbrace)$-measurable function $f \geq 0$ we have $\int f \, d\mu =
\nu([0,f))$.

We take a sequence of positive simple functions $f_n \uparrow f$ which
exists by Lemma \ref{PointwiseApproximationBySimpleSigmaRing}.
Since $[0,f_n) \uparrow [0,f)$ we can use the definition of integral
with respect to $\mu$, continuity of measure with respect to $\nu$ and
Claim 3 to see
\begin{align*}
\int f \, d \mu &= \lim_{n \to \infty} \int f_n \, d\mu = \lim_{n \to
  \infty} \nu([0,f_n)) = \nu([0,f))
\end{align*}

By definition we have arranged for all $f \in
\mathcal{L}$ to be $\mathcal{R}/\mathcal{B}(\reals \setminus \lbrace 0
\rbrace)$-measurable so by Claim 4 and the definition of $\nu$, for $f
\in \mathcal{L}$ with $f \geq 0$ we have $\int f \, d \mu = \nu([0,f))
= I(f)$.  For arbitrary $f \in \mathcal{L}$ we write $f = f_+ - f_-$
with $f_+, f_- \in \mathcal{L}$ and $f_+, f_- \geq 0$ and use
linearity of integral and pre-integral to conclude that $\int f \, d
\mu = \int f_+ \, d\mu - \int f_- \, d\mu = I(f_+) - I(f_-) = I(f)$.
\end{proof}

It should be remarked that one can develop a good deal of measure an
integration theory starting from some of the concepts introduced in
this section; indeed for a short period of time it was fashionable to
do this instead of taking the approach of developing the theory of
$\sigma$-algebras, measure and integral in the way we have done.
Alas, that fashion has passed so we content ourselves with the most
streamlined presentation of these ideas we know that gives us Theorem \ref{DaniellStoneTheorem}.

\section{Standard Spaces}

TODO: We have called a field of sets (collection of sets closed under union,intersection and complementation) a Boolean algebra above.

\begin{defn}A field $\mathcal{F}$ on $\Omega$ has the \emph{countable extension property} if it is countable and 
every set function $\mu : \mathcal{F} \to [0, +\infty]$ that satisfies 
\begin{itemize}
\item[(i)] $\mu(A) \geq 0$ for all $A \in \mathcal{F}$
\item[(ii)] $\mu(\Omega) = 1$
\item[(iii)] For every $n \in \naturals$ and disjoint $A_1, \dotsc, A_n \in \mathcal{F}$ we have $\mu(\cup_{i=1}^n A_i) =  \sum_{i=1}^n \mu(A_i)$
\end{itemize}
also is countably additive, i.e. satisfies  for every disjoint $A_1, A_2, \dotsc\in \mathcal{F}$ we have $\mu(\cup_{i=1}^\infty A_i) =  \sum_{i=1}^\infty \mu(A_i)$.
\end{defn} 

\begin{defn}A measurable space $(\Omega, \mathcal{A})$ has the \emph{countable extension property} if there exists a field $\mathcal{F}$ such that $\mathcal{A} = \sigma(\mathcal{F})$ 
and $\mathcal{F}$ has the countable extension property.
\end{defn}

\begin{defn}Let $\mathcal{F}$ be a field then $A \in \mathcal{F}$ is an \emph{atom}  $B \in \mathcal{F}$ and $B \subset A$ implies either $\emptyset = B$ or $A = B$.
\end{defn}

\begin{defn}Let $A_1, \dotsc, A_n$ be subsets of a set $\Omega$ then an \emph{intersection set} is any non-empty set of the form $A_1^{*_1} \cap \dotsc \cap A_n^{*_n}$ where $A_i^{*_i}$ equals either $A_i$ or $A_i^c$.  It is often convenient to index intersection sets by a subset $\mathfrak{B} \subset \lbrace 0, 1 \rbrace^n$ such that $\beta \in \mathfrak{B}$ corresponds to $A_1^{*_1} \cap \dotsc \cap A_n^{*_n}$ where $A_i^{*_i} = A_i$ if and only if $\beta_i=1$
\end{defn}

\begin{prop}Any two intersection sets on $A_1, \dotsc, A_n$ are disjoint.
\end{prop}
\begin{proof}
Let $A_\beta$ and $A_\gamma$ be intersection sets for $\beta, \gamma \in \mathfrak{B}$.  Then since there is some $1 \leq i \leq n$ such that $\beta_i \neq \gamma_i$ it follows that either $A_\beta \subset A_i$ and $A_\gamma \subset A_i^c$ or $A_\beta \subset A_i^c$ and $A_\gamma \subset A_i$.  In either case we have $A_\beta \cap A_\gamma = \emptyset$. 
\end{proof}

\begin{prop}\label{AtomsAsIntersectionSets}Let $\mathcal{F}$ be a finite field  generated by $A_1, \dotsc, A_n$.  Then the set of atoms is the set of intersection sets on $A_1, \dotsc, A_n$.  The set of atoms is pairwise disjoint and generates $\mathcal{F}$.  
\end{prop}
\begin{proof}
Let $\mathfrak{B} \subset \lbrace 0,1 \rbrace^n$ be the indexing set for the intersection sets on $A_1, \dotsc, A_n$.
\begin{clm} Every $A_i$ may be written as a finite union of intersection sets on $A_1, \dotsc, A_n$.  The same is true of each $A_i^c$.  
\end{clm}
For every $\beta \in \mathfrak{B}$ such that $\beta_i = 1$ we have $A_\beta \subset A_i$; hence $\cup_{\substack{\beta \in \mathfrak{B} \\ \beta_i=1}} A_\beta \subset A_i$.   On the other hand, let $\omega \in A_i$ and define $\beta \in \mathfrak{B}$ as follows.  Let $\beta_i = 1$ and for $j \neq i$ let $\beta_j = 1$ if $\omega \in A_j$ and let $\beta_j =0$ otherwise.  Then it follows that $\omega \in A_\beta$.  Since $A_\beta \neq \emptyset$, we have $\beta \in \mathfrak{B}$ and therefore $\cup_{\substack{\beta \in \mathfrak{B} \\ \beta_i=1}} A_\beta = A_i$.  A similar argument shows that for every $1 \leq i \leq n$ we also have $A_i^c = \cup_{\substack{\beta \in \mathfrak{B}\\ \beta_i = 0}} A_\beta$.  

\begin{clm} Every $A \in \mathcal{F}$ may be written as a finite union of intersection set $A_1, \dotsc, A_n$.  In particular, $\mathcal{F}$ is generated by the intersection sets.
\end{clm}
Clearly every finite union of $A_1, \dotsc, A_n$ and $A^c_1, \dotsc, A^c_n$ is a finite union of intersection sets by applying the previous claim.  Since intersection sets are pairwise disjoint, any intersection of a finite union of intersection sets is also a finite union of intersection sets and the claim follows.

\begin{clm}Every atom in $\mathcal{F}$ is an intersection set.
\end{clm}
Let $A$ be an atom.  From the previous claim, $A$ is a finite union of intersection sets.  Since $A$ is an atom the number of such intersection sets must be one.

\begin{clm}Every intersection set is an atom.
\end{clm}
Let $A_\beta$ be an intersection set for some $\beta \in \mathfrak{B}$ and suppose that $\emptyset \neq A \subset A_\beta$ is a non-empty subset with $A \in \mathcal{F}$.  By a previous claim, $A$ is a finite union of intersection sets.  Since intersection sets are pairwise disjoint this implies that $A = A_\beta$ and therefore $A_\beta$ is an atom.

By the previous two claims we know that atoms are precisely the intersection sets and since we have shown that the intersection sets generate $\mathcal{F}$ the same is true for atoms (and moreover every $A \in \mathcal{F}$ is finite union of atoms).
\end{proof}

\begin{cor}\label{RefinementOfAtoms}Let $\mathcal{F} \subset \mathcal{G}$ be finite fields on $\Omega$ then every atom of $\mathcal{F}$ is a union of atoms of $\mathcal{G}$.
\end{cor}
\begin{proof}
Let $A \in \mathcal{F}$ be an atom in $\mathcal{F}$.  Since $A \in \mathcal{G}$ we know that $A$ is a disjoint finite union of atoms in $\mathcal{G}$.  
\end{proof}

\begin{prop}\label{NestedUnionOfFieldsOfSets}Let $\mathcal{F}_1 \subset \mathcal{F}_2 \subset \dotsb$ be nested finite fields the $\mathcal{F} = \cup_{n=1}^\infty \mathcal{F}_n$ is a field.
\end{prop}
\begin{proof}
Let $A \in \mathcal{F}$, then there exists $n \in \naturals$ such that $A \in \mathcal{F}_n$.   Since $\mathcal{F}_n$ is a field it follows that $A^c \in \mathcal{F}_n$ hence $A^c \in \mathcal{F}$.  Now let $A, B \in \mathcal{F}$.  Since the $\mathcal{F}_n$ are nested there exists $n \in \naturals$ such that $A,B \in \mathcal{F}_n$.  It follows that $A \cup B \in \mathcal{F}_n$ and $A \cap B \in \mathcal{F}_n$ and therefore $A \cup B \in \mathcal{F}$ and $A \cap B \in \mathcal{F}$.
\end{proof}

\begin{defn}If $\mathcal{F}_1 \subset \mathcal{F}_2 \subset \dotsb$ are nested finite fields then we say the \emph{asymptotically generate} the field $\mathcal{F} = \cup_{n=1}^\infty \mathcal{F}_n$.  If $\sigma(\mathcal{F}) = \mathcal{A}$ then we say that the $\mathcal{F}_1 \subset \mathcal{F}_2 \subset \dotsb$ \emph{asymptotically generate} the measurable space $(\Omega, \mathcal{A})$.
\end{defn}

\begin{defn}Let $\mathcal{F}_1 \subset \mathcal{F}_2 \subset \dotsb$ are nested finite fields that asymptotically generate $\mathcal{F}$ and let $\mathcal{A}_n$ be atoms of $\mathcal{F}_n$.  We say that the sequence of atoms $\mathcal{A}_1, \mathcal{A}_2, \dotsc$ is a \emph{binary indexed sequence} if 
\begin{itemize}
\item[(i)] There exist a set $\mathfrak{B}_n \subset \lbrace 0, 1 \rbrace^n$ indexing $\mathcal{A}_n$, i.e. $\mathcal{A}_n = \lbrace A_\beta \rbrace_{\beta \in \mathfrak{B}_n}$.
\item[(ii)] For any $m > n$, $\beta_m \in \mathfrak{B}_m$, $\beta_n \in \mathfrak{B}_n$ such that $\beta_n$ is a prefix of $\beta_m$ we have $A_{beta_m} \subset A_{\beta_n}$.
\end{itemize}
\end{defn}

\begin{defn}Let $A_1, A_2, \dotsc$ be an enumeration of a countable field of subsets of $\Omega$ the \emph{canonical binary sequence function} if the function $f : \Omega \to \lbrace 0,1 \rbrace^\infty$ given by $f(\omega) = (\characteristic{A_1}(\omega), \characteristic{A_2}(\omega), \dotsc)$.
\end{defn}

Note that a canonical binary sequence function need not be injective or surjective. Any countably generated field has a binary indexed sequence.

\begin{prop}Let $\mathcal{F}$ be a countably generated field on a sample space $\Omega$ then there is a sequence of fields $\mathcal{F}_1 \subset \mathcal{F}_2 \subset \dotsb$ that asymptotically generate $\mathcal{F}$.  Moreover the $\mathcal{F}_n$ can be constructed so that the atoms $\mathcal{A}_n$ on $\mathcal{F}_n$ can be indexed by a set $\mathfrak{B}_n \subset \lbrace 0,1 \rbrace^n$ with the property that if $m > n$ and $\beta_n$ is a prefix of $\beta_m$ then $A_{\beta_m} \subset A_{\beta_n}$.  
\end{prop}
\begin{proof}
Pick an enumeration $\mathcal{F} = \lbrace A_1, A_2, \dotsc \rbrace$.  Define $\mathcal{F}_n = \mathcal{F}(A_1, \dotsc, A_n)$.  It is clear that the $\mathcal{F}_n$ asymptotically generate $\mathcal{F}$.  By Proposition \ref{AtomsAsIntersectionSets} we know that the atoms of $\mathcal{F}_n$ are precisely the intersection sets on $A_1, \dotsc, A_n$.  Therefore there exists $\mathfrak{B}_n \subset \lbrace 0,1 \rbrace^n$ such that the atoms are precisely $A_\beta$ for $\beta \in \mathfrak{B}_n$.  By construction, for $\beta \in \mathfrak{B}_n$ we have $\omega \in A_\beta$ if and only if $\omega \in A_i$ when $\beta_i = 1$ and $\omega \in A^c_i$  when $\beta_i = 0$ for all $1 \leq i \leq n$.  Therefore if $m>n$ and  $\beta_n$ is a prefix of $\beta_m$ then we have $\omega \in A_{beta_m}$ implies $\omega \in A_i$ when $\beta_i = 1$ and $\omega \in A^c_i$  when $\beta_i = 0$ for all $1 \leq i \leq m$ so it follows that $\omega \in A_{\beta_n}$.

Note that by construction for any $\omega \in \Omega$ if we let $f(\omega)$ be the image of $\omega$ under the canonical binary sequence function, and we let $\beta_n=(\characteristic{A_1}(\omega), \dotsc, \characteristic{A_n}(\omega))$ be the prefix of $f(\omega)$ of length $n$ then for any $1 \leq i \leq n$ we have $\omega \in A_i$ if and only if $\beta_n(i) = 1$ and therefore $\omega \in A_{\beta_n}$.
\end{proof}

\begin{defn}A sequence of finite fields $\mathcal{F}_1, \mathcal{F}_2, \dotsc$ is a \emph{basis} of a field $\mathcal{F}$ if 
\begin{itemize}
\item[(i)] $\mathcal{F}_1, \mathcal{F}_2, \dotsc$  asymptotically generate $\mathcal{F}$
\item[(ii)] if $G_1 \supset G_2 \supset \dotsb$ is a sequence of atoms with $G_n \in \mathcal{A}_n$ then $\cap_{n=1}^\infty G_n \neq \emptyset$.
\end{itemize}
A field $\mathcal{F}$ is said to be \emph{standard} if it has a basis.
\end{defn}

\begin{defn}A sequence of finite fields $\mathcal{F}_1, \mathcal{F}_2, \dotsc$ is a \emph{basis} of the measurable space $(\Omega, \mathcal{A})$ if and only it is a basis of some field of subsets of $\Omega$ that generate $\mathcal{A}$.  The measurable space $(\Omega, \mathcal{A})$ is \emph{standard} if it has a basis.
\end{defn}

\begin{prop}Let $\mathcal{F}$ be a field of subsets of $\Omega$ then if $\mathcal{F}$ has the countable intersection property then $\mathcal{F}$ is standard.
\end{prop}
\begin{proof}
The argument is by contradiction, so let us assume that $\mathcal{F}$ does not have a basis.
TODO: Finish
\end{proof}

\chapter{Probability}
Here we begin to focus on the special case of probability spaces.  The
development of measure theoretic probabilty begins with the
assumptions that we are given a
\begin{defn}A \emph{probability space} is a measure space $(\Omega,
  \mathcal{A}, P)$ such that $\probability{\Omega} = 1$.
\end{defn}
Given a measurable function $\xi : \Omega \to (S,
\mathcal{S})$ we will refer to $\xi$ as a \emph{random
  element} of $S$.  The special case of a measurable function $\xi : \Omega \to (\reals, \mathcal{B}(\reals))$
is called a \emph{random variable}.  For a random element $\xi$, by
Lemma \ref{PushforwardMeasure} we can
push forward the probability measure to get a measure $(\pushforward{\xi}{P})$ called the \emph{distribution} or \emph{law} of $\xi$. One
sometimes writes $\mathcal{L}(\xi)$ to denote the distribution of
$\xi$ and one writes $\xi \eqdist \eta$ to denote that $\xi$ and $\eta$ have
the same distribution. 

In probability theory the existence of a probability space is critical
to the formal development of the theory however it is almost always
the case that one is only concerned with results that don't depend on
the exact choice of probability space.  To make this statement more
precise we introduce 
\begin{defn}A probability space $(\Omega^\prime, \mathcal{A}^\prime,
  P^\prime)$ is an \emph{extension} of $(\Omega, \mathcal{A},
  P)$ if there is a surjective measurable map $\pi : \Omega^\prime \to
  \Omega$ such that $P = \pushforward{\pi}{P^\prime}$.
\end{defn}

A result is considered properly \emph{probabilistic} if it is
preserved under extension of sample space.  Note that this is a
cultural statement and not a mathematical theorem.  As an example of a
probabilistic concept, we have the ability to talk about an
\emph{event} $A$ and its probability $\probability{A}$ since given any
$\pi$  we can unambiguously refer to $\pi^{-1}(A)$ as the same event
in $\Omega^\prime$ and we know that probability is preserved.  As an
example of a non-probabilistic concept we have the cardinality of an
event.

In keeping with the philosophy that probabilistic results are
invariant under extension of the underlying probability space, we will
follow common practice and try to avoid explicit mention of the underlying
probability space in many definitions and results.  

\begin{defn}Given a random vector $\xi = (\xi_1, \dots, \xi_n)$ in
  $\reals^n$ we define the \emph{distribution function} to be 
\begin{align*}
F(x_1, \dots, x_n) = \probability {\cap_{i=1}^n \left ( \xi_i \leq x_i
    \right ) }
\end{align*}
\end{defn}
\begin{lem}\label{DistributionFunctionCharacterizeProbability}Let $\xi$ and $\eta$ be random vectors in $\reals^n$ with
  distribution functions $F$ and $G$, then
  $\xi \eqdist \eta$ if and only if $F = G$.
\end{lem}
\begin{proof}This follows from Lemma \ref{UniquenessOfMeasure} by
  noting
  that sets of the form $(-\infty, x_1] \times \cdots \times (-\infty,
  x_n]$ form a $\pi$-system that contains $\reals^n$.
\end{proof}

The construction of Lebesgue-Stieltjes measure shows that every
Borel probability measure on $\reals$ is determined uniquely by its distribution function.
\begin{lem}Probability measures of $(\reals, \mathcal{B}(\reals))$ are
  in one to one correspondence with $F : \reals \to \reals$ that are
  right continuous, nondecreasing such that $\lim_{x \to -\infty} F(x)
  = 0$ and $\lim_{x \to \infty} F(x) = 1$ via the mapping $F(x) =
  \probability{(-\infty, x]}$.
\end{lem}
\begin{proof}Clearly any probability measure is locally finite so we
  apply Lemma \ref{LebesgueStieltjesMeasure} to create a 1-1
  correspondence with $\hat{F}$, 
  right continuous and nondecreasing such that $\probability{(a,b]} =
  \hat{F}(b) - \hat{F}(a)$.  Now define $F(x) = \hat{F}(x) +
  \probability{(-\infty, 0]}$.
\end{proof}

\begin{defn}The \emph{expectation} of a random variable $\xi$ on a
  probability space $(\Omega, \mathcal{A}, P)$ is
  defined to be 
\begin{align*}
\expectation{\xi} = \int \xi \, dP
\end{align*}
\end{defn}
A very useful corollary to the abstract change of variables Lemma
\ref{ChangeOfVariables} is the following
\begin{lem}[Expectation Rule]\label{ExpectationRule}Let $\xi$ be a random variable and $f: \reals \to \reals$
  be a Borel measurable function.  Then 
\begin{align*}
\expectation{f(\xi)} = \int f \, d (\pushforward{\xi}{P})
\end{align*}
In particular, 
\begin{align*}
\expectation{\xi} = \int x \, d (\pushforward{\xi}{P})
\end{align*}
\end{lem}
\begin{proof}This is just a restatement of Lemma
  \ref{ChangeOfVariables} for the special case of random variables and
  measurable functions on $\reals$.
\end{proof}

The following lemma is useful for relating tail bounds and expectations.
\begin{lem}\label{TailsAndExpectations}Let $\xi$ be a positive random variable with finite
  expectation.  Then $\expectation{\xi} = \int_0^\infty \probability{\xi \geq
    \lambda} d\lambda$.
\end{lem}
\begin{proof}
This is just an application of Tonelli's Theorem,
\begin{align*}
\int_0^\infty \probability{\xi \geq \lambda}d\lambda &= \int_0^\infty \left[\int
\characteristic{\xi \geq \lambda} dP\right] d\lambda \\
&= \int\left[\int_0^\infty \characteristic{\xi \geq \lambda} d\lambda\right] dP \\
&= \int\left[\int_0^{\xi} d\lambda\right] dP \\
&= \int \xi \, dP \\
&= \expectation{\xi} \\
\end{align*}
\end{proof}

\begin{lem}[Cauchy Schwartz Inequality]\label{CauchySchwartz}Let $\xi$
  and $\eta$ satisfy $\expectation{\xi^2}, \expectation{\eta^2} < \infty$ then $\xi \eta$ is integrable and
  $\expectation{\xi \eta}^2 \leq \expectation{\xi^2}
  \expectation{\eta^2}$.
\end{lem}
\begin{proof}
Since we have both $0 \leq (\xi + \eta)^2$ and $0 \leq (\xi - \eta)^2$
we have $\abs{\xi \eta} \leq \frac{1}{2}(\xi^2 + \eta^2)$ which shows
that $\xi \eta$ is integrable.

There are a host of different proofs of Cauchy Schwartz inequality.  Here is perhaps
the simplest one.  Note that for all $t \in \reals$, $0 \leq \expectation{(t \xi +
\eta)^2} = \expectation{\xi^2} t^2 + 2 \expectation{\xi \eta} t +
\expectation{\eta^2}$.  The quadratic formula implies that
$\sqrt{4 \expectation{\xi^2} \expectation{\eta^2} - (2
  \expectation{\xi \eta})^2 } \geq 0$ which in turn implies the
result.

The proof we just provided is probably the slickest one available but
has the disadvantage of being very specific to the quadratic case.  
There is a different proof of Cauchy Schwartz that we provide that
involves two steps that have a broader application.  The idea is to
derive Cauchy Schwartz from the trival fact that for all real numbers
$x,y$ we have $xy \leq \frac{x^2}{2} + \frac{y^2}{2}$ (which we used when showing
integrability of $\xi\eta$).  Applying this fact to $\xi$ and $\eta$ we see
that
\begin{align*}
\expectation{\xi \eta} &\leq \frac{\expectation{\xi^2}}{2} + \frac{\expectation{\eta^2}}{2}
\end{align*}
To finish the proof, we apply a \emph{normalization trick} by defining
$\hat{\xi} = \frac{\xi}{\sqrt{\expectation{\xi^2}}}$ and $\hat{\eta} =
\frac{\eta}{\sqrt{\expectation{\eta^2}}}$ so that
$\expectation{\hat{\xi^2}} = \expectation{\hat{\eta^2}} =1$.  Now we apply the above bound and linearity of expectation to see that
\begin{align*}
\frac{1}{\sqrt{\expectation{\xi^2}}\sqrt{\expectation{\eta^2}}}
\expectation{\xi\eta} &= \expectation{\hat{\xi}\hat{\eta}} \leq 1
\end{align*}
which yields the result.
\end{proof}

Applications of Cauchy Schwatz are ubiquitous in analysis.  Only
slightly less common are applications of the following
generalization.  First a definition
\begin{defn}Given any $p > 0$ and random variable $\xi$, the \emph{$L^p$
  norm} of $\xi$ is 
\begin{align*}
\norm{\xi}_p = \left (\expectation{\abs{\xi}^p} \right )^ {\frac{1}{p}}
\end{align*}
\end{defn}
\begin{lem}[H\"{o}lder Inequality]\label{Holder}Given $p,q,r > 0$ such
  that $\frac{1}{r} = \frac{1}{p} + \frac{1}{q}$ and random variables
  $\xi$ and $\eta$, we have
\begin{align*}
\norm{\xi \eta}_r \leq \norm{\xi }_p \norm{\eta}_q
\end{align*}
\end{lem}
\begin{proof}
We start by assuming that $r=1$.  The proof here is a direct generalization of the second proof we
provided for Cauchy Schwartz.  To get started we need to find a
generalization of the simple fact that $xy \leq \frac{x^2}{2} +
\frac{y^2}{2}$.  
 
The inequality we need is called Young's Inequality and is derived
from the following fact.  Let $f$ be an
continuous increasing function $f : [0,c] \to \reals$ such that $f(0)
= 0$.  Then the area interpretation of integral tells us that for $0
\leq a \leq c$ and $0 \leq b \leq f(c)$ we have
\begin{align*}
ab \leq \int_0^a f(x) \, dx + \int_0^b f^{-1}(x) \, dx
\end{align*}
with equality if and only if $b=f(a)$.  

For our case, we first assume that $r = 1$.  Define $f(x) =
x^{p-1}$ then observe that $f^{-1}(x) = x^{q-1}$ since
$1 = \frac{1}{p} + \frac{1}{q}$ is equivalent to $(p-1)(q-1) = 1$.
Therefore we have Young's Inequality, $ab \leq \frac{a^p}{p} +
\frac{b^q}{q}$.

Now applying the normalization trick by defining $\hat{\xi} =
\frac{\abs{\xi}}{\norm{\xi}_p}$ and $\hat{\eta} =
\frac{\abs{\eta}}{\norm{\eta}_q}$ so that ${\norm{\hat{\xi}}_p} =
{\norm{\hat{\eta}}_q} = 1$.  We now apply Young's Inequality to
$\hat{\xi}$ and $\hat{\eta}$ to see
\begin{align*}
\frac{1}{\norm{\hat{\xi}}_p \norm{\hat{\eta}}_q} \expectation{\abs{\xi \eta}} &=
\expectation{\hat{\xi}\hat{\eta}} \leq \frac{1}{p} + \frac{1}{q} = 1
\end{align*}

Lastly we generalize to general $r>0$.  Given $\frac{1}{r} =
\frac{1}{p} + \frac{1}{q}$ we define $\hat{p} = \frac{p}{r}$ and 
$\hat{q} = \frac{q}{r}$ so that $1 =
\frac{1}{\hat{p}} + \frac{1}{\hat{q}}$ and 
\begin{align*}
\expectation{\abs{\xi \eta}^r} &\leq \norm{\xi^r}_{\hat{p}}
\norm{\eta^r}_{\hat{q}} = \norm{\xi}_p^r \norm{\eta}_q^r 
\end{align*}
Taking $r^{th}$ roots we are done.
\end{proof}
\begin{cor}\label{IncreasingMoments}For $p > r > 0$ and any random variable $\xi$, we
  have $\norm{\xi}_r \leq \norm{\xi}_p$.
\end{cor}
\begin{proof}Define $q = \frac{p -r}{pr} > 0$ and apply H\"older's
  Inequality to see that $\norm{\xi}_r \leq \norm{\xi}_p \norm{1}_q = \norm{\xi}_p $.
\end{proof}
It worth noting that the corollary above is generally true on finite
measure spaces but fails for non-finite measure spaces (e.g. consider
$f(x) = \frac{1}{x}$ which has finite $L^p$ norm on $[1,\infty)$ for
$p > 1$ but infinite $L^1$ norm on $[1,\infty)$).

\section{Convexity and Jensen's Inequality}
\begin{defn}A function $\varphi : \reals^n \to \reals$ is said to be
  \emph{convex} if for all $x, y \in \reals^n$ and $t \in [0,1]$, we
  have
\begin{align*}
\varphi( tx + (1-t)y) \leq t\varphi(x) + (1-t)\varphi(y)
\end{align*}
$\varphi$ is said to be \emph{strictly convex} if it is convex and for
all $t \in (0,1)$, 
\begin{align*}
\varphi( tx + (1-t)y) < t\varphi(x) + (1-t)\varphi(y)
\end{align*}
\end{defn}

TODO: Convex functions are continuous; in fact they are locally Lipschitz (hence differentiable a.e. by Rademacher).

Convex functions are almost surely differentiable.
\begin{lem}\label{ThreeChordLemma}Let $\varphi : [a,b] \to \reals$ be convex.  Then for every
  $a < x < b$, we have
\begin{align*}
\frac{\varphi(x) - \varphi(a)}{x - a} &\leq \frac{\varphi(b) -
  \varphi(a)}{b - a}  \leq \frac{\varphi(b) - \varphi(x)}{b - x} 
\end{align*}
If $\varphi$ is strictly convex then the inequalities may be replaced
by strict inequalities.
\end{lem}
\begin{proof}
Note that we can write $x = t a + (1-t) b$
with $t = \frac{b-x}{b-a} \in [0,1]$.  So applying the definition of
convexity we know that $\varphi(x) \leq t \varphi(a) +
(1-t)\varphi(b)$ and using the fact that $1-t = \frac{x-a}{b-a}$ we get
\begin{align*}
\frac{\varphi(x) - \varphi(a)}{x - a} &\leq \frac{t \varphi(a) +
(1-t)\varphi(b) - \varphi(a)}{x - a} = \frac{1-t}{x-a} (\varphi(b) -
\varphi(a) ) = \frac{\varphi(b) -
  \varphi(a)}{b - a} 
\end{align*}
and in a similar way,
\begin{align*}
\frac{\varphi(b) - \varphi(x)}{b - x} &\geq \frac{ \varphi(b)  - t \varphi(a) -
(1-t)\varphi(b) }{b -x} = \frac{t}{b-x} (\varphi(b) -
\varphi(a) ) = \frac{\varphi(b) -
  \varphi(a)}{b - a} 
\end{align*}
It is clear from the definition of strict convexity that the
inequalities above may be replaced by strict inequalities if $\varphi$ is strictly convex.
\end{proof}
\begin{lem}\label{ConvexHasDini}Let $\varphi : [a,b] \to \reals$ be a convex function, then
  for every $x \in (a,b)$, $D^-\varphi(x)$ and $D^+\varphi(x)$ exist
  and furthermore for $a < x < y < b$ we have
\begin{align*}
D^-\varphi(x) &\leq D^+\varphi(x) \leq \frac{\varphi(y) -
  \varphi(x)}{y - x} \leq D^-\varphi(y) \leq D^+\varphi(y)
\end{align*}
If $\varphi$ is strictly convex then we have
\begin{align*}
D^+\varphi(x) &< \frac{\varphi(y) -  \varphi(x)}{y - x} < D^-\varphi(y)
\end{align*}
Thus each of $D^\pm \varphi$ is monotone and has at most a countable number of discontinuities.
\end{lem}
\begin{proof}
Lemma \ref{ThreeChordLemma} shows that for $a < x < b$ and $h > 0$,
$\frac{\varphi(x + h) - \varphi(x)}{h}$ is an increasing function of
$h$ bounded below by $\frac{\varphi(x) - \varphi(a)}{x - a}$.  Thus $D^+\varphi(x) = \lim_{h \downarrow 0} \frac{\varphi(x + h) -
  \varphi(x)}{h}$ is a decreasing limit hence exists.
Similarly $\frac{\varphi(x - h) -
  \varphi(x)}{-h}=\frac{\varphi(x)-\varphi(x - h)}{h}$ is an decreasing
function of $h$ bounded above by $\frac{\varphi(b) - \varphi(x)}{b - x}$.  Thus $D^-\varphi(x) = \lim_{h \downarrow 0} \frac{\varphi(x - h) -
  \varphi(x)}{-h}$ is a bounded increasing limit hence
exists.  

The inequalities follow directly from Lemma \ref{ThreeChordLemma}.
For example, since $D^+\varphi(x) = \lim_{h \downarrow 0} \frac{\varphi(x + h) -
  \varphi(x)}{h}$ and for all $x < x+h < y$, we have $\frac{\varphi(x + h) -
  \varphi(x)}{h} \leq \frac{\varphi(y) -
  \varphi(x)}{y-x}$ we get $D^+\varphi(x) \leq \frac{\varphi(y) -
  \varphi(x)}{y-x}$.  In the strictly convex case, we know that for
any $w$ with
$x < w < y$ we have  by what we have just shown and
another application of Lemma \ref{ThreeChordLemma}
\begin{align*}
D^+\varphi(x) &\leq \frac{\varphi(w) -
  \varphi(x)}{w-x} < \frac{\varphi(y) -
  \varphi(x)}{y-x}
\end{align*}  The case of $D^-\varphi(y)$ follows analogously.

The fact that $D^\pm \varphi$ has at most a countable number of
discontinuities follows from the fact that the functions are nondecreasing.
On every finite interval $[c,d] \subset (a,b)$ and $n \in \naturals$ know that right
hand and left hand limits exist hence all discontinuities are jump discontinuities.  For 
every $n \in \naturals$ the number of discontinuities of size bigger than $1/n$ is finite
(bounded by $(c-d)*n$).  Now take the union over all $n$ and a countable set of intervals $[c,d]$ covering
$(a,b)$.
\end{proof}
\begin{cor}\label{ConvexHasSubderivative}Let $\varphi : [a,b] \to
  \reals$ be convex then for $x \in (a,b)$ there exists constants $A,B
  \in \reals$ such that $Ay + B \leq \varphi(y)$ for all $y \in [a,b]$
  and $Ax + B = \varphi(x)$.  If $\varphi$ is strictly convex then we
  may assume that $Ay + B < \varphi(y)$ for $y \neq x$.
\end{cor}
\begin{proof}
By Lemma \ref{ConvexHasDini} we can pick $D^-\varphi (x) \leq A \leq D^+\varphi (x)$.
Also by that result we know that for all $h > 0$, in fact we
have
\begin{align*}
\frac{\varphi(x) - \varphi(x-h)}{h} &\leq A \leq \frac{\varphi(x+h) - \varphi(x)}{h}
\end{align*}
which gives the result upon clearing denominators and defining $B =
\varphi(x)$.
Once again, the strictly convex case follows easily.
\end{proof}
TODO: Extend this to $\reals^n$ (presumably this can be done by taking
partial Dini Derivatives.

\begin{thm}[Jensen's Inequality]\label{Jensen}Let $\xi$ be a random vector in $\reals^n$
  and $\varphi : \reals^n \to \reals$ be a convex function such that
  $\xi$ and $\varphi(\xi)$ are integrable.  Then 
\begin{align*}
\varphi(\expectation{\xi}) \leq \expectation{\varphi(\xi)}
\end{align*}
If $\varphi$ is strictly convex then we have $\varphi(\expectation{\xi}) = \expectation{\varphi(\xi)}$ if and only if $\xi =
\expectation{\xi}$ a.s.
\end{thm}
\begin{proof}We use the fact that for every $x \in \reals^n$ we have a
  subdifferential $\langle a, y \rangle + b$ that satisfies
\begin{align*}
\langle a, y \rangle + b &\leq \varphi(y) \\
\langle a,x \rangle + b &= \varphi(x)
\end{align*}
In particular, choose such an $a,b \in \reals^n$ for the choice $x =
\expectation{\xi}$.  Then by monotonicity and linearity of integral
\begin{align*}
\expectation{\varphi(\xi)} &\geq \expectation{\langle a, \xi\rangle +
  b} \\
&= \langle a, \expectation{\xi} \rangle + b = \varphi(\xi)
\end{align*}
which gives the result.

If $\varphi$ is strictly convex then when $\xi \neq \expectation{\xi}$,
we have 
\begin{align*}
0 &< \varphi(\xi) - \varphi(\expectation{\xi}) - \langle a , \xi -
\expectation{\xi} \rangle
\end{align*}  Thus if $\varphi(\expectation{\xi}) =
\expectation{\varphi(\xi)}$ using linearity of expectation
\begin{align*}
\expectation{ (\varphi(\xi) - \varphi(\expectation{\xi}) - \langle a , \xi -
\expectation{\xi}\rangle); \xi \neq \expectation{\xi}} &= \expectation{\varphi(\xi) - \varphi(\expectation{\xi}) - \langle a , \xi -
\expectation{\xi}\rangle} \\
&=0
\end{align*}
from which we conclude $\characteristic{\xi \neq \expectation{\xi}} = 0$ a.s.
\end{proof}

We will use the following fact later on in the text.  There is no harm
is skipping over it now.
\begin{lem}\label{AbsoluteContinuityOfConvexFunctions}Let $\varphi : [a,b] \to \reals$ be convex with $f(a) = f(a+)$ and $f(b) = f(b-)$, then $\varphi$ is absolutely continuous and $\varphi^\prime$ is increasing.  In particular $D^- \varphi$ and $D^+ \varphi$ are integrable on $[a,b]$ and
for every $x \in [a,b]$, 
\begin{align*}
\varphi(x) &= \varphi(a) + \int_a^x D^+\varphi(s) \, ds = \varphi(a) + \int_a^x D^-\varphi(s) \, ds 
\end{align*}
Conversely if $\varphi$ is absolutely continuous and $\varphi$ is increasing almost everywhere then $\varphi$ is convex.
\end{lem}
\begin{proof}
Suppose that $\varphi$ is convex.  Since the result is trivial for $x=a$ we may assume that $a < x \leq b$.  Pick $\alpha$ and $\beta$ with $a < \alpha < \beta < x$.  Both $D^+\varphi$ and $D^- \varphi$ are bounded on $[\alpha, \beta]$ (e.g. a lower bound of $\frac{\varphi(\alpha) - \varphi(a)}{\alpha - a}$ and an upper bound of $\frac{\varphi(b) - \varphi(\beta)}{b - \beta}$).  Let 
$n \in \naturals$ be given and consider the partition $x_i = \alpha + i \frac{\beta - \alpha}{n}$ for $i=0, \dotsc, n$.  For any $i=2, \dotsc, n-1$ we have
\begin{align*}
\int_{x_{i-2}}^{x_{i-1}} D^+\varphi(s) \, ds  &\leq D^+\varphi (x_{i-1}) \frac{\beta - \alpha}{n} \leq \varphi(x_i) - \varphi(x_{i-1}) \leq D^-\varphi (x_i) \leq \int_{x_{i}}^{x_{i+1}} D^-\varphi(s) \, ds
\end{align*}
Summing up for $i=2, \dotsc, n-1$ we get
\begin{align*}
\int_\alpha^{x_{n-2}} D^+\varphi(s) \, ds &\leq \varphi(x_{n-1}) - \varphi(x_{1}) \leq \int_{x_2}^{\beta} D^-\varphi(s) \, ds
\end{align*}
Now let $n \to \infty$ using Dominated Convergence on the integrals and continuity of $\varphi$ (TODO: Where do we show this?)  to conclude that 
\begin{align*}
\int_\alpha^{\beta} D^+\varphi(s) \, ds &\leq \varphi(\beta) - \varphi(\alpha) \leq \int_{\alpha}^{\beta} D^-\varphi(s) \, ds
\end{align*}
On the other hand from Lemma \ref{ConvexHasDini} we know that $D^- \varphi(s) \leq D^+ \varphi(s)$ for $a <  s < \beta$ and therefore we conclude
\begin{align*}
\int_\alpha^{\beta} D^+\varphi(s) \, ds &= \int_{\alpha}^{\beta} D^-\varphi(s) \, ds = \varphi(\beta) - \varphi(\alpha) 
\end{align*}

We know that $D^\pm \varphi(s)$ are bounded above by $D^+ \varphi (\beta)$ (hence also bounded above by $D^+ \varphi (b)$ right; does this help with anything?) therefore we may apply Monotone Convergence
to $\characteristic{[\alpha,\beta]}(s) (D^+ \varphi (\beta) - D^\pm \varphi(s))$ and use the assumption $f(a) = f(a+)$ to conclude
\begin{align*}
\int_a^\beta D^\pm \varphi(s) \, ds = -(\beta - a) D^+ \varphi (\beta) -\lim_{\alpha \to a^+} \int_\alpha^\beta(D^+ \varphi (\beta) - D^\pm \varphi(s)) \, ds \\
&=-(\beta - a) D^+ \varphi (\beta) - \lim_{\alpha \to a^+} ((\beta-\alpha) D^+ \varphi (\beta) - (\varphi(\beta) - \varphi(\alpha))) = \varphi(\beta) - \varphi(a) \\
\end{align*}
In particular, 
\begin{align*}
\int_a^\beta (-D^\pm \varphi(s) \maxop 0) \, ds &= -\int_a^\beta D^\pm \varphi(s) \, ds + (\beta -a) D^+ \varphi (\beta) = \varphi(a) - \varphi(\beta) + (\beta -a) D^+ \varphi (\beta) < \infty
\end{align*}
which shows that the negative part of $D^\pm \varphi$ is integrable on every interval $[a, \beta]$ with $\beta < b$.

TODO: Finish and understand the subtly of how integrability is shown...
\end{proof}

\chapter{Independence}
\begin{defn}Given a measure space $(\Omega, \mathcal{A}, P)$, a set $T$ and a
  collection of $\sigma$-algebras $\mathcal{F}_t$ for $t \in T$, we
  say that the $\mathcal{F}_t$ are \emph{k-ary independent} if for every
  finite subset $t_1,\dots, t_n \in T$ with $n \leq k$ and every $A_{t_i} \in
  \mathcal{F}_{t_i}$ we have $\probability{A_{t_1} \cap \cdots \cap A_{t_n}} =
  \probability{A_{t_1}}\cdots\probability{A_{t_n}}$.  We say that
  $\mathcal{F}_t$ are \emph{independent} if the
  $\mathcal{F}_t$ are k-ary independent for every $k>0$.  It is common
  to refer to independent events as \emph{jointly independent}
  or \emph{mutually independent}
  events when it is desirable to provide emphasis that we are not
  considering k-ary independence for some particular value of k.  Furthermore, 2-ary independent events are often referred to
  as \emph{pairwise independent} events.
\end{defn}
\begin{defn}Given a probability space $(\Omega, \mathcal{A}, P)$, a set $T$ and a
  collection of random elements $\xi_t : (\Omega, \mathcal{A}) \to
  (S_t, \mathcal{S}_t)$ for $t \in T$, we
  say that the $\xi_t$ are \emph{independent} if the $\sigma$-algebras
  $\sigma(\xi_t)$ are independent.
\end{defn}
\begin{examp}Given two sets $A,B \in \mathcal{A}$ it is easy to see that $\sigma(A)$
  and $\sigma(B)$ are independent if and only if $\probability{A \cap B} =
  \probability{A}\cdot \probability{B}$ thus the notion of
  independence of $\sigma$-algebras generalizes the
  simple notion of independence from elementary probability.
\end{examp}
\begin{examp}Consider the space of triples
  $\{ (0,0,0),(0,1,1),(1,0,1),(1,1,0) \}$ with a uniform distribution.  Let
  $\xi_1,\xi_2,\xi_3$ be the coordinate functions.  Note that each of
  $\xi_i$ is uniformly distributed and that each joint distribution
  $(\xi_i,\xi_j)$ for $i \neq j$ is uniformly distributed as well.
  This shows that the $\xi_i$ are pairwise independent.  On the other
  hand, note that joint distribution $(\xi_1,\xi_2,\xi_3)$ is also
  uniformly distributed hence does not equal the product of the
  marginal distributions hence the $\xi_i$ are not jointly
  independent.  Intuitively the source of the dependence is clear; we
  have arranged the sample space so that specifying two coordinate
  values determines the value of the third coordinate.  Note this
  example can also be framed in a more elementary way in terms of
  events.  Consider the events $A_1 = \{(0,0,0),(0,1,1)\}$, $A_2 =
  \{(0,0,0),(1,0,1)\}$ and $A_3 = \{(0,1,1),(1,0,1)\}$.  Note that the
  events are pairwise independent but not independent.
\end{examp}
\begin{examp}\label{IndependenceOnProductSpaces}For $i=1, \dotsc, n$ let $(\Omega_i, \mathcal{A}_i, P_i)$
  be a probability space, $(S_i, \mathcal{S}_i)$ be a measurable space and suppose that $f_i : \Omega_i \to S_i$ is
a random element.  Define the probability space $(\Omega,
  \mathcal{A}, P)$ to be the product $(\Omega_1 \times \dotsb \times \Omega_n,
  \mathcal{A}_1 \otimes \dotsb \otimes \mathcal{A}_n, P_1 \otimes
  \dotsb \otimes P_n)$, for each $i=1, \dotsc, n$ let $\pi_i : \Omega \to \Omega_i$ be the
$i^{th}$ coordinate projection and let $g_i = f_i \circ \pi_i$ (i.e. $g_i(\omega_1, \dotsc, \omega_n) = f_i(\omega_i)$) then
$g_1, \dotsc, g_n$ are independent.  This follows by noting that for $A_i \in \mathcal{S}_i$ we have by definition of the 
product measure
\begin{align*}
\probability{g_1^{-1} (A_1) \cap \dotsb \cap g_n^{-1} (A_n)} &= \probability{\pi_1^{-1}(f_1^{-1} (A_1)) \cap \dotsb \cap \pi_n^{-1}(f_n^{-1} (A_n))} \\
&= \probability{\pi_1^{-1}(f_1^{-1} (A_1))} \dotsb \probability{\pi_n^{-1}(f_n^{-1} (A_n))} \\
&= \probability{g_1^{-1} (A_1)} \dotsb \probability{g_n^{-1} (A_n)} 
\end{align*}
\end{examp}

\begin{lem}\label{IndependenceProductMeasures}Suppose we are given a finite collection of
  random elements $\xi_1, \dots, \xi_n$ in measurable spaces
  $S_1, \dots, S_n$ with distributions $\mu_1, \dots, \mu_n$.  The
  $\xi_i$ are independent if and only if the distribution of $(\xi_1,
  \dots, \xi_n)$ on $S_1 \times \cdots \times S_n$ is $\mu_1 \otimes
  \cdots \otimes \mu_n$.
\end{lem}
\begin{proof}If we assume that joint distribution of the $\xi_i$ is $\mu_1 \otimes
  \cdots \otimes \mu_n$ then clearly the $\xi_i$ are independent since 
\begin{align*}
\probability{\xi_1^{-1}(B_1) \cap \cdots \cap \xi_n^{-1}(B_n)} &=
\probability{(\xi_1, \dots, \xi_n)^{-1}(B_1 \times \cdots \times B_n)}
  \\
&= (\mu_1 \otimes \cdots \otimes \mu_n)(B_1 \times \cdots \times B_n) \\
&= \mu_1(B_1) \cdots \mu_n(B_n) \\
&=\probability{\xi_1^{-1}(B_1)} \cdots \probability{\xi_n^{-1}(B_n)}
\end{align*}

On the other hand, if we assume that the $\xi_i$ are independent then
above calculation shows that $(\pushforward{(\xi_1, \dots, \xi_n)}{P}) =
\mu_1 \otimes \cdots \otimes \mu_n$ on cylinder sets which together
with the finiteness of probability measures shows that
they are equal everywhere by the uniqueness of product measure proved
in Theorem \ref{Fubini}.
\end{proof}

Having proven that the joint distribution of independent random
elements is a product measure we can apply Fubini's Theorem to compute
expectations of functions of independent random elements as iterated
integrals.  We make that statement precise in the following result.
Note that there is an important generalization of this fact that
eliminates the assumption of independence but requires the development
of the notion of a conditional distribution (see Theorem
\ref{Disintegration}).  The current result is much simpler than the notation
required to state it.

\begin{lem}\label{DisintegrationIndependentLaws}Let $\xi$ and $\eta$ be
  independent random elements in measurable spaces $(S, \mathcal{S})$
  and $(T, \mathcal{T})$ respectively.  Let $f : S \times T \to
  \reals$ be a measurable function and define 
  $g(s) = \expectation{f(s, \eta)}$ and 
  $h(s) = \expectation{\abs{f(s, \eta)}}$. Suppose that either $f$ is
  non-negative or $h(\xi)$ is integrable, then
$\expectation{f(\xi, \eta)} = \expectation{g(\xi)} =
\expectation{\expectation{f(s, \eta)}\mid_{s=\xi}} $.
\end{lem}
\begin{proof}
Let $\mu$ be the distribution of $\xi$ and $\nu$ be the distribution
of $\eta$; by Lemma \ref{IndependenceProductMeasures} we know that the
joint distribution of $(\xi, \eta)$ is $\mu \otimes \nu$.  Suppose
that $f$ is non-negative and use the Expectation Rule (Lemma \ref{ExpectationRule}) and
Tonelli's Theorem \ref{Fubini} to calculate
\begin{align*}
\expectation{f(\xi, \eta)} &= \int f(x,y) \, d(\mu \otimes \nu)(x,y)
\\
&=
\int \left [ \int f(x, y) \, d\nu(y) \right ] \, d\mu(x) = \int g(x)
\, d\mu(x) = \expectation{g(\xi)}
\end{align*}

If instead assuming $f \geq 0$, we assume that $h(\xi)$ is integrable.
Applying the result just proven for the non-negative case to $\abs{f}$
shows that in fact $\expectation{ \abs{f(\xi, \eta)}} < \infty$ so we
may replay the same argument for $f$ without the absolute value using Fubini's Theorem
\ref{Fubini} in place of Tonelli's Theorem.
\end{proof}

The fact that the joint distribution of independent random variables
only depends on the distribution of the underlying random variables
has the important consequence that the distribution of \emph{sums} of
independent random variables also only depends on the distribution of
the underlying random variables.  However we can actually be a bit
more precise than that.

\begin{defn}A \emph{measurable group} is a group $G$ with a
  $\sigma$-algebra $\mathcal{G}$ such that the group inverse is
$\mathcal{G}$-measurable and the group operation is $\mathcal{G}
\otimes \mathcal{G} / \mathcal{G}$-measurable.
\end{defn}
\begin{defn}Given two $\sigma$-finite measures $\mu$ and $\nu$ on a measurable group
  $(G, \mathcal{G})$, the \emph{convolution} $\mu * \nu$ is the measure on $G$ defined
  by taking the pushforward of $\mu \otimes \nu$ under the group operation.
\end{defn}
\begin{lem}\label{Convolution}Convolution of measures on a measurable group $(G,
  \mathcal{G})$ is associative.  Furthermore, if $G$ is Abelian, then
  convolution of measures is commutative and we have the formula
\begin{align*}
\mu * \nu (B) &= \int \mu(B - g) \, d \nu(g) = \int \nu(B - g) \,  d \mu(g)
\end{align*}
\end{lem}
\begin{proof}First we derive the formula for the convolution of two
  measures as integrals.
Suppose we are given
  $\sigma$-finite measures $\mu, \nu$ and  a measurable
  $A \in \mathcal{G}$.  Define $A^2 = \lbrace(g,h) \mid gh \in
  A\rbrace$ and then the definition of the pushforward of a
  measure,the construction of product measure and Tonelli's Theorem we get
\begin{align*}
\left(\mu*\nu \right)(A) &= \left(\mu \otimes \nu\right) (A^2) \\
&= \int
\int \characteristic{A^2}(g,h) \, d(\mu\otimes \nu)(g,h) \\
&= \int \left [
\int \characteristic{A^2}(g,h) \, d \mu(g) \right ]
d\nu(h) \\
&= \int \left [
\int \characteristic{A^2}(g,h) \, d\nu(h) \right ]
d \mu(g) \\
\end{align*}
Now consider the inner integral for a fixed $h \in G$ and define for
each such fixed $h$ the right translation $Ah^{-1}$ and note that as a
function of $g$ alone, $\characteristic{A^2}(g,h) =
\characteristic{Ah^{-1}}(g)$.  Similarly, for fixed $g$ we introduce
the left translation $g^{-1}A$ and have  $\characteristic{A^2}(g,h) =
\characteristic{g^{-1}A}(h)$. Substituting into the integrals above, 
\begin{align*}
\left(\mu*\nu \right)(A) &= \int \mu(A\cdot g^{-1}) \, d \nu(g) = \int
\nu(g^{-1}\cdot A) \, d \mu(g)
\end{align*}
In particular, if $G$ is Abelian then $g^{-1}\cdot A = A\cdot g^{-1}$ and we have
the formula above.

To see the associativity is an application of Tonelli's
  Theorem with a bit of messy notation.  Suppose we are given
  $\sigma$-finite measures $\mu_1, \mu_2, \mu_3$ and  a measurable
  $A \in \mathcal{G}$.  Define $A^3 = \lbrace(g,h,k) \mid ghk \in
  A\rbrace$ and note that for fixed $h,k$ we have
  $\characteristic{A^3}(g,h,k) = \characteristic{Ak^{-1}h^{-1}}(g)$
  and for fixed $g,h$ we have $\characteristic{A^3}(g,h,k) =
  \characteristic{k^{-1}g^{-1}A}(k)$ 
Now applying this observation and the integral formula above
\begin{align*}
\left((\mu_1 * \mu_2)*\mu_3 \right)(A) &= \int (\mu_1 *
\mu_2)(Ak^{-1}) \, d\mu_3(k) \\
&= \int \int \mu_1(Ak^{-1}h^{-1}) \, d\mu_2(h) d\mu_3(k) \\
&= \int \int \int \characteristic{A^3}(g,h,k) \, d \mu_1(g) d\mu_2(h) d\mu_3(k) \\
&= \int \int \int \characteristic{A^3}(g,h,k) \, d\mu_3(k) d\mu_2(h) d \mu_1(g) \\
&= \int \int \mu_3(h^{-1} g^{-1} A) \, d\mu_2(h) d\mu_1(g) \\
&= \int  (\mu_2 *\mu_3)(g^{-1} A) \, d\mu_1(g) \\
&= \left(\mu_1 *  (\mu_2 *\mu_3) \right )(A)
\end{align*}
\end{proof}
\begin{defn}A measure $\mu$ on a measurable group
  $(G, \mathcal{G})$ is said to be \emph{left invariant} if for every
  $g \in G$ and $A \in \mathcal{G}$, $\mu(g\cdot A) = \mu(A)$.  A
  measure is said to be \emph{right invariant} if for every
  $g \in G$ and $A \in \mathcal{G}$, $\mu(A\cdot g) = \mu(A)$.  A
  measure that is both right invariant and left invariant is said to
  be \emph{invariant}.
\end{defn}
\begin{lem}\label{ConvolutionDensity}Let $\lambda$ be an invariant measure on a measurable Abelian group
  $(G, \mathcal{G})$ and let $\mu = f \cdot \lambda$ and $\nu = g
  \cdot \lambda$ be measures which have densities with respect to
  $\lambda$.  Then $\mu * \nu$ has the $\lambda$-density
\begin{align*}
(f * g)(x) &= \int f(x-y) g(y) \, d\lambda(y)
\end{align*}
\end{lem}
\begin{proof}
By the integral formula for convolution, given $A \in \mathcal{G}$,
\begin{align*}
(\mu * \nu)(A) &= \int \mu(A - y) \, d \nu(y) \\
&= \int \int\characteristic{A - y}(x) f(x) g(y) \, d\lambda(x)
d\lambda(y) \\
&= \int \int\characteristic{A}(x+y) f(x) g(y) \, d\lambda(x)
d\lambda(y) \\
&= \int \int\characteristic{A}(x) f(x-y) g(y) \, d\lambda(x)
d\lambda(y) \\
&= \int \characteristic{A}(x) \left [ \int f(x-y) g(y) \, d\lambda(y)
  \right ] d\lambda(x) \\
&= ((f*g) \cdot \lambda)(A)
\end{align*}
\end{proof}
\begin{examp}Let $\xi$ and $\eta$ be independent $N(0,1)$ random
  variables.  Then $\xi + \eta$ is an $N(0,2)$ random variable.  From
  Corollary \ref{ConvolutionDensity}, we
  know $\xi + \eta$ has density given by the convolution of Gaussian
  densities.
\begin{align*}
\frac{1}{2\pi} \int e^\frac{-(x-y)^2}{2} e^\frac{-y^2}{2} dy &=
\frac{1}{2\pi} \int e^{-(y^2 -xy + \frac{1}{2}x^2)} dy =
\frac{1}{2\pi} e^\frac{-x^2}{4} \int e^{-(y - \frac{x}{2})^2} dy =
\frac{1}{\sqrt{4\pi}} e^\frac{-x^2}{4}
\end{align*}
\end{examp}

\begin{lem}\label{IndependencePiSystem}Suppose we are given two
  $\pi$-systems $\mathcal{S}$ and $\mathcal{T}$ in a probability space
  $(\Omega, \mathcal{A}, P)$ such that
  $\probability{A \cap B} = \probability{A} \probability{B}$ for all
  $A \in \mathcal{S}$ and $B \in \mathcal{T}$.  Then
  $\sigma(\mathcal{S})$ and $\sigma(\mathcal{T})$ are independent.
\end{lem}
\begin{proof}
This is simply a pair of monotone class arguments.  First pick
arbitrary element $A
\in \mathcal{A}$.  We define $\mathcal{C} = \lbrace B \in
  \mathcal{A} \mid  \probability{A \cap B} =  \probability{A}
  \probability{B} \rbrace$.  We claim that $\mathcal{C}$ is a
  $\lambda$-system.  First it is clear that $\Omega \in \mathcal{C}$.
  Next assume that $B,C \in \mathcal{C}$ with $C \supset B$.  Then $C
\setminus B \in \mathcal{C}$ because
\begin{align*}
\probability{A\cap ( C \setminus B)} &= \probability{ (A \cap C)
  \setminus (A \cap B)} \\
&=\probability{ A \cap C}
  - \probability{A \cap B} \\
&=\probability{ A} \probability{C}
  - \probability{A} \probability{B} \\
&=\probability{A} \left(\probability{C}
  - \probability{B} \right) =  \probability{ A}
\probability{C\setminus B}\\
\end{align*}
Next assume that $B_1 \subset B_2 \subset \cdots$ with $B_i \in
\mathcal{C}$. We have $\cup_{n=1}^\infty B_n \in \mathcal{C}$ by the calculation
\begin{align*}
\probability{A\cap\cup_{n=1}^\infty B_n} &=
\probability{\cup_{n=1}^\infty A \cap B_n} & & \text{by DeMorgan's
  Law} \\
&=\lim_{n \to \infty} \probability{ A \cap B_n} & & \text{by
  Continuity of Measure} \\
&=\lim_{n \to \infty} \probability{ A} \probability{B_n} & &
\text{since $B_n \in \mathcal{C}$}\\
&=\probability{ A} \lim_{n \to \infty} \probability{B_n} \\
&=\probability{A} \probability{\cup_{n=1}^\infty B_n} & & \text{by
  Continuity of Measure}\\
\end{align*}

Our assumption is that if we pick $A \in \mathcal{S}$, then $\mathcal{T} \subset \mathcal{C}$ so the
$\pi$-$\lambda$ Theorem (Theorem \ref{MonotoneClassTheorem}) shows
that $\sigma(\mathcal{T}) \subset \mathcal{C}$.  Since our choice of
$A \in \mathcal{S}$ can be arbitrary, we know for every for every $A\in \mathcal{S}$ and
every $B \in \sigma(\mathcal{T})$ we have $\probability{A \cap B} =
\probability{A} \probability{B}$.  

It remains to extend $\mathcal{S}$ to $\sigma(\mathcal{S})$.  This is
done in exactly the same way.   Pick a $B \in \sigma(\mathcal{T})$ and define $\mathcal{D} \lbrace A \in
  \mathcal{A} \mid  \probability{A \cap B} =  \probability{A}
  \probability{B} \rbrace$.  We have shown that  $\mathcal{D}$ is a
  $\lambda$-system and that $\mathcal{S} \subset \mathcal{D}$ hence
  the $\pi$-$\lambda$ Theorem gives us $\mathcal{D} \supset
  \sigma(\mathcal{S})$.  Since $B\in \sigma(\mathcal{T})$ was
  arbitrary we have shown independence of $\sigma(\mathcal{S})$ and $\sigma(\mathcal{T})$.
\end{proof}

\begin{lem}\label{IndependenceGrouping}Let $\mathcal{A}_t$ for $t \in
  T$ be an independent family of $\sigma$-algebras on $\Omega$.  The
  for any disjoint partition $\mathcal{T}$ of $T$ we have
  $\sigma(\bigcup_{s \in S} \mathcal{A}_s)$ are independent where $S
  \in \mathcal{T}$.
\end{lem}
\begin{proof}
For $S$ and element of the partition of $T$, let $\mathcal{C}_S$ be
the set of all finite intersections of elements from $\cup_{s \in S}
\mathcal{A}_s$.  Clearly each $\mathcal{C}_S$ is a $\pi$-system that
generates $\sigma (\bigcup_{s \in S} \mathcal{A}_s)$.  Moreover,
the independence of the $\mathcal{A}_t$ for all $t \in T$ shows that
the $\mathcal{C}_S$ are independent $\pi$-systems by
associativity of finite intersection of sets and multiplication in $\reals$.
Thus Lemma \ref{IndependencePiSystem} shows the result.
\end{proof}

In order to prove independence of a countable collection of
$\sigma$-algebras it can be useful to reduce the task to showing a
sequence of pairwise independent relationships as in the following
Lemma.
\begin{lem}\label{IndependenceByPairwiseIndependence}Let
  $\mathcal{A}_1, \mathcal{A}_2, \dotsc$ be $\sigma$-algebras, then
  they are independent if and only if $\bigvee_{k=1}^n \mathcal{A}_k$
  is independent of $\mathcal{A}_{n+1}$ for all $n \geq 1$.
\end{lem}
\begin{proof}
The only if direction is an application of Lemma
\ref{IndependenceGrouping}.  The if direction will be shown by
induction.  To set notation, suppose that $A_{k_1} \in
\mathcal{A}_{k_1}, \dotsc, A_{k_m} \in \mathcal{A}_{k_m}$ are chosen
and we must show $\probability{A_{k_1} \cap \dotsb \cap A_{k_m}} = 
\probability{A_{k_1} } \dotsb \probability{ A_{k_m}} $ where without
log of generality we assume $1 \geq k_1 < \dotsb < k_m$.  If we let $n
= k_1 \vee \dotsb \vee k_m=k_m$ the induction variable is $n$.  The case
of $n=1$ is trivial as there is nothing to prove, so suppose the
result is true for $n-1$ and we are given $A_{k_1} \in
\mathcal{A}_{k_1}, \dotsc, A_{k_m} \in \mathcal{A}_{k_m}$ with $k_m =
n$.  Using the hypothesis, the fact that $k_{m-1} < n$ and induction
hypothesis we know that 
\begin{align*}
\probability{A_{k_1} \cap \dotsb \cap A_{k_m}} &= \probability{A_{k_1}
  \cap \dotsb \cap A_{k_{m-1}}} \probability{A_{k_m}} \\
&=\probability{A_{k_1}} \dotsb \probability{A_{k_m}}
\end{align*}
and the result is proven.
\end{proof}

Note that the previous lemma can be taken as demonstrating that
independence of sets cannot be destroyed by applying the operations of
complementation, countable union and countable intersection.  The
property of independence is also very robust in the sense that it
cannot be destroyed by composition with any measurable mapping.
\begin{lem}\label{IndependenceComposition}A
  finite collection of random elements $\xi_1, \dots, \xi_n$ in
  measurable spaces $(S_1,\mathcal{S}_1), \cdots, (S_n,
  \mathcal{S}_n)$ is independent if and only $f_1 \circ \xi_1, \cdots,
  f_n \circ \xi_n$ is independent for every measurable $f_1, \cdots, f_n$.
\end{lem}
\begin{proof}The reverse implication is clear because the identity on
  every $(S_i, \mathcal{S}_i)$ is measurable.

Now if $\xi_i$ are independent then by definition $\sigma(\xi_i)$ are
independent $\sigma$-algebras.  But for any measurable $f_i$, $\sigma(f_i \circ \xi_i) \subset
\sigma(\xi_i)$ and therefore the $f_1 \circ \xi_1, \cdots,
  f_n \circ \xi_n$ are independent.
\end{proof}

Implicit in a few of the above proofs is the fact that independence
among groups of independent objects can be reduced to checking
independence of finite subsets within the groups.  Here is a
codification of this fact stated in the simple case of checking
pairwise independence.
\begin{lem}\label{IndependenceFinitary}Let $\mathcal{F}_t$ and
  $\mathcal{G}_s$ be sets of $\sigma$-algebras. Then
  $\sigma(\bigcup_{t \in T} \mathcal{F}_t)$ is independent of
  $\sigma(\bigcup_{s \in S} \mathcal{G}_s)$ if an only if for every
  finite subset $T^\prime \subset T$ and $S^\prime \subset S$, we have $\sigma(\bigcup_{t \in T^\prime} \mathcal{F}_t)$ is independent of
  $\sigma(\bigcup_{s \in S^\prime} \mathcal{G}_s)$
\end{lem}
\begin{proof}
One direction of this is trivial.  For the other direction suppose we
have independence over each of the finite subsets.  To prove the
result note that set of finite intersections of elements of
$\bigcup_{t \in T} \mathcal{F}_t$
is a $\pi$-system that generates $\sigma(\bigcup_{t \in T}
\mathcal{F}_t)$ (and similarly with $S$).  Our assumption tells us
that these $\pi$-systems are independent hence we appeal to Lemma \ref{IndependencePiSystem}.
\end{proof}

TODO: In the following I think we can generalize to the requirement
that the $f_j(\xi_j)$ are integrable.
\begin{lem}\label{IndependenceExpectations}A
  finite collection of random elements $\xi_1, \dots, \xi_n$ in
  measurable spaces $(S_1,\mathcal{S}_1), \cdots, (S_n,
  \mathcal{S}_n)$ is
  independent if and only if 
\begin{align*}
\expectation{f_1(\xi_1) \cdots f_n(\xi_n)} =
  \expectation{f_1(\xi_1)} \cdots \expectation{f_n(\xi_n)}
\end{align*}
for all $f_i : S_n \to \reals$ that are either bounded
measurable or positive measurable.
\end{lem}
\begin{proof}
Note that for the special case $f_i = \characteristic{A_i}$ for 
Borel sets $A_i \in \mathcal{B}(\reals)$, $f_i(\xi_i) =
\characteristic{f_i^{-1}(A_i)}$ and therefore the claim is equivalent to
the definition of independence as we can see by the following
calculation
\begin{align*}
\expectation{f_1(\xi_1) \cdots f_n(\xi_n)} &=
\expectation{\characteristic{f_1^{-1}(A_1)} \cdots
  \characteristic{f_n^{-1}(A_n)}} \\
&= \probability{f_1^{-1}(A_1) \cap \cdots \cap f_n^{-1}(A_n)} \\
&= \probability{f_1^{-1}(A_1) } \cdots \probability{f_n^{-1}(A_n)} \\
&=\expectation{f_1(\xi_1)} \cdots \expectation{f_n(\xi_n)}
\end{align*}
Therefore if we assume the result for all positive or bounded measurable
$f$ then we certainly have independence.  

On the other hand if we assume independence of the $\xi_i$ then we
know that the desired result holds for $f_i$ that are indicator
functions.  It remains to apply the standard machinery to derive the
result for more general $f_i$.

For $f_i$ simple functions we simply use linearity of expectation.  If
we write $f_i = c_{1,i} \characteristic{A _{1_i,i}} + \cdots + c_{m_i,i}
\characteristic{A _{m_i, i}}$ then 
\begin{align*}
\expectation{f_1(\xi_1) \cdots f_n(\xi_n)} &= \sum_{k_1 = 1}^{m_1}
\cdots \sum_{k_n = 1}^{m_n} c_{k_1, 1} \cdots c_{k_n, n}
\expectation{\characteristic{A_{k_1,1}}(\xi_1)
  \cdots\characteristic{A_{k_n,n}}(\xi_n)} \\
&= \sum_{k_1 = 1}^{m_1}
\cdots \sum_{k_n = 1}^{m_n} c_{k_1, 1} \cdots c_{k_n, n}
\expectation{\characteristic{A_{k_1,1}}(\xi_1)} \cdots
\expectation{\characteristic{A_{k_n,n}}(\xi_n)} \\
&= \sum_{k_1 = 1}^{m_1} c_{k_1, 1} \expectation{\characteristic{A_{k_1,1}}(\xi_1)} 
\cdots \sum_{k_n = 1}^{m_n} c_{k_n, n}
\expectation{\characteristic{A_{k_n,n}}(\xi_n)} \\
&=\expectation{f_1(\xi_1)} \cdots \expectation{f_n(\xi_n)}
\end{align*}

To show the result for positive $f$, first start by assuming that
$f_1$ is positive and $f_2, \cdots, f_n$ are simple.  Pick $f_{i,1}$
increasing simple functions such that $f_{i,1} \uparrow f_1$.  Then we
have $f_{i,1} f_2 \cdots f_n \uparrow f_1 f_2 \cdots f_n$ we have
\begin{align*}
\expectation{f_1(\xi_1) \cdots f_n(\xi_n)} &= \lim_{i\to
  \infty}\expectation{f_{i,1}(\xi_1) \cdots f_n(\xi_n)} & & \text{by
  Monotone Convergence} \\
&=\lim_{i\to
  \infty}\expectation{f_{i,1}(\xi_1)} \cdots \expectation{f_n(\xi_n)}
& & \text{result for simple functions} \\
&=\expectation{f_1(\xi_1)} \cdots \expectation{f_n(\xi_n)} & &\text{by
  Monotone Convergence}
\end{align*}
Having shown the result for $f_1$ positive and $f_2, \cdots, f_n$
simple just iterate with Monotone Convergence as above to see the result for all
$f_1, \cdots, f_n$ positive.

For $f_i$ bounded, first write $f_1 = f_1^+ - f_1^-$ with $f_1^\pm
\geq 0$ and bounded and assume that $f_2, \dots, f_n$ are positive and
bounded.  Note that
$f_1^\pm \circ \xi$ is integrable by the boundedness of $f_1^\pm$.
Therefore by linearity of expectation and the fact that we have proven
the result for positive $f_i$
\begin{align*}
\expectation{f_1(\xi_1) f_2(\xi_2) \cdots f_n(\xi_n)} &=
\expectation{f_1^+(\xi_1) f_2(\xi_2) \cdots f_n(\xi_n)} -
\expectation{f_1^-(\xi_1) f_2(\xi_2) \cdots f_n(\xi_n)} \\
&=\expectation{f_1^+(\xi_1)}\expectation{f_2(\xi_2)} \cdots
\expectation{f_n(\xi_n)} \\ 
&-\expectation{f_1^-(\xi_1)}\expectation{f_2(\xi_2)} \cdots
\expectation{f_n(\xi_n)} \\
&=\expectation{f_1(\xi_1)}\expectation{f_2(\xi_2)} \cdots
\expectation{f_n(\xi_n)} 
\end{align*}
Now perform induction on $i$ to get the final result.
\end{proof}

\begin{examp} TODO:  Find an example where this fails for integrable
  $f$.  I'm pretty sure the crux is to find $f$ that is integrable for which $f \circ
  \xi$ is not.  In any case if one finds such a pair, then the result
  doesn't really even make sense since not all of the expectations are defined.
\end{examp}

\begin{cor}Suppose $f,g$ are independent integrable random variables
 then $fg$ is integrable and $\expectation{fg} = \expectation{f}\expectation{g}$.
\end{cor}
\begin{proof}By Lemma
  \ref{IndependenceExpectations}, independence of $f,g$ and positivity and measurability of $\abs{x}$, we see that 
\begin{align*}\expectation{\abs{fg}} &= \expectation{\abs{f}\cdot \abs{g}} =
  \expectation{\abs{f}}\expectation{\abs{g}} < \infty
\end{align*}
showing integrability of $f g$.

This argument also shows that $\expectation{fg} =
\expectation{f}\expectation{g}$ for positive $f,g$.  To extend to to
integrable $f,g$ write $f = f_+ - f_-$ and $g = g+ - g_-$ and use
linearity of expectation
\begin{align*}
\expectation{fg} &= \expectation{f_+ g_+} - \expectation{f_+ g_-} -
\expectation{f_- g_-} + \expectation{f_- g_-} \\
&= \expectation{f_+}\expectation{g_+} - \expectation{f_+}\expectation{ g_-} -
\expectation{f_-}\expectation{ g_-} + \expectation{f_-}\expectation{
  g_-} \\
&= \left( \expectation{f_+} - \expectation{f_-}\right) \left( \expectation{g_+} - \expectation{g_-}\right)\\
&= \expectation{f}\expectation{g}
\end{align*}
\end{proof}

\begin{examp}\label{UncorrelatedNotIndependent}This is an example of random variables $\xi$ and
  $\eta$ such that $\expectation{\xi \eta} =
  \expectation{\xi} \cdot \expectation{\eta}$ (are
  \emph{uncorrelated}) but $\xi$ and $\eta$ are
  not independent.

Consider the sample space $\Omega = \{1,2,3\}$ with uniform distribution.  A
random variable $\xi:\Omega \to \reals$ is just a vector in $\reals^3$.  Let $\xi = (1,-1,0)$
and let $\eta=(-1,-1,2)$.  Note that
$\expectation{\xi}=\expectation{\eta} = \expectation{\xi \eta} = 0$
and therefore $\xi$ and $\eta$ are uncorrelated.  On the other hand
$\xi$ and $\eta$ are not independent; for example 
\begin{align*}
0=\probability{\xi=1
  \wedge \eta = 2} \neq
\probability{\xi=1}\probability{\eta=2}=\frac{1}{9}
\end{align*}
\end{examp}

\begin{defn}Given a sequence of events $A_n$ the event that $A_n$
  occurs \emph{infinitely often} is the set $\bigcap_{n=1}^\infty
  \bigcup_{k=n}^\infty A_k = \limsup_{n \to \infty} A_n$.  The
  probability that $A_n$ occurs infinitely often is often written $\probability{A_n \text{ i.o.}}$.
\end{defn}

\begin{thm}\label{BorelCantelli}[Borel Cantelli Theorem]Let
  $(\Omega,\mathcal{A},P)$ be a probability space and let $A_1, A_2,
  \dots \in \mathcal{A}$.
 \begin{itemize}
\item[(i)] If $\sum_{i=1}^\infty \probability{A_i} < \infty$
  then $\probability{A_i \text{ i.o.}} = 0$.  
\item[(ii)]If the $A_i$ are
  independent and $\probability{A_i \text{ i.o.}} = 0$, then we have
  $\sum_{i=1}^\infty \probability{A_i} < \infty$.  More precisely, if
  $\sum_{i=1}^\infty \probability{A_i} = \infty$ then $\probability{A_i \text{ i.o.}} = 1$.
\end{itemize}
\end{thm}
\begin{proof}
To prove (i) we observe that the convergence of $\sum_{i=1}^\infty
\probability{A_i}$ implies that the partial sums converge to zero,
$\lim_{n\to \infty} \sum_{i=n}^\infty
\probability{A_i} = 0$.  Now we apply a union bound (subadditivity of
measure) and use continuity of measure to see that
\begin{align*}
\probability{A_n \text{ i.o.}} &= \lim_{n\to \infty}
\probability{\bigcup_{k=n}^\infty A_k}  \leq  \lim_{n\to \infty} \sum_{k=n}^\infty \probability{A_k} = 0
\end{align*}

To see (ii), first observe the simple calculation
\begin{align*}
\probability{\bigcap_{n=1}^\infty \bigcup_{k=n}^\infty A_k} &=
\lim_{n\to \infty} \probability{\bigcup_{k=n}^\infty A_k}  & &\text{ by continuity of measure}\\
&= \lim_{n\to \infty} \lim_{m\to \infty} \probability{\bigcup_{k=n}^m
  A_k} & &\text{ by continuity of measure}\\
&= \lim_{n\to \infty} \lim_{m\to \infty} \left ( 1 - \probability{\bigcap_{k=n}^m
  A_k^c} \right ) & &\text{ by DeMorgan's law} \\
&= \lim_{n\to \infty} \lim_{m\to \infty}\left ( 1 - \prod_{k=n}^m\probability{
  A_k^c} \right )& &\text{by independence} \\
&=  1 - \lim_{n\to \infty} \lim_{m\to \infty}\left (\prod_{k=n}^m\left
    (1 - \probability{
  A_k} \right) \right ) \\
\end{align*}

Now we recall the 
elementary bound $1+x \leq e^x$ for $x \in \reals$ from Lemma
\ref{BasicExponentialInequalities} and assume that
$\sum_{n=1}^\infty \probability{A_n} = \infty$.  By the calculation
above we have
\begin{align*}
\probability{\bigcap_{n=1}^\infty \bigcup_{k=n}^\infty A_k} &= 1 - \lim_{n\to \infty} \lim_{m\to \infty}\left (\prod_{k=n}^m\left
    (1 - \probability{
  A_k} \right) \right ) \\
&\geq 1 - \lim_{n\to \infty} \lim_{m\to \infty}\left (\prod_{k=n}^m
   e^{- \probability{
  A_k}} \right ) \\
&= 1 - \lim_{n\to \infty} \lim_{m\to \infty}
   e^{- \sum_{k=n}^m\probability{
  A_k}} \\
&= 1
\end{align*}
But of course we know that $\probability{\bigcap_{n=1}^\infty
  \bigcup_{k=n}^\infty A_k} \leq 1$ so in fact we have shown that
$\probability {A_n \text{ i.o.}} = 1$.
\end{proof}

\begin{examp}Here is a somewhat synthetic example that shows when
  $A_n$ are dependent it is possible to have $\probability{A_n \text{
      i.o.}} = 0$ while $\sum_{n=1}^\infty \probability{A_n} =
  \infty$.  Take $([0,1], \mathcal{B}([0,1]), \lambda$ as the measure
  space.  Take the intervals $[0, \frac{1}{n}]$ in a sequence such
  that $[0, \frac{1}{n}]$ occurs $n$ times (e.g. $[0,1],
  [0,\frac{1}{2}], [0,\frac{1}{2}], [0,\frac{1}{3}],[0,\frac{1}{3}],
  [0,\frac{1}{3}], \dots$). Clearly $\lbrace A_n \text{ i.o.} \rbrace
  = \lbrace 0 \rbrace$.  On the other hand it is clear that
  $\sum_{n=1}^\infty \probability{A_n} = \infty$.
\end{examp}

\begin{examp}This is a more probabilistic example.  Consider a game in
  which there is a $n$-sided die for each $n=2, 3, \dots$.  In the
  $n^{th}$ round of the game, one rolls the $n$-sided die.  If one
  gets a $1$ then one stops the game else one continues to play.  Let
  $A_n$ be the event that the player is still alive at round $n$.  It
  is clear that player has a probability of $\frac{1}{2} \cdots
  \frac{n-1}{n} = \frac{1}{n}$ of being alive at round $n$.  It is also clear that
  the probability the player never loses is bounded by $\frac{1}{n}$
  for all $n$ hence is 0.  The probability the player never loses is
  the same as $\probability{A_n \text{ i.o.}}$ on the other hand,
  $\sum_{n=1}^\infty \probability{A_n} = \sum_{n=1}^\infty \frac{1}{n}
  = \infty$.
\end{examp}

The Borel Cantelli Theorem tells us
that $\probability{A_n \text{ i.o.}}$ can only take the values $0$ and
$1$ when the $A_n$ are independent events (and in fact gives us a test
for determining which alternative holds).  The $0/1$ dichotomy is a
general feature of sequences of independent events and describing the
nature this dichotomy motivates the following definitions.

\begin{defn}Let $\mathcal{A}_n$ be a sequence of $\sigma$-algebras on
  a space $\Omega$.  The \emph{tail $\sigma$-algebra} $\mathcal{T}_\infty$ is
  defined to be 
\begin{align*}
\mathcal{T}_\infty = \bigcap_{n=1}^\infty \sigma\left(\bigcup_{k=n}^\infty \mathcal{A}_k\right)
\end{align*}
\end{defn}

\begin{thm}[Kolmogorov's $0-1$ Law]\label{Kolmogorov01Law}Let
  $\mathcal{A}_n$ be a sequence of independent $\sigma$-algebras on
  a probability space $(\Omega, \mathcal{A}, P)$ such that $\mathcal{A}_n
  \subset \mathcal{A}$ for all $n>0$.  Then for every $T \in \mathcal{T}_\infty$ we have
  $\probability{T}=0$ or $\probability{T} = 1$.
\end{thm}
\begin{proof}Let $\mathcal{T}_n = \sigma\left(\bigcup_{k=n}^\infty
    \mathcal{A}_k\right)$ and $\mathcal{S}_n = \sigma\left(\bigcup_{k=1}^{n-1}
    \mathcal{A}_k\right)$.  Then by Lemma \ref{IndependenceGrouping}
  we see that $\mathcal{T}_n$ and $\mathcal{S}_n$ are independent.
  Therefore for $A \in \mathcal{T}_n$ and $B \in \mathcal{S}_n$ we
  have $\probability{A \cap B} = \probability{A} \probability{B}$.

Now pick $A \in \mathcal{T}_\infty$, then by the above observation we
have $\probability{A \cap B} = \probability{A} \probability{B}$ for $B
\in \bigcup_{n=1}^\infty \mathcal{S}_n$.  Since $\mathcal{S}_1 \subset
\mathcal{S}_1 \subset \cdots$, we can easily see that
$\bigcup_{n=1}^\infty \mathcal{S}_n$ is a $\pi$-system.  Given $B_1,
B_2 \in \bigcup_{n=1}^\infty \mathcal{S}_n$ there exist $n_1,n_2$ such
that $B_i \in \mathcal{S}_{n_i}$ for $i=1,2$.  Then define $n =
\max(n_1,n_2)$ and $B_i \in \mathcal{S}_n$ for $i=1,2$ and therefore
$B_1 \cap B_2 \in \mathcal{S}_n \subset \bigcup_{n=1}^\infty
\mathcal{S}_n$.  Applying Lemma \ref{IndependencePiSystem} we conclude
that $\mathcal{T}_\infty$ and $\sigma(\bigcup_{n=1}^\infty
\mathcal{S}_n)$ are independent.  Note that for every $n>0$, $\mathcal{T}_n
\subset \sigma(\bigcup_{n=1}^\infty \mathcal{S}_n)$ hence the same is
true of their intersection $\mathcal{T}_\infty$.  We may conclude
that for any $A \in \mathcal{T}_\infty$ we have 
\begin{align*}
\probability{A} &= \probability{A \cap A} =
\probability{A}\probability{A}
\end{align*}
which shows that $\probability{A} = 0$ or $\probability{A} = 1$.
\end{proof}
Tail algebras arise naturally in various limiting processes involving
random variables.  In the case in which the random variables are
independent, the limits have various kinds of almost sure properties
that can be derived from Kolmogorov's $0-1$ Law.  Here are a few examples.
\begin{cor}Let $(S,d)$ be a complete metric space and let $\xi_n$ be a sequence of independent random elements in
  S.  Then either $\xi_n$ converges almost surely or diverges almost surely.
\end{cor}
\begin{proof}
Let $\mathcal{T}_n = \sigma\left( \cup_{k\geq n} \sigma(\xi_k)
\right)$ and let $\mathcal{T} = \cap_{n=1}^\infty \mathcal{T}_n $ be
the tail $\sigma$-algebra.  By Kolmogorov's $0-1$ Law it suffices to show that the event that
$\xi_n$ converges is $\mathcal{T}$-measurable. 
Since $S$ is complete, we know that $\xi_n$ converges if and only if
for every $\epsilon > 0$ there exists $N>0$ such that $d(\xi_m, \xi_n)
< \epsilon$.  With that in mind, for every $m>0$, $n>0$ and $\epsilon>0$ define
\begin{align*}
A_{n,m,\epsilon} = \lbrace d(\xi_m, \xi_n) < \epsilon \rbrace
\end{align*}
which is $\sigma \left(\sigma(\xi_m) \cup \sigma(\xi_n)
\right)$-measurable.

To prove convergence it suffices to demonstrate it for any sequence of
$\epsilon_k \to 0$.  So in particular if we choose $\epsilon_k =
\frac{1}{k}$ we see that the event that $\xi_n$ converges is
\begin{align*}
\bigcap_{k=1}^\infty \bigcup_{N=1}^\infty \bigcap_{m,n\geq N} A_{m,n,\frac{1}{k}}
\end{align*}
Note that each $\bigcap_{m,n\geq N} A_{m,n,\frac{1}{k}}$ is
$\mathcal{T}_N$-measurable and $A_{N+1} \subset A_N$ hence
$\bigcup_{N=1}^\infty \bigcap_{m,n\geq N} A_{m,n,\frac{1}{k}}$ is
$\mathcal{T}$-measurable.  Taking the countable union of
$\mathcal{T}$-measurable  sets we see the event that $\xi_n$ converges
is $\mathcal{T}$-measurable.  
\end{proof}

\begin{cor}\label{ConstantLimitOfIndependent}Let $\xi_n$ be a sequence of independent random variables.
  Then $\limsup_{n \to \infty} \xi_n$ and $\liminf_{n \to \infty} \xi_n$ are almost surely constant.
\end{cor}
\begin{proof}Because $\liminf_n \xi_n= -\limsup_n -\xi_n$ if suffices to
  show the result for $\limsup_n \xi_n$.   Let $\mathcal{T}$ be the tail $\sigma$-algebra of
  $\sigma(\xi_n)$ and let $\mathcal{T}_n = \sigma\left(\cup_{k\geq n}
  \sigma(\xi_k) \right)$.  By Kolmogorov's 0-1 Law, it suffices to show that $\limsup_{n \to \infty}
  \xi_n$ is $\mathcal{T}$-measurable.

By definition, $\limsup_{n \to \infty} \xi_n = \lim_{n \to \infty}
\sup_{k \geq n} \xi_k$.  The term $\sup_{k \geq n} \xi_k$ is
$\mathcal{T}_n$-measurable by \ref{LimitsOfMeasurable} and when taking the limit of the sequence
we can ignore any finite prefix of the sequence.  Therefore we can
express the limit as a limit of $\mathcal{T}_n$-measurable functions
for $n>0$ arbitrary.  This shows that $\limsup_{n \to \infty} \xi_n$
is $\mathcal{T}_n$-measurable for all $n>0$ hence $\mathcal{T}$-measurable.
\end{proof}

\begin{cor}\label{AlmostSureAverages}Let $\xi_n$ be a sequence of independent random variables.
  Then $\lim_{n \to \infty} \frac{1}{n}\sum_{k=1}^\infty \xi_k$ almost
  surely diverges or almost sure converges.  If it converges then the
  limit is  almost surely constant.
\end{cor}
\begin{proof}
Note that $\lim_{n \to \infty} \frac{1}{n}\sum_{k=1}^n \xi_k =
\lim_{n \to \infty} \frac{1}{n}\sum_{k=m}^n \xi_k$ for any $m >
0$.  Pick such an $m > 0$ and note that every finite partial sum
$\frac{1}{n}\sum_{k=m}^n \xi_k$ is $\mathcal{T}_m$-measurable hence so
is the limit $\lim_{n\to \infty} \frac{1}{n}\sum_{k=1}^n \xi_k$.
Since $m >0$ was arbitrary we know that $\lim_{n\to \infty}
\frac{1}{n}\sum_{k=1}^n \xi_k$ is $\mathcal{T}$-measurable.  
\end{proof}

The Borel Cantelli Theorem is a very useful technique in demonstrating
the almost sure convergence of sequences of random variables.  The
following simple version of the Strong Law of Large Numbers
illustrates the technique with a minimum of distractions.
\begin{lem}\label{SLLNL4}Let $\xi, \xi_1, \xi_2, \dots$ be independent identically
  distributed random variables with $\expectation{\xi^4} < \infty$,
  then $\lim_{n \to \infty} \frac{1}{n} \sum_{k=1}^n \xi_k =
  \expectation{\xi}$ a.s.
\end{lem}
\begin{proof}
First note that it suffices to show the result when $\expectation{\xi}
= 0$ since we can just compute
\begin{align*}
0 = \lim_{n \to \infty} \frac{1}{n} \sum_{k=1}^n (\xi_k -
\expectation{\xi}) = 
\lim_{n \to \infty} (\frac{1}{n} \sum_{k=1}^n \xi_k) -
\expectation{\xi}
\end{align*}
Furthermore by Corollary \ref{IncreasingMoments} the finite $4^{th}$ moment of $\xi$ implies finiteness of
the first four moments, hence $\expectation{\left (\xi -
    \expectation{\xi}\right )^4} <
\infty$.

Now assuming that $\xi_k$ have mean zero we fix $\epsilon > 0$ and apply Markov
bounding to see
\begin{align*}
\probability{\abs{\sum_{k=1}^n \xi_k} > n \epsilon} &=
\probability{\left ( \sum_{k=1}^n \xi_k \right ) ^4  > n^4 \epsilon^4} \\
&\leq \frac{\expectation{\left ( \sum_{k=1}^n \xi_k \right )^4}}{ n^4 \epsilon^4}
& & \text{by Markov's inequality}
\\
&= \frac{\sum_{k=1}^n\expectation{\xi_k^4}+6\sum_{k=1}^n
 \sum_{l=k+1}^n\expectation{\xi_k^2 \xi_l^2}}{ n^4 \epsilon^4} & &
\text{by independence and zero mean} \\
&= \frac{\sum_{k=1}^n\expectation{\xi_k^4}+6\sum_{k=1}^n
 \sum_{l=k+1}^n\sqrt{\expectation{\xi_k^4} \expectation{\xi_l^4}}}{ n^4 \epsilon^4} & &
\text{by Cauchy Schwartz} \\
&= \frac{\expectation{\xi^4}(n + 3(n^2 -n))}{ n^4 \epsilon^4} \leq \frac{3\expectation{\xi^4}}{ n^2 \epsilon^4} \\
\end{align*}
And therefore $\sum_{n=1}^\infty \probability{\abs{\sum_{k=1}^n \xi_k}
  > n \epsilon}  < \infty$.  Now we can apply Borel Cantelli to see
that $\probability{\frac{1}{n}\abs{\sum_{k=1}^n \xi_k
    } > \epsilon \text{ i.o.}} = 0$.

By the above argument, for every $m \in \naturals$ we get an event
$A_m$ with $\probability{A_m} = 0$ such that for every $\omega \notin
A_m$ there is $N_{\omega,m}$such that $\frac{1}{n}\abs{\sum_{k=1}^n
  \xi_k(\omega)} \leq \frac{1}{m}$ for $n > N_{\omega,m}$.  Let $A =
\cup{m=1}^\infty A_m$ and note that by countable subadditivity
$\probability{A} = 0$.  Furthermore, for every $\epsilon > 0$, $\omega
\in A$ we pick $m > \frac{1}{\epsilon}$ and then for $n > N_{\omega,
  m}$ we have $\frac{1}{n}\abs{\sum_{k=1}^n
  \xi_k(\omega)} \leq \frac{1}{m} < \epsilon$ for $n > N_{\omega,m}$
giving the result.
\end{proof}
The proof above demonstrates a general pattern in applications of
Borel Cantelli in which one applies it a countably infinite number of
times and still derive an almost sure result.  We'll prove more
refined versions of the Strong Law of Large Numbers later and those
will also use Borel Cantelli but with more complications.


It will prove to be important to be able to construct random variables
with prescribed distributions.  In particular, we will soon need to be
able to construct independent random variables with prescribed
distributions.  The standard way of constructing them is to use
product spaces, however we have only developed product spaces of
finitely many factors.  Rather than developing the full fledged theory
of infinitary products, we provide a mechanism which suffices for the
construction of countably many random variables with prescribed
distributions; in fact we show that it is possible to do so on the
probability space $([0,1], \mathcal{B}([0,1]), \lambda)$.  First
proceed by noticing that there is ready source of independence waiting
for us to harvest.  Given $x \in [0,1]$ we can take the unique binary
expansion $x = 0.\xi_1\xi_2 \cdots$ which has the property that
$\sum_{n=1}^\infty \xi_n = \infty$ (here we are resolving the ambiguity
between expansions that have a tail of $1$'s and those with a tail of
$0$'s).
\begin{lem}\label{BernoulliSequence}Let $\xi_n : [0,1] \to [0,1]$ be defined by taking the
  $n^{th}$ digit of the binary expansion of $x \in [0,1]$.  Then
  $\xi_n$ is a measurable function.  Let $\vartheta :  [0,1] \to
  [0,1]$, then $\vartheta$ has a uniform distribution if and only if
  $\xi_n \circ \vartheta$ comprise an independent sequence of
  Bernoulli random variables with probability $\frac{1}{2}$.
\end{lem}
\begin{proof}To see the measurability of $\xi_n$ we first define the
  \emph{floor function} to be $\lfloor x \rfloor = \sup \lbrace n \in
  \integers \mid n \leq x\rbrace$.  Then define
 \begin{align*}
\xi(x) &= \begin{cases}
0 & \text{if $x - \lfloor x \rfloor \in [0,\frac{1}{2})$} \\
1 & \text{if $x - \lfloor x \rfloor \in [\frac{1}{2},1)$} \\
\end{cases}
\end{align*}
It is clear that $\xi$ is a measurable function since $\xi^{-1}(0) =
\cup_n [n, n+\frac{1}{2})$ and  $\xi^{-1}(1) =
\cup_n [n+\frac{1}{2}, n+1)$.  Now define 
\begin{align*}
\xi_n(x) &= \xi(2^{n-1} x) & &\text{for $n \in \naturals$ and  $x \in \reals$}
\end{align*}
and notice that $\xi_n$ give the binary expansion of $x \in \reals$.
By measurability of $\xi$ we see that $\xi_n$ are also measurable.

Now suppose that $\vartheta$ is a $U(0,1)$ random variable on
$[0,1]$ and consider $\xi_n \circ \vartheta$.  For every $(k_1, \dots,
k_n) \in \lbrace 0,1\rbrace^n$, let $q = \sum_{j=1}^n \frac{k_j}{2^j}$ we clearly have 
\begin{align*}
\probability{\cap_{j \leq n} \lbrace \xi_j(\vartheta(x)) = k_j \rbrace}
  &= \probability{\vartheta(x) \in [q, q+\frac{1}{2^{n}})
    \rbrace} = \frac{1}{2^n}
\end{align*}
and summing over $(k_1, \dots, k_{n-1})$ we see 
\begin{align*}
\probability{\xi_n(\vartheta(x)) = k_n} &=\sum_{(k_1, \dots, k_{n-1})
  \in \lbrace 0,1\rbrace^{n-1}}\probability{\cap_{j \leq n} \lbrace \xi_j(\vartheta(x)) = k_j \rbrace}
  = \frac{1}{2}
\end{align*}
which shows that each $\xi_n \circ \vartheta$ is a Bernoulli random
variable with probability $\frac{1}{2}$.

In a similar vein, given $n_1, \dots, n_m$ and $k_{n_j} \in \lbrace 0,1 \rbrace$ , let $n = \sup(n_1, \dots,
n_m)$ for $j=1,\dots, m$ and  $A_n = \lbrace (l_1, \dots, l_n) \mid l_{n_j} = k_{n_j} \text{
  for } j=1, \dots , m\rbrace$ and we have 
\begin{align*}
\probability{ \cap_{j=1}^m \lbrace \xi_{n_j} (\vartheta(x)) =
  k_{n_j} \rbrace } &=\sum_{(k_1, \dots, k_{n})
  \in A_n}\probability{\cap_{j \leq n} \lbrace
  \xi_j(\vartheta(x)) = k_j \rbrace} \\
  &= 2^{n -m} \frac{1}{2^n} = \frac{1}{2^m}
\end{align*}
which shows that $\xi_{n_j} \circ \vartheta$ are independent.

Next, suppose that we know $\xi_n \circ \vartheta$ is an independent Bernoulli sequence with
probability $\frac{1}{2}$.  Let $\tilde{\vartheta}$ be a $U(0,1)$
random variable (e.g. $\tilde{\vartheta}(x) = x$) and then we know
from the first part of the Lemma
that $\xi_n \circ \tilde{\vartheta}$ is also a Bernoulli sequence with
probability $\frac{1}{2}$.

Because of the
independence of each the sequences and the fact that the elementwise
the two sequences have the same distribution we know that the
distribution of the sums is just the convolution of the distributions
of the terms in the sequence, hence $\sum \xi_n \circ \vartheta
\eqdist \sum \xi_n \circ \tilde{\vartheta}$.  Thus we have shown that
$\sum \xi_n \circ \vartheta$ is also $U(0,1)$.
\end{proof}

\begin{lem}\label{ReproductionOfUniform}There exist measurable functions $f_1, f_2, \dots$ on
  $[0,1]$ such
  that whenever $\vartheta$ is a $U(0,1)$ random variable, the
  sequence
  $f_n \circ \vartheta$ is a family of independent $U(0,1)$  random variables.
\end{lem}
\begin{proof}
Let $\xi_n \circ \vartheta$ denote the binary expansion of $\vartheta$
from Lemma \ref{BernoulliSequence}.  By the result of that Lemma, we
know that the $\xi_n \circ \vartheta$ are an i.i.d. sequence of
Bernoulli random variables with probability $\frac{1}{2}$.  Now choose
any bijection between $\naturals$ and $\naturals^2$ (e.g. the diagonal
mapping).  With this relabeling of of the constructed family we now
have a sequence $\xi_{n,m} \circ \vartheta$ of i.i.d. Bernoulli random
variables.  Define $f_n(x) = \sum_{m=1}^\infty
\frac{\xi_{n,m}(x)}{2^m}$ and apply Lemma \ref{BernoulliSequence} a
second time to see that each $f_n \circ \vartheta$ is a $U(0,1)$
random variable.  Furthermore, $f_n\circ \vartheta$ is $\sigma(\cup_m
\sigma(\xi_{n,m} \circ \vartheta))$-measurable so by Lemma \ref{IndependenceGrouping} we
see that the $f_n \circ \vartheta$ are independent.
\end{proof}

\begin{thm}\label{ExistenceCountableIndependentRandomVariables}For any probability measures $\mu_1, \mu_2, \dots$ on
  $(\reals, \mathcal{B}(\reals))$ there exist independent random
  variables $f_1, f_2, \dots$ on $([0,1], \mathcal{B}([0,1]),
  \lambda)$ such that $\mathcal{L}(f_n) = \mu_n$.
\end{thm}
\begin{proof}
Define $\vartheta(x) = x$ which is clearly a $U(0,1)$-random variable
on $[0,1]$ and use Lemma \ref{ReproductionOfUniform} to construct
$\vartheta_n$, a
sequence of independent $U(0,1)$ random variables.  Let $F_n$ be the
distribution function of the probability measure $\mu_n$ and let
$G_n(y) = \sup \lbrace x \in \reals \mid F(x) \geq y \rbrace$ be the generalized
inverse of $F_n$.  By the proof of Theorem \ref
{LebesgueStieltjesMeasure}, we know that $\mathcal{L}(G_n\circ
  \vartheta_n) = \mu_n$ and by Lemma \ref{IndependenceComposition} we
  know that $G_n\circ  \vartheta_n$ are still independent.
\end{proof}


\section{Convergence of Random Variables}
TODO: a.s. convergence, convergence in probability
and weak convergence (convergence in distribution), tightness of
distribution.
\begin{defn}Let $(S,d)$ be a $\sigma$-compact metric space with the Borel
  $\sigma$-algebra and let $\xi_n$ be a sequence of random elements in
  $S$.  Let $\xi$ be a random element in $S$.
\begin{itemize}
\item[(i)] $\xi_n$ \emph{converges almost surely} to $\xi$ if for almost
  every $\omega \in \Omega$, $\xi_n(\omega)$ converges to $\xi(\omega)$ in $S$.
We write $\xi_n \toas \xi$ to denote almost sure convergence.
\item[(ii)] $\xi_n$ \emph{converges in probability} to $\xi$ if for any
  $\epsilon>0$ we have 
\begin{equation*}
\lim_{n \to \infty} \probability{ \lbrace \omega : d(\xi_n(\omega),
      \xi(\omega)) >
      \epsilon \rbrace } = 0
\end{equation*}
We write $\xi_n \toprob \xi$ to denote convergence in probability.
\item[(iii)] $\xi_n$ \emph{converges in distribution} to $\xi$ if, for
  every bounded continuous function $f:S \to \reals$, one has 
\begin{equation*}
\lim_{n \to \infty} \expectation{f(\xi_n)} = \expectation{f(\xi)}.
\end{equation*} 
We write $\xi_n \todist \xi$ to denote convergence in distribution.
\item[(iv)] $\xi_n$ has a \emph{tight sequence of distributions} if, for
  every $\epsilon>0$, there exists a compact subset $K$ of $S$ such
  that $\probability{\xi_n \in K} \geq 1 - \epsilon$ for sufficiently
  large $n$.
\end{itemize}
\end{defn}

TODO: Note that convergence in distribution is really a property of
the distribution of the random variables and not the random variables
themselves.

 For the case of random variables there is another strong form of
convergence that is quite useful.
\begin{defn}If $\xi, \xi_1, \xi_2, \dots$ are random variables then $\xi_n$ \emph{converges in $L^p$} to $\xi$ if 
$\lim_{n \to \infty} \expectation{\abs{\xi_n - \xi}^p} = 0$.
We write $\xi_n \tolp{p} \xi$ to denote convergence in $L^p$.  We may
also call convergence in $L^p$ \emph{convergence in $p^{th}$ mean}.
\end{defn}

TODO: Motivation for concept of almost sure convergence via Law of
Large Numbers.  Think of modeling coin tossing using random
variables.  The $n^{th}$ coin flip is represented as a Bernoulli
random variable $\xi_n$ where $\xi_n(\omega) = 1$ means that the coin
lands with heads.  The \emph{empirical probability} of heads is in $n$
trials is $S_n = \frac{1}{n} \sum_{k=1}^n \xi_k$.  Now our intuition
is that $S_n$ converges to $1/2$ in some appropriate sense.  Now the
simple minded notion of pointwise convergence that we used in the
development of measure theory (e.g. in all of the limit theorems)
is too strong for this scenario.  Clearly, it is theoretically
possible for a person to toss a coin an infinite number of times and
get only heads.  It is possible by extremely improbable; so improbable
in fact that its probability is zero.

TODO: Motivation for concept of convergence in probability.
Motivation for convergence in mean is pretty clear.

There is also some useful technical intuition around how one might
prove that sequences converge almost surely.  The idea is implicit in
the definitions but is useful to take the time to call it out and make
it perfectly explicit; we will see it time and again.  If one looks at the contrapositive of almost
sure convergence, it means that there is probability zero that a
sequence of random elements does not converge.  The property of not
converging is that there exists an $\epsilon >0$ such that for all $N
> 0$, $d(\xi, \xi_n) \geq \epsilon$ for all $n > N$. Converting the
logic in set operations, let $A_{N, \epsilon}$ be the event that 
$d(\xi, \xi_n) \geq \epsilon$ for all $n > N$.  Convergence fails
precisely on the event $\cup_{\epsilon > 0} \cap_{N=1}^\infty A_{N,
  \epsilon}$, so almost sure convergence means that $\probability{\cup_{\epsilon > 0} \cap_{N=1}^\infty A_{N,
  \epsilon}} = 0$.  TODO: Note that one can restrict $\epsilon$ to a
countable subset of $\reals$ (e.g. $\rationals$ or $\frac{1}{n}$).  Note that the same reasoning applies when handling
almost sure Cauchy sequences as well.

Almost sure convergence is such a simple notion that it seems there
may be nothing worth explaining about it.  However the following
result ties in the definition of almost sure convergence with the
idea of events happening infinitely often that we encountered when
discussing indpedendence.  The connection proves to be quite powerful
and we'll soon see that it make the Borel-Cantelli Lemma a useful tool
for proving almost sure convergence.
\begin{lem}\label{ConvergenceAlmostSureByInfinitelyOften}Let $\xi, \xi_1, \xi_2, \dots$ be random elements in the
  metric space $(S,d)$, then $\xi_n \toas \xi$ if
and only if for every $\epsilon > 0$, $\probability{d(\xi_n, \xi) \geq
  \epsilon \text{ i.o.}}=0$ if and only if for every $\epsilon > 0$,
$\lim_{n \to \infty} \probability{\sup_{m \geq n} d(\xi_m, \xi) > \epsilon}
= 0$.  
\end{lem}
\begin{proof}
By definition if $\xi_n \toas \xi$ there is a set $A \subset \Omega$
such that $\probability{A} = 1$ and for all $\epsilon > 0$ and $\omega \in A$ there
exists $N_{\epsilon, \omega}\geq 0$ such that $d(\xi_n(\omega), \xi(\omega)) <
\epsilon$ when $n \geq
N_{\epsilon, \omega}$.  In particular for $\omega \in A$, $d(\xi_n, \xi) \geq
  \epsilon$ finitely often.  Therefore $\lbrace d(\xi_n, \xi) \geq
  \epsilon \text{ i.o.} \rbrace \subset A^c$ and $\probability{d(\xi_n, \xi) \geq
  \epsilon \text{ i.o.}} \leq \probability{A} = 0$.

In the opposite direction, let $A_\epsilon = \lbrace d(\xi_n, \xi) \geq
  \epsilon \text{ i.o.}\rbrace$ and by assumption
  $\probability{A_\epsilon} = 0$.  The event that $\xi_n$ does not
  converge to $\xi$ is precisely $A = \cup_{\epsilon > 0} A_\epsilon$
  and we might think we are done.  Unfortunately $\cup_{\epsilon > 0}
  A_\epsilon$ is an uncountable union and we can't conclude that
  $\probability{A} = 0$.  
  We resolve this by noting that in fact $A = \cup_n A_{\frac{1}{n}}$
  which is a countable union of sets of measure zero; hence has
  measure zero.

TODO: Fix inconsistency in use of $\geq$ and $>$.

To see the second equivalence, just unfold the definition of events
happening infinitely often and use continuity of measure
\begin{align*}
\probability{d(\xi_n, \xi) >
  \epsilon \text{ i.o.}} &= \probability{\cap_{n=1}^\infty
  \cup_{m=n}^\infty \lbrace d(\xi_m, \xi) >
  \epsilon\rbrace } \\
&= \lim_{n \to \infty} \probability{\cup_{m=n}^\infty \lbrace d(\xi_m, \xi) >
  \epsilon \rbrace } \\
&= \lim_{n \to \infty} \probability{\sup_{m \geq n} d(\xi_m, \xi) >
  \epsilon} \\
\end{align*}
\end{proof}

\begin{lem}\label{ConvergenceAlmostSureImpliesInProbability}Let $\xi, \xi_1, \xi_2, \dots$ be random elements in the
  metric space $(S,d)$.  If $\xi_n \toas \xi$ then $\xi_n \toprob \xi$.
\end{lem}
\begin{proof}
By Lemma \ref{ConvergenceAlmostSureByInfinitelyOften} and continuity of measure, if $\xi_n \toas \xi$ then we know that for each
$\epsilon > 0$, 
\begin{align*} 
0=\probability{d(\xi_n, \xi) \geq
  \epsilon \text{ i.o.}}=\lim_{n \to \infty} \probability{\cup_{k\geq
    n} {d(\xi_k, \xi) \geq
  \epsilon}}
\end{align*}
Now clearly we have $\probability{d(\xi_n, \xi) \geq
  \epsilon} \leq \probability{\cup_{k \geq n} d(\xi_k, \xi) \geq
  \epsilon}$ so convergence in probability follows.

Here is an alternative approach that currently has a hole in the
argument.  Is it worth patching the hole?  Suppose there exists $\epsilon , \delta >
0$ for which there is a subsequence $n_j \to \infty$ and
$\probability{d(\xi_{n_j}, \xi) > \epsilon} \geq \delta > 0$.  We
claim that $\probability{\cap_j \lbrace d(\xi_{n_j}, \xi) > \epsilon
  \rbrace } > 0$ (is this really true?).  Note $\cap_j \lbrace d(\xi_{n_j}, \xi) > \epsilon \rbrace  \subset \{ \omega \mid \xi_{n_j}(\omega) \text{ does not converge
  to }  \xi(\omega) \}$ hence $\xi_n$ does not converge on a set of
positive measure.
\end{proof}

\begin{examp}\label{ConvergeProbabilityNotAlmostSure}[Sequence converging in probability but not almost
  surely]Consider the $(\reals, \mathcal{B}(\reals))$ with Lebesgue
  measure.  For a sequence of intervals $I_n \subset \reals$ observe
  that $\characteristic{I_n} \toprob 0$ if and only if $\abs{I_n} \to
  0$.  For every $n > 0$ consider the events $A_{n,j} =
  [\frac{j-1}{n}, \frac{j}{n}]$ for $j=1, \dots, n$.  Now consider the
  sequence of random variables obtained by taking the lexicographic
  order of pairs $(n,j)$ for $n>0$ and $j=1,\dots, n$ and the
  indicator functions $\characteristic{A_{n,j}}$; call the
  resulting sequence $f_m$.  
Note that $f_m
\toprob 0$ by the
above discussion.  On the the other hand, the sequence does not
converge pointwise anywhere on $[0,1]$ because for every $x \in [0,1]$, we can
see $\limsup_{m \to \infty} f_m(x) = 1$ but $\liminf_{m \to \infty} f_m(x) = 0$.
\end{examp}

\begin{lem}\label{ConvergenceInMeanImpliesInProbability}Let $\xi,
  \xi_1, \xi_2, \dots$ be random variables, if $\xi_n \tolp{p} \xi$, then $\xi_n \toprob \xi$.
\end{lem}
\begin{proof}
This is a simple application of Markov's Inequality (Lemma \ref{MarkovInequality})
\begin{align*}
\probability{\abs{\xi_n - \xi} > \epsilon} &= \probability{\abs{\xi_n
    - \xi}^p > \epsilon^p} \leq \frac{\expectation{\abs{\xi_n
    - \xi}^p}}{\epsilon^p}
\end{align*}
but the right hand side converges to $0$ by assumption.
\end{proof}

\begin{examp}[Sequence converging in probability but in mean]To see
  that a sequence of random elements can converge in probability but
  not in mean we can modify Example \ref
  {ConvergeProbabilityNotAlmostSure}.  Using the notation from that
  example, define the random variables $n\characteristic{A_{n,j}}$ and
  order them lexicographically into the sequence $f_m$.  Note that
  point behind rescaling is that we have arrange for
  $\expectation{n\characteristic{A_{n,j}}} = 1$.  The argument
  that the $f_m \toprob 0$ follows essentially unchanged;  convergence in probability is insensitive the rescaling of
  the random variables.  On the other hand, it is clear that
  $\expectation{f_m} = 1$ for all $m>0$ and therefore $f_m$ do not
    converge in mean to $0$.
\end{examp}

There are few useful characterization of convergence in probability
that are important tools to have.  The first provides a
characterization of convergence in probability as a convergence of
expectations.  Because of the previous example, we know that
convergence in probability does not control the behavior of random
elements on arbitrarily small sets hence it alone is not capable of
controlling the values of expectations.  Adding in such control as an
explicit extra condition we can tie the concepts together.
\begin{lem}\label{ConvergenceInProbabilityAsConvergenceInExpectation}Let $\xi, \xi_1, \xi_2, \dots$ be random elements in the
  metric space $(S,d)$.  $\xi_n \toprob \xi$ if and only if
  $\lim_{n \to \infty} \expectation{d(\xi_n,\xi) \wedge 1} = 0$.
\end{lem}
\begin{proof}Suppose that $\xi_n \toprob \xi$.  We pick
  $\epsilon > 0$ and $N > 0$ such that
  $\probability{d(\xi_n,\xi) > \epsilon} < \epsilon$ for $n > N$.
Now write
\begin{align*}
d(\xi_n,\xi) \wedge 1 &= d(\xi_n,\xi)
  \wedge 1 \cdot \characteristic{d(\xi_n,\xi) > \epsilon} + d(\xi_n,\xi)
  \wedge 1 \cdot \characteristic{d(\xi_n,\xi) \leq \epsilon} \\
&\leq \characteristic{d(\xi_n,\xi) > \epsilon} + \epsilon
\end{align*}
Taking expectations we see
\begin{align*}
\expectation{d(\xi_n,\xi) \wedge 1} &\leq \probability{d(\xi_n,\xi) >
  \epsilon} + \epsilon \leq 2\epsilon & & \text{for $n > N$.}
\end{align*}

Suppose that $\lim_{n \to \infty} \expectation{d(\xi_n,\xi) \wedge
  1} = 0$.  First note that in proving convergence in probability, it
suffices to consider $\epsilon < 1$ since for any $\epsilon <
\epsilon^\prime$ we have $\probability{d(\xi_n, \xi) >
  \epsilon^\prime)} \leq \probability{d(\xi_n, \xi) >
  \epsilon)}$.  So pick $0 < \epsilon < 1$ and use Markov's Inequality
(Lemma \ref{MarkovInequality}) to see
\begin{align*}
\lim_{n \to \infty} \probability{d(\xi_n, \xi) >
  \epsilon)} &= \lim_{n \to \infty}\probability{d(\xi_n, \xi) \wedge 1 >
  \epsilon)} \leq \lim_{n \to \infty} \frac{\expectation{d(\xi_n,\xi) \wedge
  1}}{\epsilon} = 0
\end{align*}
\end{proof}

As an example of how this Lemma is can be used, note that it provides a quick alternative proof to Lemma
\ref{ConvergenceAlmostSureImpliesInProbability}:  If $\xi_n \toas \xi$
then $d(\xi_n, \xi) \wedge 1 \toas 0$ and Dominated Convergence
implies $\expectation{d(\xi_n, \xi) \wedge 1} \to 0$.

The relationship between almost sure convergence and convergence in
probability can be made even tighter than Lemma \ref{ConvergenceAlmostSureImpliesInProbability}.
\begin{lem}\label{ConvergenceInProbabilityAlmostSureSubsequence}Suppose $(S,d)$ is a metric space and let $\xi,
  \xi_1, \xi_2, \dots$ be random elements in $S$.  Then $\xi_n \toprob
  \xi$ if and only for every subsequence $N^\prime \subset \naturals$ there is a
  further subsequence $N^{\prime\prime} \subset N^\prime$ such that
  $\lim_{n \in N^{\prime\prime}} \xi_n = \xi$ a.s.
\end{lem}
\begin{proof}
Let $\xi_n \toprob \xi$.  By Lemma
\ref{ConvergenceInProbabilityAsConvergenceInExpectation}, we know that
$\lim_{n \to \infty} \expectation{d(\xi_n, \xi) \wedge 1}  = 0$.
Thus we can pick $n_k > 0$ such that $\expectation{d(\xi_{n_k}, \xi)
  \wedge 1} < \frac{1}{2^k}$.  Therefore  
\begin{align*}\sum_{k=1}^\infty \expectation{d(\xi_{n_k}, \xi)
  \wedge 1} = \expectation{\sum_{k=1}^\infty   d(\xi_{n_k}, \xi)
  \wedge 1} < \infty
\end{align*} where we have used Tonelli's Theorem
\ref{TonelliIntegralSum}.  Finiteness of the second integral implies $\sum_{k=1}^\infty   d(\xi_{n_k}, \xi)
  \wedge 1 < \infty$ almost surely and convergence of the sum implies
  that the terms $d(\xi_{n_k}, \xi)  \wedge 1 \toas 0$ which in turn
  implies  $d(\xi_{n_k}, \xi)  \toas 0$

Here is an alternative proof of the first implication using Borel-Cantelli.  Pick a sequence $n_1,
n_2, \dots$ such that $\probability{d(\xi_{n_k}, \xi) > \frac{1}{k}} <
\frac{1}{2^k}$.  Then the sets $A_k=\{\omega \mid d(\xi_{n_k}(\omega),
\xi(\omega)) > \frac{1}{k} \}$ satisfy $\sum_{k=1}^\infty \mu A_k <
\infty$ and we can apply Borel-Cantelli to conclude that $\mu (A_k
i.o.) = 0$.  Thus $\omega \notin A_k
i.o. $ we pick $N_1 > 0$ such that $\omega \notin A_k$ for $k > N_1$
and given $\epsilon >0$, we pick $N_2 >
\frac{1}{\epsilon}$.  Then for $k > \max(N_1,N_2)$ we see that $d(\xi_{n_k}(\omega),
\xi(\omega)) \leq \frac{1}{k} < \epsilon$ and we have shown that $\xi_{n_k} \toas \xi$.

To prove the converse, suppose that $\xi_n$ does not converge in
probability to $\xi$.  The definitions tell us that we can find
$\epsilon > 0$, $\delta > 0$ and a subsequence $N^\prime$ such that
$\probability{d(\xi_{n_k}, \xi) > \epsilon} > \delta$ for all $n \in
N^\prime$.  We claim that there is no subsequence of $N^{\prime
  \prime}$ for which $\xi_n \toas \xi$ along $N^{\prime \prime}$.  The
claim is verified by using the fact (shown in the proof of Lemma
\ref{ConvergenceAlmostSureImpliesInProbability}) that convergence
almost surely means that $\probability{\cup_{k\geq n} \lbrace d(\xi_k,
  \xi) > \epsilon \rbrace} \to 0$ for all $\epsilon > 0$.  For our
chosen $\epsilon$, along any
subsequence $N^{\prime \prime} \subset N^{\prime}$ every tail event
$\cup_{k \in N^{\prime \prime}, k\geq n} \lbrace d(\xi_k,
  \xi) > \epsilon \rbrace$ contains only events with probability greater
  than $\delta$ hence cannot converge to $0$.
\end{proof}

The previous lemma has a nice side effect which is a proof that the
property of convergence in probability does not actually depend on the
choice of metric.
\begin{cor}\label{ConvergenceInProbabilityIndependentOfMetric}Let $\xi, \xi_1, \xi_2, \dots$ be a random elements in a
  metrizable space $S$.  The property $\xi_n \toprob \xi$ does not
  depend on the choice of metric $d$.
\end{cor}

The previous lemma also gives us a very simply proof the extremely
useful Continuous Mapping Theorem for convergence in probability.
\begin{lem}\label{ContinuousMappingProbability}Let $\xi, \xi_1, \xi_2, \dots$ be a random elements in a
  metric space $(S,d)$ such that $\xi_n \toprob \xi$.  Let
  $(T,d^\prime)$ be a metric space and let $f : S \to
  T$ be a continuous function, then $f(\xi_n) \toprob f(\xi)$.
\end{lem}
\begin{proof}
Pick a subsequence $N^\prime \subset \naturals$ and note that by Lemma
\ref{ConvergenceInProbabilityAlmostSureSubsequence} we know there
exists a subsequence $N^{\prime \prime} \subset N^\prime$ such that
$\xi_n \toas \xi$ along $N^{\prime \prime}$.  By the continuity of $f$,
we know that $f(\xi_n) \toas f(\xi)$ along $N^{\prime \prime}$ hence
another application of Lemma
\ref{ConvergenceInProbabilityAlmostSureSubsequence}  shows that
$f(\xi_n) \toprob f(\xi)$.
\end{proof}
The full power of the Continuous Mapping Theorem for convergence in
probability is only fully appreciated in conjuction with the following
useful characterization of convergence in probability in product
spaces.  It is important to reinforce that the following Lemma fails
in the case of convergence in distribution and one of the best uses of
convergence in probability is a way of getting around that latter
limitation.
\begin{lem}\label{ConvergenceInProbabilityInProductSpaces}Let $\xi, \xi_1, \xi_2, \dots$ and $\eta, \eta_1, \eta_2,
  \dots$ be random sequences in $(S,d)$ and $(T,d^\prime)$
  respectively.  Then $(\xi_n, \eta_n) \toprob (\xi, \eta)$ if an only
  if $\xi_n \toprob \xi$ and $\eta_n \toprob \eta$.
\end{lem}
\begin{proof}
Note that by Corollary
\ref{ConvergenceInProbabilityIndependentOfMetric} we may work with any
metric on $S \times T$.  We choose the metric $d^{\prime\prime}((x,w), (y,z)) =
d(x,y) + d^\prime(w,z)$.
First we assume that $(\xi_n, \eta_n) \toprob (\xi, \eta)$.  Then we
know that for every $\epsilon > 0$, we have
\begin{align*}
\lim_{n \to \infty} \probability{d^{\prime\prime}((\xi_n,\eta_n),
  (\xi, \eta)) > \epsilon} &= 0
\end{align*}
By our choice of metric $d^{\prime\prime}$ we can see that $d(\xi_n,
\xi) \leq d^{\prime\prime}((\xi_n,\eta_n),
  (\xi, \eta))$ and $d^\prime(\eta_n,\eta) \leq d^{\prime\prime}((\xi_n,\eta_n),
  (\xi, \eta))$ and therefore we can conclude that $\xi_n \toprob \xi$
  and $\eta_n \toprob \eta$.  

On the other hand if we assume that  $\xi_n \toprob \xi$ and $\eta_n
\toprob \eta$ the for every $\epsilon > 0$ we have the union bound
\begin{align*}
\probability{d^{\prime\prime}((\xi_n,\eta_n),
  (\xi, \eta)) > \epsilon} &\leq \probability{d(\xi_n,\xi) > \frac{\epsilon}{2}}
  + \probability{d^\prime(\eta_n,\eta) > \frac{\epsilon}{2}}
\end{align*}
which shows the converse.
\end{proof}
\begin{cor}\label{ConvergenceInProbabilityAndAlgebraicOperations}Let $\xi, \xi_1, \xi_2, \dots$ and $\eta, \eta_1, \eta_2,
  \dots$ be sequences of random variables such that $\xi_n \toprob
  \xi$ and $\eta_n \toprob \eta$, then 
\begin{itemize}
\item[(i)] $\xi_n + \eta_n \toprob \xi + \eta$
\item[(ii)] $\xi_n  \eta_n \toprob \xi \eta$
\item[(iii)] $\xi_n / \eta_n \toprob \xi /\eta$ if $\eta \neq 0$ a.e.
\end{itemize}
\end{cor}
\begin{proof}
By Lemma \ref{ConvergenceInProbabilityInProductSpaces} we know that
$(\xi_n, \eta_n) \toprob (\xi, \eta)$ in $\reals^2$.  By continuity of algebraic
operations and the Continuous Mapping Theorem the result holds.
\end{proof}


\subsection{The Weak Law Of Large Numbers}

\begin{thm}[Weak Law of Large Numbers]\label{WLLN} Let $\xi_1, \xi_2, \dots$ be independent and identically
  distributed random variables with
\begin{align*}
\mu = \expectation{\xi_i} < \infty
\end{align*}
Then 
\begin{align*}
\frac{1}{n} \sum_{k=1}^n \xi_k \toprob \mu
\end{align*}
\end{thm}
\begin{proof}
If is worth first proving the result with the additional assumption of
finite variance, so assume $\sigma^2 = \variance{\xi_j} < \infty$.
The first thing to note is that it suffices to assume that $\mu =0$.
For we can replace $\xi_j$ by $\xi_j - \mu$.  Now define $\hat{S}_n = \frac{1}{n} \sum_{k=1}^n \xi_k$ and note that by
linearity of expectation, $\expectation{\hat{S}_n} = 0$ and by
independence, 
\begin{align*}
\variance{\hat{S}_n} &= \frac{1}{n^2}  \sum_{k=1}^n
\expectation{\xi_k^2} = \frac{\sigma^2}{n}
\end{align*}
Pick $\epsilon > 0$ and using Markov
  Inequality (Lemma
  \ref{MarkovInequality})
\begin{align*}
\probability{\abs{\hat{S}_n} > \epsilon} = \probability{\hat{S}_n^2 >
  \epsilon^2} \leq \frac{\variance{\hat{S}_n}}{\epsilon^2} = \frac{\sigma^2}{n \epsilon^2}
\end{align*}
so $\lim_{n \to \infty} \probability{\abs{\hat{S}_n} > \epsilon}= 0$ and
thus $\hat{S}_n \toprob 0$.

Now to extend the result to eliminate the finite variance assumption
we use a version of a \emph{truncation argument}.  One leverages the
fact that by Lemma \ref{IndependenceExpectations}, independence of random variables is
preserved under arbitrary measurable transformations.  In particular,
for every $N > 0$, define $f_N(x) = x \cdot
\characteristic{\abs{x}\leq N}$
which is easily seen to be measurable and define 
\begin{align*}
\xi_{i,\leq N} &=
f_{\leq N} \circ \xi_i \\
\xi_{i, > N} &=  \xi_i - \xi_{i,\leq N}
\end{align*}

We first establish some simple facts about the behavior of the
truncation sequences $\xi_{i,\leq N}$ and $\xi_{i,> N}$.  Since
$\xi_i$ are integrable we have the bound 
\begin{align*}
\variance{\xi_{i,\leq N}} &= \expectation{\xi^2_{i,\leq N}} -
\expectation{\xi_{i,\leq N}}^2 \leq \expectation{\xi^2_{i,\leq N}}
\leq N \expectation{\abs{\xi_i}} < \infty
\end{align*}
which shows that $\xi_{i,\leq N}$ has finite variance.  Let $\mu_N = \expectation{\xi_{i,\leq N}}$.

Next note that integrability of $\xi_i$ implies that $\abs{\xi_i} <
\infty$ a.s. hence $\lim_{N \to \infty} \xi_{i,>N} = \lim_{N \to
  \infty} \abs{\xi_{i,>N}}  = 0$ a.s.  Since
$\abs{\xi_{i,>N}} < \abs {\xi_i}$, we can apply Dominated Convergence
Theorem and linearity of expectation to see that 
\begin{align*}
\lim_{N \to \infty} \expectation{\xi_{i,>N}} &= \lim_{N \to \infty}
\expectation{\abs{\xi_{i,>N}}}   = 0 \\
\lim_{N \to \infty} \expectation{\xi_{i,\leq N}} &=
\expectation{\xi_i} - \lim_{N \to
  \infty} \expectation{\xi_{i,> N}} = \expectation{\xi_i}
\end{align*}

Now we stitch these observations together to provide the proof of the
Weak Law of Large Numbers.  Suppose we are given $\epsilon >0$ and $\delta > 0$.  Pick $N$ large enough so that 
\begin{align*}
\abs{\expectation{\xi_{i,\leq N}} - \expectation{\xi_i}} &<
\frac{\epsilon}{3} \\
\expectation{\abs{\xi_{i,> N}}} &< \frac{\epsilon \delta}{3}
\end{align*}
It is important to note these two bounds depend only on the underlying
distribution of $\xi_i$ and therefore by the identically distributed assumption
on the $\xi_i$ if we pick $N$ so the above properties are satisified
for a single $i$, in fact the properties are satisified uniformly for
all $i > 0$.  

Using the triangle inequality and a union bound (i.e. the general fact
that $\lbrace \abs{a + b} \geq \epsilon \rbrace \subset \lbrace \abs{a} \geq \frac{\epsilon}{2} \rbrace  \cup \lbrace \abs{b} \geq \frac{\epsilon}{2} \rbrace$) we have
\begin{align*}
\probability{\abs{\frac{\sum_{i=1}^n \xi_i}{n} - \mu} \geq \epsilon}
&= \probability{\abs{\frac{\sum_{i=1}^n \xi_{i,\leq N}}{n} - \mu_N +
    \mu_N - \mu + \frac{\sum_{i=1}^n \xi_{i,> N}}{n}} \geq \epsilon} \\
&\leq \probability{\abs{\frac{\sum_{i=1}^n \xi_{i,\leq N}}{n} - \mu_N} \geq \frac{\epsilon}{3}}  \\
&+ \probability{\abs{\mu_N - \mu} \geq \frac{\epsilon}{3} } + \probability{\abs{\frac{\sum_{i=1}^n \xi_{i,> N}}{n}} \geq \frac{\epsilon}{3}} \\
\end{align*}
Consider each of the three terms in turn.  The first term we apply
Chebyshev bounding
\begin{align*}
\probability{\abs{\frac{\sum_{i=1}^n \xi_{i,\leq N}}{n} - \mu_N} \geq
  \frac{\epsilon}{3}} &\leq \frac{9 \variance{\frac{\sum_{i=1}^n
      \xi_{i,\leq N}}{n}}}{\epsilon^2} \leq \frac{9 N
  \expectation{\abs{\xi_1}}}{n\epsilon^2} < \delta
\end{align*}
provided we choose $n > \frac{9 N
  \expectation{\abs{\xi_1}}}{\delta\epsilon^2}$.  
The second term is $0$ since we have assumed $N$ large enough so that
$\abs{\mu_N - \mu} < \frac{\epsilon}{3}$.  The third term we use a
Markov bound
\begin{align*}
\probability{\abs{\frac{\sum_{i=1}^n \xi_{i,> N}}{n}} \geq
  \frac{\epsilon}{3}}  &\leq \frac{3
  \expectation{\abs{\frac{\sum_{i=1}^n \xi_{i,> N}}{n}}}}{\epsilon}
\leq \frac{3 \expectation{\abs{\xi_{i,> N}}}}{\epsilon} < \delta
\end{align*}
\end{proof}

It is worth examining the proof above to see that we didn't use the
full strength of the identical distribution property.  Really all we
used was the fact that were able to provide bounds on the expectation
of the tails of the sequences \emph{uniformly}.  As an exercise, it is
worth noting that the above proof goes through almost unchanged
provided we merely assume that $\xi_n$ are independent and uniformly integrable.

\begin{examp}\label{WLLNCounterExampleBoundedFirstMoment}The following is an example of a how the Weak Law of
  Large Numbers can fail despite having a sequence of independent
  random variables with bounded first moment.

Let $\eta_n$ be a sequence of independent Bernoulli random variables
with the rate of $\eta_n$ equal to $\frac{1}{2^n}$.  Now define $\xi_n
= 2^n \eta_n$ and $S_n = \frac{1}{n}\sum_{k=1}^n \xi_k$.  It is
helpful to think in Computer Science terms and consider $\sum_{k=1}^n
\xi_k$ to be a random $n$-bit positive integer in which bit $k$ has
probability $\frac{1}{2^k}$ of being set.  Note that
$\expectation{\xi_n} = \expectation{\abs{\xi_n}}  = 1$ and therefore
$\expectation{S_n} = 1$.  On the other hand we proceed to show that
$S_n$ does not converge in probability to $1$.  We do this by
constructing a subsequence $S_{n_k}$ such that 
$\lim_{k \to \infty} \probability{ S_{n_k} < \frac{1}{2}} = 1$ (note
the choice of the constant $\frac{1}{2}$ is somewhat arbitrary; any
positive constant would do).

Consider the subsequence $S_{2^k}$ and the complementary event 
\begin{align*}
\lbrace S_{2^k} \geq  \frac{1}{2} \rbrace &= \lbrace
\sum_{n=1}^{2^k}\xi_n \geq  2^{k-1} \rbrace = \bigcup_{m=k-1}^{2^k}
\lbrace \xi_m \neq 0 \rbrace
\end{align*}
Taking expectations, we get 
\begin{align*}
\probability{ S_{2^k} \geq  \frac{1}{2} } &\leq \sum_{m=k-1}^{2^k}
\probability{ \xi_m \neq 0 } \\
&= \sum_{m=k-1}^{2^k} \frac{1}{2^m} = \frac{1}{2^{k-1}} \cdot 2 \cdot
(1 - 2^{2^k - k + 1}) < \frac{1}{2^{k-2}}
\end{align*}
which is enough to show by taking complements that $\lim_{k \to \infty} \probability{
  S_{2^k} <  \frac{1}{2} } = 1$.

TODO: Discussion about what is going on here.   Essentially, the
averages here have a distribution which is peaking around 0 but has
enough of a possibility of rare events happening (with exponentially
large impact) to move the mean of
the averages up to 1.  Thus the distribution is concentrating around 0
which is NOT the mean!

TODO: Question: does this sequence converge in distribution?  I'd
guess is converges to the Dirac measure at 0.
\end{examp}

TODO: Other weak law ``counterexamples'' such as Cauchy
distributions.  Varadhan mentions that one can tweak a Cauchy
distribution so that it has no mean but the sequence of averages
converges in probability.


\subsection{The Strong Law Of Large Numbers}

This is the most common approach to proving of the Strong Law of Large
Numbers.  The proof requires the development of some tools for proving
the almost sure converges of infinite sums of independent random
variables.

TODO: Observe how this next result is related to second moment bounds (Chebyshev
applied to sums).

\begin{lem}[Kolmogorov's Maximal Inequality]\label{KolmogorovMaximalInequality}Let
  $\xi_1, \xi_2, \dots$ be independent random variables with
  $\expectation{\xi_n^2} < \infty$ for all $n>0$.  The for every
  $\epsilon > 0$, we have
\begin{align*}
\probability{\sup_n \abs{\sum_{k=1}^n \xi_k - \expectation{\xi_k}}
  \geq \epsilon} < \frac{1}{\epsilon^2} \sum_{k=1}^\infty \variance{\xi_k}
\end{align*}
\end{lem}
\begin{proof}
It is clear we may assume that $\expectation{\xi_n} = 0$ for all $n >
0$.  

Before we start in on the result to be proven, we need an small
observation.  To clean up notation a bit we define $S_n = \sum_{k=1}^n    \xi_k$.
Pick $N > n > 0$ and observe $0 \leq (S_N-S_n)^2 =  S_N^2 - 2 S_N S_n + S_n^2
= S_N^2 - S_n^2 - 2(S_N - S_n)S_n$ and therefore 
$S_N^2 - S_n^2 \geq 2(S_N - S_n)S_n$.  Now using the fact that by Lemma
\ref{IndependenceGrouping} we know $S_N - S_n$ is independent of
$S_n$.  Therefore for any $A_n \in \sigma(S_n)$ we have
\begin{align*}
\expectation{S_N^2 - S_n^2 ; A_n} \geq 2\expectation{(S_N - S_n)S_n;
  A_n} = 2\expectation{S_N - S_n}\expectation{S_n;  A_n} = 0
\end{align*}
which gives us 
\begin{align*}
\expectation{S_N^2; A_n} \geq \expectation{S_n^2;  A_n} 
\end{align*}
by linearity of expectation.  

Now we start in on the inequality to be proven.  Note that by continuity of measure, we know that 
\begin{align*}
\probability{\sup_n \abs{S_n}  \geq \epsilon}  =
\lim_{N \to \infty} \probability{\sup_{n \leq N} \abs{S_n}  \geq \epsilon}
\end{align*}
so it suffices to show for every $N > 0$ 
\begin{align*}
\probability{\sup_{n \leq N} \abs{S_n} \geq \epsilon} \leq
\frac{1}{\epsilon^2} \sum_{k=1}^N \expectation{\xi_k^2} =
\frac{1}{\epsilon^2} \expectation{S_N^2}
\end{align*}

Consider $\probability{\sup_{n \leq N} \abs{S_n} \geq
  \epsilon}$.  Define the event $A_n = \lbrace \abs{S_k} < \epsilon
\text{ for $1\leq k < n$ and } \abs{S_n} \geq
  \epsilon$ and note that $A_n$ is $\sigma(\xi_n)$-measurable and we have the disjoint union 
\begin{align*}
\lbrace \sup_{n \leq N} \abs{S_n} \geq
  \epsilon \rbrace &= A_1 \cup \cdots \cup A_N
\end{align*}
and therefore 
\begin{align*}
\probability{ \sup_{n \leq N} \abs{S_n} \geq
  \epsilon } &= \sum_{k=1}^N \probability{A_k} & & \text{by additivity
  of measure}\\
&\leq
\frac{1}{\epsilon^2}\sum_{k=1}^N \expectation{S_k^2 ; A_k} & & \text{$\abs{S_k}
\geq \epsilon$ on the event $A_k$}\\
&\leq \frac{1}{\epsilon^2}\sum_{k=1}^N \expectation{S_N^2 ; A_k} \\
&= \frac{1}{\epsilon^2}\expectation{S_N^2 ; { \sup_{n \leq N} \abs{S_n} \geq
  \epsilon }} & & \text{by additivity
  of measure}\\
&\leq \frac{1}{\epsilon^2}\expectation{S_N^2 } & & \text{positivity of $S_N^2$}
\end{align*}
and the result is proved.  
\end{proof}

The previous lemma gives us a criterion for almost sure convergence of
sums of square integrable random variables with finite variance.
\begin{lem}[Kolmogorov One-Series Criterion]\label{VarianceCriterionSeries}Let $\xi_1, \xi_2,\dots$ be independent square integrable
  random variables.  If $\sum_{n=1}^\infty \variance{\xi_n} < \infty$
  then $\sum_{n=1}^\infty \left (\xi_n - \expectation{\xi_n}\right )$
  converges a.s.
\end{lem}
\begin{proof}
We may clearly assume that $\expectation{\xi_n} = 0$ for all $n > 0$.
Define $S_n = \sum_{k=1}^n \xi_k$.

Before giving a proper proof, it might be worth looking a simple
heuristic argument to give some intuition why this result should be
true.  For every $N > 0$, 
\begin{align*}
\probability{\abs{\sum_{n=1}^\infty \xi_n} > N} &\leq
\frac{\variance{\sum_{n=1}^\infty \xi_n}}{N^2} & & \text{by
  Chebeshev's Inequality} \\
&=\frac{\sum_{n=1}^\infty \expectation{\xi_n^2}}{N^2} & &\text{by
  independence and zero mean}
\end{align*}
and therefore we know that 
\begin{align*}
\sum_{N=1}^\infty
\probability{\abs{\sum_{n=1}^\infty \xi_n} > N} \leq \sum_{n=1}^\infty
\expectation{\xi_n^2} \sum_{N=1}^\infty \frac{1}{N^2} < \infty
\end{align*}
so Borel Cantelli implies $\probability{\abs{\sum_{n=1}^\infty \xi_n}
  > N \text{ i.o.}} = 0$ which implies almost
sure convergence. The problem with this argument is that we have
manipulated the series as if we knew it converged which is what we are
trying to prove (is this really the problem, or is the problem that we
are dealing with conditional convergence so showing the almost sure
boundedness of the sum doesn't imply convergence; in that case this
argument is completely irrelevant).  Kolmogorov's Maximal Inequality gives us a way to
make a more rigorous argument.

Pick $\epsilon > 0$ and for every $N > 0$ define $A_{N,\epsilon}= \lbrace
\sup_{n > N} \abs{S_n - S_N} \geq \epsilon \rbrace$.  Applying Lemma \ref
{KolmogorovMaximalInequality} to the sequence $\xi_n$ for $n=N+1, N+2,
\dots$, we know that
\begin{align*}
\probability{A_{N,\epsilon}} = \probability{\sup_{n > N} \abs{S_n - S_N} \geq
  \epsilon} \leq \frac{1}{\epsilon^2}\sum_{n=N+1}^\infty \expectation{\xi_n^2}
\end{align*}
and by the convergence of $\sum_{n=1}^\infty \expectation{\xi_n^2}$ we
know that 
\begin{align*}
\lim_{N \to \infty} \probability{A_{N,\epsilon}} \leq \lim_{N\to \infty}
\frac{1}{\epsilon^2}\sum_{n=N+1}^\infty \expectation{\xi_n^2} = 0
\end{align*}
which by subadditivity of measure tells us that $\probability{\cap_{N=1}^\infty
  A_{N,\epsilon}} = 0$.  Now, for every $n>0$ define $B_n
=\cap_{N=1}^\infty A_{N,\frac{1}{n}}$, define $B = \cup_n B_n$  and note that by
countable additivity of measure, $\probability{B} = 0$.  

We show that $S_n$ converges for all $\omega \notin B$.  Pick $\omega \notin
B$ .  Assume we are
given $\epsilon > 0$ and pick $n>0$ such that $\frac{1}{n} < \epsilon$.  We know $\omega \notin B_n$ and
therefore for some $N>0$, $\omega \notin A_{N,\frac{1}{n}}$ which implies that  $\abs{S_k -
  S_N} < \frac{1}{n} < \epsilon$ for all $k > N$.  This shows that
$S_n(\omega)$ is a Cauchy sequence for every $\omega \notin B$ and by
completeness of $\reals$ this shows that $S_n$ is almost surely convergent.

Here is a more concise variant of the same basic argument.  Pick $\epsilon > 0$ and applying Lemma \ref
{KolmogorovMaximalInequality} to the sequence $\xi_n$ for $n=N+1, N+2,
\dots$, we know that
\begin{align*}
\probability{\sup_{n > N} \abs{S_n - S_N} \geq
  \epsilon} \leq \frac{1}{\epsilon^2}\sum_{n=N+1}^\infty \expectation{\xi_n^2}
\end{align*}
and by the convergence of $\sum_{n=1}^\infty \expectation{\xi_n^2}$ we
know that 
\begin{align*}
\lim_{N \to \infty} \probability{\sup_{n > N} \abs{S_n - S_N} \geq
  \epsilon} \leq \lim_{N\to \infty}
\frac{1}{\epsilon^2}\sum_{n=N+1}^\infty \expectation{\xi_n^2} = 0
\end{align*}
which shows that $\sup_{n > N} \abs{S_n - S_N} \toprob 0$.  Now by
Lemma \ref{ConvergenceInProbabilityAlmostSureSubsequence} we know that a subsequence of $\sup_{n > N} \abs{S_n -
  S_N}$ converges to $0$ a.s.  However, as $\sup_{n > N} \abs{S_n -
  S_N}$ is nonincreasing in $N$ (TODO: I don't see this; in fact I
don't think it is true without a positivity assumption), the almost sure converge of the subsequence
implies the almost sure converge of the entire sequence.  The
convergence $\sup_{n > N} \abs{S_n -  S_N} \toas 0$ is just the
statement that $S_n$ is almost sure Cauchy which by completeness of
$\reals$ says that $S_n$ converges almost surely.
\end{proof}

Having just proven a convergence criterion for a sequence of partial
sums of independent random variables, we should ask ourselves how this
can help us establish criteria for the sequence of averages that the
Strong Law of Large Numbers refers to.  The key result here has
nothing to do with probability.

\begin{lem}\label{SummationByParts}Let $a_1, a_2, \dots$ and $b_1,
  b_2, \dots$ be sequences of real numbers.  Define $\Delta a_n =
  a_{n+1} - a_n$ and $\Delta b_n =
  b_{n+1} - b_n$, then for every $n > m > 0$,
\begin{align*}
\sum_{k=m}^n a_k \Delta b_k = a_{n+1} b_{n+1} - a_m b_m - \sum_{k=m}^n
b_{k+1} \Delta a_k
\end{align*}
\end{lem}
\begin{proof}Note that we have the \emph{product rule}
\begin{align*}
\Delta (a\cdot b)_k &= a_{k+1}b_{k+1} - a_k b_k \\
&= a_{k+1}b_{k+1} -
a_{k}b_{k+1} + a_{k}b_{k+1} - a_k b_k \\
&= a_k \Delta b_k +
b_{k+1} \Delta a_k
\end{align*}
and therefore 
\begin{align*}
a_{n+1} b_{n+1} - a_m b_m &= \sum_{k=m}^n \Delta (a\cdot b)_k  \\
&= \sum_{k=m}^n a_k \Delta b_k  + \sum_{k=m}^n b_{k+1} \Delta a_k
\end{align*}
\end{proof}

\begin{lem}\label{SeriesAndAverages}Let $0=b_0 \leq b_1 \leq b_2 \leq \dots$ be a non-decreasing
  sequence of positive real numers such that $\lim_{n \to \infty} b_n
  = \infty$ and define $\beta_n = b_n - b_{n-1}$ for $n > 0$.  If
  $s_1, s_2, \dots$ is a sequence of real numbers with $\lim_{n \to
    \infty} s_n = s$ then
\begin{align*}
\lim_{n \to \infty} \frac{1}{b_n} \sum_{k=1}^n \beta_k s_k = s
\end{align*}
In particular, if $x_1, x_2, \dots$ are real numbers, then if $\sum_{n=1}^\infty
\frac{x_n}{b_n} < \infty$ then $\lim_{n \to \infty}
\frac{1}{b_n}\sum_{k=1}^n x_k < \infty$.
\end{lem}
\begin{proof}
To see the first part of the Lemma, note that for any constant $s \in
\reals$,  $\frac{1}{b_n}\sum_{k=1}^n \beta_k s = s$ and therefore we
may assume that $s=0$.

Pick an $\epsilon >0$ and then select $N_1 >0$ such that $\abs{s_k} <
\frac{\epsilon}{2}$ for all $k \geq N_1$.  Define $M = \sup_{n\geq 1}
\abs{s_n}$ and then because $\lim_{n \to \infty} b_n = \infty$ we can pick $N_2 >
0$ such that $\frac{b_{N_1} M }{b_n} < \frac{\epsilon}{2}$ for all $n > N_2$.
Now for every $n > \max(N_1, N_2)$,
\begin{align*}
\abs{\frac{1}{b_n} \sum_{k=1}^n \beta_k s_k} &\leq \abs{\frac{1}{b_n}
  \sum_{k=1}^{N_1} \beta_k s_k} + \abs{\frac{1}{b_n}
  \sum_{k=N_1+1}^n \beta_k s_k} \\
&\leq \frac{b_{N_1} M }{b_n} + \frac{(b_n - b_{N_1})\epsilon }{2b_n} \leq \epsilon
\end{align*}
and we are done.

To see the second part of the Lemma, define $s_0 = 0$ and $s_n =
\sum_{k=1}^n \frac{x_k}{b_k}$, now apply summation by parts to see
\begin{align*}
\frac{1}{b_n} \sum_{k=1}^n \Delta b_{k-1} s_{k-1} &= \frac{1}{b_n}
\left ( b_n s_n - b_0 s_0 - \sum_{k=1}^n b_k \Delta s_{k-1} \right )
\\
&= s_n - \frac{1}{b_n} \sum_{k=1}^n x_k
\end{align*}
so we can take limits and apply the first part of this Lemma to find
\begin{align*}
\lim_{n \to \infty} \frac{1}{b_n} \sum_{k=1}^n x_k
&= \lim_{n \to \infty} s_n - \lim_{n \to \infty} \frac{1}{b_n}
\sum_{k=1}^n \Delta b_{k-1} s_{k-1} \\
&= s - s = 0
\end{align*}
\end{proof}

\begin{cor}\label{KolmogorovSLLNL2}Assume that $0 \leq b_1 \leq b_2 \leq \cdots$ and $\lim_{n
    \to \infty} b_n = \infty$ and let $\xi_1, \xi_2, \dots$ be
    independent square integrable random variables.  If
    $\sum_{n=1}^\infty \frac{\variance{\xi_n}}{b_n^2} < \infty$ then 
\begin{align*}
\frac{1}{b_n} \sum_{k=1}^n \left( \xi_k - \expectation{\xi_k} \right )
\toas 0
\end{align*}
\end{cor}

\begin{thm}[Strong Law of Large Numbers]\label{SLLN} Let $\xi, \xi_1, \xi_2, \dots$ be independent and identically
  distributed random variables. Then if $\xi_1$ is integrable
\begin{align*}
\lim_{n \to \infty} \frac{1}{n} \sum_{k=1}^n \xi_k &= \expectation{\xi} \quad \text{\emph{a.s.}}
\end{align*}
Conversely if $\frac{1}{n} \sum_{k=1}^n \xi_k$ converges on a set of
positive measure, then $\xi_1$ is integrable.
\end{thm}
\begin{proof}
First, one makes the standard reduction to the case in which
$\expectation{\xi_n}= 0$ for all $n>0$.  

Next we apply a truncation argument by defining 
\begin{align*}
\eta_n = \xi_{n, \leq  n} = \xi_n \cdot \characteristic{[0,n]}(\abs{\xi_n})
\end{align*}
Note
\begin{align*}
\sum_{n=1}^\infty \probability{\eta_n \neq \xi_n} &= \sum_{n=1}^\infty
\probability{\abs{\xi_n} > n} \\
&\leq \sum_{n=1}^\infty \int_{n-1}^n
\probability{\abs{\xi_n} \geq \lambda } d\lambda & & \text{since
  $\probability{\abs{\xi_n} \geq \lambda}$ is decreasing} \\
&= \int_0^\infty \probability{\abs{\xi} \geq \lambda } d\lambda & &
\text{by i.i.d. }\\
&= \expectation{\abs{\xi}} < \infty & & \text{by Lemma \ref{TailsAndExpectations}}
\end{align*}
Now we apply Borel Cantelli to conclude that $\probability{\eta_n \neq
  \xi_n \text{ i.o.}} = 0$.  Stated conversely,
$\probability{\text{there exists $N>0$ such that $\xi_n \leq n$ for
    all $n > N$}} = 1$.

Next define $\overline{\eta}_n = \frac{1}{n}\sum_{k=1}^n \eta_k$ and
$\overline{\xi}_n = \frac{1}{n}\sum_{k=1}^n \xi_k$.  We claim that
$\lim_{n \to \infty} \overline{\eta}_n = 0$ a.s. if and only if
$\lim_{n \to \infty} \overline{\xi}_n = 0$ a.s.

For almost all $\omega \in \Omega$ we can pick
$N_\omega > 0$ such that $\xi_n(\omega) = \eta_n(\omega)$ for all $n >
N_\omega$.  Let $C_\omega = \sum_{k=1}^{N_\omega} \left (
  \eta_k(\omega) - \xi_k(\omega) \right )$ so that for $n > N_\omega$,
we have $\lim_{n \to \infty} \overline{\eta}_n(\omega) = \lim_{n \to \infty}
\overline{\xi}_n(\omega) + \frac{C_\omega}{n}$ and therefore $\lim_{n
  \to \infty} \overline{\eta}_n(\omega)  = \lim_{n \to \infty}
\overline{\xi}_n(\omega) $.  

Therefore it suffices to show $\lim_{n \to \infty} \overline{\eta}_n =
0$ a.s.  Although we no longer have $\expectation{\eta_n} = 0$ because
we have truncated $\xi_n$, the \emph{average} of the means of $\eta_n$
is 0.  This follows from noting that $\lim_{n \to \infty} \xi_{ \leq n} =
\xi$ and $\abs{\xi_{\leq n}} \leq \abs{\xi}$ so 
\begin{align*}
0 &=\expectation{\xi} \\
&= \lim_{n \to \infty}\expectation{\xi_{\leq
    n}} & & \text{by Dominated Convergence}\\
&= \lim_{n \to \infty}\expectation{\xi_{n, \leq n}} & & \text{by i.i.d.} \\
&= \lim_{n \to \infty}\expectation{\eta_n}
\end{align*}
and therefore by application of Lemma \ref{SeriesAndAverages}
\begin{align*}
\frac{1}{n} \sum_{k=1}^n \expectation{\eta_n} &= \lim_{n \to \infty}\expectation{\eta_n}=0\\
\end{align*}
Therefore if we can show that $\sum_{n=1}^\infty
\frac{\variance{\eta_n}}{n^2} <\infty$, then by Corollary
\ref{KolmogorovSLLNL2} we can conclude
\begin{align*}
\lim_{n \to \infty} \frac{1}{n} \sum_{k=1}^n \eta_k = \lim_{n \to
  \infty} \frac{1}{n} \sum_{k=1}^n \expectation{\eta_k} = 0 \text { a.s.}
\end{align*}
and we'll be done.

To show the desired bound we'll need the elementary fact that $C =
\sup_{n>0} n \sum_{k=n}^\infty \frac{1}{k^2} < \infty$.  This can be
seen by viewing the sum as lower Riemann sum for an integral bounding
\begin{align*}
n \sum_{k=n}^\infty \frac{1}{k^2} &\leq n \int_{n-1}^\infty \frac{dx}{x^2}
= \frac{n}{n-1}  \leq 2
\end{align*}

Now we can finish the proof
\begin{align*}
\sum_{n=1}^\infty \frac{\variance{\eta_n}}{n^2} &\leq
\sum_{n=1}^\infty\frac{\expectation{\eta_n^2}}{n^2}  \\
&= \sum_{n=1}^\infty\frac{\expectation{\xi_n^2;\abs{\xi_n} \leq n}}{n^2} \\
&= \sum_{n=1}^\infty \sum_{k=1}^n \frac{\expectation{\xi^2;k-1 \leq \abs{\xi} \leq k}}{n^2} \\
&= \sum_{k=1}^\infty \expectation{\xi^2;k-1 \leq \abs{\xi}
    \leq k}\sum_{n=k}^\infty \frac{1}{n^2} \\
&\leq  \sum_{k=1}^\infty \frac{C}{k}\expectation{\xi^2;k-1 \leq
  \abs{\xi}\leq k} \\
&\leq  C\sum_{k=1}^\infty \frac{k}{k}\expectation{\abs{\xi};k-1 \leq
  \abs{\xi}\leq k} = C \expectation{\abs{\xi}} < \infty
\end{align*}

It remains to show the converse result; namely that if
$\overline{\xi}_n$ converges on a set of positive measure then $\xi$
is integrable.  First, note by Corollary \ref{AlmostSureAverages}, we
know that $\overline{\xi}_n$ converges almost surely.

\begin{align*}
\lim_{n \to \infty} \frac{\xi_n}{n} &= \lim_{n \to \infty} \left (
  \overline{\xi}_n - \frac{n-1}{n}   \overline{\xi}_{n-1} \right ) \\
&= \lim_{n \to \infty} \overline{\xi}_n  - 1 \cdot \lim_{n \to \infty}
\overline{\xi}_n = 0 \text{ a.s.}
\end{align*}
and therefore if we define $A_n = \lbrace \abs{\xi_n} \geq n \rbrace$
then we know that $\probability{A_n \text{ i.o.}} = 0$ (in particular
for each $\omega$ for which $\lim_{n \to \infty}
\frac{\xi_n(\omega)}{n} = 0$ and any $\epsilon > 0$, we can find $N > 0$ such that
$\abs{\xi_n(\omega)} < \epsilon n$ for all $n > N$; just choose
$\epsilon < 1$).  But we also know that $\xi_n$ are independent and
therefore by Lemma \ref{IndependenceComposition} the $A_n$ are
independent so Borel Cantelli implies $\sum_{n=1}^\infty
\probability{A_n} < \infty$.  But now we can apply a tail bound
\begin{align*}
\expectation{\abs{\xi}} &= \int_0^\infty \probability{\abs{\xi} \geq
  \lambda} \, d\lambda & & \text{by Lemma \ref{TailsAndExpectations}}
\\
&\leq \sum_{n=0}^\infty \probability{\abs{\xi} \geq n} & &
\text{bounding by an upper
  Riemann sum} \\
&= 1 + \sum_{n=1}^\infty \probability{A_n} < \infty & &\text { by i.i.d.}
\end{align*}
\end{proof}
\begin{proof}The following proof uses a different truncation argument
  (one closer to the WLLN argument we presented) and is
  taken from Tao.

TODO:  Understand that proof better and write it down completely.

So to apply Borel Cantelli we need so find a sequence $N_j$ such that 
\begin{align*}
\sum_{j=1}^\infty n_j \probability{\xi > N_j} &< \infty \\
\sum_{j=1}^\infty \frac{1}{n_j} \expectation{\xi_{\leq N_j}} &< \infty 
\end{align*}
We show that both sums are finite if we choose $N_j = n_j$.  In both
cases this follows by establishing pointwise bounds in terms of
$\xi$.  For the first sum we use Tonelli's Theorem to exchange sums
and expectations
\begin{align*}
\sum_{j=1}^\infty n_j \probability{\xi > n_j} &= \sum_{j=1}^\infty n_j
\expectation{\characteristic{\xi > n_j}}= 
\expectation{\sum_{j=1}^\infty n_j\characteristic{\xi > n_j}} \\
&=\expectation{\sum_{n_j < \xi}  n_j} 
\end{align*}
TODO: Fill this in.  Essentially the idea is that we have an
approximately geometric series so the above is $O(\xi)$.

For the second sum, 
\begin{align*}
\sum_{j=1}^\infty \frac{1}{n_j} \expectation{\xi_{\leq n_j}} &\leq
\frac{1}{n_1} \expectation{\xi} \sum_{j=1}^\infty c^{-j} =
\frac{c  \expectation{\xi}}{n_1(c-1)} < \infty
\end{align*}
\end{proof}

\begin{thm}[Strong Law of Large Numbers (Finite Variance
  Case)]\label{SLLNL2} Let $\xi_1, \xi_2, \dots$ be independent and identically
  distributed random variables.  Let
\begin{align*}
\mu = \expectation{\xi_i} \text { and } \sigma^2 = \variance{\xi_j}^2
< \infty
\end{align*}
Then 
\begin{align*}
\lim_{n \to \infty} \frac{1}{n} \sum_{k=1}^n \xi_k &= \mu \quad \text{\emph{a.s.} and in $L^2$}
\end{align*}
\end{thm}
\begin{proof}
First note that by replacing $\xi_n$ with $\xi_n - \mu$ it suffices to
prove the Theorem with $\mu = 0$.

Next it is convenient to define the terms $S_n =  \sum_{k=1}^n \xi_k$
and $\eta_n = \frac{S_n}{n}$.
and observe that by linearity $\expectation{S_n} = \expectation{\eta_n} = 0$ and by
independence 
\begin{align*}
\variance{\eta_n} &= \frac{1}{n^2} \sum_{j=1}^n \sum_{k=1}^n \expectation{\xi_j\xi_k} \\
&= \frac{1}{n^2} \sum_{k=1}^n \expectation{\xi_k^2} = \frac{\sigma^2}{n}
\end{align*}
By taking the limit we see that $\lim_{n\to \infty} \variance{\eta_n} =
0$ which implies that $\eta_n \to 0$ in $L^2$.  

To see almost sure convergence we first pass to a subsequence.
Consider the subsequence $\eta_{n^2}$ and note by the above
variance calculation and Corollary \ref{TonelliIntegralSum} that 
\begin{align*}
\expectation{\sum_{n=1}^\infty \eta_{n^2}^2} &= \sum_{n=1}^\infty
\expectation{\eta_{n^2}^2} = \sum_{n=1}^\infty \frac{\sigma^2}{n^2} < \infty
\end{align*}
Finiteness of the first expectation implies that $\sum_{n = 1}^\infty
\eta_{n^2}^2  < \infty$ almost surely which in turn implies that $\lim_{n \to \infty}
\eta_{n^2}^2 = 0$ and $\lim_{n \to \infty}
\eta_{n^2} = 0$  almost surely .  It remains to prove almost sure
convergence for the entire sequence.  

Pick an arbitrary $n > 0$ and define $p(n) = \floor{\sqrt{n}}$ so that
$p(n)$ is the integer satisfying $(p(n))^2 \leq n < (p(n) + 1)^2$.
Then we have
\begin{align*}
\eta_n - \frac{p(n)^2}{n} \eta_{p(n)^2} &= \frac{1}{n}\sum_{k=p(n)^2 + 1}^n \xi_k
\end{align*}
and calculating variances as before,
\begin{align*}
\variance{\eta_n - \frac{p(n)^2}{n} \eta_{p(n)^2}} &=
\expectation{\left ( \eta_n - \frac{p(n)^2}{n} \eta_{p(n)^2} \right
  )^2} \\
&=
\frac{1}{n^2}\sum_{k=p(n)^2 + 1}^n \expectation{\xi_k^2} \\
&= \frac{\sigma^2 (n - p(n)^2)}{n^2} \\
&< \frac{\sigma^2 (2 p(n) + 1)}{n^2} \leq \frac{3\sigma^2}{n^\frac{3}{2}}
\end{align*}
This bound tells us that 
\begin{align*}
 \expectation{ \sum_{n=1}^\infty \left ( \eta_n - \frac{p(n)^2}{n} \eta_{p(n)^2} \right
  )^2} = \sum_{n=1}^\infty \expectation{\left ( \eta_n - \frac{p(n)^2}{n} \eta_{p(n)^2} \right
  )^2}  < \infty
\end{align*}
which as before tells us that 
\begin{align*}
\sum_{n=1}^\infty \left ( \eta_n - \frac{p(n)^2}{n} \eta_{p(n)^2} \right
  )^2 < \infty 
\end{align*}
almost surely and 
\begin{align*}
\lim_{n\to \infty} \left ( \eta_n - \frac{p(n)^2}{n} \eta_{p(n)^2}
\right ) = 0
\end{align*}
almost surely.

Since we have already proven $\eta_{p(n)^2} \toas 0$ and we can see by
definition that $0 < \frac{p(n)}{n} \leq 1$ we conclude that $\eta_n
\toas 0$.
\end{proof}

\subsubsection{Empirical Distributions and the Glivenko-Cantelli Theorem}

Here is a simple application of the Strong Law of Large Numbers that
has important applications in statistics.  Consider the process of
making a sequence of independent observations for purpose of inferring
a statement about an underlying distribution of a random variable.  A
basic statistical methodology is to use the distribution of ones
sample as an approximation to the unknown distribution.  We aim to
give a demonstration of why this methodology is sound.  First we make
precise what we mean by the distribution of the sample.

\begin{defn}Given independent random variables $\xi_1, \xi_2, \dots$,
  for each $n > 0$  and $x \in \reals$, we define the \emph{empirical distribution
    function} to be
\begin{align*}
\hat{F}_n(x, \omega) &= \frac{1}{n} \sum_{k=1}^n \characteristic{\xi_k \leq x}(\omega)
\end{align*}
\end{defn}
Note that the empirical distribution function depends on both $x$ and
$\omega \in \Omega$ but it is customary to omit mention of the
argument $\omega$ and simply write $\hat{F}_n(x)$.  In general we will follow this custom but on
occasion where we feel it is important enough for clarity we'll
include it as we did in the definition.  In the statistical context we've alluded to each $\xi_k$ represents
the value of the $k^{th}$ observation.  The empirical distribution of
$n$ samples is the distribution function of the \emph{empirical
  measure} obtained by placing an equally weighted point mass at the value of each observation.

\begin{lem}\label{PointwiseConvergenceOfEmpiricalDistribution}Let $\xi_1, \xi_2, \dots$ be i.i.d. random variables with
  distribution function $F(x)$ and empirical distribution functions
  $\hat{F}_1(x), \hat{F}_2(x), \dots$.  Then for each $x \in \reals$,
\begin{align*}
\lim_{n \to \infty} \hat{F}_n(x) &= F(x) \text{ a.s.}
\end{align*}
and in addition
\begin{align*}
\lim_{n \to \infty} \lim_{y \to x^-}\hat{F}_n(y) &= \lim_{y \to x^-} F(y) \text{ a.s.}
\end{align*}
\end{lem}
\begin{proof}
This statement is a simple application of the Strong Law of Large
Numbers.  First note that for every $x \in \reals$, by Lemma
\ref{IndependenceComposition}, the functions 
$\characteristic{\xi_n \leq x}$ are independent.  Because the $\xi_n$
are identically distributed the same follows for
$\characteristic{\xi_n \leq x}$.  Lastly, the functions
$\characteristic{\xi_n \leq x}$ are bounded and therefore integrable
so we can apply the Strong Law of Large Numbers to conclude that 
\begin{align*}
\lim_{n \to \infty} \hat{F}_n(x) &= \lim_{n \to \infty} \frac{1}{n} \sum_{k=1}^n \characteristic{\xi_k
  \leq x} = \expectation{\characteristic{\xi_1 \leq x}} = F(x) \text{ a.s.}
\end{align*}

To see the almost sure pointwise convergence of the left limits, first
note that for every $x \in \reals$, we have 
\begin{align*}
\lim_{n\to \infty} \characteristic{(-\infty, x-\frac{1}{n}]}(y) &= \begin{cases}
1 & \text{if $y < x$} \\
0 & \text{if $y \geq x$}
\end{cases} = \characteristic{(-\infty, x)}(y)
\end{align*}
Therefore, 
\begin{align*}
F(x-) &= \lim_{n \to \infty} F(x - \frac{1}{n}) & &\text{by the existence of left limits in $F(x)$}\\
&= 
\expectation{\characteristic{\xi \leq x-\frac{1}{n}}} \\
&= \expectation{\lim_{n \to \infty}\characteristic{\xi \leq
    x-\frac{1}{n}}} & & \text{by Dominated Convergence Theorem} \\
&= \expectation{\characteristic{\xi < x}}
\end{align*}

By the same argument, 
\begin{align*}
\hat{F}_m(x-) &= \lim_{n \to \infty} \hat{F}_m (x - \frac{1}{n}) \\
&= \lim_{n \to \infty} \frac{1}{m} \sum_{i=1}^m \characteristic{\xi_i
  \leq x - \frac{1}{n}}\\
&=\sum_{i=1}^m \characteristic{\xi_i  < x}
\end{align*}
As in the pointwise argument above, the family $\characteristic{\xi_i
  < x}$ is an i.i.d. family of integrable random variables so using
the above computations and the Strong Law of Large Numbers we see that
\begin{align*}
\lim_{n \to \infty} \hat{F}_n(x-) &= \lim_{n \to \infty} \sum_{i=1}^n
\characteristic{\xi_i  < x} = \expectation{\characteristic{\xi  < x} }
= F(x-) \text{ a.s.}
\end{align*}
\end{proof}

In fact, with a little more work leveraging properties of distribution
functions, we can prove that the empirical
distribution function converges uniformly.
\begin{thm}[Glivenko-Cantelli Theorem]\label{GlivenkoCantelli}Let $\xi_1, \xi_2, \dots$ be i.i.d. random variables with
  distribution function $F(x)$ and empirical distribution functions
  $\hat{F}_1(x), \hat{F}_2(x), \dots$.  Then,
\begin{align*}
\lim_{n \to \infty} \sup_x \abs{\hat{F}_n(x) - F(x)} &= 0 \text{ a.s.}
\end{align*}
\end{thm}
\begin{proof} 
TODO: Give an intuitive idea of the proof (the notation is messy and a
bit opaque).  Essentially we use the properties of distribution
functions (cadlag property and the compactness of the range) to
establish that if two distrubtion functions are close at a carefully
selected finite number of points then they are uniformly close.

By leveraging the boundedness (compactness) of the range of the
distribution function, we can get some nice uniform bounds on the
growth of that distribution function.  Compare the following
construction with Lemma \ref{LebesgueStieltjesMeasure}.
Let 
\begin{align*}
G(y) = \inf \lbrace x \in \reals \mid F(x) \geq y \rbrace
\end{align*}
be the generalized left continuous inverse of $F(x)$.  For each positive integer $m >
0$, consider the partition $x_{k,m} = G(\frac{k}{m})$ for $k=1, \dots,
m-1$.  We observe the following facts: by the definition of $G(y)$, for $x < x_{k,m}$, we have $F(x) < \frac{k}{m}$
and by right continuity of $F(x)$ and the definition of $G(y)$,
$F(G(y)) \geq y$, so in particular $F(x_{k,m}) \geq \frac{k}{m}$.  These two facts provide the following statements

\begin{align*}
F(x_{k+1,m}-)  - F(x_{k,m}) &\leq \frac{1}{m} & &\text{for $1 \leq k <
  m-1$} \\
F(x_{1,m}-) &\leq \frac{1}{m} \\
F(x_{m-1,m}) &\geq 1 - \frac{1}{m} \\
\end{align*}

Now, for each $m > 0$, $n > 0$ and $\omega \in \Omega$, define 
\begin{align*}
D_{n,m}(\omega) = \max(\max_{k} \abs{\hat{F}_n(x_{m,k}, \omega) -
  F(x_{k,m})}, \max_k \abs{\hat{F}_n(x_{m,k}-, \omega) - F(x_{k,m}-)}
)
\end{align*}
and we proceed to use this quantity to bound the distance between
$\hat{F}_n(x, \omega)$ and $F(x)$.  

First, observe the bound for $x < x_{k,m}$ for $1 \leq k \leq m-1$, 
\begin{align*}
\hat{F}_n(x, \omega) &\leq \hat{F}_n(x_{k,m}-, \omega) \\
&\leq F(x_{k,m}-) + D_{n,m}(\omega) & &\text{by definition of
  $D_{n,m}(\omega)$} \\
&\leq F(x) + \frac{1}{m} + D_{n,m}(\omega)
\end{align*}
and for $x \geq x_{k,m}$ for $1 \leq k \leq m-1$
\begin{align*}
\hat{F}_n(x, \omega) &\geq \hat{F}_n(x_{k,m}, \omega) \\
&\geq F(x_{k,m}) - D_{n,m}(\omega) \\
&\geq F(x) - \frac{1}{m}  - D_{n,m}(\omega) 
\end{align*}
When we put these together for $x \in [x_{k,m}, x_{k+1,m})$ for $1\leq
k < m-1$ and we have 
\begin{align*}
\sup_{x_{1,m} \leq x < x_{m-1,m}} \abs{\hat{F}_n(x, \omega) - F(x)} < \frac{1}{m} + D_{n,m}(\omega)
\end{align*}

It remains to complete the picture of what happens when $x < x_{1,m}$ and $x \geq
x_{m-1,m}$.

For $-\infty < x < x_{1,m}$, we have 
\begin{align*}
\hat{F}_n(x, \omega) & \geq 0 \\
& \geq F(x) - \frac{1}{m} \\
&  \geq F(x) - \frac{1}{m} - D_{n,m}(\omega)
\end{align*}
and lastly we have for $x \geq x_{m-1,m}$, 
\begin{align*}
\hat{F}_n(x, \omega) &\leq 1 \\
&\leq F(x) + \frac{1}{m} \\
&\leq F(x) + \frac{1}{m} + D_{n,m}(\omega)
\end{align*}
which allows us to extend for all $x \in \reals$,
\begin{align*}
\sup_{x} \abs{\hat{F}_n(x, \omega) - F(x)} < \frac{1}{m} + D_{n,m}(\omega)
\end{align*}

Now for each $m$, $\lim_{n \to \infty} D_{n,m} = 0$ a.s. by Lemma
\ref{PointwiseConvergenceOfEmpiricalDistribution}
and by taking a countable union of sets of probability zero, we have
for all $m > 0$, $\lim_{n \to \infty} D_{n,m} = 0$ a.s.  Therefore by
taking the limit as $m \to \infty$ and $n \to \infty$, we have result.
\end{proof}

We now take a short digression into statistics to show how the
Glivenko-Cantelli Theorem can be used.  The approach taken in
demonstrating the result below has far reaching generalizations;
don't let the epsilons and deltas distract you from appreciating the
conceptual framework.
\begin{defn}Let $P$ be a Borel probability measure on $\reals$ with
  distribution function $F(x) =
  \expectation{\characteristic{(-\infty,x]}}$.  We define the
    \emph{median} of $P$ to be $\median(P) = \inf_x \lbrace F(x) \geq
    \frac{1}{2} \rbrace$.  If $\xi$ is a random variable then we will
    often write $\median(\xi)$ for the median of the distribution of $\xi$.
\end{defn}

\begin{lem}Let $\xi_1, \xi_2, \dots$ be i.i.d. random variables and
  distribution function $F(x)$.  Suppose that $F(x) > \frac{1}{2}$ for
  all $x > \median{\xi}$.   The sample median $\lim_{n \to \infty}
  \median(P_n) = \median(\xi)$ a.s.; one says that the sample median is a
  \emph{strongly consistent} estimator of $\median(\xi)$.
\end{lem}
\begin{proof}
The key to the proof is viewing the median as a functional on the
space of distribution functions.  The Glivenko-Cantelli Theorem tells
us that empirical distributions functions converge uniformly so what
we need to prove convergence of the sample medians is a continuity
property of the median functional.  We develop the required continuity
property in bare handed way without talking about metric spaces or
topologies.

Suppose we have two Borel probability measures $P$ and $Q$ with
distribution functions $F_P(x)$ and $F_Q(x)$ with $F_P(x) >
\frac{1}{2}$ for $x > \median(P)$.  Given $\epsilon > 0$, pick 
$\delta > 0$ such that 
\begin{align*}
F_P(\median(P) - \epsilon) &< \median(P) - \delta \\
F_P(\median(P) + \epsilon) &> \median(P) + \delta \\
\end{align*}
We claim that if $Q$ satisfies $\sup_x \abs{F_P(x) - F_Q(x)} \leq
\delta$ then $\abs{\median(P) - \median(Q)} \leq \epsilon$.

To see this first note that 
\begin{align*}
F_P(\median(Q)) &\geq F_Q(\median(Q)) - \delta \geq \frac{1}{2} - \epsilon
\end{align*}
which implies that $\median(Q) \geq \median(P) - \epsilon$ by choice
of $\delta$ and the increasing nature of $F_P(x)$.
Secondly note that for any $x < \median(Q)$ we have
\begin{align*}
F_P(x) \leq F_Q(x) + \delta < \frac{1}{2} + \epsilon
\end{align*}
which implies $x < \median(P) + \epsilon$ and therefore by
arbitraryness of $x$, we have $\median(Q) \leq \median(P) + \epsilon$
and we are done with the claim.

Now as per our plan we couple the continuity just proven with
Glivenko-Cantelli to derive the result.
\end{proof}

Note that the value $\sup_x \abs{\hat{F}_n(x) - F(x)}$ is called the
\emph{Kolmogorov-Smirnov statistic} and is used in the nonparametric
\emph{Kolmogorov-Smirnov Test} for goodness of fit.  The
Glivenko-Cantelli Theorem tells us that this is a consistent estimator
of goodness of fit, however the test itself requires information on
the rate of convergence.  The most common result in this area is
\emph{Donsker's Theorem}.  Mention the DKW Inequality too; weak forms
of this can be established using the Pollard proof of Glivenko
Cantelli which the one that generalizes to Vapnik-Chervonenkis
families.  We can develop that proof after we do some exponential inequalities.

TODO: Mention that there are generalizations of these results in the
closely related fields of Empirical Process Theory and Statistical
Learning Theory.  One of the goals of such generalizations is to prove
consistency of more general statistics derived from the empirical measure.

\subsection{Convergence In Distribution}
As we have already remarked convergence in distribution is really a
property of the laws of a sequence of random variables and therefore
the limit of a sequence of random variables that converge in
distribution can only be expected to be unique up to equality in
distribution.
\begin{lem}\label{UniquenessOfConvergenceInDistribution}$\eta, \xi, \xi_1, \xi_2, \dots$ be a random elements in a
  metric space $(S,d)$ such that $\xi_n \todist \xi$ and $\xi_n
  \todist \eta$, then $\eta \eqdist \xi$.
\end{lem}
\begin{proof}
Let $F$ be a closed set in $S$ and define $f_n(x) = nd(x,F) \wedge
1$.  Then the $f_n$ are bounded and continuous (look forward to Lemma
\ref{DistanceToSetLipschitz} for a proof of a stronger result) and
$f_n \downarrow \characteristic{F}$ thus by Monotone Convergence,
\begin{align*}
\probability {\xi \in F} &= \lim_{n \to \infty} \expectation{f_n(\xi)}=
\lim_{n \to \infty} \lim_{m \to \infty} \expectation{f_n(\xi_m)} =  \lim_{n \to \infty} \expectation{f_n(\eta)}=\probability {\eta \in F} 
\end{align*}
Since the closed sets are a $\pi$-system that generate the Borel
$\sigma$-algebra on $S$ we have $\xi \eqdist \eta$ by montone classes
(specifically Lemma \ref{UniquenessOfMeasure}).
\end{proof}
This result will also follow from the fact that weak convergence of
probability measures corresponds to convergence in a metric topology on the space
of probability measures (proven later in this chapter).

Our next goal is to establish that convergence in
distribution is implied by convergence in probability.
\begin{lem}\label{ConvergenceInProbabilityImpliesConvergenceInDistribution}Let $\xi, \xi_1, \xi_2, \dots$ be a random elements in a
  metric space $(S,d)$ such that $\xi_n \toprob \xi$, then $\xi_n
  \todist \xi$.
\end{lem}
\begin{proof}Pick a bounded continous function $f : S \to \reals$,
  then $\expectation{f(\xi_n)}$.  By Lemma
  \ref{ContinuousMappingProbability} we know that $f(\xi_n) \toprob
  f(\xi)$.  Because $f$ is bounded, we know that $f(\xi_n)$ and
  $f(\xi)$ are integrable and therefore $f(\xi_n) \tolp{1} f(\xi)$
  which implies the result.
\end{proof}

\begin{examp}[Sequence converging in distribution but not in
  probability]Consider the binary expansion of real numbers in
  $[0,1]$, $x=0.\xi_1\xi_2\cdots$ and consider each $\xi_i$ as a
  random variable on the probability space $([0,1],\mathcal{B}([0,1]),
  \lambda)$.  We claim that $\xi_i$ converge in distribution to the
  uniform distribution on $\lbrace 0,1 \rbrace$ but that the $\xi_i$ diverge
  in probability.  We know from Lemma \ref{BernoulliSequence} that the $\xi_i$ are
  i.i.d. Bernoulli random variables with rate $\frac{1}{2}$ so the
  convergence in distribution follows.  If the $\xi_i$ converge in
  probability, there is a subsequence that converges almost surely.

By independence of the $\xi_i$, we know that for any $i \neq j$ 
\begin{align*} 
\probability{\xi_i \neq \xi_j} &= \probability{\xi_i=0 \text{ and }
  \xi_j=1} + \probability{\xi_i=1 \text{ and }
  \xi_j=0} \\
&= \probability{\xi_i=0}
  \probability{\xi_j=1} + \probability{\xi_i=1}
  \probability{\xi_j=0}  = \frac{1}{2}
\end{align*}
and therefore for $i \neq j$, 
\begin{align*}
\expectation{d(\xi_i, \xi_j) \wedge 1} &= \expectation{d(\xi_i,
  \xi_j)} = \probability{\xi_i \neq \xi_j} = \frac{1}{2}
\end{align*}
and we conclude that $\xi_i$ has no subsequence that is Cauchy in
probability and hence $\xi_i$ does not converge in probability.
\end{examp}

\begin{examp}[Sequence converging in distribution but diverging in mean]
Let $\xi_n$ be random variable which takes the value $n^2$ with
probability $\frac{1}{n}$ and takes the value $0$ with probability
$\frac{n-1}{n}$.  Note that $\lim_{n \to \infty} \xi_n = \lim_{n \to
  \infty} n = \infty$.  On the other hand, if we let $f$ be a bounded
continuous function then 
\begin{align*}
\lim_{n \to \infty} \expectation{f(\xi_n)} &= \lim_{n \to
  \infty}\frac{n-1}{n} f(0) + \lim_{n \to \infty}\frac{1}{n} f(n^2) \\
&= f(0)
\end{align*}
where we have used the boundedness of $f$.
Therefore, $\xi_n \todist \delta_0$ even though it diverges in mean.
\end{examp}

\begin{lem}\label{ConvergeInDistributionToConstant}Let $\xi_n$ be a sequence of real valued random variables
  that converge in distribution to a random variable $\xi$ that is
  almost surely a constant, then $\xi_n$ converges to $\xi$ in probability
  as well.
\end{lem}
\begin{proof}
Suppose that $\xi_n$ converges in distribution to $c \in \reals$.
Note that the function $f(x) = \abs{x-c} \wedge 1$ is bounded and
continuous and therefore we know 
\begin{align*}
\lim_{n \to \infty} \expectation{\abs{\xi_n - c} \wedge 1} &=
  \expectation{\abs{c - c} \wedge 1} = 0
\end{align*}
which, by Lemma
\ref{ConvergenceInProbabilityAsConvergenceInExpectation}, shows that $\xi_n$ converges to $c$ in probability as well.
\end{proof}

The definition we have given for convergence in distribution has the
advantage of applying to general random elements in metric spaces but
that comes at the cost of being a bit abstract.  It is worth
connecting the abstract definition with more direct criteria that
apply for random variables.  

In fact the first equivalence is for discrete random variables.  Given
that our definition of convergence in distribution is in terms of
metric spaces, we must be specific about the metric that we put on the
range a discrete random variable.  For discussing convergence in
distribution the primary feature that we are concerned with is the
defintion of continuous functions.  If we put a metric 
\begin{align*}
d(x,y) &= \begin{cases}
1 & \text{if $x \neq y$} \\
0 & \text{if $x = y$} \\
\end{cases}
\end{align*}
then all functions are continuous.  Note that the same is true if we
consider the induced metric $\integers \subset \reals$.
\begin{lem}\label{ConvergenceInDistributionAsConvergenceOfDistributionFunctionsDiscreteCase}
Let $\xi, \xi_1, \xi_2, \dots$ be a sequence of discrete random
variables with countable range $S$.  Then $\xi_n \todist \xi$ if and
only if for every $x \in S$, we have $\lim_{n \to \infty}
\probability{\xi_n = x} = \probability{\xi = x}$.
\end{lem}
\begin{proof}
First let's assume that $\xi_n \todist \xi$.  From the discussion
preceeding the Lemma, we know that for any bounded function $f : S \to
\reals$, we have $\lim_{n \to \infty} \expectation{f(\xi_n)}  =
\expectation{f(\xi)}$.  In particular, for each $x \in S$, we may take
$f(y) = \characteristic{x}(y)$ in which case we have
$\lim_{n \to \infty} \probability{\xi_n = x}  =
\probability{\xi = x}$ as required.

So now assume the converse.  In the following, it is helpful to label
the elements of $S$ using the natural numbers.  Note that we can cast our assumption as
saying that for every $x_j \in S$, 
\begin{align*}
\lim_{n \to \infty} \expectation{\characteristic{x_j}(\xi_n)}  =\expectation{\characteristic{x_j}(\xi)} 
\end{align*}
Furthermore, any bounded function can be written as a linear
combination $f(y) = \sum_{j=1}^\infty f_j \cdot
\characteristic{x_j}(y)$.  By linearity of expectation and our
assumption it is trivial to see that for any finite linear combination $f_N(y) = \sum_{j=1}^N f_j \cdot
\characteristic{x_j}(y)$, we in fact have
\begin{align*}
\lim_{n \to \infty} \expectation{f_N(\xi_n)}  =\expectation{f_N(\xi)} 
\end{align*}
and our task is to extend this to general infinite sums.   Let $M>0$
be a bound for $f$ defined as above.  

Pick an $\epsilon >0$.  Since $\sum_{j=1}^\infty \probability{ \xi
  =x_j} =1$ we can find $J > 0$ such that $\sum_{j=1}^J \probability{ \xi
  =x_j} > 1 - \epsilon$.  For each $j =1, \dots, J$ we can find $N_j >
0$ such that $\abs{\probability{ \xi  = x_j} - \probability{ \xi  =
    x_j}} < \frac{\epsilon}{J}$ for $n > N_j$.  Now take $N = \max(N_1,
\dots, N_J)$ and then we have for all $n > N$, $\sum_{j=1}^J \probability{ \xi_n
  =x_j} > 1 - 2\epsilon$. If we let $f_j = f(x_j)$ for each $x_j \in S$, then we
have the following calculation
\begin{align*}
\abs{\expectation{f(\xi_n) - f(\xi)}} &\leq \sum_{j=1}^J f_j
\abs{\probability{\xi_n = x_j} - \probability{\xi = x_j}} +
\abs{\sum_{j=J+1}^\infty f_j \probability{\xi_n = x_j} } +
\abs{\sum_{j=J+1}^\infty f_j \probability{\xi = x_j} } \\
&\leq \sum_{j=1}^J\abs{f_j} \frac{\epsilon}{J}+ 2M\epsilon + M\epsilon <
4M \epsilon
\end{align*}
Since $\epsilon>0$ was arbitrary we have $\lim_{n \to \infty}
\expectation{f(\xi_n)} =\expectation{f(\xi)}$ and we are done.
\end{proof}

 In the case of general random variables, we can also characterize
 convergence in distribution by looking at pointwise convergence of distribution
 functions and using a proof similar in spirit to that used above for
 discrete random variables, but it comes with a subtle twist.
\begin{lem}\label{ConvergenceInDistributionAsConvergenceOfDistributionFunctions}Let $\xi, \xi_1, \xi_2, \dots$ be sequence of random
  variables with distribution functions $F(x), F_1(x), F_2(x), \dots$.
  If $\xi_n \todist \xi$ then $\lim_{n \to \infty} F_n(x) = F(x)$ for
  all $x \in \reals$ such that $F$ is continuous at $x$.  Conversely,
  if $\lim_{n \to \infty} F_n(x) = F(x)$ on a dense subset of $\reals$
  then $\xi_n \to \xi$.
\end{lem}
\begin{proof}
Let us first assume that $\xi_n \to \xi$.  Consider a function $\characteristic{(-\infty, x]}$ for $x \in
\reals$ so that $F(x) = \expectation{\characteristic{(-\infty, \xi]}}$
and $F_n(x) = \expectation{\characteristic{(-\infty, \xi_n]}}$.  Note
that we cannot just apply the definition of convergence in
distribution to derive the result because $\characteristic{(-\infty,
  x]}$ is not continuous; so our goal is to extend to defining
property of convergence in distribution to a particular class of
discontinuous functions.  The way to do this is to approximate by
continuous functions.  To this end, define for each integer $x \in \reals$,
$m > 0$ the
following bounded continuous approximations of the indicator function $\characteristic{(-\infty,
  x]}$:
\begin{align*}
f^+_{x, m}(y) = \begin{cases}
1 & \text{if $y \leq x$} \\
m(x-y) + 1 & \text{if $x < y < x + \frac{1}{m}$} \\
0 & \text{if $x + \frac{1}{m} \leq y$} \\
\end{cases}
\end{align*}
and 
\begin{align*}
f^-_{x, m}(y) = \begin{cases}
1 & \text{if $y \leq  x - \frac{1}{m} $} \\
m(x-y) & \text{if $x - \frac{1}{m} < y < x $} \\
0 & \text{if $x \leq y$} \\
\end{cases}
\end{align*}
and note that $f^-_{x,m}(y) < \characteristic{(-\infty, x]} (y) <
f^+_{x,m}(y)$ and 
\begin{align*}
\expectation{f^-_{x,m}(\xi)} &=
\lim_{n \to \infty} \expectation{f^-_{x,m}(\xi_n)}\\
&\leq \liminf_{n \to \infty} \expectation{\characteristic{(-\infty, x]} (\xi_n)}\\
&=\liminf_{n \to \infty}  F_n(x) \\
&\leq \limsup_{n \to \infty}  F_n(x) \\
&= \limsup_{n \to \infty}  \expectation{\characteristic{(-\infty, x]} (\xi_n)} \\
&\leq \limsup_{n \to \infty}  \expectation{f^+_{x,m}(\xi_n)} \\
&=\lim_{n \to \infty} \expectation{f^+_{x,m}(\xi_n)} =
\expectation{f^+_{x,m}(\xi)}  \\
\end{align*}
But we also can see that for every $x,y \in \reals$, $\lim_{m \to
  \infty} f^-_{x,m}(y) = \characteristic{(-\infty, x)} (y)$ and $\lim_{m \to
  \infty} f^+_{x,m}(y) = \characteristic{(-\infty, x]} (y)$.
By application of Dominated Convergence, we see that $\lim_{m \to \infty}
\expectation{f^-_{x,m}(\xi)} = F(x-)$ and $\lim_{m \to \infty}
\expectation{f^+_{x,m}(\xi)} = F(x)$ so if $x$ is a point of
continuity of $F$ then $F(x-)=F(x)$ which shows $\liminf_{n \to
  \infty}  F_n(x) = \limsup_{n \to \infty}  F_n(x) = F(x)$.

Now let's assume that we have a dense set $D \subset \reals$ with $\lim_{n \to \infty} F_n(x) = F(x)$ for all $x
\in D$.  Pick a bounded continuous function $f: \reals \to \reals$ and
we must show $\lim_{n \to \infty} \expectation{f(\xi_n)} \to
\expectation{f(\xi)}$.  We will again make an approximation argument. To see how to proceed, recast our hypothesis
as the statement that $\lim_{n \to \infty} \expectation{\characteristic{(-\infty,x]}(\xi_n)} \to
\expectation{\characteristic{(-\infty,x]} (\xi)}$ for every $x \in D$
and note that by taking sums of functions of the form
$\characteristic{(-\infty,x]}(y)$ allows us to create step functions.
So, the idea of the proof is to carefully approximate $f$ by step functions so
that we may leverage our hypothesis.

We pick $\epsilon > 0$.  First it is helpful to allow ourselves to concentrate on a finite
subinterval of the reals.  As $F$ is a distribution function, we know
$\lim_{x \to -\infty} F(x) = 0$ and $\lim_{n \to \infty} F(x) = 1$ and
therefore by density of $D$ we may find $r,s \in D$ such that $F(r)
\leq \frac{\epsilon}{2}$ and $F(s) \geq 1 - \frac{\epsilon}{2}$.
Because $\lim_{n \to \infty} F_n(x) = F(x)$ for $x \in D$, we can find
and $N_1 > 0$ such that $F_n(r)
\leq \epsilon$ and $F_n(s) \geq 1 - \epsilon$ for
$n>N_1$.

Now we turn our attention to the approximation of $f$ and note that by compactness of $[r,s]$ we know that
we can find a finite partition $r_0 = r < r_1 < \cdots < r_{m-1} <
r_m=s$ such that $r_j \in D$ and $\abs{f(r_{j}) - f(r_{j-1})} \leq
\epsilon$ for $1 \leq j \leq m$.  To see this we know that $f$ is
uniformly continuous on $[r,s]$ and therefore there exists $\delta >
0$ such that for any $x,y \in [r,s]$ with $\abs{x -y} < \delta$ we
  have $\abs{f(x) - f(y)} < \epsilon$.   We construct $r_j$
  inductively starting with $r_0 = r$.  Using uniform continuity as
  above and the density of $D$, given $r_{j-1}$ we can find $r_j$ with
  $r_{j-1}+\frac{\delta}{2} \leq r_j < r_{j-1} + \delta$ and we know
  that $\abs{f(r_j) - f(r_{j-1})} < \epsilon$.  In less than $\lceil
  \frac{2(s -r)}{\delta} \rceil$ steps we have $\abs{r_j - s} <
  \delta$ and we terminate the construction.  Having constructed the
  partition, define the step
function 
\begin{align*}
g(y) &= \sum_{j=1}^m f(r_j) \left (\characteristic{(-\infty,
    r_j]}(y) - \characteristic{(-\infty,    r_{j-1}]}(y) \right ) =
\sum_{j=1}^m f(r_j) \characteristic{(r_{j-1}, r_j]}(y)
\end{align*}
and note that by construction we have $\abs{f(y) - g(y)} \leq \epsilon$
for all $r \leq y \leq s$.  

So now we estimate
\begin{align*}
\abs{\expectation{f(\xi_n)} - \expectation{f(\xi)}} &\leq
\abs{\expectation{f(\xi_n)} - \expectation{g(\xi_n)}}  +
\abs{\expectation{g(\xi_n)} - \expectation{g(\xi)}}  + \abs{\expectation{g(\xi)} - \expectation{f(\xi)}} 
\end{align*}
and consider each term on the left hand side.  By boundedness of $f$
we pick $M > 0$ such that $f(x) \leq M$ for all $x \in \reals$ and
note that since $g(y) = 0$ for $y\leq r$ and $y > s$,
\begin{align*}
\abs{\expectation{f(\xi_n)} - \expectation{g(\xi_n)}} &\leq
\abs{\expectation{f(\xi_n); \xi_n \leq r} } +
\abs{\expectation{f(\xi_n)-g(\xi_n); r < \xi_n \leq s} } +
\abs{\expectation{f(\xi_n); \xi_n > s} } \\
&\leq \epsilon M + \epsilon + \epsilon M = \epsilon (2M +1)
\end{align*}
and similarly, 
\begin{align*}
\abs{\expectation{f(\xi)} - \expectation{g(\xi)}} 
&\leq \frac{\epsilon}{2} M + \epsilon + \frac{\epsilon}{2} M =
\epsilon ( M + 1)
\end{align*}

Now leveraging the fact that $\lim_{n \to \infty} F_n(r_j) = F(r_j)$
for every $0 \leq j \leq m$ and the finiteness of this set, we can
pick $N_2 > 0$ such that $\abs{F_n(r_j) - F(r_j)} \leq \frac{\epsilon}{2mM}$ for
all $n > N_2$ and all $0 \leq j \leq m$.  Using this fact and the
definition of $g$, 
\begin{align*}
\abs{\expectation{g(\xi_n)} - \expectation{g(\xi)}} &= \abs{\sum_{j=1}^m f(r_j) \left (\expectation{\characteristic{(-\infty,
    r_j]}(\xi_n)} - \expectation{\characteristic{(-\infty,
    r_{j-1}]}(\xi_n)} - \expectation{\characteristic{(-\infty,
    r_j]}(\xi)} + \expectation{\characteristic{(-\infty,
    r_{j-1}]}(\xi)}\right )} \\
&=  \abs{\sum_{j=1}^m f(r_j) \left (F_n(r_j) - F_n(r_{j-1}) - F(r_j)+
    F(r_{j-1}) \right )}  \\
&\leq \sum_{j=1}^m \abs{f(r_j)} \left( \abs{F_n(r_j) - F(r_j)} +
  \abs{F_n(r_{j-1}) - F(r_{j-1})} \right )\\
&\leq \epsilon
\end{align*}
for every $n > N_2$.  

Putting these three bounds together we have for $n > N_1 \wedge N_2$,
$\abs{\expectation{f(\xi_n)} - \expectation{f(\xi)}} \leq (3M + 3)
\epsilon$ and we are done.
\end{proof}

\begin{examp}Let $\xi_n$ be a $U(-\frac{1}{n}, \frac{1}{n})$ random
  variable and let $\xi = 0$ a.s., then $\xi_n \todist \xi$.  Note
  that the distribution function of $\xi_n$ is 
\begin{align*}
F_n(x) &= \begin{cases}
1 & \text{if $x \geq \frac{1}{n}$} \\
\frac{1}{2}(nx + 1) & \text{if $-\frac{1}{n} < x < \frac{1}{n}$} \\
0 & \text{if $x \leq \-\frac{1}{n}$} \\
\end{cases}
\end{align*}
Then it is clear that $\lim_{n \to \infty} F_n(x) = 0$ for $x <
0$ and  $\lim_{n \to \infty} F_n(x) = 1$ for $x >
0$.   Since the distribution function of $\delta_0$ is
$\characteristic{[0, \infty)}$ we apply Lemma
\ref{ConvergenceInDistributionAsConvergenceOfDistributionFunctions} to
conclude convergence in distribution.  Note that $\lim_{n \to \infty}
F_n(0) = \frac{1}{2} \neq F(0) = 1$.  It is also worth noting that the
pointwise limit of $F_n$ isn't actually a distribution function
(e.g. is not right continuous at $0$).
TODO: Is convergence in distribution easy to prove directly using the defintion?
\end{examp}

The theory of convergence in distribution is rather vast and can be
studied at many different levels of generality and sophistication.
For example, we have stated the basic definitions on general metric
spaces and for some of most basic foundations it is no more difficult
to prove things in metric spaces than in a more concrete case such as
random variables or vectors.  However it soon becomes wise to
temporarily drop the generality and concentrate on the special case of
random vectors (e.g. to prove probably the most famous result of
probability: the Central Limit Theorem).  At some point it becomes
necessary to return to the general case but at that point one needs to
be prepared to bring more powerful tools to the table as the theory
becomes much more subtle.

In this section we start the program and deal with those first
results in the theory of weak convergence that can be simply dealt
with in the context of general metric spaces.

One of the key features of dealing with probability measures (and to
a lesser extent measures in general) is that they are very \emph{well
  behaved} when viewed as functionals (i.e. linear mappings from
functions to $\reals$).  We've left that statement deliberately vague
for the moment since is properly understood within the context of the
general theory of distributions.  What we want to begin exploring is a
side effect of this good behavior: namely that weak convergence of
probability measures can be characterized by using many different
classes of functions other than the bounded continuous ones.  In one
direction one can prove results that tell us that to prove weak
convergence it is not necessary to test with all bounded continuous
functions but one only need use some subset of these. In fact, in the
case of random variables and random vectors, it is
only necessary to test with compactly supported
infinitely differentiable functions (which we won't prove quite yet
since we're still dealing with general metric spaces).
In another direction, knowing that one has a weakly convergent
sequence of probability measures one can extend the convergence with
test functions to use statements about some classes of discontinuous functions
(e.g. indicator functions).  Combining both directions, one can
characterize weak convergence by testing against certain classes of
discontinuous functions.  

Our first foray into the plasticity of weak convergence of probability
measures is the following set of conditions that characterize weak
convergence of Borel probability measures on metric spaces.  Before we
state the Theorem we need a couple of quick definitions.

\begin{defn}Let $\mu$ be a Borel probability measure on a metric space
  $S$.  We say that a subset $A \subset S$ is a $\mu$-continuity set
  if $\mu(\partial A) = 0$.
\end{defn}

\begin{defn}Let $(S,d)$ and $(S^\prime,d^\prime)$ be metric spaces.
  We say $f : S \to S^\prime$ is \emph{Lipschitz continuous} if there
  exists a $C \geq 0$ such that $d(f(x), f(y)) \leq C d(x,y)$ for all
  $x,y \in S$.  We often such a $C$ a \emph{Lipschitz constant}.  
\end{defn}
It is often convenient to refer to a Lipschitz continuous function as
being Lipschitz.
\begin{examp}As examples of continuous functions that fail to be Lipschitz
  continous consider $f(x) = x^2$ on $\reals$ and $\sin(1/x)$ on $(0,
  \infty)$.  Note that $x^2$ is Lipschitz on any compact set.  This
  latter fact can be generalized to show that any continuously
  differentiable function can be shown to be Lipschitz on any compact set.
\end{examp}
\begin{lem}A Lipschitz function $f$ is uniformly continuous.
\end{lem}
\begin{proof}Let $C$ be a Lipschitz constant for $f$.  The for
  $\epsilon > 0$, let $\delta = \frac{\epsilon}{C}$.
\end{proof}
As an example of Lipschitz function that we'll make use of in the next
Theorem, consider the following.
\begin{lem}\label{DistanceToSetLipschitz}Let $F \subset S$ be a closed subset and define $f(x) =
  d(x, F) = \inf_{y \in F} d(x,y)$.  Then $f(x)$ is Lipschitz with
  Lipschitz constant $1$.
\end{lem}
\begin{proof}
Let $\epsilon > 0$, $x,y \in S$ and pick a $z \in F$ such that $f(x)
\leq d(x,z) \leq f(x) + \epsilon$.  By the triangle inequality, we
have
\begin{align*}
f(y) &\leq d(y,z) \leq d(x,z) + d(x,y) \leq f(x) + d(x,y) + \epsilon
\end{align*}
The argument is symmetric in $x$ and $y$ so we also have that
\begin{align*}
f(x) &\leq  f(y) + d(x,y) + \epsilon
\end{align*}
and therefore $\abs{f(x) - f(y)} \leq d(x,y) + \epsilon$.
Since $\epsilon$ was arbitrary let it go to $0$ and we are done.
\end{proof}

\begin{lem}\label{MaxMinOfLipschitz}Let $f,g : S \to \reals$ be Lipschitz with Lipschitz
  constants $C_f$ and $C_g$ respectively.  Then both $f \wedge g$
  and $f \vee g$ are Lipschitz with Lipschitz constants $C_f \vee C_g$.
\end{lem}
\begin{proof}
The proof is elementary but long winded; we only do the case of $f
\wedge g$.  Pick $x,y \in S$ and
consider $\abs{(f \wedge g)(x) - (f \wedge g)(y)}$.  We break the analysis
down into four cases.  

Case (i): Suppose $(f \wedge g)(x) \geq (f \wedge g)(y)$ and $f(y) \leq g(y)$.
\begin{align*}
\abs{(f \wedge g)(x) - (f \wedge g)(y)} &= (f \wedge g)(x) - f(y) \leq
f(x) - f(y) \leq C_f d(x,y)
\end{align*}

Case (ii): Suppose $(f \wedge g)(x) \geq (f \wedge g)(y)$ and $g(y) \leq f(y)$.
\begin{align*}
\abs{(f \wedge g)(x) - (f \wedge g)(y)} &= (f \wedge g)(x) - g(y) \leq
g(x) - g(y) \leq C_g d(x,y)
\end{align*}

Case (iii): Suppose $(f \wedge g)(y) \geq (f \wedge g)(x)$ and $f(x) \leq g(x)$.
\begin{align*}
\abs{(f \wedge g)(x) - (f \wedge g)(y)} &= (f \wedge g)(y) - f(x) \leq
f(y) - f(x) \leq C_f d(x,y)
\end{align*}

Case (iv): Suppose $(f \wedge g)(y) \geq (f \wedge g)(x)$ and $g(x) \leq f(x)$.
\begin{align*}
\abs{(f \wedge g)(x) - (f \wedge g)(y)} &= (f \wedge g)(y) - g(x) \leq
g(y) - g(x) \leq C_g d(x,y)
\end{align*}
Thus we see $\abs{(f \wedge g)(x) - (f \wedge g)(y)} \leq (C_f \vee
C_g) d(x,y)$.

The case of $f \vee g$ follows in a similar way.
\end{proof}

\begin{thm}[Portmanteau Theorem]\label{PortmanteauTheorem}Let $\mu$ and $\mu_n$ be a sequence of
 Borel probability measures on a metric space $S$.  The following are equivalent
\begin{itemize}
\item[(i)] $\mu_n$ converge in distribution to $\mu$.
\item[(ii)] $\sexpectation{f}{n} \to \expectation{f}$
  for all bounded Lipschitz functions $f$.
\item[(iii)] $\limsup_{n \to \infty} \mu_n(C) \leq
  \mu(C)$ for all closed sets $C$
\item[(iv)] $\liminf_{n \to \infty} \mu_n(U) \geq
  \mu(U)$ for all open sets $U$
\item[(v)] $\lim_{n \to \infty} \mu_n(A) = \mu(A)$ for all
  $\mu$-continuity sets $A$.
\end{itemize}
\end{thm}

Before we begin the proof, we pay particular attention to the fact that one does not have
equality in the case of indicator functions.  What this
is saying is that mass can move out to the boundary during limiting
processes of distributions.  In the case of open sets that mass can be lost (to the
boundary) whereas in the case of closed sets, it can magically appear
in the limit.  An example here is the limit of point masses
$\delta_{\frac{1}{n}}$.  It is elementary that $\delta_{\frac{1}{n}} \todist
\delta_0$ but if one considers the open set $(0,1)$, then
$\delta_{\frac{1}{n}}(0,1) = 1$ but $\delta_0(0,1) = 0$.  In a similar
way, take the closed set $\lbrace 0 \rbrace$ and we see $\delta_{\frac{1}{n}} \lbrace 0 \rbrace = 0$ but $\delta_0
\lbrace 0 \rbrace = 1$.  The statement in (v) neatly captures the idea
that the only way we fail to converge with indicator functions is when
mass appears on the boundary of the set; if we rule out that
possiblity assuming the set is a continuity set then we have
convergence when the corresponding indicator function is used as the
test function.

\begin{proof}
Note that (i) implies (ii) is trivial since a bounded Lipschitz
function is also bounded and continuous.

(ii) implies (iv): Suppose we have $U \subset S$ an open set.  Let
$f_n(x) = (nd(x,U^c)) \wedge 1$.  By Lemma \ref{DistanceToSetLipschitz}
and Lemma \ref{MaxMinOfLipschitz} we know that $f_n(x)$ is Lipschitz with
constant $n$.  It is trivial to see that $f_n(x)$ is 
increasing.  Furthermore $\lim_{n \to \infty}
f_n(x) = \characteristic{U}(x)$.  This can be seen by noting that if $x \in U$, then by taking a
ball $B(x,r) \subset U$, we know that $d(x, U^c) \geq r$ and therefore
$f_n(x) = 1$ for $n \geq \frac{1}{r}$.  On the other hand, it is
trivial that $f_n(x) = 0$ for all $x \in U^c$ and all $n$.  Armed with
these facts we prove (iv)
\begin{align*}
\mu(U) &= \lim_{n \to  \infty}\expectation{f_n}  & & \text{by
  Monotone Convergence Theorem}\\
&= \lim_{n \to  \infty} \lim_{m \to  \infty} \sexpectation{f_n}{m} & &
\text{by (ii)}\\
&\leq \lim_{n \to  \infty} \liminf_{m \to  \infty}
\sexpectation{\characteristic{U}}{m}  & & \text{since $f_n \leq \characteristic{U}$}\\
&= \liminf_{m \to  \infty} \mu_m(U)\\
\end{align*}

(iii) is equivalent to (iv): Assume (iii) and use the fact $\liminf_{n \to \infty}
f_n = -\limsup_{n \to \infty} -f_n$ and (iv) to calculate for an open set $U$,
\begin{align*}
\liminf_{n \to \infty} \mu_n(U) &= -\limsup_{n \to \infty} -\mu_n(U) = -\limsup_{n \to \infty} \mu_n(U^c) + 1 \geq -\mu(U^c) + 1 = \mu(U)
\end{align*}
The proof that (iv) implies (iii) follows in an analogous way.

(iv) implies (i).  Suppose $f \geq 0$ continuous, then for every
$\lambda \in \reals$, we know that $\lbrace f > \lambda \rbrace =
f^{-1}((\lambda, \infty))$ is an open subset of $S$.  Because of that
we we may use Lemma
\ref{TailsAndExpectations}, Fatou's Lemma (Theorem \ref{Fatou}) and (iii) to see
\begin{align*}
\int f \, d\mu &= \int_0^\infty \sprobability{f > \lambda}{\mu} \,
d\lambda \\
&\leq \int_0^\infty \liminf_{n \to \infty} \sprobability{f > \lambda}{\mu_n} \, d\lambda \\
&\leq \liminf_{n \to \infty} \int_0^\infty \sprobability{f >
  \lambda}{\mu_n} \, d\lambda \\
&= \liminf_{n \to \infty} \int f \, d\mu_n\\
\end{align*}
Now we play the same trick as in the proof of Dominated Convergence.
Suppose $f$ is bounded and continuous and suppose $\abs{f} \leq c$.
By what we have just shown,
\begin{align*}
\int f \, d\mu &= -c + \int (c+f) \, d\mu \leq -c + \liminf_{n \to
  \infty} \int (c+ f) \, d\mu_n = \liminf_{n \to \infty} \int f \, d\mu_n \\
-\int f\, d\mu &= -c + \int (c-f) \, d\mu \leq -c + \liminf_{n \to
  \infty} \int (c- f) \, d\mu_n =  -\limsup_{n \to \infty} \int f \, d\mu_n
\end{align*}
Therefore 
\begin{align*}
\limsup_{n \to \infty} \int f \, d\mu_n &\leq \int f \, d\mu
\leq \liminf_{n \to \infty} \int f \, d\mu_n
\end{align*} which implies $\lim_{n
  \to \infty} \int f \, d\mu_n = \int f \, d\mu$ and (i) is proven.

 (iii) and (iv) imply (v).  Pick a $\mu$-continuity set $A$.  The first thing to
note is that $\mu(A) = \mu(\overline{A}) = \mu(\interior(A))$ because
they all differ by a subset of $\partial A$.   Now on
the one hand, 
\begin{align*}
\liminf_{n \to \infty} \mu_n(A) &\geq \liminf_{n \to \infty} \mu_n(\interior(A)) \geq \mu(\interior(A)) = \mu(A)
\end{align*}
On the other hand, 
\begin{align*}
\limsup_{n \to \infty} \mu_n(A) &\leq \limsup_{n \to \infty}
\mu_n(\overline{A}) \leq \mu(\overline{A}) = \mu(A)
\end{align*}
which shows that $\lim_{n \to \infty} \mu_n(A) = \mu(A)$.

(v) implies (iii).  Pick a closed set and for every $\epsilon > 0$
consider the closed $\epsilon$-neighborhood $F_\epsilon = \lbrace x \mid d(x,
F) \leq  \epsilon \rbrace$.  Note that $\partial F_\epsilon \subset \lbrace x \mid d(x,
F) =  \epsilon \rbrace$ since if $d(x,F) < \epsilon$ then by
continuity of the function $f(y) = d(y,F)$ we can find a ball $B(x,r)$
such that $d(y,F) < \epsilon$ for every $y \in B(x,r)$; thus proving
$x$ is in the interior of $F_\epsilon$.  The fact that $\partial F_\epsilon \subset \lbrace x \mid d(x,
F) =  \epsilon \rbrace$ shows that the $\partial F_\epsilon$ are disjoint.

Next note that $\mu (\partial F_\epsilon) \neq 0$ for at most a
countable number of $\epsilon$.  For every $n \geq 1$, there can only
be a finite number $F_\epsilon$ with $\mu (\partial F_\epsilon) \geq
\frac{1}{n}$ because of the disjointness of $F_\epsilon$ and the
countable additivity of $\mu$.  So the set of all $\epsilon$ with $\mu
(\partial F_\epsilon)  > 0$ is a countable union of finite set and
therefore countable.  Now the complement of a countable set in
$\reals$ is dense (Lemma \ref{ComplementOfCountableSetDense}) hence $F_\epsilon$ is a $\mu$-continuity set for a
dense set of $\epsilon$.  

Now deriving (iii) is easy.  Pick a decreasing sequence of $\epsilon_m$ such that
$\lim_{m \to \infty} \epsilon_m = 0$ and each $F_{\epsilon_m}$ is a
$\mu$-continuity set.  Therefore by subadditivity of measure and and our
hypothesis, for each $m$
\begin{align*}
\limsup_{n \to \infty} \mu_n(F) &\leq \lim_{n \to \infty}
\mu_n(F_{\epsilon_m}) = \mu(F_{\epsilon_m})
\end{align*}
However, by continuity of measure, we know
that 
\begin{align*}
\limsup_{n \to \infty} \mu_n(F) &\leq \lim_{m \to \infty} \mu(F_{\epsilon_m}) = \mu_n(F)
\end{align*}
and we're done.
\end{proof}

\begin{defn} Given metric spaces $(S,d)$ and $(S^\prime, d^\prime)$ and a map $g: S \to S^\prime$,
  the set of discontinuity points $D_g$ is the set of $x \in S$ such
  that for every $\epsilon > 0$ and $\delta > 0$ there exists $y \in
  S$ such that $d(x,y) < \delta$ and $d^\prime(g(x), g(y)) > \epsilon$.
\end{defn}
\begin{thm}[Continuous Mapping Theorem]\label{ContinuousMappingTheorem}Let $\xi_n$ and $\xi$ be random
  elements in a metric space $S$.  Let $S'$ be a metric space such that
  there exists a map $g: S \to S^\prime$ with the property that the
  $\probability{\xi \in D_g} = 0$.  Then 
\begin{itemize}
\item[(i)] If $\xi_n$ converges in distribution to $\xi$ then
  $g(\xi_n)$ converges in distribution to $g(\xi)$.
\item[(ii)] If $\xi_n$ converges in probability to $\xi$ then
  $g(\xi_n)$ converges in probability to $g(\xi)$.
\item[(iii)] If $\xi_n$ converges a.s. to $\xi$ then
  $g(\xi_n)$ converges a.s. to $g(\xi)$.
\end{itemize}
\end{thm}
\begin{proof}

TODO: This proof makes the assumption that $g$ is continuous.  This is
a big simplification for the distribution case in particular.  Provide
the proof with the weaker assumption.

To prove (i), suppose we are given a bounded continuous $f : S^\prime
\to \reals$.  Then $f \circ g : S \to \reals$ is also bounded and
continuous hence
\begin{align*}
\lim_{n \to \infty} \int f(g(\xi_n)) \, d\mu = \int f(g(\xi)) \, d\mu
\end{align*}
which shows that $g(\xi_n) \todist g(\xi)$.

To prove (ii), for every $\epsilon, \delta > 0$, define 
\begin{align*}
B^\epsilon_\delta = \{ x \in S \mid \exists \, y \in S \text{ with }
d(x,y) < \delta \text{ and } d^\prime(g(x), g(y)) \geq \epsilon\}
\end{align*}
Note that for $\delta^\prime < \delta$ and fixed $\epsilon$ we have
$B^\epsilon_{\delta^\prime} \subset B^\epsilon_{\delta}$.  Continuity
of $g$ implies that $\bigcap_{m=1}^\infty
  B^\epsilon_{\frac{1}{m}} = \emptyset$; and therefore by continuity of
  measure (Lemma
\ref{ContinuityOfMeasure}) we know that $\lim_{m \to \infty} \probability{\xi \in B^\epsilon_{\frac{1}{m}}} = 0$.

Now fix $\epsilon,\gamma >0$ and note that for all $n,m > 0$, we have
the bound
\begin{align*}
\probability{d^\prime(g(\xi_n),
  g(\xi)) \geq \epsilon} \leq \probability{d(\xi_n, \xi) \geq
  \frac{1}{m}} + \probability{\xi \in B^\epsilon_{\frac{1}{m}}}
\end{align*}

By the previous observation, we can find an $m>0$ such that
$\probability{\xi \in B^\epsilon_{\frac{1}{m}}} < \frac{\gamma}{2}$.
Having picked such an $m>0$, since $\xi_i$ converges to $\xi$ in
probability, we can find $N >0$ such that $\probability{d(\xi_n, \xi) \geq
  \frac{1}{m}} < \frac{\gamma}{2}$ for all $n > N$.

To prove (iii), simply note that by continuity of $g$, $\xi_n(\omega)
\to \xi(\omega)$ implies $g(\xi_n(\omega)) \to g(\xi(\omega))$.
\end{proof}

The following result is a basic tool in the theory of asymptotic
statistics.  We state and prove it here because it is a
straightforward application of the Portmanteau Theorem, but we'll wait
until we've proven the Central Limit Theorem to give examples of how
it is applied.  
\begin{thm}[Slutsky's Theorem]\label{Slutsky}Let $\xi_n$ and $\eta_n$ be two sequences of
  random elements in $(S,d)$ such that $d(\xi_n,\eta_n) \toprob 0$.  
If $\xi$ is a random element in $(S,d)$
such that $\xi_n \todist \xi$ in distribution, then
  $\eta_n \todist \xi$.
\end{thm}
\begin{proof}
By the Portmanteau Theorem (Theorem \ref{PortmanteauTheorem}) it
suffices to show $\expectation{f(\eta_n)} \to \expectation{f(\xi)}$
for all bounded Lipschitz functions $f : S \to \reals$.  Pick such an
$f$ and $M,K > 0$ such that $\abs{f(x)} \leq M$ and $\abs{f(x) - f(y)} \leq K
d(x,y)$.  Then if we pick $\epsilon > 0$,
\begin{align*}
\lim_{n \to \infty} \abs{\expectation{f(\eta_n)} -
  \expectation{f(\xi_n)}} &\leq \lim_{n \to \infty} \expectation{\abs{
    f(\eta_n) - f(\xi_n)}} \\
&\leq \lim_{n \to \infty} \expectation{\abs{
    f(\eta_n) - f(\xi_n)}\characteristic{d(\eta_n, \xi_n) \leq
    \epsilon}} +  \expectation{\abs{
    f(\eta_n) - f(\xi_n)}\characteristic{d(\eta_n ,\xi_n) >
    \epsilon} } \\
&\leq \epsilon K +  2 M \lim_{n \to \infty} \probability{d(\eta_n ,\xi_n) >
    \epsilon} \\
&= \epsilon K 
\end{align*}
Since $\epsilon$ was arbitrary, we have $\lim_{n \to \infty}\expectation{f(\eta_n)} =
\lim_{n \to \infty}\expectation{f(\xi_n)} = \expectation{f(\xi)}  $ and we are done.
\end{proof}
\begin{cor}[Slutsky's Theorem]Let $\xi_n$ and $\eta_n$ be two sequences of
random elements in $(S,d)$.  If $\xi$ is a random element in $(S,d)$
such that $\xi_n$ converges to $\xi$ in distribution and $c \in S$ is
  a constant such that $\eta_n$ converges to $c$ in probability, then for
  every continuous function $f$,
  $f(\xi_n,\eta_n)$ also converges to $f(\xi,c)$ in distribution.
\end{cor}
\begin{proof}
The critical observation here is that with the assumptions above the
random element $(\xi_n,\eta_n)$ converges to $(\xi,c)$ in
distribution.  Then we can apply the Continuous Mapping Theorem
(Theorem \ref{ContinuousMappingTheorem}) to
derive the result.  To see $(\xi_n,\eta_n) \todist (\xi,c)$, first
note that $d((\xi_n, \eta_n), (\xi_n, c)) = d(\eta_n, c) \toprob 0$ by
assumption.  Therefore by the previous lemma, it suffices to show that
$(\xi_n, c) \todist (\xi,c)$.  Pick a continuous bounded function $f :
S \times S \to \reals$ and note that $f(-,c) : S \to \reals$ is also
continuous and bounded.  Therefore $\lim_{n \to \infty}
\expectation{f(\xi_n, c)} = \expectation{f(\xi,c)}$.
\end{proof}

\subsection{Uniform Integrability}

In this section we introduce the technical notion of uniform
integrability of a family of random variables.  Informally uniform
integrability is the property that the tails of the family of
integrable random
variables can be simultaneously bounded in expectation.  Practically
one implication of this property is that one can use a single
truncation parameter to approximate all of the random variables in a
uniformly integrable family.  As an application of this fact we'll
observe that the truncation argument proof of the Weak Law of Large
Numbers extends from i.i.d. sequences of random variables to uniformly
integrable sequences of random variables.  It also worth noting that
the property of uniform integrability figures prominently in
martingale theory.

\begin{defn}A collection of random variables $\xi_t$ for $t \in T$ is
  \emph{uniformly integrable} if and only if $\lim_{M \to \infty}
  \sup_{t \in T} \expectation{\abs{\xi_t} ; \abs{\xi_t} > M} = 0$.
\end{defn}

\begin{examp}A sequence of identically distributed variables $\xi_n$ is
  uniformly integrable.  This can be seen easily by defining $g(x) =
  \abs{x}\characteristic{\abs{x}>M}$ and noting that 
\begin{align*}
\expectation{\abs{\xi_n} ; \abs{\xi_n} > M} = \expectation{g(\xi_n)} =
\int g(x) \, d\xi_n
\end{align*}
by Lemma \ref{ChangeOfVariables} which shows that the expectation is
independent of $n$ since $d\xi_n$ is independent of $n$.
\end{examp}

\begin{lem}\label{UniformIntegrabilityProperties}The random variables $\xi_t$ for $t \in T$ are uniformly
  integrable if and only if
\begin{itemize}
\item[(i)] $\sup_{t \in T} \expectation{\abs{\xi_t}} < \infty$
\item[(ii)] For every $\epsilon > 0$ there exists $\delta > 0$ such
  that if $\probability{A} < \delta$ then $\expectation{\abs{\xi_t} ;
    A} < \epsilon$ for all $t \in T$.
\end{itemize}
\end{lem}
\begin{proof}
First we assume uniform integrability of $\xi_t$.  To prove (i), pick
$M > 0$ such that $\expectation{\abs{\xi_t} ; \abs{\xi_t} > M} < 1$
for all $t \in T$.  Then for $t \in T$,
\begin{align*}
\expectation{\abs{\xi_t}} &= \expectation{\abs{\xi_t};\abs{\xi_t} \leq  M
} + \expectation{\abs{\xi_t}; \abs{\xi_t} > M} \\
&\leq M + 1
\end{align*}
To show (ii), pick $\epsilon > 0$, $M >0 $ such that
$\expectation{\abs{\xi_t} ; \abs{\xi_t} > M} < \frac{\epsilon}{2}$ and
$\delta < \frac{\epsilon}{2M}$.
Then
\begin{align*}
\expectation{\abs{\xi_t} ; A} &= \expectation{\abs{\xi_t} ; A \wedge
  \abs{\xi_t} \leq M} + \expectation{\abs{\xi_t} ; A \wedge
  \abs{\xi_t} > M} \\
&\leq M \delta + \expectation{\abs{\xi_t} ;   \abs{\xi_t} > M} \leq \epsilon
\end{align*}

Now assume (i) and (ii).  Pick $\epsilon > 0$ and $\delta >0 $ as in
(ii) and let $M > 0$ be such that $\expectation{\abs{\xi_t}} \leq M$ for all
$t \in T$.  Pick $N > \frac{M}{\delta}$ and note that
\begin{align*}
\probability{\abs{\xi_t} > N} &\leq \frac{\expectation{\abs{\xi_t}
  }}{N} \leq \frac{M}{N} < \delta
\end{align*}
so by (ii), $\expectation{\abs{\xi_t}; \abs{\xi_t} > N} < \epsilon$
and uniform integrability is proven.
\end{proof}

Here is a simple result that illustrates how the conditions for
uniform integrability in the previous Lemma can often be more
convenient than the definition.

\begin{lem}\label{SumsOfUniformlyIntegrable}Suppose $\abs{\xi_t}^p$ and $\abs{\eta_t}^p$ are both uniformly integrable
  families of random variables.  Then for every $a,b \in \reals$,
  $\abs{a\xi_t + b \eta_t}^p$ is uniformly integrable.
\end{lem}
\begin{proof}
By Lemma \ref{UniformIntegrabilityProperties} we know that $\sup_t
\expectation{\abs{\xi_t}^p} < \infty$ and $\sup_t
\expectation{\abs{\eta_t}^p} < \infty$; equivalently $\sup_t \norm{\xi_t}_p <
\infty$ and $\sup_t \norm{\eta_t}_p < \infty$.  Now by the triangle
inequality/Minkowski's inequality $\sup_t \norm{a\xi_t + b \eta_t}_p
\leq a \sup_t \norm{\xi_t}_p + b \sup_t \norm{\eta_t}_p < \infty$.
Thus condition (i) of Lemma \ref{UniformIntegrabilityProperties} is shown.

To see condition (ii) of Lemma \ref{UniformIntegrabilityProperties},
suppose $\epsilon > 0$ is given.  By this same Lemma applied to
$\abs{\xi_t}^p$ and $\abs{\eta_t}^p$ pick a $\delta > 0$ such that for
all $A$ with $\probability{A} < \delta$ we have
$\expectation{\abs{\xi_t}^p ; A} \leq \frac{\epsilon}{2^p a^p}$ and 
$\expectation{\abs{\eta_t}^p ; A} \leq \frac{\epsilon}{2^p b^p}$ for all
$t$.  Then
we have
\begin{align*}
\expectation{\abs{a\xi_t + b \eta_t}^p; A} &= \norm{a \xi_t
  \characteristic{A} + b \eta_t \characteristic{A}}_p^p \\
&\leq \left( a\norm{\xi_t \characteristic{A}}_p + b \norm{\eta_t
    \characteristic{A}}_p\right)^p \\
&\leq \left ( a \frac{\epsilon^{1/p}}{2a} + b
  \frac{\epsilon^{1/p}}{2b}\right )^p = \epsilon
\end{align*}
\end{proof}

Here is an example that shows that the condition (i) of the previous
Lemma is not sufficient to guarantee uniform integrability.
\begin{examp}Here we demonstrate a sequence $\xi_n$ with $\sup_n \expectation{\abs{\xi_n}} <
\infty$ but $\xi_n$ is not uniformly integrable.  Consider the sequence
$\xi_n$ constructed in Example \ref
{WLLNCounterExampleBoundedFirstMoment}.  Recall for that sequence,
$\expectation{\abs{\xi_n}} = 1$ for all $n>0$.  On the other hand, for any $M > 0$ and $n > 0$
we have
\begin{align*}
\expectation{\abs{\xi_n} ; \abs{\xi_n} > M} &= \begin{cases}
0 & \text{if $2^n \leq M$} \\
1 & \text{if $2^n > M$}
\end{cases}
\end{align*}
and therefore for all $M > 0$ we have $\sup_n \expectation{\abs{\xi_n}
  ; \abs{\xi_n} > M} = 1$.
\end{examp}

 TODO: Show convergence result for uniformly integrable sequences.

While we have shown that convergence in probability is strictly weaker
than convergence in mean, it turns out that adding the condition of
uniform integrability is precisely what is needed to make them
equivalent.  Before proving that result we have a Lemma that
illustrates the connection between uniform integrability and
convergence of means.
\begin{lem}\label{UniformIntegrableAndMeans}Let $\xi, \xi_1, \xi_2, \dotsc$ be positive random variables such
  that $\xi_n \todist \xi$, then $\expectation{\xi} \leq \liminf_{n \to
    \infty} \expectation{\xi_n}$.  Moreover, $\expectation{\xi} = \lim_{n \to
    \infty} \expectation{\xi_n} < \infty$ if and only if $\xi_n$ are uniformly integrable.
\end{lem}
\begin{proof}
To see the first inequality, note that for any $R \geq 0$, the function
\begin{align*}
f_R(x) &= \begin{cases}
R & \text{$x > R$} \\
x & \text{$0 \leq x \leq R$} \\
0 & \text{$x < 0$}
\end{cases}
\end{align*}
 is bounded and continuous and for fixed $x$, $f_R(x)$ is increasing
 in $R$.  The first inequality follows:
\begin{align*}
\expectation{\xi} &= \lim_{R \to \infty} \expectation{f_R(\xi)} & &
\text{by Monotone Convergence Theorem} \\
&= \lim_{R \to \infty} \lim_{n \to \infty} \expectation{f_R(\xi_n)}
& & \text{because $\xi_n \todist \xi$}\\
&\leq \liminf_n \expectation{\xi_n} & & \text{because $f_R(x) \leq x$
  for all $x \geq 0$}\\
\end{align*}
An alternative derivation is:
\begin{align*}
\expectation{\xi} &= \int \probability{\xi > \lambda} \, d \lambda & &
\text{by Lemma \ref{TailsAndExpectations}}\\
&\leq \int \liminf_n \probability{\xi_n > \lambda} \, d\lambda & &
\text{by Portmanteau Lemma \ref{PortmanteauTheorem}} \\
&\leq \liminf_n \int \probability{\xi_n > \lambda} \, d\lambda & &
\text{by Fatou's Lemma (Theorem \ref{Fatou})} \\
&= \liminf_n \expectation{\xi_n} & & \text{by Lemma \ref{TailsAndExpectations}}
\end{align*}

Now assume that $\xi_n$ is uniformly integrable.  Then by what we have
just proven and Lemma \ref{UniformIntegrabilityProperties} we have
\begin{align*}
\expectation{\xi} \leq \liminf_n \expectation{\xi_n}  \leq \sup_n
\expectation{\xi_n} < \infty
\end{align*}
So now we use the triangle inequality to write
\begin{align*}
\abs{\expectation{\xi_n} - \expectation{\xi}} &\leq
\abs{\expectation{\xi_n} - \expectation{f_R(\xi_n)}} +
\abs{\expectation{f_R(\xi_n)} - \expectation{f_R(\xi)}} + \abs{\expectation{f_R(\xi)} - \expectation{\xi}}
\end{align*}
We take the limit as $n$ goes to infinity and then as $R$ goes to
infinity and consider each term on the right side in turn.

For the first term:
\begin{align*}
\lim_{R \to \infty} \limsup_n \abs{\expectation{\xi_n} -
  \expectation{f_R(\xi_n)}} 
&= \lim_{R \to \infty} \limsup_n \left ( 
\expectation{\xi_n ; \xi_n > R} - R\probability{\xi_n > R} \right) \\
&\leq \lim_{R \to \infty} \limsup_n \expectation{\xi_n ; \xi_n > R} -
\lim_{R \to \infty} \liminf_n R\probability{\xi_n > R}  \\
&= 0
\end{align*}
where in the last line we have used uniform integrability of $\xi_n$
as well as the following
\begin{align*}
\lim_{R \to \infty} R \liminf_n \probability{\xi_n > R} &\leq \lim_{R \to \infty} R \sup_n \probability{\xi_n > R}  \leq \lim_{R \to \infty} \sup_n \expectation{\xi_n ; \xi_n > R} = 0
\end{align*}

The second term we have $\limsup_n \abs{\expectation{f_R(\xi_n)} -
  \expectation{f_R(\xi)}}  = 0$ because $f_R$ is bounded continuous
and $\xi_n \todist \xi$.  The third term we have $\lim_{R \to \infty}
\abs{\expectation{f_R(\xi)} - \expectation{\xi}} = 0$ by Monotone
Convergence.

Putting the bounds on the three terms of the right hand side together
we have $\limsup_{n \to \infty} \abs{\expectation{\xi_n} -
  \expectation{\xi}} = 0$ which by positivity shows $\lim_{n \to \infty} \abs{\expectation{\xi_n} -
  \expectation{\xi}} = 0$.

TODO:  Here is an alternative proof of the same fact by approximating $x
\characteristic{x \leq R}$ from above by continuous functions.  I
might like this proof better (since I came up with it?)

Now assume that $\lim_{n \to \infty} \expectation{\xi_n}  =
\expectation{\xi} < \infty$ and we need to show uniform integrability
of $\xi_n$.  The idea is to approximate $x \characteristic{x\geq R}$
by a continuous function so that we can use the weak convergence of
$\xi_n$.  The trick is that this function isn't bounded but is the
difference between a bounded function and the function $f(x) =x$; the
behavior of this latter function is covered by the hypothesis that the
means converge.  To make all of this precise, define the following bounded continuous function
\begin{align*}
g_R(x) &= x \wedge (R - x)_+ = \begin{cases}
0 & \text{if $x<0$ or $x > R$} \\
x & \text{if $0 \leq x \leq \frac{R}{2}$} \\
R - x & \text{if $\frac{R}{2} < x \leq R$} \\
\end{cases}
\end{align*}
and note that 
\begin{align*}
x - g_R(x) &= \begin{cases}
0 & \text{if $x<\frac{R}{2}$} \\
2x -R & \text{if $\frac{R}{2} \leq x \leq R$} \\
x & \text{if $R < x$} \\
\end{cases}
\end{align*}
so $x \characteristic{x \geq R}  \leq x - g_R(x) \leq x$, and $\lim_{R
  \to \infty}  x - g_R(x) = 0$.  Putting these facts together we see
\begin{align*}
&\lim_{R \to \infty} \limsup_{n \to \infty} \expectation{\xi_n ; \xi_n
  \geq R} \\
&\leq \lim_{R \to \infty} \limsup_{n \to \infty}
\expectation{\xi_n - g_R(\xi_n)} \\
&= \lim_{R \to \infty} \left (\lim_{n \to \infty}
\expectation{\xi_n} - \lim_{n \to \infty} \expectation{g_R(\xi_n)} \right) \\
&=\lim_{R \to \infty} \expectation{\xi} -  \expectation{g_R(\xi)} & &
\text{by assumption and $\xi_n \todist \xi$}\\
&=\lim_{R \to \infty} \expectation{\xi - g_R(\xi)} = 0 & & \text{by
  Dominated Convergence}
\end{align*}
\end{proof}

Converge in mean and convergence of means become equivalent in the
presence of almost sure convergence.
\begin{lem}\label{ConvergenceInMeanConvergenceOfMeans}Suppose $\xi,
  \xi_1, \xi_2, \dotsc$ are random variables
\begin{itemize}
\item[(i)]$\xi_n \tolp{p} \xi$ implies $\norm{\xi_n}_p \to \norm{\xi}_p$
\item[(ii)]If  $\xi_n \toas \xi$ and $\norm{\xi_n}_p \to \norm{\xi}_p$
then $\xi_n \tolp{p} \xi$
\end{itemize}
\end{lem}
\begin{proof}
To see (i), suppose $\xi_n \tolp{p} \xi$ and note that by the triangle inequality, 
\begin{align*}
\lim_{n \to \infty} \norm{\xi_n}_p &\leq \lim_{n \to \infty}
\norm{\xi_n - \xi}_p + \norm{\xi}_p = \norm{\xi}_p
\end{align*}
and
\begin{align*}
\norm{\xi}_p &\leq \lim_{n \to \infty}
\norm{\xi_n - \xi}_p + \lim_{n\to \infty} \norm{\xi_n}_p = \lim_{n\to
  \infty} \norm{\xi_n}_p
\end{align*}
therefore $\lim_{n\to  \infty} \norm{\xi_n}_p = \norm{\xi}_p$.

To see (ii), if $\xi_n \toas \xi$ and $\norm{\xi_n}_p \to \norm{\xi}_p$ then we
know that $\abs{\xi_n - \xi}^p \toas 0$ and we have the bound
\begin{align*}
\abs{\xi_n - \xi}^p &\leq \left( \abs{\xi_n} + \abs{\xi}\right )^p
\leq 2^p \max(\abs{\xi_n}^p, \abs{\xi}^p) \leq 2^p \left(
  \abs{\xi_n}^p +  \abs{\xi}^p\right)
\end{align*}
and our assumption tells us that $\lim_{n \to \infty} 2^p \expectation{
  \abs{\xi_n}^p +  \abs{\xi}^p} = 2^{p+1} \norm{\xi}_p^p < \infty$.
Therefore we can apply Dominated Convergence (Theorem \ref{DCT}) to conclude that $\lim_{n
  \to \infty} \norm{\xi_n - \xi}_p = 0$.
\end{proof}

To summarize and complete the discussion, we have the following 

TODO: Fix the statement here; this is taken from Kallenberg but it
feels imprecise to me (e.g. the equivalence of (ii) and (iii) doesn't
really require convergence in probability but only convergence in
distribution; (i) implies convergence in probability (by Markov)).
The only new content here is the extension of (ii) implies (i) to the
context of almost sure convergence to convergence in probability by
the argument along subsequences).
\begin{lem}\label{LpConvergenceUniformIntegrability}Let $\xi, \xi_1, \xi_2, \dots$ be random variables in $L^p$
  for $p > 0$ and suppose $\xi_n \toprob \xi$.  Then the following are
  equivalent:
\begin{itemize}
\item[(i)] $\xi_n \tolp{p} \xi$
\item[(ii)] $\norm{\xi_n}_p \to \norm{\xi}_p$
\item[(iii)]The sequence of random variables $\abs{\xi_1}^p,
  \abs{\xi_2}^p, \dots$ is uniformly integrable.
\end{itemize}
\end{lem}
\begin{proof}
To see (i) implies (ii), this is the first part of Lemma \ref{ConvergenceInMeanConvergenceOfMeans}.

Note that since $\xi_n \toprob \xi$ implies $\xi_n \todist \xi$ we
know that (ii) and (iii) are equivalent by Lemma \ref{UniformIntegrableAndMeans}.

To see that (ii) implies (i), suppose that $\norm{\xi_n - \xi}_p$
does not converge to zero.  They there exists an $\epsilon > 0$ and a subsequence $N^\prime
\subset \naturals$ such that $\norm{\xi_n - \xi}_p \geq \epsilon$
along $N^\prime$.  Since $\xi_n \toprob \xi$ by Lemma
\ref{ConvergenceInProbabilityAlmostSureSubsequence} there is a further
subsequence $N^{\prime \prime} \subset N^\prime$ such that $\xi_n
\toas \xi$ along $N^{\prime \prime}$.  However, Lemma
\ref{ConvergenceInMeanConvergenceOfMeans} tells us that $\norm{\xi_n -
  \xi}_p$ converges to $0$ along $N^{\prime \prime}$ which is a
contradiction.

An alternative argument is to show that (iii) implies (i) directly.
Since we have $\abs{\xi_n}^p$ is uniformly integrable and trivially the
singleton collection $\abs{\xi}^p$ is uniformly integrable, it follows from
Lemma \ref{SumsOfUniformlyIntegrable} that $\abs{\xi_n - \xi}^p$ is
uniformly integrable.  Now suppose $\epsilon>0$ is given and take $R >
\epsilon$ so that by use of convergence in probability and uniform
integrability we get
\begin{align*}
\lim_{n \to \infty} \expectation{\abs{\xi_n - \xi}^p} &= \lim_{R \to
  \infty} \limsup_{n \to \infty} \left( \expectation{\abs{\xi_n - \xi}^p;
\abs{\xi_n - \xi}^p \leq \epsilon } + \expectation{\abs{\xi_n - \xi}^p;
\epsilon < \abs{\xi_n - \xi}^p < R} + \expectation{\abs{\xi_n - \xi}^p;
\abs{\xi_n - \xi}^p \geq R} \right)\\
&\leq \epsilon + \lim_{R \to \infty} \lim_{n \to \infty} R \probability{\epsilon <
  \abs{\xi_n - \xi}^p} + 
\lim_{R \to \infty} \sup_n \expectation{\abs{\xi_n - \xi}^p;
\abs{\xi_n - \xi}^p \geq R} \\
&= \epsilon
\end{align*}
and since $\epsilon > 0$ was arbitrary, we have $\lim_{n \to \infty} \expectation{\abs{\xi_n - \xi}^p} =0$.
\end{proof}

TODO: Show how the proof of the Weak Law of Large Numbers applies to
uniformly integrable sequences not just i.i.d.

\begin{lem}$\xi_t$ is uniformly integrable if and only if there exists
  a convex and increasing $f : \reals_+ \to \reals_+$ such that
  $\lim_{x \to \infty} \frac{f(x)}{x} = \infty$ and $\sup_t
  \expectation{f(\abs{\xi_t})} < \infty$.
\end{lem}
\begin{proof}
Suppose we have $f : \reals_+ \to \reals_+$ such that $\lim_{x \to
  \infty} \frac{f(x)}{x} = \infty$ and
$\sup_t  \expectation{f(\abs{\xi_t})} < \infty$ (it doesn't have to be
increasing or convex).  Let $\epsilon > 0$ be given and pick $R > 0$
such that $\frac{f(x)}{x} \geq \frac{\sup_t \expectation{f(\abs{\xi_t})}}{\epsilon }$
for $x \geq R$.  Then for all $t \in T$,
\begin{align*}
\expectation{\abs{\xi_t}; \abs{\xi_t} \geq R} &\leq
\frac{\epsilon}{\sup_t \expectation{f(\abs{\xi_t})}}
\expectation{f(\abs{\xi_t}); \abs{\xi_t} \geq R} 
\leq \epsilon
\end{align*}
thus $\lim_{R \to \infty} \sup_t  \expectation{\abs{\xi_t};
  \abs{\xi_t} \geq R} = 0$ and uniform integrability is shown.

The key step to finding $f$ is the following observation.  Suppose we
are given an increasing $f : \reals_+ \to \reals_+$, then if we use
Lemma \ref{TailsAndExpectations} then for any positive $\xi$ we 
\begin{align*}
\expectation{f(\xi)} &= \int_0^\infty \probability{f(\xi) \geq \lambda} \, d\lambda =
\int_{f^{-1}(0)}^{f^{-1}(\infty)} \probability{\xi \geq \eta} f^\prime(\eta) \, d\eta & &
\text{letting $f(\eta) = \lambda$}
\end{align*}
so the problem of finding $f$ can be recast as finding a function $g$
such that $\int \probability{\xi \geq \eta} g(\eta) \, d\eta < \infty$
and $\lim_{x \to \infty} g(x) = \infty$.  Though the computation above
isn't rigorous since we haven't justified the change of variables in
the integral, this idea tells us that we
should assume $f$ of the form $f(x) = \int_0^x g(y) \, dy$ and for
such an $f$ we can rigorously calculate using Tonneli's Theorem
\begin{align*}
\expectation{f(\xi)} &= \expectation{\int_0^{\abs{\xi}} g(y) \, dy} =
\expectation{\int_0^\infty g(y) \characteristic{\abs{\xi} \geq y} \,
  dy} = \int_0^\infty g(y) \probability{\abs{\xi} \geq y}\,
  dy < \infty
\end{align*}
Furthermore, 
$\lim_{x \to \infty} \frac{f(x)}{x} = \infty$ by L'Hopital's Rule
(TODO: can do this without differentiation)
Moreover if $g(x)$ is increasing then we know that $f(x)$ is convex.

So our goal is to find $g(x)$ such that $\lim_{x \to \infty} g(x) =
\infty$ and $\sup_t \int_0^\infty \probability{\abs{\xi_t} \geq \eta}
g(\eta) \, d\eta < \infty$.

The existence of $g(x)$ for any positive integrable $\phi(x)$ can be
established by the following explicit construction.  Let 
\begin{align*}
g(x) &=
\frac{1}{\sqrt{\int_x^\infty \phi(x) \, dx}}
\end{align*}
and note that Dominated Convergence shows $\lim_{x \to \infty} g(x) =
\infty$ and the Fundamental Theorem of Calculus (Theorem
\ref{FundamentalTheoremOfCalculus}) shows that (TODO: this also requires the Chain Rule
which isn't trivial in this context)
\begin{align*}
g(x) \phi(x)  &= -2 \frac{d}{dx} \sqrt{\int_x^\infty \phi(x) \, dx} \\
\intertext{and therefore}
\int_0^\infty g(x) \phi(x) \, dx &= 2 \sqrt{\int_0^\infty \phi(x) \,
  dx} < \infty
\end{align*}


Now suppose that $\xi_t$ is uniformly integrable.
\begin{align*}
\expectation{\abs{\xi_t} ; \abs{\xi_t} \geq R} &= 
\int_0^\infty \probability{\abs{\xi_t} \characteristic{\abs{\xi_t}
    \geq R} \geq \lambda} \,
d\lambda \\
&= \int_R^\infty \probability{\abs{\xi_t} \geq \lambda} \,
d\lambda + 
\int_0^R \probability{\abs{\xi_t} \geq R} \, d\lambda \\
&= \int_R^\infty \probability{\abs{\xi_t} \geq \lambda} \,
d\lambda + 
R \probability{\abs{\xi_t} \geq R} 
\end{align*}
and since $\lim_{R \to \infty} \sup_t \expectation{\abs{\xi_t} ;
  \abs{\xi_t} \geq R} = 0$ we also get $\lim_{R \to \infty} \sup_t \int_R^\infty \probability{\abs{\xi_t} \geq \lambda} \,
d\lambda = 0$
which shows that if we define 
\begin{align*}
g(x) &= \frac{1}{\sqrt{\sup_t \int_x^\infty \probability{\abs{\xi_t} \geq \lambda} \,
d\lambda}} \\
\intertext{then we have}
\lim_{x \to \infty} g(x) &= \infty \\
\intertext{and moreover for any $t \in T$, }
g(x) &\leq
\frac{1}{\sqrt{\int_x^\infty \probability{\abs{\xi_t} \geq \lambda} \,
    d\lambda}}\\
\intertext{so by the previous construction we know that}
\int_0^\infty \probability{\abs{\xi_t} \geq x} g(x) \, dx 
&\leq \int_0^\infty \probability{\abs{\xi_t} \geq x} 
\frac{1}{\sqrt{\int_x^\infty \probability{\abs{\xi_t} \geq \lambda} \,
d\lambda}} \, dx < \infty
\end{align*}
TODO: Finish and address any issues related to the fact that we only
have almost everywhere differentiability of an integral in Lebesgue
theory (e.g. chain rule, u-substitution) (also is L'Hopital valid).
\end{proof}

\subsection{Topology of Weak Convergence}
We have defined convergence of a sequence of probability measures but
have skirted describing the topology underlying this notion of
convergence. An intuitively appealing approach would be to define a
metric on the space of probability measures such that two measures are
close if their values on some chosen collection of sets are close.  A
moments reflection on the Portmanteau Theorem \ref{PortmanteauTheorem}
tells us that such a condition is likely to be too strong.  For
example, if we pick a closed set $F$ then we know that it is possible
for $\mu_n \toweak \mu$ but to have $\mu(F)$ strictly larger than all
of the $\mu_n(F)$; even more precisely by considering the standard
delta mass example it is possible for $\mu(F)$ to be equal to one but
for all $\mu_n(F)$ to be zero.  

As it turns out the intuitive idea can be rescued with a small
modification.  Again thinking about the delta mass example, we can see
that while $\mu_n(F)$ remains zero for all $n$ the mass of $\mu_n$ get
arbitrary close to $F$ so that we can potentially measure how close
$\mu$ and $\mu_n$ are by looking at how much we have to \emph{thicken}
the set $F$ to capture the mass of $\mu_n$.  Generalizing we may want
to say that $\mu$ and $\nu$ are close if for every set $A$ in some
collection we don't have to thicken $A$ very much for the $\mu(A)$ and
$\nu(A)$ to be close; in fact the amount of thickening required may be a
quantitative measure of closeness.  We now proceed to make this idea precise.

\begin{defn}Given a metric space $(S,d)$ a subset $A \subset S$ and
$\epsilon > 0$ define
\begin{align*}
A^\epsilon &= \lbrace x \in S \mid \inf_{y \in A} d(x,y) < \epsilon \rbrace
\end{align*}
\end{defn} 

\begin{lem}For any set $A \subset S$ we have 
\begin{itemize}
\item[(i)]$A^\epsilon$ is an open set.
\item[(ii)]$A^\epsilon = (\overline{A})^\epsilon$.
\item[(iii)]$(A^\epsilon)^\delta \subset A^{\epsilon + \delta}$
\end{itemize}
\end{lem}
\begin{proof}
To see (i) pick an $x \in A^\epsilon$ and pick $y \in A$ such that
$d(x,y) < \epsilon$.  Then by the triangle inequality for all $z \in
S$ such that $d(x,z) < (\epsilon - d(x,y))/2$ we have
\begin{align*}
d(y,z) &\leq d(x,y) + d(x,z) < (\epsilon + d(x,y))/2 < \epsilon
\end{align*}
showing $z \in A^\epsilon$.

To see (ii), it is clear from the definition that $A^\epsilon \subset
(\overline{A})^\epsilon$ since $A \subset \overline{A}$.  To see the
opposite inclusion suppose $x \in (\overline{A})^\epsilon$ and pick $y
\in \overline{A}$ such that $d(x,y) < \epsilon$ then by density of $A$
in $\overline{A}$ pick $z \in A$ such that $d(y,z) < (\epsilon -
d(x,y))/2$.  The triangle inequality as before shows $d(z,x) < \epsilon$ and
therefore $x \in A^\epsilon$.

To see (iii) suppose $z \in (A^\epsilon)^\delta$ and pick $y \in
A^\epsilon$ such that $d(z,y) < \delta$.  Now pick $x \in A$ such that
$d(x,y) < \epsilon$ and use the triangle inequality to conclude
\begin{align*}
d(x,z) &\leq d(x,y) + d(y,z) < \epsilon + \delta
\end{align*}
\end{proof}

\begin{lem}[Levy-Prohorov Metric]\label{LevyProhorovMetric}Let $(S,d)$ be a metric space and let $\mathcal{P}(S)$
  denote the set of probability measures.  Define 
\begin{align*}
\rho(\mu, \nu) &= \inf \lbrace \epsilon > 0 \mid \mu(F) \leq
\nu(F^\epsilon) + \epsilon \text{ for all closed subsets } F \subset S \rbrace
\end{align*}
Then in fact 
\begin{align*}
\rho(\mu, \nu) &= \inf \lbrace \epsilon > 0 \mid \mu(A) \leq
\nu(A^\epsilon) + \epsilon \text{ for all Borel  subsets } A \subset S \rbrace
\end{align*}
and furthermore $\rho$ is a metric on $\mathcal{P}(S)$.
\end{lem}
\begin{proof}
Claim 1: For every $\alpha, \beta > 0$ if $\mu(F) \leq \nu(F^\alpha) +
\beta$ for all closed subsets $F \subset S$ then $\nu(F) \leq \mu(F^\alpha) +
\beta$ for all closed subsets $F \subset S$.

The proof of the claim relies on the observation that $F \subset
(((F^\alpha)^c)^\alpha)^c$.
To see the observation note that if $x \in F$ and $x \in
((F^\alpha)^c)^\alpha$ then we can find $y \notin F^\alpha$ such that
$d(x, y)  < \alpha$ which contradicts the definition of $F^\alpha$.
The claim follows by using inclusion in the observation in addition to
the fact that $F^\alpha$ is open (hence
$(F^\alpha)^c$ is closed) so
\begin{align*}
\nu(F) &\leq \nu((((F^\alpha)^c)^\alpha)^c) = 1 -
\nu(((F^\alpha)^c)^\alpha) \leq 1 - \mu((F^\alpha)^c) + \beta =
\mu(F^\alpha) + \beta
\end{align*}

With Claim 1 in hand symmetry of $\rho$ now follows as the sets
$\lbrace \epsilon > 0 \mid \mu(F) < \nu(F^\epsilon) + \epsilon 
\rbrace$ and $\lbrace \epsilon > 0 \mid \nu(F) < \mu(F^\epsilon) + \epsilon
\rbrace$ are equal a fortiori the infimum are equal.

Clearly by continuity of measure (Lemma \ref{ContinuityOfMeasure}) we
have $\mu(A) = \lim_{\epsilon \to 0} \mu(A^\epsilon)$ and therefore
$\rho(\mu, \mu) = 0$.  Conversely if $\rho(\mu, \nu) = 0$ then we pick
a closed set $F$ and for every $\epsilon > 0$ we have $\mu(F) <
\nu(F^\epsilon) + \epsilon$.  Again using continuity of measure we can
conclude that $\mu(F) \leq \nu(F)$.  By symmetry of $\rho$ that we
have already proven we can conclude $\nu(F) \leq \mu(F)$ and there
$\mu(F) = \nu(F)$ for all closed set $F \subset S$.  Since closed sets
are a $\pi$-system that generate the Borel subsets of $S$ we conclude
that $\mu = \nu$ by a monotone class argument (Lemma
\ref{UniquenessOfMeasure}).

To show the triangle inequality let $\mu, \nu, \zeta$ be probability
measures and suppose $\epsilon > 0$ is such that $\mu(F) <
\nu(F^\epsilon) + \epsilon$ and $\delta > 0$ is such that $\nu(F) <
\zeta(F^\delta) + \delta$ for all closed sets $F \subset S$.  Now
choose a particular $F \subset S$ be closed then
\begin{align*}
\mu(F) 
&\leq \nu(F^\epsilon) + \epsilon 
\leq \nu(\overline{F^\epsilon})+ \epsilon 
\leq \zeta((\overline{F^\epsilon})^\delta)+ \epsilon + \delta \\
&=\zeta((F^\epsilon)^\delta)+ \epsilon + \delta 
\leq \zeta(F^{\epsilon+\delta})+ \epsilon + \delta
\end{align*}
Thus $\rho(\mu, \zeta) \leq \rho(\mu, \nu) + \rho(\nu, \zeta)$.  TODO:
This last conclusion is more or less obvious but there are some minor details that
could be filled in here.
\end{proof}

\chapter{Lindeberg's Central Limit Theorem}

The Law of Large Numbers tells us that when we are given i.i.d. random
variables $\xi_i$ with finite expectation, we have almost sure
convergence of $\frac{1}{n} \sum_{k=1}^n \xi_k =
\expectation{\xi_i}$.  Using different notation we can say, 
\begin{align*}
\sum_{k=1}^n \xi_k - n \expectation{\xi_i}  = o(n)
\end{align*}
From one point of view, the Central Limit Theorem arises from asking
the question about whether $o(n)$ can be replaced by $o(n^p)$ or
$\mathcal{O}(n^p)$ for $p < 1$.  In this sense the Central Limit
Theorem gives some information about the rate of convergence of the sums
$\frac{1}{n} \sum_{k=1}^n \xi_k$ to their limit.

First some intuition about the Central Limit Theorem.  Let's assume
that we have a sequence of i.i.d. random variables $\xi_i$ such that
$\xi_i$ has moments of all orders (a much stronger assumption than one
needs for the CLT).  We also assume 
\begin{align*}
\expectation{\xi_i} = 0, \expectation{\xi_i^2} =1
\end{align*}
Consider the following computation of the moments
of the partial sums of $\xi_i$.  Let $S_n = \xi_1 + \cdots + \xi_n$.
\begin{align*}
\expectation{S_n^{m+1}} &= \expectation{(\xi_1
  + \cdots + \xi_n) (\xi_1
  + \cdots + \xi_n)^{m}} \\
&= \sum_{i=1}^n \expectation{\xi_i (\xi_n + S_{n-1})^m }\\
&= n \expectation{\xi_n (\xi_n + S_{n-1})^m} \textrm{ TODO: don't know how
  to prove this step}\\
&= n \sum_{j=0}^m \binom{m}{j} \expectation{\xi_n^{j+1}}
\expectation{S_{n-1}^{m-j}} \\
&= n m \expectation{S_{n-1}^{m-1}} + n \sum_{j=2}^m \binom{m}{j} \expectation{\xi_n^{j+1}}
\expectation{S_{n-1}^{m-j}} \\
\end{align*}
Now define $\hat{S}_n = S_n/\sqrt{n}$, and divide both sides of the
above by $n^{\frac{m+1}{2}}$ and we see
\begin{align*}
\expectation{\hat{S}_n^{m+1}} = m \expectation{\hat{S}_n^{m-1}} + \sum_{j=2}^m \binom{m}{j} \frac{1}{n^\frac{j-1}{2}}\expectation{\xi_n^{j+1}}
\expectation{\hat{S}_{n-1}^{m-j}}
\end{align*}
An induction on $m$ together with the observation that $\expectation{\hat{S}_n^0} =
1$ and $\expectation{\hat{S_n}} = 0$ shows that 
\begin{align*}
\lim_{n \to \infty} \expectation{\hat{S}_n^{2m+1}}&=0 \\
\lim_{n \to \infty} \expectation{\hat{S}_n^{2m}}&=\prod_{j=1}^m (2j-1)
= \frac{(2m)!}{2^m m!}\\
\end{align*}
We can recognize that these are the moments of the standard normal
distribution.

The above argument is one path to use to see how Gaussian
distributions might arise when looking at sums of i.i.d random
variables but relies on an unecessarily strong set of assumptions (not
to mention it ignores the fact that moments alone to not characterize
a distribution).

In fact convergence to normal distributions is more general than
i.i.d. variables and we look for a version that has a rather precise set of
assumptions called the Lindeberg conditions.  The statement of the
result and the corresponding notation is unwieldy but the proof itself
doesn't seem to suffer much from the added complexity.  Furthermore
the added generality provides a useful space to explore when examining the limits of asymptotic normality.

\begin{thm}[Lindeberg]\label{LindebergTheorem}Let $\xi_1, \xi_2, \dots$ be independent square
  integrable random variables $\expectation{\xi_m} = 0$ and
  $\expectation{\xi_m^2} = \sigma_m^2 > 0$.  Define
\begin{align*}
S_n &= \sum_{i=1}^n \xi_i \\
\Sigma_n &= \sqrt{\sum_{i=1}^n \sigma_i^2}\\
\hat{S}_n &= \frac{S_n}{\Sigma_n} \\
r_n &= \max_{1 \leq i \leq n} \frac{\sigma_i}{\Sigma_n} \\
g_n(\epsilon) &= \frac{1}{\Sigma_n^2} \sum_{i=1}^n \expectation{\xi_i^2
  \characteristic{|\xi_i| \geq \epsilon \Sigma_n}}
\end{align*}
and let $d \gamma = \frac{1}{\sqrt{2\pi}} e^{\frac{-x^2}{2}} dx$ be the
  distribution of an $N(0,1)$ random variable.  For all $\epsilon > 0$, $\varphi \in C^3(\reals;\reals)$ with
bounded 2nd and 3rd derivative,
\begin{align*}
\left \lvert \expectation{\varphi(\hat{S_n})} - \int_\reals \varphi
  d\gamma \right \rvert \leq
\left ( \frac{\epsilon}{6} + \frac{r_n}{2}\right ) \norm{\varphi^{\prime\prime\prime}}_\infty + g_n(\epsilon) \norm{\varphi^{ \prime
\prime}}_\infty
\end{align*}
and 
\begin{align*}
r_n^2 \leq \epsilon^2 + g_n(\epsilon)
\end{align*}
In particular, if $\lim_{n \to \infty} g_n(\epsilon) = 0$ for every
$\epsilon > 0$, then 
\begin{align*}
\lim_{n \to \infty} \left \lvert \expectation{\varphi(\hat{S_n})} - \int_\reals \varphi
  d\gamma \right \rvert = 0
\end{align*}
\end{thm}
Before attacking the proof we note how everything specializes in the
case of i.i.d. random variables.  In this case $\Sigma_n = \sqrt{n}
\sigma$, $\hat{S}_n = \frac{\sum_{i=1}^n \xi_i}{\sqrt{n} \sigma}$ and
$g_n(\epsilon) = \frac{1}{\sigma^2} \expectation{\xi^2 ; \abs{\xi}
  \geq \epsilon \sqrt{n} \sigma}$.  Because $\expectation{\xi^2}
< \infty$ we know that $\xi^2 < \infty$ a.s.  and we have
$\xi^2 \characteristic{\abs{\xi}  \geq \epsilon \sqrt{n} \sigma}
\toas 0$.  Noting $\xi^2 \characteristic{\abs{\xi}  \geq \epsilon \sqrt{n}
  \sigma} \leq \xi^2$, Dominated Convergence tells us that $\lim_{n
\to \infty} g_n(\epsilon) = 0$.

This special case also sheds some light on aspects of the hypotheses.
For example, the $\sqrt{n}$ in the denominator is the only possible
choice to acheive convergence to a random variable with finite
non-zero variance; it is precisely the term requires to make
$\sigma(\hat{S}_n)$ converge to a finite non-zero number (in fact in
the i.i.d. case it makes the sequence constant).  

It is also worth spending some time understanding the nature of
$g_n(\epsilon)$.  First, it is clear from independence and definitions that 
\begin{align*}
\expectation{\hat{S}_n^2} &= \sum_{i=1}^n \expectation{\left
    (\frac{\xi_i}{\Sigma_n} \right )^2} = \frac{1}{\Sigma_n^2}
\sum_{i=1}^n \sigma_i^2 = 1
\end{align*}
but we can also write 
\begin{align*}
g_n(\epsilon) &= \frac{1}{\Sigma_n^2} \sum_{i=1}^n \expectation{\xi_i^2
  \characteristic{|\xi_i| \geq \epsilon \Sigma_n}} = \sum_{i=1}^n
\expectation{\left (\frac{\xi_i}{\Sigma_n}\right )^2
  ; \abs{\frac{\xi_i}{\Sigma_n}} \geq \epsilon}
\end{align*}
So the $\hat{S}_n$ is the sum of $\xi_i$ normalized to maintain a
constant unit variance.  Our assumption that $\lim_{n \to \infty}
g_n(\epsilon) = 0$ is an assertion that in the limit, all of that unit
variance is contained in a bounded region around $0$.  In the
i.i.d. case that is clearly true because all of the unscaled $\xi_n$
have their ``energy'' in a constant fashion, so rescaling is able to
concentrate that energy arbitrarily close to $0$.  It is permissible
to have the energy of the $\xi_n$ moving off to infinity but only if
it travels at a rate less than $\sqrt{n}$.

TODO: Question is it possible to satisfy the Lindeberg condition when
$\lim_{n \to \infty} \Sigma_n < \infty$?

\begin{proof}
Fix an $n >0$ and define $\hat{\xi}_m = \frac{\xi_m}{\Sigma_n}$ and
$\hat{S}_n = \hat{\xi}_1 + \cdots + \hat{\xi}_n$.  Note that
$\expectation{\hat{S}_n^2} = 1$.  Let $\eta_1, \eta_2, \dots$ be independent $N(0,1)$ random variables
that are also independent of the $\xi_i$.  Note that we may have to
extend $\Omega$ in order to arrange this (e.g. extend by $[0,1]$ and
use Theorem \ref{ExistenceCountableIndependentRandomVariables}).  We
rescale each $\eta_i$ so that it has the same variance as
$\hat{\xi_i}$; define $\hat{\eta}_i =
\frac{\sigma_i \eta_i}{\Sigma_n}$ and $\hat{T}_n = \hat{\eta}_1 +
\cdots + \hat{\eta}_n$.  Notice that
$\expectation{\hat{\eta}_m^2} =
\expectation{\hat{\xi}_m^2} = \frac{\sigma_m^2}{\Sigma_n^2}$ and $\hat{T}_n$ is also a $N(0,1)$
random variable.  Therefore, by the Expectation Rule (Lemma
\ref{ExpectationRule}) $\int \varphi \, d\gamma =
\expectation{\varphi(\hat{T}_n)}$ and we can write 
\begin{align*}
\left \lvert \expectation{\varphi(\hat{S_n})} - \int_\reals \varphi
  d\gamma \right \rvert = \left \lvert \expectation{\varphi(\hat{S_n})} - \expectation{\varphi(\hat{T_n})} \right \rvert
\end{align*}
By having arranged for $\hat{\xi}_i$ and $\hat{\eta}_i$ to have same
first and second moments so one should be thinking that we have
constructed a ``second order approximation''.  TODO: What is critical
is that the approximation of the individual $\hat{\xi}_i$ may not be a
good one, the approximation $\hat{S_n}$ by $\hat{T_n}$ is a good one.
Find the critical point(s) in the proof where this comes to light.

The real trick of the proof is to interpolate between
$\varphi(\hat{S_n})$ and $\varphi(\hat{T_n})$ by exchanging
$\hat{\xi}_i$ and $\hat{\eta}_i$ one summand at a time.  By varying
only one summand we will then be able use Taylor's Theorem to
estimate the differences between the terms.  Concretely we write,
\begin{align*}
\varphi(\hat{S_n}) - \varphi(\hat{T_n}) &= \varphi(\hat{\xi}_1 + \cdots + \hat{\xi}_n) -
\varphi(\hat{\eta}_1 + \cdots + \hat{\eta}_n) \\
&= \varphi(\hat{\xi}_1 + \cdots + \hat{\xi}_n) - \varphi(\hat{\eta}_1 +
\hat{\xi}_2 + 
\cdots + \hat{\xi}_n) \\
&+ \varphi(\hat{\eta}_1 +
\hat{\xi}_2 +
\cdots + \hat{\xi}_n) - 
\varphi(\hat{\eta}_1 + \hat{\eta}_2 + \hat{\xi}_3 + \cdots + \hat{\xi}_n) \\
&+ \cdots\\
&+ \varphi(\hat{\eta}_1 +
\cdots  +
\hat{\eta}_{n-1} + \hat{\xi}_n) - 
\varphi(\hat{\eta}_1 + \cdots + \hat{\eta}_n) \\
\end{align*}
Since we have to manipulate these terms a bit, it helps to clean up
the notation by defining:
\begin{align*}
U_m &= \begin{cases}
\hat{\xi}_2 + \cdots + \hat{\xi}_n & \text{if $m=1$} \\
\hat{\eta}_1 + \cdots + \hat{\eta}_{m-1}+\hat{\xi}_{m+1} + \cdots +
\hat{\xi}_n & \text{if $1 < m<n$} \\
\hat{\eta}_1 + \cdots + \hat{\eta}_{n-1} & \text{if $m=n$} \\
\end{cases}
\end{align*}
and then we can write the above interpolation as 
\begin{align*}
\varphi(\hat{S_n}) - \varphi(\hat{T_n}) &= \sum_{m=1}^n \varphi(U_m + \hat{\xi}_m) -\varphi(U_m + \hat{\eta}_m)
\end{align*}
Now we can take absolute values, use the triangle inequality and use
linearity of expectation to see
\begin{align*}
\abs{\expectation{\varphi(\hat{S_n}) - \varphi(\hat{T_n})}} &\leq
\sum_{m=1}^n \abs{\expectation{\varphi(U_m + \hat{\xi}_m)}
  -\expectation{\varphi(U_m + \hat{\eta}_m)}}\\
&=\sum_{m=1}^n \abs{\expectation{\varphi(U_m + \hat{\xi}_m)
  -\varphi(U_m + \hat{\eta}_m)}}
\end{align*}

Now we focus on each term $\varphi(U_m + \hat{\xi}_m) -\varphi(U_m +
\hat{\eta}_m)$ by applying Taylor's Formula (Theorem
\ref{TaylorsTheorem}) to see 
\begin{align*}
\varphi(U_m + x) = \varphi(U_m) + x \varphi^\prime(U_m) +
\frac{x^2}{2} \varphi^{\prime\prime}(U_m) + R_m(x)
\end{align*}
where
\begin{align*}
R_m(x) &=  \int_{U_m}^{U_m+x}
\frac{(U_m + x - t)^2}{2} \varphi^{\prime \prime \prime}(t) \, dt
\end{align*}

For example,applying this expansion with $x = \hat{\xi}_m$, using
linearity of expectation, independence of
$\hat{\xi}_m$ and $U_m$ and Lemma \ref{IndependenceExpectations} we get
\begin{align*}
\expectation{\varphi(U_m + \hat{\xi}_m) } &= \expectation{
  \varphi(U_m)+ \hat{\xi}_m \varphi^\prime(U_m) +
\frac{\hat{\xi}_m^2}{2} \varphi^{\prime\prime}(U_m) +
R_m(\hat{\xi}_m)} \\
&=\expectation{
  \varphi(U_m)} + \frac{\sigma_m^2}{2\Sigma_n^2}\expectation{\varphi^{\prime\prime}(U_m)} + \expectation{R_m(\hat{\xi}_m)}
\end{align*}
and in exactly the same way because we have arrange for $\hat{\xi}_m$
and $\hat{\eta}_m$ to share the first two moments, we get
\begin{align*}
\expectation{\varphi(U_m + \hat{\eta}_m) } &=\expectation{
  \varphi(U_m)} + \frac{\sigma_m^2}{2\Sigma_n^2}\expectation{\varphi^{\prime\prime}(U_m)} + \expectation{R_m(\hat{\eta}_m)}
\end{align*}
Thus, $\expectation{\varphi(U_m + \hat{\xi}_m)
  -\varphi(U_m + \hat{\eta}_m)} = \expectation{R_m(\hat{\xi}_m)} -
\expectation{R_m(\hat{\eta}_m)}$ and 
\begin{align*}
\abs{\expectation{\varphi(\hat{S_n}) - \varphi(\hat{T_n})}} &\leq 
\sum_{m=1}^n \abs{\expectation{R_m(\hat{\xi}_m) }} +  \sum_{m=1}^n \abs{\expectation{R_m(\hat{\eta}_m) }}
\end{align*}

We complete the proof by bounding each expectation above.  On the one hand, there is the Lagrange Form for the
remainder term (Lemma \ref{LagrangeFormRemainder}) that shows that $R_m(x) =
\varphi^{\prime\prime\prime}(c) \frac{x^3}{6}$ for some $c \in [U_m,
U_m+x]$ hence $\abs{R_m(x)} \leq
\norm{\varphi^{\prime\prime\prime}}_\infty \frac{\abs{x}^3}{6}$. On
the other hand, sticking with the integral form of the remainder term, since $t \in [U_m, U_m + x]$ we can bound the term $(U_m + x - t)^2 \leq \abs{x}^2$ in
the integral and integrate to conclude 
\begin{align*}
\abs{R_m(x)} &=  \int_{U_m}^{U_m+x}
\frac{(U_m + x - t)^2}{2} \varphi^{\prime \prime \prime}(t) \, dt \leq
\frac{\abs{x}^2}{2} \int_{U_m}^{U_m+x}
\varphi^{\prime \prime \prime}(t) \, dt \\
&=\frac{\abs{x}^2}{2} \left(\varphi^{\prime \prime}(U_m+x) - \varphi^{\prime \prime}(x)\right) \leq \norm{\varphi^{\prime\prime}}_\infty \abs{x}^2
\end{align*}

With this setup, pick $\epsilon >0$ and first consider the remainder term $R_m(\hat{\xi}_m) $
and a note that we have to be a little careful.  We would like to use
the stronger $3^{rd}$ moment bound however we have not
assumed that $\hat{\xi}_m$ has a finite $3^{rd}$ moment.  So what we
do is truncate $\hat{\xi}_m$ and take a $2^{nd}$ moment bound over the
tail (valid because of the finite variance assumption) and use a
$3^{rd}$ moment bound on the truncated $\hat{\xi}_m$.  The details follow:
\begin{align*}
\abs{\expectation{R_m(\hat{\xi}_m) }} &\leq
\abs{\expectation{R_m(\hat{\xi}_m) ; \abs{\hat{\xi}_m} \leq \epsilon}}
+ \abs{\expectation{R_m(\hat{\xi}_m) ; \abs{\hat{\xi}_m} > \epsilon}}
\end{align*}
We take the sum of first terms and apply the Taylor's formula bound to see 
\begin{align*}
\sum_{m=1}^n \abs{\expectation{R_m(\hat{\xi}_m) ; \abs{\hat{\xi}_m} \leq
    \epsilon}}
&\leq
\frac{\norm{\varphi^{\prime\prime\prime}}_\infty}{6} \sum_{m=1}^n \abs{\expectation{\abs{\hat{\xi}_m}^3
    ; \abs{\hat{\xi}_m} \leq \epsilon}} \\
&\leq
\epsilon \frac{\norm{\varphi^{\prime\prime\prime}}_\infty}{6} \sum_{m=1}^n
\abs{\expectation{\abs{\hat{\xi}_m}^2}} \\
&=
\epsilon \frac{\norm{\varphi^{\prime\prime\prime}}_\infty}{6} \sum_{m=1}^n
\frac{\sigma_m^2}{\Sigma_n^2} = \epsilon \frac{\norm{\varphi^{\prime\prime\prime}}_\infty}{6} \\
\end{align*}
Next take the sum of the second terms to see
\begin{align*}
\sum_{m=1}^n \abs{\expectation{R_m(\hat{\xi}_m) ; \abs{\hat{\xi}_m} >
    \epsilon}}
&\leq
\norm{\varphi^{\prime\prime}}_\infty\sum_{m=1}^n \abs{\expectation{\abs{\hat{\xi}_m}^2
    ; \abs{\hat{\xi}_m} > \epsilon}} \\
&=
\norm{\varphi^{\prime\prime}}_\infty \frac{1}{\Sigma_n^2}\sum_{m=1}^n \abs{\expectation{\abs{\xi_m}^2
    ; \abs{\xi_m} > \epsilon \Sigma_n}} \\
&= \norm{\varphi^{\prime\prime}}_\infty g_\epsilon(n)
\end{align*}
Lastly, to bound the remainder term on $\hat{\eta}_m$ we can directly
appeal to the $3^{rd}$ moment bound because as a normal random
variable $\hat{\eta}_m$ has finite moments of all orders:
\begin{align*}
\sum_{m=1}^n \abs{\expectation{R_m(\hat{\eta}_m) }}
&\leq
\frac{\norm{\varphi^{\prime\prime\prime}}_\infty}{6} \sum_{m=1}^n
\abs{\expectation{\abs{\hat{\eta}_m}^3}} \\
&= \frac{\norm{\varphi^{\prime\prime\prime}}_\infty}{6} \sum_{m=1}^n \frac{\sigma_m^3}{\Sigma_n^3}
\abs{\expectation{\abs{\eta_m}^3}} \\
&\leq \frac{r_n \norm{\varphi^{\prime\prime\prime}}_\infty}{6} \sum_{m=1}^n \frac{\sigma_m^2}{\Sigma_n^2}
\abs{\expectation{\abs{\eta_m}^3}} \\
&= \frac{r_n \norm{\varphi^{\prime\prime\prime}}_\infty}{6}
\frac{2\sqrt{2}}{\sqrt{\pi}} < \frac{r_n \norm{\varphi^{\prime\prime\prime}}_\infty}{2}
\end{align*}

TODO: We used a calculation of the $3^{rd}$ absolute moment of the
standard normal distribution ($\frac{2\sqrt{2}}{\sqrt{\pi}}$).  We need to record that calculation somewhere.

The last thing to show is the bound on $r_n^2$.  For each $n >0$ and
$1 \leq m \leq n$,
\begin{align*}
\frac{\sigma_m^2}{\Sigma_n^2} &= \frac{1}{\Sigma_n^2} \left(
  \expectation{\xi_m^2 ; \abs{\xi_m} < \epsilon \Sigma_n} +
  \expectation{\xi_m^2 ; \abs{\xi_m} \geq \epsilon \Sigma_n} \right) \\
&\leq \frac{1}{\Sigma_n^2} \left(\epsilon^2 \Sigma_n^2 + \Sigma_n^2
  g_n(\epsilon) \right) = \epsilon^2 + g_n(\epsilon)
\end{align*}
hence $r_n^2 = \max_{1\leq m \leq n} \frac{\sigma_m^2}{\Sigma_n^2}
\leq \epsilon^2 + g_n(\epsilon)$.
\end{proof}

Note that the Lindeberg condition is a sufficient condition but not a
necessary condition for convergence to a normal distribution; but is
not too far off.  Thus it is useful to examine a case in which we
don't satisfy the condition.
\begin{examp}[Failure of Lindeberg Condition]Let $\xi_n$ be a sequence
  of independent random
  variables such that $\xi_n = n$ with probability $\frac{1}{2n^2}$, $\xi_n =
  -n$ with probability $\frac{1}{2n^2}$ and $\xi_n = 0$ with
  probability $1-\frac{1}{n^2}$.  Note that $\variance{\xi_n} =
  (-n)^2 \cdot \frac{1}{2n^2} + 0 \cdot (1 - \frac{1}{2n^2}) +  n^2 \cdot
  \frac{1}{2n^2} = 1$.  
$\sum_{n=1}^\infty
  \probability{\xi_n \neq 0} = \sum_{n=1}^\infty \frac{1}{n^2} <
  \infty$ so by Borel Cantelli, we have $\xi_n$ are eventually $0$
  a.s.; hence $S_n = \sum_{i=1}^n \xi_i$ is bounded a.s. and $\lim_{n
    \to \infty} \frac{S_n}{\sqrt{n}} = 0$ a.s.  Therefore,
  $\frac{S_n}{\sqrt{n}}$ does not converge to a Gaussian in
  distribution.

We know that $\xi_n$ must not satisfy the Lindeberg condition and it
is instructive to perform that calculation explicitly.  Using the
notation of Theorem \ref{LindebergTheorem}, $\Sigma_n = \sqrt{n}$,
thus for any $\epsilon >
0$, and $n > \epsilon^2$, we have 
\begin{align*}
\xi_n \cdot \characteristic{\abs{\xi_n} >
  \epsilon \Sigma_n} &= \xi_n \cdot \characteristic{\abs{\xi_n} >
  \epsilon \sqrt{n}} = \xi_n
\end{align*}
so only a finite number of summands of $\expectation{ \xi_n^2 ;
  \abs{\xi_n} >  \epsilon \sqrt{n}}$ are different from $1$, hence
\begin{align*}
\lim_{n \to \infty} g_n(\epsilon) &=\lim_{n \to \infty} \frac{1}{n} \sum_{i=1}^n \expectation{ \xi_n^2 ;
  \abs{\xi_n} >  \epsilon \sqrt{n}} = 1
\end{align*}
\end{examp}

TODO: Mention Feller-Lindeberg Theorem that adds an addition
hypothesis that makes the Lindeberg condition equivalent to
asymptotic normality.

The Lindeberg Theorem above doesn't actually prove weak convergence
because of the differentiability assumption on the function $\varphi$.
Our next step is to use approximation arguments to show that we in
fact get weak convergence.  The argument has broader applicability
than the Central Limit Theorem and is just a validation that proving weak
convergence for random vectors only requires use compactly supported
smooth test functions.
\begin{lem}\label{WeakConvergenceWithSmoothTestFunctions}Let $\xi, \xi_1, \xi_2, \dots$ be random vectors in $\reals^N$,
  then $\xi_n \todist \xi$ if and only if $\lim_{n \to \infty}
  \expectation{f(\xi_n)} = \expectation{f(\xi)}$ for all $f \in
  C^\infty_c(\reals^N; \reals)$.
\end{lem}
\begin{proof}
Since any $f \in  C^\infty_c(\reals^N; \reals)$ is bounded we
certainly see that $\xi_n \todist \xi$ implies $\lim_{n \to \infty}
  \expectation{f(\xi_n)} = \expectation{f(\xi)}$.

In the other direction, take an arbitrary $f \in C_b(\reals^N;
\reals)$ and pick $\epsilon > 0$.  By Lemma \ref{ApproximationByMollifiers}, we can find $f_n
\in C^\infty_c(\reals^N; \reals)$ such that $f_n$ converges uniformly
on compact sets and $\norm{f_n}_\infty \leq \norm{f}_\infty$.  
The idea of the proof is to note that for any
$n, k \geq 0$, we have
\begin{align*}
\abs{\expectation{f(\xi_n) - f(\xi)}} &\leq \abs{\expectation{f(\xi_n)
    - f_k(\xi_n)}} + \abs{\expectation{f_k(\xi_n)    - f_k(\xi)}} + \abs{\expectation{f_k(\xi) - f(\xi)}}
\end{align*}
and then to bound each term on the right hand side.  The second term
will be easy to handle because of our hypothesis and the smoothness of
$f_k$.   The first and
third terms will require that we examine the approximation provided by
the uniform convergence of the $f_k$ on all compact sets.


The first task we have
is to pick that compact set; it turns out that it suffices to consider
closed balls centered at the origin.  For any $R \in \reals$ with $R>0$,
 there exists a $\psi_R \in C^\infty_c(\reals^N;\reals)$
with $\characteristic{\abs{x} \leq \frac{R}{2}} \leq \psi_R(x) \leq
\characteristic{\abs{x} \leq R}$,
therefore 
\begin{align*}
\lim_{n \to \infty} \probability{\abs{\xi_n} > R} &= 1 - \lim_{n \to
  \infty} \expectation{\characteristic{\abs{\xi_n} \leq R}} \\
&\leq 1 - \lim_{n \to
  \infty} \expectation{\psi_R(\xi_n)} \\
&=1 - \expectation{\psi_R(\xi)} \\
&\leq 1 - \expectation{\characteristic{\abs{\xi} \leq \frac{R}{2}}} \\
&=\probability{\abs{\xi} > \frac{R}{2}} 
\end{align*}
On the other hand, we know that $\lim_{R \to \infty}
\characteristic{\abs{\xi} \leq \frac{R}{2}} = 0$ a.s. and therefore by
Monotone Convergence, $\lim_{R \to \infty} \probability{\abs{\xi} >
  \frac{R}{2}} = 0$.  Select $R > 0$ such that  
\begin{align*}
\probability{\abs{\xi} >R} &\leq \probability{\abs{\xi} >\frac{R}{2}} \leq \frac{\epsilon}{4\norm{f}_\infty}
\end{align*}
Then we can
pick $N_1 > 0$ such that $\probability{\abs{\xi_n} >R} \leq \frac{\epsilon}{2\norm{f}_\infty}$ for all $n > N_1$.

Having picked $R>0$, we know that $f_n$ converges uniformly to $f$ on
$\abs{x} \leq R$ and therefore we can find a $K > 0$ such that for $k
> K$ and $\abs{x} \leq R$  we have $\abs{f_k(x) - f(x)} < \epsilon$.
Therefore,
\begin{align*}
\abs{\expectation{f_k(\xi) - f(\xi)}} &\leq \expectation{\abs{f_k(\xi)
    - f(\xi)} ; \abs{\xi} \leq R} + \expectation{\abs{f_k(\xi)
    - f(\xi)} ; \abs{\xi} > R} \\
&\leq \epsilon \probability{ \abs{\xi} \leq R} + 2\norm{\xi}_\infty \probability{ \abs{\xi} > R} \\
&\leq \epsilon + \frac{\epsilon}{2} < 2\epsilon
\end{align*}
and via the same calculation, for $n > N_1$
\begin{align*}
\abs{\expectation{f_k(\xi_n) - f(\xi_n)}} &\leq \epsilon + 2\norm{\xi_n}_\infty \probability{ \abs{\xi_n} > R} \leq 2\epsilon
\end{align*}

To finish the proof, pick a single $k > K$ and then we can find $N_2 >
0$ such that for all $n > N_2$, we have $\abs{\expectation{f_k(\xi_n)
    - f_k(\xi)}} < \epsilon$.  Putting these three estimates together,
we have for $n > \max(N_1, N_2)$, 
\begin{align*}
\abs{\expectation{f(\xi_n) - f(\xi)}} &\leq 5 \epsilon
\end{align*}
\end{proof}

We are not going to prove the following but we should talk about it:
\begin{thm}\label{Berry-Esseen Theorem}Let $\xi, \xi_1, \xi_2, \dots$
  be i.i.d with $\expectation{\abs{\xi}^3} < \infty$.  Let $\Phi(x)$
  be the cdf of standard normal and let $G(x) = \probability{\frac{S_n
      - \mu}{\sigma\sqrt{n}} \leq x}$ be the empirical cdf.  Then
  there exists a constant $C > 0$ such that
\begin{align*}
\sup_x \abs{G(x) - \Phi(x)} \leq \frac{C
  \expectation{\abs{\xi}^3}}{\sigma^3 \sqrt{n}}
\end{align*}
\end{thm}
Note the upper bound of the constant $C$ has been reduced to about
$0.5600$.  

The proof of the Lindeberg Theorem \ref{LindebergTheorem} above extends to a more general
situation which (somewhat surprisingly) is required for applications.  We state the the Theorem here and encourage the 
motivated reader to prove it herself by closely following the proof of Theorem \ref{LindebergTheorem}.    On the other hand
the reader that is anxious to move onto new topics will lose nothing by skipping this topic and returning to it when it is
required later on.
\begin{defn}A \emph{triangular array} of random variables is for each $n \in \naturals$
\begin{itemize}
\item [(i)] a probability space $(\Omega_n, \mathcal{A}_n, P_n)$ 
\item [(ii)] an natural number $m_n \in \naturals$ and random variables $\xi_{n1}, \dotsc, \xi_{n m_n}$ defined on  $(\Omega_n, \mathcal{A}_n, P_n)$ 
\end{itemize}
such that
\begin{itemize}
\item[(i)] for each $n \in \naturals$ the $\xi_{n1}, \dotsc, \xi_{n m_n}$ are independent
\item[(ii)] $\expectation{\xi_{ni}} = 0$ for each $n \in \naturals$ and $1 \leq i \leq m_n$
\item[(iii)] $\sum_{i=1}^{m_n} \expectation{\xi_{ni}^2} > 0$ for each $n \in \naturals$
\end{itemize}
\end{defn}

The Lindeberg Central Limit Theorem for triangular arrays follows
\begin{thm}[Triangular Arrays]\label{LindebergTheoremTriangular}  Let $\xi_{ni}$ be a triangular array and define
\begin{align*}
S_n& = \sum_{i=1}^{m_n} \xi_{ni},&
\sigma_{ni}& = \sqrt{\expectation{\xi_{ni}^2}},& 
\Sigma_n& = \sqrt{\sum_{i=1}^{m_n} \sigma_{ni}^2}, \\
\hat{S}_n& = \frac{S_n}{\Sigma_n},&
r_n& = \max_{1 \leq i \leq m_n} \frac{\sigma_{ni}}{\Sigma_n},&
g_n(\epsilon)& = \frac{1}{\Sigma_n^2} \sum_{i=1}^{m_n} \expectation{\xi_{ni}^2 \characteristic{|\xi_{ni}| \geq \epsilon \Sigma_n}}
\end{align*}
and let $d \gamma = \frac{1}{\sqrt{2\pi}} e^{\frac{-x^2}{2}} dx$ be the
  distribution of an $N(0,1)$ random variable.  For all $\epsilon > 0$, $\varphi \in C^3(\reals;\reals)$ with
bounded 2nd and 3rd derivative,
\begin{align*}
\left \lvert \expectation{\varphi(\hat{S_n})} - \int_\reals \varphi
  d\gamma \right \rvert \leq
\left ( \frac{\epsilon}{6} + \frac{r_n}{2}\right ) \norm{\varphi^{\prime\prime\prime}}_\infty + g_n(\epsilon) \norm{\varphi^{ \prime
\prime}}_\infty
\end{align*}
and 
\begin{align*}
r_n^2 \leq \epsilon^2 + g_n(\epsilon)
\end{align*}
In particular, if $\lim_{n \to \infty} g_n(\epsilon) = 0$ for every
$\epsilon > 0$, then 
\begin{align*}
\lim_{n \to \infty} \left \lvert \expectation{\varphi(\hat{S_n})} - \int_\reals \varphi
  d\gamma \right \rvert = 0
\end{align*}
\end{thm}
\begin{proof}
Fix an $n >0$ and define for $1 \leq i \leq m_n$, $\hat{\xi}_{ni} = \frac{\xi_{ni}}{\Sigma_n}$ and
$\hat{S}_n = \hat{\xi}_{n1} + \cdots + \hat{\xi}_{nm_n}$.  Note that
$\expectation{\hat{S}_n^2} = 1$.  Let $\eta_1, \eta_2, \cdots, \eta_{m_n}$ be independent $N(0,1)$ random variables
that are also independent of the $\xi_{ni}$.  Note that we may have to
extend $\Omega_n$ in order to arrange this (e.g. extend by $[0,1]$ and
use Theorem \ref{ExistenceCountableIndependentRandomVariables}).  We
rescale each $\eta_i$ so that it has the same variance as
$\hat{\xi_{ni}}$; define $\hat{\eta}_i =
\frac{\sigma_{ni} \eta_i}{\Sigma_n}$ and $\hat{T}_n = \hat{\eta}_1 +
\cdots + \hat{\eta}_{m_n}$.  Notice that
$\expectation{\hat{\eta}_i^2} =
\expectation{\hat{\xi}_{ni}^2} = \frac{\sigma_{ni}^2}{\Sigma_n^2}$ and $\hat{T}_n$ is also a $N(0,1)$
random variable.  Therefore, by the Expectation Rule (Lemma
\ref{ExpectationRule}) $\int \varphi \, d\gamma =
\expectation{\varphi(\hat{T}_n)}$ and we can write 
\begin{align*}
\left \lvert \expectation{\varphi(\hat{S_n})} - \int_\reals \varphi
  d\gamma \right \rvert = \left \lvert \expectation{\varphi(\hat{S_n})} - \expectation{\varphi(\hat{T_n})} \right \rvert
\end{align*}
By having arranged for $\hat{\xi}_{ni}$ and $\hat{\eta}_i$ to have same
first and second moments so one should be thinking that we have
constructed a ``second order approximation''. 

The real trick of the proof is to interpolate between
$\varphi(\hat{S_n})$ and $\varphi(\hat{T_n})$ by exchanging
$\hat{\xi}_i$ and $\hat{\eta}_i$ one summand at a time.  By varying
only one summand we will then be able use Taylor's Theorem to
estimate the differences between the terms.  Concretely we write,
\begin{align*}
\varphi(\hat{S_n}) - \varphi(\hat{T_n}) &= \varphi(\hat{\xi}_{n1} + \cdots + \hat{\xi}_{nm_n}) -
\varphi(\hat{\eta}_1 + \cdots + \hat{\eta}_{m_n}) \\
&= \varphi(\hat{\xi}_{n1} + \cdots + \hat{\xi}_{nm_n}) - \varphi(\hat{\eta}_1 +
\hat{\xi}_{n2} + 
\cdots + \hat{\xi}_{nm_n}) \\
&+ \varphi(\hat{\eta}_1 +
\hat{\xi}_{n2} +
\cdots + \hat{\xi}_{nm_n}) - 
\varphi(\hat{\eta}_1 + \hat{\eta}_2 + \hat{\xi}_{n3} + \cdots + \hat{\xi}_{nm_n}) \\
&+ \cdots\\
&+ \varphi(\hat{\eta}_1 +
\cdots  +
\hat{\eta}_{m_n-1} + \hat{\xi}_{nm_n}) - 
\varphi(\hat{\eta}_1 + \cdots + \hat{\eta}_{nm_n}) \\
\end{align*}
Since we have to manipulate these terms a bit, it helps to clean up
the notation by defining:
\begin{align*}
U_i &= \begin{cases}
\hat{\xi}_{n2} + \cdots + \hat{\xi}_{nm_n} & \text{if $i=1$} \\
\hat{\eta}_1 + \cdots + \hat{\eta}_{i-1}+\hat{\xi}_{n,i+1} + \cdots +
\hat{\xi}_{nm_n} & \text{if $1 < i <m_n$} \\
\hat{\eta}_1 + \cdots + \hat{\eta}_{m_n-1} & \text{if $i=m_n$} \\
\end{cases}
\end{align*}
and then we can write the above interpolation as 
\begin{align*}
\varphi(\hat{S_n}) - \varphi(\hat{T_n}) &= \sum_{i=1}^{m_n} \varphi(U_i + \hat{\xi}_{ni}) -\varphi(U_i + \hat{\eta}_{i})
\end{align*}
Now we can take absolute values, use the triangle inequality and use
linearity of expectation to see
\begin{align*}
\abs{\expectation{\varphi(\hat{S_n}) - \varphi(\hat{T_n})}} &\leq
\sum_{i=1}^{m_n} \abs{\expectation{\varphi(U_i + \hat{\xi}_{ni})}
  -\expectation{\varphi(U_i + \hat{\eta}_i)}}\\
&=\sum_{i=1}^{m_n} \abs{\expectation{\varphi(U_i + \hat{\xi}_{ni})
  -\varphi(U_i + \hat{\eta}_i)}}
\end{align*}

Now we focus on each term $\varphi(U_i + \hat{\xi}_{ni}) -\varphi(U_i +
\hat{\eta}_i)$ by applying Taylor's Formula (Theorem
\ref{TaylorsTheorem}) to see 
\begin{align*}
\varphi(U_i + x) = \varphi(U_i) + x \varphi^\prime(U_i) +
\frac{x^2}{2} \varphi^{\prime\prime}(U_i) + R_i(x)
\end{align*}
where
\begin{align*}
R_i(x) &=  \int_{U_i}^{U_i+x}
\frac{(U_i + x - t)^2}{2} \varphi^{\prime \prime \prime}(t) \, dt
\end{align*}

For example,applying this expansion with $x = \hat{\xi}_{ni}$, using
linearity of expectation, independence of
$\hat{\xi}_{ni}$ and $U_i$ and Lemma \ref{IndependenceExpectations} we get
\begin{align*}
\expectation{\varphi(U_i + \hat{\xi}_{ni}) } &= \expectation{
  \varphi(U_i)+ \hat{\xi}_{ni} \varphi^\prime(U_i) +
\frac{\hat{\xi}_{ni}^2}{2} \varphi^{\prime\prime}(U_i) +
R_i(\hat{\xi}_{ni})} \\
&=\expectation{
  \varphi(U_i)} + \frac{\sigma_{ni}^2}{2\Sigma_n^2}\expectation{\varphi^{\prime\prime}(U_i)} + \expectation{R_i(\hat{\xi}_{ni})}
\end{align*}
and in exactly the same way because we have arrange for $\hat{\xi}_{ni}$
and $\hat{\eta}_i$ to share the first two moments, we get
\begin{align*}
\expectation{\varphi(U_i + \hat{\eta}_{ni}) } &=\expectation{
  \varphi(U_i)} + \frac{\sigma_{ni}^2}{2\Sigma_n^2}\expectation{\varphi^{\prime\prime}(U_i)} + \expectation{R_i(\hat{\eta}_i)}
\end{align*}
Thus, $\expectation{\varphi(U_i + \hat{\xi}_{ni})
  -\varphi(U_i + \hat{\eta}_i)} = \expectation{R_i(\hat{\xi}_{ni})} -
\expectation{R_i(\hat{\eta}_i)}$ and 
\begin{align*}
\abs{\expectation{\varphi(\hat{S_n}) - \varphi(\hat{T_n})}} &\leq 
\sum_{i=1}^{m_n} \abs{\expectation{R_i(\hat{\xi}_{ni}) }} +  \sum_{i=1}^{m_n} \abs{\expectation{R_i(\hat{\eta}_i) }}
\end{align*}

We complete the proof by bounding each expectation above.  On the one hand, there is the Lagrange Form for the
remainder term (Lemma \ref{LagrangeFormRemainder}) that shows that $R_i(x) =
\varphi^{\prime\prime\prime}(c) \frac{x^3}{6}$ for some $c \in [U_i,U_i+x]$ hence $\abs{R_i(x)} \leq
\norm{\varphi^{\prime\prime\prime}}_\infty \frac{\abs{x}^3}{6}$. On
the other hand, sticking with the integral form of the remainder term, since $t \in [U_i, U_i + x]$ we can bound the term $(U_i + x - t)^2 \leq \abs{x}^2$ in
the integral and integrate to conclude 
\begin{align*}
\abs{R_i(x)} &=  \int_{U_i}^{U_i+x}
\frac{(U_i + x - t)^2}{2} \varphi^{\prime \prime \prime}(t) \, dt \leq
\frac{\abs{x}^2}{2} \int_{U_i}^{U_i+x}
\varphi^{\prime \prime \prime}(t) \, dt \\
&=\frac{\abs{x}^2}{2} \left(\varphi^{\prime \prime}(U_i+x) - \varphi^{\prime \prime}(x)\right) \leq \norm{\varphi^{\prime\prime}}_\infty \abs{x}^2
\end{align*}

With this setup, pick $\epsilon >0$ and first consider the remainder term $R_i(\hat{\xi}_{ni}) $
and a note that we have to be a little careful.  We would like to use
the stronger $3^{rd}$ moment bound however we have not
assumed that $\hat{\xi}_{ni}$ has a finite $3^{rd}$ moment.  So what we
do is truncate $\hat{\xi}_{ni}$ and take a $2^{nd}$ moment bound over the
tail (valid because of the finite variance assumption) and use a
$3^{rd}$ moment bound on the truncated $\hat{\xi}_{ni}$.  The details follow:
\begin{align*}
\abs{\expectation{R_i(\hat{\xi}_{ni}) }} &\leq
\abs{\expectation{R_i(\hat{\xi}_{ni}) ; \abs{\hat{\xi}_{ni}} \leq \epsilon}}
+ \abs{\expectation{R_i(\hat{\xi}_{ni}) ; \abs{\hat{\xi}_{ni}} > \epsilon}}
\end{align*}
We take the sum of first terms and apply the Taylor's formula bound to see 
\begin{align*}
\sum_{i=1}^{m_n} \abs{\expectation{R_i(\hat{\xi}_{ni}) ; \abs{\hat{\xi}_{ni}} \leq \epsilon}}
&\leq
\frac{\norm{\varphi^{\prime\prime\prime}}_\infty}{6} \sum_{i=1}^{m_n}
\abs{\expectation{\abs{\hat{\xi}_{ni}}^3    ; \abs{\hat{\xi}_{ni}} \leq \epsilon}} \\
&\leq
\epsilon \frac{\norm{\varphi^{\prime\prime\prime}}_\infty}{6} \sum_{i=1}^{m_n}
\abs{\expectation{\abs{\hat{\xi}_{ni}}^2}} \\
&=
\epsilon \frac{\norm{\varphi^{\prime\prime\prime}}_\infty}{6} \sum_{i=1}^{m_n}
\frac{\sigma_{ni}^2}{\Sigma_n^2} = \epsilon \frac{\norm{\varphi^{\prime\prime\prime}}_\infty}{6} \\
\end{align*}
Next take the sum of the second terms to see
\begin{align*}
\sum_{i=1}^{m_n} \abs{\expectation{R_i(\hat{\xi}_{ni}) ; \abs{\hat{\xi}_{ni}} >    \epsilon}}
&\leq
\norm{\varphi^{\prime\prime}}_\infty\sum_{i=1}^{m_n} 
\abs{\expectation{\abs{\hat{\xi}_{ni}}^2    ; \abs{\hat{\xi}_{ni}} > \epsilon}} \\
&=
\norm{\varphi^{\prime\prime}}_\infty \frac{1}{\Sigma_n^2}\sum_{i=1}^{m_n}
\abs{\expectation{\abs{\xi_{ni}}^2 ; \abs{\xi_{ni}} > \epsilon \Sigma_n}} \\
&= \norm{\varphi^{\prime\prime}}_\infty g_\epsilon(n)
\end{align*}
Lastly, to bound the remainder term on $\hat{\eta}_i$ we can directly
appeal to the $3^{rd}$ moment bound because as a normal random
variable $\hat{\eta}_i$ has finite moments of all orders:
\begin{align*}
\sum_{i=1}^{m_n} \abs{\expectation{R_i(\hat{\eta}_i) }}
&\leq
\frac{\norm{\varphi^{\prime\prime\prime}}_\infty}{6} \sum_{i=1}^{m_n}
\abs{\expectation{\abs{\hat{\eta}_m}^3}} \\
&= \frac{\norm{\varphi^{\prime\prime\prime}}_\infty}{6} \sum_{i=1}^{m_n} \frac{\sigma_{ni}^3}{\Sigma_n^3}
\abs{\expectation{\abs{\eta_i}^3}} \\
&\leq \frac{r_n \norm{\varphi^{\prime\prime\prime}}_\infty}{6} \sum_{i=1}^{m_n} \frac{\sigma_{ni}^2}{\Sigma_n^2}
\abs{\expectation{\abs{\eta_i}^3}} \\
&= \frac{r_n \norm{\varphi^{\prime\prime\prime}}_\infty}{6}
\frac{2\sqrt{2}}{\sqrt{\pi}} < \frac{r_n \norm{\varphi^{\prime\prime\prime}}_\infty}{2}
\end{align*}

TODO: We used a calculation of the $3^{rd}$ absolute moment of the
standard normal distribution ($\frac{2\sqrt{2}}{\sqrt{\pi}}$).  We need to record that calculation somewhere.

The last thing to show is the bound on $r_n^2$.  For each $n >0$ and
$1 \leq i \leq n$,
\begin{align*}
\frac{\sigma_{ni}^2}{\Sigma_n^2} &= \frac{1}{\Sigma_n^2} \left(
  \expectation{\xi_{ni}^2 ; \abs{\xi_{ni}} < \epsilon \Sigma_n} +
  \expectation{\xi_{ni}^2 ; \abs{\xi_{ni}} \geq \epsilon \Sigma_n} \right) \\
&\leq \frac{1}{\Sigma_n^2} \left(\epsilon^2 \Sigma_n^2 + \Sigma_n^2
  g_n(\epsilon) \right) = \epsilon^2 + g_n(\epsilon)
\end{align*}
hence $r_n^2 = \max_{1\leq i \leq m_n} \frac{\sigma_{ni}^2}{\Sigma_n^2}
\leq \epsilon^2 + g_n(\epsilon)$.
\end{proof}


\chapter{Characteristic Functions And Central Limit Theorem}
In this section we study the weak convergence of random vectors more carefully.
Our first goal is to develop just enough of the theory of
characteristic functions in order to prove the classical Central Limit Theorem.
After that we delve more deeply into theory of characteristic
functions.  

The motivation for the theory we are about to develop is the intuition
that most of the behavior of a probability distribution on $\reals$ is
captured by its moments.  If one could put the information about all
of the distribution's moments into a single package simulataneously
then the resulting package might characterize the probability
distribution in a useful way.  A initial naive approach might be to
use a \emph{generating function} methodology.  For example, one might
try to define a function $f(t) = \sum_{n=0}^\infty M_n t^n$ where
$M_n$ denotes the $n^{th}$ moment.  Alas, such a approach fails rather
miserably as it is a very rare thing for moments to decrease quickly
enough for the formal power series for $f(t)$ to ever converge and make a useful
function object.  A better approach is to scale the moments to give
the series a chance to converge.  For example, being a bit sloppy we
could write
\begin{align*}
f(t) &= \int e^{tx} dP = \sum_{n=0}^\infty \frac{M_n}{n!} t^n
\end{align*}
This idea has a lot more merit and can be used effectively but it has
the distinct disadvantage that it only works for distributions that
have moments of all orders.

The wonderful idea that we will be exploring in this chapter is that
by passing into the domain of complex numbers we get a
characterization of the distribution that is always defined and (at
least conceptually) captures all moments in a generating function.
Specifically, we define 
\begin{align*}
f(t) &= \int e^{itx} dP 
\end{align*}
which is the \emph{Fourier Transform} of the probability distribution
and we get an object that uniquely determines the distribution and can
often be much easier to work with.  In particular we we will see that
convergence in distribution is described as pointwise convergence of
characteristic functions and through that connection we will get
another proof of the Central Limit Theorem.

In this section we start to make use of integrals of complex valued
measurable functions.  Let's establish the basic definitions and facts
that we require.
\begin{defn}A function $f : (\Omega, \mathcal{A}, \mu) \to \complexes$
  is measurable if and only $f = h + i g$ where $h,g : (\Omega,
  \mathcal{A}, \mu) \to \reals$ are measurable.  Equivalently,
  $\complexes$ is given the Borel $\sigma$-algebra.
\begin{itemize}
\item[(i)] If $\mu(A) < \infty$, then $\abs{\int f \, d\mu} \leq \int
  \abs{f} \, d\mu$.
\end{itemize}
\end{defn}
\begin{proof}
By the triangle inequality for the complex norm, we know that given
any two $z,w \in \complexes$ and $t \in [0,1]$, $\abs{(1-t) z + t w}
\leq (1-t) \abs{z} + t \abs{w}$ and therefore the complex norm is
convex.  Then by Jensen's Inequality (Theorem \ref{Jensen}, $\abs{\int f \, d\mu} \leq \int
\abs{f} \, d\mu$.
\end{proof}

\begin{defn}Let $\mu$ be a probability measure on $\reals^n$. Its
  \emph{Fourier Transform} is denoted $\hat{\mu}$ and is the complex function on
  $\reals^n$ defined by 
\begin{align*}
\hat{\mu}(u) &= \int e^{i\langle u,x \rangle} d \mu(x) = \int \cos(\langle u,x \rangle)
\, d \mu(x) + i \int \sin(\langle u,x \rangle) \, d\mu(x)
\end{align*}
\end{defn}
The first order of business is to establish the basic properties of
the Fourier Transform of a probability measure including the fact that
the definition makes sense.
\begin{thm} \label{CharacteristicFunctionBoundedAndContinuous}Let $\mu$ be a probability measure, then $\hat{\mu}$
  exists and is a bounded uniformly continuous function with
  $\hat{\mu}(0) = 1$.
\end{thm}
\begin{proof}
To see that $\hat{\mu}$ exists, use the representation
\begin{align*}
\hat{\mu}(u) &= \int \cos(\langle  u,x \rangle) \, d\mu(x)+ i \int \sin(\langle
  u,x \rangle) \, d\mu(x)
\end{align*} and use the facts that $\abs{\cos \theta } \leq 1$
and $\abs {\sin \theta} \leq 1$ to conclude that both integrals are
bounded.

To see that $\hat{\mu}(0) = 1$, simply calculate
\begin{align*}
\hat{\mu}(0) &= \int \cos(\langle  0,x \rangle) \, d\mu(x)+ i \int \sin(\langle
  0,x \rangle) \, d\mu(x) = \int d\mu(x) = 1
\end{align*} 
In a similar way, boundedness is a simple calculation
\begin{align*}
\abs{\hat{\mu}(u)} &\leq \int \abs{e^{i \langle u,x \rangle}} \, d\mu(x) =
\int \, d\mu(x) = 1
\end{align*}
Lastly, to prove uniform continuity, first note that for any $u,v \in
\reals^n$, we have
\begin{align*}
\abs{e^{i \langle  u,x \rangle} - e^{i \langle  v,x \rangle} }^2 &=
\abs{e^{i \langle  u-v,x \rangle} - 1}^2 \\
&= (\cos(\langle  u-v,x
\rangle) - 1)^2 + \sin^2(\langle  u-v,x \rangle) \\
&= 2(1 - \cos(\langle  u-v,x
\rangle) \\
&\leq \langle  u-v,x\rangle^2 & & \text{by Lemma
  \ref{BasicExponentialInequalities}} \\
&\leq \norm{u-v}^2_2 \norm{x}^2_2 & & \text{by Cauchy Schwartz}
\end{align*}
On the other hand, it is clear from the triangle inequality that
\begin{align*}
\abs{e^{i \langle  u,x \rangle} - e^{i \langle  u,x \rangle} } \leq
\abs{e^{i \langle  u,x \rangle}} + \abs{e^{i \langle  u,x \rangle}  }
\leq 2
\end{align*}
and therefore we have the bound $\abs{e^{i \langle  u,x \rangle} -
  e^{i \langle  u,x \rangle} } \leq \norm{u-v}_2 \norm{x}_2 \wedge
2$.  
Note that pointwise in $x \in \reals^n$, $\lim_{n \to \infty}
\frac{1}{n} \norm{x}_2 \wedge 2 = 0$ and trivially $\frac{1}{n}
\norm{x}_2 \wedge 2 \leq 2$ so Dominated Convergence shows
that $\lim_{n \to \infty} \int \frac{1}{n} \norm{x}_2 \wedge
  2 \, d\mu(x) = 0$.  Given an $\epsilon >0$, pick $N > 0$ such that
$\int \frac{1}{N} \norm{x}_2 \wedge
  2 \, d\mu(x) < \epsilon$ then for  $\norm{u-v}_2
\leq \frac{1}{N}$,
\begin{align*}
\abs{\hat{\mu}(u) - \hat{\mu}(v)} &\leq \int \abs{e^{i \langle  u,x \rangle} -
  e^{i \langle  u,x \rangle}} \, d\mu(x) \\
&\leq \int \norm{u-v}_2 \norm{x}_2 \wedge
2  \, d\mu(x) \\
&\leq \int \frac{1}{N} \norm{x}_2 \wedge 
2  \, d\mu(x) < \epsilon
\end{align*}
proving uniform continuity.
\end{proof}
\begin{defn}Let $\xi$ be an $\reals^n$-valued random variable. Its
  characteristic function is denoted $\varphi_\xi$ and is the complex
  valued function on
  $\reals^n$ defined by 
\begin{align*}
\varphi_\xi(u) &= \expectation{e^{i\langle u,\xi \rangle}} \\
&= \int e^{i\langle u,x \rangle} \distribution{\xi}(dx) =
\hat{\distribution{\xi}}(u) 
\end{align*}
\end{defn}

We motivated the definition of the characteristic function by
considering how we might encode information about the moments of a
probability measure.  To make sure that we've succeeded we need to
show how to extract moments from the characteristic function.  To see
what we should expect, let's specialize to $\reals$ and suppose that
we can write out a power series:
\begin{align*}
\hat{\mu}(t) &= \int e^{itx} \, d\mu = \sum_{n=0}^\infty \frac{i^n
  M_n}{n!} t^n
\end{align*}
Still working formally, we see that we can differentiate the series with respect
to $t$ to isolate each individual moment $M_n$
\begin{align*}
\frac{d^n}{dt^n} \hat{\mu}(0) &= i^n M_n
\end{align*}
The above computation was rather formal and we won't try to make the
entire thing rigorous (specifically we won't consider the series
expansions).  What we make rigorous in the next Theorem is the connection between
moments of $\mu$ and derivatives of the characteristic function.
\begin{thm}\label{MomentsAndDerivatives} Let $\mu$ be a probability measure on $\reals^n$ such that
  $f(x) = \left| x \right| ^ m$ is integrable with respect to $\mu$.
  Then $\hat{\mu}$ is has continuous partial derivatives up to order
  $m$ and 
\begin{equation*}
 \frac{\partial^m \hat{\mu}} {\partial x_{j_1} \ldots \partial x_{j_m}}(u) =
 i^m \int x_{j_1} \ldots x_{j_m}e^{i\langle u,x \rangle} \mu(dx)
\end{equation*}
\end{thm}
\begin{proof}First we proceed with $m=1$.  Pick $1 \leq j \leq n$ and let $v \in \reals^n$ be the
  vector with $v_j = 1$ and $v_i = 0$ for $i \neq j$.  Then for $u \in
  \reals^n$ and $t > 0$,
\begin{align*}
\frac{\hat{\mu}(u + t v_j) - \hat{\mu}(u)}{t} &= \frac{1}{t} \int e^{i
  \langle u + t v_j, x \rangle} - e^{i\langle u,x \rangle} d\mu(x) \\
&= \frac{1}{t} \int e^{i\langle u,x \rangle} \left ( e^{i  t x_j} - 1 \right ) d \mu(x)
\end{align*}
But note that 
\begin{align*}
\abs{\frac{1}{t} e^{i\langle u,x \rangle} \left ( e^{i t x_j} - 1
  \right )}^2 &= \abs{\frac{ e^{i  t x_j} - 1 }{t} }^2 \\
&= \frac{\cos^2(t
  x_j) - 2 \cos(t x_j) + 1 + \sin^2(t x_j)}{t^2} \\
&= 2\left ( \frac{1 - \cos(t x_j)}{t^2} \right ) \\
&\leq x_j^2 & & \text{by Lemma \ref{BasicExponentialInequalities}}
\end{align*}
But $\abs{x_j}$ is assumed to be integrable hence we can apply the
Dominated Convergence Theorem to see 
\begin{align*}
\frac{\partial}{\partial x_j}  \int e^{i\langle u, x \rangle} d \mu(x)
&= \lim_{t \to 0} \frac{1}{t} \int e^{i\langle u + t v_j, x \rangle} -
e^{i\langle u, x \rangle} d \mu(x) \\
&=  \int \lim_{t \to 0}  \frac{e^{i\langle u + t v_j, x \rangle} -
e^{i\langle u, x \rangle}}{t} d \mu(x) \\
&= i \int x_j e^{i\langle u, x \rangle} d \mu (x)
\end{align*}

Continuity of the derivative follows from the formula we just proved.
Suppose that $u_n \to u \in \reals^n$.  Then we have shown that
\begin{align*}
\frac{\partial}{\partial x_j} \hat{\mu} (u_n)= i \int x_j e^{i\langle u_n, x \rangle} d \mu (x)
\end{align*}
and we have the bound on the integrands $\abs{x_j e^{i\langle u_n, x
    \rangle}} < \abs{x_j}$ with $\abs{x_j}$ integrable by assumption.
We apply Dominated Convergence to see that 
\begin{align*}
\lim_{n \to \infty} \frac{\partial}{\partial x_j} \hat{\mu} (u_n) &=  i \int \lim_{n\to \infty} x_j
e^{i\langle u_n, x \rangle} d \mu (x) \\
&= i \int x_j e^{i\langle u, x
\rangle} d \mu (x) \\
&= \frac{\partial}{\partial x_j} \hat{\mu}(u)
\end{align*}

TODO: Fill in the details of the induction step (it is pretty obvious
that argument above IS the induction step).
\end{proof}

The key in unlocking the relationship between weak convergence and
characteristic functions is a basic property of Fourier Transforms
that is often called the Plancherel Theorem.  In our particular case
the Plancherel Theorem shows that one may evaluate integrals of
continuous functions against probability measures equally well using
Fourier Transforms; in this way we'll see that the characteristic
function of a probability measure is a faithful representation of the
measure when viewed as a functional (the point of view implicit in
the definition of weak convergence).
\begin{thm}\label{PlancherelTheorem}Let 
\begin{align*}
\rho_\epsilon(x) &= \frac{1}{\epsilon\sqrt{2 \pi}}
  e^{-\frac{x^2}{2\epsilon^2}}
\end{align*}  be the Gaussian density with variance $\epsilon^2$.  Given a Borel probability measure
  $(\reals, \mathcal{B}(\reals), \mu)$ and an integrable $f : \reals
  \to \reals$, then for any $\epsilon > 0$,
\begin{align*}
\int_{-\infty}^\infty f * \rho_\epsilon (x) \, d\mu(x) &=
\frac{1}{2\pi} \int_{-\infty}^\infty e^{-\frac{\epsilon^2 u^2}{2}} \hat{f}(u) \overline{\hat{\mu}(u)}\, du
\end{align*}
If in addition, $f \in C_b(\reals)$ and $\hat{f}(u)$ is integrable
then
\begin{align*}
\int_{-\infty}^\infty f \, d\mu &= \frac{1}{2 \pi}
\int_{-\infty}^\infty \hat{f}(u) \overline{\hat{\mu}(u)}\, du
\end{align*}
\end{thm}
\begin{proof}
This is a calculation using Fubini's Theorem (Theorem \ref{Fubini}) to
the triple integral
\begin{align*}
\int \int \int e^{-\frac{\epsilon^2 u^2}{2}} f(x) e^{iux} e^{-iuy}
\, d\mu(y) \, dx \, du
\end{align*}
Note that by Tonneli's Theorem, 
\begin{align*}
\int \int \int \abs{ e^{-\frac{\epsilon^2 u^2}{2}} f(x) e^{iux} e^{-iuy}}
\, d\mu(y) \, dx \, du &= \int \int \int e^{-\frac{\epsilon^2 u^2}{2}} \abs{ f(x)} 
\, d\mu(y) \, dx \, du \\
&=\int \abs{ f(x)}  \, dx \int e^{-\frac{\epsilon^2 u^2}{2}} \, du < \infty
\end{align*}
and therefore we are justified in using using Fubini's Theorem to
calculate via iterated integrals
\begin{align*}
 \frac{1}{2 \pi} \int_{-\infty}^\infty e^{-\frac{\epsilon^2 u^2}{2}} \hat{f}(u)
 \overline{\hat{\mu}(u)}\, du &=  \frac{1}{2 \pi} \int_{-\infty}^\infty
 e^{-\frac{\epsilon^2 u^2}{2}} \left( \int_{-\infty}^\infty f(x)e^{iux} \, dx \right )
 \left (\int_{-\infty}^\infty e^{-iuy} \, d\mu(y) \right )\, du \\
&=  \frac{1}{2 \pi} \int_{-\infty}^\infty f(x)
\left( \int_{-\infty}^\infty 
 \left (\int_{-\infty}^\infty e^{iu(x-y)}  e^{-\frac{\epsilon^2 u^2}{2}}
   \, du \right ) \, d\mu(y) \right ) \, dx\\
\end{align*}
Now the inner integral is just the Fourier Transform of a Gaussian
with mean $0$ and variance $\frac{1}{\epsilon^2}$ which we have calculated in Exercise \ref {FourierTransformGaussian},
so we have by that calculation, another application of Fubini's
Theorem and the definition of convolution,
\begin{align*}
&= \frac{1}{2 \pi} \int_{-\infty}^\infty f(x)
\left( \int_{-\infty}^\infty \frac{\sqrt{2\pi}}{\epsilon}
  e^{-(x-y)^2/2\epsilon^2} \, d\mu(y) \right ) \, dx\\
&= \int_{-\infty}^\infty f(x)
\left( \int_{-\infty}^\infty \rho_\epsilon(x -y) \, d\mu(y) \right ) \, dx\\
&= \int_{-\infty}^\infty 
\left( \int_{-\infty}^\infty f(x) \rho_\epsilon(x -y) \, dx \right )
\, d\mu(y) \\
&= \int_{-\infty}^\infty f * \rho_\epsilon(y) \, d\mu(y) \\
\end{align*}

The second part of the theorem is just an application of Lemma
\ref{UniformApproximationByGaussians} and the first part of the
Theorem.  By the Lemma, we know that for any $f \in C_c(\reals;
\reals)$, we have $\lim_{\epsilon \to 0} \sup_x
\abs{f*\rho_\epsilon (x) - f(x)} = 0$.  So we have, 
\begin{align*}
\lim_{\epsilon \to 0} \abs{\int_{-\infty}^\infty f - f*\rho_\epsilon
  \, d\mu} &\leq \lim_{\epsilon \to 0} \int_{-\infty}^\infty \abs{ f -
  f*\rho_\epsilon} \, d\mu &\leq \lim_{\epsilon \to 0} \sup_x \abs{ f -
  f*\rho_\epsilon} = 0 \\
\end{align*}
and by integrability of $\hat{f}(u)$, the fact that
$\abs{\hat{\mu}} \leq 1$ (Lemma \ref{CharacteristicFunctionBoundedAndContinuous}) we may use Dominated Convergence
to see that 
\begin{align*}
\lim_{\epsilon \to 0} \int_{-\infty}^\infty f*\rho_\epsilon  \, d\mu
&= \frac{1}{2\pi} \lim_{\epsilon \to 0} \int_{-\infty}^\infty
e^{-\frac{1}{2}\epsilon^2 u^2} \hat{f}(u) \overline{\hat{\mu}(u)} \,
du \\
&= \frac{1}{2\pi}\int_{-\infty}^\infty  \lim_{\epsilon \to 0} 
e^{-\frac{1}{2}\epsilon^2 u^2} \hat{f}(u) \overline{\hat{\mu}(u)} \,
du \\
&=\frac{1}{2\pi}\int_{-\infty}^\infty  \hat{f}(u)
\overline{\hat{\mu}(u)} \, du 
\end{align*}
and therefore we have the result.
\end{proof}
As it turns out, we'll get a lot more mileage our the first statement
of the Theorem above.  We won't really ever be in a position in which
we have the required integrability of the Fourier Transform
$\hat{f}(t)$ to use the second part.  However, the technique used in the proof
of the second part of the Theorem will be replayed several times.
First we show that the characteristic function completely
characterizes probability measures.
\begin{thm}\label{EqualCharacteristicFunctionEqualMeasures} Let $\mu$ and $\nu$ be a probability measures on
  $\reals^n$ such that $\hat{\mu} = \hat{\nu}$, then $\mu=\nu$.
\end{thm}
\begin{proof}
Let $f \in C_c(\reals)$, then we know by Lemma
\ref{UniformApproximationByGaussians} that $\lim_{\epsilon \to 0}
\norm{\rho_\epsilon * f - f}_\infty = 0$.  Then for each $\epsilon >
0$, and using the Plancherel Theorem
\begin{align*}
\abs{\int f \, d\mu - \int f \, d\nu} &\leq \abs{\int \rho_\epsilon *
  f \, d\mu - \int \rho_\epsilon * f \, d\nu} + \int \abs{\rho_\epsilon *
  f  - f}\, d\mu + \int \abs{\rho_\epsilon *
  f  - f}\, d\nu \\
&\leq \abs{\frac{1}{2\pi} \int_{-\infty}^\infty e^{-\frac{\epsilon^2
      u^2}{2}} \hat{f}(u) (\overline{\hat{\mu}(u)} - \overline{\hat{\nu}(u)})\, du}+ 2\norm{\rho_\epsilon *
  f  - f}_\infty \\
&=  2\norm{\rho_\epsilon *
  f  - f}_\infty 
\end{align*}
Taking the limit as $\epsilon$ goes to $0$, we see that $\int f \,
d\mu = \int f \, d\nu$ for all $f \in C_c(\reals)$.

Now, take a finite interval $[a,b]$ and approximate
$\characteristic{[a,b]}$ by the compactly supported continuous
functions
\begin{align*}
f_n(x) &= \begin{cases}
1 & \text{for $a \leq x \leq b$} \\
0 & \text{for $x < a- \frac{1}{n}$ or $x > b + \frac{1}{n}$} \\
n(x-a) + 1 & \text{for $a-\frac{1}{n} \leq x < a$} \\
1 - n(x-b)& \text{for $b < x \leq b + \frac{1}{n}$} \\
\end{cases}
\end{align*}
It is clear that $f_n(x)$ is decreasing in $n$ and $\lim_{n \to
  \infty}f_n(x) = \characteristic{[a,b]}$ so by Monotone Convergence
\begin{align*}
\mu([a,b]) &= \lim_{n \to \infty} \int f_n \,
d\mu = \lim_{n \to \infty} \int f_n \,
d\nu = \nu([a,b])
\end{align*}
Since the Borel $\sigma$-algebra is generated by the closed intervals,
we see that $\mu = \nu$.
\end{proof}
\begin{thm} Let $\xi = \left( \xi_1, \ldots , \xi_n \right)$ be an
    $\reals^n$-valued random variable.  Then the $\reals$-valued
    random variables $\xi_i$ are independent if and only if 
\begin{equation*}
\varphi_\xi(u_1, \ldots , u_n) = \prod_{j=1}^n \varphi_{\xi_j}(u_j)
\end{equation*}
\end{thm}
\begin{proof}
TODO: This is a simple corollary that follows by calculating the
characteristic function of the product and then using the fact that
the characteristic function uniquely defines the distribution.
First suppose that the $\xi_i$ and independent.  Then we calculate
\begin{align*}
\varphi_\xi(u) &= \expectation{e^{i\langle u, \xi
    \rangle}} = \expectation{\prod_{k=1}^ne^{i u_k \xi_k}}
=\prod_{k=1}^n \expectation{e^{i u_k \xi_k}} = \prod_{k=1}^n \varphi_{\xi_k}(u_k) 
\end{align*}
Note that here we have used Lemma \ref{IndependenceExpectations} on
a bounded complex valued function.
TODO: Do the simple validation that the Lemma extends to this situation.

On the other hand, if we assume that $\varphi_\xi(u_1, \ldots , u_n) =
\prod_{j=1}^n \varphi_{\xi_j}(u_j)$, then we know that if we pick
independent random variables $\eta_j$ where each $\eta_j$ has the same
distribution as $\xi_j$ then by the above calculation $\varphi_\xi(u)
= \varphi_\eta(u)$.
By Theorem \ref{EqualCharacteristicFunctionEqualMeasures} we know
that $\xi$ and $\eta$ have the same distribution.  Thus the $\xi_j$
are also independent by Lemma \ref{IndependenceProductMeasures} and
the equality of the distributions of each $\xi_j$ and $\eta_j$.
\end{proof}
\begin{lem}Let $\xi$ and $\eta$ be independent random vectors in
  $\reals^n$.  Then $\varphi_{\xi + \eta}(u) = \varphi_\xi(u) \varphi_\eta(u)$.
\end{lem}
\begin{proof}
This follows from the calculation 
\begin{align*}
\varphi_{\xi + \eta}(u) &= \expectation{e^{i\langle u, \xi + \eta
    \rangle}} = \expectation{e^{i\langle u, \xi
    \rangle}e^{i\langle u, \eta \rangle}} \\
&=\expectation{e^{i\langle u, \xi
    \rangle}} \expectation{e^{i\langle u, \eta \rangle}} =
\varphi_{\xi }(u)\varphi_{\eta}(u)& & \text{by
  Lemma \ref{IndependenceExpectations}}
\end{align*}
\end{proof}

\begin{examp}\label{FourierTransformGaussian}Let $\xi$ be an $N(0,1)$ random variable.  Then
  $\varphi_\xi(u) = e^{\frac{-u^2}{2}}$.  The least technical way of
  seeing this requires a bit of a trick.  First note that because $\sin
  ux$ is an odd function we have
\begin{align*}
\varphi_\xi(u) &= \frac{1}{\sqrt{2\pi}}\int_{-\infty}^\infty e^{iux}
e^{\frac{-x^2}{2}} \, dx \\
&= \frac{1}{\sqrt{2\pi}}\int_{-\infty}^\infty e^{\frac{-x^2}{2}}\cos ux
\, dx + \frac{i}{\sqrt{2\pi}}\int_{-\infty}^\infty e^{\frac{-x^2}{2}}\sin ux
\, dx \\
&=  \frac{1}{\sqrt{2\pi}}\int_{-\infty}^\infty e^{\frac{-x^2}{2}}\cos
ux \, dx
\end{align*}
On the other hand by Lemma \ref{MomentsAndDerivatives} and the fact
that $x \cos ux$ is an odd function we have 
\begin{align*}
\frac{d\varphi_\xi(u)}{du} &=
\frac{i}{\sqrt{2\pi}}\int_{-\infty}^\infty x e^{iux}
e^{\frac{-x^2}{2}} \, dx \\
&= \frac{i}{\sqrt{2\pi}}\int_{-\infty}^\infty x e^{\frac{-x^2}{2}}\cos ux
\, dx - \frac{1}{\sqrt{2\pi}}\int_{-\infty}^\infty x e^{\frac{-x^2}{2}}\sin ux
\, dx \\
&=  \frac{-1}{\sqrt{2\pi}}\int_{-\infty}^\infty x e^{\frac{-x^2}{2}}\sin ux
\, dx
\end{align*}
This last integral can be integrated by parts (let $df = x
e^{\frac{-x^2}{2}} dx$ and $g = \sin ux$, hence $f =
-e^{\frac{-x^2}{2}}$ and $dg = u\cos ux$) to yield
\begin{align*}
\frac{d\varphi_\xi(u)}{du} &=\frac{-u}{\sqrt{2\pi}}\int_{-\infty}^\infty e^{\frac{-x^2}{2}}\cos
ux \, dx
\end{align*}
and therefore we have shown that characteristic function satisfies the
simple first order differential equation $\frac{d\varphi_\xi(u)}{du} =
-u \varphi_\xi(u)$ which has the general solution $\varphi_\xi(u) = C
e^\frac{-u^2}{2}$ for some constant $C$.  To determine the constant,
we use Lemma
\ref{CharacteristicFunctionBoundedAndContinuous} to see that
$\varphi_\xi(0) = C = 1$ and we are done.
\end{examp}

To extend the previous example to arbitrary normal distributions, we
prove the following result that has independent interest.
\begin{lem}\label{CharacteristicFunctionOfAffineTransform}Let $\xi$ be a random vector in $\reals^N$ then for $a \in
  \reals^M$ and $A$ an $M \times N$ matrix, we have
\begin{align*}
\varphi_{a + A\xi}(u) &= e^{i\langle a, u \rangle} \varphi_\xi(A^* u)
\end{align*}
where $A^*$ denotes the transpose of $A$.
\end{lem}
\begin{proof}
This is a simple calculation
\begin{align*}
\varphi_{a + A\xi}(u) &= \expectation{e^{i \langle u, a + A \xi
    \rangle}} = \expectation{e^{i \langle u, a \rangle} e^{i \langle u,  A \xi
    \rangle}}
= e^{i \langle u, a \rangle}\expectation{e^{i \langle A^* u,  \xi
    \rangle}} = e^{i\langle a, u \rangle} \varphi_\xi(A^* u)
\end{align*}
where we have used the elementary fact from linear algebra that
\begin{align*}
\langle u,  Av \rangle &= u^*Av = (u^*Av)^* = v^* A^*u = \langle A^*u,  v \rangle
\end{align*}
\end{proof}

\begin{examp}\label{FourierTransformGeneralGaussian}Let $\xi$ be an $N(\mu,\sigma^2)$ random variable.  Then
  $\varphi_\xi(u) = e^{iu\mu - \frac{1}{2} u^2\sigma^2}$.  We know
  that if $\eta$ is an $N(0,1)$ random variable then $\mu +
  \sigma\eta$ is $N(\mu, \sigma^2)$, so by the previous Lemma
  \ref{CharacteristicFunctionOfAffineTransform} and Example \ref{FourierTransformGaussian}
\begin{align*}
\varphi_\xi(u) &= e^{iu\mu} \varphi_\eta(\sigma u) = e^{iu\mu - \frac{1}{2} u^2\sigma^2}
\end{align*}
\end{examp}


The last piece of the puzzle that we need to put into place before
proving the Central Limit Theorem is a result that shows we can test
convergence in distribution by looking at pointwise convergence of
associated characteristic functions.
\begin{thm}[Glivenko-Levy Continuity
  Theorem]\label{GlivenkoLevyContinuity}If $\mu, \mu_1, \mu_2, \dots$
  are probability measures on $(\reals^n, \mathcal{B}(\reals^n))$,
  then $\mu_n$ converge weakly to $\mu$ if and only if $\hat{\mu}_n(u)$
  converge to $\hat{\mu}(u)$ pointwise.
\end{thm}
\begin{proof}
By Theorem \ref{WeakConvergenceWithSmoothTestFunctions} it suffices to
show that for every $f \in C^\infty_c(\reals^n, \reals)$, we have $\lim_{n
  \to \infty} \int f
\, d\mu_n = \int f \, d\mu$.  By \ref{UniformApproximationByGaussians}
we know that $\lim_{\epsilon \to 0}
\norm{\rho_\epsilon * f - f}_\infty = 0$.  Pick $\delta > 0$ and find
$\epsilon > 0$ such that $\norm{\rho_\epsilon * f - f}_\infty <
\delta$.
Now, 
\begin{align*}
\abs{\int f \, d\mu_n - \int f \, d\mu} &\leq \abs{\int (f - \rho_\epsilon * f) \, d\mu_n} + \abs{\int \rho_\epsilon * f \, d\mu_n -
  \int \rho_\epsilon * f \, d\mu} + \abs{\int (\rho_\epsilon * f - f)
  \, d\mu} \\
&\leq \delta + \frac{1}{2\pi} \abs{\int \hat{f}(t)
  e^{-\frac{1}{2}\epsilon^2t^2} (\hat{\mu}_n(t) - \hat{\mu}(t)) \, dt}
+ \delta 
\end{align*}
where we have used the Plancherel Theorem (Theorem
\ref{PlancherelTheorem}) and the uniform
approximation of $f$ by $\rho_\epsilon * f$ in going from the first to
the second line.

Because $f$ is compactly supported, we know that $\hat{f}(t) \leq
\norm{f}_\infty$ and together with Lemma
\ref{CharacteristicFunctionBoundedAndContinuous} we see that 
\begin{align*}
\abs{\hat{f}(t)
  e^{-\frac{1}{2}\epsilon^2t^2} (\hat{\mu}_n(t) - \hat{\mu}(t))} \leq
2\norm{f}_\infty e^{-\frac{1}{2}\epsilon^2t^2}
\end{align*} where the upper bound
is an integrable function of $t$.  Therefore by Dominated Convergence
we see that $\limsup_{n \to \infty} \abs{\int f \, d\mu_n - \int f \,
  d\mu}  \leq 2\delta$.  Since $\delta>0$ was arbitrary, we have $\int f \, d\mu_n = \int f \,  d\mu$.
\end{proof}

Note that part of the hypothesis in the above theorem is the fact that
the pointwise limit of the characteristic functions is assumed to be
the characteristic function of a probability measure.  There is a stronger form of the above theorem that
characterizes when a pointwise limit of charactersistic functions is
in fact the characteristic function of a probability measure.  That
stronger result is not needed to prove the Central Limit Theorem so we
postpone its statement and proof until later.

\begin{thm}[Central Limit Theorem]\label{CentralLimitTheorem}Let
  $\xi, \xi_1, \xi_2, \dots$ be i.i.d. random variables with $\mu =
  \expectation{\xi}$ and 
  $\sigma = \variance{\xi_n} < \infty$, then 
\begin{align*}
\sqrt{n} (\frac{1}{n}\sum_{i=1}^n \xi_i -  \mu) \todist N(0,\sigma^2)
\end{align*}
\end{thm}
\begin{proof}
The first thing to note is that by using the Theorem on $\frac{\xi_i -
  \mu}{\sigma}$, it suffices to assume that $\mu = 0$
and $\sigma = 1$.  Thus we only have to show that $\frac{1}{\sqrt{n}}
\sum_{k=1}^n \xi_k \todist N(0,1)$.  

Define $S_n = \sum_{k=1}^n \xi_k$.  By Theorem
\ref{GlivenkoLevyContinuity} it suffices to show that 
\begin{align*}
\lim_{n \to \infty} \expectation{e^{itS_n/\sqrt{n}}} &= e^{t^2/2}
\end{align*}
To calculate the limit, first note that by independence and i.i.d. we
have 
\begin{align*}
\expectation{e^{itS_n/\sqrt{n}}} &= \prod_{k=1}^n
\expectation{e^{it\xi_k/\sqrt{n}}}  = \left[\expectation{e^{it\xi/\sqrt{n}}}\right]^n
\end{align*}

In order to evaluate the limit, we take the Taylor expansion of the exponential $e^{ix} = 1 + i x -
\frac{1}{2}x^2 + R(x)$ where by Lagrange form of the remainder and the
fact that $\abs{\frac{d}{dx} e^{ix}} \leq 1$, we
see that $\abs{R(x)} \leq \frac{1}{6}\abs{x}^3$.  Note that this
estimate isn't very good for large $\abs{x}$ but it is easy to do
better for $\abs{x} > 1$ just using the triangle inequality
\begin{align*}
\abs{e^{ix} - 1 - ix + \frac{1}{2}x^2} &\leq 2 + \abs{x} +
\frac{1}{2}x^2 \leq \frac{7}{2}x^2
\end{align*}
Therefore we have the bound $\abs{R(x)} \leq \frac{7}{2}(\abs{x}^3
\wedge x^2)$.  Applying the Taylor expansion and using the zero mean
and unit variance assumption, we get
\begin{align*}
\expectation{e^{itS_n/\sqrt{n}}} &= \left(1 - \frac{t^2}{2n} + \expectation{R(\frac{t\xi}{\sqrt{n}})}\right)^n
\end{align*}
By our estimate on the remainder term, we can see that
\begin{align*}
n \abs{\expectation{R(\frac{t\xi}{\sqrt{n}})}} &\leq
\frac{7}{2}\expectation{\frac{t^3\abs{\xi}^3}{\sqrt{n}} \wedge t^2
  \xi^2} \\
&\leq \frac{7}{2} \expectation{ t^2
  \xi^2} = \frac{7t^2}{2} 
\end{align*}
By the above inequalities and Dominated Convergence we can
conclude that 
\begin{align*}
\lim_{n \to \infty} n
\abs{\expectation{R(\frac{t\xi}{\sqrt{n}})}} = 0
\end{align*}
so if we define $\epsilon_n = \frac{2n}{t^2}
\abs{\expectation{R(\frac{t\xi}{\sqrt{n}})}}$ then we have $\lim_{n
  \to \infty} \epsilon_n = 0$ and 
\begin{align*}
\lim_{n \to \infty} \expectation{e^{itS_n/\sqrt{n}}} &= \lim_{n \to 
  \infty} \left(1 - \frac{t^2}{2n} (1 + \epsilon_n) \right)^n =\lim_{n \to 
  \infty}  e^{n \log(1 - \frac{t^2}{2n} (1 + \epsilon_n))} =
e^{-t^2/2}
\end{align*}
\end{proof}

It is also useful to call out a useful corollary of the continuity
theorem that allows one to characterize convergence in distribution of
random vectors by considering one dimensional projections.
\begin{cor}[Cramer Wold Device]\label{CramerWoldDevice}Let $\xi, \xi_1, \xi_2, \dotsc$ be 
  random vectors in $\reals^N$.  Then $\langle c, \xi_n \rangle
  \todist \langle c, \xi \rangle$ for all $c \in \reals^N$ if and only
  if $\xi_n \todist \xi$.
\end{cor}
\begin{proof}
Simply note that for all random vectors $\xi$,
$c \in \reals^N$ and $x \in \reals$
\begin{align*}
\varphi_{\langle c , \xi\rangle}(x) &= \expectation{e^{i x \langle
    c , \xi \rangle} }= \varphi_\xi(x c)
\end{align*}
 
Therefore if $\langle c, \xi_n \rangle
  \todist \langle c, \xi \rangle$ for all $c \in \reals^N$ then by the
  Glivenko-Levy Continuity Theorem \ref{GlivenkoLevyContinuity}
we know that 
\begin{align*}
\lim_{n \to \infty} \varphi_{\xi_n}(c) &= \lim_{n \to \infty}
\varphi_{\langle c, \xi_n \rangle}(1) = \varphi_{\langle c, \xi
  \rangle}(1) = \varphi_\xi(c)
\end{align*}
and applying the Theorem again we conclude that $\xi_n \todist \xi$.  In
a completely analogous way, if we assume that $\xi_n \todist \xi$ then
for all $c \in \reals^N$ and $x\in \reals$, then
\begin{align*}
\lim_{n \to \infty} \varphi_{\langle c, \xi_n \rangle}(x) &= \lim_{n \to \infty}
\varphi_{\xi_n}(x c) = \varphi_{\xi}(x c) = \varphi_{\langle c, \xi \rangle}(x)
\end{align*}
from which we conclude that $\langle c, \xi_n \rangle \todist \langle
c, \xi \rangle$.
\end{proof}

\begin{thm}[Prokhorov's Theorem, special case]Let $\mu_n$ be a tight sequence of
  measures on $\reals^n$.  Then there is a subsequence of that
  converges in distribution.
\end{thm}
\begin{proof}
TODO
\end{proof}
 
TODO: Do the full Levy Continuity Theorem (and Prokhorov's Theorem) that shows a characteristic
function that is continuous at $0$ is the characterisitic function of
a probability measure (the basic point is that the pointwise limit of
characteristic functions of probability measures is almost the
characteristic function of a probability measure; the associated
distribution function may not have the correct limits at $\pm
\infty$ due to mass escaping to infinity.  If we assume continuity at $0$, then we can prove tightness
which keeps the mass from escaping and shows that the limits are $0,1$ as required of a distribution
function.  Note that the pointwise limit of a sequence of
characteristic functions is the characteristic function of a measure
(though not necessarily a probability measure); this fact is often
know as the Helly Selection Theorem.  It can be restated in terms of a
topology on the space of locally finite measures called the vague
topology and the Helly Selection Theorem can be restated as saying
that the space of probability measures is relatively sequentially
compact in the vague topology on the locally finite measures on $\reals^n$.

\section{Gaussian Random Vectors and the Multidimensional Central
  Limit Theorem}
There is a version of the Central Limit Theorem for random vectors in
$\reals^N$ in which Gaussian distributions also occur.  The nature of
Gaussians in this context is a bit more subtle than in the one
dimensional case.  We lead with a definition
\begin{defn}A random vector $\xi$ in $\reals^N$ is said to be a \emph{Gaussian random
  vector} is for every $a \in \reals^N$, the random variable $\langle
  a,\xi \rangle$ is a univariate normal or is almost surely 0 (which
  we take as the degenerate univariate normal $N(0,0)$).
\end{defn}
The first theorem that we prove gives an alternative characterization
of the property in terms of characteristic functions.  This result is
sometimes used as the definition of a Gaussian random vector; the only
real benefit to the definition we've given is that it is more
elementary.
\begin{thm}\label{GaussianVectorCharacteristicFunction}A random vector
  $\xi$ in $\reals^N$ is Gaussian if and only if there is a $\mu \in
  \reals^N$ and a symmetric positive semi-definite matrix $Q \in
  \reals^{N\times N}$ such that 
\begin{align*}
\varphi_\xi(u) &= e^{i\langle u, \mu\rangle - \frac{1}{2}\langle u, Qu
\rangle}
\end{align*}
For $\xi$ with characteristic function of this form, $\mu =
\expectation{\xi}$ and $Q = \covariance{\xi}$; we say that $\xi$ is
$N(\mu, Q)$.
\end{thm}
\begin{proof}
First we assume that we have a characteristic function of the above
form.  Let $a \in \reals^N$ and consider the random variable $\langle
a, \xi \rangle$.  Notice that $\langle a, \xi \rangle = a^* \xi$ is a
special case of an affine transformation so we
can apply Lemma
\ref{CharacteristicFunctionOfAffineTransform} to calculate
\begin{align*}
\varphi_{\langle a, \xi \rangle}(u) &= \varphi_\xi(a u) = e^{i u \langle
  a, \mu\rangle - \frac{1}{2}\langle a, Q a\rangle u^2} 
\end{align*}
Now, by Example \ref{FourierTransformGeneralGaussian}
we see that $\langle a, \xi \rangle$ is $N(\langle
  a, \mu\rangle, \langle a, Q a\rangle)$.  Since $a$ was arbitrary,
  this shows that $\xi$ is Gaussian.

Now we assume that $\xi$ is Gaussian.  Let $\mu = (\mu_1, \dots,
\mu_N) = \expectation{\xi}$ and let $Q = \covariance{\xi}$.  Pick $a
\in \reals^N$ and note that 
\begin{align*}
\expectation{\langle a,\xi \rangle} &= \langle a, \mu \rangle \\
\variance{\langle a,\xi \rangle} &= \expectation{(\langle a,\xi
  \rangle - \expectation{\langle a,\xi \rangle})^2} \\
&=  \expectation{(\langle a,\xi - \mu  \rangle)^2} \\
&=  \expectation{a^* (\xi - \mu) (\xi - \mu)^* a} \\
&= a^*\expectation{ (\xi - \mu) (\xi - \mu)^*}a = \langle a, Q a
\rangle \\
\end{align*}
Now we know by our assumption and the expectation and variance
calculation above  that $\langle a, \xi \rangle$ is
$N(\langle a, \mu \rangle, \langle a, Q a \rangle)$ and by Example
\ref{FourierTransformGeneralGaussian}, we have
\begin{align*}
\varphi_{\langle a, \xi \rangle}(u) &= e^{iu \langle a, \mu \rangle -
  \frac{1}{2}\langle a, Q a \rangle u^2}
\end{align*}
As above we can apply Lemma
\ref{CharacteristicFunctionOfAffineTransform}
to see
\begin{align*}
\varphi_\xi(a) &= \varphi_{\langle a, \xi \rangle}(1) =  e^{i \langle a, \mu \rangle -  \frac{1}{2}\langle a, Q a \rangle}
\end{align*}
Together with the fact two measures with the same characteristic
function must be equal (Theorem
\ref{EqualCharacteristicFunctionEqualMeasures} ), this also proves the last part of the Theorem since we have shown by
construction that $\mu = \expectation{\xi}$ and $Q = \covariance{\xi}$.
\end{proof}

\begin{examp}Let $\xi_1, \dots, \xi_N$ be independent random variables
  with $\xi_i$ being normal $N(\mu_i, \sigma_i^2)$.  Then $\xi =
  (\xi_1, \dots, \xi_N)$ is a Gaussian random vector.  In fact, if we
  let $\mu = (\mu_1, \dots, \mu_N)$ and 
\begin{align*}
Q = \diag(\sigma_1^2, \dots, \sigma_N^2)
\end{align*}
then $\xi = N(\mu, Q)$.
\end{examp}

\begin{examp}\label{LinearTransformationGaussian}Let $(\xi_1, \dots, \xi_d)$ be $\xi = N(\mu, Q)$.  Let
  $A$ be a $r \times d$ matrix then $A \xi$ is $N( A \mu, A Q
  A^T)$.

To see this note that if $a \in \reals^d$ then $\langle a , A \xi
\rangle = \langle a^T A, \xi \rangle$ so we see that $A \xi$ is in
fact Gaussian.  The fact that $\expectation{A \xi} =
A\expectation{\xi}$ is immediate from linearity of expectation.  To
calculate the covariance we 
\begin{align*}
\covariance{A\xi} &= \expectation{A(\xi - \mu) (\xi - \mu)^T A^T} =
A\expectation{(\xi - \mu) (\xi - \mu)^T}A^T = 
A Q A^T
\end{align*}
\end{examp}

The characterization of Gaussian random vectors using characteristic
functions allows us to see that almost sure limits of Gaussian random vectors are
Gaussian random vectors.  We will need this result when we construct
Brownian motion later on.
\begin{lem}\label{LimitOfGaussianRandomVectors}Let $\xi_1, \xi_2, \dots$ be a sequence of random
  vectors in $\reals^N$ with $\xi_n$ an $N(\mu_n, C_n)$ Gaussian
  random vector.  Suppose that $\xi$ is a random vector such
  that $\xi_n$ converges to $\xi$ almost surely.  If $\lim_{n \to
    \infty} \expectation{\xi_n} = \mu$ and $\lim_{n \to \infty}
  \covariance{\xi_n} = C$ then $\xi$ is a $N(\mu, C)$ Gaussian random vector.
\end{lem}
\begin{proof}
Since $\xi_n$ converges almost surely to $\xi$ then it converges in
distribution.  We know from Lemma
\ref{GaussianVectorCharacteristicFunction} and the Glivenko-Levy
Continuity Theorem (Theorem \ref{GlivenkoLevyContinuity}) we see
\begin{align*}
\varphi_\xi(u) &= \lim_{n \to \infty} \varphi_{\xi_n}(u) = \lim_{n \to
  \infty} e^{i\langle u, \mu_n\rangle - \frac{1}{2}\langle u, C_n
  y\rangle} = e^{i\langle u, \mu\rangle - \frac{1}{2}\langle u, C
  y\rangle} 
\end{align*}
where we have used continuity of $e^{ix}$.  Thus, using Lemma
\ref{GaussianVectorCharacteristicFunction}
again shows that $\xi$ is $N(\mu, C)$.
\end{proof}

TODO: Gaussian Random Variables in $\reals^n$ and the multidimensional
CLT.  Pretty sure this can be derived from the Cramer Wold device.
TODO: Show that a
given two independent Gaussian random variables their sum and
difference are independent Gaussian (that probably doesn't require
Gaussian random vectors).  Not sure we really need to call this out as a Lemma.

One last thing we need in the sequel are estimates on the tails of
normal random variables.  These results are not required yet nor do
they add anything significant to the conceptual picture so the
reader can safely skip over them and return to them when they are
referenced.

\begin{lem}\label{GaussianTailsElementary}Given an $N(0,1)$ random
  variable $\xi$ we have for all $\lambda > 0$, 
\begin{align*}
\frac{\lambda}{\sqrt{2\pi}(1+\lambda^2)} e^{-\lambda^2/2}&\leq \probability{\xi \geq
    \lambda} \leq \frac{1}{\sqrt{2\pi} \lambda} e^{-\lambda^2/2}
\end{align*}
\end{lem}
\begin{proof}
We start by showing the upper bound 
\begin{align*}
\probability{\xi \geq \lambda} &= \frac{1}{\sqrt{2\pi}} \int_\lambda^\infty
e^{\frac{-x^2}{2}} \, dx 
\leq  \frac{1}{\sqrt{2\pi}} \int_\lambda^\infty
\frac{x}{\lambda} e^{\frac{-x^2}{2}} \, dx 
= \frac{1}{\sqrt{2\pi}\lambda} e^{\frac{-\lambda^2}{2}}
\end{align*}
Interestingly, the lower bound follows from the upper bound.  Define 
\begin{align*}
f(\lambda) &= \lambda e^{-\lambda^2/2} - (1 + \lambda^2)
\int_\lambda^\infty e^{-x^2/2} \, dx
\end{align*}
and notice that $f(0) = -\int_0^\infty e^{-x^2/2} \, dx =
-\frac{\sqrt{2\pi}}{2} < 0$.  Furthermore if we use the upper bound
just proven
\begin{align*}
\lim_{\lambda \to \infty} f(\lambda)
&= \lim_{\lambda \to \infty} \lambda^2 \int_\lambda^\infty e^{-x^2/2}
\, dx \leq \lim_{\lambda \to \infty} \lambda e^{-\lambda^2/2} = 0
\end{align*}
and therefore $\lim_{\lambda \to \infty} f(\lambda) = 0$.  In addition
we have for $\lambda \geq 0$,
\begin{align*}
\frac{d}{d\lambda} f(\lambda) &= e^{-\lambda^2/2} - \lambda^2
e^{-\lambda^2/2} + (1+\lambda^2) e^{-\lambda^2/2} -2\lambda
\int_\lambda^\infty e^{-x^2/2} \, dx \\
&=2\lambda \left(\frac{1}{\lambda}e^{-\lambda^2/2} -
  \int_\lambda^\infty e^{-x^2/2} \, dx \right ) \geq 0
\end{align*}
where the last inequality follows from the upper bound just proven.
This shows that $f(\lambda) \geq 0$ for all $\lambda \geq 0$ and we
are done.
\end{proof}

\section{Laplace Transforms}
It turns out to be useful to specialize characteristic functions for
the case in which we have a measure that is supported on the positive
orthant $\reals^N_+$.  

\begin{defn}Let $\mu$ be a probability measure on $\reals^N_+$. Its
  \emph{Laplace Transform} is denoted $\tilde{\mu}$ and is the function on
  $\reals^N_+$ defined by 
\begin{align*}
\tilde{\mu}(u) &= \int e^{-\langle u,x \rangle} d \mu(x)
\end{align*}
\end{defn}

Next we observe that the behavior of the Laplace transform near zero
corresponds to the behavior of the measure near infinity.
\begin{lem}\label{LaplaceTailEstimate}Let $\mu$ be a probability
  measure on $\reals^N_+$ and let $\mathds{1}=(1, \dotsc, 1) \in
  \reals^N_+$.  Then for each $r > 0$ we have
\begin{align*}
\mu\lbrace \abs{x} \geq r\rbrace &\leq 2(1 - \tilde{\mu}(\mathds{1}/r))
\end{align*}
\end{lem}
\begin{proof}
In order to see how simple the estimate is, first assume that $N=1$.
Observe that because $e^{-ux}$ is a decreasing function of $x$ for $u
> 0$ we have $e^{-ux}  \leq e^{-1} < 1/2$ for all $x
\geq 1/u$ and $e^{-ux} \leq 1$ for all $x \geq 0$.  Therefore for a fixed $r > 0$,
\begin{align*}
\tilde{\mu}(r) &= \int e^{-rx} \, d\mu(x) = \int \characteristic{[0,1/r)}(x)  e^{-rx}
\, d\mu(x)  + \int \characteristic{[1/r,\infty)}(x)  e^{-rx} \,
d\mu(x)  \\
&\leq \mu[0,1/r) + \frac{1}{2} \mu[1/r, \infty) = 1 - \frac{1}{2} \mu[1/r, \infty) 
\end{align*}

To extend to the case of general $N$, we need a little bit more
information.  Note that minimum value of
$\langle \mathds{1}, x\rangle = \sum_{j=1}^N x_j$ on $\reals^N_+ \cap \lbrace \abs{x} \geq u \rbrace$ is
$u$ (it occurs at the points $(0, \dotsc, 0,u,0, \dotsc,0)$).  TODO:
Show this...
Therefore we know that for all fixed $r \in \reals_+$ we have $e^{-\langle r
\mathds{1}, x \rangle} \leq e^{-1} < 1/2$ for all $x \in \reals^N_+$
with $\abs{x} \geq 1/r$.  Now we can playback the same argument as the
case $N=1$:
\begin{align*}
\tilde{\mu}(r \cdot \mathds{1}) &= \int e^{-r\langle \mathds{1}, x \rangle}
\, d\mu(x) \\
&= \int \characteristic{\abs{x} < 1/r}(x)  e^{-r\langle \mathds{1}, x \rangle}
\, d\mu(x)  + \int \characteristic{\abs{x} \geq 1/r}(x)  e^{-r\langle \mathds{1}, x \rangle} \,
d\mu(x)  \\
&\leq \mu \lbrace \abs{x} < 1/r \rbrace + \frac{1}{2} \mu \lbrace
\abs{x} \geq 1/r \rbrace = 1 - \frac{1}{2} \mu \lbrace \abs{x} \geq
1/r \rbrace
\end{align*}
\end{proof}

\begin{lem}\label{LaplaceEquicontinuityAndTightness}Let $\lbrace
  \mu_\alpha \rbrace$
  be a family of probability measures on $\reals^N_+$, then the family
  $\lbrace \mu_\alpha \rbrace$ is tight if and only if the family
  $\lbrace \tilde{\mu}_\alpha \rbrace$ is equicontinuous at $0$.  If
  this is true then $\lbrace \tilde{\mu}_\alpha \rbrace$ is uniformly equicontinuous on all of $\reals^N_+$.
\end{lem}
\begin{proof}
First we assume that the family $\lbrace \mu_\alpha \rbrace$ is
equicontinuous and show tightness.  To do this, note that if $\epsilon
> 0$ is given,
 then by equicontinuity we can find
$\delta>0$ such that $1 - \tilde{\mu}_\alpha(u \mathds{1}) < \epsilon/2$ for all
$0 \leq u < \delta$ and all $\alpha$.  By Lemma
\ref{LaplaceTailEstimate} we get for every $r > 1/\delta$ 
\begin{align*}
\mu_\alpha\lbrace \abs{x} \geq r \rbrace &\leq 2 (1 - \tilde{\mu_\alpha}(\mathds{1}/r)) < \epsilon
\end{align*}
and therefore tightness is proven. 

Now assume that the family $\lbrace \tilde{\mu}_\alpha \rbrace$ is
tight.  For each $\alpha$ let $\xi_\alpha$ be a random vector with
distribution $\mu_\alpha$.  Using the elementary bound $\abs{e^{-x} - e^{-y}} \leq \abs{x
  -y} \wedge 1$ for $0 \leq x,y < \infty$ and Cauchy-Schwartz (Lemma
\ref{CauchySchwartz}) we see that for any
$0 < \epsilon < 2$,
\begin{align*}
\abs{\tilde{\mu}_\alpha(u) - \tilde{\mu}_\alpha(v)} &=
\abs{\expectation{ e^{-\langle u, \xi_\alpha \rangle} - e^{-\langle v,
      \xi_\alpha}}} \\ 
&\leq
\expectation{\abs{ e^{-\langle u, \xi_\alpha \rangle} - e^{-\langle v,
      \xi_\alpha \rangle}}} \\ 
&\leq \expectation{\abs{\langle u-v, \xi_\alpha \rangle} \wedge 1} \\ 
&= \expectation{\abs{\langle u-v, \xi_\alpha \rangle} \wedge 1 ;
  \langle u-v, \xi_\alpha \rangle < \epsilon/2}  + 
\expectation{\abs{\langle u-v, \xi_\alpha \rangle} \wedge 1 ;
  \langle u-v, \xi_\alpha \rangle \geq \epsilon/2} \\ 
&\leq \epsilon/2 + 
\probability{\langle u-v, \xi_\alpha \rangle \geq \epsilon/2} \\ 
&\leq \epsilon/2 + 
\probability{\abs{\xi_\alpha } \geq \frac{\epsilon}{2\abs{u-v}}} \\ 
\end{align*}
Thus by tightness for all $u,v \in \reals^N_+$ with $\abs{u-v}$
sufficiently small we have $\abs{\tilde{\mu}_\alpha(u) -
  \tilde{\mu}_\alpha(v)} < \epsilon$ uniformly in $\alpha$.  Thus we see that
the family
$\lbrace \tilde{\mu}_\alpha \rbrace$ is uniformly equicontinuous on
all of $\reals^N_+$ and in particular at $0$.
\end{proof}

The following result is analogous to the Glivenko-Levy Continuity
Theorem \ref{GlivenkoLevyContinuity} for characteristic functions.  As
with that result, here we point out that the assumption that $\mu_n$
converge to probability measure is critical and we will return to the
question of how to remove the assumption (using the notion of tightness) later on.
\begin{thm}[Glivenko-Levy Continuity
  Theorem]\label{GlivenkoLevyContinuityLaplace}If $\mu, \mu_1, \mu_2, \dots$
  are probability measures on $(\reals^N_+, \mathcal{B}(\reals^N_+))$,
  then $\mu_n$ converge weakly to $\mu$ if and only if $\tilde{\mu}_n(u)$
  converge to $\tilde{\mu}(u)$ pointwise.  Moreover if this is true
  then the convergence is uniform on bounded sets.
\end{thm}
\begin{proof}
Since $\mu_n$ converge to $\mu$ weakly and $e^{-\langle u,x \rangle}$
is bounded and continuous we know that $\tilde{\mu}_n(u) \to
\tilde{\mu}(u)$ pointwise.  In fact, by Lemma
\ref{WeakConvergenceImpliesTight} we know that the family $\mu_n$
is tight and therefore by Lemma
\ref{LaplaceEquicontinuityAndTightness} it is uniformly equicontinuous
on $\reals^N_+$.  this convergence is uniform on
every bounded set.

Now we assume that $\tilde{\mu}_n(u)$converges to $\tilde{\mu}(u)$ for
every $u \in \reals^N_+$.  We now want to approximate general bounded
continuous functions by functions $e^{-\langle u,x \rangle}$ in order
derive weak convergence.  To do this, we will consider $[0,\infty]^n$
which is a compact Hausdorff space and therefore amenable to
application of the Stone Weierstrass Theorem
\ref{StoneWeierstrassApproximation}.  To use an approximation of
functions on $[0,\infty]^n$ derive an effective approximation on
$\reals^N_+$ is going to require that we are able to control behavior
of the measures $\mu_n$ at infinity and therefore we first show that
$\mu_n$ is a tight family.  Suppose $\epsilon > 0$ is given and use
the continuity of $\tilde{\mu}(u)$ and the fact that
$\tilde{\mu}(0)=1$ to find $r_0 > 0$ such that $1 - \mu(\mathds{1}/r_0) <
\epsilon/2$.  By pointwise convergence $\mu_n(\mathds{1}/r_0) \to \mu(\mathds{1}/r_0)$ we
can find an $N > 0$ such that $1 - \tilde{\mu_n}(\mathds{1}/r_0) < \epsilon$ for all $n
\geq N$ and therefore by Lemma \ref{LaplaceTailEstimate}, $\mu_n(\abs{x} \geq r)  \leq 1 - \mu_n(\mathds{1}/r_0) <
\epsilon$.  For each $n = 1, \dotsc, N-1$ by continuity of measure  we
can find $r_n > 0$ such that $\mu_n\lbrace \abs{x} \geq r_n \rbrace <
\epsilon$.  Therefore taking the maximum $r = r_0 \vee r_1 \vee \dotsb \vee
r_{N-1}$ we get $\mu_n\lbrace \abs{x} \geq r \rbrace <
\epsilon$ for all $n$ and we have shown $\mu_n$ is tight.

Suppose that $\epsilon > 0$ is given and
pick $r > 0$ such that $\mu_n(\reals^N_+ \setminus B(0, r)) <
\epsilon$ and $\mu(\reals^N_+ \setminus B(0, r)) < \epsilon$.

Having shown that $\mu_n$ is tight we return to the task of creating
an approximation.  Since $e^{-\langle u,x \rangle}$
has limits (either $0$ or $1$) as $x \to \infty$ we can extend each
such function to
a continuous function on $[0,\infty]^n$.  Note also that the family $e^{-\langle k, x
  \rangle}$ for $k \in \integers^n_+$ contains the constant functions
and separates points therefore we can apply the Stone Weierstrass
Theorem to conclude that any
continuous function on $[0,\infty]^n$ can be uniformly approximated by
a linear combination of $e^{-\langle k, x  \rangle}$. 

Given a bounded continuous function $f : \reals^N_+ \to \reals$ such
that $\abs{f(x)} \leq M$ we
apply a continuous cutoff $1 - d(x,B(0,r)) \vee 0$ to create a function $\hat{f} : \reals^N_+
\to \reals$ such that $\hat{f}(x) \leq M$ for all $x \in \reals^N_+$, $f(x) = \hat{f}(x)$ for $x \in B(0,r)$ and
$\hat{f}(x) = 0$ for $\abs{x} > 2r$.  Note that for every $n$ we have
\begin{align*}
\int \abs{f - \hat{f}} \, d\mu_n &= 
\int_{\abs{x} < r} \abs{f -  \hat{f}} \, d\mu_n + 
\int_{\abs{x} \geq r} \abs{f -  \hat{f}} \, d\mu_n < 2 M \epsilon
\end{align*}
and similarly for $\mu$.

The function $\hat{f}$ can be
extended by zero to define a continuous function on $[0,\infty]^n$ and
therefore we can find some finite linear combination $g = \sum_k c_k
e^{-\langle k, x \rangle}$ such that $\abs{\hat{f}(x) - g(x)}<
\epsilon$ for all $x \in [0, \infty]^n$ so \emph{a fortiori} for all $x \in \reals^N_+$.  Therefore
\begin{align*}
&\abs{\int f \, d\mu_n - \int f \, d \mu} \\
&\leq \int \abs{f - \hat{f}}
\, d\mu_n + \int \abs{\hat{f} - g}\, d\mu_n + \abs{\int  g\, d\mu_n -
  \int g \, d\mu} + \int \abs{\hat{f} - g}\, d\mu + \int \abs{f - \hat{f}}
\, d\mu \\
&\leq 2M \epsilon + \epsilon + \abs{\sum_k c_k ( \mu_n(k) - \mu(k))} +
\epsilon + 2M \epsilon
\end{align*}
Now take the limit as $n \to \infty$ and use the fact that $\mu_n \to
\mu$ pointwise (recall that the above sum
over $k$ is finite) and then let $\epsilon \to 0$.
\end{proof}

The Cramer-Wold device for Laplace transforms is a simple corollary.
\begin{cor}[Cramer Wold Device]\label{CramerWoldDeviceLaplace}Let $\xi, \xi_1, \xi_2, \dotsc$ be 
  random vectors in $\reals^N_+$.  If $\langle c, \xi_n \rangle
  \todist \langle c, \xi \rangle$ for all $c \in \reals^N_+$ then it
  follows that $\xi_n \todist \xi$.
\end{cor}
\begin{proof}
Since  $\langle c, \xi_n \rangle  \todist \langle c, \xi \rangle$ we
know that that $\expectation {e^{-\langle c, \xi_n \rangle } } \to
\expectation{e^{-\langle c, \xi \rangle}}$ for all $c \in \reals^N_+$
by definition of weak convergence
and therefore by Theorem \ref{GlivenkoLevyContinuityLaplace} we
conclude $\xi_n \todist \xi$.
\end{proof}

\chapter{Conditioning}

\section{$L^p$ Spaces}
Prior to discussing the general formulation of the notion of
conditional probabilities we shall need to lay down some techniques of
functional analysis pertaining to spaces of measurable (and
integrable) random variables.

\begin{defn}Given a measure space $(\Omega, \mathcal{A}, \mu)$ and $p
  \geq 1$ we let $L^p(\Omega, \mathcal{A}, \mu)$ be the space of equivalence
  classes of measurable functions such that $\int \abs{f}^p \, d\mu <
  \infty$ under the equivalence relation of
  almost everywhere equality.  For any element $f \in   L^p(\Omega,
  \mathcal{A}, \mu)$ we define 
\begin{align*}
\norm{f}_p &= \left ( \int \abs{f}^p \, d\mu \right )^{\frac{1}{p}}
\end{align*}
\end{defn}

It is clear that the spaces $L^p(\Omega, \mathcal{A}, \mu)$ but our first goal is to establish that each is a complete
normed vector space (a.k.a. Banach space).  As our first step in that
direction we need to prove the triangle inequality
\begin{lem}[Minkowski Inequality]\label{MinkowskiInequality}Given $f,g
  \in L^p(\Omega,  \mathcal{A}, \mu)$ then $f+g \in L^p(\Omega,
  \mathcal{A}, \mu)$ and $\norm{f+g}_p \leq \norm{f}_p + \norm{g}_p$.
\end{lem}
\begin{proof}
Note that it suffices to assume that $f \geq 0$ and $g \geq 0$ since
if we have the inequality for positive elements then it follows for
all elements by applying the ordinary triangle inequality on $\reals$
and using the fact that $x^p$ is increasing to see
\begin{align*}
\norm{f+g}_p &\leq \norm{\abs{f}+\abs{g}}_p \leq
\norm{\abs{f}}_p+\norm{\abs{g}}_p = \norm{f}_p+\norm{g}_p 
\end{align*}
The case $p=1$ follows immediately from linearity of integral (in fact
we have equality).  

For $1 < p < \infty$, first use the following
crude pointwise bound to see that $f+g \in L^p(\Omega,  \mathcal{A}, \mu)$:
\begin{align*}
(f+g)^p &\leq (f \vee g + f \vee g)^p = 2^p (f^p \vee g^p)\leq 2^p (f^p
+ g^p)
\end{align*}
and therefore $\norm{f+g}_p^p \leq 2^p (\norm{f}_p^p + \norm{g}_p^p) <
\infty$.  To see the triangle inequality, note that we can assume that
$\norm{f+g}_p > 0$ for otherwise the triangle inequality follows by
positivity of the norm.  Write
\begin{align*}
\norm{f+g}_p^p &= \int (f+g)^p \, d\mu = \int f (f+g)^{p-1} \, d\mu +
\int g (f+g)^{p-1} \, d\mu \\
\end{align*}
Now we can apply the H\"{o}lder Inequality (Lemma
\ref{Holder}) to each of the terms on the right hand side and use the
fact that $\frac{1}{p} + \frac{1}{q}=1$ is equivalent to $p = (p-1)q$ to see
\begin{align*}
\int f (f+g)^{p-1} \, d\mu &\leq \left(\int f^p \,
  d\mu\right)^\frac{1}{p}\left(\int (f+g)^{(p-1)q} \,
  d\mu\right)^\frac{1}{q} =\norm{f}_p \norm{f+g}_p^{p/q}
\end{align*}
Applying this argument to the term $\int g (f+g)^{p-1} \, d\mu$ as
well we get
\begin{align*}
\norm{f+g}_p^p &\leq  (\norm{f}_p + \norm{g}_p) \cdot
\norm{f+g}_p^{p/q}
\end{align*}
and dividing through by $\norm{f+g}_p^{p/q}$ and using $p - \frac{p}{q}=1$ we get $\norm{f+g}_p \leq  \norm{f}_p + \norm{g}_p$.
\end{proof}

\begin{lem}\label{CompletenessOfLp}For $p \geq 1$ the normed vector
  space $L^p(\Omega,  \mathcal{A}, \mu)$ is complete.
\end{lem}
\begin{proof}
Let $f_n$ be a Cauchy sequence in $L^p(\Omega,  \mathcal{A}, \mu)$.
The first step of the proof is to show that there is a subsequence of
$f_n$ that converges almost everywhere to an element $f \in
L^p(\Omega,  \mathcal{A}, \mu)$.

By the Cauchy property, for each $j \in \naturals$ we can find an $n_j > 0$ such that
$\norm{f_m - f_{n_j}}_p \leq \frac{1}{2^j}$ for all $m > n_j$.  In
this way we get a subsequence $f_{n_j}$ such that $\norm{f_{n_{j+1}} -
    f_{n_j}}_p \leq \frac{1}{2^j}$ for all $j \in \naturals$.  Now by
  applying Monotone Convergence and the triangle inequality we have
\begin{align*}
\norm{\sum_{j=1}^\infty \abs{f_{n_{j+1}} - f_{n_j}} }_p &= \lim_{N \to
  \infty} \norm{\sum_{j=1}^N \abs{f_{n_{j+1}} - f_{n_j}} }_p \\
&\leq \lim_{N \to
  \infty} \sum_{j=1}^N \norm{f_{n_{j+1}} - f_{n_j}}_p \\
&\leq \lim_{N \to
  \infty} \sum_{j=1}^N \frac{1}{2^j} < \infty
\end{align*}
and therefore we know that $\sum_{j=1}^\infty \abs{f_{n_{j+1}} -
  f_{n_j}}$ is almost surely finite.  Anywhere this sum is finite it
follows that $f_{n_j}$ is a Cauchy sequence in $\reals$.  To see this,
suppose we are given
$\epsilon > 0$ we pick $N > 0$ such that $\sum_{j=N}^\infty \abs{f_{n_{j+1}} -
  f_{n_j}} < \epsilon$, then for any $k \geq j \geq N$ we have 
\begin{align*}\abs{f_{n_k} -
  f_{n_j}} = \abs{\sum_{m=j}^k (f_{n_{m+1}} -  f_{n_m})} \leq
\sum_{m=j}^k \abs{ f_{n_{m+1}} -  f_{n_m}} < \epsilon
\end{align*}

We know that the set where $f_{n_j}$ converges is measurable (TODO:
Where is this?) so we can
define $f$ to be the limit of the Cauchy sequence $f_{n_j}$ where
valid and define it to be zero elsewhere (a set of measure zero).

To see that $f \in L^p(\Omega,  \mathcal{A}, \mu)$ and to show that
$f_n$ converges to $f$, suppose $\epsilon > 0$ is given and pick $N
\in \naturals$ such that for all $m,n \geq N$ we have
$\norm{f_m-f_n}_p < \epsilon$.  Now we can use Fatou's
Lemma (Theorem \ref{Fatou})  to see for any $n \geq N$, 
\begin{align*}
\int \abs{f-f_n}^p \, d\mu &\leq \liminf_{j \to \infty} \int
\abs{f_{n_j} - f_n}^p \, d\mu \leq \sup_{m\geq n} \int
\abs{f_{m} - f_n}^p \, d\mu < \epsilon^p
\end{align*}
Therefore by the Minkowski Inequality, we see that $f = f_n + (f -
f_n)$ is in  $L^p(\Omega,  \mathcal{A}, \mu)$ and $f_n \tolp{p} f$.
\end{proof}

We know that measurable functions can be approximated by simple
functions (Lemma \ref{PointwiseApproximationBySimple}) with pointwise
convergence.  It is useful to extend this approximation to $L^p$ spaces.
\begin{lem}\label{LpApproximationBySimple}Simple functions are dense in $L^p(\Omega, \mathcal{A}, \mu)$.
\end{lem}
\begin{proof}
Pick a positive function $f\in L^p(\Omega, \mathcal{A}, \mu)$ and sequence of simple functions
such that $0 \leq f_n \uparrow f$.  Then it is also true that $f_n^p \uparrow
f^p$ and Dominated Convergence tells us that $\lim_{n \to \infty}
\norm{f_n}_p  = \norm{f}_p$.  By Lemma \ref{ConvergenceInMeanConvergenceOfMeans} we conclude that $f_n
\tolp{p} f$.

To finish the proof, take an arbitrary $f$ and write it as $f = f_+ -
f_-$.  Now take positive
simple functions $g_n \uparrow f_+$ and $h_n \uparrow f_-$  and use the
triangle inequality to see that
\begin{align*}
\lim_{n \to \infty} \norm{f - (g_n - h_n)}_p &\leq \lim_{n \to \infty}
(\norm{f_+ - g_n}_p + \norm{f_- - h_n}_p ) = 0
\end{align*} 
\end{proof}

Note that for any $\sigma$-algebra $\mathcal{F} \subset \mathcal{A}$
we can also consider the space $ L^p(\Omega,  \mathcal{F}, \mu)$.  
As we shall soon see, it will become important to understand a bit
about these spaces as $\mathcal{F}$ vary.  The first thing to note is
that for $\mathcal{G} \subset \mathcal{F}$, $L^p(\Omega,  \mathcal{G},
\mu)$ is a closed linear subspace of $L^p(\Omega,  \mathcal{F},
\mu)$.  The inclusion is trivial since any $\mathcal{G}$-measurable
function is also $\mathcal{F}$-measurable; closure follows from the
completeness of the space $L^p(\Omega, \mathcal{G}, \mu)$ (Lemma
\ref{CompletenessOfLp}).

The following approximation result will be used only occasionaly.
\begin{lem}\label{LpDensityUnionSubsigmaAlgebras}$\cup_n L^p(\Omega, \mathcal{F}_n, \mu)$ is dense in
  $L^p(\Omega, \bigvee_n \mathcal{F}_n, \mu)$
\end{lem}
\begin{proof}The first thing to show the result for indicator
  functions.  A general fact, suppose $V$ is a closed linear subspace of
  $L^p$ and let $\mathcal{C} = \lbrace A \mid \characteristic{A} \in
   V\rbrace$.  We claim that $\mathcal{C}$ is a $\lambda$-system.
    Given $A, B \in \mathcal{C}$ with $A \subset B$, we have $B
    \setminus A \in \mathcal{C}$ since $\characteristic{B \setminus A}
    = \characteristic{B} - \characteristic{A}$ and $V$ is a linear
    space.  Now assume that $A_1 \subset A_2 \subset \cdots \in
    \mathcal{C}$.  We have that $\characteristic{A_n} \uparrow
    \characteristic{A}$ and continuity of measure (Lemma
    \ref{ContinuityOfMeasure}) tells us that
    $\lim_{n \to \infty} \norm{\characteristic{A_n}}_p =
    \norm{\characteristic{A}}_p$ so Lemma
    \ref{ConvergenceInMeanConvergenceOfMeans} implies $\characteristic{A_n}
    \tolp{p} \characteristic{A}$.  Since $V$ is closed we know
    $\characteristic{A} \in V$.
\end{proof}

\begin{lem}\label{LpApproximationByContinuous}Let $S$ be a metric
  space and let $\mu$ be a finite Borel measure on $S$.  Then the
 space of bounded continuous functions is dense in $L^p(S,
  \mathcal{B}(S), \mu)$.
\end{lem}
\begin{proof}
Note that the finiteness of $\mu$ guarantees that any bounded
measurable function is also in $L^p(S, \mathcal{B}(S), \mu)$ so the
proof will focus on establishing boundedness of functions involved and
not concern itself with verifying $p$-integrability.  Suppose we have $U \subset S$ an open set.  Let
$f_n(x) = (nd(x,U^c)) \wedge 1$.  We know that $f_n(x)$ is 
increasing, bounded and continuous with $\lim_{n \to \infty} f_n(x) =
\characteristic{U}(x)$ and therefore $ f^p_n(x) \uparrow
\characteristic{U}$ as well.  By Monotone Convergence we have $\lim_{n \to
  \infty} \norm{f_n}_p^p  = \mu(U) = \norm{\characteristic{U}}_p^p$
hence $f_n \tolp{p} \characteristic{U}$.  Now we
extend to general Borel sets $A$ by a monotone class argument.  We claim
that 
\begin{align*}
\mathcal{C} &= \lbrace A \in \mathcal{B}(S) \mid \text{ there
  exist bounded continuous } f_n \text{ such that } f_n \tolp{p}
\characteristic{A} \rbrace
\end{align*}
is a $\lambda$-system.  Supposing $A \subset B$ with $A,B \in
\mathcal{C}$ we get bounded continuous $f_n$ such that $f_n \tolp{p}
\characteristic{A}$ and bounded continuous $g_n$ such that $g_n \tolp{p}
\characteristic{B}$ by Lemma \ref{LpConvergenceUniformIntegrability}.
Then
\begin{align*}
\lim_{n \to \infty} \norm{\characteristic{B\setminus A} - (g_n
  -f_n)}_p & \leq \lim_{n \to \infty} \norm{\characteristic{B} -
  g_n}_p   + \lim_{n \to \infty} \norm{\characteristic{A} -
  f_n}_p = 0
\end{align*}
and therefore $B \setminus A \in \mathcal{C}$.  
If $A_1 \subset A_2 \subset \dotsb$ with $A_n \in \mathcal{C}$ then 
\begin{align*}
\lim_{n \to \infty} \norm{\characteristic{A_n}}_p &= \lim_{n \to
  \infty} \mu(A_n)^{1/p} = \mu(\cup_{n=1}^\infty A_n) ^{1/p} = \norm{\characteristic{\cup_{n=1}^\infty A_n}}_p
\end{align*}
Now for each $A_n$ there exists a sequence bounded continuous $f_{n,m}$ with $\lim_{m \to
  \infty} \norm{\characteristic{A_n} - f_{n,m}}_p = 0$.  Now we can
find a subsequence $f_{n,m_n}$ such that $\lim_{n \to \infty}
\norm{\characteristic{\cup_{n=1}^\infty A_n} - f_{n,m_n}}_p = 0$ which
shows $\cup_{n=1}^\infty A_n \in \mathcal{C}$.  Now as open sets are
clearly a $\pi$-system the
$\pi$-$\lambda$ Theorem \ref{MonotoneClassTheorem} shows that
$\mathcal{B}(S) \subset \mathcal{C}$.  Now for any simple function $f
= \sum_{j=1}^m c_j \characteristic{A_j}$ we can find $f_{j,n}$ such
that $\lim_{n \to \infty} \norm{\characteristic{A_j} - f_{j,n}}_p = 0$
and by the triangle inequality
\begin{align*}
\lim_{n \to \infty} \norm{f - \sum_{j=1}^m c_j f_{j,n}}_p &\leq
\lim_{n \to \infty} \sum_{j=1}^m \abs{c_j} \norm{\characteristic{A_j}
  - f_{j,n}}_p = 0
\end{align*}
and the fact that each $\sum_{j=1}^m c_j f_{j,n}$ is bounded and continuous is clear.

The last step is to use the fact that simple functions are dense in
$L^p(S, \mathcal{B}(S), \mu)$ (Lemma \ref{LpApproximationBySimple}).
\end{proof}

TODO: Pretty sure the above result will have an extension to
$\sigma$-finite case.

TODO: Develop inner product and projection for $L^2$ spaces.

\section{Conditional Expectation}
 Before getting into the technical details we want to get set the
intuition for the problem and the form that solutions will take.
Given a random element $\xi$ in $S$ and a random variable $\eta$, we want to
formulate the notion of the expected value of $\eta$ given a value of
$\xi$.  The immediate way to think of representing such an object is as a map from
$S$ to $\reals$.  In practice the representation is expressed in a
different but equivalent way.  Recall from Lemma
\ref{FunctionalRepresentation} that any random variable
$\gamma$ that is $\xi$-measurable can be factored as $f \circ
\xi$ for some measurable $f : S \to \reals$.  In this way the
conditional expectation may equally be considered as $\xi$-measurable
random variable.  It is this latter representation that is most
convenient for working with (and constructing) conditional
expectations.  To remove matters a little further from the initial
intuition, one often makes use of the fact that the conditional
expection winds up only depending on the $\sigma$-field induced by
$\xi$ and discusses conditioning with respect to arbitrary sub
$\sigma$-fields.

TODO: Elaborate on the three faces of conditional expectation:
projection, density/Radon-Nikodym derivative and disintegration.

Existence via Radon-Nikodym.  The Radon-Nikodym theorem (Theorem
\ref{RadonNikodym}) can
be given a martingale proof (hence derived in some sense from the
existence of conditional expectations).  However, the standard proof
for Radon-Nikodym using
Hahn Decomposition does not depend on the existence of conditional
expection and in fact, the Radon-Nikodym theorem can easily be used to
prove the existence of conditional expectations.
Given $\xi \geq 0$ and $\mathcal{F} \subset \mathcal{A}$, then define
the probability measure $\nu(A) =
\expectation{\xi \characteristic{A}}$.  Note that $\nu$ is absolutely
continuous with respect to $\mu$ on $\mathcal{F}$.  Therefore, the Radon-Nikodym
derivative with respect to $(\Omega, \mathcal{F})$ exists and
satisfies 
\begin{align*}
\nu(A) = \expectation{\xi \characteristic{A}} =
\expectation{\frac{d\nu}{d\mu} \characteristic{A}}
\end{align*}
for all $A \in \mathcal{F}$.  This equality shows that
$\frac{d\nu}{d\mu}$ is a conditional expectation of $\xi$.  For
general $\xi$, write $\xi=\xi_+ - \xi_-$ and proceed as above.
 
TODO: Make sure we have covered the following:  Definition of $L^p$
spaces, completeness of $L^p$ spaces, definition of Hilbert space,
orthogonal projections in Hilbert spaces.  Density of $L^2$ in $L^1$.
Unique extension of a bounded linear operator from a dense subspace
of a complete normed linear space.

On the other hand, there is very appealing construction of conditional
expectation using function spaces that we provide here.  Recall that
for a measurable space $(\Omega, \mathcal{A}, \mu)$ we have associated
Banach spaces of $p$-integrable functions $L^p(\Omega, \mathcal{A}, \mu)$ with norm $\norm{f}_p =
\left ( \int \abs{f}^p \, d \mu \right ) ^ \frac{1}{p}$.  In the
special case $p=2$ we actually have a Hilbert space $L^2(\Omega,
\mathcal{A}, \mu)$ with inner product $<f, g> = \int f g \, d \mu$.
Suppose we have a sub $\sigma$-algebra $\mathcal{F} \subset
\mathcal{A}$ and we have a canonical inclusion $L^p(\Omega, \mathcal{F},
\mu) \subset L^p(\Omega, \mathcal{A},
\mu)$ as a subspace.  In fact by the completeness of $L^p(\Omega,
\mathcal{F},\mu)$, we know that this is a \emph{closed} subspace.
Therefore if we specialize to the case of $L^2(\mathcal{F}) \subset
L^2(\mathcal{A})$ then we have the orthogonal projection onto
$L^2(\mathcal{F})$.  For square integrable random variables, this
orthogonal projection defines the conditional expectation.  In the
following, we extend this defintion to all integrable random variables
and prove the basic properties.

TODO: Elaborate on the ``a.s. uniqueness'' in the definition.

\begin{thm}[Conditional Expectation]\label{ConditionalExpectation}For
  any $\mathcal{F} \subset \mathcal{A}$ there exists a unique linear
  operator $\cexpectationop{\mathcal{F}} : L^1 \to L^1(\mathcal{F})$
  such that 
\begin{itemize} 
\item[(i)]$\expectation{\cexpectation{\mathcal{F}}{\xi} ; A} = \expectation{\xi ;
    A}$ for all $\xi \in L^1$, $A \in \mathcal{F}$
\end{itemize}
The following properties also hold for $\xi, \eta \in L^1$,
\begin{itemize}
\item[(ii)]$\expectation{\abs{\cexpectation{\mathcal{F}}{\xi}}} \leq \expectation{\abs{\xi}}$ a.s.
\item[(iii)]$\xi \geq 0$ implies $\cexpectation{\mathcal{F}}{\xi} \geq
  0$ a.s.
\item[(iv)]$0 \leq \xi_n \uparrow \xi$ implies
  $\cexpectation{\mathcal{F}}{\xi_n}  \uparrow
  \cexpectation{\mathcal{F}}{\xi}$ a.s.
\item[(v)]$\cexpectation{\mathcal{F}}{\xi \eta} = \xi
  \cexpectation{\mathcal{F}}{\eta}$ if $\xi$ is
  $\mathcal{F}$-measurable and $\xi\eta,
  \xi\cexpectation{\mathcal{F}}{\eta} \in L^1$
\item[(vi)]$\expectation{\cexpectation{\mathcal{F}}{\xi}
    \cdot \cexpectation{\mathcal{F}}{\eta}} = \expectation{\xi \cdot
    \cexpectation{\mathcal{F}}{\eta}} = \expectation{
    \cexpectation{\mathcal{F}}{\xi} \cdot \eta} $
\item[(vii)]$\cexpectation{\mathcal{F}}{\cexpectation{\mathcal{G}}{\xi}}
= \cexpectation{\mathcal{F}}{\xi}$ a.s. for all $\mathcal{F} \subset \mathcal{G}$.
\end{itemize}
\end{thm}
\begin{proof}
Begin by defining $\cexpectationop{\mathcal{F}} : L^2 \to
L^2(\mathcal{F})$ as orthogonal projection.  If we pick $A \in
\mathcal{F}$, then $\characteristic{A} \in L^2(\mathcal{F})$ and
therefore, $\xi - \cexpectation{\mathcal{F}}{\xi} \perp
\characteristic{A}$ which shows
\begin{align*}
\expectation{\xi ; A} &= <\xi, \characteristic{A}> =
<\cexpectation{\mathcal{F}}{\xi}, \characteristic{A}> =
\expectation{\cexpectation{\mathcal{F}}{\xi} ; A}
\end{align*}
If we define $A = \lbrace \cexpectation{\mathcal{F}}{\xi} \geq 0
\rbrace$ the above implies
\begin{align*}
\expectation{\abs{\cexpectation{\mathcal{F}}{\xi}}} &=
\expectation{\cexpectation{\mathcal{F}}{\xi} ; A} -
\expectation{\cexpectation{\mathcal{F}}{\xi} ; A^c} & & \text{by
  linearity of expectation}\\
&= \expectation{\xi ; A} - \expectation{\xi ; A^c} & &\text{by (i)} \\
&\leq \expectation{\abs{\xi}; A} + \expectation{\abs{\xi}; A^c} & &
\text{since $\xi \leq \abs{\xi}$ and 
  $-\xi \leq \abs{\xi}$} \\
&= \expectation{\abs{\xi}} & & \text{by linearity of expecation}
\end{align*}
This inequality shows us that the linear operator
$\cexpectationop{\mathcal{F}}$ is bounded in the $L^1$ norm as well as
in the $L^2$ norm.  On the other hand, we know that $L^2$ is dense in
$L^1$ and $L^1$ is complete so there is a unique extension of $\cexpectationop{\mathcal{F}}$
to a bounded linear operator $L^1 \to L^1{\mathcal{F}}$.  Concretely,
for any $\xi \in L^1$, we pick a sequence $\xi_n \in L^2$ such that
$\lim_{n \to \infty} \xi_n \to \xi$ in the $L^1$ norm and define
$\cexpectation{\mathcal{F}}{\xi} = \lim_{n \to \infty}
\cexpectation{\mathcal{F}}{\xi_n}$ where the limit is in the $L^1$
norm.  Since the $L^1$ closure of $L^2(\mathcal{F})$ is
$L^1(\mathcal{F})$, we see that the definition is plausible.  

TODO: Show independence, linearity and boundedness of the extension.
Perhaps factor this out into a separate Lemma; it is a generic
construction.

To see that the condition (i) uniquely defines
$\cexpectation{\mathcal{F}}{\xi} $ a.s., suppose we had two
$\mathcal{F}$-measurable random variables $\eta$ and $\rho$ for which
$\expectation{\eta ; A} = \expectation{\rho ; A}$ for all $A \in
\mathcal{F}$.  Let $A = \lbrace \eta > \rho \rbrace$ which is
$\mathcal{F}$-measurable and so we have assumed
$\expectation{\eta - \rho ; A} = 0$.  If we apply Lemma \ref{ZeroIntegralImpliesZeroFunction} we
know that $(\eta - \rho)\characteristic{A} = 0$ a.s. which shows that
$\probability{A} = 0$.   The same argument shows that
$\rho > \eta$ with probability $0$, hence $\eta = \rho$ a.s.

To see (iii), let $A = \lbrace \cexpectation{\mathcal{F}}{\xi} < 0 \rbrace$ and
observe that 
\begin{align*}
0 &\leq \expectation{-\cexpectation{\mathcal{F}}{\xi} ; A} =
\expectation{-\xi ; A} \leq 0
\end{align*}
and therefore $\expectation{-\cexpectation{\mathcal{F}}{\xi} ; A} = 0$
which applying Lemma \ref{ZeroIntegralImpliesZeroFunction} implies
$\probability{A}=0$.

 To see (iv), suppose $0 \leq \xi_n \uparrow \xi$ a.s.  Then by Monotone
 Convergence, $\lim_{n \to \infty} \expectation{\abs{\xi - \xi_n}} =
 0$.  Now by (ii) and linearity of conditional expection, 
\begin{align*}
0 \leq \lim_{n \to \infty} \expectation{\abs{\cexpectation{\mathcal{F}}{\xi}
  - \cexpectation{\mathcal{F}}{\xi_n}} } \leq \lim_{n \to \infty} \expectation{\abs{\xi - \xi_n}} =
 0
\end{align*}
which shows that $\cexpectation{\mathcal{F}}{\xi_n}$ converges to
$\cexpectation{\mathcal{F}}{\xi}$ in $L^1$.  Now by Lemma
\ref{ConvergenceInMeanImpliesInProbability} this
implies that the converges is in probability and by Lemma \ref{ConvergenceInProbabilityAlmostSureSubsequence} there is a
subsequence that converges a.s.  By (iii) we know that $\cexpectation{\mathcal{F}}{\xi_n}$
 is non-decreasing so we know by Lemma \ref{IncreasingSequenceWithConvergentSubsequence} that that almost sure convergence of the
 subsequence extends to the almost sure convergence of the entire sequence.


To see (v), note that if $\xi$ is $\mathcal{F}$-measurable then for
every $\eta \in L^1$, we know $\xi\cexpectation{\mathcal{F}}{\eta}$ is
$\mathcal{F}$-measurable and by simple calculation
\begin{align*}
\expectation{\xi\cexpectation{\mathcal{F}}{\eta}; A} &= \expectation{\xi\eta; A}
\end{align*}
by the apply the extension of the property (i) to the
$\mathcal{F}$-measurable function
$\xi\characteristic{A}$.  Now by (v) follows by applying (i) again.

For the property (vi), by symmetry we only have to prove $\expectation{\cexpectation{\mathcal{F}}{\xi}
    \cdot \cexpectation{\mathcal{F}}{\eta}} = \expectation{\xi \cdot
    \cexpectation{\mathcal{F}}{\eta}}$.  To prove this first assume
  that $\xi, \eta \in L^2$.  In that case, we know that
  $\cexpectation{\mathcal{F}}{\eta}
  \in L^2(\mathcal{F})$ and $\xi - \cexpectation{\mathcal{F}}{\xi}
  \perp L^2(\mathcal{F})$, so 
\begin{align*}
\expectation{\cexpectation{\mathcal{F}}{\xi}
    \cdot \cexpectation{\mathcal{F}}{\eta}} &= <\cexpectation{\mathcal{F}}{\xi}
    ,\cexpectation{\mathcal{F}}{\eta}> \\
&= <\cexpectation{\mathcal{F}}{\xi} - \xi, \cexpectation{\mathcal{F}}{\eta}> + <\xi, \cexpectation{\mathcal{F}}{\eta}>\\
&= <\xi, \cexpectation{\mathcal{F}}{\eta}> = \expectation{\xi \cdot \cexpectation{\mathcal{F}}{\eta}}\\
\end{align*}
Now by the density of $L^2 \subset L^1$, for general $\xi, \eta \in
L^1$ we pick $\xi_n \tolp{1} \xi$ and $\eta_n \tolp{1} \eta$ with
$\xi_n, \eta_n \in L^2$.  By the above 
Lastly, we prove (vii).  Suppose we are given $\sigma$-algebras
$\mathcal{F} \subset \mathcal{G}$.  Then for $A \in \mathcal{F}
\subset \mathcal{G}$,
\begin{align*}
\expectation{\cexpectation{\mathcal{G}}{\xi} ; A} &= \expectation{\xi
  ; A} & & \text{by (i) applied to
  $\cexpectation{\mathcal{G}}{\xi}$}\\
&= \expectation{\cexpectation{\mathcal{F}}{\xi}
  ; A} & & \text{by (i) applied to
  $\cexpectation{\mathcal{F}}{\xi}$}\\
\end{align*}
where are the equalities are a.s.   By definition $\cexpectation{\mathcal{F}}{\xi}$ is
$\mathcal{F}$-measurable which shows by (i) that
$\cexpectation{\mathcal{F}}{\cexpectation{\mathcal{G}}{\xi}}
= \cexpectation{\mathcal{F}}{\xi}$ a.s.
\end{proof}

When verifying the defining property of conditional expectation it is
often useful to observe that it suffices to check indicator functions
for sets in a generating $\pi$-system.
\begin{lem}\label{ConditionalExpectationExtension}Suppose $\xi, \eta$ are integrable or non-negative random
  variables and $\mathcal{F}$ is a $\pi$-system such that $\Omega \in
  \mathcal{F}$ and for all $A
  \in \mathcal{F}$, we have $\expectation{\xi ; A} = \expectation{
    \eta; A}$.  Then we have $\expectation{\xi ; A} = \expectation{
    \eta; A}$ for all $A \in \sigma(\mathcal{F})$.
\end{lem}
\begin{proof}We first let $\mathcal{G}$ be the set of all $A$ such that  $\expectation{\xi ; A} = \expectation{
    \eta; A}$ and show that it is a $\lambda$-system.  If $A, B \in
  \mathcal{G}$ and $B \supset A$ then
\begin{align*}
\expectation{ \xi ; B\setminus A} &= \expectation{\xi;B} -
\expectation{\xi; A} = \expectation{\eta;B} -
\expectation{\eta; A} = \expectation{ \eta ; B\setminus A}
\end{align*}

Now suppose that we have $A_1 \subset A_2 \subset \cdots \in
\mathcal{G}$.  We claim that $\lim_{n \to \infty} \expectation{ \xi ;
  A_n} = \expectation{ \xi ; \cup_n A_n}$ and similarly with $\eta$.
In the case that we assume $\xi$ is integrable then we have $\abs{\xi
  \characteristic{A_n}} \leq \abs{\xi}$, so we may use Dominated
Convergence whereas in the case that $\xi$ is non-negative we may use
Monotone Convergence.  In either case,
\begin{align*}
\expectation{ \xi ; \cup_n A_n} &= \lim_{n \to \infty} \expectation{ \xi ;
  A_n} = \lim_{n \to \infty} \expectation{ \eta ;
  A_n} = \expectation{ \eta ; \cup_n A_n}
\end{align*}
We have assumed that $\Omega \in \mathcal{G}$ therefore we have shown $\mathcal{G}$ is a $\lambda$-system and our
assumption is that $\mathcal{F} \subset \mathcal{G}$ so we apply the
$\pi$-$\lambda$ Theorem (Theorem \ref{MonotoneClassTheorem}) to get
the result.
\end{proof}

Occasionally it can be useful to extend the defining property of
conditional expectation beyond indicator functions.
\begin{lem}Let $\xi \in L^1$ then for a $\sigma$-algebra $\mathcal{F}$
  and for any $\eta \in L^1(\mathcal{F})$ such that $\eta \xi$ and
  $\eta \cexpectation{\mathcal{F}}{\xi}$ are both integrable, 
  $\expectation{\cexpectation{\mathcal{F}}{\xi}\cdot\eta} = \expectation{\xi\cdot\eta} $.
\end{lem}
\begin{proof}
This is a simple application of the standard machinery.
Property (i) is exactly this statement for $\mathcal{F}$-measurable indicator functions.
Linearity of expectation shows that the statement then holds for
$\mathcal{F}$-measurable simple functions.  For
$\mathcal{F}$-measurable $\eta \geq 0$ satisfying the requirements of
the Lemma, we pick an increasing
approximation by simple functions $\eta_n \uparrow \eta$.
Now we can
apply Dominated Convergence to the sequences
$\cexpectation{\mathcal{F}}{\xi} \cdot \eta_n$ and $\xi \cdot \eta_n$,
\begin{align*}
\expectation{\xi \cdot \eta} &= \lim_{n \to \infty} \expectation{\xi
  \cdot \eta_n} & & \text{by Dominated Convergence} \\
&=\lim_{n \to \infty} \expectation{\cexpectation{\mathcal{F}}{\xi} 
  \cdot \eta_n} \\
&=\expectation{\cexpectation{\mathcal{F}}{\xi} 
  \cdot \eta} & & \text{by Dominated Convergence} \\
\end{align*}
For general integrable $\eta$ split into its positive and negative
parts $\eta = \eta_+ - \eta_-$ and use linearity of expectation.
\end{proof}

It is important to extend our basic limit theorems of integration
theory to conditional expectations.  We have already proven the
analogue of montone convergence.  Here we address the other cases of
importance.  The proofs are essentially identical to the
non-conditional cases.
\begin{lem}[Fatou's Lemma for Conditional Expectation]\label{FatouConditional}Let $\xi_1, \xi_2, \dotsc$ be
  positive random variables then 
\begin{align*}
\cexpectationlong{\mathcal{F}}{\liminf_{n \to \infty} \xi_n} &\leq \liminf_{n \to \infty} \cexpectationlong{\mathcal{F}}{\xi_n}
\end{align*}
\end{lem}
\begin{proof}
The proof is essentially identical to the case for ordinary
expectations (Theorem \ref{Fatou}) since we have montone convergence
and monotonicity of conditional expectation
\begin{align*}
\cexpectationlong{\mathcal{F}}{\liminf_{n \to \infty} \xi_n} &=
\lim_{n  \to \infty}\cexpectationlong{\mathcal{F}}{\inf_{k \geq n} \xi_k} \\
&\leq \lim_{n  \to \infty}\inf_{k \geq n}\cexpectationlong{\mathcal{F}}{ \xi_k} \\
 &= \liminf_{n \to \infty} \cexpectationlong{\mathcal{F}}{\xi_n}
\end{align*}
where all of the equalities and inequalities are taken to be almost sure.
\end{proof}

\begin{lem}[Dominated Convergence for Conditional
  Expectation]\label{DCTConditional}Let $\xi, \xi_1, \xi_2, \dotsc$ be
  random variables such that $\xi_n \toas \xi$ and $\eta$ be a
  positive random variables such that $\abs{\xi_n} \leq \eta$,
$\expectation{\eta} <  \infty$ then 
\begin{align*}
\cexpectationlong{\mathcal{F}}{\xi} &= \lim_{n \to \infty}
\cexpectationlong{\mathcal{F}}{\xi_n} \text{ a.s.}
\end{align*}
\end{lem}
\begin{proof}
Note that $\eta \pm \xi_n \geq 0$ so we may apply Fatou's Lemma
\ref{FatouConditional} to both sequences.
\begin{align*}
\cexpectationlong{\mathcal{F}}{\eta} \pm
\cexpectationlong{\mathcal{F}}{\xi}  &=\cexpectationlong{\mathcal{F}}{\eta \pm \xi} \\
&=\cexpectationlong{\mathcal{F}}{\lim_{n \to \infty} \eta \pm \xi_n}
\\
&\leq \liminf_{n \to \infty} \cexpectationlong{\mathcal{F}}{ \eta \pm \xi_n} \\
&= \cexpectationlong{\mathcal{F}}{ \eta} + \liminf_{n \to \infty} \cexpectationlong{\mathcal{F}}{ \pm \xi_n} \\
\end{align*}
where all of the comparisons are in an almost sure sense.  Now by
integrability of $\eta$ and the chain rule of conditional expectation 
we know that $\expectation
{\cexpectationlong{\mathcal{F}}{ \eta}}=
\expectation{\eta} < \infty$ and therefore
$\cexpectationlong{\mathcal{F}}{ \eta} < \infty$ a.s.  Thus it is
permissible to subtract $\cexpectationlong{\mathcal{F}}{ \eta}$ from
both sides of the inequality above and deduce the pair of inequalities
\begin{align*}
\pm\cexpectationlong{\mathcal{F}}{\xi}  &\leq \liminf_{n \to \infty}
\cexpectationlong{\mathcal{F}}{ \pm \xi_n} \text{ a.s.}\\
\end{align*}
Now using this pair of inequalities
\begin{align*}
\limsup_{n \to \infty} \cexpectationlong{\mathcal{F}}{ \xi_n} &=
-\liminf_{n \to \infty} \cexpectationlong{\mathcal{F}}{ -\xi_n} \leq \cexpectationlong{\mathcal{F}}{\xi} \leq \liminf_{n \to \infty}
\cexpectationlong{\mathcal{F}}{\xi_n} \text{ a.s. }
\end{align*}
which shows us that $\cexpectationlong{\mathcal{F}}{\xi} = \lim_{n \to \infty}
\cexpectationlong{\mathcal{F}}{\xi_n}$ a.s.
\end{proof}

\begin{lem}Suppose that $\xi_t$ for $t \in T$ is a uniformly
  integrable family of random variables and then
  $\cexpectationlong{\mathcal{F}}{\xi_t}$ is uniformly integrable.
  Moreover if $\xi$ is a random variable and $\xi_n$ is a uniformly integrable family of random
  variables such that $\xi_n \toas \xi$ then
  $\cexpectationlong{\mathcal{F}}{\xi_n} \toas \cexpectationlong{\mathcal{F}}{\xi}$.
\end{lem}
\begin{proof}
To see uniform integrability of
$\cexpectationlong{\mathcal{F}}{\xi_t}$ we use Lemma
\ref{UniformIntegrabilityProperties}.  Since conditional expectation
is an $L^1$ contraction, the $L^1$ boundedness of
$\cexpectationlong{\mathcal{F}}{\xi_t}$ follows from the $L^1$
boundedness of $\xi_t$.  Now if we let $A$ be measurable and pick $R >
0$, then by using monotonicity and the tower property of conditional expectation 
\begin{align*}
\expectation{\abs{\cexpectationlong{\mathcal{F}}{\xi_t}} ; A} &\leq
\expectation{\cexpectationlong{\mathcal{F}}{\abs{\xi_t}} ; A} \\
&=\expectation{\cexpectationlong{\mathcal{F}}{\abs{\xi_t}; \abs{\xi_t}
  \leq R} ; A} +
\expectation{\cexpectationlong{\mathcal{F}}{\abs{\xi_t}; \abs{\xi_t} >
  R} ; A} \\
&\leq R \probability{A} + \expectation{\abs{\xi_t} ; \abs{\xi_t} > R} \\
\end{align*}
and therefore taking $\sup_t$, $\lim_{\probability{A} \to 0}$ and
$\lim_{R \to \infty}$ and using the uniform integrability of $\xi_t$
we get uniform integrability of
$\cexpectationlong{\mathcal{F}}{\xi_t}$.

If we assume that $\xi_1, \xi_2, \dotsc$ are uniformly integrable and $\xi_n \toas
\xi$ then picking a measurable $A$ and using the first part of this
lemma and Lemma \ref{BoundedTimesUniformlyIntegrable} we know that
both families $\xi_1
\characteristic{A}, \xi_2 \characteristic{A}, \dotsc$ and $\cexpectationlong{\mathcal{F}}{\xi_1}
\characteristic{A}, \cexpectationlong{\mathcal{F}}{\xi_2}
\characteristic{A}, \dotsc$ are uniformly integrable.  So know using
using Lemma \ref{LpConvergenceUniformIntegrability} to justify
exchanging limits and expectations we get
\begin{align*}
\expectation{\xi ; A} &= \lim_{n \to \infty} \expectation{\xi_n ; A} =
\lim_{n \to \infty} \expectation{\cexpectationlong{\mathcal{F}}{\xi_n}
  ; A} = \expectation{\lim_{n \to \infty} \cexpectationlong{\mathcal{F}}{\xi_n}
  ; A} 
\end{align*}
Since $\lim_{n \to \infty} \cexpectationlong{\mathcal{F}}{\xi_n}$
is $\mathcal{F}$-measurable (Lemma \ref{LimitsOfMeasurable}) we know that 
$\cexpectationlong{\mathcal{F}}{\xi} = \lim_{n \to \infty}
\cexpectationlong{\mathcal{F}}{\xi_n}$ by the defining property of conditional expectation.
\end{proof}
 
TODO: Provide an example of conditional expectation and a dyadic
$\sigma$-algebra.

A last observation is that conditional expectations depend only 
``local'' information in both the random variable and the
$\sigma$-algebra.  This has an intuitive appeal as one can think of
the $\sigma$-algebra against which the conditional expectation is
taken as a specifying a coarser resolution of the random variable and
this coarsening is obtained by averaging/integration.  So long as the
domains over which we integrate are contained entirely inside of a
set we are interested in, the conditional expectation should only
depend on the $\sigma$-algebra restricted to that set and the values
of the random variable on that set.  We proceed to make this idea more
formal and give a proper proof.

\begin{defn}Given $\sigma$-algebras $\mathcal{F}$, $\mathcal{G}$ and
  $\mathcal{A}$ with $\mathcal{F} \subset \mathcal{A}$ and
  $\mathcal{G} \subset \mathcal{A}$ and a set $A \in \mathcal{F} \cap
  \mathcal{G}$, we way that $\mathcal{F}$ and $\mathcal{G}$
  \emph{agree on $A$} if for every $B \subset A$, $B \in \mathcal{F}$
  if and only if $B \in \mathcal{G}$.
\end{defn}
\begin{lem}\label{ConditionalExpectationIsLocal}Given $\sigma$-algebras $\mathcal{F}$, $\mathcal{G}$ and
  $\mathcal{A}$ with $\mathcal{F} \subset \mathcal{A}$ and
  $\mathcal{G} \subset \mathcal{A}$ and a set $A \in \mathcal{F} \cap
  \mathcal{G}$ such that $\mathcal{F}$ and $\mathcal{G}$ agree on $A$
  and random variables $\xi$ and $\eta$ such that $\xi$ and $\eta$
  agree almost surely on $A$ then
\begin{align*}
\cexpectationlong{\mathcal{F}}{\xi} &=
\cexpectationlong{\mathcal{G}}{\eta} \text{ a.s. on $A$}
\end{align*}
\end{lem}
\begin{proof}
We first claim that if $B \subset A$ and $B \in \mathcal{F} \vee
\mathcal{G}$ then in fact $B \in \mathcal{F} \cap
\mathcal{G}$.  To see the claim, $A \cap \mathcal{F} \vee
\mathcal{G}$ is a $\sigma$-algebra of subsets of $A$ generated by $A
\cap \mathcal{F} = A \cap \mathcal{G} = A \cap \mathcal{F} \cap
\mathcal{G}$ hence $A \cap \mathcal{F} \vee
\mathcal{G} \subset A \cap \mathcal{F} \cap
\mathcal{G}$.  The opposite inclusion is trivial.

Consider the set $\lbrace
\cexpectationlong{\mathcal{F}}{\xi} >
\cexpectationlong{\mathcal{G}}{\eta} \rbrace \cap A$ and observe by
the above claim that it
is contained in $\mathcal{F} \cap \mathcal{G}$.  Therefore by
monotonicity of conditional expectation, the averaging property of
conditional expectation and the fact that $\xi = \eta$ almost surely
on $A$ we have
\begin{align*}
0 &\leq \expectation{(\cexpectationlong{\mathcal{F}}{\xi} -
\cexpectationlong{\mathcal{G}}{\eta} ) ; 
\lbrace
\cexpectationlong{\mathcal{F}}{\xi} >
\cexpectationlong{\mathcal{G}}{\eta} \rbrace \cap A} \\
&=\expectation{\cexpectationlong{\mathcal{F}}{\xi} ; 
\lbrace \cexpectationlong{\mathcal{F}}{\xi} >
\cexpectationlong{\mathcal{G}}{\eta} \rbrace \cap A} -
\expectation{\cexpectationlong{\mathcal{G}}{\eta} ; 
\lbrace
\cexpectationlong{\mathcal{F}}{\xi} >
\cexpectationlong{\mathcal{G}}{\eta} \rbrace \cap A} \\
&=\expectation{\xi; 
\lbrace \cexpectationlong{\mathcal{F}}{\xi} >
\cexpectationlong{\mathcal{G}}{\eta} \rbrace \cap A} -
\expectation{\eta ; 
\lbrace
\cexpectationlong{\mathcal{F}}{\xi} >
\cexpectationlong{\mathcal{G}}{\eta} \rbrace \cap A} \\
&=0
\end{align*}
which shows $\cexpectationlong{\mathcal{F}}{\xi} \leq
\cexpectationlong{\mathcal{G}}{\eta}$ almost surely on $A$.  Switching
the roles of $\mathcal{F}$ and $\mathcal{G}$ yields the opposite
inequality and the result follows.
\end{proof}

The definition of conditional expectation as given is rather abstract
but in the case of random variables with densities, we can make the
concept more concrete.

TODO: Where to put this?
\begin{lem}Let $(\xi, \eta)$ be a random vector in $\reals^2$.
  Suppose that $(\xi, \eta)$ has a density $f$, then 
\begin{itemize}
\item[(i)]Both  $\xi$ and $\eta$ have a densities given by the
  formulas
\begin{align*}
f_{\xi}(y) = \int_{-\infty}^\infty f(y,z) \, dz & & f_{\eta}(z) = \int_{-\infty}^\infty f(y,z) \, dy
\end{align*}
\item[(ii)]$\xi$ and $\eta$ are independent if and only if $f(y,z) = f_{\xi}(y) f_{\eta}(z) $.
\item[(iii)]For any $y \in \reals$ such that $f_{\xi}(y) \neq 0$, we
  have the density
\begin{align*}
f_{\xi=y}(z) &= \frac{f(y,z)}{f_{\xi}(y)}
\end{align*}
\item[(iv)]If we define $h_{\eta}(y) = \int_{-\infty}^\infty z f_{\xi=y}(z)
  \, dz$ then for every measurable $g : \reals \to \reals$ such that
$g(\xi)$ is integrable, we have
\begin{align*}
\expectation{g(\xi) \cdot h_{\eta}(\xi)} &= \expectation{\xi \cdot \eta}
\end{align*}
\end{itemize}
\end{lem}

If we consider $\eta$ a random element in some $(T, \mathcal{T})$, $\xi$ an integrable random
variable then we usually write $\cexpectationlong{\sigma(\eta)}{\xi} =
\cexpectationlong{\eta}{\xi}$ and speak of the \emph{conditional
  expectation of $\xi$ with respect to $\eta$}.  
\begin{lem}There exists a measurable function $f : T \to \reals$ such
  that $\cexpectationlong{\eta}{\xi} = f(\eta)$, furthermore such an
  $f$ is unique almost surely $\pushforward{\eta}{P}$.  If we are
  given another pair $\tilde{\xi}$ and $\tilde{\eta}$ such that $(\xi,
  \eta) \eqdist (\tilde{\xi}, \tilde{\eta})$ then $\cexpectationlong{\tilde{\eta}}{\tilde{\xi}} = f(\tilde{\eta})$.
\end{lem}
\begin{proof}This is a simple corollary of Lemma
  \ref{FunctionalRepresentation} and the almost sure uniqueness of
  conditional expectations.
\end{proof}

Having defined $\cexpectationlong{\eta}{\xi}$ in terms of conditional
expectation of $\xi$ with respect the $\sigma$-algebra $\sigma(\eta)$
is natural to think of the latter as being the more general case.
However note that if we are given $\mathcal{F}$ and define $\eta :
(\Omega, \mathcal{A}) \to (\Omega, \mathcal{F})$ to be identity
function then in fact we see the two notions are equivalent.  In some
cases, authors (Kallenberg in particular) will refer to conditional
expectation with respect to a $\sigma$-algebra as the special case.
We'll try to avoid making statements about the relative level of
generality of the two ideas but will try to avoid using the notation
$\cexpectationlong{\eta}{\xi}$ when we know that $\eta$ is an
identity map.

\begin{lem}\label{ConditionalExpectationCompletions}Let $\mathcal{F}$ be a $\sigma$-algebra and let $\xi$ be
  integrable, then $\cexpectationlong{\mathcal{F}}{\xi} =
  \cexpectationlong{\overline{\mathcal{F}}}{\xi}$ a.s.
\end{lem}
\begin{proof}
Let $A \in \overline{\mathcal{F}}$.  We know from Lemma ??? that there
exist $A_\pm \in \mathcal{F}$ such that $A_- \subset A \subset A_+$ and
$\probability{A_+ \setminus A_-} = 0$. It is clear that for any $\xi
\geq 0$ we have 
\begin{align*}
\expectation{\xi ; A_-} &\leq \expectation{\xi ; A} \leq
\expectation{\xi ; A_+} = \expectation{\xi; A_-} + \expectation{\xi;
  A_+ \setminus A_-} = \expectation{\xi; A_-} 
\end{align*}
and therefore $\expectation{\xi ; A_-} = \expectation{\xi; A} =
\expectation{\xi; A_+}$.  By linearity this clearly extends to
integrable $\xi$.  Therefore we get
\begin{align*}
\expectation{\xi; A} = \expectation{\xi; A_-} =
\expectation{\cexpectationlong{\mathcal{F}}{\xi}; A_-} = \expectation{\cexpectationlong{\mathcal{F}}{\xi}; A} 
\end{align*}
which gives the result.
\end{proof}

\section{Conditional Independence}

\begin{defn}Given $\sigma$-algebras $\mathcal{F}$, $\mathcal{G}$ and
  $\mathcal{H}$ we say that $\mathcal{F}$ and $\mathcal{H}$ are
  \emph{conditionally independent given} $\mathcal{G}$ if for all $F
  \in \mathcal{F}$ and all $H \in \mathcal{H}$ we have 
\begin{align*}
\cprobability{\mathcal{G}}{F \cap H} &= \cprobability{\mathcal{G}}{F} \cprobability{\mathcal{G}}{H} 
\end{align*}
We often write $\cindependent{\mathcal{F}}{\mathcal{H}}{\mathcal{G}}$.
\end{defn}

A technical result that can be helpful when trying to prove
conditional independence is the following analogue of Lemma
\ref{IndependencePiSystem}
\begin{lem}\label{ConditionalIndependencePiSystem}Suppose we are given
  a $\sigma$-algebra $\mathcal{G}$ and two
  $\pi$-systems $\mathcal{S}$ and $\mathcal{T}$ in a probability space
  $(\Omega, \mathcal{A}, P)$ such that
  $\cprobability{\mathcal{G}}{A \cap B} = \cprobability{\mathcal{G}}{A} \cprobability{\mathcal{G}}{B}$ for all
  $A \in \mathcal{S}$ and $B \in \mathcal{T}$.  Then
  $\sigma(\mathcal{S})$ and $\sigma(\mathcal{T})$ are conditionally independent
  given $\mathcal{G}$.
\end{lem}
\begin{proof}
TODO: A straightforward extension of the proof of Lemma
\ref{IndependencePiSystem}.
\end{proof}

\begin{lem}\label{ConditionalIndependenceDoob}Given $\sigma$-algebras $\mathcal{F}$, $\mathcal{G}$ and
  $\mathcal{H}$, then
  $\cindependent{\mathcal{F}}{\mathcal{H}}{\mathcal{G}}$ if and only
  if for all $H \in \mathcal{H}$, we have
  $\cprobability{\mathcal{G}}{H} =
  \cprobability{\mathcal{F},\mathcal{G}}{H}$.
In particular, $\cindependent{\mathcal{F}}{\mathcal{H}}{\mathcal{G}}$
if and only if $\cindependent{\left (\mathcal{F}, \mathcal{G} \right )}{\mathcal{H}}{\mathcal{G}}$
\end{lem}
\begin{proof}
We first assume that
$\cindependent{\mathcal{F}}{\mathcal{H}}{\mathcal{G}}$.  Let $F \in
\mathcal{F}$ and $G \in \mathcal{G}$ and calculate
\begin{align*}
\expectation{\characteristic{F}\characteristic{G}\characteristic{H}}
&=
\expectation{\cexpectationlong{\mathcal{G}}{\characteristic{F}\characteristic{G}\characteristic{H}}}
\\
&=
\expectation{\characteristic{G}\cexpectationlong{\mathcal{G}}{\characteristic{F}\characteristic{H}}} \\
&=
\expectation{\characteristic{G}\cexpectationlong{\mathcal{G}}{\characteristic{F}}\cexpectationlong{\mathcal{G}}{\characteristic{H}}} \\
&=
\expectation{\cexpectationlong{\mathcal{G}}{\characteristic{F}\characteristic{G}}\cexpectationlong{\mathcal{G}}{\characteristic{H}}} \\
&=
\expectation{\characteristic{F}\characteristic{G}\cexpectationlong{\mathcal{G}}{\characteristic{H}}} \\
\end{align*}
Now note that set of all intersections $F \cap G$ is a $\pi$-system
that contains $\Omega$ and therefore by Lemma
\ref{ConditionalExpectationExtension} and the defining property of
conditional expectation we have
$\cexpectationlong{\mathcal{G}}{\characteristic{H}} =
\cexpectationlong{\mathcal{F},\mathcal{G}}{\characteristic{H}}$.

To show the converse, we take $F \in \mathcal{F}$ and $H \in
\mathcal{H}$ and
\begin{align*}
\cexpectationlong{\mathcal{G}}{\characteristic{F}\characteristic{H}} &=
\cexpectationlong{\mathcal{G}}{\cexpectationlong{\mathcal{F},
    \mathcal{G}}{\characteristic{F}\characteristic{H}}} \\
&= \cexpectationlong{\mathcal{G}}{\characteristic{F}\cexpectationlong{\mathcal{F},
    \mathcal{G}}{\characteristic{H}}} \\
&= \cexpectationlong{\mathcal{G}}{\characteristic{F}} \cexpectationlong{\mathcal{F},
    \mathcal{G}}{\characteristic{H}}\\
&= \cexpectationlong{\mathcal{G}}{\characteristic{F}} \cexpectationlong{\mathcal{G}}{\characteristic{H}}\\
\end{align*}
Now the last claim follows simply we have shown both
statements are equivalent to the fact that
$\cprobability{\mathcal{G}}{H} =
\cprobability{\mathcal{F},\mathcal{G}}{H}$ for all $H \in \mathcal{H}$.
\end{proof}

\begin{lem}\label{ConditionalIndependenceChainRule}Given $\sigma$-algebras $\mathcal{G}$, $\mathcal{H}$ and
  $\mathcal{F}_1, \mathcal{F}_2, \dots$, then
  $\cindependent{\mathcal{H}}{\left( \mathcal{F}_1, \mathcal{F}_2,
      \dots \right)}{\mathcal{G}}$ if and only if $\cindependent{\mathcal{H}}{ \mathcal{F}_{n+1}}{\left(  \mathcal{G},\mathcal{F}_1, \mathcal{F}_2,
      \dots , \mathcal{F}_n\right)}$ for all $n \geq 0$.
\end{lem}
\begin{proof}
If we assume the second property then we can conclude from Lemma
\ref{ConditionalIndependenceDoob} and an induction on $n \geq 0$ that
for every $H \in \mathcal{H}$,
\begin{align*}
\cprobability{\mathcal{G}}{H} &= 
\cprobability{\mathcal{G}, \mathcal{F}_1}{H} = 
\cprobability{\mathcal{G}, \mathcal{F}_1, \mathcal{F}_2}{H} = \cdots
\end{align*}
and therefore by another application of Lemma
\ref{ConditionalIndependenceDoob}, we know that
$\cindependent{\mathcal{H}}{\left (\mathcal{F}_1, \dots,
    \mathcal{F}_n\right)}{\mathcal{G}}$ for every $n \geq 1$.  Now
$\cup_n \sigma(\mathcal{F}_1, \dots, \mathcal{F}_n)$ is a $\pi$-system that generates
$\sigma(\mathcal{F}_1, \mathcal{F}_2, \dots)$ and therefore
application of Lemma \ref{ConditionalIndependencePiSystem} shows us
that  $\cindependent{\mathcal{H}}{\left( \mathcal{F}_1, \mathcal{F}_2,
      \dots \right)}{\mathcal{G}}$.

On the other hand, if we assume $\cindependent{\mathcal{H}}{\left( \mathcal{F}_1, \mathcal{F}_2,
      \dots \right)}{\mathcal{G}}$ then for any $n \geq 1$, and $H \in
  \mathcal{H}$, we apply the telescoping rule, Lemma
  \ref{ConditionalIndependenceDoob} and the pull out rule to get
\begin{align*}
\cprobability{\mathcal{G}, \mathcal{F}_1, \dots, \mathcal{F}_n}{H} &=
\cexpectationlong{\mathcal{G}, \mathcal{F}_1, \dots,
  \mathcal{F}_n}{\cprobability{\mathcal{G}, \mathcal{F}_1,
    \mathcal{F}_2, \dots }{H}} \\
&=\cexpectationlong{\mathcal{G}, \mathcal{F}_1, \dots,
  \mathcal{F}_n}{\cprobability{\mathcal{G}}{H}} \\
&=\cprobability{\mathcal{G}}{H}
\end{align*}
so in particular, for all $n \geq 0$,
\begin{align*}
\cprobability{\mathcal{G}, \mathcal{F}_1, \dots, \mathcal{F}_n}{H} &= \cprobability{\mathcal{G}, \mathcal{F}_1, \dots, \mathcal{F}_{n+1}}{H}
\end{align*}
Another application of Lemma \ref{ConditionalIndependenceDoob} shows
that $\cindependent{\mathcal{H}}{ \mathcal{F}_{n+1}}{\left(  \mathcal{G},\mathcal{F}_1, \mathcal{F}_2,
      \dots , \mathcal{F}_n\right)}$ for all $n \geq 0$.
\end{proof}

\begin{lem}Suppose $\cindependent{\mathcal{F}}{\mathcal{H}}{\mathcal{G}}$ and $\mathcal{A} \subset \mathcal{F}$,
  then $\cindependent{\mathcal{F}}{\mathcal{H}}{\mathcal{A},\mathcal{G}}$.
\end{lem}
\begin{proof}
By Lemma \ref{ConditionalIndependenceDoob}, we know for all $H \in
\mathcal{H}$, 
$\cprobability{\mathcal{G}}{H} =
\cprobability{\mathcal{F},\mathcal{G}}{H}$. On the other hand, since
$\mathcal{A} \subset \mathcal{F}$ we also have $\mathcal{G} \subset
\sigma(\mathcal{A}, \mathcal{G}) \subset \sigma(\mathcal{F},
\mathcal{G})$ and therefore we can conclude
$\cprobability{\mathcal{F},\mathcal{G}}{H} = \cprobability{\mathcal{A},\mathcal{G}}{H}$.
Since $\mathcal{A} \subset \mathcal{F}$ we know that $\sigma(\mathcal{A},
\mathcal{F}, \mathcal{G}) = \sigma(\mathcal{F}, \mathcal{G})$ and we
get $\cprobability{\mathcal{F},\mathcal{A},\mathcal{G}}{H} =
\cprobability{\mathcal{A},\mathcal{G}}{H}$.
Another application of Lemma \ref{ConditionalIndependenceDoob} tells
us that $\cindependent{\mathcal{F}}{\mathcal{H}}{\mathcal{A},\mathcal{G}}$.
\end{proof}

\section{Conditional Distributions and Disintegration}
Now for a more subtle concept in conditioning.  Consider a random
element $\xi$ in a measurable space $(S,\mathcal{S})$ and a random
element $\eta$ in a measurable space $(T,\mathcal{T})$.  We'd like to
make sense of the conditional distribution of $\xi$ given a value of
$\eta$.  Two things should occur to us.  First, such an object sounds
like it should a mapping from $T$ to a space of measures on $S$.  Second, we
expect that we'll actually define this object in terms of the
conditional expectation and that it will likely wind up as an
$\eta$-measurable random measure on $\Omega$.  A third thing might also
occur to us: namely these two representations are equivalent.  As it
turns out, due to the fact that conditional expectations are only
defined up to almost sure equivalence, this last supposition is not true and we often must make
additional assumptions to arrange for the existence of the mapping of
$T$ to the space of measures on $S$. 

\subsection{Probability Kernels}
Before jumping into the development of conditional distributions
proper we need to step back a bit and make sure we've laid a proper
foundation for the discussion.  We wrote heuristically above about a
mapping to a space of measures.  This is a concept that will come up in a variety of contexts from this point
on and we glossed over the fact that we want such a mapping to have
measurability properties.  There are a couple of equivalent ways of
formulating the notion of a measurable family of measures;  we
explore these now.
To formalize, we have the following definition
\begin{defn}
Let $(S, \mathcal{S})$ and $(T, \mathcal{T})$ be measurable spaces.  A
\emph{probability kernel} from $S$ to $T$ is a function $\mu : S
\times \mathcal{T} \to [0,1]$ 
such that for every fixed $s \in S$, $\mu(s, \cdot) : \mathcal{T} \to
[0,1]$ is a probability measure and for every fixed $A \in
\mathcal{T}$, $\mu(\cdot, A) : S \to [0,1]$ is Borel measurable.
\end{defn}

It is useful to have some alternative characterizations of the
measurability properites of kernels but before we can state them we
need another definition.
\begin{defn}Given a measurable space $(S, \mathcal{S})$, then
  $\mathcal{P}(S)$ is the space of probability measures on $S$ with
  the $\sigma$-algebra generated by all sets of the form $\lbrace \mu
  \mid \mu(A) \in B \rbrace$ for $A \in \mathcal{S}$ and $B \in
  \mathcal{B}([0,1])$.  Alternatively, for each $A \in \mathcal{S}$,
  define the evaluation map $\pi_A : \mathcal{P}(S) \to [0,1]$ by
  $\pi_A(\mu) = \mu(A)$ and then take the $\sigma$-algebra generated
  by all of the evaluation maps.
\end{defn}

\begin{examp}\label{ProbabilityKernelFiniteSampleSpace}
The following special case of a probability kernel is easy to
understand and also comes up in the theory of finite Markov chains.
Suppose $S$ and $T$ are two finite probability spaces each equipped with the power
set $\sigma$-algebra.  In this case a probability measure on $T$ is just
a set of non-negative real numbers $p_t$ for $t \in T$ such $\sum_{t
  \in T} p_t = 1$.  Therefore a probablity kernel from $S$ to $T$ is
just a set of such vectors, one for each $s \in S$.   It is customary in
the theory of finite Markov chains to view probabilities on $T$ as row
vectors and thus view a probability kernel $\mu$ as an $S \times T$ matrix
$\mu_{s,t}$ such that $\mu_{s,t} \geq 0$ and for each fixed $s \in S$
we have $\sum_{t \in T} \mu_{s,t} = 1$.  Such a matrix with row sums
equal to $1$ is sometimes called a \emph{stochastic matrix}.
Note that because we are using power set $\sigma$-algebras the
measurability conditions in the definition of a kernel are trivially
satisfied.
\end{examp}

Many mappings on the space of probability measures are measurable.
\begin{lem}\label{MeasurableMappingsOfMeasures}Let $(S, \mathcal{S})$ and $(T, \mathcal{T})$ be measurable
  spaces and let $f : S \to T$ be a measurable function then the following mappings are measurable:
\begin{itemize}
\item[(i)]$\mu \mapsto \mu_A$ for every $A \in \mathcal{S}$.
\item[(ii)]$\mu \mapsto \int f \, d\mu$ for every measurable function $f: S \to \reals$.
\item[(iii)]$(\mu,\nu) \mapsto \mu \otimes \nu$.
\item[(iii)]$\mu \mapsto \pushforward{f}{\mu}$.
\end{itemize}
\end{lem}
\begin{proof}
To see (i) simply note that for every $B \in \mathcal{S}$ and $C \in
\mathcal{B}(\reals)$, we have $\lbrace \mu \mid \mu_A(B) \in C \rbrace
= \lbrace \mu \mid \mu(A \cap B) \in C \rbrace$ which is
measurable since $A \cap B \in \mathcal{S}$.

For (ii) note that for $f=\characteristic{A}$ an indicator function we have $\int f \,
d\mu = \mu(A)$ is a measurable function of $\mu$ be definition of the
$\sigma$-algebra on $\mathcal{P}(S)$.  By Lemma
\ref{ArithmeticCombinationsOfMeasurableFunctions} we then see that
$\int f \, d\mu$ is measurable for simple functions.  For positive
functions $f$ we take an increasing sequence of simple functions $f_n
\uparrow f$ so that $\int f \, d\mu = \lim_{n \to \infty} \int f_n \,
d\mu$ which is measurable by Lemma \ref{LimitsOfMeasurable}.  For
general $f$ we write $f = f_+ - f_-$ and use Lemma
\ref{ArithmeticCombinationsOfMeasurableFunctions} again.

To see (iii) we first note that for $A \in \mathcal{S}$ and $B \in
\mathcal{T}$ we have $(\mu \otimes \nu)(A \times B) = \mu(A)\nu(B)$
which is a measurable function of $(\mu, \nu)$ by definition of the $\sigma$-algebras on
$\mathcal{P}(S)$
and $\mathcal{P}(T)$, definition of the product $\sigma$-algebra and
continuity (hence Borel measurability) of multiplication on $\reals$.  Now we extend
to general $A \in \mathcal{S} \otimes \mathcal{T}$ by a monotone class
argument.  Let $\mathcal{C} = \lbrace A \in \mathcal{S} \otimes
\mathcal{T} \mid \mu \otimes \nu(A) \text{ is a measurable function of
} (\mu, \nu) \rbrace$.  We claim that $\mathcal{C}$ is a
$\lambda$-system.  If $A,B \in \mathcal{C}$ such that $A \subset B$
then $(\mu \otimes \nu)(B \setminus A) = (\mu \otimes \nu)(B) - (\mu
\otimes \nu)(A)$ which is measurable by Lemma
\ref{ArithmeticCombinationsOfMeasurableFunctions}.  If $A_1 \subset
A_2 \subset \cdots$ with $A_n \in \mathcal{C}$ for $n = 1, 2, \dotsc$ then by continuity of
measure (Lemma \ref{ContinuityOfMeasure}) we have $(\mu \otimes
\nu)(A) = \lim_{n \to \infty} (\mu \otimes \nu)(A_n)$ which is
measurable by Lemma \ref{LimitsOfMeasurable}.  Since the sets of the
form $A \times B$ are a $\pi$-system generating $\mathcal{S} \otimes
\mathcal{T}$ we can apply the $\pi$-$\lambda$ Theorem (Theorem
\ref{MonotoneClassTheorem}) to conclude $\mathcal{S} \otimes
\mathcal{T} \subset \mathcal{C}$ and the claim is verified.  By the
result of the claim we now know that for every $C \in
\mathcal{B}(\reals)$ and every $A \in \mathcal{S} \otimes \mathcal{T}$
we have 
\begin{align*}
\lbrace (\mu, \nu) \in \mathcal{P}(S) \times \mathcal{P}(T)
\mid (\mu \otimes \nu) (A) \in C \rbrace &=
\otimes^{-1} \lbrace \mu \in \mathcal{P}(S \times T) \mid \mu (A) \in C \rbrace
\end{align*}
 is a measurable subset of
$\mathcal{P}(S) \times \mathcal{P}(T)$.  Since sets of the form
$\lbrace \mu \in \mathcal{P}(S \times T) \mid \mu (A) \in C \rbrace$
generate the $\sigma$-algebra on $\mathcal{P}(S \times T)$ we have
that $\otimes$ is measurable (Lemma \ref{MeasurableByGeneratingSet}).

To see (iv), we know that $\pushforward{f}{\mu}$ is indeed a
probability measure (Lemma \ref{PushforwardMeasure}).  To see the
measurability of the pushforward, suppose $A \in \mathcal{T}$ and $B
\in \mathcal{B}([0,1])$ and note that 
\begin{align*}
\lbrace \mu \in \mathcal{P}(S) \mid \pushforward{f}{\mu}(A) \in B
  \rbrace
&=
\lbrace \mu \in \mathcal{P}(S) \mid \mu(f^{-1}(A)) \in B
  \rbrace
\end{align*}
which is measurable since $f^{-1}(A) \in \mathcal{S}$.  Now the
general result follows from Lemma \ref{MeasurableByGeneratingSet}.
\end{proof}

As promised, we have the following lemma that gives a couple of
alternative characterizations of the measurability condition of a
kernel; including the obligatory monotone class argument.
\begin{lem}\label{KernelMeasurability}Let $(S, \mathcal{S})$ and $(T, \mathcal{T})$ be measurable
  spaces and $\mu_s$ be a family of probability measures on $T$.  Then
  the following are equivalent
\begin{itemize}
\item[(i)]$\mu : S \times \mathcal{T} \to [0,1]$ is a probability kernel
\item[(ii)]$\mu : S \to \mathcal{P}(T)$ is measurable
\item[(iii)]$\mu(s, A) : S \to [0,1]$ is Borel measurable for every $A$
  belonging to a $\pi$-system that generates $\mathcal{S}$.
\end{itemize}
\end{lem}
\begin{proof}
First suppose that $\mu$ is a kernel, $A \in \mathcal{T}$
and $B$ is a Borel
measurable subset of $[0,1]$.  Then 
\begin{align*}
\mu^{-1}(\lbrace \nu \mid \nu(A)
\in B \rbrace) &= \lbrace s \in S \mid \mu(s,A) \in B \rbrace =
\mu(\cdot, A)^{-1}(B)
\end{align*}
which is measurable by the kernel property.  Since sets of the form $\lbrace \nu \mid \nu(A)
\in B \rbrace$ generate the $\sigma$-algebra on $\mathcal{P}(T)$ we
see that $\mu$ is measurable by Lemma \ref{MeasurableByGeneratingSet}.

To see that (ii) implies (i), observe that for a fixed $A \in
\mathcal{T}$ and let $\pi_A(\nu) = \nu(A)$ be the evaluation map.  By
construction the $\pi_A$ are measurable.  For such a fixed $A$, we see
that $\mu(s, A) = \pi_A(\mu)$ therefore as a composition of measurable
maps we see that $\mu(s,A)$ is $\mathcal{S}$-measurable (Lemma
\ref{CompositionOfMeasurable}).

The implication (i) implies (iii) is immediate.  If we assume (iii)
then we derive (i) by a monotone class argument.  By Theorem
\ref{MonotoneClassTheorem} it suffices to show that $\mathcal{C} =
\lbrace A \mid \mu(s, A) : S \to [0,1] \text { is measurable}\rbrace$
is a $\lambda$-system.  If $A \subset B$ with $A,B \in \mathcal{C}$
then $\mu(s, B \setminus A) = \mu(s, B) - \mu(s,A)$ is measurable.  If
$A_1 \subset A_2 \subset \cdots$ with $A_n \in \mathcal{C}$ then by
continuity of measure (Lemma \ref{ContinuityOfMeasure}) applied
pointwise in $s$, we see $\mu(s, \cup_n A_n) = \lim_n \mu(s, A_n)$
which shows measurability by Lemma \ref{LimitsOfMeasurable}.
\end{proof}

A point that shall occasionally come up is the fact that we shall use
the previous lemma to shift interpretations of a kernel: sometimes
thinking of it as a map $\mu : S \times \mathcal{T} \to [0,1]$ and
sometimes as a map $\mu : S \to \mathcal{P}(T)$.  Often we will make
such transitions between these perspectives without  comment but there
are times in which we may use the notation $\mu(s,A)$ when thinking of
the first realization and $\mu(s)$ when thinking of the second.  It is
also the case that the notation for integrals with respect to kernels
needs to be considered.  Up to this point we have notation $\int f \,
d\mu$ for integrals and in those cases in which we wanted to make it
clear what the integration variable is we might write $\int f(x) \,
d\mu(x)$.  In a world with kernels the latter notation is unfortunate
as it becomes difficult to construe whether the $x$ dependence indicated
for the measure means an integration variable or whether it
indicates that the measure is a kernel with $x$ dependence.  To resolve
this issue we shall adopt a different convention when discussing
integrals against kernels and write $\int f(x) \, \mu(dx)$ to denote
that $x$ is the integration variable.  This notation allows us to
capture both integration variables and measure dependence in
expressions such as $\int f(x) \, \mu(s, dx)$ which should be
interpreted as the integral of $f(x)$ against the measure $\mu(s)$ for
some particular value of $s$.  The reader may already be wondering
whether an expression such as this is a measurable function of the
parameter $s$; we will state and prove a slightly more general
fact below.

There is a useful generalization of the product measure construction
involving kernels.  It is a type of ``twisted'' product construction.
\begin{defn}Let $\mu : S \times \mathcal{T} \to [0,1]$ be a
  probability kernel from $S$ to $T$ and $\nu : S \times T \times
  \mathcal{U} \to [0,1]$ be a probability kernel from $S \times T$ to
  $U$, we then define $\mu \otimes \nu : S \times \mathcal{T} \otimes
  \mathcal{U} \to [0,1]$ by
\begin{align*}
\mu \otimes \nu(s, A) &= \iint \characteristic{A}(t,u) \,
d\nu(s,t,du) \, d\mu(s, dt)
\end{align*}
We also have the special restriction $\mu \nu : S \times 
  \mathcal{U} \to [0,1]$ defined by 
\begin{align*}
\mu \nu(s, B) &= \mu \otimes  \nu(s, T \times B) = \iint \nu(s,t,B) \, d\mu(s, dt)
\end{align*}
\end{defn}

The fact that this construction defines a probability kernel is the
content of the next Lemma.
\begin{lem}\label{KernelTensorProductMeasurability}Suppose $\mu : S \times \mathcal{T} \to [0,1]$ is a
  probability kernel from $S$ to $T$ and $\nu : S \times T \times
  \mathcal{U} \to [0,1]$ be a probability kernel from $S \times T$ to
  $U$.  Let $f : S \times T \to \reals_+$ and $g
  : S \times T  \to U$  be measurable then 
\begin{itemize}
\item[(i)] $\int f (s, t) \, d\mu(s,dt)$ is a measurable function of $s \in S$.
\item[(ii)] $\mu_s \circ (g(s, \cdot))^{-1}$ is a kernel from $S$ to $U$.
\item[(iii)] $\mu \otimes \nu$ is a kernel from $S$ to $T \times U$.
\end{itemize}
\end{lem}
\begin{proof}
To see (i), we apply the standard machinery.  First consider $f(s,t) = \characteristic{A\times B}(s,t)$
for $A \in \mathcal{S}$ and $B \in \mathcal{T}$.  In this case, 
\begin{align*}
\int \characteristic{A \times B} (s, t) \, d\mu(s,dt) &=
\characteristic{A} (s)\int \characteristic{B} (t) \, d\mu(s,dt)
=\characteristic{A} (s) \mu(s,B)
\end{align*}
which is $\mathcal{S}$-measurable by measurability of $A$ and the fact
that $\mu$ is a kernel.  We extend to the case of general characteristic
functions by observing that products $A \times B$ are a generating
$\pi$-system for the $\sigma$-algebra $\mathcal{S} \otimes
\mathcal{T}$.  Additionally we must show that $\mathcal{C} = \lbrace C
\in \mathcal{S} \otimes \mathcal{T} \mid \int \characteristic{C} (s, t) \, d\mu(s,dt) \text { is measurable} \rbrace$ is a
$\lambda$-system.  To see this first assume that $A \subset B$ with
$A,B \in \mathcal{C}$.  Then by linearity of integral, $\int
\characteristic{B \setminus A} (s, t) \, d\mu(s,dt) = \int
\characteristic{B} (s, t) \, d\mu(s,dt) - \int
\characteristic{A} (s, t) \, d\mu(s,dt)$ which shows $B \setminus A \in
\mathcal{C}$.  Secondly if $A_1 \subset A_2 \subset \cdots$ is a chain
in $\mathcal{C}$ then by Monotone Convergence applied pointwise in
$s$, we have $\int \characteristic{\cup_n A_n} (s, t) \, d\mu(s,dt) =
\lim_{n\to \infty} \int \characteristic{A_n} (s, t) \, d\mu(s,dt)$
which shows $\cup_n A_n \in \mathcal{C}$ because limits of measurable functions are measurable
(Lemma \ref{LimitsOfMeasurable}).  Now an application of Theorem
\ref{MonotoneClassTheorem} shows the result.

By $\mathcal{S}$-measurability for characteristic functions and
linearity of integral, we see that $\int f (s, t) \, d\mu(s,dt)$ is
$\mathcal{S}$-measurable for simple functions and by definition of
integral we see that for any positive measurable $f$ with an
approximation by simple functions $f_n \uparrow f$ we note that for
each fixed $s$, $f_n$ are simple functions of $t$ alone so $\int f (s,
t) \, d\mu(s,dt) = \lim_{n} \int f_n (s,
t) \, d\mu(s,dt)$ showing $\mathcal{S}$-measurability by another
application of Lemma \ref{LimitsOfMeasurable}.  Lastly extending to
general integrable $f$, write $f = f_+ - f_-$ and use linearity of
integral.

Having proven (i) we derive (ii) and (iii) from it.  To see (ii)
assume that $A \in \mathcal{U}$ and note that for fixed $s$, if we
denote the section of $g$ at $s$ by $g_s : T \to U$ then it is
elementary that $\characteristic{g_s^{-1}(A)}(t) =
\characteristic{g^{-1}(A)}(s,t)$ and thus
\begin{align*}
\mu_s \circ (g(s, \cdot))^{-1}(A) &= \mu(s, g^{-1}(s, A)) = \mu(s, g^{-1}(A)) 
\end{align*}
which we have shown is $\mathcal{S}$-measurable in (i).

To see (iii), pick $A \in \mathcal{T} \otimes \mathcal{U}$ and recall
that by definition
\begin{align*}
\mu \otimes \nu (A) (s) &= \iint \characteristic{A}(t,u) \,
d\nu(s,t,du) \, d\mu(s, dt)
\end{align*}
We know that $\characteristic{A}(t,u)$ is $\mathcal{T} \otimes
\mathcal{U}$-measurable hence also $\mathcal{S} \otimes \mathcal{T} \otimes
\mathcal{U}$-measurable.  Therefore we can apply (i) to conclude that $\int \characteristic{A}(t,u) \,
d\nu(s,t,du)$ is $\mathcal{S} \otimes \mathcal{T}$-measurable.  Now
apply (i) again to conclude that $\mu \otimes \nu (A) (s)$ is $\mathcal{S}$-measurable.
\end{proof}

TODO: There should be a ``twisted'' Fubini Theorem for kernels that says
\begin{align*}
\iint f(y,z) \, (\mu \otimes \nu)(x, dy, dz) &= \iint f(y,z) \,\nu(x,y,dz) \mu(x, dy)
\end{align*}
which should certainly be true of probability and finite kernels.  Given that Fubini is not true for 
non $\sigma$-finite measures then presumably the result above will not hold for that case either (in fact
I haven't yet been through the exercise of figuring out exactly where we use the finiteness of measures in
the kernel results presented).  To be honest, I am sure I have used this twisted Fubini result without mention
elsewhere.

\begin{examp}\label{ProbabilityKernelProductFiniteSampleSpace}
This continues Example \ref{ProbabilityKernelFiniteSampleSpace}.  For finite probability spaces $S$, $T$ and $U$ a probability kernel
$\mu : S  \to \mathcal{P}(T)$ is a stochastic matrix $\mu_{s,t}$ and a
probability kernel $\nu : S \times T \to \mathcal{P}(U)$ is a $(S
\times T) \times U$
stochastic matrix $\nu_{s,t, u}$ where we consider the pair $(s,t)$ to
the row index.  If we now identify $(t,u)$ as column index in the 
$S \times (T \times U)$ matrix $\mu \otimes \nu$ then 
\begin{align*}
(\mu \otimes
\nu)_{s,t,u} 
&= (\mu \otimes\nu)(s, \lbrace (t,u) \rbrace) = \iint
\characteristic{\lbrace (t,u) \rbrace} (x,y) \, d\nu(s,x,dy) \,
d\mu(s, dx) \\
&= \int \characteristic{\lbrace (t) \rbrace} (x) \nu(s,x, \lbrace{u}) \,
d\mu(s, dx) \\
&=\mu_{s,t} \nu_{s,t,u} 
\end{align*}

There is a particularly important special case of this special case.
Consider the case of $\mu : S \to \mathcal{P}(T)$  and $\nu : T \to
\mathcal{P}(T)$.  We can apply the kernel product $\mu \otimes \mu : S
\to T \times U$ to sets of the form $T \times \lbrace u \rbrace$ for
$u \in U$ and we get
\begin{align*}
(\mu \nu)(s, \lbrace u \rbrace) &= (\mu \otimes \nu)(s, T \times
\lbrace u \rbrace) \\
&=\sum_{t \in T}  (\mu \otimes \nu)(s,  \lbrace (t,u) \rbrace) \\
&= \sum_{t \in T} \mu_{s,t} \nu_{s,t,u} 
\end{align*}
so the product $\mu \nu$ is simply the matrix product.
\end{examp}

\begin{examp}
Let $(S, \mathcal{S})$ be a measurable space and let $f : S \to
[0,\infty)$ be a bounded measurable function.  Let $\mu : S \times \mathcal{S} \to [0,\infty)$ be defined by 
$\mu(x, A) = f(x) \delta_x(A)$, then $\mu$ is a kernel (not a probability kernel however).  Measurability of $\mu(x,A)$ for 
fixed $A$ follows immediately from the measurability of $f(x)$ and the measurability of $A$ (apply Proposition \ref{prop:SimpleFunctions} noting $\delta_x(A) = \characteristic{A}(x)$).
\end{examp}

It shall also be useful to show that we can construct a parameterized
family of random elements whose distributions are given by a specified
kernel.
\begin{lem}\label{RandomizationAndKernels}Let $(S,\mathcal{S})$ and $(T, \mathcal{T})$ be measurable
  spaces with $T$ a Borel space and let $\mu : T \times \mathcal{S}
  \to [0,1]$ be a probability kernel.  There exists a measurable
  function $G : S \times [0,1] \to T$ such if $\vartheta$ is a
  $U(0,1)$ random variable then $G(s, \vartheta)$ has distribution
  $\mu(s, \cdot)$ for all $s \in S$.
\end{lem}
\begin{proof}
First assume that $T = [0,1]$; we replay the argument of
Lemma \ref{LebesgueStieltjesMeasure} pointwise in $S$.  
Let 
\begin{align*}
G(s, t) &= \sup \lbrace u \in [0,1] \mid \mu(s,[0,u]) < t \rbrace
\text{ for $s \in S$ and $t \in [0,1]$}
\end{align*}
We claim that $G(s,t)$ is $\mathcal{S} \otimes
\mathcal{B}([0,1])$-measurable.  First note that if we define
\begin{align*}
G^{\rationals} (s, t) &= \sup \lbrace u \in [0,1] \cap \rationals \mid \mu(s,[0,u]) < t \rbrace
\text{ for $s \in S$ and $t \in [0,1]$}
\end{align*}
then in fact $G^{\rationals} = G$.  To see this, it is clear that
$G^{\rationals} \leq G$.  For the other inequality, let $s \in S$ and
$t \in [0,1]$ be given and pick an arbitrary $\epsilon > 0$;  let $u
\in [0,1]$ be such that $G(s,t) - \epsilon < \mu(s,[0,u])$.  Now take
a sequence of $q_n \in [0,1] \cap \rationals$ such that $q_n
\downarrow u$ and use continuity of measure to conclude that $\lim_{n
  \to \infty } \mu(s, [0,q_n]) = \mu(s, [0,u]) < t$ so there is a $q \in
  [0,1] \cap \rationals$ such that $q \geq x$ and $\mu(s,[0,q]) < t$.
  This proves that $G^{\rationals}(s,t) \geq G(s,t) - \epsilon$ and since
  $\epsilon > 0$ was arbitrary we have the desired equality.  Now for
  any $y \in [0,1]$ we can write
\begin{align*}
\lbrace (s,t) \mid G(s,t) \leq y \rbrace &= \cap_{\substack{q \leq y
    \\ q \in \rationals}} \lbrace (s,t) \mid \mu(s,
[0, q] ) \leq y \rbrace
\end{align*}
and each $\lbrace (s,t) \mid \mu(s,[0, q]) \leq y \rbrace$ is
measurable for fixed $q$ since $\mu(s,[0,q])$ is a measurable function
of $s$ (e.g. observe $\lbrace (s,t) \mid \mu(s,[0, q]) \leq y \rbrace
= \lbrace (s,t) \mid t - \mu(s,[0, q]) \geq 0 \rbrace$ and use the
measurability of the function $g(s,t) = t - \mu(s,[0, q])$).

Now note that
\begin{align*}
\probability{ G(s,\vartheta) \leq u } &= \probability{\vartheta \leq
  \mu(s,[0,u])} = \mu(s,[0,u])
\end{align*}
and therefore $G(s,\vartheta) \eqdist \mu(s,\cdot)$ by Lemma \ref{DistributionFunctionCharacterizeProbability}.

To extend to general Borel spaces $T$, first suppose that $T \in
\mathcal{B}([0,1])$.  Given a probability kernel $\mu : S \times \mathcal{T}
\to [0,1]$ we define $\tilde{\mu} : S \times \mathcal{B}([0,1]) \to
[0,1]$ by $\tilde{\mu}(s, A) = \mu(s, A \cap T)$.  It is clear that
$\tilde{\mu}(s, \cdot)$ is a probability measure for all $s \in S$ and
furthermore since $A \cap T \in \mathcal{T}$ we know that
$\tilde{\mu}(s, A)$ is $\mathcal{S}$-measurable for every $A \in
\mathcal{B}([0,1])$ hence $\tilde{\mu}$ is a probability kernel (Lemma
\ref{MeasurableMappingsOfMeasures} and Lemma \ref{KernelMeasurability}).  Note
that by construction for all $s \in S$ we have $\tilde{\mu}(s, T^c) =
\mu(s, T \cap T^c) = 0$.
Applying the result for $[0,1]$ we get a measurable $\tilde{G} : S \times
[0,1] \to [0,1]$ such that $\probability{ \tilde{G}(s,\vartheta) \in A
} = \mu(s, A)$.  Pick an arbitrary point $t_0 \in T$ and define 
\begin{align*}
G(s,t) = \characteristic{\tilde{G}^{-1}(T)}(s,t) G(s,t) + t_0 \characteristic{\tilde{G}^{-1}(T^c)}(s,t)
\end{align*}
$G(s,t)$ is a measurable function $G : S \times [0,1]
\to T$.  Furthermore for all $s in S$ and $A \in \mathcal{T}$, 
\begin{align*}
\probability{G(s, \vartheta) \in A} &= \begin{cases}
\probability{\tilde{G}(s, \vartheta) \in A } & \text{if $t_0 \notin A$} \\
\probability{\tilde{G}(s, \vartheta) \in A } +
\probability{\tilde{G}(s,t) \in T^c} & \text{if $t_0 \in A$}
\end{cases} \\
&= \probability{\tilde{G}(s,\vartheta) \in A} = \tilde{\mu}(s, A) = \mu(s, A \cap T) = \mu(s,A)
\end{align*}
proving the result for Borel subsets of $[0,1]$.

Lastly suppose $T$ is Borel isomorphic to a Borel subset of $[0,1]$
and let $\mu: S \times \mathcal{T} \to [0,1]$ be a probability
kernel.  If $A \in \mathcal{B}([0,1])$ and $g : T \to A$ is a Borel
isomorphism then note that $\pushforward{g}{\mu}(s, A) = \mu(s,
g^{-1}(A)$ defines a probability kernel $\pushforward{g}{\mu} : S
\times A \cap \mathcal{B}([0,1]) \to [0,1]$.  It is clear that if
select $G : S \times [0,1] \to A$ as above then $G \circ g^{-1} : S
\times [0,1] \to T$ is measurable and 
\begin{align*}
\probability{g^{-1}(G(s, \vartheta)) \in B } &= \probability{G(s,
  \vartheta) \in g(B) } \\
&=\pushforward{g}{\mu}(s, g(B)) = \mu(s, B)
\end{align*}
so $G \circ g^{-1}$ proves the result.
\end{proof}

\begin{lem}\label{InducedBijectionOnSigmaAlgebras}Let $(S,
  \mathcal{S})$ and $(T, \mathcal{T})$ be measurable spaces and let $f
  : S \to T$ be a Borel isomorphism then $f^{-1} : \mathcal{T} \to
  \mathcal{S}$ is a bijection.
\end{lem}
\begin{proof}
Since a Borel isomorphism is a bijection we know that $f^{-1} : 2^T
\to 2^S$ is a bijection (Lemma
\ref{BijectivityOfInducedSetMap}).  By measurability of $f$ we know
that $f^{-1} (\mathcal{T}) \subset \mathcal{S}$.  Moreover for any $A \in
\mathcal{S}$ by measurability of we know that $f^{-1}$ $\lbrace t \in T \mid f(t) \in A \rbrace \in
\mathcal{T}$ and clearly $f^{-1} \lbrace
t \in T \mid f(t) \in A \rbrace = \lbrace s \in S \mid f(f^{-1}(s))
\in A \rbrace = A$ since $f$ is a bijection.
\end{proof}


\begin{lem}\label{InducedBorelIsomorphismOnProbabilityMeasures}Let $(S, \mathcal{S})$ and $(T, \mathcal{T})$ be measurable
  spaces and let $f : S \to T$ be a Borel isomorphism, then the map
  $f_* : \mathcal{P}(S) \to \mathcal{P}(T)$ given by $f_*(\mu)(A) =
  \mu(f^{-1}(A))$ is a Borel isomorphism with $(f_*)^{-1} = (f^{-1})_*$.
\end{lem}
\begin{proof}
We first show $f_*$ is measurable.  Let $F \in \mathcal{T}$ and let $G
\in \mathcal{B}([0,1])$ and consider the measurable set $\lbrace \mu
\mid \mu(F) \subset G \rbrace \subset \mathcal{P}(T)$.  Since $f$ is
measurable we know that $f^{-1}(F) \in \mathcal{S}$ and therefore
\begin{align*}
f_*^{-1} \lbrace \mu \mid \mu(F) \subset G \rbrace &= \lbrace \mu \mid
f_*\mu(F) \subset G \rbrace= \lbrace \mu \mid \mu(f^{-1}(F)) \subset G \rbrace
\end{align*}
is measurable in $\mathcal{P}(S)$.
Since the $\sigma$-algebra on
$\mathcal{P}(T)$ is generated by sets of the form $\lbrace \mu
\mid \mu(F) \subset G \rbrace$, measurability of $f_*$ follows from
Lemma \ref{MeasurableByGeneratingSet}.

Since $f$ is a Borel isomorphism, we know $(f^{-1})_* :
\mathcal{P}(T) \to \mathcal{P}(S)$ is well defined and measurable and
we can compute that for all $A \in \mathcal{S}$ and $\mu \in
\mathcal{P}(S)$ we have
\begin{align*}
(f^{-1})_* f_*\mu(A) &= f_*\mu(f(A)) = \mu(f^{-1}(f(A))) = \mu(A)
\end{align*}
so that $(f^{-1})_* \circ f_* = id$.  By symmetry we have $f_* \circ (f^{-1})_* = id$ and the result is shown.
\end{proof}

\begin{thm}\label{ExistenceConditionalDistribution}Let $(S, \mathcal{S})$ be a Borel space and
  $(T, \mathcal{T})$ be an arbitrary measuable space.  Let $\xi$ be a
  random element in $S$ and $\eta$ be a random element in $T$.  There
  is exists a probability kernel $\mu : T \times \mathcal{S} \to
  \reals$ such that $\cprobability{\eta }{\xi \in A}(\omega) =
  \mu(\eta(\omega), A)$ for all $A \in \mathcal{S}$ and $\omega \in
  \Omega$.  Furthermore, if $\tilde{\mu}$ is another probability
  kernel satisfying this property then $\mu = \tilde{\mu}$ almost
  surely with respect to $\mathcal{L}(\eta)$.
\end{thm}
\begin{proof}
TODO:  Reduce to the case of $S = \reals$ and use density of rationals
and properties of distribution functions to create a regular version.

Now we show how to handle the case of general Borel $S$.  Let $A$ be a
Borel subset of $\reals$ and let $j : S \to A$ be a Borel
isomorphism.  We apply the result just proven to $j  \circ \xi :
\Omega \to A$ and get the existence of a probability kernel $\tilde{\mu} : T
\to \mathcal{P}(A)$ such that $\cprobability {\eta}{j \circ \xi \in B}
= \tilde{\mu}(\eta, B)$ for all Borel subsets $B \subset A$.  By Lemma
\ref{InducedBorelIsomorphismOnProbabilityMeasures} we know that $j_* :
\mathcal{P}(S) \to \mathcal{P}(A)$ is a Borel isomorphism so we can
define $\mu = j^{-1}_* \circ \tilde{\mu}$ which is a
probability kernel by Lemma \ref{KernelMeasurability}.  Because $j$ is
a Borel isomorphism, we know that every measurable subset of $S$ is of
the form $j^{-1}B$ for some Borel $B \subset A$ (Lemma
\ref{InducedBijectionOnSigmaAlgebras}) and we have
\begin{align*}
\cprobability{\eta}{\xi \in j^{-1}B} &= \cprobability{\eta}{j \circ
  \xi \in B} = \tilde{\mu}(\eta, B) = \mu(\eta, j^{-1}B)
\end{align*}
\end{proof}

The following theorem is an absolutely essential tool for computing
conditional expectations.  Suppose we are given a random variable that is a real valued function applied
to  a pair of random elements $f(\xi, \eta)$.   In the case that $\xi$ and $\eta$ are indendent we applied the Expectation
Rule and Fubini's Theorem in Lemma \ref{DisintegrationIndependentLaws} to calculate the expected value of $f(\xi, \eta)$ as an
iterated integral.  Naively we might expect that in the general case we can calculate the expectation
of $f$ conditioned on $\eta$ by fixing the value of $\eta$ and then taking an ``appropriate'' expectation'.  The 
appropriate notion of expectation is given by integration against the distribution of $\xi$ conditional on $\eta$.

TODO: Show how this works in the discrete case

\begin{thm}\label{Disintegration}Let $(S, \mathcal{S})$ and $(T, \mathcal{T})$ be measurable
  spaces and let $\xi$ be a random element in $S$ and $\eta$ be a
  random element in $T$.  Suppose 
\begin{itemize}
\item[(i)] $\cprobability{\mathcal{F}}{\xi \in
    \cdot}$ has a regular version $\nu : \Omega \times \mathcal{S} \to
  \reals$ 
\item[(ii)] $\eta$ is  $\mathcal{F}$-measurable 
\item[(iii)] $f : S \times T \to \reals$ is
  measurable with either $f \geq 0$ or $\expectation{\abs{f(\xi,
      \eta)}}< \infty$
\end{itemize}
Then 
\begin{align*}
\expectation{f(\xi, \eta)} &= \expectation{\int f(s, \eta) \, d\nu(s)}\\
\intertext{and moreover}
\cexpectationlong{\mathcal{F}}{f(\xi, \eta)} &= \int f(s, \eta) \, d\nu(s)
\text{ a.s.} \\
\end{align*}
\end{thm}
\begin{proof}
The proof is an application of the standard machinery.
To start with we assume that $f = \characteristic{A \times B}$ for $A
\in \mathcal{S}$ and $B \in \mathcal{T}$.  
Then 
\begin{align*}
\expectation{f(\xi, \eta)}  &= 
\expectation{\characteristic{A}(\xi) \characteristic{B}(\eta)} \\
&= \expectation{\cexpectationlong{\mathcal{F}}{\characteristic{A}(\xi)} \characteristic{B}(\eta) } \\
&= \expectation{\nu(A) \characteristic{B}(\eta)} \\
&= \expectation{\int \characteristic{A}(s) \characteristic{B}(\eta)
  d\nu(s) } \\
&= \expectation{\int f(s, \eta)  d\nu(s) } \\
\end{align*}

Now we extend to the set of all $C \in \mathcal{S}\otimes
\mathcal{T}$ by using a Monotone Class Argument (Theorem
\ref{MonotoneClassTheorem}).  Let $\mathcal{C} = \lbrace C \in \mathcal{S}\otimes
\mathcal{T} \mid \expectation{\characteristic{C}(\xi, \eta)} = \expectation{\int
  \characteristic{C}(s, \eta)  d\nu(s) } \rbrace$
Since the set of all $A \times B$ is a $\pi$-system
containing $S \times T$ it suffices to show that $\mathcal{C}$ is a $\lambda$-system.
Suppose $C, D \in \mathcal{D}$ and $C \subset D$; then we see $D
\setminus C \in \mathcal{C}$ by
noting $\characteristic{D \setminus C} = \characteristic{D} -
\characteristic{C}$ and applying linearity of expectation and
integral.  If we assume $C_1 \subset C_2 \subset \cdots$ with $C_n \in
\mathcal{C}$, then
$\characteristic{\cup_n C_n} = \lim_{n \to \infty}
\characteristic{C_n}$ and the Monotone Convergence Theorem implies
$\expectation{\characteristic{\cup_n C_n}(\xi, \eta)} = \lim_{n \to
  \infty}\expectation{\characteristic{C_n}(\xi, \eta)}$.  Similarly
for fixed $\omega \in \Omega$, $\int \characteristic{\cup_n C_n}(s,
\eta) \, d\nu(s) = \lim_{n \to \infty} \int \characteristic{C_n}(s,
\eta) \, d\nu(s)$, moreover monotonicity of integral implies that $\int \characteristic{C_n}(s,
\eta) \, d\nu(s)$
is increasing in $n$.  Therefore we may apply Monotone Convergence a
second time to conclude that
\begin{align*}
\expectation{\int \characteristic{\cup_n C_n}(s,\eta) \, d\nu(s) } &=
\lim_{n\to \infty}\expectation{\int \characteristic{C_n}(s,\eta) \, d\nu(s) } 
\end{align*}
Therefore we see that $\cup_n C_n \in
\mathcal{C}$.  

Extending the result to simple functions is trivial since both sides
are linear in $f$.

Now we suppose that $f : S \times T \in \reals$ is positive
measurable.  We pick an approximation of $f$ by an increasing sequence
of positive simple functions $0 \leq f_n \uparrow f$.  Now $f_n(\xi,
\eta)$ is an increasing sequence of positive simple functions with
$\lim_{n \to \infty} f_n(\xi, \eta) = f(\xi, \eta)$ and therefore by
definition of expectation, $\expectation{f(\xi, \eta)} = \lim_{n \to
  \infty} \expectation{f_n(\xi, \eta)}$.  Similarly for fixed $\omega
\in \Omega$ we have $f_n(s, \eta)$ are positive simple functions increasing to
$f(s, \eta)$ and therefore $\int f(s, \eta) \, d\nu(s) = \lim_{n \to
  \infty} \int f_n(s, \eta) \, d\nu(s)$.  Monotonicity of integral shows
that  the sequence $\int f_n(s, \eta) \, d\nu(s)$ is positive and
increasing and therefore we may apply Monotone Convergence and the
fact that result holds for the $f_n$ to show
that 
\begin{align*}
\expectation{\int f(s, \eta) \, d\nu(s)} &= \lim_{n \to
  \infty} \expectation{\int f_n(s, \eta) \, d\nu(s)} = \lim_{n \to \infty} \expectation{f_n(\xi, \eta)} = \expectation{f(\xi, \eta)}
\end{align*}  
Therefore the
result for positive measurable $f$.

Lastly for general integrable $f$, we know by the result for positive
$f$ that
\begin{align*}
\expectation{\int \abs{f(s, \eta)} \, d\nu(s)} &=
\expectation{\abs{f(\xi, \eta)}} < \infty
\end{align*}
Which shows us that $\int \abs{f(s, \eta)} \, d\nu(s) < \infty$ almost
surely.  Then we can write $f = f_+ - f_-$ and use the the result for
postive $f$ and linearity.

The last thing to do is to extend the result to the case of
conditional expectations.  Let $f : S \times T \to \reals_+$ be
positive and let $A \in \mathcal{F}$.  Consider $(\eta,
\characteristic{A})$ as a random element of $T \times \lbrace 0,1\rbrace$.  Note that this
random element is $\mathcal{F}$-measurable since $\eta$ is and $A \in
\mathcal{F}$.  Therefore we can apply the case just proven to the
function $\tilde{f} : S \times T \times \lbrace 0,1 \rbrace \to
\reals_+$ given by $\tilde{f}(s,t,u) = u f(s,t)$ and the elements $\xi$
and $(\eta, \characteristic{A})$ to get
\begin{align*}
\expectation{f(\xi, \eta) ; A} &= \expectation {\int f(s, \eta)
  \characteristic{A} \, d\nu(s)} = \expectation {\int f(s, \eta)
  \, d\nu(s) ; A}
\end{align*}
which shows that $\cexpectationlong{\mathcal{F}}{f(\xi,\eta)} = \int f(s, \eta)
  \, d\nu(s)$ a.s. for $f \geq 0$.  The case of integrable $f$ follows as
  usual by taking differences.
\end{proof}

\begin{thm}[Jensen's Inequality]\label{JensenConditionalExpectation}Let $\xi$ be a random vector and $\mathcal{F}$ be a
  $\sigma$-algebra.  If $\varphi$ is a convex function then
  $\varphi(\cexpectationlong{\mathcal{F}}{\xi}) \leq
    \cexpectationlong{\mathcal{F}}{\varphi(\xi)}$ a.s.
If $\varphi$ is strictly convex then $\varphi(\cexpectationlong{\mathcal{F}}{\xi}) =
    \cexpectationlong{\mathcal{F}}{\varphi(\xi)}$ if and only if $\xi =
      \cexpectationlong{\mathcal{F}}{\xi}$ a.s.
\end{thm}
\begin{proof}
Since $\reals^n$ is Borel by Theorem
\ref{ExistenceConditionalDistribution} we know
$\cprobability{\mathcal{F}}{\xi \in \cdot}$ has regular version
$\mu$.  Now by Theorem \ref{Disintegration} and the ordinary Jensen
Inequality (Lemma \ref{Jensen}) applied pointwise we know that 
\begin{align*}
\varphi(\cexpectationlong{\mathcal{F}}{\xi}) &= 
\varphi \left( \int s \, \mu(ds) \right) \leq \int \varphi (s)
\, \mu(ds) = \cexpectationlong{\mathcal{F}}{\varphi(\xi)}
\end{align*}

TODO: The strictly convex/equality case
\end{proof}

As another application of Theorem \ref{Disintegration} we give a
little result about the interaction between conditional indpendence
and conditional expectations.
\begin{cor}\label{ConditionalIndependenceConditionalExpectations}Let $\xi$ be a random element in $S$ such that
  $\cprobability{\mathcal{G}}{\xi  \in \cdot}$
has a regular version.  Then if
$\cindependent{\xi}{\mathcal{G}}{\mathcal{F}}$ and $f : S \to \reals$ is
measurable then
$\cexpectationlong{\mathcal{G}}{f(\xi)} = \cexpectationlong{\mathcal{F}, \mathcal{G}}{f(\xi)}$.
\end{cor}
\begin{proof}
Let $\mu$ be a regular version of $\cprobability{\mathcal{G}}{\xi  \in
  \cdot}$ .  By Lemma \ref{ConditionalIndependenceDoob} we know that
$\cprobability{\mathcal{G}}{\xi  \in \cdot} =
\cprobability{\mathcal{F}, \mathcal{G}}{\xi  \in \cdot}$ and therefore
$\mu$ is a regular version for $\cprobability{\mathcal{F}, \mathcal{G}}{\xi  \in \cdot}$ as well and
by Theorem \ref{Disintegration}
\begin{align*}
\cexpectationlong{\mathcal{G}}{f(\xi)} &= \int f(s) \, \mu(ds) =
\cexpectationlong{\mathcal{F}, \mathcal{G}}{f(\xi)} \text { a.s.}
\end{align*}

TODO: Is there a proof of this result that doesn't require the
existence of regular versions?
\end{proof}
 
Special case of random vectors with densities.  Suppose we are given
$\xi : \Omega \to \reals^m$ and $\eta: \Omega \to \reals^n$ such that
$(\xi,\eta)$ has density $f$ on $\reals^{m+n}$.  Then $\xi$ and $\eta$ have
densities $f_{\xi}$ and $f_\eta$ called the marginal densities and we
get a conditional densities $f(x,y)/f_\xi(x)$ and $f(x,y)/f_\eta(y)$.
TODO: Tie this back to conditional distributions as defined in the
general case (this is an exercise in Kallenberg for example).

For random vectors, the existence of regular versions allows us to
bring the theory of characteristic functions to bear on problems.

\begin{lem}\label{ConditionalCharacteristicFunctions}Let $\xi$ be a random vector in $\reals^n$ and let
  $\mathcal{F}$ be a $\sigma$-algebra.  Suppose that $\phi : \reals^n
  \times \Omega \to \complexes$ is a function such that for each fixed
  $u \in \reals^n$ we have
\begin{align*}
\phi(u, \omega) &= \cexpectationlong{\mathcal{F}}{e^{i\langle u,\xi
    \rangle}} \text{ a.s.}
\end{align*}
If for every $\omega \in \Omega$ there is a probability measure
$\mu(\omega)$ on $(\reals^n, \mathcal{B}(\reals^n))$ such that
$\phi(u, \omega) = \int e^{i\langle u, x \rangle} \, \mu(\omega, dx)$
then it follows that for every $A \in \mathcal{B}(\reals^n)$ we have
\begin{align*}
\cprobability{\mathcal{F}}{\xi \in A}(\omega) &= \mu(\omega, A) \text{ a.s.}
\end{align*}
\end{lem}
\begin{proof}
By Theorem \ref{ExistenceConditionalDistribution} we know that we may
chose a regular version $\nu$ for $\cprobability{\mathcal{F}}{\xi \in
  \cdot}$.  By Theorem \ref{Disintegration} we know that for every
fixed $u \in \reals^n$ we have
\begin{align*}
\cexpectationlong{\mathcal{F}}{e^{i \langle u, \xi \rangle}} &=
\phi(u, \omega) = \int
e^{i \langle u, x \rangle} \, \nu(\omega, dx)
\end{align*} 
almost surely and by
taking a countable intersection of almost sure events we may assume
that $\phi(u, \omega) = \int
e^{i \langle u, x \rangle} \, \nu(\omega, dx)$ for all $u \in
\rationals^n$ almost surely.  For each fixed $\omega$, both sides of
this equation are characteristic functions of a probability measure
hence each side is uniformly continuous (Lemma
\ref{CharacteristicFunctionBoundedAndContinuous}) and therefore
equality on $\rationals^n$ can be upgraded to equality on $\reals^n$.
Now the characteristic function uniquely identifies the underlying
probability measure Theorem
\ref{EqualCharacteristicFunctionEqualMeasures} and therefore 
\begin{align*}
\mu(\omega, \cdot) = \nu(\omega, \cdot) =
\cprobability{\mathcal{F}}{\xi \in A}(\omega) \text{ a.s.}
\end{align*}
\end{proof}

 We've seen that given a specified distribution we can always find a
random variable with that specfied distribution.  Moreover, we know
that if we allow ourselves to to extend the probability space then we
can construct such a random variable to be independent of any existing
random elements (or $\sigma$-algebras).  We now turn our attention to
the analogous problem space for conditional distributions.  The
simplest such result shows that given a random element and a
prescribed probability kernel we can always find a second random
element whose conditional distribution given the first random element is the kernel.
\begin{lem}Let $(S, \mathcal{S})$ and $(T, \mathcal{T})$ be measurable
  spaces, $\mu : T \times \mathcal{S} \to \reals$ be a
  probability kernel and $\eta$ be a random element in $T$.  There
  exists an extension $\hat{\Omega}$ and a random element $\xi : \hat{\Omega} \to S$
  such that $\cprobability{\eta}{\xi \in \cdot} = \mu(\eta, \cdot)$
  a.s.   and $\cindependent{\xi}{\zeta}{\eta}$ for every random element
  $\zeta$ defined on $\Omega$.
\end{lem}
\begin{proof}
The appropriate construction is thrust upon us by Theorem
\ref{Disintegration}.  Note that if we succeed in constructing $\xi$
then that result tells how to compute expectations on $\hat{\Omega}$.
Following that lead, let $(\Omega, \mathcal{A}, P)$ be the probability
space underlying the random element $\eta$ and define
$(\hat{\Omega}, \hat{\mathcal{A}}) = (S \times \Omega,
\mathcal{S} \otimes \mathcal{A})$.  Define the probability measure 
\begin{align*}
\hat{P}(A) &= \expectation{\int \characteristic{A}(s,\omega) \,
  d\mu(\eta(\omega), s)}
\end{align*}
Note that $(\hat{\Omega}, \hat{\mathcal{A}},\hat{P})$ is an extension
of $(\Omega, \mathcal{A}, P)$ since for $A \in \mathcal{A}$, 
\begin{align*}
\hat{P}(S \times A) &= \expectation{\int \characteristic{S}(s)\characteristic{A}(\omega) \,
  d\mu(\eta(\omega), s)} = \expectation{\characteristic{A}(\omega)} = P(A)
\end{align*}

Now define $\xi(s, \omega) = s$ and note that for $A \in \mathcal{S}$
and $B \in \mathcal{A}$, 
\begin{align*}
\hat{P}(\xi \in A ; B) &=  \expectation{\int \characteristic{A}(s)\characteristic{B}(\omega) \,
  d\mu(\eta(\omega), s)} = \expectation{\mu(\eta, A) ; B}
\end{align*}
which shows $\cprobability{\mathcal{A}}{\xi \in A}= \mu(\eta, A)$ a.s. 
by the defining property of conditional expectation (note that since
$\mu(\eta, A)$ and $\characteristic{B}$ are both
$\mathcal{A}$-measurable, their expectation with respect to $P$ is the
same as their expectation with respect to $\hat{P}$).  In particular,
since we know that $\mu(\eta, A)$ is $\eta$-measurable we also know
that $\cprobability{\mathcal{A}}{\xi \in A}=\cprobability{\eta}{\xi \in A} = \mu(\eta, A)$.

This last observation also shows $\cindependent{\xi}{\mathcal{A}}{\eta}$ by an application of Lemma \ref{ConditionalIndependenceDoob}.
\end{proof}

The next result is closely related to the previous lemma and provides
an answer to a very natural question.  Suppose that one is given
measure $\mu$ on a product space $S \times T$.  It is trivial that one can
construct random elements $\xi$ and $\eta$ in $S$ and $T$ respectively
such that the law of $(\xi, \eta)$ is $\mu$ (just take the probability
space to be $(S \times T, \mathcal{S} \times \mathcal{T}, \mu)$ and
then use the identity map).  Now suppose that one is given a random
element $\eta$ in $T$ such that the law of $\eta$ is the marginal of
$\mu$ (i.e. $\probability{\eta \in B} = \mu(S \times B)$); one may ask
whether one can find a random element $\xi$ in $S$ such that the law
of $(\xi, \eta)$ is $\mu$.  If $\mu$ is a product measure then this
follows whenever $S$ is Borel by creating a $\xi$ independent of
$\eta$ such that $\probability{\xi \in A} = \mu(A \times T)$.   In
fact due to the existence of conditional distributions a similar proof
works for general $\mu$.
The result is expressed in terms of random elements in $S \times T$
rather a probability measure $\mu$ and the proof shows that the
construction can be done with a single uniform randomization variable
in addition to $\eta$.

\begin{lem}\label{Transfer}Let $(S, \mathcal{S})$ be a Borel space and $(T, \mathcal{T})$ be 
  a general measurable space.  Let $\xi$ be a random element in $S$
  and let $\eta$ be a random element in $T$ both defined on a 
  probability space $(\Omega, \mathcal{A})$.  Let $\tilde{\eta}$ be a
  random element in $T$ defined on a probability space
  $(\tilde{\Omega}, \tilde{\mathcal{A}})$ and assume that
  $\eta \eqdist \tilde{\eta}$.  Then there
  exists a measurable function $f : T \times [0,1] \to S$ such that if
  $\vartheta$ is a $U(0,1)$ random variable defined on
  $(\tilde{\Omega}, \tilde{\mathcal{A}})$ with
  $\cindependent{\vartheta}{\tilde{\eta}}{}$ and we define
  $\tilde{\xi} = f(\tilde{\eta}, \vartheta)$ then $(\xi, \eta) \eqdist
  (\tilde{\xi}, \tilde{\eta})$.
\end{lem}
\begin{proof}
By Theorem \ref{ExistenceConditionalDistribution} we have a probability kernel
$\mu : T \times \mathcal{S} \to \reals$ such that
$\cprobability{\eta}{\xi \in \cdot} = \mu(\eta, \cdot)$.  

Furthermore, we know by Lemma \ref{RandomizationAndKernels} we can find measurable $f : T
\times [0,1] \to S$ such that for every $t \in T$ the distribution of $f(t, \vartheta)$ is
$\mu(t)$.  Now define $\tilde{\xi} = f(\tilde{\eta}, \vartheta)$,
assume we have a measurable $g : S \times T \to \reals_+$ and
calculate
\begin{align*}
&\expectation{g(\tilde{\xi}, \tilde{\eta})} \\
&=\expectation{g(f(\tilde{\eta}, \vartheta), \tilde{\eta})} \\
&= \expectation{\int_0^1 g(f(\tilde{\eta},x), \tilde{\eta}) \, dx} & &
\text{$\tilde{\eta} \Independent \vartheta$, Lemma
  \ref{DisintegrationIndependentLaws} and Lemma \ref{ExpectationRule}}\\
&= \expectation{\int_0^1 g(f(\eta,x), \eta) \, dx} & &
\text{since $\eta \eqdist \tilde{\eta}$}\\
&= \expectation{\int g(s, \eta) \, d\mu(\eta,s)} & & \text{by
  Expectation Rule
  Lemma \ref{ExpectationRule} and $\mathcal{L}(f(t,\vartheta)) = \mu(t)$} \\
&= \expectation{g(\xi, \eta)} & & \text{by Theorem \ref{Disintegration}}
\end{align*}
which shows in particular that $(\xi, \eta) \eqdist (\tilde{\xi},
\tilde{\eta})$.  Note that in applying the fact that $\eta \eqdist
\tilde{\eta}$ we are moving from taking expectations against
$\tilde{\Omega}$ to taking expectations against $\Omega$.
\end{proof}

\begin{lem}\label{SolvingStochasticEquations}Let $S$ and $T$ be Borel spaces with $f : S \to T$
  measurable and let $\xi$ be a random
  element in $S$ and $\eta$ be a random element in $T$ such that
  $f(\xi) \eqdist \eta$.  Then there exists a random element in $S$
  $\tilde{\xi}$ such that $\xi \eqdist \tilde{\xi}$ and $f(\tilde{\xi}) = \eta$ a.s.
\end{lem}
\begin{proof}
Since $f(\xi) \eqdist \eta$ and $S$ is Borel by Lemma \ref{Transfer}
we can find $\tilde{\xi} \eqdist \xi$ such that $(\xi, f(\xi)) \eqdist
(\tilde{\xi}, \eta)$.  Now applying the measurable function $f \times
id : S \times T \to T \times T$ we conclude that $(f(\xi), f(\xi))
\eqdist (f(\tilde{\xi}), \eta)$.  Because the diagonal $\Delta \subset
T \times T$ is measurable (TODO: Do we really need Borel-ness for
this?) we can conclude 
\begin{align*}
\probability{f(\tilde{\xi}) = \eta} &= \probability{(f(\tilde{\xi}),
  \eta) \in \Delta} = 
\probability{(f(\xi), f(\xi)) \in \Delta} =1
\end{align*}
\end{proof}


\section{Applications}

\subsection{A Bootstrap Central Limit Theorem}

TODO: Give some description of the Bootstrapping procedure of Efron.

Let $\xi_1, \dotsc, \xi_n$ be an independent random sample from the distribution of the random vector $\xi$.  
\begin{thm}
Let $\xi, \xi_1, \xi_2, \dotsc$ be i.i.d. random vectors in $\reals^d$
with $\expectation{\xi^2} < \infty$, for
every $n \in \naturals$ let $\xi^B_{n1}, \dotsc, \xi^B_{nn}$ be
conditionally i.i.d. relative to $\xi_1, \dotsc, \xi_n$ with 
\begin{align*}
\cprobability{\xi_1, \dotsc, \xi_n}{\xi^B_{nj} \in \cdot} &= \frac{1}{n} \sum_{j=1}^n \delta_{\xi_j} \text{ for all $1 \leq j \leq n$}
\end{align*}
Let $\overline{\xi}_n = \frac{1}{n} \sum_{j=1}^n \xi_j$ and $C = \covariance{\xi}$ then almost surely
\begin{align*}
\cprobability{\xi_1, \dotsc, \xi_n}{\frac{1}{\sqrt{n}} \sum_{j=1}^n \left( \xi^B_{nj} - \overline{\xi}_n \right) } \todist N(0, C)
\end{align*}
\end{thm}
\begin{proof}
TODO: Reduce to the case $d=1$ and $\expectation{\xi} = 0$.  Dispense with the case of $\expectation{\xi^2} = 0$ (i.e. $\xi = 0$ a.s.).

Having reduced to the case in which $\expectation{\xi}=0$, $\expectation{\xi^2}=\sigma^2$ with $0 < \sigma < \infty$ we define the sample standard deviation
\begin{align*}
s^\prime_n &= \frac{1}{\sqrt{n}} \left( \sum_{j=1}^n \left( \xi_j - \overline{\xi}_n \right)^2 \right)^{1/2} = \left( \frac{1}{n} \sum_{j=1}^n \xi^2_j - \overline{\xi}^2_n \right)^{1/2}
\end{align*}
Applying the Strong Law of Large Numbers Theorem \ref{SLLN} to the $\xi_n$  and $\xi^2_n$ we know that almost surely
\begin{align*}
\lim_{n \to \infty} (s^\prime_n)^2 &=\lim_{n \to \infty} \frac{1}{n} \sum_{j=1}^n (\xi_j - \overline{\xi}_n)^2 \\
&=\lim_{n \to \infty} \frac{1}{n} \sum_{j=1}^n \xi^2_j - \left(\lim_{n \to \infty} \overline{\xi}_n\right)^2 = \sigma^2
\end{align*}

Also we have by Theorem \ref{Disintegration}
\begin{align*}
\cexpectationlong{\xi_1, \dotsc, \xi_n} { (\xi^B_{nj} - \overline{\xi}_n)} &= \frac{1}{n} \int x \sum_{j=1}^n \delta_{\xi_j}(dx) - \overline{\xi}_n \\
= \left(  \overline{\xi}_n - \overline{\xi}_n \right) = 0
\end{align*}
and by another application of Theorem \ref{Disintegration} and the pullout property of conditional expectations/
\begin{align*}
\cexpectationlong{\xi_1, \dotsc, \xi_n} { (\xi^B_{nj} - \overline{\xi}_n)^2} 
&=\cexpectationlong{\xi_1, \dotsc, \xi_n} { (\xi^B_{nj})^2} - 2 \cexpectationlong{\xi_1, \dotsc, \xi_n} {\xi^B_{nj} \overline{\xi}_n} + \cexpectationlong{\xi_1, \dotsc, \xi_n} {\overline{\xi}_n^2}  \\
&= \frac{1}{n} \int x^2 \sum_{j=1}^n \delta_{\xi_j}(dx) - 2 \overline{\xi}_n \cexpectationlong{\xi_1, \dotsc, \xi_n} {\xi^B_{nj} } +\overline{\xi}_n^2 \\
&= \frac{1}{n} \sum_{j=1}^n \xi_j^2 - \overline{\xi}_n^2 = (s^\prime_n)^2
\end{align*}

From the above computations if we define 
\begin{align*}
\eta_{nj} &= \frac{1}{s_n^\prime \sqrt{n}} (\xi^B_{nj} - \overline{\xi}_n)
\end{align*}
it follows that $\eta_{n1}, \dotsc, \eta_{nn}$ are conditionally i.i.d. given $\xi_1, \dotsc, \xi_n$ and satisfy $\cexpectationlong{\xi_1, \dotsc, \xi_n} {\eta_{nj}}=0$, 
$\sum_{j=1}^n \cexpectationlong{\xi_1, \dotsc, \xi_n} {\eta^2_{nj}}=\sum_{j=1}^n 1/n = 1$ almost surely.  Thus the regular conditional distributions $\cprobability{\xi_1, \dotsc, \xi_n}{\eta_{nj} \in \cdot}$ for $1 \leq j \leq n$ are almost surely a triangular array.  To conclude that $\cprobability{\xi_1, \dotsc, \xi_n}{\eta_{nj} \in \cdot} \todist N(0,1)$ almost surely it suffices by the Lindeberg Central Limit Theorem \ref{LindebergTheoremTriangular} to show that for every $\epsilon > 0$ we have $\lim_{n \to 0} \sum_{j=1}^n \cexpectationlong{\xi_1, \dotsc, \xi_n}{\eta^2_{nj} ; \abs{\eta_{nj}} > \epsilon} = 0$ almost surely.

As noted above $\overline{\xi}_n \toas 0$ and $s^\prime_n \toas \sigma$.  Thus if we are given $\epsilon > 0$ then almost surely there exists a random $N \in \naturals$ such that $\abs{s^\prime_n - \sigma} < \sigma/4$ and
$\overline{\xi}_n < \frac{1}{4} \sigma \epsilon \leq \frac{1}{4} \sqrt{n} \sigma \epsilon$ for all $n \geq N$.  Therefore for such $n \geq N$, if we assume $\abs{\eta_{nj}} > \epsilon$,  (equivalently $\abs{\xi^B_{nj} - \overline{\xi}_n} < \sqrt{n} s^\prime_n \epsilon$) it follows that
\begin{align*}
\abs{\xi^B_{nj}} &\geq \abs{\xi^B_{nj} - \overline{\xi}_n} - \abs{\overline{\xi}_n} > \sqrt{n} s^\prime_n \epsilon - \frac{1}{4} \sqrt{n} \sigma \epsilon \\
&\geq \sqrt{n} \sigma \epsilon - \sqrt{n} (\sigma-s^\prime_n) \epsilon - \frac{1}{4} \sqrt{n} \sigma \epsilon > \sqrt{n} \sigma \epsilon /2
\end{align*}
TODO: Finish
\end{proof}

\section{Martingales and Optional Times}
TODO:  First introduce discrete time martingales then do stopping
times and lastly extend to continuous time martingales (at least the
basics).

We first begin with a very general notion of \emph{stochastic process}
which we rather quickly specialize.  
\begin{defn}Suppose one has a measurable space $(S, \mathcal{S})$ and an
  index set $T$.  We let $S^T$ denote the set of all functions $f : T
  \to S$.  Then $\mathcal{S}^T$ is the $\sigma$-algebra generated by
  all the evaluation maps $\pi_t : S^T \to S$ defined by $\pi_t(f) =
  f(t)$.  That is to say
\begin{align*}
\mathcal{S}^T &= \sigma \left ( \lbrace \lbrace f \mid f(t) \in U
  \rbrace \mid t \in T \text{, } U \in
  \mathcal{S} \rbrace\right )
\end{align*}
\end{defn}
Measurability with respect to the $\sigma$-algebra $\mathcal{S}^T$ has a useful
alternative characterization.  First we establish some notation.  If
we consider a set function $X : \Omega \to S^T$ then can equivalently
view this as a set function $\tilde{X} : \Omega \times T \to S$ via the
identification $\tilde{X}(\omega, t) = X(\omega)(t)$ (the process of
transforming $\tilde{X}$ to $X$  is called
\emph{currying} in computer science).  We can also curry $\tilde{X}$ on $\Omega$
 to get an element $\hat{X} : T \to S^\Omega$.  It is customary to
write $\hat{X}(t)$ as $X_t$.
\begin{lem}\label{ProcessMeasurableProjections}Suppose one has a probability space $(\Omega,
  \mathcal{A})$, a measurable space $(S, \mathcal{S})$, an
  index set $T$ and a subset $U \subset S^T$.  Then $X : \Omega \to U$
  is $U \cap \mathcal{S}^T$-measurable if and only if $X_t : \Omega
  \to S$ is $\mathcal{S}$-measurable for all $t \in T$.
\end{lem}
\begin{proof}
We know by definition of $\mathcal{S}^T$ that every projection $\pi_t
: S^T \to S$ is measurable.  Moreover, we know that $X_t = \pi_t \circ
X$.  Therefore if we assume $X$ is
$\mathcal{S}^T$-measurable then $X_t$ is a composition of measurable
functions and it follows from Lemma
\ref{CompositionOfMeasurable} that $X_t$ is measurable.

In the opposite direction, assume that each $X_t$ is measurable.  Let
$A \in \mathcal{S}$ and $t \in T$ and consider the set $\pi_t^{-1}(A)
\in \mathcal{S}^T$.  By definition we can see that
\begin{align*}
X^{-1}(\pi_t^{-1}(A)) &= \lbrace \omega \in \Omega \mid
\pi_t(X(\omega)) \in A \rbrace = X_t^{-1}(A)
\end{align*}
which is measurable by assumption.  Since sets of the form
$\pi_t^{-1}(A)$ generate $\mathcal{S}^T$ application of Lemma \ref{MeasurableByGeneratingSet}
shows that $X$ is measurable.
\end{proof}

It can be useful to know what operations on set functions are
measurable with respect to the product topology on $S^T$.  Here we
record a simple fact that we will use.
\begin{lem}\label{MeasurableFunctionGroup}Let $G$ be a measurable
  group and $T$ be an index set, with group operations defined pointwise, $(G^T, \mathcal{G}^T)$ is a  measurable group.
\end{lem}
\begin{proof}
With the identity defined by the constant function $f(t) = e$, the
fact that $G^T$ is a group is immediate.  To see measurability of the
group operation, let $A \in \mathcal{G}$ and pick $t \in T$.  Note
that $(\pi_t, \pi_t) : \mathcal{G}^T \otimes \mathcal{G}^T / \mathcal{G} \otimes \mathcal{G}$ - measurable by
definition of the product $\sigma$-algebra (both on $G^T$ and
generically) and we know the group operation is $\mathcal{G} \otimes
\mathcal{G}/\mathcal{G}$-measurable therefore $\lbrace (f,g) \mid (f
\cdot g)(t) \in A \rbrace$ is $\mathcal{G}^T \otimes
\mathcal{G}^T$-measurable.  The proof for the inverse operation
follows similarly.
\end{proof}

\begin{defn}Suppose one has a probability space $(\Omega,
  \mathcal{A}, \mu)$, a measurable space $(S, \mathcal{S})$, an
  index set $T$ and a subset $U \subset S^T$.  A $U \cap
  \mathcal{S}^T$-measurable $X : \Omega \to U$ is called a
  \emph{stochastic process}.
\end{defn}

Note that we do not require $U$ to be a measurable subset of $S^T$
(and in most case that we consider it will not be).
According to this definition, a stochastic process is simply a random
element in subset of a path space $(U, U \cap \mathcal{S}^T)$.  As such it has a
distribution $\pushforward{X}{\mu}$ which is a measure on $U$; as usual we will say
that two stochastic processes $X$ and $Y$ are equal in distribution
when their laws are equal.  Because of the nature of the
$\sigma$-algebra on $\mathcal{S}^T$ there is a simple way to measure
whether two processes are equal in distribution.

TODO: Build some intuition about the definition of a process.  In
particular, the reason for considering subsets $U \subset S^T$ is
clear because $S^T$ is just too big.  It is very rare for one to be
interested in arbitrary set functions; almost always one wants some
kind of condition to be imposed such as continuity or at least some
restriction on the discontinuities that can occur (e.g. allowing jump
discontinuities but outlawing
oscillatory discontinuities is common in stochastic processes).  These
subsets very often come with additional structure that either implies or constrains their measure theoretic
structure (e.g. a metric topology that implies a Borel
$\sigma$-algebra).  A subtle point that shall come up is that one will
want the implied measure theoretic structure be compatible with the
measure theoretic structure of the general definition.
TODO: Is there something deep about the use of the product
$\sigma$-algebra in this context or is a technical convenience/least
common denominator that allow one to prove general results?  Well,
arguably it is made in this way so that a stochastic process is just a
family of random elements $X_t$ indexed by $T$; where do we even state
this fact?

\begin{lem}\label{RestrictionOfSigmaAlgebra}Let $(S, \mathcal{S})$ be a measurable space an let $U
  \subset S$ be a (not necessarily measurable) subset, then $U
  \cap \mathcal{S}$ is a $\sigma$-algebra on $U$.  Furthermore if
  $\mathcal{C} \subset 2^S$ is a set of subsets of $S$ that
  generates $\mathcal{S}$ then $\mathcal{D} = \lbrace U \cap C \mid C
  \in \mathcal{C} \rbrace$ generates $U \cap \mathcal{S}$.  Lastly
  given a measurable space $(T, \mathcal{T})$ and an $\mathcal{S}/\mathcal{T}$-measurable
  function $f : S \to T$, the restriction $f \vert_U : U \to T$ is $U
  \cap \mathcal{S}/\mathcal{T}$-measurable.
\end{lem}
\begin{proof}
The fact that $U \cap \mathcal{S}$ is a $\sigma$-algebra follows
easily from the fact that $\mathcal{S}$ is a $\sigma$-algebra and the set theoretic identities $\cap_{i=1} (U \cap A_i) = U
\cap \cap_{i=1} A_i$ and $U \setminus (U \cap A) = U \cap (U \cap A)^c
= U \cap A^c$.

Given the generating set $\mathcal{C}$ for $\mathcal{S}$ and
$\mathcal{D}$ defined as above it is immediate from the fact that
$\mathcal{C} \subset \mathcal{S}$ that we have $\mathcal{D} \subset U
\cap \mathcal{S}$.  As we have just proven that $U \cap \mathcal{S}$
is a $\sigma$-algebra it follows that $\sigma(\mathcal{D}) \subset U
\cap \mathcal{S}$.  

On the other hand, let $\mathcal{E} = \lbrace A
\subset S \mid U \cap A \in \sigma(\mathcal{D}) \rbrace$.  We
claim that $\mathcal{E}$ is a $\sigma$-algebra.  Indeed if $A, A_1, A_2,
\dotsc \in \mathcal{E}$ then we have $U \cap \cup_{i=1}^\infty A_i =
\cup_{i=1}^\infty U \cap A_i \in \sigma(\mathcal{D})$ which implies
$\cup_{i=1}^\infty A_i \in \mathcal{E}$ and $U \cap A^c
= (U \cap U^c) \cup (U \cap A^c) = U \cap (U \cap A)^c \in
\mathcal{D}$ which implies $A^c \in \mathcal{E}$.  By the definition
of $\mathcal{D}$, we know
$\mathcal{C} \subset \mathcal{E}$ and therefore $\mathcal{S} =
\sigma(\mathcal{C}) \subset \sigma(\mathcal{E}) = \mathcal{E}$.  Thus we have shown the
reverse inclusion $U \cap
\mathcal{S} \subset \sigma(\mathcal{D})$ and we have $U \cap
\mathcal{S} = \sigma(\mathcal{D})$.

Lastly, $U \cap \mathcal{S}/\mathcal{T}$-measurability of the
restriction of a
$\mathcal{S}/mathcal{T}$-measurable $f$ follows from the identity
$(f\vert_U)^{-1}(A) = \lbrace s \in S \mid f(s) \in A \text{ and } s
\in U \rbrace = U \cap f^{-1}(A)$ which shows $(f\vert_U)^{-1}(A) \in U \cap
\mathcal{S}$ whenever $f^{-1}(A) \in \mathcal{S}$.
\end{proof}

\begin{lem}\label{ProcessLawsAndFDDs}   Let $X$ be a stochastic process
  with values in $U \subset S^T$, then for every $t_1,
  \dotsc, t_n \in T$ then $(X_{t_1}, \dotsc, X_{t_n}) \in S^n$ is
  $\mathcal{S}^{\otimes n}$-measurable and the measures
  $\pushforward{(X_{t_1}, \dotsc, X_{t_n})}{\mu}$ are called the
  \emph{finite dimensional distributions} of $X$. Given $U \subset
  S^T$ then any two probability measures on $(U, U \cap
  \mathcal{S}^T)$ are equal if and only if their finite
  dimensional distributions are equal.  In particular, if $X$ and $Y$ are
  two stochastic processes with values in $U \subset S^T$ then $X \eqdist Y$ if and only if their
  finite dimensional distributions are equal (written $X \eqfdd Y$).
  It is also that the case that $X \eqfdd Y$ if and only if
  $\pushforward{(X_{t_1}, \dotsc, X_{t_n})}{\mu} =
  \pushforward{(Y_{t_1}, \dotsc, Y_{t_n})}{\mu}$ for all $t_1, \dotsc,
  t_n \in T$ with the $t_j$ distinct.
\end{lem}
\begin{proof}
Suppose that $t_1, \dotsc, t_n$ are given and define the $n$-dimensional projection
$(\pi_{t_1}, \dotsc, \pi_{t_n}) : S^T \to S^n$.   We claim that 
$(\pi_{t_1}, \dotsc, \pi_{t_n})$ is 
$\mathcal{S}^T/\mathcal{S}^{\otimes n}$ measurable.  Indeed if we let $A_1 \times \dotsb \times A_n \in
\mathcal{S}^{\otimes n}$ then $(\pi_{t_1}, \dotsc, \pi_{t_n})^{-1}(A_1
\times \dotsb \times A_n) = 
\pi_{t_1}^{-1}(A_1) \cap \dotsb \cap \pi_{t_n}^{-1}(A_n) $, hence $(\pi_{t_1}, \dotsc, \pi_{t_n})^{-1}(A_1
\times \dotsb \times A_n) \in
\mathcal{S}^T$ by the measurability of each $\pi_{t_j}^{-1}(A_j)$ for
$j=1, \dotsc, n$.  Since sets of the form $A_1 \times \dotsb \times
A_n$ generate $\mathcal{S}^{\otimes n}$ 
we see that $(\pi_{t_1}, \dotsc, \pi_{t_n})$ is measurable by
application of Lemma \ref{MeasurableByGeneratingSet}.

The $\mathcal{S}^{\otimes n}$-measurability of $(X_{t_1}, \dotsc,
X_{t_n})$ now follows directly from Lemma
\ref{CompositionOfMeasurable} and \ref{RestrictionOfSigmaAlgebra} since we can write $(X_{t_1}, \dotsc,
X_{t_n}) = (\pi_{t_1}, \dotsc, \pi_{t_n}) \circ X$ as a composition of
a $U \cap \mathcal{S}^T/\mathcal{S}^{\otimes n}$-measurable function
$(\pi_{t_1}, \dotsc, \pi_{t_n})\vert_U$ and $U \cap \mathcal{S}^T$-measurable function $X$.

Suppose that $\mu$ and $\nu$ are probability measures on $(U, U \cap
\mathcal{S}^T)$ whose finite dimensional projections are equal; that
is to say for every $t_1, \dotsc, t_n \in T$ we have 
$\pushforward{(\pi_{t_1},\dotsc, \pi_{t_n})}{\mu} =
\pushforward{(\pi_{t_1},\dotsc, \pi_{t_n})}{\nu}$.  This fact that
shows that $\mu$ and $\nu$ agree on all sets of the form
$(\pi_{t_1},\dotsc, \pi_{t_n})^{-1}(A)$ where $n > 0$, $t_1, \dotsc,
t_n \in T$ and $A \subset \mathcal{S}^{\otimes n}$.  Let the
set of sets of this form be called $\mathcal{C}$.  We claim
$\mathcal{C}$ generates $U \cap \mathcal{S}^T$.  Indeed it is the case
that sets of the form $\pi_t^{-1}(A)$ for $t \in T$ and $A \subset
\mathcal{S}$ generate $U \cap \mathcal{S}^T$.  One can see this by
observing that $\pi_t^{-1}(A) = U \cap \tilde{\pi}_t^{-1}(A)$ where
$\tilde{\pi}_t : S^T \to S$ is the evaluation map extended to the entireity of $S^T$.  By
definition $\mathcal{S}^T$ is generated by the sets
$\tilde{\pi}^{-1}(A)$ and therefore by Lemma
\ref{RestrictionOfSigmaAlgebra} we conclude $U \cap \mathcal{S}^T$ is
generated by sets of the form $U \cap \tilde{\pi}_t^{-1}(A) =
\pi_t^{-1}(A)$.

Next we claim that $\mathcal{C}$ is a $\pi$-system.  This follows
immediately as we can write $(\pi_{t_1},\dotsc, \pi_{t_n})^{-1}(A)
\cap (\pi_{s_1},\dotsc, \pi_{s_m})^{-1}(B) = (\pi_{t_1},\dotsc,
\pi_{t_n}, \pi_{s_1},\dotsc, \pi_{s_m})^{-1}(A \times B)$.  Now we may
conclude $\mu = \nu$
by a monotone class argument (specifically Lemma \ref{UniquenessOfMeasure}).

The statement about stochastic processes follows by applying the fact just proven the laws
of $X$ and $Y$.

It is trivial that if $X \eqdist Y$ then the finite dimensional
distributions with distinct $t_j$ are equal.  To see the converse note
that the projection $(\pi_{t_1},\dotsc,  \pi_{t_n})$ for not necessarily distinct $t_j$ may be written
as a composition $i \circ (\pi_{s_1}, \dotsc, \pi_{s_m})$ with $s_1,
\dotsc, s_m$ the set of distinct $t_j$ and $i : S^m \to S_n$ that
depends only on the $t_j$.  Now the result follows from functoriality
of the pushforward of a measure.
\end{proof}

The previous result shows that the finite dimensional distributions
uniquely characterize the distribution of a stochastic process.  We
now turn an associated existence problem.  Namely given a family of
distributions that are candiates to be the finite dimensional
distributions of a stochastic process, is there in fact a stochastic
process with these FDDs.  In general this is not the case and the
result requires topological assumptions.  It is sufficient to assume
that the spaces involved are Borel.

\begin{defn}Let $(S_1, \mathcal{S}_1), (S_2, \mathcal{S}_2), \dotsc$ be a sequence of measurable spaces,
  and for each $n \in \naturals, $ let  $\mu_n$ be a probability
  measure on on $S_1 \times \dotsb \times S_n$.  We say that the
  sequence of measures $\mu_1, \mu_2, \dotsc$ is \emph{projective} if
  for every $n \in \naturals$ and every $A \in \mathcal{S}_1 \otimes
  \dotsb \otimes \mathcal{S}_n$ we have $\mu_{n+1}(A \times S_{n+1}) =
  \mu_n(A)$.
\end{defn}

\begin{thm}[Daniell Theorem]\label{DaniellExtension}Let $(S_1,
  \mathcal{S}_1), (S_2, \mathcal{S}_2), \dotsc$ be a sequence of
  measurable spaces, with $S_2, S_3 , \dotsc$ Borel and let $\mu_1,
  \mu_2, \dotsc$ be a projective sequence of measures then there exist
  random elements $\xi_n$ in $S_n$ for $n \in \naturals$ such that
  $\mathcal{L}(\xi_1, \dotsc, \xi_n) = \mu_n$ for all $n \in
  \naturals$.  In particular, there exists a probability measure $\mu$
  on $S_1 \times S_2 \times \dotsb$ such that for every $n \in
  \naturals$ and $A \in \mathcal{S}_1 \otimes \dotsb \otimes
  \mathcal{S}_n$ we have $\mu(A \times S_{n+1} \times \dotsb) = \mu_n(A)$.
\end{thm}
\begin{proof}
Trivially we can create $\xi_1$ with $\mathcal{L}(\xi_1) = \mu_1$
(just take $\Omega = S_1$ and define $\xi_1$ to be the identity).  Now
by extending $\Omega$ to $S_1 \times [0,1]$ we applying Lemma
\ref{ReproductionOfUniform} we can find independent $U(0,1)$ random
variables $\vartheta_2, \vartheta_3, \dotsc$ which are also
independent of $\xi_1$.  We construct the remaining $\xi_2, \xi_3,
\dotsc$ by an induction argument using Lemma \ref{Transfer}.  

Suppose that we have constructed $\xi_1, \dotsc, \xi_n$ where for each
$m > 1$ there exists a measurable function $f_m$ such that $\xi_m =
f_m(\xi_1, \vartheta_2, \dotsc, \vartheta_m)$.  Let $\eta_1, \dotsc,
\eta_{n+1}$ we arbitrary random elements such that
$\mathcal{L}(\eta_1, \dotsc, \eta_{n+1}) = \mu_{n+1}$ (e.g. define
$\tilde{\Omega} = S_1 \times \dotsb \times S_{n+1}$ with probability
measure $\mu_{n+1}$ and define $\eta_m(s_1, \dotsc, s_{n+1}) = s_m$).
By the induction hypothesis and the projective property of the sequence $\mu_n$ we have for each $A
\in \mathcal{S}_1 \otimes \dotsb \otimes \mathcal{S}_n$
\begin{align*}
\probability{(\eta_1, \dotsc, \eta_n) \in A} &= \probability{(\eta_1,
  \dotsc, \eta_n, \eta_{n+1}) \in A \times S_{n+1}} = \mu_{n+1}(A
\times S_{n+1}) \\
&= \mu_n(A) = \probability{(\eta_1, \dotsc, \eta_n) \in A} 
\end{align*}
and therefore $(\eta_1, \dotsc, \eta_n) \eqdist (\xi_1, \dotsc,
\xi_n)$.  Now we may apply Lemma \ref{Transfer} to conclude that there
is a measurable function $g : S_1 \times \dotsb \times S_n \times
[0,1] \to S_{n+1}$ such that $\xi_{n+1} = g(\xi_1, \dotsc, \xi_n,
\vartheta_{n+1})$ satisfies 
\begin{align*}
\mathcal{L}(\xi_1, \dotsc, \xi_{n+1}) &=
\mathcal{L}(\eta_1, \dotsc, \eta_{n+1})  = \mu_{n+1}
\end{align*}
Moreover we may define 
\begin{align*}
f_{n+1}(x_1, \dotsc, x_{n+1}) &= g(x_1,
f_2(x_1, x_2), \dotsc, f_n(x_1, \dotsc, x_n), x_{n+1})
\end{align*} so that
$\xi_{n+1} = f_{n+1}(\xi_1, \vartheta_2, \dotsc, \vartheta_{n+1})$.

For the last part of the theorem, define $\mu = \mathcal{L}(\xi_1,
\xi_2, \dotsc)$.  It then follows that for every $n \in \naturals$ and
$A \in \mathcal{S}_1 \otimes \dotsb \otimes \mathcal{S}_n$ we have
\begin{align*}
\mu(A \times S_{n+1} \dotsb) &= \probability{(\xi_1, \xi_2, \dotsc)
  \in A \times S_{n+1} \dotsb} \\
&=  \probability{(\xi_1, \dotsc, \xi_n)
  \in A } = \mu_n(A) 
\end{align*}
\end{proof}

We now generalize the Daniell Theorem to arbitrary index sets $T$.
First we generalize the notion of a projective sequence of measures to
a projective family on an arbitrary index set.
\begin{defn}Let $T$ be a set and suppose we are given a measurable
  space $(S_t, \mathcal{S}_t)$ for every $t \in T$ and for every
  finite subset $I \subset T$ we are given a probability measure
  $\mu_I$ on $\times_{t \in I} S_t$.  For any subset $U \subset T$
  define $(S_U, \mathcal{S}_U) = (\times_{t \in U} S_t , \otimes{t \in
    U} \mathcal{S}_t)$.  We say that $\lbrace \mu_I
  \rbrace$ is a \emph{projective family} if for every
  finite subset $J \subset T$ and $I \subset J$ we have $\mu_J( \cdot
  \times S_{J \setminus I}) = \mu_I( \cdot)$. If in the definition
  above we replace the set of finite subsets of $T$ by the set of
  countable subsets of $T$ the we say $\mu_I$ is a \emph{countable projective family}.
\end{defn}

Before we attack the theorem we give a description of the structure of the
infinite product $\sigma$-algebra that will prove useful in the proof
of the extension theorem.
\begin{lem}\label{ProductSigmaAlgebraAsCountableCylinderSets}Let $T$ be a set and $(S_t, \mathcal{S}_t)$ be a family of
  measurable spaces then the $\sigma$-algebra $\otimes_{t \in T}
  \mathcal{S}_t$ is precisely the set of sets of the form $A \times
  S_{T \setminus U}$ where $U \subset T$ is a countable subset and $A
  \in \otimes_{t \in U} \mathcal{S}_t$.
\end{lem}
\begin{proof}
We claim that 
\begin{align*}
\mathcal{C} &= \lbrace A \times
  S_{T \setminus U} \mid U \subset T \text{ is countable and } A
  \in \otimes_{t \in U} \mathcal{S}_t \rbrace
\end{align*}
is a $\sigma$-algebra.  Obviously $\mathcal{C}$ is non-empty.  To see that $\mathcal{C}$ is closed under set
complement take $A \times S_{T \setminus U}$ with $U \subset T$
countable and $A \in \mathcal{S}_U$.  Then $A^c  \in \mathcal{S}_U$
and moreover $(A \times S_{T \setminus U})^c = A^c \times S_{T
  \setminus U} \in \mathcal{C}$.  
Given a sequence $C_1, C_2, \dotsc \in \mathcal{C}$
with $C_j = A_j \times S_{T \setminus U_j}$, again by passing to the
union $\cup_{j=1}^\infty U_j$ we may assume that the $U_j$ are all the
same countable subset of $T$ and therefore $C_j = A_j \times S_{T
  \setminus U}$ with $A_j \in \mathcal{S}_U$.  It follows that
$\cup_{j=1}^\infty A_j \in \mathcal{S}_U$ and therefore
$\cup_{j=1}^\infty C_j \in \mathcal{C}$.  Closure under countable
intersection follows by De Morgans Law.  Since it is clear that each
$\pi_t$ is $\mathcal{C}$-measurable we see that $\otimes_{t \in T}
\mathcal{S}_t \subset \mathcal{C}$.

To see the reverse conclusion fix a countable subset $U \subset T$ and
we need to show that for every $A \in \mathcal{S}_U$ we have $A \times
S_{T \setminus U} \in \otimes_{t \in T}
\mathcal{S}_t$.  We note that the set of all  $A \subset S_U$ such that $A \times
S_{T \setminus U} \in \mathcal{S}_T$ is a $\sigma$-algebra since 
it is precisely the pullback and pushforward of $\mathcal{S}_U$ under the projection
$\pi_U : S_T \to S_U$ (Lemma \ref{SigmaAlgebraPullback}).  It clearly
contains all sets of the form $B \times S_{U \setminus \lbrace t
  \rbrace}$ for $t \in U$ and $B \in \mathcal{S}_t$.  Such sets
generate the $\sigma$-algebra $\mathcal{S}_U$  and therefore we
conclude that $A \times S_{T \setminus U} \in \mathcal{S}_T$.
\end{proof}

\begin{thm}[Daniell-Kolmogorov
  Theorem]\label{DaniellKolmogorovExtension}Let $T$ be a set, $(S_t,
  \mathcal{S}_t)$ for $t \in T$ be a family of Borel sets and $\mu_I$
  be a projective family of probability measures.  There exists a
  random element $\xi_t$ in $S_t$ for all $t \in T$ such that for
  every $I \subset T$ we have $\mathcal{L}(\xi_I) = \mu_I$.
\end{thm}
\begin{proof}
Let $\overline{T}$ be the set of countable subsets of $T$.  It is
clear that the restriction of the projective family $\mu_I$ to any
subset $U \subset T$ is a projective sequence and therefore we can
apply Theorem \ref{DaniellExtension} to construct a probability
measure $\mu_U$ on $S_U$ such that for every finite subset $J \subset
U$ we have $\mu_U(\cdot \times S_{U \setminus J}) = \mu_J(\cdot)$.  

Now assume that we have a \emph{countable} subset $V \subset U$ and
consider the probability measure $\mu_U( \cdot \times S_{U \setminus
  V}$ on $S_V$.  From what we have just shown, for every finite subset
$J \subset V$ and every $A \in \mathcal{S}_J$ we have
\begin{align*}
\mu_U(A \times S_{V \setminus J} \times S_{U \setminus V}) &= \mu_U(A
\times S_{U \setminus J}) = \mu_J(A)
\end{align*}
and therefore $\mu_U(\cdot \times S_{U \setminus V})$ and $\mu_V$ have
the same finite dimensional distributions and therefore by Lemma
\ref{ProcessLawsAndFDDs} we know that $\mu_U(\cdot \times S_{U
  \setminus V}) =\mu_V$.  Thus we have extended the projective
family $\mu_I$ to a countable projective family.  Since by Lemma
\ref{ProductSigmaAlgebraAsCountableCylinderSets} we know that
$\otimes_{t \in T} \mathcal{S}_t$ is the precisely the set of
countable cylinder sets, we can define a measure as a set function on
said sets.  Pick $U \subset T$ and let $A \in \mathcal{S}_U$ then we
define $\mu$ by $\mu(A \times S_{T \setminus U} ) = \mu_U(A)$.  We first
claim that the $\mu$ is well defined.  Suppose we have countable
subset $U,V \subset T$, $A \in \mathcal{S}_U$ and $B \in
\mathcal{S}_V$ such that $A \times S_{T setminus U} = B \times S_{T
  \setminus V}$.  We can write 
\begin{align*}
A \times S_{T setminus U} &= A \times S_{V \setminus U} \times S_{T
  setminus (U \cup V)}
\intertext{and}
B \times S_{T setminus V} &= B \times S_{U \setminus V} \times S_{T
  setminus (U \cup V)}
\end{align*}
from which it follows that $A \times S_{V \setminus U}  =  B \times
S_{U \setminus V}$.  Using this equality along with projectivity we get
\begin{align*}
\mu(A \times S_{T \setminus U}) &= \mu_U(A) = \mu_{U \cup V}(A \times
S_{V \setminus U}) \\
&=\mu_{U \cup V}(B \times S_{U \setminus V}) = \mu_V(B) = \mu(B \times S_{T \setminus V}) 
\end{align*}
which shows that $\mu$ is well defined.

It is clear that $\mu(\emptyset) = \mu_I(\emptyset)$ for any $I
\subset T$ and therefore $\mu(\emptyset) = 0$.  

To see countable additivity of $\mu$ suppose we are given set $A_1
\times S_{T \setminus U_1}, A_2 \times S_{T \setminus U_2}, \dotsc$
where each $U_j \subset T$ is a countable subset and $A_j \in
\mathcal{S}_{U_j}$ for $j \in \naturals$.  If we define $U =
\cup_{j=1}^\infty U_j$ and redefine $A_j$ as $A_j \times S_{U
  \setminus U_j}$ then we may assume that the $U_j$ are all the same.
  Therefore we now have by countable additivity of $\mu_U$,
\begin{align*}
\mu(\cup_{j=1}^\infty (A_j \times S_{T\setminus U})) &=
\mu((\cup_{j=1}^\infty A_j) \times S_{T\setminus U}) \\
&=\mu_U(\cup_{j=1}^\infty A_j) = \sum_{j=1}^\infty \mu_U(A_j) =
\sum_{j=1}^\infty \mu(A_j \times S_{T \setminus U}) 
\end{align*}

Having defined $\mu$ on $(S_T, \mathcal{S}_T)$ we now define $\xi_t$
to be the projection $\xi_t : S_T \to S_t$ for each $t \in T$.  It is
clear from the definition of $\mu$ that for every finite subset of $I
\subset T$ and $A \in \mathcal{S}_I$ we have
\begin{align*}
\probability{ \xi_I \in A} &= \mu(A \times S_{T \setminus I}) = \mu_I(A)
\end{align*}
and therefore $\mathcal{L}(\xi_I) = \mu_I$.
\end{proof}

There are a great many things to be said about stochastic processes in
general, however we will wait a bit to travel that road and instead begin to
look at a special subclass of stochastic processes.

The first specialization is to assume our index set $T \subset \overline{\reals}$ (e.g. $\integers$,
$\reals$).  A good intuition here is that $T$ represents time and that
$X_t$ repsesents the dynamics of a time-varying random variable.

Remaining in the land of intuition, we know that as time progress we
learn from our experience; more things become known (or at least
knowable).  If we translate the term ``knowable'' into the term
``measurable'' we get a mathematically precise description of the
increasing flow of information with time.
\begin{defn}Suppose we have a probability space $(\Omega,
  \mathcal{A})$.  A collection of $\sigma$-algebras $\mathcal{F}_t
  \subset \mathcal{A}$ for $t
  \in T$ is called a \emph{filtration} if $\mathcal{F}_s \subset
  \mathcal{F}_t$ for all $s < t$.
\end{defn}

Given a stochastic process one can easily construct a filtration
associated with observations of said process.
\begin{defn}Given a probability space $(\Omega,  \mathcal{A})$, an
  index set $T \subset \overline{\reals}$ and a stochastic process $X
  : \Omega \to U$, the filtration \emph{generated by} $X$ is 
\begin{align*}
\mathcal{F}_t &= \sigma(\lbrace \sigma(X_s) \mid s \leq t \rbrace)
\end{align*}
\end{defn}

We then need to tie back the notion of a stochastic process with the
notion of a filtration.  In particular one wants to call out the case
in which a filtration contains enough information to be able to
measure the values of the process (i.e. contains at least as much
information as the knowledge of the values of the process itself).
\begin{defn}Given a probability space $(\Omega,  \mathcal{A})$, an
  index set $T \subset \overline{\reals}$, a
  filtration $\mathcal{F}_t$ for $t \in T$ and a stochastic process $X
  : \Omega \to U$, we say that $X$ is \emph{adapted} to $\mathcal{F}$
  if $X_t$ is $\mathcal{F}_t$-measurable for every $t \in T$.
\end{defn}

\begin{examp}$X$ is adapted to its generated filtration (and the
  generated filtration is the smallest filtration adapted to $X$).
\end{examp}

Now we are able to define the special class of stochastic processes
with which well spend some time.
\begin{defn}Given a probability space $(\Omega,  \mathcal{A})$, an
  index set $T \subset \overline{\reals}$ and a
  filtration $\mathcal{F}_t$ for $t \in T$, a stochastic process $M :
  \Omega \to \reals^T$ is called an \emph{$\mathcal{F}$-martingale} if 
\begin{itemize}
\item[(i)]$M_t$ is integrable for all $t \in T$
\item[(ii)]$M$ is adapted to $\mathcal{F}$
\item[(iii)]$\cexpectation{\mathcal{F}_s}{M_t} = M_s$ a.s. for all
  $s,t \in T$ with $s \leq t$.
\end{itemize}
If we replace the condition (iii) by the condition
$M_s \leq \cexpectation{\mathcal{F}_s}{M_t}$ a.s., then $M$ is said to
be a \emph{submartingale} and if we replace it with
$M_s \geq \cexpectation{\mathcal{F}_s}{M_t} $ a.s. then $M$ is said to be a
\emph{supermartingale}.
\end{defn}

A entire class of examples of martingales can be constructed via the
following Lemma.  
\begin{lem}\label{ClosedMartingales}Given a probability space $(\Omega,  \mathcal{A})$, an
  index set $T \subset \overline{\reals}$, a
  filtration $\mathcal{F}_t$ for $t \in T$ and a integrable random
  variable $\xi$, the process $M_t = \cexpectation{\mathcal{F}_t}{\xi}$ is
  an $\mathcal{F}$-martingale.
\end{lem}
\begin{proof}Integrability $\mathcal{F}$-adaptedness of $M_t$ follows from the definition of
  conditional expectation.  Since for $s,t \in T$ with $s \leq t$ we
  have $\mathcal{F}_s \subset \mathcal{F}_t$, the chain rule for
  conditional expectation shows 
\begin{align*}
\cexpectation{\mathcal{F}_s}{M_t} &=
\cexpectation{\mathcal{F}_s}{\cexpectation{\mathcal{F}_t}{\xi}} =
\cexpectation{\mathcal{F}_s}{\xi} = M_s
\end{align*}
\end{proof}
A martingale that can be expressed in the form given by the Lemma is
referred to as a \emph{closed} martingale.  We call the reader's attention the the fact
that the index set in the above Lemma is allowed to include $\infty$.
For in that case, we might as well assume that $\xi$ is
$\mathcal{F}_\infty$-measurable (or equivalently assume that $\xi =
M_\infty$ a.s.)  This consideration points to an
arguably less transparent defintion
definition of a closed martingale as one for which $\sup T \in T$ (see
Kallenberg for example; what we call a closed martingale he calls a
\emph{closable} martingale).  Also thinking about the case  in which
$\infty \ in T$ (or more generally when $\sup T \in T$) suggests a
relationship between a closing $\xi$ and a limit $\lim_{t \to \infty}
M_t$.  Such a relationship indeed exists and is explained in
Martingale convergence theorems that follow.

The unbiased random walk provides one of the simplest examples of a martingale.
\begin{examp}\label{RandomWalkAsMartingale}Suppose we are given a collection of independent random
  variables $\xi_1, \xi_2, \dots$ with $\expectation{\xi_n} = 0$ for
  all $n > 0$.  Define the filtration
  $\mathcal{F}_0 = \lbrace \emptyset, \Omega \rbrace$ and
  $\mathcal{F}_n = \sigma(\xi_1, \dots, \xi_n)$ for $n > 0$ and define
  the process $M_0=0$ and $M_n = \xi_1 + \cdots + \xi_n$.  Then $M_n$
  is an $\mathcal{F}$-martingale.
\end{examp}
From the point of view of gambling, if we think of each $\xi_n$ as
representing the outcome of a fair game based on a bet of one dollar,
then $M_n$ represents the wealth at time $n$ of a gambler that places a one dollar
bet on every game.  The gambling interpretation of martingales doesn't
really depend on the random walk structure of the example.  Given any
martingale we can interpret $M_n$ as the wealth at time $n$ and then
use use a telescoping sum
\begin{align*}
M_n &= M_0 + \sum_{j=1}^n (M_j - M_{j-1})
\end{align*}
to represent the wealth at time $n$ as the initial wealth $M_0$ plus
the sum of the return $M_j - M_{j-1}$ on the first $j$ bets.

The second example shows how one can make a martinagale out of the
variance of a base martinagle.
\begin{examp}Suppose we have the setup of Example
  \ref{RandomWalkAsMartingale} except that we also assume a constant variance
  $\expectation{\xi_n^2} = \sigma^2$ for all $n > 0$.  Then $M_n^2 - n
  \sigma^2$ is an $\mathcal{F}$-martingale.  Integrability and
  $\mathcal{F}$-adaptedness are immediate from our assumptions.  The
  martingale property requires a small computation
\begin{align*}
\cexpectationlong{\mathcal{F}_{n-1}}{M_n^2 - n \sigma^2} &=
\cexpectationlong{\mathcal{F}_{n-1}}{M_{n-1}^2 + 2 M_{n-1} \xi_n +
  \xi_n^2 - n \sigma^2} \\
&=M_{n-1}^2 +  2 M_{n-1} \cexpectationlong{\mathcal{F}_{n-1}}{\xi_n}
+  \cexpectationlong{\mathcal{F}_{n-1}}{ \xi_n^2} - n \sigma^2 \\
&=M_{n-1}^2 +  2 M_{n-1} \expectation{\xi_n}
+  \expectation{ \xi_n^2} - n \sigma^2 \\
&= M_{n-1}^2 - (n-1)\sigma^2
\end{align*}
\end{examp}

Returning to our gambling interpretation of martingales we discussed in Example
\ref{RandomWalkAsMartingale}, one can 
ask whether the ``unit bet'' assumption can be relaxed.  That is we
think of each increment $M_n - M_{n-1}$ as the return on a game in
which one has wagered on dollar.  It
would be very interesting indeed to know whether there is a betting
strategy that could make a fair game into an advantageous game (either
for the gambler or the house).  As manifested in our view of the world
as a wealth process and a returns process, the bet on the $n^{th}$
game is simply a multiplier $A_n$ applied to the return $M_n -
M_{n-1}$.  Thus the betting strategy is also a stochastic process.  To
model reality, there is an important
constraint on a betting strategy.  A bet on the $n^{th}$ game must be
made prior to the $n^{th}$ game being played and therefore should only
should only be able to make use of information about the outcome of
the first $n-1$ games.  Thus a betting strategy must not only bet
adapted the filtration $\mathcal{F}$ but satisfy the stonger condition
of the following definition.
\begin{defn}Given a filtration $\mathcal{F}_0 \subset \mathcal{F}_1
  \subset \cdots$, we say a process $A_n$ is \emph{$\mathcal{F}$-previsible} or
  \emph{$\mathcal{F}$-non-anticipating} if $A_n$ is $\mathcal{F}_{n-1}$-measurable.
\end{defn}
We make the assumption that a betting strategy is previsible and model
the strategy as providing the amount that a gambler will bet.  Of
interest is that we allow the gambler to ``short'' the bet (i.e. bet a
negative amount).  It turns out that under reasonable conditions
betting strategies alone cannot alter the fairness of a game.
\begin{lem}\label{DiscreteTimeMartingaleTransform}Let $M_n$ be a martingale and let $A_n$ be an
  $\mathcal{F}$-previsible process with each $A_n$ bounded and $A_0 =
  1$.  
Define the \emph{martingale transform}
  $\tilde{M}_n = \sum_{j=0}^n A_j\left(M_j - M_{j-1}\right )$ (we
  define $M_{-1} = 0$ for simplicity).  Then $\tilde{M}_n$ is a martingale.
\end{lem}
\begin{proof}
Clearly $\tilde{M}_n$ is $\mathcal{F}_n$-measurable as $M_j$ and
$A_j$ are for each $j\leq n$.  Integrability
of $\tilde{M}_n$ follows from the integrability of the $M_n$ and the
boundedness of $A_n$.  The martingale property follows from a simple
computation
\begin{align*}
\cexpectationlong{\mathcal{F}_{n-1}}{\tilde{M}_n} &= \sum_{j=0}^n
\cexpectationlong{\mathcal{F}_{n-1}}{A_j(M_j - M_{j-1})} \\
&=A_n \cexpectationlong{\mathcal{F}_{n-1}}{(M_n - M_{n-1})}  +
\sum_{j=0}^{n-1} A_j(M_j - M_{j-1})\\
&= \tilde{M}_{n-1}
\end{align*}
\end{proof}

\begin{lem}Let $M_t$ be a martingale then $\expectation{M_t}$ is
  constant in $t \in T$.
\end{lem}
\begin{proof}For $s,t \in T$ with $s < t$, by the
  martingale property and the chain rule of
  conditional expectations we have
$\expectation{M_s} = \expectation{\cexpectation{\mathcal{F}_s}{M_t}} = \expectation{M_t}$.
\end{proof}

\begin{defn}Given a set $T \subset \overline{\reals}$, we call a $T
  \cup \lbrace \sup T \rbrace$-valued random variable a \emph{random
    time}.  A random time is called an
  \emph{$\mathcal{F}$-optional time} (also called an \emph{$\mathcal{F}$-stopping time}) if and only if $\lbrace \tau \leq
  t \rbrace \in \mathcal{F}_t$ for all $t \in T$.
\end{defn}

An $\mathcal{F}$-optional time $\tau$ represents a random decision
rule of when to stop a game such that the decision  to stop at time $t$can be made based only on
information acccumulated up to and including time $t$ (i.e. without
seeing the future).  Note that we allow a random time to take the value $\sup T$ (think of
this as infinity) but the condition of being an optional time does not
place a condition on what happens at $\sup T$.

Provided with an optional time there is a $\sigma$-algebra of events
that is associated with it.
\begin{defn}Given an optional time $\tau$, we define 
\begin{align*}
F_\tau &= \lbrace
  A \in \mathcal{A} \mid A \cap \lbrace \tau \leq t \rbrace\in
    \mathcal{F}_t \text{ for all } t \in T \rbrace
\end{align*}
\end{defn}

 Note that we have not taken the generated $\sigma$-algebra in the
above definition, because of the following.
\begin{lem}\label{StoppedFiltration}Given an optional time $\tau$, $\mathcal{F}_\tau$ is a
  $\sigma$-algebra.  Furthermore, $\tau$ is $\mathcal{F}_\tau$-measurable.
\end{lem}
\begin{proof}Since $\Omega \cap \lbrace \tau \leq t \rbrace  = \lbrace
  \tau \leq t \rbrace \in \mathcal{F}_t$ by definition of optional
  time, we see that $\Omega \in \mathcal{F}_\tau$.  If we suppose that
  $A \in \mathcal{F}_\tau$ then for all $t \in T$, we apply elementary
  Boolean  algebra and $\sigma$-algebra properties of $\mathcal{F}_t$ to see $A^c \cap \lbrace
  \tau \leq t \rbrace = (A \cap \lbrace
  \tau \leq t \rbrace)^c \cap \lbrace
  \tau \leq t \rbrace \in \mathcal{F}_t$.  Lastly, given $A_1, A_2,
  \dots \in \mathcal{F}_\tau$, we have $(\cap_{n=1}^\infty A_n)
  \cap \lbrace
  \tau \leq t \rbrace = \cap_{n=1}^\infty (A_n
  \cap \lbrace
  \tau \leq t \rbrace ) \in \mathcal{F}_t$ and thus $\mathcal{F}_\tau$
  is a $\sigma$-algebra.

For every $s,t \in T$, we have 
$\lbrace \tau \leq s \rbrace \cap \lbrace \tau \leq t \rbrace =
\lbrace \tau \leq s \wedge t \rbrace \in \mathcal{F}_{s \wedge t}
\subset \mathcal{F}_t$ which shows every set $\lbrace \tau \leq s
\rbrace \in \mathcal{F}_\tau$ for $s \in T$.  Now for $s \in \reals
\setminus T$,
$\lbrace \tau \leq s \rbrace = \cup_{t \in T ; t < s} \lbrace \tau
\leq t \rbrace$; the trick is that this is an uncountable
union so we have to be a bit more careful in handling this
case.  Let $\tilde{s} = \sup \lbrace t \leq s \mid t \in T \rbrace$.  The
first thing to note is that $\lbrace \tau \leq s \rbrace = \lbrace
\tau \leq \tilde{s} \rbrace$.  The inclusion $\supset$ is obvious
since $s \geq \tilde{s}$.  To see the inclusion $\subset$ note that we
cannot have $\tilde{s} < \tau(\omega) \leq s$ since $\tau(\omega) \in
T$.  If $\tilde{s} \in T$ then we have show $\lbrace \tau \leq s
\rbrace \in \mathcal{F}_\tau$.  Lets assume that $\tilde{s} \notin T$.
By definition, we can find an increasing sequence $s_n \leq \tilde{s}$
such that $s_n \in T$ and $\lim_{n
  \to \infty} s_n = \tilde{s}$.  Now we claim that $\cup_n \lbrace \tau \leq
s_n \rbrace = \lbrace \tau \leq \tilde{s} \rbrace$.  The inclusion
$\subset$ follows since $s_n \leq \tilde{s}$.  To see the other
inclusion, suppose $\tau(\omega) \leq \tilde{s}$.  Because we have
assumed $\tilde{s} \notin T$ then in fact $\tau(\omega) < \tilde{s}$
and we can find $s_n$ such that $\tau(\omega) < s_n < \tilde{s}$
showing $\omega \in \cup_n \lbrace \tau \leq
s_n \rbrace$.  Putting the two equalities together
\begin{align*}
\lbrace \tau \leq s \rbrace &= \lbrace \tau \leq \tilde{s} \rbrace =
\cup_n \lbrace \tau \leq s_n \rbrace \in \mathcal{F}_\tau
\end{align*}
and we have shown that for all $s \in \reals$, $\lbrace \tau \leq s
\rbrace \in \mathcal{F}_\tau$.
This suffices to show $\mathcal{F}_\tau$-measurability by Lemma \ref{MeasurableByGeneratingSet}.
\end{proof}

Conceptually, one thinks of the $\sigma$-algebra $\mathcal{F}_\tau$ as being
events $A$ such that if $\tau \leq t$ then one only needs information
available at time $t$ to determine whether $A$ has occurred or not.
More suggestively one may say that $\mathcal{F}_\tau$ as being the
events that happen before $\tau$.

\begin{lem}Let $\sigma$ and $\tau$ be optional times with $\sigma \leq
  \tau$, then $\mathcal{F}_\sigma \subset \mathcal{F}_\tau$.
 \end{lem}
\begin{proof}Suppose we have an $A \in \mathcal{F}_\sigma$.  Because
  $\sigma \leq \tau$, we know that$\lbrace \tau \leq t
  \rbrace \subset \lbrace \sigma \leq t \rbrace$  for all $t \in T$.  Take a 
  $t \in T$, then $A \cap \lbrace \tau \leq t \rbrace = (A \cap
  \lbrace \sigma \leq t \rbrace) \cap \lbrace \tau \leq t \rbrace \in \mathcal{F}_t$.
\end{proof}

\begin{lem}Let $T \subset \overline{\reals}$ be a countable subset of
  the extended reals, let $\mathcal{F}_t$ be a filtration and $\tau :
  \Omega \to T$ be a random time.  Then $\tau$ is an optional time if
  and only if $\lbrace \tau = t \rbrace \in \mathcal{F}_t$ for every
  $t \in T$.
\end{lem}
\begin{proof}
Suppose that $\lbrace \tau = t \rbrace \in \mathcal{F}_t$ then we see
that 
\begin{align*}
\lbrace \tau \leq t \rbrace &= \cup_{s \leq t} \lbrace \tau = s \rbrace
\end{align*}
which is a countable union of sets $\lbrace \tau = s \rbrace \in
\mathcal{F}_s \subset \mathcal{F}_t$ hence is in $\mathcal{F}_t$.

Now if $\tau$ is $\mathcal{F}$-optional then similarly we may write
\begin{align*}
\lbrace \tau = t \rbrace &= \lbrace \tau \leq t \rbrace \cap \left(
  \cup_{s < t} \lbrace \tau \leq s \rbrace\right )^c
\end{align*}
which shows that $\lbrace \tau = t \rbrace \in \mathcal{F}_t$.
\end{proof}

If we think of an optional time as a random stopping rule for a game, then a
useful construct is the random stopping element associated with a
process and the stopping rule.  An interesting aspect of the proof is
that it shows stopped processes can be represented as martingale transforms.
\begin{lem}Let $\tau$ be an $\mathcal{F}$-optional time on a countable index set $T
  \subset \overline{\reals}$ and let $X$ be a stochastic
  process on $T$ adapted to $\mathcal{F}$. Then the random element $X_\tau$ is $\mathcal{F}_\tau$-measurable.
\end{lem}
\begin{proof}
TODO
\end{proof}

The $\sigma$-algebra maybe thought of as being constructed by patching
together the individual $\sigma$-algebras $\mathcal{F}_t$ of the
filtration; many arguments make use of this idea.  A precise statement
that allows localization of conditional expectations with respect to
$\mathcal{F}_\tau$ is given here.  The reader should translate the
following lemma into the intuitively obvious prose assertion that ``given
that $\tau = t$, an event $A$ happens before
$\tau$ if and only if $A$ happens before $t$''.
\begin{lem}\label{LocalizationOfStoppedFiltration}Given a filtration $\mathcal{F}_t$ and an
  $\mathcal{F}$-optional time $\tau$, for every $t \in T$, the $\sigma$-algebras
  $\mathcal{F}_t$ and $\mathcal{F}_\tau$ agree on the set $\lbrace
  \tau = t\rbrace$.
\end{lem}
\begin{proof}
Suppose $A \in \mathcal{F}_\tau$ and $A \subset \lbrace \tau = t
\rbrace$.  Then by definition of $\mathcal{F}_\tau$ we know that $A = A
\cap \lbrace \tau \leq t \rbrace \in \mathcal{F}_t$.  On the other
hand, if $A \in \mathcal{F}_t$ we know that for all $s \in T$,
\begin{align*}
A \cap \lbrace \tau \leq s \rbrace &= A \cap \lbrace \tau = t
\rbrace \cap\lbrace \tau \leq s \rbrace = \begin{cases}
A & \text{if $s \geq t$} \\
\emptyset & \text{if $s < t$}
\end{cases}
\in \mathcal{F}_s
\end{align*}
\end{proof}

Another useful fact is 
\begin{prop}\label{StoppedAlgebraMinOfOptionalTimes}Let $\sigma$ and $\tau$ be $\mathcal{F}$-optional times
  then $\mathcal{F}_\sigma \cap \lbrace \sigma \leq \tau \rbrace \subset
  \mathcal{F}_{\sigma \wedge \tau} = \mathcal{F}_\sigma \cap \mathcal{F}_\tau$.
\end{prop}
\begin{proof}
TODO:
\end{proof}

\subsection{Discrete Time Martingales}
For the special case of index set $T = \integers_+$, we often call a
martingale a \emph{discrete time martingale}.  Discrete martingales are well
undertsood objects and as it turns out many important results about discrete
martingales can be used to prove corresponding results for general martingales via approximation
arguments.  Thus, we will start our study of martingales by studying
discrete martingales.

The first thing to note is a simple observation that the definition
for the special case of discrete martingales can be simplified.
\begin{lem}Let $\mathcal{F}_n$ be a filtration and $M_n$ be a sequence
  of $\mathcal{F}$-adapted integrable random variables.  If
  $\cexpectation{\mathcal{F}_{n-1}}{M_n} = M_{n-1}$ for $n > 0$ then $M_n$
  is an $\mathcal{F}$-martingale.
\end{lem}
\begin{proof}We only have to show that
  $\cexpectation{\mathcal{F}_m}{M_n} = M_m$ for all $m \leq n$.
  Because we know $M_n$ is $\mathcal{F}_n$-measurable then we have
  $\cexpectation{\mathcal{F}_n}{M_n} = M_n$.  If $m < n-1$, then we
  proceed by induction
  assuming the result is true for $m+1$,
\begin{align*}
\cexpectation{\mathcal{F}_m}{M_n} &=
\cexpectation{\mathcal{F}_m}{\cexpectation{\mathcal{F}_{m+1}}{M_n}} \\
&=\cexpectation{\mathcal{F}_m}{M_{m+1}} & & \text{by induction hypothesis}\\
&= M_m & & \text{by hypothesis}
\end{align*}
\end{proof}

Furthermore in discrete time we have a simple version of a
construction of a useful class of optional times.
\begin{defn}Let $\mathcal{F}$ be a filtration on $\integers_+$ and let
  $X_n$ be an $\mathcal{F}$-adapted process with values in a meaurable
  space $(S, \mathcal{S})$.  For every $A \in \mathcal{S}$ be can
  define the \emph{hitting time} by
\begin{align*}
\tau_A &= \min \lbrace n \mid X_n \in A \rbrace
\end{align*}
where by convention we assume the minimum of the empty set is positive infinity.
\end{defn}

For the moment the only thing we want to record about hitting times is
that they are indeed optional times.  They will soon thereafter start to prove their utility.
\begin{lem}\label{HittingTimesDiscrete}A hitting time is an
  $\mathcal{F}$-optional time.
\end{lem}
\begin{proof}
Simply write for every finite $n$,
\begin{align*}
\lbrace \tau_A \leq n \rbrace &= \cup_{0 \leq m \leq n} \lbrace X_m
\in A \rbrace
\end{align*}
and note that by $\mathcal{F}$-adaptedness of $X$, we have $\lbrace X_m
\in A \rbrace \in \mathcal{F}_m \subset \mathcal{F}_n$.
\end{proof}

\begin{lem}\label{ExpectationStoppedMartingaleDiscrete}Let $M_n$ be a martingale and let $\tau$ be an optional
  time such that $\tau \leq C < \infty$, then $\expectation{M_\tau} =
  \expectation{M_0}$.
\end{lem}
\begin{proof}
\begin{align*}
\expectation{M_\tau} &= \sum_{n=0}^C \expectation{M_n ; \tau=n} \\
&=\sum_{n=0}^C \expectation{\cexpectationlong{\mathcal{F}_n}{M_C} ;
  \tau=n} \\
&=\sum_{n=0}^C \expectation{M_C ;
  \tau=n} \\
&=\expectation{M_C ;
  \cup_{n=0}^C \tau=n} = \expectation{M_C}\\
\end{align*}

Therefore the result follows from the case of a constant deterministic
time.  This latter case is just a simple induction on $n$.
\end{proof}


\begin{thm}[Optional Sampling Theorem]\label{OptionalSamplingDiscrete}Let $\sigma$ and $\tau$ be bounded
  $\mathcal{F}$-optional times and
  let $M_n$ be a martingale, then 
\begin{align*}
\cexpectationlong{\mathcal{F}_\sigma}{M_\tau} &\geq M_{\sigma \wedge
  \tau}
\text{ a.s.}
\end{align*}
TODO: The assumption that $\sigma$ is bounded can be removed (see
Kallenberg's proof for a demonstration of that).  How to fix up the
proof below or amend them?

TODO: This result assumes that we have a martingale on $\integers$;
the result holds with arbitrary countable index sets.
\end{thm}
\begin{proof}
We warn the reader that the following proof is a bit longer than many
you'll see in the literature.  It intentionally avoids any of the
tricks that make for short proofs in hopes of making a clearer
explanation for why the result is in fact true.

We first begin with a simple special case with $\sigma$ deterministic that captures the essence of
the result.  Suppose $\tau$ is $\mathcal{F}$-optional and there exist
constants $k, m$ such that $k \leq \tau \leq m$.  We need to prove
that $\cexpectationlong{\mathcal{F}_k}{M_\tau} = M_k$.  We do this by
induction on $m-k$.  For $m-k=0$, the result is trivial since in this
case $M_\tau = M_k$.  For the induction step suppose we have $k \leq
\tau \leq m$ with $m-k >0$ and note that we can use the induction
hypothesis on the stopping time $k+1 \leq \tau\vee k+1 \leq m$.  We
get
\begin{align*}
\cexpectationlong{\mathcal{F}_k}{M_\tau} &=
\cexpectationlong{\mathcal{F}_k}{M_{\tau \vee k+1}} +
\cexpectationlong{\mathcal{F}_k}{(M_k-M_{k+1})\characteristic{\tau=k}}
\\
&=\cexpectationlong{\mathcal{F}_k}{\cexpectationlong{\mathcal{F}_{k+1}}{M_{\tau
      \vee k+1}} } +
\characteristic{\tau=k}\cexpectationlong{\mathcal{F}_k}{(M_k-M_{k+1})}
\\
&=\cexpectationlong{\mathcal{F}_k}{M_{k+1}} + 0 = M_k
\end{align*}

To get the general result, we suppose that we are given $\sigma, \tau
\leq N < \infty$
and we suppose we are given $A \in \mathcal{F}_\sigma$.  Note that we
can write $A = \cup_{n=0}^N A \cap \lbrace \sigma=n\rbrace$ where
$A \cap \lbrace \sigma=n\rbrace \in \mathcal{F}_n$ for all $0 \leq n
\leq N$.
\begin{align*}
\expectation{M_\tau ; A} &= \sum_{n=0}^N \sum_{m=0}^N \expectation{M_n
  \characteristic{\tau=n} \characteristic{\sigma=m}
  \characteristic{A}} \\
&= \sum_{n=0}^N \left (
\sum_{m=n}^N \expectation{M_m
  \characteristic{\tau=m} \characteristic{\sigma=n} \characteristic{A}}+
\sum_{m=n+1}^N \expectation{M_n
  \characteristic{\tau=n} \characteristic{\sigma=m}\characteristic{A}}
\right) \\
&=\sum_{n=0}^N \expectation{(M_{\tau \vee n} - M_n
  \characteristic{\tau < n})\characteristic{\sigma=n}
  \characteristic{A}} + 
\expectation{M_n
  \characteristic{\tau=n} \characteristic{\sigma \geq
    n+1}\characteristic{A}} \\
&=\sum_{n=0}^N \expectation{M_n  \characteristic{\tau\geq n}\characteristic{\sigma=n}
  \characteristic{A}} + 
\expectation{M_n
  \characteristic{\tau=n} \characteristic{\sigma \geq
    n+1}\characteristic{A}} \\
&=\sum_{n=0}^N \expectation{M_n \characteristic{\tau \wedge \sigma =
    n}  \characteristic{A}} \\
&= \expectation{M_{\tau \wedge \sigma} ; A}
\end{align*}
and therefore by the defining property of conditional expectations we
are done.

Here is another rather direct proof that seems quite transparent and
is completely self contained.
Suppose $\sigma, \tau \leq N$.  Pick $A \in \mathcal{F}_\sigma$ and
compute
\begin{align*}
\expectation{M_\tau ; A} &= \sum_{n=0}^N \sum_{m=0}^N \expectation{M_n
  ; A \cap \lbrace \tau = n \rbrace \cap \lbrace \sigma = m\rbrace} \\
&=\sum_{n=0}^{N-1} \sum_{m=n+1}^N \expectation{M_n
  ; A \cap \lbrace \tau = n \rbrace \cap \lbrace \sigma = m\rbrace} + \sum_{n=0}^{N} \sum_{m=n}^N \expectation{M_m
  ; A \cap \lbrace \tau = m \rbrace \cap \lbrace \sigma = n\rbrace}\\
&=\sum_{n=0}^{N-1} \expectation{M_n
  ; A \cap \lbrace \tau = n \rbrace \cap \lbrace \sigma > n\rbrace} + \sum_{n=0}^{N} \sum_{m=n}^N \expectation{M_N
  ; A \cap \lbrace \tau = m \rbrace \cap \lbrace \sigma = n\rbrace}\\
&=\sum_{n=0}^{N-1} \expectation{M_n ; A \cap \lbrace \tau = n \rbrace
  \cap \lbrace \sigma > n\rbrace} + 
\sum_{n=0}^{N} \expectation{M_N  ; A \cap \lbrace \tau \geq n \rbrace \cap \lbrace \sigma = n\rbrace}\\
&=\sum_{n=0}^{N-1} \expectation{M_n ; A \cap \lbrace \tau = n \rbrace
  \cap \lbrace \sigma > n\rbrace} + 
\sum_{n=0}^{N} \expectation{M_n  ; A \cap \lbrace \tau \geq n \rbrace
  \cap \lbrace \sigma = n\rbrace}\\
&=\sum_{n=0}^{N} \expectation{M_n  ; A \cap \lbrace \tau \wedge \sigma = n \rbrace}\\
&=\expectation{M_{\tau \wedge \sigma} ; A}
\end{align*}
\end{proof}

\begin{cor}Let $M_n$ be a martingale and let $\tau$ be an optional
  time, then $M_{\tau \wedge n}$ is a martingale.
\end{cor}
\begin{proof}
This is an immediate consequence of Optional Sampling as $\tau \wedge
n$ and $n-1$ are both bounded optional times and therefore 
\begin{align*}
\cexpectationlong{\mathcal{F}_{n-1}}{M_{\tau \wedge n}} &= M_{\tau
  \wedge n \wedge (n-1)} = M_{\tau \wedge (n-1)}
\end{align*}

Note that this can also be proven by a direct computation using the
fact that $\lbrace \tau \geq n \rbrace = \lbrace \tau \leq n-1
\rbrace^c \in \mathcal{F}_{n-1}$:
\begin{align*}
\cexpectationlong{\mathcal{F}_{n-1}}{M_{\tau \wedge n}} &=
\sum_{m=0}^{n-1} \cexpectationlong{\mathcal{F}_{n-1}}{M_m
  \characteristic{\tau=m}} + \cexpectationlong{\mathcal{F}_{n-1}}{M_n
\characteristic{\tau \geq n}} \\
&= \sum_{m=0}^{n-1} M_m  \characteristic{\tau=m} + M_{n-1}
\characteristic{\tau \geq n} \\
&= \sum_{m=0}^{n-2} M_m  \characteristic{\tau=m} + M_{n-1}
\characteristic{\tau \geq n-1} = M_{\tau \wedge (n-1)}\\
\end{align*}
\end{proof}

\begin{lem}[Doob Decomposition]\label{DoobDecompositionDiscrete}Let
  $X_n$ be a submartingale, then there exists a martingale $M_n$ and
  an almost surely increasing non-negative $\mathcal{F}$-previsible process $A_n$ such that $X_n
  = X_0 + M_n + A_n$.
\end{lem}
\begin{proof}
We start with $M_0 = A_0 = 0$ and proceed to define $M_n$ by induction
for $n >0$ in the most natural way possible
\begin{align*}
M_n &= X_n - \cexpectationlong{\mathcal{F}_{n-1}}{X_n} + M_{n-1} \\
A_n &= X_n - M_n - X_0 = \cexpectationlong{\mathcal{F}_{n-1}}{X_n} - M_{n-1}
+ X_0
\end{align*}
a simple induction validating that $M_n$ is
$\mathcal{F}_n$-measurable, $A_n$ is $\mathcal{F}_{n-1}$-measurable
and $M_n$ is integrable.

The martingale property follows immediately from the definition and
the $\mathcal{F}_{n-1}$-measurability of
$\cexpectationlong{\mathcal{F}_{n-1}}{X_n}$ and $M_{n-1}$:
\begin{align*}
\cexpectationlong{\mathcal{F}_{n-1}}{M_n} &=
\cexpectationlong{\mathcal{F}_{n-1}}{X_n} -
\cexpectationlong{\mathcal{F}_{n-1}}{\cexpectationlong{\mathcal{F}_{n-1}} {X_n}} + 
\cexpectationlong{\mathcal{F}_{n-1}}{M_{n-1}} =M_{n-1}
\end{align*}

The fact that $A_n$ is increasing follows from
\begin{align*}
A_n &= \cexpectationlong{\mathcal{F}_{n-1}}{X_n} - M_{n-1} =
\cexpectationlong{\mathcal{F}_{n-1}}{X_n} - X_{n-1} + A_{n-1}
\end{align*}
so that 
\begin{align*}
A_n - A_{n-1} &= \cexpectationlong{\mathcal{F}_{n-1}}{X_n} - X_{n-1}
\geq 0 \text{ a.s.}
\end{align*}
by the submartingale property of $X_n$.  Non-negativity of $A_n$
follows from the facts that $A_n$ is increasing and $A_0 = 0$.
\end{proof}

The Doob Decomposition is generally a useful tool to transfer results
about martingales over to submartingales.  As a first illustration we
present the following optional
sampling theorem to submartingales
\begin{cor}\label{OptionalSamplingSubmartingaleDiscrete}Let $X_n$ be a submartingale and let $\sigma$ and $\tau$ be
  bounded optional times, then
  $\cexpectationlong{\mathcal{F}_\sigma}{X_\tau} \geq X_{\sigma \wedge
    \tau}$ a.s.

TODO: See comments about relaxing boundedness hypotheses.
\end{cor}
\begin{proof}
We write $X_n = M_n + A_n + X_0$ with $M_n$ a martingale and $A_n$ positive
increasing previsible.  Applying optional sampling (Theorem
\ref{OptionalSamplingDiscrete}) and the Doob Decomposition we get
\begin{align*}
\cexpectationlong{\mathcal{F}_\sigma}{X_\tau} &=
\cexpectationlong{\mathcal{F}_\sigma}{M_\tau + A_\tau + X_0} = M_{\sigma
  \wedge \tau} + \cexpectationlong{\mathcal{F}_\sigma}{A_\tau} + X_0
\end{align*}
so by a reverse application of the Doob Decomposition 
we just need to show $\cexpectationlong{\mathcal{F}_\sigma}{A_\tau}
\geq A_{\sigma \wedge \tau}$ a.s.

To see last fact first note that the monotonicity of $A_n$ and the
fact that $\sigma \wedge \tau \leq \tau$ shows us that $A_{\sigma
  \wedge \tau} \leq A_\tau$ a.s.  
Also we know that $\mathcal{F}_{\sigma \wedge
  \tau} \subset \mathcal{F}_\sigma$ and therefore the
$\mathcal{F}_{\sigma \wedge \tau}$-measurability of $A_{\sigma \wedge
  \tau}$ implies $\mathcal{F}_\sigma$-measurability.  Therefore
applying these observations and monotonicity of conditional
expectation we get
\begin{align*}
\cexpectationlong{\mathcal{F}_\sigma}{A_\tau} - A_{\sigma \wedge \tau}
&= \cexpectationlong{\mathcal{F}_\sigma}{A_\tau - A_{\sigma \wedge
    \tau}} \geq 0 \text{ a.s.}
\end{align*}
and we are done.
\end{proof}

There are other decomposition results that are of use.  While the Doob
Decomposition shows that a submartingale is bounded below by a
martingale the following shows that an $L^1$-bounded submartingale is
bounded above by a martingale as well.
\begin{lem}[Krickeberg
  Decomposition]\label{KrickebergDecompositionDiscrete}Let $X_n$ be an
  $L^1$-bounded submartingale then there exists an $L^1$-bounded martingale $M_n$ and
  a nonnegative $L^1$-bounded supermartingale $A_n$ such that $X_n = M_n - A_n$.
\end{lem}
\begin{proof}Fix an $m \geq 0$ and for every $n\geq m$ define $M_{n,m}
  = \cexpectationlong{\mathcal{F}_m}{X_n}$.  Note that by the
  submartingale property
\begin{align*}
M_{n,m} &= \cexpectationlong{\mathcal{F}_m}{X_n}  \leq
\cexpectationlong{\mathcal{F}_m}{\cexpectationlong{\mathcal{F}_n}
  {X_{n+1}}} = \cexpectationlong{\mathcal{F}_m}
  {X_{n+1}} = M_{n+1,m} \text{ a.s.}
\end{align*}
and therefore we can define $M_m = \lim_{n \to \infty} M_{n,m}$.
Furthermore we know that 
\begin{align*}
\expectation{\abs{M_{n,m}}} &\leq
\expectation{\cexpectationlong{\mathcal{F}_m}{\abs{X_n}}} 
=\expectation{\abs{X_n}} \leq \sup_n \expectation{\abs{X_n}} < \infty
\end{align*}
and therefore we can apply the Monotone Convergence Theorem to
conclude $\expectation{\abs{M_m}} < \infty$ so $M_m$ are integrable.
Clearly by definition of conditional expectation, each $M_{n,m}$ is $\mathcal{F}_m$-measurable and
therefore by Lemma \ref{LimitsOfMeasurable} we know that $M_m$ is
$\mathcal{F}_m$-measurable showing $M_m$ is $\mathcal{F}$-adapted.
Lastly applying the monotone convergence property of conditional
expectation and the tower rule 
for conditional expectation we get
\begin{align*}
\cexpectationlong{\mathcal{F}_{m}}{M_{m+1}} &= \lim_{n \to \infty}
\cexpectationlong{\mathcal{F}_{m}}{\cexpectationlong{\mathcal{F}_{m+1}}{X_n}}
= \lim_{n \to \infty}\cexpectationlong{\mathcal{F}_{m}}{X_n} = M_m
\end{align*}
which shows that $M_m$ is indeed a non-negative martingale.
$L^1$-boundedness of $M_m$ follows from the argument that showed $M_m$
was integrable.

Now define $A_n = M_n - X_n$ and note that 
\begin{align*}
A_n = \lim_{m \to \infty} \cexpectationlong{\mathcal{F}_n}{X_m} - X_n \geq
0 \text{ a.s.}
\end{align*}
by the submartingale property of $X_n$.  To see that $A_n$ is an $L^1$-bounded
supermartingale, note that integrability and $L^1$-boundedness of $A_n$ follow by the
triangle inequality and the corresponding properties of $X_n$ and $M_n$, $\mathcal{F}$-adaptedness follows from the
$\mathcal{F}$-adaptedness of $X_n$ and $M_n$ and the supermartingale
property follows using the submartingale and martingale properties of
$X_n$ and $M_n$ respectively
\begin{align*}
\cexpectationlong{\mathcal{F}_n}{A_{n+1}} &=
\cexpectationlong{\mathcal{F}_n}{M_{n+1}} -
\cexpectationlong{\mathcal{F}_n}{X_{n+1}} \leq M_n - X_n = A_n
\end{align*}
\end{proof}

\begin{lem}Let $M_n$ be an $L^1$-bounded martingale then there exist
  non-negative martingales $Y_n^+$ and $Y_n^-$ such that $M_n = Y_n^+
  - Y_n^-$ a.s. and $\norm{Y_n^\pm}_1 \leq \norm{M_n}_1$.
\end{lem}
\begin{proof}
This is a corollary of the proof of Lemma
\ref{KrickebergDecompositionDiscrete}.  If we apply that construction
to each of the submartingales $M_n^\pm$ we get that $Y_n^\pm = \lim_{m
  \to \infty} \cexpectationlong{\mathcal{F}_m}{M_m^\pm}$ defines a
pair of nonnegative martingales.  By linearity of conditional
expectation and the martingale property of $M_n$ we see that
\begin{align*}
Y_n^+ - Y_n^- = \lim_{m \to \infty}
\cexpectationlong{\mathcal{F}_n}{M_m^+ - M_m^-} = M_n^+ - M_n^- = M_n
\text{ a.s.}
\end{align*}
\end{proof}

\subsubsection{Martingale Inequalities}
Intuitively one thinks of martingales as being essentially constant
and submartingales as essentially increasing.  These intuitions can be
helpful when thinking of the types of properties that martingales
should have.  Probably the most important such property is that
boundedness of  a martingale or submartingale implies convergence (analogous
to the fact that a bounded increasing sequence in $\reals$ must converge).  

There are several fundamental inequalities that describe
these intuitions in a precise way.  The first result we prove is a maximal
inequality that can be viewed as an analogue of Kolmogorov's Maximal
Inequality (Lemma \ref{KolmogorovMaximalInequality}) for a special
class of dependent random variables.
\begin{lem}[Doob's Maximal
  Inequality]\label{DoobMaximalInequalityDiscrete}Let $X_t$ be a
  submartingale on a countable index set $T$, then for every $\lambda
  > 0$ and $t \in T$,
\begin{align*}
\lambda \probability{\sup_{s \leq t} X_s \geq \lambda} &\leq
\expectation{X_t ; \sup_{s \leq t} X_s  \geq \lambda} \leq \expectation{X_t^+}
\end{align*}
where $X_t^+ = X_t \vee 0$.
\end{lem}
\begin{proof}
First we assume that $T$ is a finite set.  By reindexing we may as
well assume that $T = \lbrace n \mid n \leq N \text{ and } n \in
\integers_+ \rbrace$ for some $N \geq 0$.  Now pick an $n \in T$.
The first thing to note is that for any submartingale $X_n$, $n\geq m$ and $A_m
\in \mathcal{F}_m$, $\expectation{X_n ; A_m} =
\expectation{\cexpectationlong{\mathcal{F}_m}{X_n} ; A_m}
\geq \expectation{X_m ; A_m}$.

Now the event $\lbrace \sup_{0 \leq k \leq n} X_k \geq \lambda
\rbrace$ can be nicely reexpressed in terms of optional times.  Define
\begin{align*}
\tau &= \min \lbrace  n \mid X_n \geq \lambda \rbrace
\end{align*}
where we assume the minimum of the empty set is positive infinity. 
Note that $\lbrace \sup_{0 \leq k \leq n} X_k \geq \lambda
\rbrace = \lbrace \tau \leq n \rbrace$.  If we consider the stopped
process $X_\tau \characteristic{\tau \leq n} = \sum_{m=0}^n X_m
\characteristic{\tau = m}$, take expectations and use the initial
observation,
$\expectation{X_\tau \characteristic{\tau \leq n}} \leq \sum_{m=0}^n
\expectation{X_n \characteristic{\tau = m}} = \expectation {X_n
  \characteristic{\tau \leq n}}$.  But on the other hand, by defintion
of $\tau$, we know that $\expectation{X_\tau \characteristic{\tau \leq
    n}} \geq \lambda \expectation{\characteristic{\tau \leq n}} =
\lambda \probability{\sup_{0 \leq k \leq n} X_k \geq \lambda}$ which
shows the first inequality.

The second inequality is true because nonnegativity of $X_n^+$ implies 
\begin{align*}
0 \leq X_n
\characteristic{\sup_{0 \leq k \leq n} X_k \geq \lambda} \leq X_n^+
\end{align*} so we can apply
monotonicity of expectation.

Now we want to extend the result to martingales on arbitrary countable
index sets $T$.  The proof above shows that the result holds for
finite subsets of $T$.  Now note that for any finite subsets $T^\prime
\subset T^{\prime\prime}$ such that $t \in T^\prime$ we have 
\begin{align*}
\lbrace \sup_{\substack{s \leq t\\
s \in T^\prime}} X_s \geq \lambda  \rbrace &\subset
\lbrace \sup_{\substack{s \leq t\\
s \in T^{\prime\prime}}} X_s \geq \lambda  \rbrace
\end{align*}
so if we write $T$ as an increasing union of finite sets $T_0 \subset
T_1 \subset \cdots$ then by continuity of measure (Lemma
\ref{ContinuityOfMeasure}) we have 
\begin{align*}
\probability{\sup_{\substack{s \leq t \\ s \in T}} X_s \geq \lambda}
&= 
\lim_{m \to \infty} \probability{\sup_{\substack{s \leq t \\
s \in T_m}}  X_s \geq \lambda} 
\end{align*}
and by the integrability of $X_t$ and the bound $\abs{X_t
  \characteristic{\sup_{\substack{s \leq t \\ s \in T}} X_s  \geq
    \lambda}} \leq \abs{X_t}$ we can apply Dominated Convergence to conclude
\begin{align*}
\expectation{X_t ; \sup_{\substack{s \leq t \\ s \in T}} X_s  \geq \lambda} &= 
\lim_{m \to \infty} \expectation{X_t ; \sup_{\substack{s \leq t \\ s
      \in T_m}} X_s  \geq \lambda}
\end{align*}
proving the result for countable $T$.
\end{proof}

A lesser known inequality is
\begin{lem}[Doob's Minimal
  Inequality]\label{DoobMinimalInequalityDiscrete}Let $X_t$ be a
  submartingale on a countable index set $T$, then for every every
  interval $[s,t] \subset T$ and $\lambda > 0$, 
\begin{align*}
\lambda \probability{\inf_{s \leq q \leq t} X_q \leq -\lambda} &\leq
\expectation{X_t^+} - \expectation{X_s}
\end{align*}
where $X_t^+ = X_t \vee 0$.
\end{lem}
\begin{proof}
We start by assuming that $T$ is finite and as in the proof of the
Maximal inequality we assume that $T = \lbrace 0, \dots, n \rbrace$.
Let $\tau = \min \lbrace k \mid X_k \leq -\lambda \rbrace$ be the
hitting time for the interval $(-\infty, -\lambda]$ and note that it
is an optional time.  Furthermore we have by this definition $\lbrace \min_{0
  \leq k \leq n} X_k \leq -\lambda \rbrace=\lbrace \tau \leq n\rbrace$.  By
Optional Sampling Theorem \ref{OptionalSamplingDiscrete} we know that 
\begin{align*}
\expectation{X_0} &\leq \expectation{X_{\tau \wedge n}}
\end{align*}
We write $X_{\tau \wedge n} = X_\tau \characteristic{\tau \leq n}
+ X_n \characteristic{\tau > n}$ and note that $ X_n \characteristic{\tau
  > n} \leq  X_n^+ \characteristic{\tau > n} \leq X_n^+$.  Putting
these facts together,
\begin{align*}
\expectation{X_0} &\leq \expectation{X_{\tau \wedge n}} \\
&=\expectation{X_{\tau} ; \tau \leq n} + \expectation{X_n ; \tau > n}
\\
&\leq -\lambda \probability{\tau \leq n} + \expectation{X_n^+} 
\end{align*}
and the result is proven.

TODO: Extend from finite to countable as in the proof of the Maximal
Inequality Lemma \ref{DoobMaximalInequalityDiscrete}.
\end{proof}
 
Having proven a tail inequality it is often a good idea to see what it
might say about expectations via Lemma \ref{TailsAndExpectations}.  In
this case, with a bit of care we get the following result of Doob that
can be interpreted as giving a bound on the extent to which a
non-negative submartingale can deviate from being increasing.
\begin{lem}[Doob's $L^p$
  Inequality]\label{DoobLpInequalityDiscrete}Let $X_t$ be a
  non-negative submartingle on a countable index set $T$, then for all
  $p > 1$ and $t \in T$,
\begin{align*}
\norm{\sup_{s \leq t} X_s}_p &\leq \frac{p}{p-1}\norm{X_t}_p
\end{align*}
\end{lem}
\begin{proof}
As with the proof of the maximal inequality we begin by assuming that
$T$ is finite and by reindexing equal to $\lbrace n \in \integers_+
\mid n \leq N\rbrace$ for some $N \geq 0$.
We begin let us start by assuming that $\expectation{(\sup_{0 \leq k
    \leq n} X_k)^p} < \infty$.  With this assumption in place we can
apply Lemma \ref{TailsAndExpectations}, the Maximal Inequality Lemma
\ref{DoobMaximalInequalityDiscrete} and Tonelli's Theorem \ref{Fubini}
to get
\begin{align*}
\expectation{(\sup_{0 \leq k \leq n} X_k)^p} &= p \int_0^\infty
\lambda^{p-1} \probability{\sup_{0 \leq k \leq n} X_k \geq \lambda} \,
d\lambda \\
&\leq p \int_0^\infty
\lambda^{p-2} \expectation{X_n ; \sup_{0 \leq k \leq n} X_k \geq \lambda} \,
d\lambda \\
&=p \expectation{X_n \int_0^\infty
\lambda^{p-2} \characteristic{\sup_{0 \leq k \leq n} X_k \geq \lambda} \,
d\lambda} \\
&=p \expectation{X_n \int_0^{\sup_{0 \leq k \leq n} X_k}
\lambda^{p-2} \, d\lambda} \\
&=\frac{p}{p-1} \expectation{X_n (\sup_{0 \leq k \leq n} X_k)^{p-1} } \\
&\leq \frac{p}{p-1} \norm{X_n}_p \expectation{(\sup_{0 \leq k \leq n}
  X_k)^{p}}^{\frac{p-1}{p}} & & \text{by H\"{o}lder's Inequality}
\end{align*}
But now, we can divide both sides by $\expectation{(\sup_{0 \leq k \leq n}
  X_k)^{p}}^{\frac{p-1}{p}}$ to get the result.

It remains to remove the assumption that $\expectation{(\sup_{0 \leq k \leq n}
  X_k)^{p}} < \infty$.  Obviously if $\norm{X_n}_p = \infty$ then the
result is trivially true so we may assume that $\norm{X_n}_p < \infty$.
Now we have for all $k\leq n$, by the
submartingale property, Jensen's Inequality (Theorem \ref{JensenConditionalExpectation}) and the tower rule for conditional expectation
\begin{align*}
\expectation{X_k^p} \leq
\expectation{\cexpectationlong{\mathcal{F}_k}{X_n}^p} \leq 
\expectation{\cexpectationlong{\mathcal{F}_k}{X_n^p}} =
\expectation{X_n^p} < \infty
\end{align*}
which shows that $\norm{X_k}_p < \infty$ for all $0\leq k \leq n$.
But this implies that $\norm{\sup_{0 \leq k \leq n} X_k}_p < \infty$
(e.g. for any $\xi, \eta \in L^p$, write $\xi \vee \eta =
\xi\characteristic{\xi > \eta} + \eta\characteristic{\xi \leq \eta}$
and induct) and so the previous calculation proves the lemma for
finite index sets.

Now to extend the result to arbitrary countable index sets $T$, simply
observe if $t \in T^\prime \subset T^{\prime \prime}$ then 
\begin{align*}
\sup_{\substack{s \leq t \\ s \in T^\prime }} X_s &\leq
\sup_{\substack{s \leq t \\ s \in T^{\prime \prime}}} X_s
\end{align*}
so we may take finite sets $T_0 \subset T_1 \subset \cdots$ such that
$t \in T_0$ and $\cup_n T_n = T$ and use Monotone Convergence to
conclude 
\begin{align*}
\expectation{\sup_{\substack{s \leq t \\ s \in T }} X_s} &= 
\lim_{n \to    \infty} \expectation{\sup_{\substack{s \leq t \\ s \in T_n }} X_s} \leq 
\frac{p}{p-1}\norm{X_t}_p
\end{align*}
\end{proof}

It is worthwhile emphasizing that the results above cover the case in
which $\infty \in T$.  

Conceptually there are two ways that a real valued sequence can fail
to converge: either the sequence escapes to infinity or the sequence
oscillates.  Our next goal is a result that puts explicit bounds on
the expected amount of oscillation in any submartingle. More
specifically, 
assume that we have fixed two reals numbers $a < b$; then
we can focus in on the oscillations between the values $a$ and $b$.
Alternatively one can measure the number of times the value of the
submartingale pass from below the lower bound $a$ to above the upper
bound $b$; each such transition is referred to as an
\emph{upcrossing}.  To describe upcrossings precisely we first define
the times at which pass below $a$ and then the time we pass above $b$.
\begin{lem}\label{UpcrossingMeasurabilityDiscrete}Let $\mathcal{F}_n$ be a filtration, $M_n$ be a $\mathcal{F}$-
  adapted process on $\integers_+$ and let $a<b$ be real numbers.  Let $\tau_0 =0$ and for each $j \geq 0$ define
  inductively
\begin{align*}
\sigma_j &= \inf \lbrace n \mid t \geq \tau_j \text{ and } M_n \leq a \rbrace \\
\tau_{j+1} &= \inf \lbrace n \mid t \geq \sigma_j \text{ and } M_n \geq b \rbrace \\
\end{align*}
then each $\tau_j$ and $\sigma_j$ is an $\mathcal{F}$-optional time
(note that we treat the infimum of the empty set to be infinity).
Furthermore if we define 
\begin{align*}
&U_a^b(n) = \sup \lbrace m \mid \tau_m \leq n \rbrace \\
&= \sup \lbrace m \mid \exists  j_1 < k_1 < \dotsb < j_m <
k_m \leq n \text{ such that } X_{j_i} \leq a \text{ and } X_{k_i} \geq b
\text{ for all } i=1, \dotsc, m \rbrace
\end{align*}
to be the number of upcrossings of $X_m$ before $n$, then each $U_a^b(n)$
is $\mathcal{F}_n$-measurable.
\end{lem}
\begin{proof}
To see that $\tau_j$ and $\sigma_j$ is an induction.  Assume that
$\tau_j$ is $\mathcal{F}$-optional for $j \leq n$.  We write
\begin{align*}
\lbrace \sigma_n = m \rbrace &= \bigcup_{k<m}
\left ( \lbrace \tau_n = k \rbrace
\cap \bigcap_{k < l < m} \lbrace X_l > a\rbrace\right) \cap \lbrace X_m \leq a\rbrace
\end{align*}
and by $\mathcal{F}$-adaptedness of $X_n$ and the fact
that $\tau_n$ is $\mathcal{F}$-optional we see that $\lbrace
\sigma_n = m \rbrace \in \mathcal{F}_m$.  In a similar way we can
express
\begin{align*}
\lbrace \tau_{n+1} = m \rbrace &= \bigcup_{k < m} \left ( \lbrace \sigma_n = k \rbrace
\cap \bigcap_{k < l < m} \lbrace X_l < b \rbrace\right) \cap \lbrace
X_m \geq b \rbrace
\end{align*}
and by $\mathcal{F}$-adaptedness of $X_n$ and the just
proven fact
that $\sigma_n$ is $\mathcal{F}$-optional we see that $\lbrace
\tau_{n+1} = m \rbrace \in \mathcal{F}_m$.

To see the $\mathcal{F}_n$-measurability of $U_a^b(n)$ we just express for $n \in \integers_+$
\begin{align*}
\lbrace U_a^b(n) = m \rbrace &= \lbrace \tau_m \leq n \rbrace \cap
\bigcap_{k >m} \lbrace \tau_k > n\rbrace
\end{align*}
and 
\begin{align*}
\lbrace U_a^b(n) = \infty \rbrace &= \cap_{m=1}^\infty \lbrace \tau_m \leq n\rbrace 
\end{align*}
both of which are $\mathcal{F}_n$-measurable because we have just shown each $\tau_m$ is an
optional time.

To see the last equality let the $\tilde{U}_a^b(n)$ be the right hand
side of the equality to be shown.  First note that if $\tau_m \leq n$ then
taking $j_i = \sigma_{i-1}$ and $k_i = \tau_i$ we get an upcrossing sequence $j_1 <
k_1 < \dotsc < j_m < k_m \leq n$; therefore $U_a^b(n) \leq
\tilde{U}_a^b(n)$.  
On the other hand, given such an upcrossing sequence we claim that
this implies $\sigma_i \leq j_i$ and $\tau_i \leq k_i$ for $i=1,\dotsc,m$ so in
particular, $\tau_m \leq n$.  This follows from an induction argument
that has two cases.  First if
$\tau_{i-1} \leq k_i$ and $X_{j_i} \leq a$ then we clearly see
$\sigma{i} \leq j_i$.  On the other hand if $\sigma_i \leq j_i$ then
we clearly see that $\tau_i \leq k_i$.  From $\tau_m \leq n$ it follows that
$\tilde{U}_a^b(n) \leq U_a^b(n)$ so the desired equality is proven.
\end{proof}

The second definition of $U_a^b(n)$ provided in the previous result
generalizes nicely to arbitrary time indexes; in particular for
countable time indexes we get a workable definition and measurability.
\begin{cor}\label{UpcrossingMeasurabilityCountable}Let $\mathcal{F}_t$ be a filtration, $M_t$ be a $\mathcal{F}$-
  adapted process with a countable time index $T$ and let $a<b$ be real numbers.  
If we define 
\begin{align*}
&U_a^b(t) \\
&= \sup \lbrace m \mid \exists  j_1 < k_1 < \dotsb < j_m <
k_m \leq t \text{ such that } X_{j_i} \leq a \text{ and } X_{k_i} \geq b
\text{ for all } i=1, \dotsc, m \rbrace
\end{align*}
to be the number of upcrossings of $X_s$ before $t$, then each $U_a^b(t)$
is $\mathcal{F}_t$-measurable.
\end{cor}
\begin{proof}
The previous result shows the $\mathcal{F}_t$-measurability for finite time indexes
$T$ that contain $t$.  Now write $T = \cup_{n=0}^\infty T_n$ where $t
\in T_0 \subset T_1 \subset \dotsb$ is a nested sequence of finite
sets.  It is easy to see that $U(t, a,b, T) = \lim_{n \to \infty} U(t,
a,b, T_n)$ and therefore the result follows from Lemma
\ref{UpcrossingMeasurabilityDiscrete} and Lemma \ref{LimitsOfMeasurable}.
\end{proof}

\begin{lem}[Doob's Upcrossing
  Inequality]\label{UpcrossingInequalityDiscrete}Let $X_t$ be a
  submartingale with a countable time index $T$ and let $U_a^b(t)$ be the number of upcrossings up to
  time $t \in T$.  Then
\begin{align*}
\expectation{U_a^b(t)} &\leq \frac{\expectation{(X_t-a)_+}}{b-a}
\end{align*}
\end{lem}
\begin{proof}
As the first reduction note that we may assume that $T$ is in fact
finite.  To see why let us temporarily change notation to make the dependence of $U_a^b(t)$ on the
time index $T$ explicit by writing $U(t, a, b, T)$.  As noted in the
proof of Corollary \ref{UpcrossingMeasurabilityCountable} if we consider a nested set of finite time indexes $t
\in T_0 \subset T_1 \subset \dotsb$ such that $T = \cup_{n=0}^\infty
T_n$ then in fact $\lim_{n \to \infty} U(t, a, b, T_n) = U(t,
a,b,T)$.  Now by Monotone Convergence we get $\lim_{n \to \infty}
\expectation{U(t, a, b, T_n)} = \expectation{ U(t,a,b,T)}$ and the
result for $T$ will follow from the result for finite time indexes.

The second step of the proof is a reduction to a notationally simpler
case.  As the function $f(x) = (x -a)_+$ is convex and nondecreasing we know
that $(X_t - a)_+$ is a positive submartingale.  Furthermore $X_t \geq b$ if
and only if $(X_t - a)_+ \geq b-a$ and $X_t \geq a$ if and only if
$(X_t -a)_+ = 0$ and therefore the number of upcrossings of $X_t$
between $a$ and $b$ is the same as the number of upcrossings of $(X_t
- a)_+$ between $0$ and $b-a$.  Therefore the result is proven if we
show that for every positive submartingale $X_t$ and $b>0$ we have 
\begin{align*}
U_0^b (t) &\leq \frac{\expectation{X_t}}{b}
\end{align*}

To finish the proof, let $n$ be the cardinality of $T$ so that we know
$\sigma_n = \tau_n = \infty$ and we can write the finite telescoping sum
\begin{align*}
X_t &= X_{\tau_0 \wedge t} + \sum_{j=0}^n \left ( X_{\sigma_j \wedge t} - X_{\tau_j
    \wedge t} \right) + \sum_{j=0}^n \left ( X_{\tau_{j+1} \wedge t} - X_{\sigma_j
    \wedge t} \right)
\end{align*}
Taking expectations we note that from the positivity of $X_t$ we have
$\expectation{X_{\tau_0 \wedge t}}\geq 0$ and because $\sigma_j \geq
\tau_j$ and the optional sampling theorem for submartingles
(Corollary \ref{OptionalSamplingSubmartingaleDiscrete}) we have
\begin{align*}
\expectation{X_{\sigma_j \wedge t} - X_{\tau_j    \wedge t} } &=
\expectation{\cexpectationlong{\mathcal{F}_{\tau_j \wedge t}}{X_{\sigma_j \wedge t}
    - X_{\tau_j    \wedge t}}} 
\geq \expectation{X_{\tau_j \wedge t}
    - X_{\tau_j    \wedge t}} = 0\\
\end{align*}
and we also have
\begin{align*}
X_{\tau_{j+1} \wedge t} - X_{\sigma_j \wedge t} &\geq b && \text{ if  $\tau_{j+1} \leq n$} \\
X_{\tau_{j+1} \wedge t} - X_{\sigma_j \wedge t} &=X_n \geq 0 && \text{ if  $\sigma_j \leq n < \tau_{j+1}$} \\
X_{\tau_{j+1} \wedge t} - X_{\sigma_j \wedge t} &= 0 && \text{ if $n < \sigma_j$} \\
\end{align*}
so by considering only the terms in sum for which $\tau_{j+1} \leq n$
we get $\sum_{j=0}^n \left( X_{\tau_{j+1} \wedge t} - X_{\sigma_j
    \wedge t} \right ) \geq b U_0^b(n)$.   
Putting this all together
\begin{align*}
\expectation{X_n} &= \expectation{X_{\tau_0 \wedge t}} + 
\sum_{j=0}^n \expectation{ X_{\sigma_j \wedge t} - X_{\tau_j \wedge t}} + 
\sum_{j=0}^n \expectation{X_{\tau_{j+1} \wedge t} - X_{\sigma_j \wedge t}} \\
&\geq \sum_{j=0}^n \expectation{X_{\tau_{j+1} \wedge t} - X_{\sigma_j \wedge t}} \\
&\geq b \expectation{U_0^b(n)}
\end{align*}
and therefore the result is proved.
\end{proof}

TODO: Add comments about the result that $\expectation{X_{\sigma_j
    \wedge n} - X_{\tau_j    \wedge n} }\geq 0$.  Given the definition
of $\sigma_j$ and $\tau_j$ this result might seem a bit
counterintuitive since one is expecting $X_{\sigma_j} \leq a < b \leq X_{\tau_j}$.  The explanation for how this result can hold is
that in fact is very unlikely that $\sigma_j < n$; with high
probability $\sigma_j \geq n$ and moreover $X_{\sigma_j    \wedge n} = X_n \geq
X_{\tau_j \wedge n}$ and not $X_{\sigma_j    \wedge n} = X_{\sigma_j}
\leq a$.  This explanation is completely consistent with
the conceptual model that submartingales are not oscillating much and
is really one of the two main points of the result (the other main point
being the fact that a lower bound for the terms
$\expectation{X_{\tau_{j+1} \wedge n} - X_{\sigma_j   \wedge n}}$ is given by $(b-a) U_a^b(n)$).

The Upcrossing Lemma leads immediately to a proof that $L^1$-bounded
submartingales converge almost surely.  This result is usually stated
for discrete submartingales $X_n$ but with a little attention to details
we get a stronger result that applies over countable time indexes
(e.g. $\rationals_+$) and paves the way for consideration of
continuous time indexes such as $\reals_+$.
\begin{thm}[$L^1$ Submartingale Convergence
  Theorem]\label{MartingaleConvergenceBoundedL1Discrete}Let $X_t$ be a
  $\mathcal{F}$-submartingale with a countable time index $T$ such that $\sup_{t \in
    T} \norm{X_t}_1 <
  \infty$ then there exists an $A \in \mathcal{F}_\infty$ with
  $\probability{A} = 1$ such that for every increasing or
  decreasing sequence $t_n$ in $T$ there exists an integrable random variable $X$ such
  that $X_{t_n} \to X$ on $A$.
\end{thm}
\begin{proof}
The first order of business here is leverage the Doob Upcrossing
Inequality to show that $X_t$ is not oscillatory almost surely and therefore has a limit (possibly infinite) almost
surely.  To do that for every $a \in \reals$, we note the elementary inequality $(x - a)_+ \leq
\abs{x} + \abs{a}$ and therefore we can that $\expectation{(X_t -
  a)_+} \leq \sup_{t \in T} \norm{X_t}_1 + \abs{a}< \infty$.
Supposing $a,b \in \reals$ with $a<b$ and $U_a^b(t)$ be the number of upcrossings of $[a,b]$ before $t$,
we can see that $U_a^b(t)$ is positive and increasing in $t$ and Lemma
\ref{UpcrossingInequalityDiscrete} and Monotone Convergence tell us
that if we pick any sequence $t_1, t_2, \dotsc$ such that $\lim_{n \to
  \infty} t_n = \sup T$ then
\begin{align*}
\expectation{\lim_{n \to \infty} U_a^b(t_n)} &= \lim_{n \to \infty}
\expectation{U_a^b(t_n)} \leq 
\lim_{n \to \infty} \frac{\norm{X_{t_n}}_1 + \abs{a}}{b-a} \leq 
\frac{\sup_{t \in T} \norm{X_t}_1 + \abs{a}}{b-a} < \infty
\end{align*}
If we let $U_a^b(\infty) = \lim_{n \to \infty} U_a^b(t_n) = \sup_{t
  \in T} U_a^b(t)$ be the number
of upcrossing on $T$, then $U_a^b(\infty)$ is
$\mathcal{F}_\infty$-measurable by Lemma \ref{LimitsOfMeasurable}, 
$U_a^b(\infty)$ is integrable and therefore almost
surely finite.

Let $A = \cap_{\substack{a < b\\ a,b \in \rationals}} \lbrace
U_a^b(\infty) < \infty \rbrace$ which is a countable intersection of
$\mathcal{F}_\infty$-measurable sets of probability one hence is a
$\mathcal{F}_\infty$-measurable set of probability one.  Let $t_n$ be any
increasing or decreasing sequence in $T$.  For each $a,b \in \rationals$ with $a<b$ define
\begin{align*}
\Lambda_a^b &=
\lbrace \liminf_{n \to \infty} X_{t_n} < a < b < \limsup_{n \to
  \infty} X_{t_n} \rbrace
\end{align*}
and note that $\Lambda_a^b \subset \lbrace U_a^b(\infty)
= \infty \rbrace$ (we can pick subsequences $N$ and $M$ such that
$X_{t_n}$ converges to $\liminf_{n \to \infty} X_{t_n}$ along $N$ and 
$\limsup_{n \to  \infty} X_{t_n}$ along $M$ and in this way construct an
infinite number of upcrossings of $[a,b]$; it is here that we require
that the sequence $t_n$ is increasing or decreasing).  Thus
\begin{align*}
\lbrace \liminf_{n \to \infty} X_{t_n} < \limsup_{n \to  \infty}
X_{t_n} \rbrace
&= \bigcup_{
\substack{
a,b \in \rationals \\
a<b
}} \Lambda_a^b \\
&\subset \bigcup_{
\substack{
a,b \in \rationals \\
a<b
}} \lbrace U_a^b(\infty) = \infty \rbrace  \\
&=A^c
\end{align*}
and therefore $\lim_{n \to \infty} X_{t_n}$ exists on the set $A$
(in particular almost surely since $\probability {A^c} = 0$).

Let $X = \lim_{n \to \infty} X_{t_n}$  on $A$ and for concreteness
define it to be $0$ on $A^c$.  Our last task is to show that
$X$ is integrable (hence almost surely finite as well).  This follows
from Fatou's Lemma
\begin{align*}
\expectation{\abs{X}} &\leq \liminf_{n \to \infty}
\expectation{\abs{X_{t_n} }} \leq \sup_{t \in T} \norm{X_t}_1 < \infty
\end{align*}
and we are done.
\end{proof} 
Note that despite the fact that the limit of the submartingale is
integrable in the above theorem, it is not necessarily the case that the
convergence is $L^1$.
TODO: Provide example of a non-uniformly integrable martingale with
almost sure but not $L^1$ convergence.

In the martingale case we can characterize the conditions under which
the convergence to a limit is in $L^1$.  Furthermore in this case, the
martingale is closed (see Lemma \ref{ClosedMartingales} for the
definition of closed martingales).
\begin{thm}[Martingale Closure Theorem]\label{L1MartingaleConvergenceTheoremDiscrete}Let $X_n$ be a martingale then the following are equivalent
\begin{itemize}
\item[(i)]$X_n$ is uniformly
  integrable
\item[(ii)]there exists an integrable $X$  such that
  $X_n \tolp{1} X$
\item[(iii)]there exists an integrable $X$ such that
  $X_n = \cexpectationlong{\mathcal{F}_n}{X}$ almost surely.
\end{itemize}
\end{thm}
\begin{proof}
To see (i) implies (ii) we know from Lemma
\ref{UniformIntegrabilityProperties} that $X_n$ uniformly integrable
implies $L^1$ boundedness, hence we can apply Theorem
\ref{MartingaleConvergenceBoundedL1Discrete}
to conclude the existence of an integrable $X$ such that $X_n \toas
X$.  However almost sure convergence implies convergence in
probability (Lemma \ref{ConvergenceAlmostSureImpliesInProbability})
which together with uniform integrability implies $X_n \tolp{1} X$
(Lemma \ref{LpConvergenceUniformIntegrability}).

To that (ii) implies (iii) suppose that $\epsilon > 0$ is given and
let $N>0$ be such that $\norm{X_n - X}_1 = \expectation{\abs{X_n - X}}
< \epsilon$ for 
all $n \geq N$.  Pick an $m \in \integers_+$, $n \geq N \vee m$ and let $A \in
\mathcal{F}_m$.  We calculate
\begin{align*}
\abs{\expectation{X ; A} - \expectation{X_m ; A} } &=
\abs{\expectation{X ; A} - \expectation{X_n ; A} } & & \text{since
  $\cexpectationlong{\mathcal{F}_m}{X_n} = X_m$}\\
&\leq \expectation{ \abs{X - X_n} ; A} \\
&\leq \expectation{ \abs{X -   X_n}} < \epsilon
\end{align*}
and since $\epsilon$ is arbitrary, we conclude $\expectation{X ; A} =
\expectation{X_m ; A}$ and therefore
$\cexpectationlong{\mathcal{F}_m}{X} = X_m$ a.s.

To see that (ii) implies (iii), we use Lemma
\ref{UniformIntegrabilityProperties}.  First note that by contraction
property of conditional expectation, we have $\sup_n
\expectation{\abs{\cexpectationlong{\mathcal{F}_n}{X}}} \leq
\expectation{\abs{X}}$ so the first condition of the lemma holds.  To
see the second condition, let $\epsilon > 0$ be fixed and pick $R > 0$
such that $\expectation{\abs{X} ; \abs{X} > R} < \frac{\epsilon}{2}$
and pick $A$
such that $\probability{A} < \frac{\epsilon}{2R}$.  Now, for every $n$,
\begin{align*}
\abs{\expectation{\cexpectationlong{\mathcal{F}_n}{X} ; A}} &\leq
\expectation{\cexpectationlong{\mathcal{F}_n}{\abs{X}} ; A} \\
&=\expectation{\abs{X} \cdot \cexpectationlong{\mathcal{F}_n}{
    \characteristic{A}}} \\
&=\expectation{\abs{X} \cdot \cexpectationlong{\mathcal{F}_n}{
    \characteristic{A}} ; \abs{X} \leq R} + 
\expectation{\abs{X} \cdot \cexpectationlong{\mathcal{F}_n}{
    \characteristic{A}} ; \abs{X} > R} \\
&\leq R \expectation{\cexpectationlong{\mathcal{F}_n}{ \characteristic{A}}} + 
\expectation{\abs{X} ; \abs{X} > R} \\
&\leq \epsilon
\end{align*}
and therefore we have condition (ii) of Lemma
\ref{UniformIntegrabilityProperties} satisfied and uniform
integrability is shown.
\end{proof}

It should be noted that the proof of (iii) implies (i) in previous
argument did not depend on the fact that we were dealing with a
filtration; in fact we have following corollary to the proof.
\begin{cor}\label{ConditionalExpectationsUniformlyIntegrable}Suppose $\xi$ is an integrable random variable the
  collection of random variables $\cexpectationlong{\mathcal{F}}{\xi}$
  for all $\sigma$-algebras $\mathcal{F}$ is uniformly integrable.
\end{cor}
\begin{proof}For any $\mathcal{F}$ just replay the argument that (iii) implies (i) in the
  previous result.

Just for grins here is the proof that Kallenberg gives that is very
similar up to a point to the proof in the previous result but instead of
using the uniform integrability of $\xi$ to make the elementary
argument invokes some standard workhorse theorems.  The resulting
argument seems to me to be more difficult to understand.  Maybe there
is a problem with my argument but I don't see it.  He says
just as we do that for any $\mathcal{F}$,
\begin{align*}
\abs{\expectation{\cexpectationlong{\mathcal{F}}{\xi} ; A}} &\leq
\expectation{\cexpectationlong{\mathcal{F}}{\abs{\xi}} ; A} =\expectation{\abs{\xi} \cdot \cexpectationlong{\mathcal{F}}{\characteristic{A}}} \\
\end{align*}
Now he observes that if the right hand side doesn't converge to zero
uniformly in $\mathcal{F}$ as $\probability{A} \to 0$ then there
exists an $\epsilon > 0$, $\sigma$-algebras $\mathcal{F}_n$ and $A_n$
with $\lim_{n \to \infty} \probability{A_n} = 0$ such that 
\begin{align*}
\expectation{\abs{\xi} \cdot
  \cexpectationlong{\mathcal{F}_n}{\characteristic{A_n}}} \geq
\epsilon \text{ for all $n$}
\end{align*}
so in particular no subsequence can converge to zero.  Now we derive a
contradiction. We know that
$\expectation{\cexpectationlong{\mathcal{F}_n}{\characteristic{A_n}}}
= \probability{A_n} \to 0$ and therefore
$\cexpectationlong{\mathcal{F}_n}{\characteristic{A_n}} \toprob 0$
(Lemma \ref{ConvergenceInMeanImpliesInProbability}) and
$\cexpectationlong{\mathcal{F}_n}{\characteristic{A_n}} \toas 0$ along
some subsequence $N$ (Lemma
\ref{ConvergenceInProbabilityAlmostSureSubsequence}).  Now since $\xi$
is integrable it is almost surely finite and therefore
$\abs{\xi} \cexpectationlong{\mathcal{F}_n}{\characteristic{A_n}} \toas 0$ along
the subsequence $N$ and by
Dominated Convergence we get $\expectation{\abs{\xi} \cdot
  \cexpectationlong{\mathcal{F}_n}{\characteristic{A_n}}} \to 0$
along $N$ which is a contradiction.
\end{proof}

Convergence of martingales in $L^p$ spaces with $p > 1$ is equivalent
to boundedness.  An even stronger condition holds, if a martingale
converges in $L^1$ to a $p$-integrable limit then the convergence can
be upgraded to $L^p$ convergence.
\begin{thm}[$L^p$ Martingale
  Convergence]\label{LpMartingaleConvergenceDiscrete}Given a
  martingale $M_n$, then for $p > 1$, there exists an $M \in L^p$ such that $M_n
  \tolp{p} M$ if and only if $M_n$ is $L^p$ bounded.  In fact, if $M_n
  \tolp{1} M$ with $M \in L^p$ then $M_n \tolp{p} M$ and $M_n$
  is $L^p$ bounded.
\end{thm}
\begin{proof}
Suppose $M_n$ is an $L^p$ bounded martingles.  By $L^p$ boundedness,
we know that $M_n$ is uniformly integrable thus by Theorem
\ref{L1MartingaleConvergenceTheoremDiscrete}
we know there is an integrable $M$ such that $M_n \toas M$ (thus $\abs{M_n}^p \toas \abs{M}^p$) and $M_n
\tolp{1} M$.  By Doob's $L^p$ inequality, for every $n$ we have
\begin{align*}
\norm{\sup_{0 \leq k \leq n} \abs{M_k}}_p &\leq \frac{p}{p-1} \norm{M_n}_p <
\frac{p}{p-1} \sup_n \norm{M_n}_p < \infty
\end{align*}
therefore by Monotone
Convergence we have $\norm{\sup_{0 \leq k \leq \infty} \abs{M_k}}_p = \lim_{n
  \to \infty}\norm{\sup_{0 \leq k \leq n} \abs{M_k}}_p < \infty$.  Now we
clearly have $\abs{M_n}^p \leq (\sup_{0 \leq k \leq \infty} \abs{M_k})^p$ and
Dominated Convergence gives us $M_n \tolp{p} M$.

Now assume that $M_n \tolp{1} M$ with $M \in L^p$ and $p > 1$. Theorem \ref{L1MartingaleConvergenceTheoremDiscrete}
implies that $M_n = \cexpectationlong{\mathcal{F}_n}{M}$ a.s. for
every $n$.  Now convexity of $x^p$ for $p > 1$ and  Jensen's
Inequality (Theorem \ref{JensenConditionalExpectation}) imply
\begin{align*}
\expectation{\abs{M_n}^p} &=
\expectation{\abs{\cexpectationlong{\mathcal{F}_n}{M}}^p} \leq 
\expectation{\cexpectationlong{\mathcal{F}_n}{\abs{M}^p}} =
\norm{M}_p^p < \infty
\end{align*}
which shows that not only is $M_n$ $p$-integrable but that the
martingale $M_n$ is
$L^p$-bounded.  The first part of the Theorem shows that $M_n \tolp{p} M$.
\end{proof}

Martingale convergence also allows us to extend the optional sampling
theorem to unbounded optional times.
\begin{lem}Let $M_n$ be a uniformly integrable martingale and let
  $\sigma$ and $\tau$ be optional times, then $M_\tau$ is integrable
  and $\cexpectationlong{\mathcal{F}_\sigma}{M_\tau} = M_{\sigma
    \wedge \tau}$.
\end{lem}
\begin{proof}
To see integrability of $M_\tau$ we use the Martingale Convergence
Theorem \ref{L1MartingaleConvergenceTheoremDiscrete} to conclude that
there exists integrable $M_\infty$ such that $M_n =
\cexpectationlong{\mathcal{F}_n}{M_\infty}$.  By Lemma
\ref{ConditionalExpectationIsLocal} and Lemma
\ref{LocalizationOfStoppedFiltration} for every $n$ we can compute
\begin{align*}
M_\tau &= M_n = \cexpectationlong{\mathcal{F}_n}{M_\infty} =
\cexpectationlong{\mathcal{F}_\tau}{M_\infty} \text{ on
  $\lbrace \tau = n \rbrace$}
\end{align*}
and therefore $M_\tau = \cexpectationlong{\mathcal{F}_\tau}{M_\infty}
$ proving integrability.  Note that this was proven for arbitrary
optional times so in particular $M_{\tau \wedge \sigma}$ is integrable
as well.

To show the optional sampling equality we first observe by the result
in the bounded case that for every $n$, $\cexpectationlong{\mathcal{F}_\sigma}{M_{\tau
    \wedge n}} = M_{\sigma
    \wedge \tau \wedge n}$ and we just need to justify taking limits
  in the equality.  Pick $A \in
  \mathcal{F}_\sigma$.  We know that $M_n \toas M_\infty$ as well and therefore 
we have $M_{\tau
  \wedge n} \characteristic{A} \toas M_\tau \characteristic{A}$ and $M_{\tau
  \wedge \sigma \wedge n} \characteristic{A} \toas M_{\tau \wedge \sigma} \characteristic{A}$.  To
show 
\begin{align*}
\expectation{M_\tau \characteristic{A}} 
&= \lim_{n\to \infty} \expectation{M_{\tau
  \wedge n} \characteristic{A}} = \lim_{n\to \infty} \expectation{M_{\tau
  \wedge \sigma \wedge n} \characteristic{A}} 
= \expectation{M_{\tau \wedge \sigma} \characteristic{A}}
\end{align*} 
it will suffice to show that $M_{\tau \wedge n}$
is uniformly integrable for an arbitrary optional time $\tau$.  Suppose $\epsilon > 0$ is given.  By the
integrability of $M_\tau$ we can find $R_1 > 0$ such that
$\expectation{\abs{M_\tau}; \abs{M_\tau} > R_1} < \epsilon/2$ and by
uniform integrability of $M_n$ we can find $R_2 > 0$ such that $\sup_n
\expectation{\abs{M_n} ; \abs{M_n} > R_2} < \epsilon/2$.  Now let $R =
R_1 \vee R_2$ and compute
\begin{align*}
\sup_n \expectation{\abs{M_{\tau \wedge n}} ; \abs{M_{\tau \wedge n}}
  > R} &= \sup_n \expectation{\abs{M_{\tau \wedge n}} ; \abs{M_{\tau \wedge n}}
  > R \text{ and } \tau \leq n} + \\
&\sup_n \expectation{\abs{M_{\tau \wedge n}} ; \abs{M_{\tau \wedge n}}
  > R \text{ and } \tau > n} \\
&\leq \expectation{\abs{M_{\tau}} ; \abs{M_{\tau}}
  > R } + \sup_n \expectation{\abs{M_{n}} ; \abs{M_{n}}
  > R } \\
&< \epsilon
\end{align*}
\end{proof}

\begin{cor}Let $M_n$ be a uniformly integrable martingale, then the
  set of random variables $\lbrace M_\tau \mid \tau \text{ is an optional time}\rbrace$ is
  uniformly integrable.
\end{cor}
\begin{proof}
By uniform integrability there is $M_\infty$ such that $M_n \to
M_\infty$ a.s. and in $L^1$.  By the previous result we have $M_\tau =
\cexpectationlong{\mathcal{F}_\tau}{M_\infty}$ and therefore the
result follows from Corollary \ref{ConditionalExpectationsUniformlyIntegrable}.
\end{proof}

We now give a result that we'll use in the transition to continuous
time.
\begin{thm}\label{JessenConditioningLimits}Let $\xi$ be an integrable
  random variable and let $\mathcal{F}_0 \subset \mathcal{F}_1 \subset
  \cdots$ be filtration, then $\cexpectationlong{\mathcal{F}_n}{\xi}$
  converges to $\cexpectationlong{\bigvee_n \mathcal{F}_n}{\xi}$ both
  almost surely and in $L^1$.  If $\dotsb \subset \mathcal{F}_{-1}
  \subset \mathcal{F}_0$ is a filtration then as $n \to -\infty$,
  $\cexpectationlong{\mathcal{F}_n}{\xi}$ converges to $\cexpectationlong{\bigcap_n \mathcal{F}_n}{\xi}$
both  almost surely and in $L^1$.
\end{thm}
\begin{proof}
First we take the unbounded above case.  We know from the tower property of conditional expectation and
Corollary \ref{ConditionalExpectationsUniformlyIntegrable} that $\cexpectationlong{\mathcal{F}_n}{\xi}$ is a
uniformly integrable martingle and is closable and converges both
almost surely and in $L^1$.  Let $M$ be the limit and we need to show
that $\cexpectationlong{\bigvee_n \mathcal{F}_n}{\xi} = M$ almost
surely.  We know that $M$ is $\bigvee_n \mathcal{F}_n$-measurable since it is an
almost sure limit of $M_n$ each of which is $\bigvee_n
\mathcal{F}_n$-measurable.  Furthermore by Theorem
\ref{L1MartingaleConvergenceTheoremDiscrete} we also know that
$\cexpectationlong{\mathcal{F}_n}{M} =
\cexpectationlong{\mathcal{F}_n}{\xi}$ almost surely.  Now suppose
that we have $A \in \mathcal{F}_n$ for some $n$.  We have
\begin{align*}
\expectation{M ; A} &=
\expectation{\cexpectationlong{\mathcal{F}_n}{M} ; A} \\
&=\expectation{\cexpectationlong{\mathcal{F}_n}{\xi} ; A} \\
&=\expectation{\cexpectationlong{\mathcal{F}_n}{\cexpectationlong{\bigvee_n
      \mathcal{F}_n}{\xi}} ; A} \\
&=\expectation{\cexpectationlong{\bigvee_n
      \mathcal{F}_n}{\xi} ; A} \\
\end{align*}
thus $\expectation{M ; A}  = \expectation{\cexpectationlong{\bigvee_n
    \mathcal{F}_n}{\xi} ; A}$ for all $A$ belonging to the
$\pi$-system $\cup_n \mathcal{F}_n$.  By a monotone class argument
(Lemma \ref{ConditionalExpectationExtension}) we conclude that $M = \cexpectationlong{\bigvee_n
    \mathcal{F}_n}{\xi}$ almost surely.

Now we treat the unbounded below case.  As before we know that
$M_n = \cexpectationlong{\mathcal{F}_n}{\xi}$ is a uniformly integrable
martingale.  By Theorem \ref{L1MartingaleConvergenceTheoremDiscrete}
we know that there is an integrable $M_{-\infty}$ such that $\lim_{n \to -\infty}
M_n = M_{-\infty}$ a.s. and by uniform
integrability and Lemma \ref{LpConvergenceUniformIntegrability} we
know that the convergence is also in $L^1$.  We need to show that $M_{-\infty} =
\cexpectationlong{\bigcap_n \mathcal{F}_n}{\xi}$ a.s.  The first step
is to observe that since $\mathcal{F}_n$ is a filtration $\bigcap_n
\mathcal{F}_n$ is the tail $\sigma$-algebra and therefore $M_{-\infty}$ is
$\bigcap_n \mathcal{F}_n$-measurable.  If we let $A \in \bigcap_n
\mathcal{F}_n$ then for all $n \leq 0$ we have $\expectation{\xi ; A}
= \expectation{M_n;A}$.  Since $M_n$ is uniformly integrable it
follows that $M_n \characteristic{A}$ is uniformly integrable as well and therefore can conclude
that $\expectation{\xi ; A} = \lim_{n \to -\infty} \expectation{M_n;A}
= \expectation{M_{-\infty};A}$.  The result is proven.
\end{proof}

TODO: This result can be proven directly without appealing to the martingle
convergence theorems (Stroock does this).  Is there any point in doing
so here?  Should we move this result further down and put it in the
context of the discussion of approximating continuous optional times
by discrete ones?  Stroock has some other interesting consequences of
this theorem too.
Here is the proof that depends only on the Doob Maximal Inequality.
\begin{proof}
Before we begin, we can clean up the notation that follows by assuming
that $\mathcal{A} = \bigvee_n \mathcal{F}_n$.  For if $\xi$ is
integrable then we know that
$\cexpectationlong{\bigvee_n \mathcal{F}_n}{\xi}$ is also integrable
and convergence in $L^1(\Omega, \bigvee_n \mathcal{F}_n, \mu)$
implies convergence in $L^1(\Omega, \mathcal{A}, \mu)$.

First goal is to validate the following claim:
\begin{align*}
\lambda \probability{\sup_{n \in \integers_+}
  \abs{\cexpectationlong{\mathcal{F}_n}{\xi}} \geq \lambda} &\leq
\expectation{ \abs{\xi} ; \sup_{n \in \integers_+}
  \abs{\cexpectationlong{\mathcal{F}_n}{\xi}} \geq \lambda}  \leq \expectation{\abs{\xi}}
\end{align*}
\emph{Here is where Stroock reduces this to Doob's Maximal Inequality
  along the way claiming that we may assume $\xi \geq 0$.  I don't
  understand how to validate his claim about the positivity assumption
  and I am stuck trying to use
  Doob's Maximal Inequality as we've stated it. However it is easy to
  rescue the situation by adapting the proof of the
  Maximal Inequality to prove the above as you'll see}.
We first prove the claim for a finite index set.  Since we 
know from Lemma \ref{ClosedMartingales} that
$\cexpectationlong{\mathcal{F}_n}{\xi}$ is an
$\mathcal{F}$-martingale. we know from
that $\abs{\cexpectationlong{\mathcal{F}_n}{\xi}}$ is a
submartingale.  We let $\tau$ be hitting time of the interval
$[\lambda, \infty)$ and note that 
\begin{align*}
\lbrace \sup_{n \in \integers_+}
  \abs{\cexpectationlong{\mathcal{F}_n}{\xi}} \geq \lambda \rbrace =
  \cup_{0 \leq m \leq n} \lbrace \tau = m \rbrace
\end{align*}
where the union is disjoint.  Since $\tau$ is an optional time
(Lemma \ref{HittingTimesDiscrete}) we also know that $\lbrace \tau =
m \rbrace \in \mathcal{F}_m$ and therefore
\begin{align*}
\expectation{\abs{\xi} ; \tau = m} &= 
\expectation{\cexpectationlong{\mathcal{F}_m}{\abs{\xi}} ; \tau =  m} \geq 
\expectation{\abs{\cexpectationlong{\mathcal{F}_m}{\xi}} ; \tau =  m}
\geq 
\lambda \probability{\tau = m}
\end{align*}
and summing for $m$ from 0 to $n$ yields
\begin{align*}
\lambda \probability{\max_{0 \leq m \leq n}
  \abs{\cexpectationlong{\mathcal{F}_m}{\xi}} \geq \lambda} &\leq
\expectation{ \abs{\xi} ; \max_{0 \leq m \leq n}
  \abs{\cexpectationlong{\mathcal{F}_m}{\xi}} \geq \lambda} 
\end{align*}
The result is completed by taking the limit as $n$ goes to infinity
and using continuity of measure (Lemma \ref{ContinuityOfMeasure}) and
Montone Convergence.

\emph{Here is the result from Stroock}
We know from Lemma \ref{ClosedMartingales} that
$\cexpectationlong{\mathcal{F}_n}{\xi}$ is an
$\mathcal{F}$-martingale.  By Doob's Maximal Inequality (Lemma
\ref{DoobMaximalInequalityDiscrete}), the
$\mathcal{F}_n$-measurability of  $\lbrace \sup_{0 \leq k
    \leq n}\cexpectationlong{\mathcal{F}_k}{\xi} \geq \lambda \rbrace$ and another application of the
tower property we know that 
\begin{align*}
\lambda \probability{\sup_{0 \leq k \leq n}
  \cexpectationlong{\mathcal{F}_k}{\xi} \geq \lambda} &\leq
\expectation{\cexpectationlong{\mathcal{F}_n}{\xi}; \sup_{0 \leq k
    \leq n}\cexpectationlong{\mathcal{F}_k}{\xi} \geq \lambda} \\
&= \expectation{\xi; \sup_{0 \leq k
    \leq n}\cexpectationlong{\mathcal{F}_k}{\xi} \geq \lambda}
\end{align*}
By continuity of measure (Lemma \ref{ContinuityOfMeasure}) we know
that 
\begin{align*}
\probability{\sup_{k}
  \cexpectationlong{\mathcal{F}_k}{\xi} \geq \lambda} &= \lim_{n \to \infty} \probability{\sup_{0 \leq k \leq n}
  \cexpectationlong{\mathcal{F}_k}{\xi} \geq \lambda} \\
\intertext{and by Dominated Convergence}
\expectation{\xi; \sup_{k}\cexpectationlong{\mathcal{F}_k}{\xi} \geq \lambda} &=
\lim_{n \to \infty} \expectation{\xi; \sup_{0 \leq k
    \leq n}\cexpectationlong{\mathcal{F}_k}{\xi} \geq \lambda} \\
\intertext{so we have shown}
\lambda \probability{\sup_{k}
  \cexpectationlong{\mathcal{F}_k}{\xi} \geq \lambda} &\leq
\expectation{\xi; \sup_{k}\cexpectationlong{\mathcal{F}_k}{\xi} \geq \lambda}
\end{align*}
\emph{End result from Stroock}

To show almost sure convergence, we let $\mathcal{G}$ denote the set
of all integrable $\xi$ such that
$\cexpectationlong{\mathcal{F}_n}{\xi} \toas \xi$.  Note that any
$\mathcal{F}_n$-measurable $\xi$ is in $\mathcal{G}$ since the sequence
of conditional expectations is eventually almost surely constant and
equal to $\xi$.  On the other hand we know that $\cup_n
L^1(\Omega, \mathcal{F}_n, \mu)$ is dense in $L^1(\Omega, \bigvee_n
  \mathcal{F}_n, \mu) = L^1(\Omega, \mathcal{A}, \mu)$ (Lemma
  \ref{LpDensityUnionSubsigmaAlgebras}) so it suffices to show that
  $\mathcal{G}$ is closed in $L^1$.  So suppose that $\xi_n$ is a
  sequence in $\mathcal{G}$ such that $\xi_n \tolp{1} \xi$.  We show
  that $\cexpectationlong{\mathcal{F}_n}{\xi} \toas \xi$ by using
  Lemma \ref{ConvergenceAlmostSureByInfinitelyOften}.  Suppose
  $\epsilon > 0$ is given, then for every $m,n$
\begin{align*}\begin{split}
\probability{\sup_{k \geq m} \abs{\cexpectationlong{\mathcal{F}_k}{\xi} - \xi} >
  \epsilon} &\leq \probability{\sup_{k \geq m}
  \abs{\cexpectationlong{\mathcal{F}_k}{\xi-\xi_n}} >
  \frac{\epsilon}{3}} + \\
&\qquad \probability{\sup_{k \geq m}
  \abs{\cexpectationlong{\mathcal{F}_k}{\xi_n} - \xi_n} >
  \frac{\epsilon}{3}} + \probability{\abs{\xi_n - \xi} >
  \frac{\epsilon}{3}} \\
&\leq \frac{6}{\epsilon} \expectation{\abs{\xi - \xi_n} } +
\probability{\sup_{k \geq m}
  \abs{\cexpectationlong{\mathcal{F}_k}{\xi_n} - \xi_n} >
  \frac{\epsilon}{3}} \\
\end{split}
\end{align*}
where the first term is bounded by our claim at the beginning of
proof applied to $\xi_n - \xi$ and the third term is
bounded by the Markov Inequality (Lemma
\ref{MarkovInequality}).

Taking the limit as $m$ goes to infinity and using our assumption that
$\xi_n \in \mathcal{G}$ and the characterization of almost sure
convergence from Lemma
\ref{ConvergenceAlmostSureByInfinitelyOften} we see that $\lim_{m \to
  \infty} \probability{\sup_{k \geq m}
  \abs{\cexpectationlong{\mathcal{F}_k}{\xi_n} - \xi_n} >
  \frac{\epsilon}{3}} = 0$.  
Therefore
\begin{align*}
\lim_{m \to \infty} \probability{\sup_{k \geq m} \abs{\cexpectationlong{\mathcal{F}_k}{\xi} - \xi} >
  \epsilon} &\leq \frac{6}{\epsilon} \expectation{\abs{\xi - \xi_n} }
\\
\end{align*}
and by taking the limit as $n$ goes to infinity we get
\begin{align*}
\lim_{m \to \infty} \probability{\sup_{k \geq m} \abs{\cexpectationlong{\mathcal{F}_k}{\xi} - \xi} >
  \epsilon}&=0
\end{align*}
 so $\cexpectationlong{\mathcal{F}_n}{\xi} \toas \xi$ by another application of Lemma \ref{ConvergenceAlmostSureByInfinitelyOften}.

Since we know that the family $\cexpectationlong{\mathcal{F}_n}{\xi}$
is uniformly integrable by Corollary
\ref{ConditionalExpectationsUniformlyIntegrable},
$\cexpectationlong{\mathcal{F}_n}{\xi} \tolp{1} \xi$ follows from the
almost sure convergence and Lemma \ref{LpConvergenceUniformIntegrability}.
\end{proof}

\subsection {Continuous Time Martingales and Weakly Optional Times}

Our next goal is to extend the theory we've developed to a continuous
time setting.  For the most part we proceed by using approximation
arguments to reduce results to the discrete time analogues proven in
the last section.  First we have to come to grips with some subtleties
related to filtrations, optional times and measurability in continuous
time.

\begin{defn}A $T$-valued random variable is called a
  \emph{weakly $\mathcal{F}$-optional time} (also called a
  \emph{weak $\mathcal{F}$-stopping time}) if and only if $\lbrace \tau <
  t \rbrace \in \mathcal{F}_t$ for all $t \in T$.  
\end{defn}
Just as with optional times, if the filtration $\mathcal{F}$ is clear
from context, we'll simply refer to a weakly optional time.

A weakly $\mathcal{F}$-optional time $\tau$ is a decision rule to stop
at $t$ that requires an arbitrarily small amount of future information
to determine that one should stop at $t$.  Alternatively one can
characterize it as a decision rule such that $\tau + \epsilon$ is
$\mathcal{F}$-optional for all $\epsilon>0$.

Let $\mathcal{F}^+ = \cup_{s>t} \mathcal{F}_s$ (note that $\mathcal{F}
= \mathcal{F}^+$ if and only if $\mathcal{F}$ is right continuous).

One way of defining the $\sigma$-algebra associated with a weakly
$\mathcal{F}$-optional time is as a limit of the $\sigma$-algebras
associated the $\mathcal{F}$-optional times $\tau + \epsilon$
\begin{align*}
\mathcal{F}_{\tau^+} = \cup_{\epsilon > 0} \mathcal{F}_{\tau + \epsilon}
\end{align*}
\begin{lem}\label{WeaklyOptionalCharacterization}$\tau$ is $\mathcal{F}^+$-optional if and only if $\tau$ is
  weakly $\mathcal{F}$-optional.  In this case, 
\begin{align*}
\mathcal{F}^+_\tau = \mathcal{F}_{\tau^+} = \lbrace A \in \mathcal{A}
\mid A \cap \lbrace \tau < t \rbrace \in \mathcal{F}_t \text { for all
} t \in T \rbrace
\end{align*}
\end{lem}
\begin{proof}
The first thing is to notice that for any random time $\tau$ (not just
optional or weakly optional times) we have the equalities
\begin{align*}
&\lbrace \tau \leq t \rbrace = \bigcap_{\substack{
r \in \rationals \\
r >  t}} 
\lbrace \tau < r \rbrace
&\lbrace \tau < t \rbrace = \bigcup_{\substack{
r \in \rationals \\ 
r < t}} 
\lbrace \tau \leq r \rbrace
\end{align*}
TODO: Justify (but it's kinda obvious by density of $\rationals$)

Armed with these facts we proceed to show the equality 
\begin{align*}
\mathcal{F}^+_{\tau} = \lbrace A \in \mathcal{A}
\mid A \cap \lbrace \tau < t \rbrace \in \mathcal{F}_t \text { for all
} t \in T \rbrace
\end{align*} 
for any random time $\tau$.

Suppose $A \cap \lbrace \tau \leq t \rbrace \in \mathcal{F}^+_t =
\cap_{s>t} \mathcal{F}_s$ for every $t \in T$.  Then for all $t  \in T$,
\begin{align*}
A \cap \lbrace \tau < t \rbrace &= A \cap \left (\bigcup_{\substack{r \in \rationals \\ r <
  t}} \lbrace \tau \leq r \rbrace \right ) =\bigcup_{\substack{r \in \rationals \\ r <
  t}}  \left ( A \cap \lbrace \tau \leq r \rbrace \right ) \in \mathcal{F}_t\\
\end{align*}
since for any $r < t$, $\mathcal{F}^+_r \subset \mathcal{F}_t$.

On the other hand, if $A \cap \lbrace \tau < t \rbrace \in
\mathcal{F}_t$ for all $t \in T$, then
\begin{align*}
A \cap \lbrace \tau \leq t \rbrace &= A \cap \left ( \bigcap_{\substack{r \in \rationals \\ r >
  t}} \lbrace \tau < r \rbrace \right ) =  \bigcap_{\substack{r \in \rationals \\ r >
  t}} \left ( A \cap  \lbrace \tau < r \rbrace \right ) \in \mathcal{F}^+_t
\end{align*}
where the last inclusion follows from the fact that for any $r < s$, $ A \cap
\lbrace \tau < r \rbrace \subset  A \cap  \lbrace \tau < s
\rbrace$, so for any $s \in T$ with $s > t$ we in fact
have
\begin{align*}
\bigcap_{\substack{r \in \rationals \\ r >
  t}} \left ( A \cap  \lbrace \tau < r \rbrace \right ) &= 
\bigcap_{\substack{r \in  \rationals \\ 
s \geq r >  t}} \left ( A \cap  \lbrace \tau < r \rbrace \right ) \in \mathcal{F}_s
\end{align*}

Now note that by definition, $\tau$ is weakly $\mathcal{F}$-optional if and only if
$\Omega \in \lbrace A \in \mathcal{A}
\mid A \cap \lbrace \tau < t \rbrace \in \mathcal{F}_t \text { for all
} t \in T \rbrace$ and $\tau$ is $\mathcal{F}^+$-optional
if and only if $\Omega \in \mathcal{F}^+_\tau$.  Therefore the equality just shown tells us that
$\tau$ is weakly $\mathcal{F}$-optional if and only if $\tau$ is 
$\mathcal{F}^+$-optional.

We finish by showing that $\mathcal{F}^+_\tau =
\mathcal{F}_{\tau^+}$.  To see this, note that $A \in
\mathcal{F}_{\tau^+}$ if and only if $A \in
\mathcal{F}_{\tau + \epsilon}$ for all $\epsilon > 0$ which is true if and only
if $A \cap \lbrace \tau + \epsilon \leq t 
\rbrace =  A \cap \lbrace \tau \leq t - \epsilon
\rbrace\in \mathcal{F}_{t}$ for all $t \in T$, $\epsilon > 0$ which is
true if and only if $A \cap \lbrace \tau \leq t 
\rbrace\in \mathcal{F}_{t+\epsilon}$ for all $t \in T$, $\epsilon >
0$.  This last statement is simply that $A \cap \lbrace \tau \leq t 
\rbrace\in \mathcal{F}^+_t$ for all $t \in T$ so we are done.
\end{proof}

In the previous section we identified a useful class of optional times
that we called hitting times.  Hitting times can be defined in
continuous time but there are more stringent requirements on when they
are optional times.
\begin{lem}\label{HittingTimesContinuous}Let $\mathcal{F}$ be a
  filtration on $\reals_+$, let $X_t$ be an $\mathcal{F}$-adapted
  process with values in a measurable space $(S, \mathcal{S})$ where
  $S$ is topological and $\mathcal{S}$ contains the Borel $\sigma$-algebra, 
  $B \in \mathcal{S}$ and $\tau_B = \inf \lbrace t > 0 \mid X_t \in
  B\rbrace$.  Then if $S$ is a metric space, $B$ is closed and $X_t$ is continuous
$\tau_B$ is $\mathcal{F}$-optional and if $B$ is open and $X_t$ is right continuous then $\tau_B$ is weakly $\mathcal{F}$-option.
\end{lem}
\begin{proof}
To see the first case, by the countability and density of the
rationals in $\reals_+$, continuity of $X_t$ and closedness of $B$ we
know that $X_{\tau_B} \in B$ and therefore $\tau_B \leq t$ if and only if there is an $0 < s \leq t$
such that $X_s \in B$.  This latter statement is true if and only if there is an
integer $m>0$ and points $X_q$ with
$q \in \rationals$ and $1/m \leq q \leq t$ that are arbitrarily close to
$B$.  Translating this observation into set operations we get
\begin{align*}
\lbrace \tau_B \leq t \rbrace &= \cup_{m=1}^\infty \cap_{n =1}^\infty \cup_{\substack{1/m
    \leq q \leq t \\ q \in \rationals}} \lbrace d(X_q, B) < \frac{1}{n}
  \rbrace \in \mathcal{F}_t
\end{align*}
because each $\lbrace x \in S \mid d(x, B) < \frac{1}{n}  \rbrace$ is
open and thus $\lbrace d(X_q, B) < \frac{1}{n}  \rbrace \in
\mathcal{F}_t$ because $X$ is $\mathcal{F}$-adapted.
To see the second case note that by similar considerations $\tau_B < t$ if and only if there
exists a $q \in \rationals$ such that $0 \leq q < t$ with $X_q \in B$
thus
\begin{align*}
\lbrace \tau_B \leq t \rbrace &= \cup_{\substack{0 \leq q < t \\ q \in
    \rationals}} \lbrace X_q \in B \rbrace \in \mathcal{F}_t
\end{align*}

\end{proof}

When passing from discrete time results to continuous time results it
is often useful to approximate an optional time on a continuous domain
by a discrete one.  The following approximation scheme is so useful it
deserves to be called out.

\begin{lem}\label{DiscreteApproximationOptionalTimes}Let $\tau$ be a weakly optional time on $\reals_+$, then define
\begin{align*}
\tau_n = \frac{1}{2^n} \floor{2^n \tau + 1}
\end{align*}
$\tau_n$ is a sequence of optional times with values in a
countable index set such that $\tau_n\downarrow \tau$.
\end{lem}
\begin{proof}
The fact that each $\tau_n$ is an optional time follows from the
definition and the fact that $\tau$ is a weakly optional time:
\begin{align*}
\lbrace \tau_n \leq \frac{k}{2^n} \rbrace &= \lbrace \frac{k-1}{2^n}
\leq \tau < \frac{k}{2^n} \rbrace = \lbrace \tau < \frac{k-1}{2^n}
\rbrace^c \cap \lbrace
\tau < \frac{k}{2^n} \rbrace \in \mathcal{F}_{\frac{k}{2^n}}
\end{align*}

To see the fact that $\tau_n$ is decreasing, note $\tau_n =
\frac{k}{2^n}$ if and only if $\frac{k-1}{2^n} \leq \tau <
\frac{k}{2^n}$  
which implies 
\begin{align*}
\tau_{n+1} &= \begin{cases}
\frac{k}{2^n} & \text{if $\frac{2k-1}{2^{n+1}} \leq \tau < \frac{k}{2^n}$} \\
\frac{2k-1}{2^{n+1}} & \text{if $\frac{k-1}{2^{n}} \leq \tau < \frac{2k-1}{2^{n+1}}$} \\
\end{cases}
\end{align*}
Convergence to $\tau$ follows easily since $\abs{\tau - \tau_n} \leq \frac{1}{2^n}$ by definition.
\end{proof}

If we have approximation scheme for an optional time we may also want
to understand how the associated $\sigma$-algebras behave.  For the
decreasing approximation of the previous lemma, part (ii) of the
following gives us the answer.
\begin{lem}\label{InfSupStoppedFiltration}If we have a finite or countable collection of optional
  times $\tau_n$ then $\sup_n \tau_n$ is an optional time.  If we have
  a finite or countable collection of weakly optional times $\tau_n$
  then $\tau = \inf_n \tau_n$ is a weakly optional time and
  furthermore
\begin{align*}
\mathcal{F}^+_\tau &= \cap_n \mathcal{F}^+_{\tau_n}
\end{align*}
\end{lem}
\begin{proof}
If $\tau_n$ are optional times then if follows from the definition of
supremum that $\lbrace \tau \leq t \rbrace = \cap_n \lbrace \tau_n
\leq t \rbrace$ and therefore $\tau$ is an optional time.

If $\tau_n$ are weakly optional times then if follows from the definition of
infimum that $\lbrace \tau < t \rbrace = \cup_n \lbrace \tau_n
< t \rbrace$ and therefore $\tau$ is a weakly optional time.
Furthermore because $\tau \leq \tau_n$ for all $n$ we know that
$\mathcal{F}^+_\tau \subset \mathcal{F}^+_{\tau_n}$ for all $n$.  On
the other hand by Lemma \ref{WeaklyOptionalCharacterization}, if we know that $A \in \cap_n \mathcal{F}^+_{\tau_n}$
then $A \cap \lbrace \tau_n < t \rbrace \in \mathcal{F}_t$ for all $n$
and $t$.  Therefore we can write $A \cap \lbrace \tau < t \rbrace =  \cup_n A \cap \lbrace \tau_n
< t \rbrace \in \mathcal{F}_t$ which shows that $A \in
\mathcal{F}^+_\tau$ by another application of Lemma \ref{WeaklyOptionalCharacterization}.
\end{proof}

We shall have a need for the following characterization of uniform
integrability for martingales on $\integers_-$ (sometimes called a
\emph{backward submartingale}).
\begin{lem}\label{BackwardSubmartingaleBoundedUniformlyIntegrable}Let $X_n$ be an $\mathcal{F}$-submartingale on
  $\integers_-$, then $\expectation{X_n}$ is bounded if and only if $X_n$ is
  uniformly integrable.
\end{lem}
\begin{proof}
As a first simple observation, we know that since $X_n$ is a
submartingale then 
\begin{align*}
\expectation{X_n}  &=
\expectation{\cexpectationlong{\mathcal{F}_{n-1}}{X_n}} \geq
\expectation{X_{n-1}}
\end{align*}
so boundedness of $\expectation{X_n}$ is equivalent to $\lim_{n \to -\infty}
\expectation{X_n} = \inf_n \expectation{X_n} > -\infty$.

Assume that $\expectation{X_n}$ is $L^1$ bounded.  We proceed by constructing the
analogue of the Doob Decomposition for time index $\integers_-$ and
then invoking results for martingales.  Recall in the Doob
Decomposition we write a submartingale $X_n$ on $\integers_+$ as $M_n
+ A_n$ where $M_n$ is a martingale and $A_n = \sum_{m=1}^n
\cexpectationlong{\mathcal{F}_{m-1}}{X_m} - X_{m-1}$.  So to make this
work for $\integers_-$ we have to handle the fact that the desired
definitions now involve an infinite sum which must converge for things
to make sense.  To
that end, define for $n \leq 0$,
\begin{align*}
\alpha_n = \cexpectationlong{\mathcal{F}_{n-1}}{X_n} - X_{n-1} =
\cexpectationlong{\mathcal{F}_{n-1}}{X_n - X_{n-1}} \geq 0
\end{align*}
so that $\alpha_n$ is a predictable process.  Observe that by Monotone Convergence
\begin{align*}
\expectation{\sum_{n \leq 0} \alpha_n}
&= \lim_{m \to \infty} \sum_{-m \leq n \leq 0} \expectation{ \alpha_n} \\
&= \lim_{m \to \infty} \sum_{-m \leq n \leq 0}
\expectation{X_n} - \expectation{X_{n-1}} \\
&= \expectation{X_0} - \inf_n \expectation{X_n} < \infty \\
\end{align*}
Therefore we know that $\sum_{n \leq 0} \alpha_n$ is almost surely
finite.  With that in hand we can define for each $n \leq 0$
\begin{align*}
A_n &= \sum_{m \leq n} \alpha_m = \sum_{m \leq n}
\cexpectationlong{\mathcal{F}_{m-1}}{X_m} - X_{m-1}
\end{align*}
so that $A_n$ is integrable.  Moreover since $A_n$ is almost surely
increasing we know that $\sup_{n} A_n \leq A_0$ and therefore the
sequence $A_n$ is
uniformly integrable (e.g. see Example \ref{DominatedImpliesUniformlyIntegrable}).
Now we define
\begin{align*}
M_n &= X_n - A_n
\end{align*}
so that by integrability of $A_n$ we have $M_n$ is integrable and
moreover
\begin{align*}
\cexpectationlong{\mathcal{F}_{n-1}}{M_n} &=
\cexpectationlong{\mathcal{F}_{n-1}}{X_n} - A_n \\
&=\cexpectationlong{\mathcal{F}_{n-1}}{X_n} -
\cexpectationlong{\mathcal{F}_{n-1}}{X_n} + X_{n-1} - A_{n-1} = M_{n-1}\\
\end{align*}
so that $M_n$ is a martingale.  Since $M_n$ is closed we conclude from 
Theorem \ref{L1MartingaleConvergenceTheoremDiscrete} that $M_n$ is
uniformly integrable.  The uniform integrability of $A_n$ and $M_n$
together imply the uniform integrability of $X_n$ (Lemma \ref{SumsOfUniformlyIntegrable}).

Now if we assume that $X_n$ is uniformly integrable then it follows
that $X_n$ is $L^1$ bounded (Lemma
\ref{UniformIntegrabilityProperties}) and therefore
$\expectation{X_n}$ is bounded since $\abs{\expectation{X_n}} \leq \expectation{\abs{X_n}}$.
\end{proof}

 TODO: Introduce complete right continuous filtration and existence of
a cadlag version of martingales.  Say something about the nature of
the modification (e.g. preservation of f.d.d.'s) and the fact that it
is a modification.

The martingale results for discrete time tell us quite a
bit about what can happen in continuous time as well.  If we are given
a submartingale on $\reals_+$ then we can restrict it to
$\rationals_+$ and ask what we know about the restricted process; as
we'll  soon see we know quite a lot!  The first issue which we examine
gets to the heart of whether we can extrapolate from the discrete case
to the continuous case.  If there are no restrictions on the
regularity/continuity of sample paths then there is very little that we can say
about what happens on $\reals_+ \setminus \rationals_+$ based on what
is happening on $\rationals_+$.  Thus our first task is to understand
the ways in which we can modify a continuous time submartingale to get
a different submartingale that has some continuity in sample paths. Note here
that the use of the word modify is quite a bit subtle: we mean to use
the word both in its colloquial sense of \emph{how can we change
  continuous time submartingale to make it have regular sample paths}
as well as the technical sense of \emph{when are the changes that we
  make to a continuous time submartingale a modification of the
  stochastic process}.   The
specific type of regularity we aim for is that sample paths of the
submartingale are right continuous and have left limits.  It is
traditional to refer to such paths as cadlag which is an acronym
derived from the French phrase \emph{continue \`{a} droits limite
  \`{a} gauche}.  The reader may encounter the acronym derived from
the English rcll but it seems to be less popular.

The formal development of these ideas comes with a lot of technical
baggage so before we jump into the details let's step
back and think about what we can expect.  We discuss the martingale
case here even though almost all of what we say applies equally to
submartingales.  Lets suppose that we have an
$\mathcal{F}$-martingale $X$ on $\reals_+$.  If we restrict a
$X$ to $\rationals_+$ then the convergence
theorems (and at their the base the upcrossing lemma) tells us that
almost surely on
any bounded interval the restricted martingale has limits along all
sequences.  Therefore at worst the restricted martingale on
$\rationals_+$ has jump discontinuities (almost surely!).  This gives us hope that we
can modify (in the colloquial sense) $X$ to create a new process $Y$
on $\reals_+$ such that $Y$ is cadlag: simply define $Y_t$ for $t \in
\reals_+$ such that $Y_t =
\lim_{\substack{q\downarrow t \\ q \in \rationals}} X_q$.  This gives
us a process to be sure but it isn't even $\mathcal{F}$-adapted in
general: by the definition of $Y_t$ as limit of $X_q$ for $q > t$  we know that $Y_t$ is
$\mathcal{F}^+_t$-measurable but it is not necessarily $\mathcal{F}_t$-measurable.  This
introduces one of the key ideas: if we have any hope of  changing $X$
to get an adapted cadlag $Y$ we had
either be prepared to pass to the filtration $\mathcal{F}^+$ or start
with a right continuous one.  

We've already glossed over an issue that brings up a second key idea.
The construction described only works \emph{almost surely}; we have to
come up a different plan on the null set where $X_q$ is wild.  The easiest thing
to do is just to set $Y_t \equiv 0$ when this occurs.  Since the event
of $X$ on $\rationals_+$ being ill-behaved has probability zero whatever it is we decide
to do won't prevent $Y$ from being a version of $X$.  The issue is that
the event of $X$ on $\rationals_+$ being ill-behaved depends on all of
$X_t$ for all $t \geq 0$ hence is in $\mathcal{F}_\infty$; thus as we
modify $X_t$ to get $Y_t$, in accounting for the ill-behavedness of $X_t$
we are changing each $Y_t$ on an event in $\mathcal{F}_\infty$ further
destroying adaptedness of $Y$.  The good news is that we do know that
the event in question is a null event and therefore we come to the
second key idea: to get a cadlag $Y$ from $X$ we had either be
prepared to add all of the null events of $\mathcal{F}_\infty$ to each
$\mathcal{F}^+_t$ or assume that they are there to begin with.  The
filtration that is right continuous and has null sets added is
referred to as the \emph{partial augmentation} of $\mathcal{F}$ (it is
distinguished from the full augmentation in that it does not assume
that the filtration is complete).

These first two ideas are enough to get us an adapted process $Y$ but
more is true: $Y$ is a martingale with respect the partially
augmented filtration.  This is not obvious and requires checking using
discrete time results.  The remaining issue and question
is whether $Y$ is indeed a version of $X$.  The following example shows
that this may not be true without further hypotheses on $X$.

\begin{examp}\label{CadlagNotAModification}Let $\Omega = \lbrace -1, 1 \rbrace$ with the probability
  measure $\probability{1} = \probability{-1} = \frac{1}{2}$.  Let
  $\mathcal{F}_t = \lbrace \Omega, \emptyset \rbrace$ for $0 \leq t
  \leq 1$ and $\mathcal{F}_t = \lbrace \Omega, \emptyset, \lbrace 1
  \rbrace, \lbrace -1 \rbrace\rbrace$ for $t > 1$ let 
\begin{align*}
X_t(\omega) &= \begin{cases}
0 & \text{for $0 \leq t \leq 1$} \\
\omega & \text{for $t > 1$}
\end{cases}
\end{align*}
It is easy to see that $X_t$ is an $\mathcal{F}$-martingale.  Now
define 
\begin{align*}
Y_t(\omega) &= \begin{cases}
0 & \text{for $0 \leq t < 1$} \\
\omega & \text{for $t \geq 1$}
\end{cases}
\end{align*}
and note that $Y_1$ is not $\mathcal{F}_1$-measurable.  However, it is
easy to see that 
 $\mathcal{F}^+_t = \lbrace \Omega, \emptyset \rbrace$ for $0 \leq t
  < 1$,  $\mathcal{F}^+_t = \lbrace \Omega, \emptyset, \lbrace 1
  \rbrace, \lbrace -1 \rbrace\rbrace$ for $t \geq 1$ and 
  $Y_t$ is an $\mathcal{F}^+$-martingale.  Note however that
  $\probability{X_1 = Y_1} = 0 \neq 1$ and thus $Y$ is not a version
  of $X$.

Note that $X$ is not a $\mathcal{F}^+$-martingale (or
$\mathcal{F}^+$-sub/supermartingale) as for $t > 1$ we have
$\cexpectationlong{\mathcal{F}^+_1}{X_t} = X_t$ and therefore
$\cexpectationlong{\mathcal{F}^+_1}{X_t}> X_1$ with probability $1/2$
(i.e. when $\omega = 1$)
and $\cexpectationlong{\mathcal{F}^+_1}{X_t} < X_1$ with probability
$1/2$ (i.e. when $\omega = -1$).
\end{examp}

Example \ref{CadlagNotAModification} shows that there is a limit to
what we can accomplish by taking a martingale with respect to an
arbitrary filtration and trying find a version that is cadlag.  
Nonetheless, the method we've outlined to make a cadlag process $Y$
from an arbitrary process $X$ can be shown to result in a version if
$X$ is assumed to be a martingale with respect to the the right continuous
filtration in the first place (plus some extra conditions if $X$ is
only assumed to be a submartingale).  Thus the impediment to the
existence of a cadlag version in Example \ref{CadlagNotAModification}
is in the final comment about $X$ not being a martingale with respect
to the right continuous filtration. 

\begin{thm}\label{CadlagModificationContinuousMartingale}Let $X_t$  be a $\mathcal{F}$-submartingale on $\reals_+$
  and let $Y_q$ denote the restriction to $\rationals_+$.  
\begin{itemize}
\item[(i)]There
  exists a set $A \subset \mathcal{F}_\infty$ with $\probability{A}=1$ on which
  $\lim_{q \to t^+} Y_q$ and $\lim_{a \to t^-} Y_q$ exist for all $t
  \in \reals_+$.  If we
  define 
\begin{align*}
Z_t(\omega) = 
\begin{cases}
\lim_{q \to t^+} Y_q(\omega) & \text{if $\omega \in A$} \\
0 & \text{if $\omega \notin A$} \\
\end{cases}
\end{align*} 
then $Z$ is a cadlag
  $\overline{\mathcal{F}_+}$-submartingale.
\item[(ii)]$X$ has a right
continuous version if and only if $Z$ is a version of $X$.
\item[(iii)]If $\mathcal{F}$ is right continuous then $Z$ is a version of $X$ if and only if $t \mapsto
\expectation{X_t}$ is right continuous.  
\end{itemize}
\end{thm}
\begin{proof}
Pick $N \in \naturals$ and note that since $Y^+_q$ is a submartingale
we have for all $q \in [0,N] \cap \rationals$
\begin{align*}
\expectation{Y^+_q} &\leq
\expectation{\cexpectationlong{\mathcal{F}_q}{Y^+_N}} = \expectation{Y^+_N}
\end{align*}
and by the same reasoning using the fact that $Y_q$ is a submartingale
we know that $\expectation{Y_0} \leq \expectation{Y_q}$.  Together
these imply 
\begin{align*}
\expectation{\abs{Y_q}} &= \expectation{Y^+_q}  + \expectation{Y^-_q}
= 2\expectation{Y^+_q}  - \expectation{Y_q} \leq 2\expectation{Y^+_N}  - \expectation{Y_0} 
\end{align*}
which implies that $Y_q$ restricted to $[0,N]$ is $L^1$-bounded.
We can apply Theorem \ref{L1MartingaleConvergenceTheoremDiscrete} to
the restricted submartingale $Y_q$ to construct $A_N \in \mathcal{F}_N$ with
$\probability{A_N}=1$ such that for all increasing and decreasing
sequences $q_n$ in $\rationals \cap [0,N]$ we have $Y_{q_n}$ converge
on $A_N$.  So in particular $\lim_{q \to t^-} Y_q$ and $\lim_{q
  \to t^+} Y_q$ exist for every $t \in [0,N]$ on $A_N$.  Taking the
intersection of $A = \cap_{N=1}^\infty A_N$ we see that  $\lim_{q \to t^-} Y_q$ and $\lim_{q
  \to t^+} Y_q$ exist for every $t \in \reals_+$ on $A \in
\mathcal{F}_\infty$.  Note that the process $Z_t$ is
$\overline{\mathcal{F}_+}$-adapted (in fact is adapted with respect to
the smaller filtration $\mathcal{G}_t = \sigma(A, \mathcal{F}^+_t)$).
We claim that the process $Z_t$ is cadlag.  Clearly sample paths on
$A^c$ are continuous since they are constant.  So we consider a sample
path on $A$.  If we fix $t \geq 0$ and $\epsilon > 0$ then
by definition of $Z_t$ on $A$ we may find a $\delta > 0$ such that
$\abs{Y_q - Z_t} < \epsilon/2$ for all $t < q < t +\delta$ and $q \in
\rationals$.  If we take an arbitrary $t < s < t + \delta$ then again
applying the definition of $Z_s$ we may pick a $s < q < t+\delta$ such
that $\abs{Y_q - Z_s} < \epsilon/2$, therefore by the triangle
inequality we have $\abs{Z_t - Z_s} < \epsilon$ and right continuity
is established.  Similarly if we define $Y^-_t = \lim_{q \to t^-} Y_q$, we may find a $\delta$ such that
$\abs{Y^-_t - Y_q} < \epsilon/2$ for all $t - \delta < q < t$.  For
any $t - \delta < s < t$ by the definition of $Z_s$ we may pick $s < q
< t$ such that $\abs{Y_q-Z_s} < \epsilon/2$ and again the triangle
inequality implies $\abs{Y^-_t - Z_s} < \epsilon$.
This shows $\lim_{s \to t^-} Z_s = \lim_{q \to t^-} Y_q$
and in particular $Z_t$ has left limits.

Now to see that $Z$ is a submartingale, let $0 \leq s < t < \infty$ be
arbitrary and pick decreasing sequence $t_n \in \rationals_+$
such that $t_n \downarrow t$ and a decreasing sequence $s_n \in
\rationals_+$ such that $s_n < t$ for all $n$ and  $s_n \downarrow s$.
For each $n$ and $m$ we have
$Y_{s_m} \leq \cexpectationlong{\mathcal{F}_{s_m}}{Y_{t_n}}$ a.s. by the
submartingale property of $X$.  By the Levy Downward Theorem
\ref{JessenConditioningLimits} we know that $\lim_{m \to \infty}
\cexpectationlong{\mathcal{F}_{s_m}}{Y_{t_n}} =
\cexpectationlong{\mathcal{F}^+_{s}}{Y_{t_n}}$ a.s. and by definition
$Z_s = \lim_{m \to \infty} Y_{s_m}$ therefore $Z_s
\leq \cexpectationlong{\mathcal{F}_{s}}{Y_{t_n}}$ a.s.
Again, by definition  $Y_{t_n} \toas Z_t$, furthermore as the sequence $t_n$ is bounded
we have already shown
$Y_{t_n}$ is $L^1$-bounded.  This allows us to apply Lemma
\ref{BackwardSubmartingaleBoundedUniformlyIntegrable} to conclude that
$Y_{t_n}$ is uniformly continuous hence $Y_{t_n} \tolp{1} Z_t$ by
Lemma \ref{LpConvergenceUniformIntegrability}.  Thus we have 
\begin{align*}
Z_s &\leq \lim_{n \to \infty}
\cexpectationlong{\mathcal{F}^+_s}{Y_{t_n}} =
\cexpectationlong{\mathcal{F}^+_s}{Z_{t}} =
\cexpectationlong{\overline{\mathcal{F}}^+_s}{Z_{t}}
\end{align*}
where the last equality follows by Lemma \ref{ConditionalExpectationCompletions}.

It is worth noting that the submartingale property also holds with
respect to the smaller filtration $\mathcal{G}_t$ alluded to above; the fact that the result is
expressed in terms of the augmented filtration is due to the fact that
the augmented filtration proves to be necessary in subsequent theory.
In fact when we get to the discussion of Girsanov theory it will be
inconvenient to require the full completion of $\mathcal{F}^+$.

Suppose that $X$ has a right continuous version $W$.  Then by
taking an intersection of almost sure events, we see that almost
surely $Y_q = W_q$ for all $q \in \rationals_+$ ($Y$ continues to
denote the restriction of $X$ to $\rationals$).  If we fix a
particular $t \geq 0$ and use the fact that $W$ is a version
of $X$, the right continuity of $W_q$ and the definition of $Z$ to
see that almost surely we see
\begin{align*}
X_t &= W_t = \lim_{q \to t^+} W_q = \lim_{q \to t^+} Y_q = Z_t
\end{align*}
and therefore $Z$ is a version of $X$.

We now assume that $\mathcal{F}$ is right continuous.  Before
proceeding to show that $Z$ is a version of $X$ if and only if
$\expectation{X_t}$ is right continuous we need two small computations.
We have already observed that for every sequence $t_n \downarrow t$ with $t
\geq 0$ and $t_n \in \rationals_+$ we have 
not only does $Y_{t_n} \toas Z_t$ but also $Y_{t_n} \tolp{1}
Z_t$. From this fact and the definition of $Y$ we get
\begin{align*}
\lim_{t_n \to t} \expectation{X_{t_n}} &= \lim_{t_n \to
  t}\expectation{Y_{t_n}} = \expectation{Z_t}
\end{align*}
Moreover using the submartingale property of $X$, the definition of
$Y$, the fact that $Y_{t_n} \tolp{1} Z_t$, Lemma \ref{ConditionalExpectationCompletions} and the
$\overline{\mathcal{F}}$-adaptedness of $Z$ we get
\begin{align*}
X_t &\leq \lim_{t_n \to t} \cexpectationlong{\mathcal{F}_t}{X_{t_n}} =
\lim_{t_n \to t} \cexpectationlong{\mathcal{F}_t}{Y_{t_n}} =
\cexpectationlong{\mathcal{F}_t}{Z_t} =\cexpectationlong{\overline{\mathcal{F}}_t}{Z_t} = Z_t \text{ a.s.}
\end{align*}

Now we suppose that $\expectation{X_t}$ is a right continuous function
of $t$ and we want to show that $Z$ is a version of $X$.  From the
above two observations and the right continuity of $\expectation{X_t}$
we get
\begin{align*}
\expectation{\abs{Z_t - X_t}} &= \expectation{Z_t - X_t} =
\expectation{Z_t} - \expectation{X_t} = 0
\end{align*}
which shows $X_t = Z_t$ a.s. (i.e. $Z$ is a version of $X$).

Now if we assume that $Z$ is a version of $X$ then playing the above
argument backward, we conclude that 
\begin{align*}
\expectation{X_t} &= \expectation{Z_t - X_t}  + \expectation{Z_t} =
\expectation{Z_t} = \lim_{t_n \to t} \expectation{X_{t_n}}
\end{align*}
which shows that $\expectation{X_t}$ is right continuous.
\end{proof}
Since the condition of right continuity of $\expectation{X_t}$ is
trivially satisfied in the case of a martingale we know that given any
martingale $X$ on a right continuous filtration we may find a version
of $X$ that is a cadlag martingale on the completion of that
filtration (actually the completion is quite a bit more than is required as
seen from the proof).  For the most part we don't worry too much about
the fact that the filtration has to be enlarged and in much of the
theory we define the problem away by assuming that our filtration is
both right continuous and complete to begin with.  There are however
cases in which passing to the completion can cause real issues
(e.g. one loses the Borel space property by adding in additional
sets).  Note that need to enlarge the space to the completion is deal
with those places in which right limits do not exist.  If one know a priori
(or is will to assume) that these limits exist then one can dispense
with the addition of null sets and it suffices to take the right
continuous filtration.

\begin{lem}\label{DoobMaximalInequalityContinuous}Let $X_t$ be a
  cadlag submartingale on $\reals_+$, then for any $t$ and $\lambda$ we have
\begin{align*}
\lambda \probability{\sup_{s \leq t} X_s \geq \lambda} &\leq
\expectation{X_t ; \sup_{s \leq t} X_s \geq \lambda} \leq \expectation{X_t^+}
\end{align*}
Furthermore if $X_t$ is non-negative then for any $p > 1$ we have
\begin{align*}
\expectation{\sup_{s \leq t} X_s} &\leq \frac{p}{p-1}\norm{X_t}_p
\end{align*}
\end{lem}
\begin{proof}
Claim 1:  For any $\omega \in \Omega$ such that
$X_t(\omega)$ is cadlag, we have
\begin{align*}
\sup_{\substack{s \leq t \\ s \in \rationals \cup \lbrace t \rbrace }}
X_s(\omega) &= \sup_{\substack{s \leq t \\ s \in \reals }}
X_s(\omega)
\end{align*}
To see this note that given any $\epsilon > 0$ we can find $s \leq t$
with $s \in \reals$ such that $X_s(\omega) > \sup_{\substack{s \leq t \\ s \in \reals }}
X_s(\omega) - \frac{\epsilon}{2}$.  By right continuity and density of
rationals, we can find
$r \in \rationals \cup \lbrace t \rbrace$ such that $s \leq r \leq t$
and $\abs{X_r(\omega) - X_s(\omega)} < \frac{\epsilon}{2}$ which by
the triangle inequality tells us that $X_r(\omega) > \sup_{\substack{s \leq t \\ s \in \reals }}
X_s(\omega) - \epsilon$.  Therefore 
\begin{align*}
\sup_{\substack{s \leq t \\ s \in \rationals \cup \lbrace t \rbrace }}
X_s(\omega) &\geq \sup_{\substack{s \leq t \\ s \in \reals }}
X_s(\omega) -\epsilon
\end{align*}  Since $\epsilon > 0$ was arbitrary we can set
it to zero to get 
\begin{align*}
\sup_{\substack{s \leq t \\ s \in \rationals \cup \lbrace t \rbrace }}
X_s(\omega) &\geq \sup_{\substack{s \leq t \\ s \in \reals }}
X_s(\omega)
\end{align*}
The opposite inequality is immediate from the definition of supremum
so the claim is verified.

By the Claim 1 and the countable index set maximal inequality (Lemma
\ref{DoobMaximalInequalityDiscrete}) we get the first result.  By
Claim 1 and the countable index set $L^p$ inequality we get the second result.
\end{proof}

\begin{lem}[Doob's $L^p$
  Inequality]\label{DoobLpInequalityContinuous}Let $X_t$ be a
  non-negative submartingle on $\reals_+$ with $X_t$ and $\mathcal{F}$ right continuous, then for all
  $p > 1$ and $0 \leq t < \infty$,
\begin{align*}
\norm{\sup_{0 \leq s \leq t} X_s}_p &\leq \frac{p}{p-1}\norm{X_t}_p
\end{align*}
\end{lem}
\begin{proof}
TODO:
\end{proof}

\begin{thm}[$L^1$ Submartingale Convergence
  Theorem]\label{MartingaleConvergenceBoundedL1Continuous}Let $X_t$ be a cadlag
  $\mathcal{F}$-submartingale on $\reals_+$ such that $\sup_{0 \leq t
    < \infty} \norm{X_t}_1 <
  \infty$ then there exists an $X \in L^1$ such that $X_t \toas X$ a.s.
\end{thm}
\begin{proof}
Restricting $X_t$ to $\rationals_+$ and applying Theorem
\ref{MartingaleConvergenceBoundedL1Discrete} we know that there
exists $X$ such that $\lim_{\substack{q \to \infty \\ q \in
    \rationals_+}} X_q = X$ almost surely.  By right continuity of $X$
we also get that $\lim_{t \to \infty} X_t = X$ almost surely (let
$\epsilon > 0$ be given, for
almost every $\omega$ we pick $N_\omega$ such that $\abs{X_q(\omega) -
  X(\omega)} \leq \epsilon$ for all $q > N_\omega$ then for any $t >
N_\omega$ we have $\abs{X_t(\omega) -
  X(\omega)} = \lim_{\substack{q \downarrow t \\ q \in \rationals_+}} \abs{X_t(\omega) -
  X(\omega)} \leq \epsilon$).
\end{proof}


NOTE: It is also true that a UI submartingale is $L^1$ convergent
but it is not true that an $L^1$ convergence submartingale must be
UI (while that is true in discrete time).  Make this into a result
somewhere and provide the counterexample (I am pretty sure Rogers and
Williams has one).

\begin{thm}[Martingale Closure
  Theorem]\label{L1MartingaleConvergenceTheoremContinuous}Let $X_t$ be
  a cadlag martingale then the following are equivalent
\begin{itemize}
\item[(i)]$X_t$ is uniformly
  integrable
\item[(ii)]there exists an integrable $X$  such that
  $X_t \tolp{1} X$
\item[(iii)]there exists an integrable $X$ such that
  $X_t = \cexpectationlong{\mathcal{F}_t}{X}$ almost surely.
\end{itemize}
\end{thm}
\begin{proof}
Given the result Theorem
\ref{MartingaleConvergenceBoundedL1Continuous}, the proof is
essentially identical to the discrete time case.  
To see (i) implies (ii) we know from Lemma
\ref{UniformIntegrabilityProperties} that $X_t$ uniformly integrable
implies $L^1$ boundedness, hence we can apply Theorem
\ref{MartingaleConvergenceBoundedL1Continuous}
to conclude the existence of an integrable $X$ such that $X_t \toas
X$.  However almost sure convergence implies convergence in
probability (Lemma \ref{ConvergenceAlmostSureImpliesInProbability})
which together with uniform integrability implies $X_t \tolp{1} X$
(Lemma \ref{LpConvergenceUniformIntegrability}).

To see that (ii) implies (iii) from $X_t \tolp{1} X$ we get for any
$\sigma$-algebra $\mathcal{G}$,
\begin{align*}
\lim_{t \to \infty}
\norm{\cexpectationlong{\mathcal{G}}{X_t}
    -\cexpectationlong{\mathcal{G}}{X} }_1 &\leq
\lim_{t \to \infty}\expectation{\cexpectationlong{\mathcal{G}}{\abs{
      X_t- X}}} = \lim_{t \to \infty}\expectation{\abs{X_t -X}} = 0
\end{align*}
and therefore for any fixed $s \geq 0$ and the martingale property
$X_s = \cexpectationlong{\mathcal{F}_s}{X_t}$ a.s. we have
\begin{align*}
\norm{X_s - \cexpectationlong{\mathcal{F}_s}{X}}_1 &\leq \lim_{t \to
  \infty} \norm{X_s -
  \cexpectationlong{\mathcal{F}_s}{X_t}}_1  + \lim_{t \to
  \infty} \norm{\cexpectationlong{\mathcal{F}_s}{X_t}-
  \cexpectationlong{\mathcal{F}_s}{X}}_1 \\
&= \lim_{t \to
  \infty} \norm{\cexpectationlong{\mathcal{F}_s}{X_t}-
  \cexpectationlong{\mathcal{F}_s}{X}}_1=0
\end{align*}
and we get that $X_s = \cexpectationlong{\mathcal{F}_s}{X}$ a.s.

To see that (ii) implies (iii), we simply invoke Corollary \ref{ConditionalExpectationsUniformlyIntegrable}.
\end{proof}

\begin{thm}\label{OptionalSamplingContinuous}Let $X_t$ be an $\mathcal{F}$-submartingale on $\reals_+$
  with $X_t$ and $\mathcal{F}$ right continuous, let $\sigma$ and
  $\tau$ be optional times with $\tau$ bounded, then $X_\tau$ is
  integrable and
\begin{align*}
\cexpectationlong{\mathcal{F}_\sigma}{X_\tau} &\leq X_{\tau \wedge
  \sigma} \text{ a.s.}
\end{align*}
\end{thm}
\begin{proof}
For each $n>0$, restrict $X$ to the dyadic rationals $X_{k/2^n}$ on the
filtration $\mathcal{F}_{k/2^n}$.  Is is immediate that this is a
discrete submartingale.

Define the discrete approximations of optional times $\tau_n =
\frac{1}{2^n} \floor{2^n \tau + 1}$ and $\sigma_n = \frac{1}{2^n}
\floor{2^n \sigma + 1}$ so that $\tau_n$ and $\sigma_n$ are optional
times such that $\tau_n \downarrow \tau$ and $\sigma_n \downarrow
\sigma$ (Lemma \ref{DiscreteApproximationOptionalTimes}) and
furthermore $\mathcal{F}_\sigma = \cap_n \mathcal{F}_{\sigma_n}$ (Lemma
\ref{InfSupStoppedFiltration} and right
continuity of $\mathcal{F}$).
We can now apply the Optional Sampling Theorem Corollary
\ref{OptionalSamplingSubmartingaleDiscrete} to conclude that for each
$m,n>0$,
\begin{align*}
\cexpectationlong{\mathcal{F}_{\sigma_m}}{X_{\tau_n}} &\geq X_{\tau_n \wedge
  \sigma_m} \text{ a.s.}
\end{align*}
Now holding $n$ fixed we note that since $\sigma_m$ is decreasing in
$m$ we have $\mathcal{F}_{\sigma_1} \supset
\mathcal{F}_{\sigma_{2}} \supset \cdots$ and therefore we can apply
the downward Levy-Jessen Theorem \ref {JessenConditioningLimits} and
right continuity of $X_t$ to
conclude 
\begin{align*}
\cexpectationlong{\mathcal{F}_{\sigma}}{X_{\tau_n}} &=\lim_{m \to
  \infty} \cexpectationlong{\mathcal{F}_{\sigma_m}}{X_{\tau_n}}  \geq
\lim_{m \to \infty} X_{\tau_n \wedge  \sigma_m} 
= X_{\tau_n \wedge \sigma} \text{ a.s.}
\end{align*}

Now we need to justify taking the limit $n \to \infty$.  To do this,
we claim that the sequence of random variable $X_{\tau_n}$ 
is a backward submartingale; that is to say if we consider
$X_{\tau_{-n}}$ and the filtration
$\mathcal{F}_{\tau_{-n}}$ for every $n < 0$
then $X_{\tau_{-n}}$ is an $\mathcal{F}_{\tau_{-n}}$-submartingale on
$\integers_-$.  The fact that $X_{\tau_{-n}}$ is a submartingale
follows from the fact that $\tau_n$ is decreasing and Optional
Sampling (Corollary \ref{OptionalSamplingSubmartingaleDiscrete})
together
with our awkward indexing
\begin{align*}
\cexpectationlong{\mathcal{F}_{\tau_{-(n-1)}}}{X_{\tau_{-n}}} &\geq
X_{\tau_{-(n-1)}} \text{ a.s.}
\end{align*}
Furthermore we can show that $\expectation{X_{\tau_n}}$ is bounded
because $\tau$ is bounded.
If we pick $T > 0$ such that $0 \leq \tau \leq T$ then we have an upper bound
\begin{align*}
\expectation{X_{\tau_{n}}}
&=\expectation{\cexpectationlong{\mathcal{F}_{\tau_n}}{X_{T}}} =
\expectation{X_T} < \infty
\end{align*}
and a lower bound from 
\begin{align*}
-\infty &< \expectation{X_0} \leq
\expectation{\cexpectationlong{\mathcal{F}_0}{X_{\tau_n}}} = \expectation{X_{\tau_n}}
\end{align*}
By Lemma \ref{BackwardSubmartingaleBoundedUniformlyIntegrable} we can
now conclude that $X_{\tau_{n}}$ is uniformly integrable.
So now we pick $A \in \mathcal{F}_{\sigma}$ and by right continuity of
$X_t$ we have
\begin{align*}
\lim_{n \to \infty} X_{\tau_n} \characteristic{A} = X_\tau
\characteristic{A} \text{ a.s.}
\end{align*}
and
\begin{align*}
\lim_{n \to \infty} X_{\tau_n \wedge \sigma} \characteristic{A} =
X_{\tau \wedge \sigma}
\characteristic{A} \text{ a.s.}
\end{align*}
and therefore by uniform integrability and Lemma
\ref{LpConvergenceUniformIntegrability} we get
\begin{align*}
\expectation{X_\tau ; A} &= \lim_{n \to \infty}
\expectation{X_{\tau_n} ; A} \geq 
\lim_{n \to \infty} \expectation{X_{\tau_n \wedge \sigma} ; A} =
\expectation{X_{\tau \wedge \sigma} ; A}
\end{align*}
which shows $\cexpectationlong{\mathcal{F}_\sigma}{X_\tau} \geq
X_{\tau \wedge \sigma}$ a.s. by the defining property and monotonicity
of conditional expectation.
\end{proof}

\subsection{Progressive Measurability}
For many applications the notion of an adapted process
suffices.  However when dealing with continuous time processes there
are anomalies that can occur with such processes that are inconvenient
and it is best to define a stronger notion of measurability.  To
understand the issue we're trying to address, note that adaptedness only addresses the behavior of
$X_t(\omega)$ as a function of $\omega$ for fixed $t$.  If we take the
sample path point of view and think of $X_t(\omega)$ as a function of
$t$ for fixed $\omega$ then there little constraint on how horribly it
can behave.  In fact the general definition of a process allows $T$ to
be an arbitrary set so it isn't even in scope to talk about an type of
regularity of sample paths.  

For the special case of processes indexed by $\reals_+$ we can discuss
measurability, continuity and even differentiability of sample paths.
For the moment, there is a
very mild restriction that we make.

\begin{defn}Let $(\Omega, \mathcal{A})$, $(S, \mathcal{S})$  and
  $(T, \mathcal{T})$  be measurable spaces.  A process $X$ on $T$ with
  values in $S$ is said to be \emph{jointly
    measurable} or simply \emph{measurable} if $X : \Omega \times T
  \to S$ is $\mathcal{A} \otimes \mathcal{T}/\mathcal{S}$ measurable.
\end{defn}


\begin{lem}\label{JointMeasurabilityOfSampleContinuous}Let $S$ be a metric space and let $T$ be a separable metric
  space both given the Borel $\sigma$-algebra.  Suppose a process $X$
  on $T$ with values in $S$ has continuous sample paths, then $X$ is
  jointly measurable.
\end{lem}
\begin{proof}
Let $\lbrace t_n \rbrace$ be a countable dense set of points in $T$.
We use this dense set to provide a sequence of approximations to $X$.
To this end, for each $n\geq 1$ and $k \geq 1$ define
\begin{align*}
B_{n,k} &= \lbrace t \in T \mid d(t, t_k) < 1/n \rbrace \\
V_{n,k} &= B_{n,k} \setminus \cup_{j=1}^{k-1} B_{n,j}
\end{align*}
Clearly the $V_{n,k}$ are Borel measurable since the $B_{n,k}$ are open.
By construction they are disjoint and by density of the $t_k$ they
cover $T$.  Define
\begin{align*}
X^n_t (\omega) &= X_{t_k}(\omega) \text{ for $t \in V_{n,k}$}
\end{align*}
Since for any $A \in \mathcal{B}(S)$ we have $\lbrace (\omega, t) \mid
X^n_{t}(\omega) \in A \rbrace = \cup_{k=1}^\infty V_{n,k} \times \lbrace
X_{t_k} \in A \rbrace$ we see that $X^n$ is jointly measurable.

Now by density of $X^n$ and the continuity of sample paths of $X$ we
have $\lim_{n \to \infty} X^n = X$ and joint measurability of $X$
follows from Lemma \ref{LimitsOfMeasurableMetricSpace}.
\end{proof}

\begin{defn}A process $X$ is said to be \emph{progressively
    measurable} or simply \emph{progressive} if for every $t$, the restriction of $X$ to the time
  interval $[0,t]$, $X : \Omega \times [0,t] \to S$ is
  $\mathcal{F}_t \otimes \mathcal{B}([0,t])$ measurable.
\end{defn}

\begin{defn}The set of \emph{progressively measurable sets} is
  defined as
\begin{align*}
\mathcal{P}\mathcal{M} = \lbrace A \subset \Omega
  \times \reals_+ \mid A \cap \Omega \times [0,t] \in \mathcal{F}_t
  \otimes \mathcal{B}([0,t]) \text{ for all } t \geq 0\rbrace
\end{align*}
\end{defn}

\begin{lem}\label{ProgressivelyMeasurableProcesses}The set $\mathcal{P}\mathcal{M}$ is a sub $\sigma$-algebra
  of $\mathcal{A} \otimes \mathcal{B}(\reals_+)$.  A process $X :
  \Omega \times \reals_+ \to S$ is progressive if and only if $X$ is
  $\mathcal{P}\mathcal{M}$-measurable, in particular a progressive
  process is jointly measurable.
\end{lem}
\begin{proof}
Since for all $t\geq 0$, $\Omega \times \reals_+ \cap \Omega \times
[0,t] = \Omega \times [0,t] \in \mathcal{F}_t  \otimes
\mathcal{B}([0,t])$ we have $\Omega \times \reals_+ \in
\mathcal{P}\mathcal{M}$.  Suppose $A \in \mathcal{P}\mathcal{M}$ and
then note by the elementary set theory equality $B^c \cap C = (B \cap
C)^c \cap C$ and the fact that $\mathcal{F}_t \otimes
\mathcal{B}([0,t])$ is a $\sigma$-algebra
\begin{align*}
A^c \cap \Omega \times [0,t] &= (A \cap \Omega \times [0,t])^c \cap
\Omega \times [0,t] \in \mathcal{F}_t \otimes \mathcal{B}([0,t])
\end{align*}
thus showing $\mathcal{P}\mathcal{M}$ is closed under set complement.
Lastly if we assume that $A_1, A_2, \dots \in \mathcal{P}\mathcal{M}$,
then clearly  for every $t \geq 0$,
\begin{align*}
\left (\cap_n A_n \right )\cap \Omega \times [0,t] &= \cap_n \left (
  A_n \cap \Omega \times [0,t]\right ) \in \mathcal{F}_t \otimes \mathcal{B}([0,t])
\end{align*}
so we see that $\mathcal{P}\mathcal{M}$ is a $\sigma$-algebra.

To see that $\mathcal{P}\mathcal{M}$ is a sub $\sigma$-algebra of
$\mathcal{A} \otimes \mathcal{B}(\reals_+)$, 
if for $A \in \mathcal{P}\mathcal{M}$ we define $A_n = A \cap \Omega
\times [0,n]$ then by definition of $\mathcal{P}\mathcal{M}$ we know
$A_n \in \mathcal{F}_n \otimes \mathcal{B}([0,n]) \subset \mathcal{A}
\otimes \mathcal{B}(\reals_+)$.  But we can write $A = \cup_n A_n $
thus showing $A \in \mathcal{A}
\otimes \mathcal{B}(\reals_+)$.

To see the characterization of progressive processes, assume $X$ is a
process and that $A
\in \mathcal{S}$ and observe
\begin{align*}
\lbrace X \in A \rbrace \cap \Omega \times [0,t] &= \lbrace (\omega,
s) \in \Omega \times [0,t] \mid X_s(\omega) \in A \rbrace
\end{align*}
which shows that $X$ is progressive if and only if it is $\mathcal{P}\mathcal{M}$-measurable.
\end{proof}

\begin{examp}The following is an example of a measurable adapted process that is not
  progressively measurable.  Take $\Omega=[0,1]$ and $S = \reals$ all
  supplied with the Borel $\sigma$-algebra and Lebesgue measure.
Let $A \subset [0,1]$ be non-measurable.  Define
\begin{align*}
X_t(\omega) &= \begin{cases}
t + \omega & \text{for $t \in A$} \\
-t - \omega & \text{for $t \notin A$}
\end{cases}
\end{align*}
with filtration defined by $\mathcal{F}_t = \mathcal{B}([0,1])$. 
(Note that for every $t\geq 0$, $\sigma(X_t) =
\mathcal{B}([0,1])$ hence this is the filtration induced by $X$).   It
is easy to see that this is a process (i.e. is measurable) since for
each fixed $t$, $X_t : [0,1] \to \reals$ is continuous hence measurable.
However that $\lbrace (\omega,s) \mid X_s(\omega) \geq 0\rbrace = \Omega
\times A$ hence is not measurable thus showing that $X$ is not
progressively measurable.

There is the simpler example but the current example also provides an
example of the type of anomaly that can occur.

Define a random time 
\begin{align*}
\tau(\omega) &= \inf \lbrace t \mid 2t \geq \abs{X_t(\omega)} \rbrace = \inf \lbrace t
\mid 2t \geq t + \omega \rbrace = \inf \lbrace t
\mid t \geq \omega \rbrace = \omega
\end{align*}
which because $\lbrace \tau \leq t \rbrace = [0,t] \in
\mathcal{B}([0,1])$ is seen to be an optional time.
Because $\mathcal{F}_t = \mathcal{B}([0,1])$ we see that for every
Borel measurable $A$, $A \cap \lbrace \tau \leq t \rbrace = A \cap
[0,t] \in \mathcal{F}_t$ so we also have $\mathcal{F}_\tau =
\mathcal{B}([0,1])$.  On the other hand, the stopped process
\begin{align*}
X_\tau(\omega) &= \begin{cases}
2\omega & \text{if $\omega \in A$} \\
-2\omega & \text{if $\omega \notin A$} \\
\end{cases}
\end{align*}
and again we see that $\lbrace X_\tau > 0 \rbrace = A$  is not
$\mathcal{F}_\tau$-measurable.
\end{examp}

Note that because sections are measurable (Lemma
\ref{MeasurableSections}) a progressively measurable process is
adapted.

\begin{lem}\label{ContinuityAndProgressiveMeasurability}Let $X$ be a
  process on $\reals_+$ with values in a metric space $(S,
  \mathcal{B}(S))$ adapated to the filtration
  $\mathcal{F}$.  Suppose $X$ has left or right continuous sample
  paths, then $X$ is $\mathcal{F}$-progressively measurable.
\end{lem}
\begin{proof}
The proof is analogous to Lemma
\ref{JointMeasurabilityOfSampleContinuous}.  We give the proof for
right continuous sample paths with the case of left continuous sample
paths being very similar.

Let $t \geq 0$ be given $X^n$ be the process on $[0,t]$ be defined by $X^n_s(\omega) =
X_{\frac{k+1}{2^n} \wedge t}(\omega)$ for $\frac{k}{2^n} < s \leq \frac{k+1}{2^n} \wedge t$.
It is clear that for any $A \in \mathcal{B}(S)$ we have
\begin{align*}
&\lbrace (\omega, s) \mid 0 \leq s \leq t ; X^n_s(\omega) \in A \rbrace \\
&= \bigcup_{k=0}^{\floor{2^n t}} \lbrace (\omega, s) \mid 0 \leq s \leq t ; \frac{k}{2^n} < s \leq \frac{k+1}{2^n}
\wedge t ;X_{\frac{k+1}{2^n} \wedge t}(\omega) \in A \rbrace \\
&=\bigcup_{k=0}^{\floor{2^n t}} \lbrace X_{\frac{k+1}{2^n} \wedge t}(\omega) \in A \rbrace
\times (\frac{k}{2^n} < s \leq \frac{k+1}{2^n}] \in \mathcal{F}_t \otimes \mathcal{B}[0,t]\\
\end{align*}
which shows that $X^n$ is progressively measurable.  By right
continuity, we see that $\lim_{n \to \infty} X^n = X\mid_{\Omega
  \times [0,t]}$ and therefore by Lemma
  \ref{LimitsOfMeasurableMetricSpace} we have $X$ is progressively measurable.
\end{proof}

\begin{lem}\label{StoppedProgressivelyMeasurableProcess}Let $X$ be an $\mathcal{F}$-progressively measurable process on
  $\reals_+$ with values in a measurable space $(S, \mathcal{S})$ and
  let $\tau$ be an $\mathcal{F}$-optional time, then $X_\tau$ is
  $\mathcal{F}_\tau$-measurable.  Moreover, the stopped process $X^\tau$ is
$\mathcal{F}$-progressively measurable, in particular $X_{\tau \wedge
  t}$ is  $\mathcal{F}_t$ measurable for all $t \geq 0$.
\end{lem}
\begin{proof}
We first claim that if we can prove $X_{\tau \wedge t}$ is
$\mathcal{F}_t$-measurable for all $t \geq 0$ then it follows that
$X_\tau$ is $\mathcal{F}_\tau$-measurable.  This follows from picking
a measurable set $A \in \mathcal{S}$ and noting that
\begin{align*}
\lbrace \tau \leq t \rbrace \cap \lbrace X_\tau \in A \rbrace &=
\lbrace \tau \leq t \rbrace \cap \lbrace X_{\tau \wedge t} \in A \rbrace
\end{align*}
which is $\mathcal{F}_t$ since $\tau$ is $\mathcal{F}$-optional and we
have assumed $\lbrace X_{\tau \wedge t} \in A \rbrace \in
\mathcal{F}_t$.

To see that $X^{\tau}$ is an $\mathcal{F}$-progressively
measurable process, pick a $t \geq 0$ and consider the restriction of
$X^{\tau}$ to $\Omega \times [0,t]$.   Note that by replacing $\tau$
with $\tau \wedge t$, we can assume that $\tau \leq t$ which implies
$\tau$ is $\mathcal{F}_t$-measurable (to see this note that for $s
\leq t$, $\lbrace \tau \leq s \rbrace \in \mathcal{F}_s \subset
\mathcal{F}_t$ and for $s > t$,
$\lbrace \tau \leq s \rbrace = \Omega$). 
Now we can factor the restriction of
$X_{\tau \wedge t}$ to $\Omega \times [0,t]$ as $X^{\tau} =
X\mid_{\Omega \times [0,t]} \circ T^t$ where
$T^t : \Omega \times [0,t] \to \Omega \times [0,t]$ is defined by
$T^t(\omega, s) = (\omega, \tau(\omega) \wedge s)$.   We claim that $T^t$ is
$\mathcal{F}_t \otimes \mathcal{B}([0,t])$-measurable.  This follows
from the $\mathcal{F}_t \otimes \mathcal{B}([0,t])$-measurability of
$(\omega, s) \mapsto \tau(\omega) \wedge s$ which follows by noting
that for every $0 \leq u \leq t$,
\begin{align*}
\lbrace \tau \wedge s \leq u \rbrace &= \lbrace \tau \leq u \rbrace
\times [0,t] \cup \Omega \times [0,u] \in \mathcal{F}_t \otimes \mathcal{B}([0,t])
\end{align*}
As $X\mid_{\Omega \times [0,t]}$ is $\mathcal{F}_t \otimes
\mathcal{B}([0,t])/\mathcal{S}$-measurable by progressive
measurability of $X$, the claim follows from Lemma
\ref{CompositionOfMeasurable}.  The fact that $X_{\tau \wedge t}$ is
$\mathcal{F}_t$-measurable for all $t \geq 0$ follows from the fact
that progressive measurability implies adaptedness.
\end{proof}

\chapter{Concentration Inequalities}
\begin{lem}[Markov Inequality]\label{MarkovInequality}Let $\xi$ be a positive integrable random variable.  Then $\probability{\xi>t} \leq \frac{E(\xi)}{t}$\end{lem}
\begin{proof}
$E(\xi) \leq E(\xi\characteristic{\{\xi>t\}}) \leq E(t\characteristic{\{\xi>t\}})=t\probability{\xi>t}$
\qedhere
\end{proof}

\begin{lem}[Chebeshev's Inequality]\label{ChebInequality}Let $\xi$ be a random variable with finite mean $\mu$ and finite variance $\sigma$.  Then $\probability{|\xi-\mu|>t} \leq \frac{\sigma^2}{t^2}$\end{lem}
\begin{proof}
$\probability{|\xi-\mu|>t} = \probability{(\xi-\mu)^2 > t^2} \leq \frac{\expectation{(\xi-\mu)^2}}{t^2}=\frac{\sigma^2}{t^2}$
\qedhere
\end{proof}

\begin{lem}[One Sided Chebeshev's Inequality]\label{OneSidedChebInequality}Let $\xi$ be a random variable with finite mean $\mu$ and
  finite variance $\sigma$.  Then $\probability{\xi-\mu>\lambda} \leq
  \frac{\sigma^2}{\sigma^2 + \lambda^2}$\end{lem}
\begin{proof}
First we assume $\expectation{\xi}=0$.  We prove a family of
inequalities for a real parameter $c > 0$.
\begin{align*}
\probability{\xi>\lambda}  
&= \probability{\xi + c > \lambda + c} \\
& \leq \probability{(\xi + c)^2 > (\lambda + c)^2} & \textrm{ because }
\lambda+c>0 \\
& \leq \frac{\expectation{\xi^2} + c^2}{(\lambda + c)^2}
\end{align*}
Now we extract the best estimate by finding the minimum of the right
hand side with respect to $c$.  Differentiating we get a vanishing
first derivative when
$
(\lambda^2 + c^2) 2c = (\expectation{\xi^2} + c^2) 2(\lambda +
c)$.  Divide by $2(\lambda+ c)$  and subtract $c^2$ to get the
  minimum at $c=\expectation{\xi}/\lambda > 0$.  Plug this value in to
  get the final estimate.
\begin{align*}
\frac{
  \expectation{\xi^2} +
  (\frac{\expectation{\xi^2}}{\lambda})^2
}
{(\lambda +
    \frac{\expectation{\xi^2}}{\lambda})^2
} & =
    \frac{\expectation{\xi^2}(1 +
      \frac{\expectation{\xi^2}}{\lambda^2})}
{
  \lambda^2 (1 + \frac{\expectation{\xi^2}}{\lambda^2}
)^2
} \\
& = \frac{\expectation{\xi^2}}{\lambda^2 + \expectation{\xi^2}}
\end{align*}
Now apply the above inequality to the centered random variable $\xi -
\mu$ to get the general result.
\qedhere
\end{proof}

\begin{defn}
We say that a random variable $\xi$ is \emph{subgaussian} if and only if
there exist constants $c, C > 0$ such that $\probability { \abs{\xi}
  \geq \lambda} \leq C e^{-c \lambda^2}$ for all $\lambda > 0$.
\end{defn}

TODO: Show that any Gaussian is subgaussian (independent of its mean?).

TODO: Show any bounded (or almost surely bounded) random variable is
subgaussian.

\begin{examp}Given the nomenclature it isn't surprising that Gaussian
  random variables are subgaussian.  As it turns out it is useful to
  analyze the case of a $N(0,\sigma^2)$ random variable separately
  since it has slightly different behavior than the general $N(\mu,
  \sigma^2)$ case.  Let us assume that $\xi$ is a normal random
  variable with mean $0$ and variance $\sigma^2$.  We have a standard tail estimate for $\lambda \geq \sigma$
\begin{align*}
\probability{\xi \geq \lambda} &= 
\frac{1}{\sqrt{2\pi}\sigma} \int_\lambda^\infty e^{-x^2/2\sigma^2} \, dx \leq 
\frac{1}{\sqrt{2\pi}\sigma} \int_\lambda^\infty \frac{x}{\sigma} e^{-x^2/2\sigma^2} \, dx =
\frac{1}{\sqrt{2\pi}} e^{-\lambda^2/2\sigma^2}
\end{align*}
The $0 \leq \lambda \leq \sigma$ case can easily be handled with a constant
multiplier but we can actually find the constant that gives a tight
bound.  Note that $\frac{1}{\sqrt{2\pi}\sigma} \int_0^\infty e^{-x^2/2\sigma^2} =
\frac{1}{2}$ so we can't do any better than $\probability{\xi \geq
  \lambda} \leq \frac{1}{2} e^{-\lambda^2/2\sigma^2}$; in fact this bound
works for all $\lambda \geq 0$.  We've already shown this for $\lambda
\geq 1$ and $\lambda=0$.  To show the bound on $[0,1]$ we calculate
the derivative 
\begin{align*}
\frac{d}{d\lambda} \left( \frac{1}{2} e^{-\lambda^2/2\sigma^2} -
  \frac{1}{\sqrt{2\pi}\sigma} \int_\lambda^\infty e^{-x^2/2\sigma^2} \, dx \right)
&=\left( -\frac{\lambda}{2\sigma^2} + \frac{1}{\sqrt{2\pi}\sigma}\right) e^{-\lambda^2/2}
\end{align*}
from which we conclude there is a unique maximum of the function at
$\lambda=\sigma \sqrt{\frac{2}{\pi}} \in (0,\sigma)$.  We have already validated
that the function is nonnegative at the endpoints of $[0,\sigma]$ so it must
be nonnegative on the entire interval.  Now by symmetry of $\xi$, the
calculation also shows that $\probability{\xi \leq -\lambda} \leq
\frac{1}{2}e^{-\lambda^2/2\sigma^2}$ and therefore $\probability{\abs{\xi}
  \geq \lambda} \leq e^{-\lambda^2/2\sigma^2}$.

Now for a general $N(\mu, \sigma)$ normal random variable $\xi$ we
have by change of variables
\begin{align*}
\probability{\xi \geq \lambda} &= 
\frac{1}{\sqrt{2\pi} \sigma} \int_\lambda^\infty e^{-(x-\mu)^2/2\sigma^2} \, dx = 
\frac{1}{\sqrt{2\pi}} \int_{(\lambda-\mu)/\sigma}^\infty e^{-x^2/2} \,
dx \leq \frac{1}{\sqrt{2\pi}} e^{-(\lambda-\mu)^2/2\sigma^2}
\end{align*}

TODO: Finish
\end{examp}

\begin{lem}Let $\{\xi_i\}_{i=1}^m$ be jointly independent subgaussian random variables.  Then $\expectation{e^{\sum_{i=1}^m \xi}} =
  \prod_{i=1}^m \expectation{e^{\xi_i}}$.
\end{lem}
\begin{proof}
First show that for a subgaussian $\xi$, we have by dominated
convergence the Taylor expansion 
$$\expectation{e^{t\xi}} = 1 + \sum_{k=1}^\infty
\frac{t^k}{k!}\expectation{\xi^k}$$
The proof of this fact is to exhibit an integrable function that
dominates the sequence of partial sums $1+\sum_{k=1}^n
\frac{t^k\xi^k}{k!}$.  This is obvious if $\xi$ is almost surely
bounded but it's not obvious to me that this should be true for a
subgaussian $\xi$.  TODO: Perhaps we need to use uniform integrability or
something like that in the subgaussian/subexponential case.

In any case, assuming the validity of the above identity for each
$\xi$, we turn to the case of the sum.
\end{proof}

\begin{lem}\label{SubgaussianEquivalence}$\xi$ is subgaussian if and only if there exists $C$ such
  that $\expectation{e^{t\xi}} \leq C e^{Ct^2}$ and if and only if
    there exists $C$ such that $\expectation{|\xi|^k} \leq
    \left(Ck\right)^{\frac{k}{2}}$ for all $t \in \reals$.
\end{lem}
\begin{proof}
Suppose $\xi$ is subgaussian and calculate:
\begin{align*}
\expectation{e^{t\xi}} &= \int_0^\infty \probability{e^{t\xi} \geq
    \lambda} d\lambda 
= \int_{-\infty}^\infty \probability{e^{t\xi} \geq e^{t\eta}} t
e^{t\eta} d\eta \\
&= \int_{-\infty}^\infty \probability{\xi \geq \eta} t
e^{t\eta} d\eta 
\leq \int_{-\infty}^\infty C t e^{t\eta - c\eta^2} d\eta  
= C t e^{\frac{t^2}{4c}}\int_{-\infty}^\infty 
e^{-\left(\sqrt{c}\eta - \frac{t}{2\sqrt{c}}\right)^2} d\eta \\
&= C^\prime t e^{\frac{t^2}{4c}} 
\leq C^\prime e^{\frac{5c t^2}{4c}} 
\end{align*}

Now assume that we have $\expectation{e^{t\xi}} \leq C e^{Ct^2}$ for
all $t$.  Pick an arbitrary $t>0$ to be
chosen later and proceed by using first order
moment method:
\begin{align*}
\probability{\xi \geq \lambda} &= \probability{e^{t\xi} \geq
  e^{t\lambda}} 
\leq \frac{\expectation{e^{t\xi}}}{e^{t\lambda}} 
\leq C e^{Ct^2 - t\lambda} \\
\end{align*}
Now we pick $t$ to minimize the upper bound derived above; simple
calculus shows this occcurs at $t=\frac{\lambda}{2C}$.  Subtituting
yields the bound 
\begin{align*}
\probability{\xi \geq \lambda} \leq C e ^ {-\frac{\lambda^2}{4C}}
\end{align*}
For the other tail, we note that our assumption holds equally well for
$-\xi$.  Thus we can use the same method to bound 
\begin{align*}
\probability{\xi \leq -\lambda} = \probability{-\xi \geq \lambda} \leq C e ^ {-\frac{\lambda^2}{4C}}
\end{align*}
therefore taking the union bound we get
\begin{align*}
\probability{|\xi| \geq \lambda} \leq 2 C e ^ {-\frac{\lambda^2}{4C}}
\end{align*}

Now consider absolute moments of subgaussian variables.  We can assume
that $\xi \geq 0$ and calculate as before:
\begin{align*}
\expectation{\xi^k} 
&= \int_0^\infty \probability{\xi^k \geq x} dx 
= k\int_0^\infty \probability{\xi^k \geq y^k} y^{k-1} dy \\
&= k C \int_0^\infty y^{k-1} e^{-cy^2} dy 
= k C \frac{c^{k-3}}{2} \int_0^\infty x^{\frac{k}{2}-1} e^{-x} dx \\
&= k C \frac{c^{k-3}}{2} \Gamma\left(\frac{k}{2}\right) 
\leq k C \frac{c^{k-3}}{2} \left(\frac{k}{2}\right)^\frac{k}{2} \\
\end{align*}
To go the other direction, assume $\expectation{|\xi|^k} \leq
(Ck)^\frac{k}{2}$ and pick a constant $0 < c < \frac{e}{2C}$ 
\begin{align*}
\expectation{e^{K\xi^2}} &= 1 + \sum_{k=1}^\infty
\frac{t^k\expectation{\xi^{2k}}}{k!}\\
&\leq 1 + \sum_{k=1}^\infty
\frac{(2tCk)^k}{k!}\\
&\leq 1 + \sum_{k=1}^\infty
\left(\frac{2tC}{e}\right)^k < \infty \\
\end{align*}
Now use the elementatry bound $ab \leq \frac{(a^2 + b^2)}{2}$ so see
\begin{align*}
\expectation{e^{t\xi}} &\leq 
\end{align*}
\end{proof}

The definition of subgaussian random variables differs in a minor way
from another in common use in the literature.  In particular, in some descriptions a random
variable $\xi$ is called subgaussian if and only if
$\expectation{e^{t\xi}} \leq e^{\frac{c^2 t^2}{2}}$ for all $t \in \reals$.  The important
difference here compared with the characterization in Lemma
\ref{SubgaussianEquivalence} is that the constant on the right hand side is $1$.
With this defintion, we must add the hypothesis $\expectation{\xi}=0$
to get equivalence with the other definition.

\begin{lem}Suppose $\xi$ is a random variable such that there exists
  $c>0$ for which
\begin{align*}
\expectation{e^{t\xi}} &\leq e^{\frac{c^2 t^2}{2}} \text{ for all $t \in \reals$}
\end{align*}
then $\expectation{\xi}=0$ and $\expectation{\xi^2} \leq c^2$.
\end{lem}
\begin{proof}
By Dominated Convergence and the hypothesis we get
\begin{align*}
\sum_{n=0}^\infty \frac{t^n}{n!}\expectation{\xi^n} &=
\expectation{e^{t\xi}} \leq e^{\frac{c^2 t^2}{2}} = \sum_{n=0}^\infty
\frac{c^{2n}}{2^n n!}t^{2n}
\end{align*}
so in particular by taking only terms up to order $t^2$ and using the
fact that the constant term in on both sides is $1$, we have
\begin{align*}
t \expectation{\xi} + \frac{t^2}{2} \expectation{\xi^2} &= \frac{c^2 t^2}{2}
+
o(t^2) \text { as $t \to 0$}
\end{align*}
If we divide both sides by $t > 0$ and take the limit as $t \to 0^+$ then we
get $\expectation{\xi} \leq 0$.  If we divide by $t < 0$ and take the
limit as $t \to 0^-$ then we get $\expectation{\xi} \geq 0$.  Thus we
can conclude $\expectation{\xi} = 0$.  If we plug that in and divide
by $t^2$ and take the limit as $t \to 0$ then see $\expectation{\xi^2}
\leq c^2$.
\end{proof}
Note that the argument in the proof above doesn't even get off the
ground unless the constant of the bounding exponential is assumed to
be $1$.

The following lemma is useful for the second moment method for
deriving tail bounds.
\begin{lem}Let $\{\xi_i\}_{i=1}^m$ be pairwise independent random
  variables and $c_i$ be scalars.  Then $\variance{\sum_{i=1}^m c_i \xi} =
  \sum_{i=1}^m |c_i|^2\variance{\xi_i}$.
\end{lem}
\begin{proof}
TODO
\end{proof}

\begin{lem}[Bennett's Inequality]\label{Bennett} Let $\{\xi_i\}_{i=1}^m$ be independent random variables with means $\mu_i$ and variances $\sigma_i$.  Set $\Sigma^2 = \sum_{i=1}^m \sigma_i^2$.  If for every $i$, $|\xi_i - \mu_i| \leq M$ almost everywhere then for every $\lambda > 0$ we have $$
\probability{\sum_{i=1}^m [\xi_i - \mu_i] > \lambda} \leq 
e^{
	-\frac{\lambda}{M}\{(1 + \frac{\Sigma^2}{M\lambda})\log(1+\frac{M\lambda}{\Sigma^2}) - 1\}
}
$$
\end{lem}
\begin{proof}
First it is easy to see that by subtracting means we may assume that
$\mu_i=0$.  Then we have $\sigma_i = \expectation{\xi_i^2}$.  We use
the exponential moment method.  We show a family of inequalities
depending on a real parameter $c > 0$ which we will pick later.  First we have 
\begin{align*}
\probability{\sum_{i=1}^m \xi_i > \lambda} & =
\probability{c\sum_{i=1}^m \xi_i > c\lambda}  && \textrm{since } c >0.\\
  & =\probability{e^{c\sum_{i=1}^m \xi_i} > e^{c\lambda}} &&
  \textrm{since } e^x \textrm{ is increasing}\\
  & \leq e^{-c\lambda} \expectation{e^{c\sum_{i=1}^m \xi_i}} && \textrm
  {by Markov's Inequality}\eqref{MarkovInequality}\\
  & = e^{-c\lambda} \prod_{i=1}^m \expectation{e^{c\xi_i}} &&
  \textrm{by independence and boundedness.  TODO: do we really need boundedness?}
\end{align*}
Now we consider an individual term $\expectation{e^{c\xi_i}}$ for an
almost surely bounded $\xi_i$ with zero mean.
\begin{align*}
\expectation{e^{c\xi_i}} &= \expectation{\sum_{k=0}^\infty
  \frac{c^k\xi_i^k}{k!}} = \sum_{k=0}^\infty
  \frac{c^k}{k!}\expectation{\xi_i^k} && \textrm{by dominated
    convergence} \\
& = 1  + \sum_{k=2}^\infty
  \frac{c^k}{k!}\expectation{\xi_i^k} && \textrm{by mean zero} \\
& \leq 1  + \sum_{k=2}^\infty
  \frac{c^k M^{k-2} \sigma_i^2}{k!} && \textrm{by boundedness and
    definition of variance} \\
& \leq e^{\sum_{k=2}^\infty  \frac{c^k M^{k-2} \sigma_i^2}{k!}} &&
\textrm{since } 1+x \leq e^x \eqref{BasicExponentialInequalities} \\
& = e^{\frac{(e^{cM} -1 - cM)\sigma_i^2}{M^2}}
\end{align*}
Therefore,
\begin{align*}
\probability{\sum_{i=1}^m \xi_i > \lambda} &\leq e^{-c\lambda} \prod_{i=1}^m
e^{\frac{(e^{cM} -1 - cM)\sigma_i^2}{M^2}} \\
& = e^{\frac{(e^{cM} -1 - cM)\Sigma^2}{M^2}}
\end{align*}
Now we pick $c>0$ to minimize the bound above ($e^{cM} -1 =
\frac{M\lambda}{\Sigma^2}$ or equivalently $c = \frac{1}{M}\ln(1 + \frac{M\lambda}{\Sigma^2})$).
Substituting yields the final bound
\begin{align*}
\probability{\sum_{i=1}^m \xi_i > \lambda} &\leq e^{-(\lambda + \frac{\Sigma^2}{M})
  \frac{1}{M}\ln(1 + \frac{M\lambda}{\Sigma^2}) + \frac{\lambda}{M}}
\\
& = e^{-\frac{\lambda}{M} \{
  (1+\frac{\Sigma^2}{\lambda M}) \ln(1 + \frac{M\lambda}{\Sigma^2}) -1
\}}
\end{align*}
\end{proof}

\begin{lem}[Bernstein's or Chernoff's Inequality]\label{Bernstein} Let
  $\{\xi_i\}_{i=1}^m$ be independent random variables with means
  $\mu_i$ and variances $\sigma_i$.  Set $\Sigma^2 = \sum_{i=1}^m
  \sigma_i^2$.  If for every $i$, $|\xi_i - \mu_i| \leq M$ almost
  everywhere then for every $\lambda > 0$ we have 
\begin{align*}
\probability{\sum_{i=1}^m [\xi_i - \mu_i] > \lambda} &\leq 
e^{
	-\{\frac{\lambda^2}{2(\Sigma^2 + \frac{1}{3}M\lambda)}\}
}
\end{align*}
\end{lem}
\begin{proof}
TODO
\end{proof}

The next inequality has a pleasing form because the resulting bound is
of the form of a Gaussian random variable.  Such bounds are
interesting enough that they warrant the following definition.
\begin{defn} Let $\xi$ be a real valued random variable with mean
  $\mu$.  We say that $\xi$ has a \emph{subgaussian upper tail} if there
  exists a  constants $C > 0$ and $c > 0$ such that for all $\lambda > 0$,
$$
\probability{[\xi-\mu] > \lambda} \leq Ce^{-c\lambda^2}.
$$
We say that $\xi$ has a \emph{subgaussian tail up to} $\lambda_0$ if the
above bound holds for $\lambda < \lambda_0$.  We say that $\xi$ has a
\emph{subgaussian tail} if both $\xi$ and $-\xi$ have subgaussian upper
tails (or equivalently if $|\xi|$ has a subgaussian tail.
\end{defn}

 The boundedness assumption on the individual random variables in the
above sums can be relaxed to an assumption that the individual random
variables has subgaussian tails.  Moreover, one can generalize the sum
of random variables to an arbitrary linear combination of random
variables on the unit sphere.

\begin{lem}\label{Matousek} Let $\{\xi_i\}_{i=1}^m$ be independent
  random variables with $E[\xi_i]=0$ and $E[\xi_i^2]=1$ and uniform
  subgaussian tails.  Let $\{\alpha_i\}_{i=1}^m$ be real coefficients
satisfying $\sum_{i=1}^m \alpha_i^2 = 1$.  The then random variable
$\eta=\sum_{i=1}^m \alpha_i\xi_i$ has $E[\eta]=0$, $E[\eta^2]=1$ and a
subgaussian tail.
\end{lem}
\begin{proof}
TODO
\end{proof}

\begin{lem}[Exercise 7 Lugosi] Let $\{\xi_i\}_{i=1}^n$ be independent
  random variables with values in $[0,1]$.  Let $S_n = \sum_{i=1}^n
  \xi_i$ and let $\mu=\expectation{S_n}$.  Show that for any
  $\lambda\geq \mu$,
\begin{align*}
\probability{S_n \geq \lambda} \leq
\left(\frac{\mu}{\lambda}\right)^\lambda \left(\frac{n-\mu}{n-\lambda}\right)^{n-\lambda}.
\end{align*}
\end{lem}
\begin{proof}
Use Chernoff bounding.  Looking at the solution, we can pattern match
that we may want to use the convexity of $e^x$ since the solution
seems to reference the endpoints of the interval $[0,n]$; indeed that
is the way to proceed.  TODO: convert the argument below for $n=1$ to
cover general $n$.
To estimate $\expectation{e^{s\xi_i}}$ we first use convexity of
$e^sx$ on the interval $x\in[0,1]$,
\begin{align*}
e^{sx} \leq xe^s + (1-x)
\end{align*}
Substituting $\xi_i$ and taking expectations we get
\begin{align*}
\expectation{e^{s\xi_i}} \leq \mu_i e^s + (1-\mu_i).
\end{align*}
So now we minimize the Chernoff bound by using elementary calculus
\begin{align*}
\frac{d}{ds} \mu_i e^{s(1-\lambda)} + (1-\mu_i)e^{-s\lambda} = \mu_i
(1-\lambda) e^{s(1-\lambda)} + \lambda (1-\mu_i)e^{-s\lambda} 
\end{align*}
which equals $0$ when
$s=\ln\left(\frac{\lambda(1-\mu_i)}{\mu_i(1-\lambda)}\right)$.  This
value is positive when $\lambda \geq \mu$.
Backsubstituing this value and doing some algebra shows
\begin{align*}
e^{-s\lambda}\expectation{e^{s\xi_i}} \leq
\left(\frac{\mu_i}{\lambda}\right)^\lambda
\left(\frac{1-\mu_i}{1-\lambda}\right)^{1-\lambda}
\end{align*}
Note also an argument for a related estimate (Exercise 8) that uses bounds similar to those in Bennett can be made
as follows.  Since $\xi_i \in [0,1]$, we have that $\xi_i^k \leq
\xi_i$.  With this observation, 
\begin{align*}
\expectation{e^{s\xi_i}} &= 1 + \sum_{k=1}^\infty \frac{s^k
  \expectation{\xi_i^k}}{k!} \\
& \leq 1 + \sum_{k=1}^\infty \frac{s^k \mu_i}{k!} \\
&= 1 + \mu_i (e^s -1) \\
&\leq e^{\mu_i(e^s - 1)}
\end{align*}
Now we select $s$ to minimize the Chernoff bound $e^{\mu_i(e^s - 1)
  -s\lambda}$ which simple calculus shows happens at
$s=\ln\left(\frac{\lambda}{\mu_i}\right)$; the location of the minimum
being positive precisely when $\lambda \geq \mu_i$.  Backsubstituting
yields a bound $\left(\frac{\mu_i}{\lambda}\right)^\lambda e^{\lambda
  - \mu_i}$.
\end{proof}


\section{Likelihood Theory}
TODO:
\begin{itemize}
\item[(i)] Definition of Likelihood function
\item[(ii)] Definition of Maximum Likelihood estimate
\item[(iii)] Fisher information: regularity conditions (FI and Le Cam), score function
  and information matrix; information matrix as Riemannian metric on
  manifold of parameters
\item[(iv)] Cramer-Rao Lower Bound
\item[(v)] Asymptotic distribution/Asymptotic Normality : Delta Method and Second Order Delta Method
\item[(vi)] Asymptotic consistency of MLEs 
\item[(vii)] Asymptotic efficiency of MLEs
\item[(viii)] Hypothesis testing with MLE: Likelihood Ratio Tests 
  Wilks Theorem(Schervish Thm 7.125, van der Vaart 16.9), Wald Tests and Score Tests
\item[(ix)] Problems with boundaries lack of regularity
\item[(x)] M-estimators
\item[(xi)]Observed information matrix...
\end{itemize}

As a quick motivation for where maximum likelihood estimation comes
from, consider the following measure of distance between two
probability distributions that was motivated by information theory.
\begin{defn}Suppose $\mu$ and $\nu$ such that $\mu << \nu$.  The the \emph{Kullback-Liebler divergence} or \emph{relative
    entropy} of $\mu$ and $\nu$ is defined as
\begin{align*}
\kldiv{\mu}{\nu} &= \sexpectation{\log \frac{d\mu}{d\nu}}{\mu}
\end{align*}
If $\mu$ is not absolutely continuous with respect to $\nu$ then by
convention $\kldiv{\mu}{\nu} = \infty$.
\end{defn}

\begin{examp}
Suppose $\mu$ and $\nu$ are probability measures that are both
absolutely continuous with respect to a third measure $\lambda$ and
furthermore $\mu << \nu$.  Then we may write $\mu = f \cdot \lambda$
and $\nu = g \cdot \lambda$ where we assume that $\lambda$- almost
surely $g=0$ implies $f=0$ (otherwise the event $A=\lbrace g=0; f>0 \rbrace$
satisfies $\nu(A)=0$ but $\mu(A)\neq 0$).  In this case we can make sense of the
ratio $\frac{f}{g}$ if we agree that $\frac{0}{0} = 0$ and then $\frac{d\mu}{d\nu} = \frac{f}{g}$.

In this case we get the formula 
\begin{align*}
\kldiv{\mu}{\nu} &= \int \log(\frac{f}{g}) f \, d\lambda
\end{align*}
that the user may have encountered before.
\end{examp}

\begin{examp}One interpretation of relative entropy is that is the
  number of bits of information that one gains updating ones that
  belief that a probability distribution is $\nu$ to a belief that a
  probability distribtuion is $\mu$.  The following simple example
  illustrates the point.  In what follows we interpret $\log$ to be
  the base 2 logarithm as opposed to the standard assumption that it
  represents the natural logarithm.  Suppose you believe that a coin is fair.  In
  this case you believe that the distribution is $\nu(H) = \nu(T) =
  1/2$.  If someone tells you that the coin is a trick coin that only
  lands with heads up then you change belief to $\mu(H) = 1$ and
  $\mu(T)=0$.  It is easy to see that $\mu <<  \nu$ and using the formula for relative entropy in terms of
  densities in the previous example we compute
\begin{align*}
\kldiv{\mu}{\nu} &= \log(\frac{1}{1/2}) \cdot 1 + \log
(\frac{0}{1/2}) \cdot 0 = \log 2 = 1
\end{align*}
Thus one has gained $1$ bit of information; which is intuitively
correct because on updating one's view of the probability distribution
one has learned the outcome of a single binary trial.

It is also instructive to consider the example with the roles of $\mu$
and $\nu$ reversed.  In this case $\mu(T) = 0$ but $\nu(T) \neq 0$
hence $\nu$ is not absolutely continuous with respect $\mu$ and
therefore we have agreed that the relative entropy is infinite.  The convention
is corroborated by the heuristic calculation
\begin{align*}
\kldiv{\nu}{\mu} &= \log(\frac{1/2}{1}) \cdot \frac{1}{2} + \log
(\frac{1/2}{0}) \cdot \frac{1}{2} = \infty
\end{align*}
The intuition here is that in going from $\mu$ to $\nu$ we are
learning that something that was formerly thought to be impossible is
in fact possible and that the information gained from this is
infinitely large.  Along the lines of this example one will often hear
the relative entropy referred to as \emph{information gain} :
particularly in the machine learning literature.
\end{examp}

\begin{lem}[Gibbs Inequality]\label{GibbsInequality}For all
  probability distributions $\mu$ and $\nu$, $\kldiv{\mu}{\nu} \geq 0$
  with equality if and only if $\mu$ and $\nu$ agree except on a set
  of measure zero with respect to $\nu$.
\end{lem}
\begin{proof}
It suffices to handle the case in which $\nu << \mu$.  In this case we
can simply use the strict convexity of $x \log x$ and apply Jensen's
inequality and the definition of the Radon-Nikodym derivative to see
\begin{align*}
\kldiv{\mu}{\nu} &= \sexpectation{\log \frac{d\mu}{d\nu}}{\mu} = \sexpectation{\frac{d\mu}{d\nu}\log \frac{d\mu}{d\nu}}{\nu} \geq
\sexpectation{\frac{d\mu}{d\nu}}{\nu} \log \sexpectation{\frac{d\mu}{d\mu}}{\nu} = 
\sexpectation{1}{\mu} \log \sexpectation{1}{\mu} = 0
\end{align*}
By strict convexity of $x \log x$, we have equality if and only if $\frac{d\mu}{d\nu}$ is
almost surely (with respect to $\nu$) a constant.  This constant must be $1$ because $\mu$ and
$\nu$ are both probability measures.  
\end{proof}

\begin{examp}Continuing the previous example we specialize to case in
  which we consider a family of densities indexed by a set $\Theta$.
  Specifically for each $\theta \in \Theta$, we suppose we have a
  density $f(x \mid \theta)$ with respect to a base measure
  $\lambda$.  The problems of (parametric) statistical estimation generally start
  with such an assumption and and assume there is distinguished
  \emph{true} value $\theta_0$ from among the elements of the set
  $\Theta$.  Lemma \ref{GibbsInequality} suggests a potential path.  We know
  from the previous example that 
\begin{align*}
\kldiv{\theta_0}{\theta} &= \sexpectation{ \log( \frac{f(x \mid \theta_0)}{f(x
  \mid \theta)})}{\theta_0} = \sexpectation{ \log( f(x \mid \theta_0))
}{\theta_0} - \sexpectation{ \log( f(x \mid \theta))
}{\theta_0} \geq 0
\end{align*}
with equality if an only $f(x \mid \theta_0)$ and $f(x \mid \theta)$
give the same measure (which we generally assume to imply that
$\theta_0 = \theta$; a condition referred to as
\emph{identifiability}).  So this means that $\sexpectation{ \log( f(x \mid \theta))
}{\theta_0}$ has a unique maximum at the value $\theta_0$.  Now this
isn't of much use directly since it assume knowledge of the density
$f(x \mid \theta_0)$ in order to compute the expectations, but it
suggests that we should consider using an approximation of the measure
defined by the density such as one defined by sampling and consider
contexts in which we maximize the function $f(x \mid \theta)$
considered as a function of $\theta$.  This insight leads to the
method of maximum likelihood which we shall study in some detail in
the following chapter.  
\end{examp}

Now we apply this idea in the context of parametric estimation.  If we
suppose that we are given a parametric family of densities $f(x;
\theta)$ relative to some measure $\nu$.

TODO: To be continued...

\subsection{The Delta Method}

\begin{defn}Given a metric space $(S,d)$ and arbitrary index set $A$, a set of random elements
  $\xi_\alpha$ in $S$ with $\alpha \in A$ is said to be \emph{tight} if for
  every $\epsilon > 0$ there exists a compact set $K \subset S$ such
  that $\sup_\alpha \probability{\xi_\alpha \notin K} < \epsilon$.  In
  the case in which $\xi_\alpha$ are random vectors in some $\reals^n$
  it is also common to that a tight set of random vectors is
  \emph{bounded in probability}.
\end{defn}

Just as with convergence in distribution, note that tightness is
really a property of the law of the random elements $\xi_\alpha$.  
We will eventually see that tightness is a type of sequential
compactness; if one goes a bit farther than we intend to go, one can in
fact show that there is a metric on the space of measures (the
Levy-Prohorov metric which metrizes convergence in distribution) and that tight sets are compact sets of
measures in the corresponding metric space (are all compact sets tight???).

The first thing that we shall see about tightness is the fact that
sequence that converge in distribution are tight.

\begin{lem}\label{WeakConvergenceImpliesTight}Suppose $\xi_n \todist
  \xi$ with $\xi, \xi_1, \xi_2, \dots$ random vectors, then $\xi_n$ is a tight sequence.
\end{lem}
\begin{proof}
TODO: Can we use Portmanteau and clean up the argument by making the
continuous approximation unnecessary?  Answer is certainly yes but
it's not clear how much simpler it makes the argument.

Suppose we are given an $\epsilon > 0$.  First since $\xi$ is almost surely finite, continuity of measure shows
that $\lim_{M \to \infty} \probability{\abs{\xi} > M} = 0$ and
therefore we can find $M_1 > 0$ such $\probability{\abs{\xi} > M_1} <
\frac{\epsilon}{2}$.   Now pick an arbitrary $M_2 > M_1$ and let $f$ be a
bounded continuous function such that $\characteristic{\abs{x} > M_2}
\leq f \leq \characteristic{\abs{x} > M_1}$.  Then we have 
\begin{align*}
\probability{\abs{\xi_n} > M_2} &\leq \expectation{f(\xi_n)} 
\end{align*}
and 
\begin{align*}
\expectation{f(\xi)} \leq \probability{\abs{\xi} > M_1} \leq \frac{\epsilon}{2}
\end{align*}
but also we can find $N > 0$ such that $\abs{\expectation{f(\xi_n)} -
  \expectation{f(\xi)} } < \frac{\epsilon}{2}$ for all $n \geq
N$.  Putting the pieces together we have for all $n \geq N$, 
\begin{align*}
\probability{\abs{\xi_n} > M_2} &\leq \expectation{f(\xi_n)} \leq
\expectation{f(\xi)} + \abs{\expectation{f(\xi_n)} -
  \expectation{f(\xi)}} < \epsilon
\end{align*}
Now for each $0 \leq n \leq N$, we can find $M^\prime_n$ such that
$\probability{\abs{\xi_n} > M^\prime_n} < \epsilon$, so if we take
$M=\max(M_2, M^\prime_1, \dots, M^\prime_{n-1})$ then we get
$\sup_n\probability{\abs{\xi_n} > M} < \epsilon$ and tightness is shown.
\end{proof}
\begin{lem}\label{ScaledTightSequenceConvergeZeroProb}Suppose $r_n$ is a sequence of real numbers such that
  $\lim_{n \to \infty} \abs{r_n} = \infty$ and
  $\eta, \xi, \xi_1, \xi_2, \dots$ is a sequence of random vectors such that $r_n(\xi_n -
  \xi) \todist \eta$.  Then $\xi_n \toprob \xi$.
\end{lem}
\begin{proof}
The proof only relies on the fact that $r_n(\xi_n -  \xi)$ is a
tight sequence (Lemma \ref{WeakConvergenceImpliesTight}).  Suppose we
are given $\epsilon, \delta > 0$.  By
tightness, we can pick $M > 0$ such that 
\begin{align*}
\sup_n \probability{\abs{r_n(\xi_n - \xi)} > M} &= \sup_n \probability{\abs{\xi_n - \xi} > \frac{M}{\abs{r_n}}} < \delta
\end{align*}
Because $\lim_n \abs{r_n} = \infty$ we pick $N>0$ such that
$\frac{M}{\abs{r_n}} \leq \epsilon$ for $n \geq N$.  Then 
\begin{align*}
\probability{\abs{r_n(\xi_n - \xi)} > \epsilon} &\leq \sup_n \probability{\abs{\xi_n - \xi} > \frac{M}{\abs{r_n}}} < \delta
\end{align*}
for $n \geq N$ and we have show $\xi_n \toprob \xi$.
\end{proof}

In this result we have restricted ourselves to random vectors in
$\reals^n$ because it is an important special case (especially in
parametric statistics) and because it is a trivial matter to show that all random
vectors are tight.  Generalization to arbitrary metric spaces is
subtle because it is no longer the case that an arbitrary random
element is tight.  One can repair the argument above by adding the
assumption that the elements of the sequence are tight random elements
or one can explore what conditions on a metric space guarantee that
all random elements are tight.  Though we don't go into it at the
moment, it turns out separability and completeness (i.e. Polishness) 
are sufficient to guarantee tightness of arbitrary random
elements and there is also a more subtle necessary and sufficient
condition that has been identified (universal measurability see
Dudley's RAP).

Part of the importance of tightness is lies in its role as a
compactness property (that is to say the fact that it implies weak
convergence of a subsequence).  On the other hand, in some cases one
uses only the boundedness aspect.  This is particularly true in
asymptotic statistics.  TODO: Introduce the $O_P(r_n)$ and $o_p(r_n)$
notation.
\begin{lem}\label{AlgebraOfStochasticConvergence}Let $\xi_1,
  \xi_2, \dots$ and
  $\eta_1, \eta_2, \dots$ be sequences of random vectors.
\begin{itemize}
\item[(i)]If $\xi_n \toprob 0$ then $\xi_n$ is tight. ($o_p(1) = O_P(1)$).
\item[(ii)]If $\xi_n \toprob 0$ and $\eta_n \toprob 0$ then $\xi_n +
  \eta_n \toprob 0$.  ($o_P(1) + o_P(1) = o_P(1)$).
\item[(iii)]If $\xi_n$ is tight and $\eta_n \toprob 0$ then $\xi_n +
  \eta_n$ is tight.  ($O_P(1) + o_P(1) = O_P(1)$).
\item[(iv)]If $\xi_n$ is tight and $\eta_n \toprob 0$ then $\xi_n *
  \eta_n \toprob 0$ (this is true for many kinds of multiplication;
  scalar multiplication, dot product, matrix multiplication). ($O_P(1)
  o_P(1) = o_P(1)$.
\item[(v)]If $\eta_n$ is tight sequence of random variables and
  $\xi_n \eta_n \toprob 0$ then $\xi_n \toprob 0$. ($o_P(O_P(1)) = o_P(1)$).
\end{itemize}
\end{lem}
\begin{proof}
To prove (i) simply note that $\xi_n \toprob 0$ implies $\xi_n
\todist 0$ (Lemma
\ref{ConvergenceInProbabilityImpliesConvergenceInDistribution}) the
therefore we know $\xi_n$ is tight by Lemma
\ref{WeakConvergenceImpliesTight}.

The statement of (ii) is a corollary to the Continuous Mapping Theorem
(Corollary \ref{ConvergenceInProbabilityAndAlgebraicOperations}).

TODO: Finish...
\end{proof}

Here is a slightly more involved fact that we shall use in the sequel.
\begin{lem}\label{InvertMatrixInProbability}Let $\Psi_n$ be a sequence of
  random matrices such that $\Psi_n \toprob \Psi$ with $\Psi$ almost surely
  equal to a constant nonsingular matrix.  Suppose $\xi_n$ is a
  sequence of random vectors such that $\Psi_n \xi_n$ is tight, then
  $\xi_n$ is tight.
\end{lem}
\begin{proof}
Recall that because convergence in
probability only depends on the underlying topology induced by a
metric (Corollary \ref{ConvergenceInProbabilityIndependentOfMetric})
and that all norms on a finite dimensional vector space are
equivalent; this means that we are free to choose the operator norm when dealing with
the convergence of the matrices $\Psi_n$.  

We remind the reader of some basic facts about the operator norm.  
In any normed vector space of linear operators with the operator norm
we have Neumann series for inverting perturbations of the identity
operator.  Specifically for any $A$ with $\norm{A} < 1$, we have
\begin{align*}
(1 - A)^{-1} &= \sum_{n=0}^\infty A^n & & \text{converges absolutely} \\
\norm{(1 - A)^{-1}} &\leq \sum_{n=0}^\infty \norm{A^n} \leq 
\sum_{n=0}^\infty \norm{A}^n = (1 - \norm{A})^{-1} \\
(1-A) (1 - A)^{-1} &= \sum_{n=0}^\infty A^n - \sum_{n=1}^\infty A^n =
1 \\
(1 - A)^{-1}(1-A) &= \sum_{n=0}^\infty A^n - \sum_{n=1}^\infty A^n =
1 \\
\end{align*}
which shows that $(1-A)$ is invertible with inverse $(1-A)^{-1}$
defined by the Neumann series.  We now extend this argument
to show there is an  open neighborhood of any invertible operator in
the space of invertible operators.  Suppose $T$ is invertible and let
$\norm{T - A} < \frac{1}{\norm{T^{-1}}}$.  Then we can write $T - A =
T(1 - T^{-1}A)$ where $\norm{T^{-1}A} \leq \norm{T^{-1}}\norm{A} < 1$
so that $(1-T^{-1}A)$ is invertible.  This shows $T-A$ is product of
invertible operators hence is itself invertible.  Moreover we have the
norm bound
\begin{align*}
\norm{(T-A)^{-1}} &\leq \norm{T}\norm{(1 - T^{-1}A)^{-1}} \leq
\frac{\norm{T}}{1 -
\norm{T^{-1}A}} \leq \frac{\norm{T}}{1 -
\norm{T^{-1}}\norm{A}}
\end{align*}

With that little piece of operator theory out of the way we can return
statistics proper.  We have assumed $\Psi_n \toprob
\Psi$ with $\Psi$ an invertible a.s. constant matrix.  Pick $\delta > 0$ and $0
< \epsilon <
\frac{1}{2\norm{\Psi^{-1}}}$, then we know that
there exists an $N > 0$ such that $\probability{\norm{\Psi_n - \Psi} \leq \epsilon} \geq
1 - \frac{\delta}{2}$ for all $n > N$.  By the preceeding
discussion we know that whenever $\norm{\Psi_n - \Psi} \leq \epsilon$,
  $\Psi_n$ is invertible and $\norm{\Psi_n^{-1}} < 2 \norm{\Psi^{-1}}$.  By
  tightness of $\Psi_n \xi_n$ we can find $M > 0$ such that 
\begin{align*}
\sup_n \probability{\norm{\Psi_n \xi_n}>M} < \frac{\delta}{2}
\end{align*}
Therefore by applying the inverse of $\Psi_n$ and using
its operator norm bound we get
\begin{align*}
\sup_{n>N} \probability{\norm{\xi_n}>2M \norm{\Psi^{-1}}}< \delta
\end{align*} 
Because random vectors in $\reals^n$ are tight, we know that there is
an $M^\prime$ such that $\probability{\norm{\xi_n}>M^\prime} < \delta$ for all $0 < n \leq N$ and
therefore $\xi_n$ is tight.
\end{proof}

\begin{defn}Given an open set $U \subset \reals^m$ and function $\phi :
  U\to \reals^n$  we say that $\phi$ is \emph{Frechet differentiable} at a
  point $x\in U$ if there is a linear map $A : \reals^m \to
  \reals^n$ such that for every sequence $h_n \in \reals^m$ such that
$\lim_{n \to \infty} \abs{h_n} = 0$ we have 
\begin{align*}
\lim_{n \to \infty}
  \frac{\phi(x + h_n) - \phi(x) - A h_n}{\abs{h_n}} &= 0 
\end{align*}
The linear map $A$ is called the \emph{Frechet derivative} of $\phi$
at $x$ is usually written $D\phi(x)$.
\end{defn}

\begin{thm}[Delta Method]\label{DeltaMethod}Let $\phi : D \subset
  \reals^k \to \reals^m$ be Frechet differentiable at $\theta \in D$.  Let
  $\xi, \xi_1, \xi_2, \dots$ be random vectors with values in $D$ and $r_n$ be a sequence
  of real numbers such that $\lim_{n \to \infty} r_n = \infty$ and
  $r_n(\xi_n - \theta) \todist \xi$.  Then 
\begin{align*}
r_n(\phi(\xi_n) -
  \phi(\theta)) \todist D\phi(\theta) \xi
\end{align*}
and moreover 
\begin{align*}
\abs{r_n(\phi(\xi_n) - \phi(\xi)) - D\phi(\theta) r_n(\xi_n - \theta)} \toprob 0
\end{align*}
\end{thm}
\begin{proof}
By Lemma \ref{ScaledTightSequenceConvergeZeroProb} be know that $\xi_n
- \theta
\toprob 0$.  By differentiability of $\phi$ we know that  for every
sequence $h_n \to 0$,
\begin{align*}
\lim_n \frac{\phi(\theta + h_n) - \phi(\theta) - D\phi(\theta)h_n}{\abs{h_n}} = 0
\end{align*}

The first thing to show is that we can extend this fact to random
sequences.  We state this as a general fact.  Suppose $\psi(x)$ is a
function such that for every $h_n \to 0$ we have
$\frac{\psi(h_n)}{\abs{h_n}} \to 0$.  We claim that if we are given
random vectors $\eta_n$ such that $\eta_n \toprob 0$ then
$\frac{\psi(\eta_n)}{\abs{\eta_n}} \toprob 0$.  To see this define a
new function by 
\begin{align*}
f(x) &= \begin{cases}
\frac{\psi(x)}{\abs{x}} & \text{for $x \neq 0$} \\
0 & \text{for $x=0$}
\end{cases}
\end{align*}
and note that by assumption $f$ is continuous at $0$.  Now by the
Continuous Mapping Theorem (Theorem \ref{ContinuousMappingTheorem}) we know that $f(\eta_n)
\toprob f(0) = 0$.

Having shown the above fact, we can use $\xi_n - \theta \toprob 0$ to
conclude
\begin{align*}
\frac{\phi(\xi_n) - \phi(\theta) - D\phi(\theta) (\xi_n -
  \theta)}{\abs{\xi_n - \theta}} &\toprob 0
\end{align*}
and if we multiply top and bottom by $r_n$ and use linearity of the
Frechet derivative we get
\begin{align*}
\frac{r_n(\phi(\xi_n) - \phi(\theta)) - D\phi(\theta) r_n(\xi_n -
  \theta)}{\abs{r_n(\xi_n - \theta)}}  &\toprob 0
\end{align*}
Tightness of $r_n(\xi_n - \theta)$ allows us to conclude that 
\begin{align*}
r_n(\phi(\xi_n) - \phi(\theta)) - D\phi(\theta) r_n(\xi_n -  \theta)  \toprob 0
\end{align*}
which gives us the second conclusion of the Theorem.

To prove this last fact suppose $\xi_n, \eta_n$ are random vectors
such that $\frac{\xi_n}{\abs{\eta_n}} \toprob 0$ and $\eta_n$ is
tight.  Suppose we are given $\epsilon, \delta > 0$.  Use tightness to
pick an $M>0$ such that $\sup_n \probability{\abs{\eta_n} > M} <
\frac{\delta}{2}$ and use $\frac{\xi_n}{\abs{\eta_n}} \toprob 0$ to
pick an $N$ such that $\probability{\abs{\frac{\xi_n}{\eta_n}} >
  \frac{\epsilon}{M}} < \frac{\delta}{2}$ for all $n \geq N$.
Then
\begin{align*}
\probability{\abs{\xi_n}>\epsilon} &=
\probability{\abs{\xi_n}>\epsilon ; \abs{\eta_n} > M} +
\probability{\abs{\xi_n}>\epsilon ; \abs{\eta_n} \leq M} \\
&\leq \probability{\abs{\eta_n} > M} +
\probability{\frac{\abs{\xi_n}}{\abs{\eta_n}}>\frac{\epsilon}{M}} \\
&< \delta
\end{align*}
for all $n \geq N$ which shows $\xi_n \toprob 0$.  TODO: Is it better to
think of this as $O_P(1) o_P(1) = o_P(1)$; probably better to think of
this as $o_P(O_P(1)) = o_P(1)$?

To get the first conclusion we simply use the fact that matrix
multiplication is continuous and the Continuous Mapping Theorem
(Theorem \ref{ContinuousMappingTheorem}) to see
that $D\phi(\theta) r_n(\xi_n - \theta) \todist D\phi(\theta) \xi$ and Slutsky's
Lemma (Lemma \ref{Slutsky})) and the part of this Theorem just proven
to conclude $r_n(\phi(\xi_n) - \phi(\theta)) \todist D\phi(\theta) \xi$.
\end{proof}

\begin{examp}One of the most common problems in statistics is the
  comparison of binomial populations.  For example, to estimate
  treatment effectiveness one might want to compare the proportion of
  postive responses between a treated group and a control group.  One
  common way to estimate the difference in proportions between two
  independent populations is the \emph{risk ratio}
\begin{align*}
\hat{RR} &= \frac{\hat{p}_1}{\hat{p}_2}
\end{align*}
where $\hat{p}_i$ denotes the sample proportion.  Here we calculate
the asymptotic distribution of the risk ratio by using the Delta method.

The trick is to apply a logarithm to convert the division into
subtraction.  First we consider a single sample proportion $\hat{p}$.  Since
$\hat{p} = \frac{1}{n} \sum_n \xi_i$ for $\xi_i$ a Bernoulli random
variable with rate $p$, we can apply the Central Limit Theorem to
conclude that 
\begin{align*}
\sqrt{n} (\hat{p} - p) &\todist N(0, p(1-p))
\end{align*}
Assuming $p \neq 0$, the Delta Method (Theorem \ref{DeltaMethod}) yields
\begin{align*}
\sqrt{n} (\ln (\hat{p}) - \ln(p)) &\todist \frac{1}{p}N(0, p(1-p)) =
  N(0, \frac{1-p}{p})
\end{align*}
Therefore is we apply this reasoning to the risk ratio and use the
fact that a sum of independent normal random variables is normal, we
see that 
\begin{align*}
\sqrt{n} (\ln (\hat{RR}) - \ln(RR)) &\todist N(0, \frac{1-p_1}{p_1} + \frac{1-p_2}{p_2})
\end{align*}

This result can then be used to create asymptotic confidence intervals
for the estimation of risk ratio 
\begin{align*}
\ln(\hat{p}_1/\hat{p}_2) \pm z_{\alpha/2}\sqrt{\frac{1-\hat{p}_1}{n_1
  \hat{p}_1} + \frac{1-\hat{p}_2}{n_2 \hat{p}_2}}
\end{align*}


TODO: Discuss the implications of substituting the variance estimate
into this formula.
\end{examp}
TODO: Lay down the conceptual framework in which parametric statistics
is modeled.  Basic problem statement is this.  Assume that one has a
probability space $(\Omega, \mathcal{A}, P)$ and a family of random
elements $\xi_\theta$ in a measure space $(X, \mathcal{X}, \mu)$ with
$\theta \in \Theta$ an unknown parameter that determines the
distribution of $\xi_\theta$.   Assume we make observations of the
value of $\xi$ (or more properly observations of generally independent
random variables with the same distribution as $\xi$), we want to find an estimate of the value (or the
distribution) of $\theta$.

There is the subtlety around the notion of
having a random variable $\xi$ with \emph{conditional density}
$f(x\mid \theta)$.  The question is how rigorously one needs to think
about the parameter $\theta$.  In the simplest form, one can just
think of having a family of random variables $\xi_\theta$ for $\theta
\in \Theta$ and not concern oneself with measurability in $\theta$.
This seems to be sufficient when discussing frequentist methods for
example.  Note also that the notation $f(x\mid \theta)$ seems to hedge
on how we want to think of the functional dependence on $\theta$.
We'll see that understanding the dependence on $\theta$ is important but doesn't map
nicely to standard probabilisitc or measure theoretic notions and has
its own somewhat idiosyncratic notions of regularity.
In the Bayesian formulation it appears that one wants to
view $\theta$ as a random quantity as well and one assumes the
existence of a random element $\theta$ in $\Theta$ and a random
element $\xi$ in $X$ and take the conditional distribution
$P_\theta = \probability{\xi \in \cdot \mid \theta}$.  Then one assumes that the
conditional distributions are all absolutely continuous with respect
to $\mu$ and thereby get the conditional densities $f(x \mid
\theta)$ such that $P_\theta = f(x \mid \theta) \cdot \mu$.  It is not yet clear to me at what point one is forced to
take the latter approach.

Here is one account of the FI regularity conditions.

\begin{defn}
Suppose we are given a measure space $(X, \mathcal{X}, \mu)$ and a family of
probability measures $P_\theta$ with $\theta \in
\Theta \subset \reals^n$ for some $n > 0$.  Suppose that  such that there
exist densities $f(x \mid \theta)$  for each $P_\theta$ with respect so $\mu$.  The $f(x
\mid \theta)$ are said to satisfy the \emph{FI regularity constraints}
if the following are true:
\begin{itemize}
\item[(i)] $\Theta \subset \reals^n$ is convex and contains and open
  set.  There exists a set $B \in \mathcal{X}$ with $\mu(B^c) = 0$ such
  that $\frac{\partial}{\partial \theta_i} f(x \mid \theta)$ exists
  for every $i=1, \dots, n$, every $\theta \in \Theta$ and every $x
  \in B$.
\item[(ii)] For every $k = 1,\dots, n$, 
\begin{align*}
\frac{\partial}{\partial \theta_i} \int f(x \mid \theta)
  \, d \mu(x) = \int \frac{\partial}{\partial \theta_i} f(x \mid
  \theta) \, d \mu(x)
\end{align*}
\item[(iii)]The set $C = \lbrace x \in X \mid f(x \mid \theta) >
  0$ does not depend on $\theta$.
\end{itemize}
\end{defn}

\begin{defn}Let $\xi$ be a random element in the measure space $(X,
  \mathcal{X}, \mu)$ with conditional density
  $f(x \mid \theta)$ with respect to $\mu$.  Suppose that $f(x \mid
  \theta)$ satisfy the FI regularity constraints.  Then the random
  vector 
\begin{align*}
U(\xi \mid \theta) &= \left( \frac{\partial}{\partial \theta_1} \log f(\xi \mid \theta),
  \dots ,  \frac{\partial}{\partial \theta_n} \log f(\xi \mid \theta)\right)
\end{align*}
is called the \emph{score function}.
\end{defn}

The basic calculation with the score function is that if we assume
that $\xi$ is a random element with density $f(x \mid \theta)$ then
\begin{align*}
\sexpectation{\frac{\partial}{\partial \theta_i} \log f(\xi \mid
  \theta)}{\theta} &= \int \frac{\frac{\partial}{\partial \theta_i}  f(x \mid \theta)}{f(x \mid
  \theta)} f(x \mid \theta) \, d\mu(x) \\
&=\int \frac{\partial}{\partial \theta_i}  f(x \mid \theta) \, d\mu(x) \\
&=\frac{\partial}{\partial \theta_i}\int f(x \mid \theta) \, d\mu(x) =
\frac{\partial}{\partial \theta_i} 1 = 0
\end{align*}
and therefore $\sexpectation{U(\xi \mid \theta)}{\theta} = 0$ under
the FI regularity constraints.

If we differentiate both side of this latter equality 
\begin{align*}
0 &= \frac{\partial}{\partial \theta_j} \int \frac{\partial}{\partial
  \theta_i}\log f(x \mid  \theta) f(x \mid \theta) \, d\mu(x) \\
&=\int \frac{\partial^2}{\partial
  \theta_i\partial
  \theta_j}\log f(x \mid  \theta) f(x \mid \theta) + \frac{\partial}{\partial
  \theta_i}\log f(x \mid  \theta) \frac{\partial}{\partial
  \theta_j} f(x \mid  \theta)\, d\mu(x) \\
&=\int (\frac{\partial^2}{\partial
  \theta_i\partial
  \theta_j}\log f(x \mid  \theta) + \frac{\partial}{\partial
  \theta_i}\log f(x \mid  \theta) \frac{\partial}{\partial
  \theta_j} \log f(x \mid  \theta)) f(x \mid \theta)\, d\mu(x) \\
\end{align*}
which shows that when $\xi$ has density $f(x \mid \theta)$, we have
the identity
\begin{align*}
-\sexpectation{\frac{\partial^2}{\partial
  \theta_i\partial
  \theta_j}\log f(\xi \mid  \theta) }{\theta} = \sexpectation{\frac{\partial}{\partial
  \theta_i}\log f(\xi \mid  \theta) \frac{\partial}{\partial
  \theta_j} \log f(\xi \mid  \theta)}{\theta}
\end{align*}
This quantity is called the \emph{Fisher information matrix}.   TODO:
The Fisher information as a Riemannian metric on $\Theta$.

TODO: What kind of object is the score function (i.e. what domain and
range).  More specifically, how does one think of the $\theta$ dependence in the score
function?  In the Bayesian formulation everything is fine because
$\xi$ is an honest random element and we are just composing it with a
deterministic function.  In the formulation in which we don't think of
$\theta$ as being random, then are we thinking of $\xi$ as having
$\theta$-dependence when we plug it in?  The answer to this is YES. 

\begin{examp}Let $\xi$ be a parameteric Gaussian family with
  $\theta=(\mu, \sigma)$.  Then $f(x \mid \theta) =
  \frac{1}{\sqrt{2\pi\sigma^2}} e^{\frac{-(x-\mu)^2}{2\sigma^2}}$ and
  $U(\xi | \theta) = \frac{\xi - \mu}{\sigma^2}$.
\end{examp}

\begin{defn}Let $\xi$ be a random element with conditional density
  $f(x \mid \theta)$ with respect to a measure space $(X,
  \mathcal{X}, \mu)$.  For every $x \in X$, the function 
\begin{align*}
L(\theta) &= f(x \mid \theta)
\end{align*}
is called the \emph{likelihood function}.

Any random element $\hat{\theta}$ in $\Theta$ that satisfies
\begin{align*}
\max_{\theta \in \Theta} f(\xi \mid \theta) = f(\xi \mid \hat{\theta})
\end{align*}
is called a \emph{maximum likelihood estimator} of $\theta$.
\end{defn}

It is important to note that in most statistical applications the
random element $\xi$ whose likelihood we are investigating is a random
vector that corresponds to sampling from a population.  This is to say
that is some underlying distribution of interest that corresponds to
some random element $\xi$ and that we model repeated sampling as a
random element $\sample{\xi} = (\xi_1, \dotsc, \xi_n)$ in a product
space $\mathcal{X}^n$.  In all cases we shall be concerned about for
the moment, we assume that the samples are i.i.d. hence the joint
density of the sample is just the product of the density of $\xi$.  In
some cases it may be convenient to emphasize that the likelihood
function is of such a form; in those cases we may choose to write
$L_n(\theta)$ for the sample likelihood.

The fact that likelihood functions for independent samples are products
is leveraged constantly in what follows and is in large part
responsible for the nice asymptotic properties of maximum likelihood
estimators.  To release the power of this fact we simply convert the
product into a sum by taking log and create the log likelihood.  Note that because the log is
monotonic, one can perform maximum likelihood estimation equally well
by taking maxima of the log likelihood.  We shall usually write
$\ell(x \mid \theta)$ to denote a log likelihood and the case of
i.i.d. samples we shall use a subscript to emphasize the
dependence on sample size $\ell_n(\sample{\xi} \mid \theta) =
\sum_{i=1}^n\log f(\xi_i \mid \theta)$.  The maximum likelihood estimator
associated with i.i.d. samples of size $n$ is denoted:
\begin{align*}
\hat{\theta}_n &= \max_{\theta \in \Theta} \sum_{i=1}^n \log f(\xi_i \mid \theta)
\end{align*}
and it is the estimator that we shall spend some time studying.  The
motivation behind this mechanism is that we know from the Gibbs
Inequality (Lemma \ref{GibbsInequality}) that the true parameter
$\theta_0$ is characterized as the maximum of $\sexpectation{\log f(x
  \mid \theta)}{\theta_0}$.  Now we can view $\hat{\theta}_n$ as the
result of substituting the (random) empirical measure in the
expectation.  To the extent that the empirical measure converges we
may hope that the estimator converges as well.  Less abstractly, we
know from the Strong Law of Large Numbers that
$\frac{1}{n}\sum_{i=1}^n \log f(\xi_i \mid \theta) \toas
\sexpectation{ \log f(x \mid
  \theta)}{\theta_0}$ so thinking of this as convergence of functions
of $\theta$ we may hope that the convergence is strong enough so that
the maxima converge.

Note that the definition of the maximum likelihood estimator is using
the $\max$ and not the $\sup$; this means that in the case the
supremum is not actually attained on the set $\Theta$ (e.g. $\Theta$
is open and the supremum is attained on the boundary) then MLE may not
exist.  In some accounts of the theory, the maximum is taken over the
closure of the parameter domain (should we do this?)

\begin{examp}\label{MLENormalParameters}Consider the case parameter estimation in a normal
  distribution $\frac{1}{\sqrt{2\pi}\sigma} e^{-(x - \mu)^2 /
    2\sigma^2}$.  If we consider $\mu$ unknown and $\sigma$ known the
    the MLE for the mean is given by setting the derivative with
    respect to $\mu$ to be zero
\begin{align*}
\frac{\partial}{\partial \mu} \sum_{i=1}^n \log \frac{1}{\sqrt{2\pi}\sigma} e^{-(\xi_i - \mu)^2 /
    2\sigma^2} &= - \frac{1}{\sigma^2} \sum_{i=1}^n (\xi_i - \mu) = 0 
\end{align*}
which implies it is the sample mean $\hat{\mu}_n = \frac{1}{n}
\sum_{i=1}^n \xi_n$.

If we assume that $\mu$ is known and $\sigma$ is unknown the finding
the maximum by differentiation we get
\begin{align*}
\frac{\partial}{\partial \sigma} \sum_{i=1}^n \log \frac{1}{\sqrt{2\pi}\sigma} e^{-(\xi_i - \mu)^2 /
    2\sigma^2} &= \frac{1}{\sigma^3} \sum_{i=1}^n (\xi_i - \mu) - n \frac{1}{\sigma}= 0 
\end{align*}
and therefore the biased estimate of standard deviation $\hat{\sigma}_n = \frac{1}{n} \sum_{i=1}^n (\xi_n-\mu)^2$.
\end{examp}

TODO: Example of estimating the rate of a Bernoulli r.v.  Note the
boundary behavior.

TODO:  Example of $\xi$ as a random vector of independent observations
(factoring the likelihood function).

Note that we have allowed an MLE to be an arbitrary random element in
$\Theta$.  It makes intuitive sense however that the estimator should
depend on the value of $\xi$.  That is indeed the case in many cases
of interest and one of our goals shall be to understand the conditions
under which that dependence holds.

\begin{thm}If there is a sufficient statistic and the MLE exists, then
  the MLE is a function of the sufficient statistic.
\end{thm}
\begin{proof}
TODO: Apply the factorization theorem.
\end{proof}

TODO: Bring up the notion of \emph{identifiability}; clearly if the
likelihood function attains its maximum value for multiple values of
$\theta$ then it is subtle to describe what consistency means (which
is the correct value of $\theta$).

As we've seen in Example \ref{MLENormalParameters} we cannot expect
that maximum likelihood estimators will be consistent.  However it is
often the case that they will be asymptotically consistent.  
TODO: Define weakly and strongly asymptotically consistent.
The following theorem provides a set of sufficient conditions under
which a maximum likelihood estimator is strongly asymptotically consistent.
\begin{thm}[Asymptotic Consistency of MLE]\label{AsymptoticConsistencyMLE}Let $\xi, \xi_1, \xi_2, \dots$ be i.i.d. parametric
  family with distribution $f(x \mid \theta) \, d \mu$ with respect to measure
  space $(X, \mathcal{X}, \mu)$.  Assume that  $\theta_0$ is fixed and
  define 
\begin{align*}
Z(M, x) &= \inf_{\theta \in M} \log \frac{f(x \mid \theta_0)}{f(x \mid \theta)}
\end{align*}
Assume that for all $\theta \neq \theta_0$ there is an open neighborhood $U_\theta$
such that $\theta \in U_\theta$ and $\sexpectation{Z(U_\theta, \xi)}{\theta_0} >
0$.

If $\Theta$ is not compact, assume that there is a compact $K \subset
\Theta$ such that $\theta_0 \in K$ and $\sexpectation{Z(\Theta
  \setminus K, \xi)}{\theta_0} > 0$.  Then 
\begin{align*}
\lim_{n \to \infty} \hat{\theta}_n = \theta_0
\end{align*}
almost surely with respect to $P_{\theta_0}$.
\end{thm}

Before starting in on the proof make sure to understand the nature of
the hypotheses.  Given the observation $x$ we have $Z(U, x) < 0$ if
there is a $\theta \in U$ such that a $\theta$ this more likely than
$\theta_0$, whereas $Z(U, x) > 0$ tells us that $\theta_0$ is more
likely than any $\theta \in U$.  Thus the conditions $\sexpectation{Z(U_\theta, \xi)}{\theta_0} >
0$ are statements that on average there is no better explanation than
$\theta_0$.  One thing that is interesting about the result is that it
is only required that $\theta_0$ be the best average estimate locally
in $\Theta$ (admittedly the weakening to a local property is only
allowed over a compact set).

\begin{proof}
By Lemma \ref{ConvergenceAlmostSureByInfinitelyOften}, the Theorem is proven if we can show that
$\sprobability{ d(\hat{\theta}_n, \theta_0) \geq \epsilon \text{
  i.o.}}{\theta_0}= 0$
for every $\epsilon > 0$.  
So assume that we have fixed $\epsilon > 0$ and let $B(\theta_0,
\epsilon)$ be the $\epsilon$-ball around $\theta_0$.  Since $K
\setminus B(\theta_0, \epsilon)$ is compact and $U_\theta$ is a cover,
we can find an finite subcover $U_1, \dots, U_{m-1}$ of $K
\setminus B(\theta_0, \epsilon)$ such that each $U_j$ satisfies
$\sexpectation{Z(U_j, \xi)}{\theta_0} > 0$.  If we define $U_m
= \Theta\setminus K$ then we by hypothesis have a finite cover $U_1, \dots,
U_m$ of $\Theta
\setminus B(\theta_0, \epsilon)$ with each $U_j$ satisfying the same
property.

Now on each $U_j$ we can apply the Strong Law of Large Numbers to
conclude that for each $j$, $\frac{1}{n} \sum_{i=1}^n Z(U_j, \xi_i)
\toas \sexpectation{Z(U_j, \xi)}{\theta_0} > 0$ a.s. The key
point from this point on is to understand that if we assume that
$\hat{\theta}_n \in U_j$ infinitely often it would force the
expectation $\sexpectation{Z(U_j, \xi)}{\theta_0}$ to be nonpositive.  Precisely,
\begin{align*}
&\sprobability{\hat{\theta}_n \notin B(\theta_0, \epsilon) \text{
    i.o.}}{\theta_0} \\
&\leq \sprobability{\hat{\theta}_n
  \in \cup_{j=1}^m U_j \text{ i.o.}}{\theta_0} & & \text{since $B^c
  \subset \cup_{j=1}^m U_j $}\\
&= \sprobability{\cup_{j=1}^m \lbrace \hat{\theta}_n
  \in U_j \text{ i.o.} \rbrace }{\theta_0}  & & \text{by
    finiteness of $n$}  \\
&\leq \sum_{j=1}^m \sprobability{\hat{\theta}_n
  \in U_j \text{ i.o.}}{\theta_0} & & \text{by
  subadditivity} \\
&\leq \sum_{j=1}^m \sprobability{\inf_{\theta \in U_j} \sum_{i=1}^n
  \log \frac{f(\xi_i, \theta_0)}{f(\xi_i, \theta)} \leq 0 \text{
    i.o.}}{\theta_0} & & \text{because $\sum_{i=1}^n
  \log \frac{f(\xi_i, \theta_0)}{f(\xi_i, \hat{\theta}_n)} \leq 0$ } \\
&\leq \sum_{j=1}^m \sprobability{\sum_{i=1}^n \inf_{\theta \in U_j} 
  \log \frac{f(\xi_i, \theta_0)}{f(\xi_i, \theta)} \leq 0 \text{
    i.o.}}{\theta_0} \\
&= \sum_{j=1}^m \sprobability{\lim_{n \to \infty} \frac{1}{n}\sum_{i=1}^n \inf_{\theta \in U_j} 
  \log \frac{f(\xi_i, \theta_0)}{f(\xi_i, \theta)} \leq 0}{\theta_0}
\\
&= \sum_{j=1}^m \sprobability{\lim_{n \to    \infty} \frac{1}{n}
  \sum_{i=1}^n  Z(U_j, \xi_i) \leq 0 }{\theta_0}\\
&=0
\end{align*}
since as noted the last equality follows from fact that the Strong Law of Large Numbers tells us that almost
surely for all $1 \leq j \leq m$,
\begin{align*}
\lim_{n \to    \infty} \frac{1}{n} \sum_{i=1}^n  Z(U_j, \xi_i) &=
  \sprobability{Z(U_j, \xi)}{\theta_0} > 0
\end{align*}
\end{proof}

Note that the proof above has a gap in it from the outset.  The
functions $Z(M, x)$ for a fixed $M \subset \Theta$ are defined as an
infimum of an uncountable collection of random variables hence we do
not know that they are measurable.  On the other hand we clearly need
them to be in order to take expectations.  TODO:  How do we get around
these issues?  I suspect there are two paths to explore: 1) take a
countable dense subset and show that the infimum can be reduced to a
countable one or 2) abandon measurability and see if we can make due
with outer expectations (a la empirical process theory).

\begin{examp}
Consider the problem of estimating the parameter $\theta \in [0,
\infty)$ in the family $U(0, \theta)$.  Assume that $\theta_0$ is the
true parameter and we want to show consistency of the maximum
likelihood estimator.  The likelihood function in
this case is 
\begin{align*}
f(x \mid \theta) &= \begin{cases}
\frac{1}{\theta} & \text{if $0 \leq x \leq \theta$} \\
0 & \text{if $x < 0$ or $x > \theta$}
\end{cases}
\end{align*}
Note that you should be thinking of $f(x \mid \theta)$ as a function
of $\theta$ with $x$ fixed.  To apply Theorem
\ref{AsymptoticConsistencyMLE}
we need to show $\sexpectation{Z(U, \theta)}{\theta_0} > 0$ for
appropriately chosen $U \subset [0, \infty)$.  Since $\sprobability{x <
  0}{\theta_0} = \sprobability{x > \theta_0}{\theta_0} = 0$ for
purposes of computing the expectations we may assume that $0 \leq x
\leq \theta_0$.  With this in mind, for such an $x$, we have the likelihood ratio
\begin{align*}
\log \frac{f(x \mid \theta_0)}{f (x \mid \theta)} &= \begin{cases}
+\infty & \text{if $0 \leq \theta < x$} \\
\log \frac{\theta}{\theta_0} & \text{if $x \leq \theta$} 
\end{cases}
\end{align*}

So now we find our neighborhoods.  Pick $\theta > \theta_0$ and define
$U_\theta = (\frac{\theta + \theta_0}{2}, \infty)$ (any left hand
endpoint between $\theta_0$ and $\theta$ would suffice).  In this
case,
\begin{align*}
Z(U_\theta, x) &= \inf_{\psi > \frac{\theta + \theta_0}{2}} \frac{f(x
  \mid \theta_0)}{f(x \mid \theta)} = 
\inf_{\psi > \frac{\theta +    \theta_0}{2}} \log \frac{\psi}{\theta_0} =
\log \frac{\theta + \theta_0}{2 \theta_0} > 0
\end{align*}
therefore $\sexpectation{Z(U_\theta, x)}{\theta_0} > 0$.

If we pick $\theta < \theta_0$ then pick $U_\theta = (\theta/2,
\frac{\theta + \theta_0}{2})$ and note that 
\begin{align*}
Z(U_\theta, x) &= \begin{cases}
\log \frac{\theta}{2\theta_0} & \text{if $x \leq \frac{\theta}{2}$} \\
\log \frac{x}{\theta_0} & \text{if $\frac{\theta}{2} < x < \frac{\theta + \theta_0}{2}$} \\
+\infty & \text{if $\frac{\theta + \theta_0}{2} \leq x \leq \theta_0$}
\end{cases}
\end{align*} 
and therefore $\sexpectation{Z(U_\theta, x)}{\theta_0} = +\infty$.

Lastly we have to find a compact set $K$ such that
$\sexpectation{Z(\reals_+ \setminus K, x)}{\theta_0} > 0$.  Pick $a >
1$ and consider the interval $[\theta_0/a, a\theta_0]$.  Note that 
\begin{align*}
Z(\reals_+ \setminus [\theta_0/a, a\theta_0], x) &= \begin{cases}
\log \frac{x}{\theta_0} & \text{if $x < \frac{\theta_0}{a}$} \\
\log a & \text{if $\frac{\theta_0}{a} \leq x \leq \theta_0$} \\
\end{cases}
\end{align*}
so integrating,
\begin{align*}
\sexpectation{Z(\reals_+ \setminus [\theta_0/a, a\theta_0],
  x)}{\theta_0} &= 
\frac{1}{\theta_0} \int_0^{\frac{\theta_0}{a}} \log \frac{x}{\theta_0}
\, dx
+ \frac{\theta_0 - \frac{\theta_0}{a}}{\theta_0} \log a \\
&=(\frac{1}{a} \log \frac{1}{a} - \frac{\theta_0}{a}) + \frac{\theta_0 - \frac{\theta_0}{a}}{\theta_0} \log a \\
\end{align*}
Note that the first term goes to $0$ as $a$ goes to $\infty$ and the
second term goes to $\infty$ as $a$ goes to $\infty$ and therefore for
sufficiently large $a$ we have $\sexpectation{Z(\reals_+ \setminus [\theta_0/a, a\theta_0],
  x)}{\theta_0} > 0$.

Note also that as $a$ approaches $1$ the expectation approaches
$-\theta_0 \leq 0$.  In this specific sense if we allow
ourselves to consider regions of parameter space like $(\theta,
\theta_0+\epsilon)$ for $\epsilon > 0$ small, then under sampling we expect the there
is an estimate that is better (more likely) than the true parameter
value.  TODO: Think more carefully about this fact and how to
interpret it; should this disturb us?  Perhaps this shouldn't disturb
us because the thing that allows us to create these regions on which 
$\sexpectation{Z(U, x)}{\theta_0} < 0$ is precisely the fact that we
are allowing ourselves to include $\theta_0 \in U$; without allowing
that we can't create such a set.

The basic
phenomenon in this example can be summarized as:
\begin{itemize}
\item[(i)] Given a single observation $x$ then the
MLE is $x$ with likelihood $1/x$; any $\theta > x$ has strictly smaller
likelihood $1/\theta$ while any $\theta < x$ has likelihood $0$.
\item[(ii)] For any $\theta \geq \theta_0$ we know that $\theta_0$ is
  always a better estimator since we can only observe $x \leq
  \theta_0$ and for these observations $\theta_0$ is always better.
\item[(iii)] For any $\theta < \theta_0$ for any observations $x \leq
  \theta$ we know that $\theta$ is a better estimator than $\theta_0$
  by a finite factor, however for $\theta < x \leq \theta_0$ then
  $\theta_0$ is a infinitely better estimator than $\theta$.
\end{itemize}
\end{examp}

\begin{examp}
This example illustrates the difficulties that can arise in applying
the above results to conclude that an MLE is consistent when the
parameter space is not compact.  Consider a
normal family with parameter $\Theta = \lbrace (\mu, \sigma) \mid
\sigma > 0 \rbrace$ given by
$\frac{1}{\sqrt{2 \pi}\sigma} e^{-(x - \mu)^2/2\sigma}$.  We show that
for any compact $K \subset \reals^2$ we have $\sexpectation{Z(K^c, x)}{\theta_0}
= -\infty$ and therefore Theorem \ref{AsymptoticConsistencyMLE} does
not apply.  In fact we show that for any compact $K$ we have $Z(K^c,
x) = -\infty$.  This follows by noting that any compact $K$ is bounded
hence there exists a value of $\mu$ such that $\lbrace (\mu, \sigma)
\mid \sigma > 0\rbrace \subset K^c$.  Now we see that for such a
$\mu$, 
\begin{align*}
\lim_{\sigma \to 0^+} \frac{f(x\mid \mu_0, \sigma_0)}{f(x \mid \mu,
  \sigma)} &= \lim_{\sigma \to 0^+} \left( \log \sigma - \log \sigma_0
  - \frac{(x - \mu_0)^2}{2\sigma_0} + \frac{(x - \mu)^2}{2\sigma}
\right ) \neq -\infty
\end{align*}
TODO: Fix this argument; it is broken.  The limit is only negative
infinity when $x$ is large enough so that $\lbrace (x, \sigma) \mid
\sigma > 0 \rbrace \subset \Theta \setminus K$.  That should be enough
if we can show that the integral over the rest of the domain is not $+\infty$.

On the other hand, one can compute the MLE explicitly in this case and
verify that it is asymptotically consistent so we have shown that
conditions of the theorem are sufficient but not necessary.
\end{examp}

TODO: The following Theorem only requires upper semi-continuity.
\begin{thm}Let $\xi, \xi_1, \xi_2, \dots$ be i.i.d. parametric
  family with distribution $f(x \mid \theta) \, d \mu$ with respect to measure
  space $(X, \mathcal{X}, \mu)$.  Assume that  $\theta_0$ is fixed and
  define 
\begin{align*}
Z(M, x) &= \inf_{\theta \in M} \log \frac{f(x \mid \theta_0)}{f(x \mid \theta)}
\end{align*}
Assume that for all $\theta \neq \theta_0$ there is an open neighborhood $U_\theta$
such that $\theta \in U_\theta$ and $\sexpectation{Z(U_\theta, \xi)}{\theta_0} >
-\infty$.  Assume $f(x \mid \theta)$ is
  a continuous function of $\theta$ for almost all $x$ with respect to
  $P_{\theta_0}$

If $\Theta$ is not compact, assume that there is a compact $K \subset
\Theta$ such that $\theta_0 \in K$ and $\sexpectation{Z(\Theta
  \setminus K, \xi)}{\theta_0} > 0$.  Then 
\begin{align*}
\lim_{n \to \infty} \hat{\theta}_n = \theta_0
\end{align*}
almost surely with respect to $P_{\theta_0}$.
\end{thm}
\begin{proof}
We show that for all $\theta \neq \theta_0$ there exists a
neighborhood $\theta \in U_\theta$ such that
$\sexpectation{Z(U_\theta, \xi)}{\theta_0} > 0$ and then apply the
previous Theorem \ref{AsymptoticConsistencyMLE}.

Pick $\theta \neq \theta_0$ and assume that we have an open
neighborhood $U_\theta$ with $\theta \in U_\theta$ and $\sexpectation{Z(U_\theta, \xi)}{\theta_0} >
-\infty$.  If $\sexpectation{Z(U_\theta, \xi)}{\theta_0} > 0$ then
have found a suitable neighborhood so we may assume
$\sexpectation{Z(U_\theta, \xi)}{\theta_0} \leq 0$ as well (we really
just need to assume that the value if finite a bit later in the proof).  Now for each $n \in \naturals$ pick a closed ball $U^n_\theta =
B(\theta, r_n) \subset U_\theta$ such that $r_n \leq \frac{1}{n}$ and
$r_n$ are non-increasing.  Furthermore because $U^{n+1}_\theta \subset
U^n_\theta$ we have for fixed $x$, $Z(U^n_\theta, x)$ is increasing in
$n$.

Now assume that we have an $x$ such that $f(x \mid \theta)$ is
continuous.  This implies $\log \frac{f(x \mid \theta_0)}{f(x \mid
  \theta)}$ is continuous as well.  This continuity coupled with the
compactness of $U^n_\theta$ implies that there exists a $\theta_n(x)
\in U^n_\theta$ such that $Z(U^n_\theta, x) = \log \frac{f(x \mid \theta_0)}{f(x \mid
  \theta_n(x))}$.  Clearly we have $\cap_n U^n_\theta =
\lbrace \theta \rbrace$ and this implies $\lim_{n \to \infty}
\theta_n(x) = \theta$.  Again by continuity we get
\begin{align*}
\lim_{n \to \infty} Z(U^n_\theta, x) &= \lim_{n \to \infty}\log \frac{f(x \mid \theta_0)}{f(x \mid
  \theta_n(x))} = \log \frac{f(x \mid \theta_0)}{f(x \mid
  \theta)}
\end{align*}

Now because $U^n_\theta \subset U_\theta$ we have $Z(U^n_\theta, x)
\geq Z(U_\theta, x)$ and $\sexpectation{Z(U_\theta, \xi)}{\theta_0}$
is finite, we may apply Fatou's Lemma (Theorem
\ref{Fatou})
\begin{align*}
\liminf_{n \to \infty} \sexpectation{ Z(U^n_\theta, x)}{\theta_0} -
\sexpectation{Z(U_\theta, x)}{\theta_0} &= \liminf_{n \to \infty} \sexpectation{ Z(U^n_\theta, x) - Z(U_\theta,
x)}{\theta_0} \\
&\geq \sexpectation{ \lim_{n \to \infty}  \left (Z(U^n_\theta, x) - Z(U_\theta,
x) \right )}{\theta_0} \\
&=\sexpectation{ \log \frac{f(x \mid \theta_0)}{f(x \mid
  \theta)}}{\theta_0} -
\sexpectation{Z(U_\theta, x)}{\theta_0} 
\end{align*}
Cancelling the (finite) common term $\sexpectation{Z(U_\theta, x)}{\theta_0}$ we get
\begin{align*}
\liminf_{n \to \infty} \sexpectation{ Z(U^n_\theta, x)}{\theta_0} \geq \sexpectation{ \log \frac{f(x \mid \theta_0)}{f(x \mid
  \theta)}}{\theta_0}  > 0
\end{align*}
where the last inequality follows from the positivity of relative
entropy (Lemma \ref{GibbsInequality}).  Now by this inequality we can
find an $N >0$ such that $\sexpectation{ Z(U^n_\theta, x)}{\theta_0} >
0$ for all $n \geq N$, but in particular there is a single
neighborhood $U^N_\theta$ with this property.  
\end{proof}
The technical conditions above are sufficient to prove asymptotic
efficient of MLEs but it is certainly not necessary.

TODO: Example showing consistency without conditions.

TODO: Note a different condition that suffices (Martingale proof:
Schervish Lemma 7.83)

Maximum likelihood estimators are asymptotically normal under certain
circumstances.  If is unfortunate that any precise statement of those
circumstances is technical and verbose.  It is also unfortunate that
there is no definitive characterization of asymptotic normality as a
set of necessary and sufficient conditions.  Instead there are a
number of sufficient conditions available with different levels of
generality and sophistication.  TODO: This is equally true about
asymptotic consistency and asymptotic results in general; move this
comment to an appropriate place and generalize.

Before stating a rather classical version of such a result let's consider the
case of a scalar parameter in a somewhat heuristic fashion.  If we
assume that we have a consistent MLE such that $\hat{\theta}_n \toas
\theta_0$ and we want to prove that $\sqrt{n}(\hat{\theta}_n -
\theta_0 ) \todist N(0, \sigma^2)$ for an appropriate $\sigma$.  We assume that $f(x \mid
\theta)$ is twice continuously differentiable as a function of
$\theta$; under these conditions the
maximum of the likelihood implies a vanishing derivative
\begin{align*}
\frac{\partial}{\partial \theta} \ell_n (\xi \mid \hat{\theta}_n) &= 0
\end{align*} 
If we apply the mean value theorem to the function
$\frac{\partial}{\partial \theta} \ell_n (\xi \mid \theta)$ to conclude
that there is a value $\theta^*_n$ that lies between $\hat{\theta}_n$
and $\theta_0$ such that 
\begin{align*}
\frac{\frac{\partial}{\partial \theta} \ell_n (\xi \mid \hat{\theta}_n)
  - \frac{\partial}{\partial \theta} \ell_n (\xi \mid
  \theta_0)}{\hat{\theta}_n - \theta_0} &= \frac{\partial^2}{\partial \theta^2}\ell_n (\xi \mid \theta^*_n)
\end{align*}
or rearranging terms to set up ourselves up to take advantage of the
Central Limit Theorem (ignore the possiblity that the denominator vanishes):
\begin{align*}
\sqrt{n}(\hat{\theta}_n - \theta_0) &=
-\frac{\sqrt{n}
 \frac{\partial}{\partial \theta} \ell_n (\xi \mid
  \theta_0)}{\frac{\partial^2}{\partial \theta^2}\ell_n (\xi \mid \theta^*_n)} 
\end{align*}
Now consider the numerator $\mu = \sexpectation{\log f(\xi \mid
  \theta_0)}{\theta_0} = 0$ and variance $i(\theta_0) = \sexpectation{\log^2 f(\xi \mid
  \theta_0)}{\theta_0}$ and we can apply the Central Limit Theorem to see
\begin{align*}
\frac{1}{\sqrt{n}} \ell^\prime_n (\xi \mid \theta_0) &= \frac{1}{\sqrt{n}}
\sum_{i=1}^n \frac{\partial}{\partial \theta} \log f(\xi_i \mid
\theta_0) \todist N(0, i(\theta_0))
\end{align*}
This looks quite promising but there is a factor of
$\frac{1}{\sqrt{n}}$ that was added that will have to be addressed.

Now if we consider the denominator things don't look so good; however a small
modification seems amenable to analysis.  If we consider
$\frac{\partial^2}{\partial \theta^2}\ell_n (\xi \mid \theta_0)$, then
we see that the Weak Law Of Large Numbers tells us that
\begin{align*}
-\frac{1}{n}\frac{\partial^2}{\partial \theta^2}\ell_n (\xi \mid \theta_0) &=
-\frac{1}{n}\sum_{i=1}^n \frac{\partial^2}{\partial \theta^2}\ell_n (\xi_i \mid
\theta_0)  \toprob \sexpectation{-\frac{\partial^2}{\partial \theta^2}
  \log f (\xi \mid
\theta_0) }{\theta_0} =  i(\theta_0)
\end{align*}
Moreover, the factor of $\frac{1}{n}$ that we needed here to apply the
Law of Large Numbers
cancelled exactly with our use of $\frac{1}{\sqrt{n}}$ in the Central
Limit Theorem application so that our Taylor expansion can be written as
\begin{align*}
\sqrt{n}(\hat{\theta}_n - \theta_0) &=
-\frac{
 \frac{\partial}{\partial \theta} \ell_n (\xi \mid
  \theta_0)}{\sqrt{n}} \cdot 
\frac{n}{\frac{\partial^2}{\partial
    \theta^2}\ell_n (\xi \mid \theta_0)} \cdot
\frac {\frac{\partial^2}{\partial
    \theta^2}\ell_n (\xi \mid \theta_0)}
{\frac{\partial^2}{\partial \theta^2}\ell_n (\xi \mid \theta^*_n)} 
\end{align*}
and we are in position to use Slutsky's Lemma to extend the asymptotic
normality of the first factor to $\sqrt{n}(\hat{\theta}_n - \theta_0)$.
The rub is that we have
a term 
\begin{align*}
\frac{\frac{\partial^2}{\partial \theta^2}\ell_n (\xi \mid \theta_0) }{\frac{\partial^2}{\partial \theta^2}\ell_n (\xi \mid \theta_n^*) }
\end{align*}
to understand.  By consistency of the estimator we know that
$\theta_n^* \toas \theta_0$ we might hope that this term converges to
$1$ (at least in probability).  In fact additional smoothness
assumptions on $f$ are sufficient to guarantee that this is the case;
the expression of these smoothness constraints is what provides the
complexity to statements of asymptotic normality of MLEs.
When that is shown, then keeping track of the factors of $i(\theta_0)$
we see that Slutsky's Lemma will tell us that
\begin{align*}
\sqrt{n}(\hat{\theta}_n - \theta_0) \todist N(0, i(\theta_0)^{-1})
\end{align*}

In the following Theorem we capture all the varied assumptions that
are required to make an argument like the above rigorous; the result
is also stated for multivariate parameters.  The details of the proof
are organized a bit differently than the outline of the scalar case
given above (e.g. dealing with boundaries in parameter space) but the
main points of the proof remain the same:
\begin{itemize}
\item[1)] Taylor expand the likelihood function around ${\theta}_0$
\item[2)] Use the Central Limit Theorem to prove convergence of the
  first derivative term at $\theta_0$
\item[3)] Use the Weak Law of Large Numbers to prove convergence of
  the second derivative term at $\theta_0$
\item[4)] Use asymptotic consistency of $\hat{\theta}_n$ and bounds on the variation of the second derivative to
  conclude that the difference between the second derivatives at
  $\theta_0$ and $\hat{\theta}_n$ go to zero in probability.
\item[5)] Use Slutsky's Lemma to glue all the pieces together.
\end{itemize}


\begin{thm}Let $\xi, \xi_1, \xi_2, \dots$ be i.i.d. parametric
  family with distribution $f(x \mid \theta) \, d \mu$ with respect to measure
  space $(X, \mathcal{X}, \mu)$ with $\Theta \subset \reals^k$
  for some $k > 0$.  Assume
\begin{itemize}
\item[(i)]$\hat{\theta}_n \toprob \theta_0$ in $P_{\theta_0}$ for every
  $\theta_0 \in \Theta$.
\item[(ii)]$f(x \mid \theta)$ has continuous second partial
  derivatives with respect to $\theta$ and that differentiation can be
  passed under the integral sign
\item[(iii)]there exists $H_r(x, \theta)$ such that for each $\theta_0
  \in int(\Theta)$ and each $k,j$,
\begin{align*}
\sup_{\norm{\theta - \theta_0} \leq r}
\abs{\frac{\partial^2}{\partial\theta_k \partial\theta_j} \log f(x
  \mid \theta_0) - \frac{\partial^2}{\partial\theta_k \partial\theta_j} \log f(x
  \mid \theta)} \leq H_r(x, \theta_0)
\end{align*}
with $\lim_{r \to 0} \sexpectation{H_r(\xi, \theta_0)}{\theta_0} = 0$.
\item[(iv)]the Fisher information matrix $\mathcal{I}_\xi(\theta_0)$ is
  finite and nonsingular.
\end{itemize}
Then 
\begin{align*}
\sqrt{n} \left(\hat{\theta}_n - \theta_0\right ) \todist N(0, \mathcal{I}^{-1}_\xi(\theta_0))
\end{align*}
\end{thm}
\begin{proof}
We start with

Claim 1: $\frac{1}{\sqrt{n}} D_{\hat{\theta}_n} \ell_n(\sample{\xi} \mid \theta)
\toprob 0$

One might jump to the conclusion that $D_{\hat{\theta}_n} \ell_n(\sample{\xi}
\mid \theta) = 0$ everywhere because $\hat{\theta}_n$ is a maximum,
however there are some details about handling the issue of boundaries on
$\Theta$.  One does know that $D_{\hat{\theta}_n} \ell_n(
\sample{\xi} \mid \theta) = 0$ when $\hat{\theta}_n \in \interior(\Theta)$ but
there is the possibility that some $\hat{\theta}_n$ lies on the
boundary of $\Theta$ and the derivative might not vanish in this case.
To handle the boundary effects, first we know that $\theta_0 \in
\interior(\Theta)$  and therefore there is an open neighborhood
$\theta_0 \in U \subset \interior(\Theta)$.  By the vanishing of the
derivative at any maximum in the interior, we know
\begin{align*}
\frac{1}{\sqrt{n}} D_{\hat{\theta}_n} \ell_n(\sample{\xi} \mid \theta) &=
\frac{1}{\sqrt{n}} D_{\hat{\theta}_n} \ell_n(\sample{\xi} \mid \theta)
\characteristic{\hat{\theta}_n  \in U} + \frac{1}{\sqrt{n}} D_{\hat{\theta}_n} \ell_n(\sample{\xi} \mid \theta)
\characteristic{\hat{\theta}_n  \notin U} \\
&= \frac{1}{\sqrt{n}} D_{\hat{\theta}_n} \ell_n(\sample{\xi} \mid \theta)
\characteristic{\hat{\theta}_n  \notin U}
\end{align*}
Using the fact that $\hat{\theta}_n \toprob
\theta_0$ allows us to conclude that 
\begin{align*}
\lim_{n \to \infty} \sprobability{\hat{\theta}_n  \notin U}{\theta_0} &= 0
\end{align*}
so in particular,
\begin{align*}
\lim_{n \to \infty} \sprobability{ \frac{1}{\sqrt{n}} D_{\hat{\theta}_n} \ell_n(\sample{\xi}
  \mid \theta) \characteristic{\hat{\theta}_n  \notin U} = 0}{\theta_0} &=0
\end{align*}
Putting these two pieces of information together we see
\begin{align*}
 \frac{1}{\sqrt{n}} D_{\hat{\theta}_n} \ell_n(\sample{\xi} \mid \theta) &=
 \frac{1}{\sqrt{n}} D_{\hat{\theta}_n} \ell_n(\sample{\xi} \mid \theta)
\characteristic{\hat{\theta}_n  \notin U} \toprob 0
\end{align*}

Now we derive a quadratic approximation to the likelihood by using a
Taylor expansion (actually just the Mean Value Theorem) of
$D_\theta \ell_n(\sample{\xi} \mid \theta)$ around ${\theta}_0$.  Once
again there is the issue of boundaries but moreover  the domain
$\Theta$ is not convex so the Taylor series only applies cleanly when
$\hat{\theta}_n$ belongs to a ball around $\theta_0$.  To handle this,
pick an $R > 0$ such that we have $B(\theta_0; R) \subset
\interior(\Theta)$.  In this case, when $\norm{\hat{\theta}_n -
  \theta_0} < R$ then we know there exists a $\theta^*_n$ between $\theta_0$ and
$\hat{\theta}_n$ such that 
\begin{align*}
D_{\hat{\theta}_n} \ell_n(\sample{\xi} \mid \theta) - D_{\theta_0} \ell_n(\sample{\xi}
\mid \theta) &= D^2_{\theta^*_n} \ell_n(\sample{\xi} \mid \theta) \cdot
(\hat{\theta}_n - \theta_0)
\end{align*}
As it turns out what happens when $\norm{\hat{\theta}_n - \theta_0}
\geq R$ won't matter since it is an event that occurs with vanishingly
small probability as $n$ grows.  Accordingly, we define
\begin{align*}
\Delta_n &= \begin{cases}
 D^2_{\theta^*_n} \ell_n(\sample{\xi} \mid \theta) & \text{when $\norm{\hat{\theta}_n -
  \theta_0} < R$} \\
0 & \text{when $\norm{\hat{\theta}_n -
  \theta_0} \geq R$} \\
\end{cases}
\end{align*}
TODO: Do we need to justify measurability here...

Claim 2: $ \frac{1}{\sqrt{n}}(D_{\theta_0} \ell_n(\sample{\xi}
\mid \theta) + \Delta_n \cdot (\hat{\theta}_n - \theta_0)) \toprob 0$

Pick an $\epsilon > 0$.  From the definition of $\Delta_n$ we have
\begin{align*}
 \frac{1}{\sqrt{n}} (D_{\theta_0} \ell_n(\sample{\xi}
\mid \theta) + \Delta_n \cdot (\hat{\theta}_n - \theta_0))
&= \begin{cases}
 \frac{1}{\sqrt{n}} (D_{\hat{\theta}_n} \ell_n(\sample{\xi}
\mid \theta) & \text{when $\norm{\hat{\theta}_n -
  \theta_0} < R$} \\
 \frac{1}{\sqrt{n}} (D_{{\theta}_0} \ell_n(\sample{\xi}
\mid \theta) & \text{when $\norm{\hat{\theta}_n -
  \theta_0} \geq R$} \\
\end{cases}
\end{align*}
and therefore
\begin{align*}
&\lim_{n \to \infty} \sprobability{ \frac{1}{\sqrt{n}} (D_{\theta_0} \ell_n(\sample{\xi}
\mid \theta) + \Delta_n \cdot (\hat{\theta}_n - \theta_0)) >
\epsilon}{\theta_0} \\
&= 
\lim_{n \to \infty} \sprobability{  \frac{1}{\sqrt{n}} D_{\hat{\theta}_n} \ell_n(\sample{\xi}
\mid \theta) > \epsilon;\norm{\hat{\theta}_n -
  \theta_0} < R}{\theta_0} \\
&+ \lim_{n \to \infty}\sprobability{  \frac{1}{\sqrt{n}} D_{{\theta}_0} \ell_n(\sample{\xi}
\mid \theta) > \epsilon;\norm{\hat{\theta}_n -
  \theta_0} \geq R}{\theta_0} \\
&\leq 
\lim_{n \to \infty} \sprobability{ \frac{1}{\sqrt{n}} D_{\hat{\theta}_n} \ell_n(\sample{\xi}
\mid \theta) > \epsilon}{\theta_0} + \lim_{n \to \infty}\sprobability{\norm{\hat{\theta}_n -
  \theta_0} \geq R}{\theta_0} = 0
\end{align*}
where we have used Claim 1 and the weak consistency of the estimator
$\hat{\theta}_0$.

Claim 3: $\frac{1}{n}\Delta_n \toprob -\mathcal{I}_\xi(\theta_0)$

Write 
\begin{align*}
\frac{1}{n}\Delta_n &= \frac{1}{n}D^2_{\theta_0} \ell_n(\sample{\xi} \mid
\theta)  \characteristic{\norm{\hat{\theta}_n -
  \theta_0} < R} \\
&+
(D^2_{\theta^*_n} \ell_n(\sample{\xi} \mid \theta) - D^2_{\theta_0} \ell_n(\sample{\xi}
\mid \theta))  \characteristic{\norm{\hat{\theta}_n -
  \theta_0} < R} \\
\end{align*}
and we address the convergence of each of the summands.  First note
that by weak consistency of the estimator $\hat{\theta}_n$ we have
$\characteristic{\norm{\hat{\theta}_n -
  \theta_0} < R} \toprob 1$.
By the Weak Law of Large Numbers and the fact we can exchange
derivatives and expectations we have
\begin{align*}
\frac{1}{n}D^2_{\theta_0} \ell_n(\sample{\xi} \mid
\theta) &= \frac{1}{n}\sum_{i=1}^n D^2_{\theta_0} \log f(\xi_i \mid
\theta) \toprob \sexpectation{D^2_{\theta_0} \log f(\xi \mid
\theta)}{\theta_0} = -\mathcal{I}_\xi(\theta_0)
\end{align*}
and therefore by Corollary
\ref{ConvergenceInProbabilityAndAlgebraicOperations} to the Continuous
Mapping Theorem we can combine these facts to conclude
\begin{align*}
\frac{1}{n}D^2_{\theta_0} \ell_n(\sample{\xi} \mid
\theta) \characteristic{\norm{\hat{\theta}_n -
  \theta_0} < R} &\toprob -\mathcal{I}_\xi(\theta_0)
\end{align*}
We turn attention to the error term which we show is $o_P(1)$.
Let $\epsilon > 0$ be given.  Pick any $0 < r \leq R$ such that
$\sexpectation{H_r(\xi, \theta_0)}{\theta_0} < \frac{\epsilon}{2}$.
Again applying the Weak Law of Large Numbers 
\begin{align*}
\frac{1}{n} \sum_{i=1}^n H_r(\xi_i, \theta_0) \toprob \sexpectation{H_r(\xi, \theta_0)}{\theta_0} < \frac{\epsilon}{2}
\end{align*}
and therefore
\begin{align*}
\lim_{n \to \infty} \sprobability{\frac{1}{n} \sum_{i=1}^n H_r(\xi_i,
  \theta_0) < \epsilon}{\theta_0} &\leq \lim_{n \to \infty} \sprobability{\abs{\frac{1}{n} \sum_{i=1}^n H_r(\xi_i,
  \theta_0) - \sexpectation{H_r(\xi, \theta_0)}{\theta_0}} <
\frac{\epsilon}{2}}{\theta_0} = 0
\end{align*}
Now apply this fact to get a bound on each entry of the Hessian matrix
\begin{align*}
&\lim_{n \to \infty} \sprobability{\frac{1}{n}\abs{D^2_{\theta^*_n, j,k} \ell_n(\sample{\xi} \mid
\theta) - D^2_{\theta_0, j,k} \ell_n(\sample{\xi} \mid
\theta) } \characteristic{\norm{\theta^*_n - \theta_0} < R} <
\epsilon}{\theta_0} \\
&\lim_{n \to \infty} \sprobability{\frac{1}{n}\abs{D^2_{\theta^*_n, j,k} \ell_n(\sample{\xi} \mid
\theta) - D^2_{\theta_0, j,k} \ell_n(\sample{\xi} \mid
\theta) } \characteristic{\norm{\theta^*_n - \theta_0} < r} <
\epsilon}{\theta_0} \\
&+ \lim_{n \to \infty} \sprobability{\frac{1}{n}\abs{D^2_{\theta^*_n, j,k} \ell_n(\sample{\xi} \mid
\theta) - D^2_{\theta_0, j,k} \ell_n(\sample{\xi} \mid
\theta) } \characteristic{r \leq \norm{\theta^*_n - \theta_0} < R} <
\epsilon}{\theta_0} \\
&\leq
\lim_{n \to \infty} \sprobability{\frac{1}{n} \sum_{i=1}^n H_r(\xi_i,
  \theta_0) < \epsilon}{\theta_0} + 
\lim_{n \to \infty} \sprobability{ \characteristic{r \leq \norm{\theta^*_n -
      \theta_0} < R} }{\theta_0} \\
&= 0
\end{align*}
and therefore we have shown $\frac{1}{n} (D^2_{\theta^*_n, j,k} \ell_n(\sample{\xi} \mid
\theta) - D^2_{\theta_0, j,k} \ell_n(\sample{\xi} \mid
\theta) ) \characteristic{\norm{\theta^*_n - \theta_0} < R} \toprob 0$.

Claim 4: $ \frac{1}{\sqrt{n}} D_{\theta_0} \ell_n(\sample{\xi}
\mid \theta) \todist N(0, \mathcal{I}_\xi(\theta_0))$

First note that 
\begin{align*}
\frac{1}{n} D_{\theta_0} \ell_n(\sample{\xi}
\mid \theta) &= \frac{1}{n} \sum_{i=1}^n D_{\theta_0} \log f(\xi_i
\mid \theta) \toprob \sexpectation{D_{\theta_0} \log f(\xi \mid \theta)}{\theta_0}
\end{align*}
since we have an i.i.d. sum and we can apply the Weak Law of Large
Numbers.  Because we assume we can exchange expectations and
derivatives for any partial derivative 
\begin{align*}
\sexpectation{\frac{\partial}{\partial \theta_i} \log f(\xi_i \mid
  \theta)}{\theta_0} &= \int \frac{\partial}{\partial \theta_i} \log f(x \mid
  \theta) f(x \mid \theta_0) dx = \int \frac{\partial}{\partial \theta_i}  f(x \mid
  \theta_0) \, dx = \frac{\partial}{\partial \theta_i} \int f(x \mid
  \theta_0) \, dx = 0
\end{align*}
and thus we conclude $\frac{1}{n} D_{\theta_0} \ell_n(\sample{\xi}
\mid \theta) \toprob 0$.  We can also calculate the covariance matrix of the random variable
$D_{\theta_0} \log f(\xi \mid \theta)$ as $\mathcal{I}_\xi(\theta_0)$.

Now we simply apply the multivariate Central Limit Theorem and the
Claim is proven.

Claim 5: $ \frac{1}{\sqrt{n}} D_{\theta_0} \ell_n(\sample{\xi}
\mid \theta) -  \sqrt{n} \mathcal{I}_\xi(\theta_0) \cdot
(\hat{\theta}_n - \theta_0) \toprob 0$

We already know from Claim 2 that $ \frac{1}{\sqrt{n}} (D_{\theta_0} \ell_n(\sample{\xi}
\mid \theta) + \Delta_n \cdot
(\hat{\theta}_n - \theta_0)) \toprob 0$ so it suffices to show that $\frac{1}{\sqrt{n}}\Delta_n \cdot
(\hat{\theta}_n - \theta_0) +  \sqrt{n} \mathcal{I}_\xi(\theta_0) \cdot
(\hat{\theta}_n - \theta_0) \toprob 0$
as well.  

By Claim 4 and Lemma \ref{WeakConvergenceImpliesTight}, we know that 
$ \frac{1}{\sqrt{n}} D_{\theta_0} \ell_n(\sample{\xi} \mid \theta)$ is
tight.  Together with Claim 2 this tells us that $\frac{1}{\sqrt{n}}\Delta_n \cdot (\hat{\theta}_n -
\theta_0)$ is $o_P(1) + O_P(1)$ hence is tight as well (Lemma
\ref{AlgebraOfStochasticConvergence}).  
Claim 3 and the invertibility of $\mathcal{I}_\xi(\theta_0)$ allows us
to apply  Lemma \ref{InvertMatrixInProbability} to conclude that $\frac{1}{\sqrt{n}}\Delta_n \cdot (\hat{\theta}_n -
\theta_0)$ is tight.
Now by Claim 3 and Lemma
\ref{AlgebraOfStochasticConvergence} we
can conclude that $ (\frac{1}{n}\Delta_n + \mathcal{I}_\xi(\theta_0))
\cdot \sqrt{n} (\hat{\theta}_n - \theta_0) \toprob 0$ as required.


Now when we combine Claim 4 and Claim 5 with Slutsky's Lemma (Theorem
\ref{Slutsky}) we conclude that $\mathcal{I}_\xi(\theta_0)
\cdot \sqrt{n} (\hat{\theta}_n - \theta_0) \todist N(0,
\mathcal{I}_\xi(\theta_0)^{-1})$.  Because $\mathcal{I}_\xi(\theta_0)$ is invertible and matrix multiplication is
continuous, the Continuous Mapping Theorem allows us to conclude 
$\sqrt{n} 
(\hat{\theta}_n - \theta_0) \todist \mathcal{I}_\xi(\theta_0)^{-1}
N(0, \mathcal{I}_\xi(\theta_0)) = N(0, \mathcal{I}_\xi(\theta_0)^{-1})$.
and we are done.
\end{proof}

As a side effect of having shown that an MLE may be asymptotically
normal we computed its asymptotic variance.  Now it is intuitively
clear that given two estimators that are equal in every other way the
one with a smaller variance is to be preferred.  So a natural question
to ask is whether a variance of $\mathcal{I}_\xi(\theta)^{-1}$ is a good by some objective
standard.  It is in fact optimal.

\begin{thm}[Cramer-Rao Lower Bound]\label{CRLB}blah blah
\end{thm}

TODO: Binomial estimation
Ideas:  Frequentist vs. Bayesian.  Two sampling approaches: sample
fixed n vs. sequentially sample till n successes.  Same means but
different variances in frequentist approaches (failure of the
likelihood principle) but same in Bayesian.
The normal approximation and confidence intervals.  Discuss issues
with coverage.  Ratio of binomial (e.g. Koopman and the Bayesian approach).

TODO: Maybe a good idea to cover logistic regression as an application
of MLE.  Expressing regression as an MLE: requires a distribution
assumption on the residual and then regression becomes a location
scale family.  I don't see that the standard proofs of consistency and
normality work in these cases though (since the observations now are
independent but have differing distributions..)  I think this is an
accurate state of affairs; there are direct proofs of MLE asymptotic
properties for GLMs (and I suppose GAMs).  See also Hjort and Pollard,
``Asymptotics for minimisers of convex processes''  As for intuition
about why i.i.d. should not be necessary to prove asymptotic results
recall that the Weak Law of Large Numbers doesn't require i.i.d. but
only uniform integrability and that the Lindeberg C.L.T. applies
without full blown i.i.d.  It'll be an interesting exercise to see how
the asymptotic theory of logistic regression unfolds.

\subsection{Logistic Regression}
To motivate the logistic regression, assume that we have a binomial
random variable $y \sim B(n,p)$ and consider the maximum likelihood
estimate of the parameter $p$.  Introduce the log odds $\theta =
logit(p) = \ln(p/(1-p))$ rewrite the binomial
distribution in terms of $\theta$.
\begin{align}
\binom{n}{m}p^m(1-p)^{n-m} & = e^{\ln(\binom{n}{m})}
e^{\ln(p^m)}e^{\ln((1-p)^{n-m}} \\
& = e^{\ln(\binom{n}{m}) + m \ln(p/(1-p)) +
  n \ln(1-p)} \\
& = e^{\ln(\binom{n}{m}) + m \ln(p/(1-p)) -  n \ln(1 + p/(1-p))} \\
& = e ^ {\ln(\binom{n}{m}) + m \theta - n \ln(1+e^\theta)}
\end{align}
This allows us to write the loglikelihood function in terms of the
parameter $\theta$ as:
$$
l(\theta; y) = y\theta - n\ln(1+e^\theta) + \ln\binom{n}{y}
$$
and then it is easy to get the score and information functions
\begin{align}
s(\theta; y) & = \frac{\partial}{\partial \theta} l(\theta;y) = y -
\frac{n e^\theta}{1+e^\theta}= y - np \\
i(\theta; y) & = -\frac{\partial}{\partial \theta} s(\theta;y) = np(1-p)
\end{align}


\section{Brownian Motion}
We begin by studying the one dimensional version of Brownian motion.
\begin{defn}A real-valued stochastic process $B_t$ on $[0, \infty)$ is said to be a
  \emph{Brownian motion} at $x \in \reals$ if 
\begin{itemize}
\item[(i)]$B(0) = x$
\item[(ii)]For all times $0 \leq t_1 \leq t_2 \leq \cdots \leq t_n$
  the increments $B_{t_2} - B_{t_1}, B_{t_3} - B_{t_2}, \dots, B_{t_n}
  - B_{t_{n-1}}$ are independent random variables
\item[(iii)]For all $0 \leq s < t$, the increment $B_t - B_s$ is normally distributed with
  expectation zero and variance $t - s$.
\item[(iv)]Almost surely the sample path $B(t)$ is continuous.
\end{itemize}
\end{defn}

The existence of Brownian motion is a non-trivial fact that was first
proved by Norbert Weiner.  Here we present a construction by Paul Levy
whose details are worth understanding because many properties of
Brownian motion follow from them.
\begin{thm}Standard Brownian motion exists.
\end{thm}
\begin{proof}
Before we construct Brownian motion on the entire real line, we
construct it on the interval $[0,1]$ (that is to say we only construct
the values $B(t)$ for $t \in [0,1]$).  
To motivate the construction of Brownian motion, we take as our
driving goals the fact that we have to construct a continuous random
path $B(x)$ for which the distribution of $B(x)$ for fixed $x \in
[0,1]$ is $N(0, x)$.  The approach to the construction is to proceed
iteratively such that at stage $n$ of the iteration we have a
piecewise linear approximation $B_n(x)$ with the distribution of $B_n(x)$ being
$N(0,x)$ at the points $x = 0, 1/2^n, \dots, 1$. The set of rational
  numbers of the form $\frac{k}{2^n}$ for $n \geq 0$ and $0 \leq k
  \leq 2^n$ is known as the \emph{dyadic rationals} in $[0,1]$.  We will sometime
  have need for the notation
\begin{align*}
\mathcal{D}_n = \lbrace \frac{k}{2^n} \mid 0 \leq k \leq 2^n \rbrace
\end{align*}
and $\mathcal{D} = \cup_{n=0}^\infty \mathcal{D}_n$ when discussing
the dyadic rationals.  To support the construction, we need a
probability space which we assume to be $([0,1], \mathcal{B}([0,1]),
\lambda)$.  As a concrete source of randomness, for
each $d \in \mathcal{D}$ let $Z_d$ be an $N(0,1)$ random variable
with the $Z_d$ independent (we may do this by Lemma
\ref{ExistenceCountableIndependentRandomVariables}).

It is worth walking through the first couple of iterations in rather
gory detail to reinforce the idea and to convince the reader that the
construction really is determined by the vague prescription given
above.  So our first goal is to construct a random piecewise linear
path that is constant at $x=0$ and has distribution $N(0,1)$ at
$x=1$.  The simplest idea turns out to be the right one to get
started: define $B_0(x) = x Z_1$.  Then $\variance{B_0(x)} = x^2$
which is correct for $x \in \lbrace 0,1\rbrace $ but nowhere in between.  The critical
point is the $x^2 < x$ for all $x \in (0,1)$ so we have \emph{too
  little} variance.  Getting a bit more variance is easy whereas we'd
be rather doomed if we already had too much.  

So recall the next step was to get the correct variance at the points
$\lbrace 0, 1/2, 1\rbrace$ not just at the points $\lbrace 0,1\rbrace$.  By the above,
$\variance{B_0(1/2)} = 1/4$ but we require that $B_1(1/2) = 1/2$ so we
need to add a random variable with distribution $N(0, 1/4)$ at $x=1/2$
satisfy our goal.
But since we had the
correct variance at ${0,1}$ we have make sure not to add any more
at either of those points.  This motivates the introduction of the function
\begin{align*}
\Delta(x) &= \begin{cases}
2x & \text{for $0 \leq x \leq \frac{1}{2}$} \\
2 - 2x & \text{for $\frac{1}{2} < x \leq 1$} \\
0 & \text{otherwise} \\
\end{cases}
\end{align*}
Now if we define $B_1(x) = B_0(x) + \frac{1}{2}\Delta(x) Z_{1/2}$
the we see that $B_1(1/2)$ is a sum of two $N(0,1/4)$ random variables
hence is $N(0,1/2)$ as desired.  Because $\Delta(0) = \Delta(1) =
0$, we have $B_1(0) = B_0(0)$ and $B_1(1) = B_0(1)$ so these two are
still in good shape.  

TODO: Make the following into an exercise.
Just to turn the crank one more time, by the definition of $B_1(x)$ we
can easily see that since in general $B_1(x)$ is an $N(0, x^2 +
\frac{1}{2}\Delta_{0,0}(x))$ random variable,
\begin{align*}
\variance{B_1(1/4)} &= \frac{1}{16} + \frac{1}{16} = 1/8 = 1/4 - 1/8 \\
\variance{B_1(3/4)} &= \frac{9}{16} + \frac{1}{16} = 5/8 = 3/4 - 1/8 \\
\end{align*}
so in both cases we need to add a variance of $1/8$ at the points
$\lbrace 1/4, 3/4 \rbrace$ without changing things at $\lbrace 0, 1/2,
1\rbrace$.  Mimicing what we have already done, we now need a ``double
sawtooth'' to modify $B_1(x)$ into $B_2(x)$.  For reasons that we'll
explain later we actually break the modification into two pieces: one
for the interval $(0, 1/2)$ and one for the interval $(1/2, 1)$.  So
define,
\begin{align*}
\Delta_{1,0} (x) &= \Delta(2x) = \begin{cases}
4x & \text{for $0 \leq x \leq \frac{1}{4}$} \\
2 - 4x & \text{for $\frac{1}{4} < x \leq \frac{1}{2}$} \\
0 & \text{otherwise} \\
\end{cases}
\end{align*}
and
\begin{align*}
\Delta_{1,1} (x) &= \Delta(2x - 1) = \begin{cases}
4x -2 & \text{for $\frac{1}{2} \leq x \leq \frac{3}{4}$} \\
4 - 4x & \text{for $\frac{3}{4} < x \leq 1$} \\
0 & \text{otherwise} \\
\end{cases}
\end{align*}
Now if we define $B_2(x) = B_1(x) + \frac{1}{\sqrt{8}}
(\Delta_{1,0}(x) Z_{/1/4} + \Delta_{1,1}(x) Z_{3/4})$, then we have
  added the appropriate variance of $1/8$ at $x=1/4$ and $x=3/4$.

To state the general construction, we first generalize the definition
of our sawtooth functions.  For $n > 0$ and $k=0, \cdots, 2^n -1$, we
define 
\begin{align*}
\Delta_{n,k} (x) &= \Delta(2^nx- k) = \begin{cases}
2^{n+1}x -2k & \text{for $\frac{2k}{2^{n+1}} \leq x \leq \frac{2k+1}{2^{n+1}}$} \\
2k + 2 - 2^{n+1}x & \text{for $\frac{2k+1}{2^{n+1}} < x \leq \frac{2k+2}{2^{n+1}}$} \\
0 & \text{otherwise} \\
\end{cases}
\end{align*}
With the definition we can complete the induction definition.  So our definition
of $B_n(x)$ can be completed.  We point out that $\Delta_{0,0}(x) =
\Delta(x)$ so the definition below is compatible with our definition
of $B_1(x)$ and $B_2(x)$ above:
\begin{align*} 
B_0(x) &= x Z_1 \\
B_n(x) &= B_{n-1}(x) + \frac{1}{\sqrt{2^{n+1}}} \sum_{k=0}^{2^{n-1} -1}
\Delta_{n-1,k}(x) Z_{\frac{2k+1}{2^{n}}} \\
&=B_0(x) + \sum_{j=0}^{n-1}\frac{1}{\sqrt{2^{j+2}}} \sum_{k=0}^{2^j -1}
\Delta_{j,k}(x) Z_{\frac{2k+1}{2^{j+1}}} & & \text{for $n > 0$} 
\end{align*}
We will sometimes find it convenient to use the definition
\begin{align*}
F_n(x) &= \frac{1}{\sqrt{2^{n+2}}} \sum_{k=0}^{2^n -1}
\Delta_{n,k}(x) Z_{\frac{2k+1}{2^{n+1}}} 
\end{align*}
so that we may write 
\begin{align*}
B_n(x) &= B_0(x) + \sum_{j=0}^{n-1} F_j(x) \\
B(x) &= B_0(x) + \sum_{j=0}^\infty F_j(x)
\end{align*}

There are host of important facts about the $B_n(x)$ and $B(x)$ that
proceed to prove.  No individual fact is difficult to prove but there
are many of them to keep track of.

\begin{lem}The following are true:
\begin{itemize}
\item[(i)] $B_n(x)$ is linear on every interval $[\frac{k}{2^n},
\frac{k+1}{2^n}]$ for $k=0,\dots,2^n -1$.
\item[(ii)]For every $n \geq 0$, and $0 < 2k+1 < 2^n$, 
\begin{align*}
B(\frac{2k+1}{2^n}) = \frac{1}{2} (B(\frac{2k}{2^n}) +
B(\frac{2k+2}{2^n})) + \frac{1}{\sqrt{2^{n+1}}} Z_{\frac{2k+1}{2^n}}
\end{align*}
\item[(iii)]For every $n \geq 0$ and every pair $0 \leq j < k \leq 2^n$,
$B(k/2^n) - B(j/2^n)$ is an $N(0, (k-j)/2^n)$ random variable.
Furthermore for $0 \leq j < k \leq l < m \leq 2^n$, the increments
$B(k/2^n) - B(j/2^n)$ and $B(m/2^n) - B(l/2^n)$ are independent.
\end{itemize}
\end{lem}
\begin{proof}
FIrst we prove (i).  This follows from a simple induction.  It is clear for $B_0(x)$.  For
$B_{n+1}(x)$ we are adding multiples of the functions $\Delta_{n,k}(x)$ each of which is linear on intervals of the form $[\frac{k}{2^{n+1}},
\frac{k+1}{2^{n+1}}]$.

Next we prove (ii).  This follows from the fact that
$B(\frac{2k+1}{2^n})=B_n(\frac{2k+1}{2^n})$, the definition of
$B_n(x)$ and the linearity of $B_{n-1}(x)$ on the interval
$[\frac{k}{2^{n-1}}, \frac{k+1}{2^{n-1}}]$.

To see (iii) first note that it suffices to prove this for increments $j+1=k$ and
$l+1=m$.  For if we have proven that then we can write a general
increment as a sum of independent increments of the former form.  
We proceed by induction on $n$.  The case $n=0$ is trivial
because the only non-trivial increment is the $N(0,1)$ random variable
$B(1) - B(0) = Z_1$.  Now consider the case for $n > 0$.  
To see this first we consider ``adjacent'' increments of the form
$B((2k+1)/2^n) - B(2k/2^n)$ and $B((2k+2)/2^n) - B((2k+1)/2^n)$.  Here
we use the formula $B((2k+1)/2^n) = \frac{B((2k+2)/2^n) +
  B(2k/2^n)}{2} + \frac{1}{\sqrt{2^{n+1}}} Z_{(2k+1)/2^n}$ to see 
\begin{align*}
B((2k+1)/2^n) - B(2k/2^n) &= \frac{B((2k+2)/2^n) -
  B(2k/2^n)}{2} + \frac{1}{\sqrt{2^{n+1}}} Z_{(2k+1)/2^n} \\
B((2k+2)/2^n) - B((2k+1)/2^n) &= \frac{B((2k+2)/2^n) -
  B(2k/2^n)}{2} - \frac{1}{\sqrt{2^{n+1}}} Z_{(2k+1)/2^n} \\
\end{align*}
The random variables $B((2k+2)/2^n)$ and $B(2k/2^n)$ only depend on
the $Z_d$ for $d \in \mathcal{D}_{n-1}$ and therefore $Z_{(2k+1)/2^n}$ is
independent of both.  The induction hypothesis is that $B((2k+2)/2^n) -
  B(2k/2^n)$ is an $N(0, \frac{1}{2^{n-1}})$ random variable therefore $ \frac{B((2k+2)/2^n) -
  B(2k/2^n)}{2}$ is $N(0, \frac{1}{2^{n+1}})$.  But both $\pm
\frac{1}{\sqrt{2^{n+1}}} Z_{(2k+1)/2^n}$ are also $N(0,
\frac{1}{2^{n+1}})$ so we've expressed the increments as a sum of two independent
$N(0, \frac{1}{2^{n+1}})$ random variable proving that each is $N(0,
\frac{1}{2^n})$.  Furthermore the increments are independent.  Because
we know they are normal it suffices to show they are uncorrelated
which is a simple computation using the formulae above and the induction hypothesis
\begin{align*}
&\expectation{(B((2k+1)/2^n) - B(2k/2^n))(B((2k+2)/2^n) -
  B((2k+1)/2^n))} \\
&= \expectation{(\frac{B((2k+2)/2^n) -
  B(2k/2^n)}{2} + \frac{1}{\sqrt{2^{n+1}}} Z_{(2k+1)/2^n} ) (\frac{B((2k+2)/2^n) -
  B(2k/2^n)}{2} - \frac{1}{\sqrt{2^{n+1}}} Z_{(2k+1)/2^n})} \\
&=\frac{1}{4}\expectation{(B((2k+2)/2^n) -  B(2k/2^n))^2} -
\frac{1}{2^{n+1}} \\
&=\frac{1}{4}\frac{1}{2^{n-1}} -  \frac{1}{2^{n+1}}  = 0
\end{align*}

It remains to show the independence of increments 
$B((k+1)/2^n) - B(k/2^n)$ and $B((j+1)/2^n) - B(j/2^n)$ with $0 \leq j
< k \leq 2^n$.  In a similar way to the case above we know that
by using the result (ii) we can see that for $0 \leq k < 2^n$,
\begin{align*}
B((k+1)/2^n) - B(k/2^n) &= \begin{cases}
\frac{B((k+1)/2^n) - B((k-1)/2^n)}{2} - \frac{1}{\sqrt{2^{n+1}}}
Z_{k/2^n} & \text{$k$ is odd} \\
\frac{B((k+2)/2^n) - B(k/2^n)}{2} + \frac{1}{\sqrt{2^{n+1}}}
Z_{(k+1)/2^n} & \text{$k$ is even} \\
\end{cases}
\end{align*}
If we assume that we are not in the case already proven then we are either
assuming that $j+1 \neq k$ or $k$ is even. The upshot is that we can
write each increment of length $\frac{1}{2^n}$ as a sum of an increment of
length $\frac{1}{2^{n-1}}$ and an independent $N(0, \frac{1}{2^{n+1}})$
 random variable.  The increments of length $\frac{1}{2^{n-1}}$ are
 independent by the induction hypothesis and therefore the original
 increments are seen to be independent.  TODO: Make this more precise.
\end{proof}

We make the following claim about $B_n(x)$: for $\frac{k}{2^n}
\leq x \leq \frac{k+1}{2^n}$ and $0 \leq k < 2^n$, we have
$\variance{B_n(x)} = 2^n(x - \frac{k}{2^n})^2 + \frac{k}{2^n}$.  We
use an induction to prove the claim.  Note
that the claim is easily seen to be true for $n=0$ (it reduces to
earlier observation that $\variance{B_0(x)} = x^2$).  Now assuming that
it is true for $n$ we extend to $n+1$.  Pick an interval
$[\frac{k}{2^{n}}, \frac{k+1}{2^{n}}]$ and consider passing from
$B_n(x)$ to $B_{n+1}(x)$ on the interval.  There are two subcases
corresponding to the subinteval $[\frac{k}{2^{n}},
\frac{2k+1}{2^{n+1}}]$ and the subinterval $[\frac{2k+1}{2^{n+1}},
\frac{k+1}{2^{n}}]$.

On the first subinterval, by the definition of $B_{n+1}(x)$ we are
adding to $B_n(x)$ a normal random variable with variance
$\left ( \frac{1}{\sqrt{2^{n+2}}} \Delta_{n,k}(x) \right)^2 = 2^n(x-\frac{k}{2^n})^2$.  So at such an $x$, $B_{n+1}(x)$ is normal
with variance 
\begin{align*}
\variance{B_{n+1}(x)} &= \variance{B_{n}(x)}  +
2^n(x-\frac{k}{2^n})^2  \\
&= 2^n(x-\frac{k}{2^n})^2 + \frac{k}{2^n} + 2^n(x-\frac{k}{2^n})^2 \\
&=
2^{n+1} (x-\frac{k}{2^n})^2 + \frac{k}{2^n}
\end{align*}

On the second subinterval, by the definition of $B_{n+1}(x)$ we are
adding to $B_n(x)$ a normal random variable with variance
$2^n(x-\frac{k+1}{2^n})^2$.  So at such an $x$, $B_{n+1}(x)$ is normal
with variance 
\begin{align*}
\variance{B_{n+1}(x)} &= \variance{B_{n}(x)}  +
2^n(x-\frac{k+1}{2^n})^2  \\
&= 2^n(x-\frac{k}{2^n})^2 + \frac{k}{2^n} + 2^n(x-\frac{k+1}{2^n})^2
\\
&=2^n \left[(x-\frac{2k+1}{2^{n+1}})^2 + \frac{1}{2^{n+1}}(x -
  \frac{2k+1}{2^{n+1}}) + \frac{1}{2^{2n+2}}\right] + \\
&2^n \left[(x-\frac{2k+1}{2^{n+1}})^2 - \frac{1}{2^{n+1}}(x -
  \frac{2k+1}{2^{n+1}}) + \frac{1}{2^{2n+2}}\right] + \frac{k}{2^n} \\
&=2^{n+1} (x-\frac{2k+1}{2^{n+1}})^2 + \frac{2k+1}{2^{n+1}}
\end{align*}
which verifies the claim.

We reiterate the importance of this fact is that the approximate path
$B_n(x)$ has the variance $x$ (the ``correct'' variance for a Brownian
path) at all $x = 0, \frac{1}{2^n}, \dots,1$, so that as $n$ increases
$B_n(x)$ has the correct variance on an increasing fine grid in
$[0,1]$.  In between the points of the grid, the variance of $B_n(x)$
is a quadratic function of $x$ that is strictly less than $x$.

Having defined the series expansion of our candidate Brownian motion,
the first order of business is to validate that it converges almost
surely.  To show convergence we need to make sure
that the increments we add at each $n$ get small fast enough; these increments are
multiples of independent standard normal random variables.
Convergence will follow if we can get an appropriate almost sure bound
on a random sample from a sequence of independent standard normals.

To see this we start with a tail bound for an $N(0,1)$ distribution.  
\begin{align*}
\probability{\abs{Z_d} \geq \lambda} &= \frac{2}{\sqrt{2\pi}} \int_\lambda^\infty
e^{\frac{-u^2}{2}} \, du \\
&\leq  \frac{2}{\sqrt{2\pi}} \int_\lambda^\infty
\frac{u}{\lambda} e^{\frac{-u^2}{2}} \, du \\
&= \frac{1}{\lambda\sqrt{2\pi}} e^{\frac{-\lambda^2}{2}}
\end{align*}
so if we pick any constant $c > 1$ and $n > 0$, then 
\begin{align*}
\probability{\abs{Z_d} \geq c \sqrt{n}} &\leq \frac{1}{c\sqrt{2\pi n}}
e^{\frac{-c^2 n}{2}} \leq e^{\frac{-c^2 n}{2}}
\end{align*}
Now using this bound, we see that
\begin{align*}
\sum_{n=0}^\infty \probability{ \text {there exists $d \in
    \mathcal{D}_n$ such that $\abs{Z_d} \geq c \sqrt{n}$}} &\leq
\sum_{n=0}^\infty  \sum_{d \in \mathcal{D}_n} \probability{\abs{Z_d}
  \geq c \sqrt{n}} \\
&\leq \sum_{n=0}^\infty 2^n e^{\frac{-c^2 n}{2}} \\
&= \sum_{n=0}^\infty
e^{-n(c^2 - 2\ln2)/2} 
\end{align*}
which converges if $c > \sqrt{2\ln2}$.  Picking such a $c$, we apply
the Borel Cantelli Theorem to conclude that 
\begin{align*}
\probability{ \text {there exists $d \in
    \mathcal{D}_n$ such that $\abs{Z_d} \geq c \sqrt{n}$ i.o.}} &= 0
\end{align*}
and therefore for almost all $\omega \in \Omega$ there exists
$N_\omega > 0$ such that $\abs{Z_d} < c \sqrt{n}$ for all $n >
N_\omega$ and $d \in \mathcal{D}_n$.  Using this result with the definition of
$F_n(x)=\sum_{k=0}^{2^n-1}
\frac{1}{\sqrt{2^{n+2}}}Z_{\frac{2k+1}{2^{n+1}}} \Delta_{n,k}(x)$, the
disjointness of the support of $\Delta_{n,k}(x)$ for fixed $n$ and the
fact that $\abs{\Delta_{n,k}(x)} \leq 1$ we
have $\norm{F_n}_\infty \leq 2^{-(n+2)/2} c \sqrt{n+1}$ which shows that
$\sum_{n=0}^\infty F_n(x)$ converges absolutely and uniformly in $x$.
Because each $F_n(x)$ is a continuous function, uniform convergence of
the series implies $B(x) = B_0(x) + \sum_{n=0}^\infty F_n(x)$ is continuous as
well (Theorem \ref{UniformLimitContinuousFunctionsIsContinuous}).

TODO: Show that for every $x \in [0,1]$, $B(x)$ is integrable and
has finite variance.  Not sure we need this because we'll prove a
stronger statement below.

The next step is to validate that $B(x)$ has independent Gaussian increments.
TODO: Show that we have Gaussian increments, independent increments,
zero mean and proper variance/covariance.  The first step is to note
that we have already proven that increments at dyadic rational numbers
are independent and Gaussian.  But we have also shown that $B(x)$ is
almost surely continuous so we may approximate arbitrary increments by
those at dyadic rationals.

Suppose we are given $0 \leq x_1 < x_2 < \cdots < x_n \leq 1$.  By the
density of the dyadic rationals we can find sequences $x_{j,m}$ of
dyadic rationals with $x_{j-1} < x_{j,m} \leq x_j$ such that $\lim_{m
  \to \infty} x_{j,m}= x_j$ (in the case $j=1$, we only require $0
\leq x_{1,m} \leq x_1$).  By almost sure continuity of $B(x)$ we know
that $B(x_{j,m}) - B(x_{j-1,m})$ converges to $B(x_j) - B(x_{j-1})$
for $1 < j \leq n$.  Moreover we know that 
\begin{align*}
\lim_{m \to \infty} \expectation{B(x_{j,m}) - B(x_{j-1,m})} &= 0
\end{align*}
and 
\begin{align*}
\lim_{m \to \infty} \textbf{Cov}\left ( B(x_{j,m}) - B(x_{j-1,m}), B(x_{i,m})
  - B(x_{i-1,m}) \right) &= \delta_{i,j} \lim_{m \to \infty} (x_{i,m} -
x_{i-1,m}) \\
&= \delta_{i,j}  (x_i - x_{i-1})
\end{align*}
and therefore by Lemma \ref{LimitOfGaussianRandomVectors} we know that
the $B(x_j) - B(x_{j-1})$ are independent $N(0, x_j - x_{j-1})$ random
variables and we are done.
\end{proof}
TODO: Note the connection of the construction to wavelets.  What we
are doing here is expressing the Brownian motion as a linear
combination of integrals of the Haar wavelet basis (in some sense we
are integrating ``white noise'' which is called an \emph{isonormal
  process} in the mathematical literature these days).  Note that the
such a form for a Brownian motion can be anticipated by examining the
covariance of Brownian motion (see Steele).

TODO: Modulus of continuity of Brownian paths; Holder continuity and
nowhere differentiability.

TODO: Some of these proofs use the specifics of the Levy construction
of Brownian motion and not just the defining properties of Brownian
motion.  In what way is this justified; i.e. to what extent is the
Levy construction unique?  The answer to this question is that Wiener
measure on $C[0,\infty)$ is uniquely defined by its finite dimensional
distributions.

\begin{defn}A function $f : (S, d) \to (T, d^\prime)$ between metric
  spaces is said to be \emph{H\"older continuous} with exponent
  $\alpha$ if there exists a constant $C > 0$ such that
  $d^\prime(f(x), f(y)) \leq C d(x,y)^\alpha$ for all $x, y \in S$.  
\end{defn}
\begin{lem}\label{HaarWaveletCoefficientHolderContinuity}Let $f : [0,1] \to \reals$ be continuous with $f(x) = c_0 +
  \sum_{n=0}^\infty \sum_{k=0}^{2^n -1} c_{n,k} \Delta_{n,k}(x)$.  Suppose $\abs{c_{n,k}} \leq
  2^{-\alpha n}$ for some $0 < \alpha < 1$ then $f \in C^{\alpha}[0,1]$.
\end{lem}
\begin{proof}
Since the condition for H\"older continuity only depends on
differences between a function we may assume that $c_0 = 0$.  Pick
$s,t \in [0,1]$ and use the triangle inequality to conclude 
\begin{align*}
\abs{f(s) - f(t)} &\leq \sum_{n=0}^\infty \abs{\sum_{k=0}^{2^n -1}
  c_{n,k} \left ( \Delta_{n,k}(s) - \Delta_{n,k}(t) \right) }
\end{align*}
To clean up our notation a bit we define 
\begin{align*}
D_n(s,t) &= 
\sum_{k=0}^{2^n -1}  c_{n,k} \left ( \Delta_{n,k}(s) - \Delta_{n,k}(t) \right)
\end{align*}
for $n\geq 0$ and we work on getting a bound on $\abs{D_n}$.  Since we have a very
concrete description of the $\Delta_{n,k}$ elementary (but detailed)
tools can be used.  Because the support of $\Delta_{n,k}$ for fixed
$n$ are disjoint $\Delta_{n,k}(s)$ is non-zero for at most one $k$ and
similarly with $\Delta_{n,k}(t)$.  Let $0 \leq k_s < 2^n$ be an
integer such that $k_s/2^{n} \leq s \leq (k_s+1)2^n$ and similarly with
$k_t$ (there is ambiguity in the choice for $s,t = k/2^n$ but it
doesn't matter since the $\Delta_{n,k}$ all vanish at such points);
with these choices, $D_n(s,t) = c_{n,k_s} \Delta_{n,k_s}(s) - c_{n, k_t}
  \Delta_{n,k_t}(t)$.  Each function $\Delta_{n,k}$ is
piecewise linear and comprises two line segments with slope $\pm
2^{n+1}$ and it is geometrically clear that $\Delta_{n,k_s}(s)$ and
$\Delta_{n,k_t}(t)$ can be no farther than if they are on the same
such line : hence $\abs{\Delta_{n,k_s}(s) - \Delta_{n,k_t}(t)} \leq
\abs{s-t} 2^{n+1}$ and by the bounds we have on the coefficients
$c_{n,k}$ we get
\begin{align*}
\abs{D_n(s,t)} &\leq \left( \abs{c_{n,k_s}} \vee \abs{c_{n,k_t}}
\right) \abs{\Delta_{n,k_s}(s) - \Delta_{n,k_t}(t)}  \leq 2^{-\alpha
  n} \abs{s-t} 2^{n+1}
\end{align*}
This is a good bound when $s,t$ are close (in fact it is a tight bound
when $k_s=k_t$ and $s,t$ are on the same line segment).  However, as
$s,t$ get farther apart we can do better just by using the fact that
$0 \leq \Delta_{n,k} \leq 1$.  Indeed by the triangle inequality
\begin{align*}
\abs{D_n(s,t)} &= \abs{c_{n,k_s} \Delta_{n,k_s}(s)} + \abs{c_{n,k_t}
  \Delta_{n,k_t}(t)} \leq \abs{c_{n,k_s}} + \abs{c_{n,k_t}} \leq
2^{-\alpha n + 1}
\end{align*}
and therefore we have the two bounds
\begin{align*}
D_n(s,t) &\leq 2^{-\alpha  n} \abs{s-t} 2^{n+1} \wedge 2^{-\alpha n + 1}
\end{align*}
As mentioned, the first of these bounds is a better estimate when $s,t$ are closer
that $2^{-n}$ and the latter is better otherwise.  So with $s,t$
given pick $N \geq 0$ such that $2^{-N -1} \leq \abs{s -t} < 2^{-N}$
and use the appropriate mix of the two estimates
\begin{align*}
\abs{f(s) - f(t)} &\leq \sum_{n=0}^N \abs{\sum_{k=0}^{2^n -1}
  c_{n,k} \left ( \Delta_{n,k}(s) - \Delta_{n,k}(t) \right) } + \sum_{n=N+1}^\infty \abs{\sum_{k=0}^{2^n -1}
  c_{n,k} \left ( \Delta_{n,k}(s) - \Delta_{n,k}(t) \right) } \\
&\leq \sum_{n=0}^N 2^{-\alpha  n} \abs{s-t} 2^{n+1} + 
\sum_{n=N+1}^\infty 2^{-\alpha n + 1} \\
&= 2 \abs{s - t} \frac{2^{(1 -\alpha)(N+1)} - 1}{2^{1 - \alpha} -1} + 2 \cdot 2^{-\alpha(N+1)} \cdot \frac{1}{1 - 2^{-\alpha}} \\
&\leq \frac{2}{2^{1 - \alpha} -1} \abs{s-t}^\alpha - \frac{2}{2^{1 -
    \alpha} -1}\abs{s-t} + \frac{2}{1 - 2^{-\alpha}} \abs{s - t}^\alpha \\
&\leq \left( \frac{2}{2^{1 - \alpha} -1} + \frac{2}{1 - 2^{-\alpha}}
\right ) \abs{s-t}^\alpha
\end{align*}
where we have used the assumption that $0 < \alpha < 1$ to determine
the sign of coefficients in the estimates (e.g. to conclude
that $\frac{2}{2^{1 - \alpha} -1}\abs{s-t} > 0$ so that this term may
be dropped from the estimate).
\end{proof}

A corollary of this result and our construction of Brownian motion is
the fact that Brownian paths are H\"older continuous with any exponent
less that $1/2$.
\begin{thm}[H\"older Continuity of Brownian
  Paths]\label{BrownianHolderContinuous}Let $B_t$ be a standard
  Brownian motion then almost surely $B_t$ is H\"older continuous for
  any exponent $\alpha < 1/2$.  Furthermore there exists a constant $C
  > 0$ (independent of $\omega$) such that almost surely there exists
  a constant $\epsilon > 0$ (depending on $\omega$) such that for all
  $0 \leq h \leq \epsilon$ and $0 \leq t \leq 1-h$ we have 
\begin{align*}
\abs{B_{t+h} - B_t} \leq C \sqrt{h \log(1/h)}
\end{align*}
\end{thm}
\begin{proof}
From our construction of Brownian motion recall that we had the
representation
\begin{align*}
B_t &= t Z_0 + \sum_{n=0}^\infty \frac{1}{\sqrt{2^{n+2}}}
\sum_{k=0}^{2^n -1} \Delta_{n,k}(t) Z_{\frac{2k+1}{2^{n+1}}}
\end{align*}
and moreover we have shown during the construction of Brownian motion
for $c > \sqrt{2 \ln 2}$ almost surely there is an $N>0$ such that 
\begin{align*}
\abs{Z_{\frac{2k+1}{2^{n+1}}}} \leq c \sqrt{n+1}
\end{align*}
for all $n \geq N$.  Note that we can ignore the leading term $t Z_0$ since is clearly
H\"older continuous, so to apply Lemma
\ref{HaarWaveletCoefficientHolderContinuity}
it suffices to observe that we have coefficients $c_{n,k} =
\frac{1}{\sqrt{2^{n+2}}}Z_{\frac{2k+1}{2^{n+1}}}$ with the bound
\begin{align*}
\abs{c_{n,k}} &\leq \frac{c\sqrt{n+1}}{\sqrt{2^{n+2}}} \leq 2^{-\alpha n}
\end{align*}
for $n$ sufficiently large.  TODO: In the previous Lemma we need to
rephrase things to note that it suffices to have the bound hold
eventually.

TODO:  Extend the estimates from the prior Lemma to yield the simple upper bound for
modulus of continuity.  Following the proof of the prior Lemma and
using our estimate on the $c_{n,k}$ directly instead of the derived
bound $\abs{c_{n,k}} \leq 2^{-\alpha n}$ we get by picking $2^{-M-2}
< \abs{s-t} \leq 2^{-M-1}$ (so that $M+1 \leq \log_2(1/\abs{s-t})$)
\begin{align*}
\abs{B_s - B_t} &\leq \sum_{n=0}^{N-1} \max_{0 \leq k < 2^n}
\abs{c_{n,k}} \abs{s-t} 2^{n+1} +
\sum_{n=N}^M \abs{s-t} 2^{n+1} \frac{c\sqrt{n+1}}{2^{(n+2)/2}} + 
2 \sum_{n=M+1}^\infty  \frac{c\sqrt{n+1}}{2^{(n+2)/2}}
\end{align*}
For the first term, we use the fact that $\lim_{\epsilon \to 0^+}
\epsilon/\sqrt{\epsilon \log(1/\epsilon)} = 0$ to find $\epsilon$ 
(depending on $\omega$) sufficiently small so that provided $\abs{s-t} \leq
\epsilon$ we have 
\begin{align*}
\sum_{n=0}^{N-1} \max_{0 \leq k < 2^n}
\abs{c_{n,k}} \abs{s-t} 2^{n+1} \leq \sqrt{\abs{s-t}
  \log(1/\abs{s-t})}
\end{align*}
For the second term, by choice of $M$ we get
\begin{align*}
\sum_{n=N}^M \abs{s-t} 2^{n+1} \frac{c\sqrt{n+1}}{2^{(n+2)/2}} 
&\leq c \abs{s-t} \sum_{n=0}^M 2^{n/2} \sqrt{n+1} \\
&\leq  c \abs{s-t}   \sqrt{M+1} \frac{2^{(M+1)/2}-1}{\sqrt{2}-1}\\
&\leq \frac{c}{\sqrt{2}-1}
  \sqrt{\abs{s-t} \log_2(1/\abs{s-t})}
\end{align*}
For the third term by choice of $M$ we get
\begin{align*}
2 \sum_{n=M+1}^\infty  \frac{c\sqrt{n+1}}{2^{(n+2)/2}}
&\leq \sqrt{M+1} \frac{c}{2^{(M+1)/2}} \sum_{n=0}^\infty
\sqrt{\frac{n+M+1}{M+1}} \frac{1}{2^{n/2}} \\
&\leq \sqrt{M+1} \frac{c}{2^{(M+1)/2}} \sum_{n=0}^\infty
\sqrt{n+1} \frac{1}{2^{n/2}} \\
&\leq C_2 \sqrt{\abs{s-t}\log_2(1/\abs{s-t})}
\end{align*}
where the constant $C_2$ depends only on the value of the convergent
series and the choice of $c$.
\end{proof}

TODO: Levy's modulus of continuity Lemmas
\begin{thm}Almost surely 
\begin{align*}
\limsup_{h \downarrow 0} \sup_{0 \leq t \leq 1-h} \frac{\abs{B_{t+h} -
    B_t}}{\sqrt{2h\log(1/h)}} &= 1
\end{align*}
(TODO: Is this $\log_e$ or $\log_2$?)
\end{thm}
\begin{proof}
My notes on the proof from Peres and Morters
Fix a $c > \sqrt{2}$ and pick $0 < \epsilon < 1/2$.  For this $\epsilon$,
by Lemma ? we pick $m>0$ such that for every $[s,t] \subset [0,1]$ we
get $[s^\prime,t^\prime] \in \Lambda(m)$ such that $\abs{t - t^\prime}
< \epsilon (t -s)$ and  $\abs{s - s^\prime}
< \epsilon (t -s)$.  Now by Lemma ? we choose $N > 0$ such that for
all $n \geq N$, almost surely for every $[s^\prime, t^\prime] \in
\Lambda_n(m)$
\begin{align*}
\abs{B_{t^\prime} - B_{s^\prime}} &\leq c \sqrt{(t^\prime - s^\prime)
  \log(1/(t^\prime -s^\prime))}
\end{align*}
(we want this to be true for the approximating $[s^\prime, t^\prime]$:
how do we know that $[s^\prime, t^\prime] \in \Lambda_n(m)$ for
sufficiently large $n$; I think it is true that $\Lambda_n(m) \subset
\Lambda_{2n}(m)$?  No I think we make the assumption that $t-s < 2^{-N}$).
But we also have Theorem \ref{BrownianHolderContinuous} (TODO: Does
this work; this result gives the bound for $h$ smaller than a
\emph{random} constant but here it seems we are assuming that it is
not random) so we can
estimate
\begin{align*}
\abs{B_{t} - B_{s}} &\leq \abs{B_{t} -
  B_{t^\prime}} + \abs{B_{t^\prime} - B_{s^\prime}} +
\abs{B_{s^\prime} - B_{s}} \\
&\leq C \sqrt{\abs{t - t^\prime} \log(1/\abs{t - t^\prime})} + 
c \sqrt{(t^\prime - s^\prime)  \log(1/(t^\prime -s^\prime))} +
C \sqrt{\abs{s - s^\prime} \log(1/\abs{s - s^\prime})}
\end{align*}
The function $x \log(1/x)$ is increasing for $0\leq x \leq 1/2$ (here
we are using $\log_2$; otherwise $1/e$) therefore if we assume $t -s <
\epsilon$ then $\abs{t - t^\prime} < \epsilon (t-s) < 1/4$ so we get
the estimate
\begin{align*}
C \sqrt{\abs{t - t^\prime} \log(1/\abs{t - t^\prime})} &\leq C
\sqrt{\epsilon(t-s)\log(1/\epsilon(t-s))} \\
&\leq C\sqrt{\epsilon(t-s)\log(1/(t-s)^2)} \\
&= \sqrt{2 \epsilon} C \sqrt{(t-s)\log(1/(t-s))}
\end{align*}
and similarly with the term involving $\abs{s- s^\prime}$.  As for the
middle term, we have by choice of $[s^\prime, t^\prime]$ that $(1 -
2\epsilon)(t -s) \leq (t^\prime - s^\prime) \leq (1+2 \epsilon)(t-s)$
and by assumption $\log(1/(t-s)) > 1$ therefore
\begin{align*}
c \sqrt{(t^\prime - s^\prime)  \log\frac{1}{t^\prime -s^\prime}} &\leq c
\sqrt{(1 + 2\epsilon) (t - s)  \log\frac{1}{(1-2\epsilon) (t -s)}} \\
&= c\sqrt{(1 + 2\epsilon) (t - s)  (\log\frac{1}{(1-2\epsilon) }
+ \log\frac{1}{ (t -s)})} \\
&\leq c\sqrt{(1 + 2\epsilon) (t - s)  \log\frac{1}{ (t -s)} (1- \log(1-2\epsilon))} 
\end{align*}
Now since $\epsilon > 0$ was arbitrary, we can put all three estimates
together conclude for any $0 < h < \epsilon$, 
\begin{align*}
\sup_{0 \leq t \leq 1-h} \abs{B_{t+h} - B_t} &\leq \left ( 2\sqrt{2\epsilon}C
  +  c\sqrt{(1 + 2\epsilon) (1- \log(1-2\epsilon)) } \right) \sqrt{h\log(1/h)}
\end{align*}
and thus 
\begin{align*}
\limsup_{h \downarrow 0} \sup_{0 \leq t \leq 1-h} \frac{\abs{B_{t+h} - B_t}}{\sqrt{h\log(1/h)}} &\leq  2\sqrt{2\epsilon}C
  +  c\sqrt{(1 + 2\epsilon) (1- \log(1-2\epsilon))} 
\end{align*}
Now since $0 < \epsilon < 1/2$ was arbitrary and $c > \sqrt{2}$ was
arbitrary we can let $\epsilon \downarrow 0$ and then $c \downarrow
\sqrt{2}$ to conclude the result.
\end{proof}

The approach above to studying the sample path properties of Brownian
motion is based on examing the (random) coefficients of the expression
of the Brownian motion in the Schauder basis.  This has advantages and
disadvantages.  The obvious advantage is a certain concreteness that
is appealing.  The disadvantage is that the analysis is less general
than it can be.  Here we provide a classical alternative to the
construction of Brownian motion and the analysis of sample paths that
relies on tools that are more general.  It is critical to have these
more general tools at hand when discussing larger classes of
stochastic process.

\begin{thm}[Kolmogorov-Centsov]Let $X_t$ be a stochastic process on
  $[0,T]^d$ with values in a complete metric space $(S,d)$ and suppose
  that there exist constant $C, \alpha, \beta$ such that 
\begin{align*}
\expectation{d(X_s, X_t)^\alpha} &\leq \abs{s-t}^{d + \beta} \text{
  for all $s,t \in \reals^d$}
\end{align*}
then $X_t$ has a continuous modification $\tilde{X}_t$ and furthermore
the paths of $\tilde{X}_t$ are almost surely H\"older continuous with
exponent $\gamma$ for every $0 < \gamma < \beta/\alpha$.
\end{thm}
\begin{proof}
We do the proof with $T=1$ and $d=1$.  

The basic idea of the proof is that via Markov bounding, the moment
condition controls the variations of $X_t$ pointwise; furthermore by careful
selection of constants we can extend this to uniform continuity of $X_t$ on a
countable subset of $[0,T]^d$.  By chosing a countable dense subset of
$[0,T]^d$ we will then be in position to create the modification.

For each $n \geq 0$, let $\mathcal{D}_n = \lbrace k/2^n \mid 0 \leq k
\leq 2^n \rbrace$ be the dyadic rationals with scale $n$ and consider
the behavior of $X_t$ on the grid $\mathcal{D}_n^d \subset [0,1]^d$.
To begin bound the variation on adjacent points in the grid using a
union bound and a Markov bound (TODO: Fix up the sum below for the
case $d>1$)
\begin{align*}
\probability{\max_{0 < k \leq 2^n} d(X_{k/2^n}, X_{(k-1)/2^n}) \geq  \epsilon}
&\leq \sum_{k=1}^{2^n} \probability{d(X_{k/2^n}, X_{(k-1)/2^n}) \geq \epsilon} \\
&\leq \sum_{k=1}^{2^n} 2^{-n(d+\beta)}/\epsilon^\alpha = 2^{-n\beta} \epsilon^{-\alpha}
\end{align*}
So if we pick $0 < \gamma < \beta/\alpha$ and $\epsilon=2^{-n\gamma}$
then we have the bound 
\begin{align*}
\sum_{n=0}^\infty \probability{\max_{0 < k \leq 2^n} d(X_{k/2^n}, X_{(k-1)/2^n}) \geq
  2^{-n\gamma}} \leq \sum_{n=1}^\infty  2^{-n(\beta - \gamma\alpha)} < \infty
\end{align*}
and Borel Cantelli tells us that there is an event $A \subset \Omega$
with $\probability{A}=1$ and for each $\omega \in A$ there exists an
$N(\omega)$ such that 
\begin{align*}
d(X_{k/2^n}(\omega),
X_{(k-1)/2^n}(\omega)) &< 2^{-n\gamma} \text{ for all $n \geq
  N(\omega)$ and $0 < k \leq 2^n$}
\end{align*}

We have gained some control on the behavior of $X_t$ on a sequence of
successively finer dyadic grids but what we need is to translate this into
control of $X_t$ simultaneously over the union of all grids (to see
what we are lacking at this point realise that we have an almost sure
bound on a term like $d(X_{k/2^n}, X_{(k-1)/2^n})$ with $k/2^n -
(k-1)/2^n = 1/2^n$ but we don't yet have a bound on a term like
$d(X_{(2k+1)/2^n}, X_{(2k-1)/2^n})$ with $(2k+1)/2^{n+1} -
(2k-1)/2^{n+1} = 1/2^n$).  

Claim: For every $n \geq N(\omega)$ and every $m > n$ we have 
\begin{align*}
d(X_t(\omega), X_s(\omega)) &\leq 2\sum_{k=n+1}^m 2^{-k\gamma} \text{
  for $s,t \in \mathcal{D}_m$ with $0 < \abs{s-t} < 2^{-n}$}
\end{align*}

The proof of the claim is by induction.  For $m=n+1$ the only way for
$0 < \abs{s-t} < 2^{-n}$ when $s,t \in \mathcal{D}_{n+1}$ is when
$s=(k-1)/2^{n+1}$ and $t=k/2^{n+1}$ and therefore by what we have
already shown $d(X_t(\omega), X_s(\omega)) \leq 2^{-(n+1)\gamma}$ so
the result holds in this case.  Now assume that the result holds for
all $n+1, \dotsc, m$ and we show it for $m+1$.  Assume without loss of
generality that $s < t$ define $s^* = \ceil{2^m s}/2^m$ and $t^* =
\floor{2^m t}/2^m$ (that is to say round $s$ up
to nearest point on the grid $\mathcal{D}_m$ and round $t$ down to the
nearest point on the grid $\mathcal{D}_m$).  Then the following are
easily seen to be true
\begin{itemize}
\item[(i)] $s^*, t^* \in \mathcal{D}_m$
\item[(ii)] $s \leq s^* \leq t^*\leq t$
\item[(iii)] $0 \leq s^* - s \leq 1/2^{m+1}$
\item[(iv)] $0 \leq t - t^* \leq 1/2^{m+1}$
\item[(v)] $0 \leq t^* - s^* < 1/2^n$
\end{itemize}
Now by the triangle inequality, the induction hypothesis and the result for adjacent points in
the grid $\mathcal{D}_{m+1}$ we get
\begin{align*}
d(X_t, X_s) &\leq d(X_t, X_{t^*}) + d(X_{t^*}, X_{s^*}) + d(X_{s^*},
X_s)  \\
&\leq 2^{-(m+1)\gamma} + 2 \sum_{k=n+1}^m 2^{-k\gamma} +
2^{-(m+1)\gamma}  = 2 \sum_{k=n+1}^{m+1} 2^{-k\gamma}
\end{align*} 
and we are done with the claim.

The claim establishes the local H\"older continuity of $X_t(\omega)$
on $\mathcal{D} = \cup_{n=1}^\infty \mathcal{D}_n$ (hence uniform
continuity).  To see this, pick $s,t \in \mathcal{D}$ such that
$\abs{s-t} < 2^{-N(\omega)}$ and find $n > N(\omega)$ such that
$2^{-(n+1)} \leq \abs{s-t} < 2^{-n}$, then $s,t \in \mathcal{D}_m$ for all
$m$ large enough and so
\begin{align*}
d(X_t(\omega), X_s(\omega)) &\leq 2 \sum_{k=n+1}^m 2^{-k\gamma} \leq
2^{-(n+1)\gamma} \frac{2}{1 - 2^{-\gamma}}  \leq \abs{s-t}^\gamma \frac{2}{1 - 2^{-\gamma}}
\end{align*}

Since $X_t$ is almost surely H\"older continuous on $\mathcal{D}^d$
which is a dense subset of $[0,1]^d$ we know that $X_t$ has a unique
extension $\tilde{X}_t$ to a continuous function on $[0,1]^d$ and that
the extension is H\"older continuous with the same exponent and
constant.  Define $\tilde{X}_t = 0$ for $\omega \notin A$.  

It remains to show that $\tilde{X}_t$ defined in this way is a
modification of $X_t$.
Assume $\epsilon
> 0$ and apply a Markov bound 
\begin{align*}
\probability{d(X_t, X_s) > \epsilon} &\leq \frac{\expectation{d(X_t,
    X_s)^\alpha}}{\epsilon^\alpha} \leq \frac{\abs{s-t}^{d + \beta}}{\epsilon^\alpha}
\end{align*}
which shows that for every $s \in [0,1]^d$ we have $X_t \toprob X_s$
as $t \to s$.

TODO: Finish the argument that this is a modification.
\end{proof}

The flip side of the positive results showing that Brownian paths are
H\"older continuous is the following result showing that a sea change
occurs at $\alpha = 1/2$.  As we'll note, in particular this shows
that Brownian paths are almost surely nowhere differentiable.

\begin{thm}\label{BrownianNotHolderContinuous}For every $\alpha > 1/2$ almost surely a Brownian path has
  no point that is locally H\"older continuous with exponent $\alpha$.
\end{thm}
\begin{proof}
Pick an $\alpha > 1/2$, $C > 0$, $\epsilon > 0$ and define
\begin{align*}
G(\alpha, C, \epsilon) &= \lbrace \omega \mid \text{ there exists } s
\in [0,1] \text{ such that } \abs{B_t(\omega) - B_s(\omega)} <
C\abs{t-s} \text{ for every } t \in [0,1] \text{ with }
\abs{t-s}<\epsilon \rbrace
\end{align*}
The set $G(\alpha,C, \epsilon)$ is not necessarily measurable so it
doesn't make sense to show that it has measure zero; however we will
show that it is contained in a set of a measure zero.  The trick to
doing this is the observation that the $\alpha$-H\"older continuity of
$B_s(\omega)$ from the definition of $G(\alpha,C,\epsilon)$ implies an
arbitrarily large number of independent increments to be small.  By
the Gaussian nature of the increments and a very crude tail
probability estimate we'll be able to conclude that the
probability of the increments all being small can be sent to zero.  At
the risk of being pedantic, note that while the positive results on
H\"older continuity relied on bounds showing it is unlike that a
collection of independent Gaussians will simulaneously be large, this
result requires a bound showing it is unlikely that a collection of
independent Gaussians will simultaneously be small.

To make this precise, pick an $\omega \in G(\alpha, C,\epsilon)$ and
let $s \in [0,1]$ be an appropriate H\"older continuous point.  Now
define $U = [0,1] \cap (s - \epsilon, s + \epsilon)$ so that the
diameter is at least $\epsilon$.  Now for any $m >0$ there is an
$N_{m,\epsilon}$ (roughly speaking $N_{m,\epsilon} = 2m/ \epsilon$)
such that for all $n \geq N_{m,\epsilon}$ there exists a $k$ with $0
\leq k < n-m$ such that for all $0\leq i < m$, $[\frac{k+i}{n},
\frac{k+i+1}{n}] \subset U$ and either $s \in [\frac{k}{n},
\frac{k+1}{n}]$ or $s \in [\frac{k+m-1}{n},\frac{k+m}{n}]$ (we only
need the last option when $s =1$).  Now using the fact that the
diameter of $U$ is less that $\epsilon$, the triangle inequality and
the H\"older continuity at $s$ we see for every $0 \leq i < m$,
\begin{align*}
\abs{B_{\frac{k+i+1}{n}}(\omega) - B_{\frac{k+i}{n}}(\omega)} &\leq
\abs{B_{\frac{k+i+1}{n}}(\omega) - B_{s}(\omega)} + \abs{B_{s}(\omega)
  - B_{\frac{k+i}{n}}(\omega)} \leq 2 C \left( \frac{m}{n} \right)^\alpha
\end{align*}
From this argument we conclude that for every $m > 0$ and every $n
\geq N_{m, \epsilon}$
\begin{align*}
G(\alpha, C, \epsilon) &\subset \cup_{k=0}^{n-m-1} \cap_{i=0}^{m-1}
\lbrace \omega \mid \abs{B_{\frac{k+i+1}{n}}(\omega) -
  B_{\frac{k+i}{n}}(\omega)} \leq 2 C \left( \frac{m}{n}
\right)^\alpha \rbrace
\end{align*}
We know that each increment $B_{\frac{k+i+1}{n}}(\omega) -
  B_{\frac{k+i}{n}}(\omega)$ is Gaussian with variance $1/n$.  Thus
 we can apply the simple bound for a $N(0,1)$ random
  variable $Z$,
\begin{align*}
\probability{\abs{Z} \leq \lambda} &= \frac{1}{\sqrt{2\pi}}
\int_{-\lambda}^\lambda e^{-x^2/2} \, dx \leq \frac{1}{\sqrt{2\pi}}
\int_{-\lambda}^\lambda \, dx = \frac{2\lambda}{\sqrt{2\pi}}
\end{align*}
to conclude 
\begin{align*}
\probability{\abs{B_{\frac{k+i+1}{n}}(\omega) -
  B_{\frac{k+i}{n}}(\omega)} \leq 2 C \left( \frac{m}{n}
\right)^\alpha} &\leq \frac{4 C \sqrt{n}}{\sqrt{2\pi}}\left( \frac{m}{n}
\right)^\alpha
\end{align*}
By a union bound and the independence of Brownian increments we know
that 
\begin{align*}
&\probability{\cup_{k=0}^{n-m-1} \cap_{i=0}^{m-1}
\lbrace \omega \mid \abs{B_{\frac{k+i+1}{n}}(\omega) -
  B_{\frac{k+i}{n}}(\omega)} \leq 2 C \left( \frac{m}{n}
\right)^\alpha \rbrace} \\
&\leq n \left( \frac{4 C \sqrt{n}}{\sqrt{2\pi}}\left( \frac{m}{n}
\right)^\alpha\right)^m =\left( \frac{4 C m^\alpha}{\sqrt{2\pi}}\right)^m n^{1 + (\frac{1}{2}-\alpha)m}
\end{align*}
The important point is if we choose any value of $m > \frac{1}{\alpha
  - 1/2}$ (possible since $\alpha > 1/2$) then the exponent $1 +
(\frac{1}{2}-\alpha)m < 0$ and taking the limit as $n \to \infty$ we
see that $G(\alpha, C, \epsilon)$ is contained in a set of measure
zero.

The proof of the Theorem is completed by taking the countable union 
over all rational $C$ and rational $\epsilon$ and noting that this is
also contained in a set of measure zero.
\end{proof}

\begin{cor}[Nondifferentiability of Brownian Motion]Almost sure a
  Brownian path is nowhere differentiable.  
\end{cor}
\begin{proof}Take $\alpha = 1$ in the Theorem \ref{BrownianNotHolderContinuous}
\end{proof}

\begin{thm}[Markov Property of Brownian motion]\label{BrownianMarkovProperty}Let $B_t$ be a Brownian motion starting at $x$ and let $s
  \geq 0$.  Then $B_{t+s} - B_s$ is a Brownian motion starting at $0$
  that is independent of $B_t$ for $0 \leq t \leq s$.
\end{thm}
\begin{proof}
The fact that $B_{t+s} - B_s$ is a Brownian motion follows from the
fact that increments of the translated process are increments of the
original Brownian motion.  More precisely if we select $t_1 \leq
\cdots \leq t_n$ then each $(B_{t_{i+1}+s} - B_s) - (B_{t_{i}+s} -
B_s) = B_{t_{i+1}+s} - B_{t_i + s}$ and therefore they are we can
conclude they are jointly independent Gaussian with variance $(t_{i+1}
- s) - (t_i - s) = t_{i+1} - t_i$.

The independence of the Brownian motion $B_{t+s} - B_s$ and $B_t$ for
$0 \leq t \leq s$ follows from the property of independent
increments.  Specifically, by the montone class argument of Lemma \ref{IndependenceFinitary} we know that it is sufficient
to show independence for finite sets $\lbrace B_{{t_1}+s} - B_s,
\dots ,B_{{t_n}+s} - B_s \rbrace$ and $\lbrace B_{s_1}, \dots,
B_{s_m}\rbrace$ for all finite sequence of times $s_1 \leq \cdots \leq s_m \leq s$ and $0 \leq t_1 \leq \cdots \leq
t_n$.  Observe that for any measurable random vectors $\xi_1, \dots ,
\xi_n$ we have $\sigma(\xi_1, \xi_2 - \xi_1, \dots,\xi_n - \xi_1) =
\sigma(\xi_1, \xi_2 - \xi_1, \dots,\xi_n - \xi_{n-1})$ (to see this
note that every term on the left is a sum of terms on the right and
vice versa).  In particular by independence of increments and Lemma
\ref{IndependenceGrouping} we know that $\sigma(B_{{t_1}+s} - B_s,
\dots ,B_{{t_n}+s} - B_{t_{n-1}})$ and $\sigma( B_{s_1} - B_0, \dots,
B_{s_m} - B_{s_{m-1}})$ are independent
which establishes the result by applying the previous observation.
\end{proof}

\subsection{Skorohod Embedding and Donsker's Theorem}
TODO: Clarify what we mean when we say a Brownian motion is
independent of a $\sigma$-algebra.

TODO: Introduce the right continuous filtration $\mathcal{F}^+_t$

TODO: Strong Markov Property

\begin{thm}[Markov  Property]\label{MarkovPropertyBrownianMotion}Let $B_t$ be a
  Brownian motion then for any $s \geq 0$ the process $B^*_t = B_{t + s} -
  B_s$ is a standard Brownian motion independent of $\lbrace B_t \mid
  0 \leq t \leq s \rbrace$.
\end{thm}
\begin{proof}
We simply walk through the defining properties of Brownian motion:
\begin{itemize}
\item[(i)] Clearly $B^*_0 = B_s - B_s = 0$.
\item[(ii)] For any $0 \leq t_1 \leq \cdots \leq t_n$ the increment
  $B^*_{t_j} - B^*_{t_{j-1}} = B^*_{s+ t_j} - B^*_{s+ t_{j-1}}$
  therefore the independence of the increments $B^*_{t_2} -
  B^*_{t_1}, \dotsc, B^*_{t_n} - B^*_{t_{n-1}}$ follows from the fact
  that $B_t$ is a Brownian motion
\item[(iii)]By the same argument as in (ii), for any $t_1 < t_2$ we
  have $B^*_{t_2} -  B^*_{t_1} = B_{s+t_2} -  B_{s+t_1}$ is normallly
  distributed with mean $0$ and variance $(s + t_2) - (s+t_1) = t_2 -
  t_1$.
\item[(iv)]The paths $B^*_t = B_{s+t}$ are almost surely continuous
  because $B_t$ is a Brownian motion
\end{itemize}

To see the independence statement pick $0 \leq t_1 \leq \cdots \leq
t_n$ and $0 \leq s_1 \leq \cdots \leq s_m \leq s$ 

TODO: Finish
\end{proof}

\begin{thm}[Strong Markov
  Property]\label{StrongMarkovPropertyBrownianMotion}Let $B_t$ be a
  Brownian motion and let $\tau$ be an almost surely finite $\mathcal{F}^+$-optional
  time, then $B^*_t = B_{\tau + t} - B_\tau$ is a standard Brownian
  motion independent of $\mathcal{F}^+_\tau$.
\end{thm}
\begin{proof}
\end{proof}

The following corollary of the strong Markov property turns out to be
a very useful tool in calculating the distributions of various
functions of Brownian motion.  It is called the reflection principle
because it shows that if one runs a Brownian motion up to an optional
time $\tau$ and then reverses the sign of all subsequent increments
(reflecting the graph of the Brownian motion with respect to the line
$y=\tau$) then the resulting process has same distribution.  TODO: Draw a picture illustrating the
geometry of reflection.
\begin{lem}[Reflection Principle]\label{ReflectionPrinciple}Let $B_t$ be a Brownian motion and let $\tau$ be an
  optional time then 
\begin{align*}
B^\prime_t &= B_{\tau \wedge t} - (B_t - B_{\tau \wedge t}) = \begin{cases}
B_t & \text{when $t \leq \tau$} \\
2 B_\tau - B_t & \text{when $t > \tau$}
\end{cases}
\end{align*}
is a Brownian motion with the same distribution as $B_t$.
\end{lem}
\begin{proof}
TODO:
\end{proof}

\begin{lem}\label{BrownianMaximumProcessLaw}Let $M_t = \sup_{0 \leq s \leq t}
  B_s$ be the maximal process associated with a standard Brownian
  motion then $M_t \eqdist \abs{B_t}$.
\end{lem}
\begin{proof}
TODO:
\end{proof}

TODO: Skorohod Embedding

\begin{thm}[Law of Iterated Logarithm]\label{LILBrownianMotion}Let
  $B_t$ be a standard Brownian motion then 
\begin{align*}
\limsup_{t \to \infty} \frac{B_t}{\sqrt{2t\log \log t}} &= 1 \text{ a.s.}
\end{align*}
\end{thm}
\begin{proof}
The basic idea of the proof is to examine the behavior of Brownian
paths sampled along the values of a geometric sequence $q^n$ for some
number $q > 1$.  Because we need to interpolate between sampling
points we must consider segments of the Brownian path between sampling
points.

To get started pick a number $q \in \rationals$ such that $q >1$ and
pick an $\epsilon > 0$ that we will later send to zero.  To clean up
the notation a bit define $\psi(t) = \sqrt{2t\log \log t}$ and let
$A_n = \lbrace \sup_{0 \leq t \leq q^n} B_t \geq (1+\epsilon)\psi(q^n)
\rbrace$.  By Lemma \ref{BrownianMaximumProcessLaw}, rescaling to a
standard normal random variable and the Gaussian tail bounds from
Lemma \ref{GaussianTailsElementary} we know that 
\begin{align*}
\probability{A_n} &= 
\probability{\abs{B_{q^n}} \geq (1 + \epsilon)  \psi(q^n)} \\
&=\probability{\frac{\abs{B_{q^n}}}{\sqrt{q^n}} \geq \frac{(1 + \epsilon)  \psi(q^n)}{\sqrt{q^n}}} \\
&\leq \frac {\sqrt{q^n}}{(1 + \epsilon)  \psi(q^n)} e^{-(1+\epsilon)^2
  \psi^2(q^n)/2q^n} \\
&=\frac {1}{(1 + \epsilon)  \sqrt{2\log \log (q^n)}} e^{-(1+\epsilon)^2
  \log \log (q^n)} 
\end{align*}
and there exists an $N_q$ depending only on $q$ such that the leading
constant is less than $1$ for $n \geq N_q$, so we have
\begin{align*}
\probability{A_n} &\leq \frac{1}{(n \log q)^{(1+\epsilon)^2}} 
\text{ for $n \geq N_q$}
\end{align*}
which shows that $\sum_{n=1}^\infty \probability{A_n} < \infty$. The
Borel Cantelli Theorem implies that almost surely at most finitely
many $A_n$ occur.  Thus almost surely there is an $N_\omega$ such that
$\abs{B_{q^n}} < (1 + \epsilon)  \psi(q^n)$ for all $n \geq N_\omega$.

Now for the other direction, again pick $q > 1$ and consider the
events
\begin{align*}
D_n &= \lbrace B_{q^n} - B_{q^{n-1}} \geq \psi(q^n - q^{n-1}) \rbrace
\end{align*}
We know that since $q \leq q^2 \leq \cdots$ so the $D_n$ are
independent events and $(B_{q^n} - B_{q^{n-1}})/\sqrt{q^n - q^{n-1}}$ is
$N(0,1)$ so we can apply Lemma \ref{GaussianTailsElementary} to see
that
for any $x \geq x_0$ we have
\begin{align*}
\probability{(B_{q^n} - B_{q^{n-1}})/\sqrt{q^n -  q^{n-1}} \geq x} \geq
\frac{x}{x^2+1} e^{-x^2/2} \geq \frac{x_0^2}{x_0^2+1} \frac{1}{x} e^{-x^2/2} 
\end{align*}
so if we let $c_1 =
\frac{2 \log \log q}{2 \log \log q+1}$ then 
\begin{align*}
\probability{D_n} &= \probability{(B_{q^n} - B_{q^{n-1}})/\sqrt{q^n -
  q^{n-1}} \geq \psi(q^n - q^{n-1})/\sqrt{q^n - q^{n-1}} } \\
&\geq c_1 \frac{e^{-\log \log (q^n - q^{n-1})}}{\sqrt{2 \log \log (q^n -
    q^{n-1})}}
\geq c_1 \frac{e^{-\log \log q^n}}{\sqrt{2 \log \log q^n}} \geq
\frac{c_2}{n \log n}
\end{align*}
so by the integral test we see that $\sum_{n=1}^\infty
\probability{D_n} = \infty$.  By Borel Cantelli we know that almost
surely there exists $N_1$ such that $B_{q^n} \geq B_{q^{n-1}} + \psi(q^n
- q^{n-1})$ for all $n \geq N_1$ (where $N_1$ depends on $q$ and $\omega
\in \Omega$).  To turn this into a lower bound on $B_{q^n}$ alone we
use the fact $-B_t$ is also a Brownian motion so we know from the
upper bound that we have already proven
\begin{align*}
\liminf_{t \to \infty} \frac{B_t}{\psi(t)} &= -\limsup_{t \to \infty}
\frac{-B_t}{\psi(t)} \geq -1 \text{ a.s.}
\end{align*}
If we pick an arbitrary $\epsilon > 0$ then almost surely there exists
$N_2$ such that for all $n \geq N_2$
$B_{q^n} \geq -(1 + \epsilon) \psi(q^n)$.  Therefore we have for all
$n \geq N_1 \wedge N_2$,
\begin{align*}
\frac{B_{q^n}}{\psi(q^n)} \geq \frac{B_{q^{n-1}} + \psi(q^n
- q^{n-1}) }{\psi(q^n)}
\geq \frac{-(1 + \epsilon) \psi(q^{n-1}) + \psi(q^n- q^{n-1})}{\psi(q^n)}
\end{align*}
Now we can provide lower bounds for $\psi(t)$ in the expressions
above.  Using the fact that $\psi(t)/\sqrt{t}$ is increasing (for
large $t$ TODO: make this precise)) we have
\begin{align*}
\frac{\psi(q^{n-1})}{\psi(q^n)} &=
\frac{\psi(q^{n-1})}{\sqrt{q^{n-1}}}
\frac{\sqrt{q^n}}{\psi(q^n)}\frac{1}{\sqrt{q}} \leq \frac{1}{\sqrt{q}} 
\end{align*}
and using the fact that $\psi(t)/t$ is increasing we have 
\begin{align*}
\frac{\psi(q^n- q^{n-1})}{\psi(q^n)} &\geq \frac{q^n - q^{n-1}}{q^n} =
1 - \frac{1}{q}
\end{align*}
so putting these facts together we get 
\begin{align*}
\limsup_{t \to \infty} \frac{B_t}{\psi(t)} &\geq \limsup_{n \to
  \infty} \frac{B_{q^n}}{\psi(q^n)} \geq
\frac{-(1+\epsilon)}{\sqrt{q}} + 1 - \frac{1}{q} \text{ a.s.}
\end{align*}
Now taking the intersection of countably many events of probability
$1$ over all $q \in \rationals$ this bound exists almost surely for
all rational numbers $q > 1$ so we may take the limit as $q \to
\infty$ and conclude that $\limsup_{t \to \infty} \frac{B_t}{\psi(t)}
\geq 1$.
\end{proof}
An additional scaling argument allows us to get a Law of Iterated
Logarithm for the limit as $t \to 0$,
\begin{cor}Let $B_t$ be a standard Brownian motion then 
\begin{align*}
\limsup_{t \to 0} \frac{B_t}{\sqrt{2 t \log \log (1/t)}} &= 1
\end{align*}
\end{cor}
\begin{proof}
We know that the rescaled process $X_t = t B_{1/t}$ for $t > 0$ is a
standard Brownian motion.  Therefore letting $h = 1/t$,
\begin{align*}
1 &= \limsup_{h \to \infty} \frac{X_h}{\sqrt{2 h \log \log (h)}} =
\limsup_{t \to 0} \frac{X_{1/t}}{\sqrt{2/t \log \log (1/t)}} =
\limsup_{t \to 0} \frac{B_t}{\sqrt{2 t \log \log (1/t)}}
\end{align*}
\end{proof}

Donsker's Theorem states roughly that Brownian motion can be
approximated in distribution by a suitably rescaled random walk.
Moreover it states that essentially all possible random walks that one
might expect could approximate Brownian motion in fact do.  This fact
shows that Brownian motion is analogous to standard normal
distributions and Donsker's Theorem is often referred to as the
Functional Central Limit Theorem.

\begin{thm}[Donsker's Invariance Principle]\label{DonskersTheorem}
Suppose we are given an i.i.d. sequence of random variables $\xi_1,
\xi_2, \dotsc$ such that $\expectation{\xi_n}=0$ and
$\variance{\xi_n}=1$ for all $n \in \naturals$.  Define the random
walk 
\begin{align*}
S_n &= \sum_{j=1}^n \xi_j
\end{align*}
its linear interpolation
\begin{align*}
S(t) &= S_{\floor{t}} + (t - \floor{t})(S_{\floor{t}+1} - S_{\floor{t}})
\end{align*}
and its rescaling from the interval $[0,n]$ to $[0,1]$
\begin{align*}
S_n^*(t) &= \frac{1}{\sqrt{n}} S(nt) & & \text{for $t\in [0,1]$}
\end{align*}
On the space $C[0,1]$ with the uniform norm, the sequence $S^*_n(t)$
converges in distribution to the standard Brownian motion.
\end{thm}

\begin{proof}
TODO
\end{proof}

TODO: Extension of Donsker's Theorem to convergence of errors of
empirical distributions to Brownian bridge.  This may be harder
because the convergence takes place not in the separable space
$C[0,1]$ but rather the space of cadlag functions (which is only
separable under the Skorohod topology).  The alternative here is
presumably to use the generalized form of weak convergence from
empirical process theory.

\chapter{Markov Processes}

TODO: Thinking about Markov processes as dynamical/deterministic
systems with (transduced) noise.

\section{Markov Processes}
The basic intuition of what a Markov process comprises is that it is a
stochastic process $X$ on a time scale $T$ such that for every time $t \in
T$ the future behavior of $X_u$ for $u \geq t$ only depends on the
past through the current value of $X_t$.  Alas, in practice the types
of problems that we concern ourselves with Markov process leads us to
a definition of a significantly more complicated object.  Rather than
pummel the reader with the definition we take the approach of starting
from simple intuition and building in the complexity by stages.  Some
readers may prefer to first jump to the end of this section to peer at
the final defintion so that it can be kept in mind during the journey.

\subsection{The Markov Property}
\begin{defn}Let $X$ be a process in $(S, \mathcal{S})$ with time scale
  $T$ which is adapted to a filtration $\mathcal{F}_t$.  We say that
  $X$ has the \emph{Markov property} if 
$\cindependent{\mathcal{F}_s}{X_t}{X_s}$ for all $s \leq t \in T$.
\end{defn}
Given any process that satisfies the Markov property it is not hard to
show using properties of conditional independence that it
automatically satisfies a seemingly stronger condition
\begin{lem}[Extended Markov Property]\label{ExtendedMarkovProperty}
Let $X$ be a process that satsifies the Markov property
  then $\cindependent{\mathcal{F}_t}{\sigma(\bigvee_{u \geq t}
    X_u)}{X_t}$ for all $t \in T$.
\end{lem}
\begin{proof}
Let $t_0 \leq t_1 \leq \cdots $ with $t_i \in T$.  By the
Markov property we know for each $0 \leq n$ that
$\cindependent{\mathcal{F}_{t_n}}{X_{t_{n+1}}}{X_{t_n}}$.  Because $X$ is
adapted to $\mathcal{F}$, we know that $X_{t_m}$ is
$\mathcal{F}_{t_n}$-measurable for $m \leq n$ and therefore
$\cindependent{\sigma(X_{t_0}, \dots
  ,X_{t_{n-1}},\mathcal{F}_{t_n})}{X_{t_{n+1}}}{X_{t_n}}$.
By Lemma \ref{ConditionalIndependenceChainRule} we conclude that $\cindependent{\mathcal{F}_{t_n}}{X_{t_{n+1}}}{X_{t_0}, \dots
  ,X_{t_n}}$ for all $n \geq 0$; because $\mathcal{F}_{t_0} \subset
\mathcal{F}_{t_n}$ we get $\cindependent{\mathcal{F}_{t_0}}{X_{t_{n+1}}}{X_{t_0}, \dots
  ,X_{t_n}}$ for all $n \geq 0$.  Another application of Lemma
\ref{ConditionalIndependenceChainRule} shows that
$\cindependent{\mathcal{F}_{t_0}}{\sigma(X_{t_1}, X_{t_2},
  \dots)}{X_{t_0}}$.

Since the union of the $\sigma$-algebras $\sigma(X_{t_1}, X_{t_2},
  \dots )$ for all $t_0 \leq t_1 \leq \cdots $ is
  clearly a $\pi$-system that generates $\sigma(\bigvee_{u \geq t_0}X_u)$, the result follows by montone classes (specifically Lemma \ref{ConditionalIndependencePiSystem}).
\end{proof}

TODO: Merge previous lemma into this propostion.
By Lemma \ref{ConditionalIndependenceDoob} the conditional independence in the Markov property can be captured by statements about conditional probabilities; it useful to have alternative equivalent characterizations of the Markov property in terms of conditional expectations of more general functions.
\begin{prop}\label{MarkovPropertyViaConditionalExpectations}Let $X$ be a process in $(S, \mathcal{S})$ with time scale
  $T$ which is adapted to a filtration $\mathcal{F}_t$.  Then the
  following are equivalent
\begin{itemize}
\item[(i)] $X$ has the Markov property
\item[(ii)] for all $t \in T$ and non-negative bounded $\sigma(\bigvee_{u \geq t}  X_u)$-measurable random variables $\xi$ we have
\begin{align*}
\cexpectationlong{\mathcal{F}_t}{\xi} &= \cexpectationlong{X_t}{\xi} \text{ a.s.}
\end{align*}
\item[(iii)] for all $s \leq t \in T$ and non-negative or bounded Borel measurable $f : S \to \reals$ we have
\begin{align*}
\cexpectationlong{\mathcal{F}_s}{f(X_t)} &= \cexpectationlong{X_s}{f(X_t)} \text{ a.s.}
\end{align*}
\end{itemize}
\end{prop}
\begin{proof}
To see that (i) implies (ii) suppose that $X$ has the Markov property and let $A \in \sigma(\bigvee_{u \geq t}  X_u)$.  By Lemma \ref{ExtendedMarkovProperty}
and Lemma \ref{ConditionalIndependenceDoob} we have
\begin{align*}
\cprobability{X_t}{A} &=
  \cprobability{\mathcal{F}_t, X_t}{A} =   \cprobability{\mathcal{F}_t}{A} \text{ a.s.}
\end{align*}
thus (ii) holds for indicator functions.  By linearity of conditional expectation (ii) holds for $\sigma(\bigvee_{u \geq t}  X_u)$-measurable simple functions.  By approximation by simple functions and Monotone Convergence for conditional expectations (ii) holds for non-negative $\sigma(\bigvee_{u \geq t}  X_u)$-measurable functions.  By linearity of conditional expectations we get (ii) for bounded $\sigma(\bigvee_{u \geq t}  X_u)$-measurable functions.

(ii) implies (iii) is immediate as any $f(X_t)$ if $\sigma(\bigvee_{u \geq t})$-measurable.

To see that (iii) implies (i) let $A \in \mathcal{S}$ and $s \leq t \in T$, then (iii) implies that 
\begin{align*}
\cprobability{X_s}{X_t \in A} 
&=  \cprobability{\mathcal{F}_s}{X_t \in A} 
= \cprobability{\mathcal{F}_s, X_s}{X_t \in A} \text{ a.s.}
\end{align*}
and Lemma \ref{ConditionalIndependenceDoob} implies that $\cindependent{\mathcal{F}_s}{X_t}{X_s}$.
\end{proof}

\subsection{Markov Transition Kernels}
 
TODO: Introduce the example of Markov Chains here as it is quite a bit
simpler and helps the
understanding of the abstract case quite a bit.

We know make a regularity assumption that for each pair $s,t \in T$
with $s \leq t$, we
have a probability kernel $\mu_{s,t} : S \times \mathcal{S} \to
\reals$ such that for every $A \in \mathcal{S}$
\begin{align*}
\mu_{s,t}(X_s, A) &= \cprobability{X_s}{X_t \in A} =
\cprobability{\mathcal{F}_s}{X_t \in A} \text{ a.s.}
\end{align*}
(e.g. if $S$ is a Borel space then this is true by Theorem
\ref{ExistenceConditionalDistribution}).
We let $\nu_t$ denote the distribution of $X_t$.
These conditional distributions characterize the distribution of the
process $X$ itself.  In particular we have the following nice formula
for finite dimensional distributions of the process.
\begin{lem}\label{MarkovDistributions}Let $X$ be a stochastic process
  on a time scale $T \subset \reals_+$ that has the Markov property,
  one dimensional distributions $\nu_t$ and transition kernels
  $\mu_{s,t}$.  Then for all $t_0 \leq \cdots \leq t_n$ and $A \in
  \mathcal{S}^{\otimes n}$ we have
\begin{align*}
\probability{(X_{t_1}, \dots, X_{t_n}) \in A} 
&= \nu_{t_1} \otimes
\mu_{t_1, t_2} \otimes \cdots \otimes \mu_{t_{n-1},t_n}(A) \\
\cprobability{\mathcal{F}_{t_0}}{(X_{t_1}, \dots, X_{t_n}) \in
  A}(\omega) 
&= \mu_{t_0, t_1} \otimes \cdots \otimes \mu_{t_{n-1},t_n}(X_{t_0}(\omega),A) \\
\end{align*}
\end{lem}
\begin{proof}
We begin by proving the first equality via induction.  The case $n=0$
is true by definition.  The induction step is 
really just a specific case of distintegration (Theorem 
\ref{Disintegration}) applied to the Markov transition kernels.  Let $A
\in \otimes_{i=0}^n \mathcal{S}$ then 
\begin{align*}
&\probability{(X_{t_0}, \cdots, X_{t_n}) \in A} \\
&=\expectation{\characteristic{A}(X_{t_0}, \cdots, X_{t_n})} \\
&=\expectation{\int \characteristic{A}(X_{t_0}, \cdots, X_{t_{n-1}},s)
  \, \mu_{t_{n-1}, t_n} (X_{n-1}, ds)} \\
&=\int \left [ \int \characteristic{A}(u_0, \cdots,u_{n-1}, s)   \,
  \mu_{t_{n-1}, t_n} (X_{n-1}, ds) \right ] \nu_{t_0} \otimes \cdots \otimes
  \mu_{t_{n-2}, t_{n-1}}(du_0, \dotsc, du_{n-1})\\
&=\nu_{t_0} \otimes \cdots \otimes  \mu_{t_{n-1}, t_{n}}(A)
\end{align*}

The second equality is derived from the first.  Suppose we have $A \in
\mathcal{S}$ and $B \in \mathcal{S}^{\otimes n}$.  Then we can compute
\begin{align*}
&\expectation{\characteristic{A}(X_{t_0}) \characteristic{B}(X_{t_1},
  \dotsc, X_{t_n})}  \\
&= \nu_{t_0} \otimes \mu_{t_0, t_1} \otimes \cdots \otimes  \mu_{t_{n-1},
  t_{n}}(A \times B) \\
&= \int \left [ \int \characteristic{B}(u_1, \dotsc, u_n) \mu_{t_0,
    t_1} \otimes \cdots \otimes \mu_{t_{n-1}, t_n} (u_0, du_1, \dotsc, du_n) \right ]
\characteristic{A}(u_0) \nu_{t_0}(u_0) \\
&=\expectation{\mu_{t_0, t_1} \otimes \cdots \otimes \mu_{t_{n-1},
    t_n}(X_0, B) \characteristic{A}(X_0)}
\end{align*}
Now the $\sigma(X_{t_0})$-measurability of $\mu_{t_0, t_1} \otimes \cdots \otimes \mu_{t_{n-1},
    t_n}(X_0, B)$ tells us that 
\begin{align*}
\cprobability{X_{t_0}}{(X_{t_1},\dotsc, X_{t_n}) \in B} 
&= \mu_{t_0, t_1} \otimes \cdots \otimes \mu_{t_{n-1},    t_n}(X_0, B)
\end{align*}

The last thing is to show that
$\cprobability{X_{t_0}}{(X_{t_1},\dotsc, X_{t_n}) \in B} =
\cprobability{\mathcal{F}_{t_0}}{(X_{t_1},\dotsc, X_{t_n}) \in B}$
a.s.  This follows from Lemma \ref{ExtendedMarkovProperty} since by
the tower property of conditional expectations and that result 
for any $A \in \mathcal{S}^{\otimes n}$ and $B \in \mathcal{F}_{t_0}$
\begin{align*}
\probability{(X_{t_1}, \dotsc, X_{t_n}) \in A ; B} 
&=\expectation{\cprobability{X_{t_0}}{(X_{t_1}, \dotsc, X_{t_n}) \in A ;  B}} \\
&=\expectation{\cprobability{X_{t_0}}{(X_{t_1}, \dotsc, X_{t_n}) \in A}
\cprobability{X_{t_0}}{B}} \\
&=\expectation{\cprobability{X_{t_0}}{(X_{t_1}, \dotsc, X_{t_n}) \in A}
\characteristic{B}} \\
\end{align*}
so the $\mathcal{F}_{t_0}$-measurability of
$\cprobability{X_{t_0}}{(X_{t_1}, \dotsc, X_{t_n}) \in A}$ gives the
result by the defining property of conditional expectations.
\end{proof}
A special case of the relations above should be called out as it
motivates a property that will assume as part of the definition of a
Markov process.  But first we need a definition.
\begin{defn}Let $\mu$ and $\nu$ be probability kernels from $S$ to
  $S$.  Then we define the probability kernel $\mu \nu$ from $S$ to
  $S$ by
\begin{align*}
\mu \nu (s, A) 
&= (\mu \otimes \nu)(s,S \times A)
= \iint \characteristic{S \times A}(t,u) \, \nu(t, du) \mu(s, dt) 
= \int \nu(t, A) \mu(s, dt)
\end{align*}
for all $s \in S$ and $A \in \mathcal{S}$.
\end{defn}
\begin{examp}
Let $S$ be a finite set and view $\mu$ and $\nu$ as $S \times S$
matrices as in Example \ref{ProbabilityKernelFiniteSampleSpace}.  Then $\mu \nu$ is just matrix multiplication:
\begin{align*}
\mu \nu (s, \lbrace t \rbrace) 
&= \int \nu(u, \lbrace t \rbrace) \mu(s,du) 
=\sum_{u \in S} \nu_{u, t} \mu_{s, u} = (\mu \nu)_{s,t}
\end{align*}
\end{examp}
\begin{cor}[Chapman-Kolmogorov
  Relations]\label{ChapmanKolmogorovWeak}Let $X$ be a stochastic
  process on a time scale $T \subset \reals$ with values in Borel
  space $(S, \mathcal{S})$ and suppose that $X$ has the Markov
  property.  Then for every $s, t, u \in T$ with $s \leq t \leq u$ we
  have
\begin{align*}
\mu_{s,t} \mu_{t,u} &= \mu_{s,u} \text{ a.s. $\nu_s$}
\end{align*}
\end{cor}
\begin{proof}
Since we have assume $S$ is a Borel space we know from Theorem \ref{ExistenceConditionalDistribution} that regular
versions $\mu_{s,t}$ exist.  By definition of $\mu_{s,t} \mu_{t,u}$, Lemma
\ref{MarkovDistributions} and the uniqueness clause of Theorem \ref{ExistenceConditionalDistribution}
\begin{align*}
\mu_{s,t} \mu_{t,u}(X_s, A) &= 
(\mu_{s,t} \otimes \mu_{t,u})(X_s, S \times A) \\
&= \cprobability{\mathcal{F}_s}{(X_t, X_u) \in S \times A} \\
&=\cprobability{\mathcal{F}_s}{ X_u \in A} \\
&=\cprobability{X_s}{ X_u \in A} \\
&= \mu_{s,u}(X_s, A) \text{  a.s.}
\end{align*}
Therefore for each $A \in \mathcal{S}$,
\begin{align*}
\nu_s(\mu_{s,t} \mu_{t,u} (\cdot, A) \neq \mu_{s,u}(\cdot, A)) &= \probability{\mu_{s,t} \mu_{t,u}(X_s, A) \neq \mu_{s,u}(X_s, A)} = 0
\end{align*}
Since $S$ is Borel we can choose a common null set for all $A \in \mathcal{S}$ (when $S=[0,1]$ just pick the union of null sets for $A$ an
interval with rational endpoints, show that this null set works for all intervals by continuity of measure and then use monotone classes; for general $S$ just
use the Borel isomorphism to reduce to the above case).
\end{proof}

\subsection{Existence of Markov Processes}

TODO: Example of process with the Markov property but for which the Chapman-Kolmogorov relations do not hold identically.

The ability to derive the almost sure version of the
Chapman-Kolmogorov relations is really just motivational for our
purposes.  In fact we will want to assume they hold identically in
what follows.  Absent a workable set of conditions from which we can
derive this fact, we build it into our definitions.  Collecting all of
the conditional independence and regularity properties we've
identified we finally make the formal definition of a Markov process.
\begin{defn}A \emph{Markov transition kernel} on time scale $T$ and 
state space $(S, \mathcal{S})$ is a probability  kernel $\mu_{s,t} : S \times \mathcal{S} \to [0,1]$ 
for each $s \leq t \in T$ such that 
\begin{align*}
\mu_{s,t} \mu_{t,u} &= \mu_{s,u}  \text{ everywhere on $S$ for  each $s \leq t \leq u$}
\end{align*}
\end{defn}

\begin{defn}
Let stochastic process $X_t$ on a time scale $T
\subset \reals_+$ and  state space $(S, \mathcal{S})$ such that $X_t$ is adapted to a filtration $\mathcal{F}_t$.  We say that
$X_t$ is a \emph{Markov process} if there exists a
Markov transition kernel $\mu_{s,t}$ such that for all $s \leq t \in T$
\begin{align*}
\cprobability{\mathcal{F}_s}{X_t \in \cdot} &= \mu_{s,t}(X_s, \cdot) \text{ a.s.}
\end{align*}
\end{defn}

Note that we have not specified in the definition that a Markov process posesses the Markov property; showing that
it does is not hard however.
\begin{prop}A Markov process has the Markov property.  Moreover for any non-negative or bounded function $f : S \to \reals$ we have
\begin{align*}
\cexpectationlong{\mathcal{F}_s}{f(X_t)} &= \cexpectationlong{X_s}{f(X_t)} = \int f(u) \, \mu_{s,t}(X_s, du)
\end{align*}
\end{prop}
\begin{proof}
Since we know $\cprobability{\mathcal{F}_s}{X_t \in A} = \mu_{s,t}(X_s, A)$ it follows that $\cprobability{\mathcal{F}_s}{X_t \in A}$ is $X_s$-measurable and therefore 
$\cprobability{\mathcal{F}_s}{X_t \in A} = \cprobability{X_s}{X_t \in A}$ a.s.  The Markov property follows by Proposition \ref{MarkovPropertyViaConditionalExpectations}.

Since $\mu_{s,t}(X_s, \cdot)$ is a regular version for $\cprobability{\mathcal{F}_s}{X_t \in \cdot}$ we apply Theorem \ref{Disintegration} to see
\begin{align*}
\cexpectationlong{\mathcal{F}_s}{f(X_t)} &= \int f(u) \, \mu_{s,t}(X_s, du)
\end{align*}
\end{proof}


TODO: Note that in the discrete (or countable?) state space case we
can in fact assume that Chapman-Kolmogorov are satisfied identically.

In lieu of general techinique for proving that a process is Markov
from general principles, we give a result that shows that we can
construct them from a set of transition kernels that obey the
Chapman-Kolmogorov relations.

TODO: There are other ways of proving a process is Markov : the
semigroup approach, the stochastic differential equation approach and
the martingale problem approach.  These are things we'll get to but
not quite yet!

\begin{thm}\label{ExistenceMarkovProcess}Suppose we are given
\begin{itemize}
\item[(i)] a  time scale starting at $0$, $T \subset \reals_+$ 
\item[(ii)]a Borel space $(S, \mathcal{S})$ 
\item[(iii)]a probability distribution $\nu$ on $(S, \mathcal{S})$
\item[(iv)]probability kernels $\mu_{s,t} : S \times
  \mathcal{S} \to [0,1]$ for each $s \leq t \in T$ such that 
\begin{align*}
\mu_{s,t} \mu_{t,u} &= \mu_{s,u} \text{ for all $s\leq
  t\leq u \in T$}
\end{align*}
\end{itemize} 
then there exists a Markov process $X_t$ with initial distribution
$\nu$ and transition kernels $\mu_{s,t}$.
\end{thm}
\begin{proof}
This is an application of the Daniell-Kolmogorov Theorem.  We first
define the finite dimensional distributions and show that they form a
projective family.  For every $n \in \naturals$ and $0 \leq t_1 \leq
\dotsb \leq t_n$ we define
\begin{align*}
\nu_{t_1, \dotsc, t_n}(A) 
&= \nu \mu_{0,t_1} \otimes \mu_{t_1, t_2} \otimes \dotsb \otimes \mu_{t_{n-1}, t_n}(A) \\
&=\nu \otimes \mu_{0,t_1} \otimes \mu_{t_1, t_2} \otimes \dotsb \otimes \mu_{t_{n-1}, t_n}(S \times A) 
\end{align*}
Let $A \in \mathcal{S}^{\otimes n-1}$ and let $1 \leq k \leq n$.
Define
\begin{align*}
A_k &= \lbrace (x_1, \dotsc, x_n) \in S^n \mid (x_1, \dotsc, x_{k-1},
x_{k+1}, \dotsc, x_n) \in A \rbrace
\end{align*}
and calculate
\begin{align*}
&\nu_{t_1, \dotsc, t_n}(A_k) = ( \nu \mu_{0,t_1} \otimes \mu_{t_1, t_2}
\otimes \dotsb \otimes \mu_{t_{n-1}, t_n})(A_k) \\
&=\int \left [ \int \left[ \dotsb \left [ \int \characteristic{A_k}(s_1,
    \dotsc, s_n) \, \mu_{t_{n-1},t_n}(s_{n-1}, ds_n) \right] \dotsb
\right] \, \mu_{t_1, t_2}(s_1, ds_2) \right ] \, \nu \mu_{0,
t_1}(ds_1) \\
&=\int \left [ \int \left[ \dotsb \left [ \int \characteristic{A}(s_1,
    \dotsc, s_{k-1}, s_{k+1}, \dotsc, s_n) \, \mu_{t_{n-1},t_n}(s_{n-1}, ds_n) \right] \dotsb \right] \, \mu_{t_1, t_2}(s_1, ds_2) \right ] \, \nu \mu_{0, t_1}(ds_1)
\end{align*}

The point here is that the integral
\begin{align*}
\int \left[ \dotsb \left [ \int \characteristic{A}(s_1,
    \dotsc, s_{k-1}, s_{k+1}, \dotsc, s_n) \, \mu_{t_{n-1},t_n}(s_{n-1}, ds_n) \right] \dotsb \right] \, \mu_{t_k, t_{k+1}}(s_{k+1}, ds_{k+2}) 
\end{align*}
is a function of $s_1, \dotsc, s_{k-1}, s_{k+1}$ only (i.e. it has no dependence on $s_k$).  From the Chapman-Kolmogorov relation
$\mu_{t_{k-1}, t_k} \mu_{t_{k}, t_{k+1}}  = \mu_{t_{k-1}, t_{k+1}}$ we
know that for any function of $f : S \to \reals$ we have 
\begin{align*}
\int \left [
  \int
f(s_{k+1}) \, \mu_{t_{k}, t_{k+1}} (s_k, ds_{k+1}) \right ] \, \mu_{t_{k-1}, t_k}
(s_{k-1}, ds_{k}) &= \int
f(s_{k+1}) \, \mu_{t_{k-1}, t_{k+1}} (s_{k-1}, ds_{k+1}) 
\end{align*}
which when applied to the integral above with $s_1, \dotsc, s_{k-2}$ fixed yields
\begin{align*}
&\int \left[ \dotsb \left [ \int \characteristic{A}(s_1,
    \dotsc, s_{k-1}, s_{k+1}, \dotsc, s_n) \,
    \mu_{t_{n-1},t_n}(s_{n-1}, ds_n) \right] \dotsb \right] \, \mu_{t_{k-1},
t_{k}}(s_{k-1}, ds_{k}) \\
&=\int \left[ \dotsb \left [ \int \characteristic{A}(s_1,
    \dotsc, s_{k-1}, s_{k+1}, \dotsc, s_n) \,
    \mu_{t_{n-1},t_n}(s_{n-1}, ds_n) \right] \dotsb \right] \,
\mu_{t_{k-1}, t_{k+1}}(s_{k-1}, ds_{k+1})
\end{align*}
Now we can use this to conclude that 
\begin{align*}
\nu_{t_1, \dotsc, t_n}(A_k) &= ( \nu \mu_{0,t_1} \otimes \mu_{t_1, t_2}
\otimes \dotsb \otimes \mu_{t_{n-1}, t_n})(A_k) \\
&= ( \nu \mu_{0,t_1} \otimes \mu_{t_1, t_2} \otimes \dotsb \otimes
\mu_{t_{k-1}, t_{k+1}} \otimes \dotsb \otimes \mu_{t_{n-1}, t_n})(A)
\\
&=\nu_{t_1, \dotsc, t_{k-1}, t_{k+1}, \dotsc, t_n}(A)
 \end{align*}
and we have show that the $\nu_{t_1, \dotsc, t_n}$ are a projective
family.
Now we can apply the Daniell-Kolmogorov Theorem
\ref{DaniellKolmogorovExtension} to conclude that there is an $S$
valued process $X$ on $T$ such that 
\begin{align*}
\mathcal{L}(X_{t_1}, \dotsc, X_{t_n}) = \nu_{t_1, \dotsc, t_n} &= \nu \mu_{0,t_1} \otimes \mu_{t_1, t_2}
\otimes \dotsb \otimes \mu_{t_{n-1}, t_n}
\end{align*}
for all $n \in \naturals$ and $0 \leq t_1 \leq \dotsb \leq t_n$.  The
case $n=1$ and $t_1 = 0$ shows us that $\mathcal{L}(X_0) = \nu
\mu_{0,0} = \nu$.

For every $t \in T$ define $\mathcal{F}_t = \sigma(X_s ; s \leq t)$ to
be filtration induced by $X$.  We must show that $X_t$ is a Markov process with transition
kernels $\mu_{s,t}$ (the fact that the initial distribution is $\nu$
was already noted).  Let $s \leq t$ be given and suppose that we have
$s_1 \leq \dotsb \leq s_n = s$.  Pick $A \in \mathcal{S}^{\otimes n}$
and $B \in \mathcal{S}$ and calculate using the FDDs of $X_t$ and the
expectation rule (Lemma \ref{ExpectationRule})
\begin{align*}
&\probability{ X_t \in B ; (X_{s_1}, \dotsc, X_{s_n}) \in A}  = \probability{(X_{s_1}, \dotsc, X_{s_n}, X_t) \in A \times B} =
\nu_{s_1, \dotsc, s_n, t}(A \times B) \\
&=\int \left [ \int \left[ \dotsb \left [ \int \characteristic{A}(u_1,
    \dotsc, u_n) \characteristic{B}(u_{n+1})\, \mu_{s,t}(u_{n}, du_{n+1}) \right] \dotsb
\right] \, \mu_{s_1, s_2}(u_1, du_2) \right ] \, \nu \mu_{0,
s_1}(du_1) \\
&=\int \left [ \int \left[ \dotsb \left [ \int \characteristic{A}(u_1,
    \dotsc, u_n) \mu_{s,t}(u_{n}, B)\, \mu_{s_{n-1},s_n}(u_{n-1}, du_{n}) \right] \dotsb
\right] \, \mu_{s_1, s_2}(u_1, du_2) \right ] \, \nu \mu_{0,
s_1}(du_1) \\
&=\expectation{ \mu_{s,t}(X_s, B) ; (X_{s_1}, \dotsc, X_{s_n}) \in A }
\end{align*}
Sets of the form $(X_{s_1}, \dotsc, X_{s_n}) \in A$ for $s_1 \leq
\dotsb \leq s_n=s$ are a $\pi$-system generating $\mathcal{F}_s$ and
therefore by a monotone class argument (specifically Lemma \ref{ConditionalExpectationExtension}) we may
conclude that $\cexpectationlong{\mathcal{F}_s}{X_t \in \cdot} =
\mu_{s,t}(X_s, \cdot)$ a.s.
\end{proof}

The previous theorem constructs a Markov process with an arbitrary initial distribution.  As it turns out in many cases
it is useful to consider the collection of Markov processes indexed by the initial distribution.  Such a collection has a
nice structure that results from the Markov property.  The uncover the structure the first thing to do is to move all of the 
constructed Markov processes into the canonical picture so that we have a family of probability measures on $S^T$.

\begin{defn}Suppose that a family of transition kernels $\mu_{s,t}$ is
  given.  For a distribution $\nu$ on $(S, \mathcal{S})$, let
  $\sprobabilityop{\nu}$ denote the distribution on $S^T$ of the
  Markov process with initial distribution $\nu$.  If $\nu=\delta_x$
  for some $x \in S$ then it is customary to write
  $\sprobabilityop{x}$ instead of $\sprobabilityop{\delta_x}$.
\end{defn}
\begin{lem}\label{MarkovMixtures}The family $\sprobabilityop{x}$ is a kernel from $S$ to
  $S^T$.  Futhermore, given an initial distribution $\nu$
\begin{align*}
\sprobability{A}{\nu} = \int \sprobability{A}{x} \, d\nu(x)
\end{align*}
\end{lem}
\begin{proof}
First assume that $A = (\pi_{t_1}, \dots, \pi_{t_n})^{-1}(B)$ for some
$B \in \mathcal{S}^{\otimes n}$ and $\pi_t : S^T \to S$ is the evaluation map $\pi_t f = f(t)$.  
We can use Lemma \ref{MarkovDistributions} and the expectation rule Lemma \ref{ExpectationRule}
to compute for any $\nu$,
\begin{align*}
\sprobability{A}{\nu} &= \sprobability{(\pi_{t_1}, \dotsc,  \pi_{t_n}) \in B}{\nu} \\
&=\sexpectation{\cprobability{\mathcal{F}_0}{(\pi_{t_1}, \dotsc,  \pi_{t_n}) \in B}}{\nu} \\
&=\sexpectation{\mu_{0, t_1} \otimes \cdots \otimes \mu_{t_{n-1}, t_n}(X_0, B)}{\nu} \\
&= \int \mu_{0, t_1} \otimes \cdots \otimes \mu_{t_{n-1}, t_n}(x,B) \, \nu(dx)
\end{align*}
In particular, for $\nu = \delta_x$ we get
\begin{align*}
\sprobability{A}{x} &=\mu_{0, t_1} \otimes \cdots \otimes \mu_{t_{n-1},  t_n}(x,B)
\end{align*}
which shows both that $\sprobability{A}{x}$ is a measurable function
of $x$ for fixed $A$ (Lemma
\ref{KernelTensorProductMeasurability}) and that
$\sprobability{A}{\nu} = \int \sprobability{A}{x} \, d\nu(x)$.

To extend to general measurable sets, we note that the set of $A$ of
the form given above is a $\pi$-system therefore we can apply Lemma
\ref{KernelMeasurability} to conclude $\sprobabilityop{x}$ is a
kernel.  Similarly we may conclude that $\sprobability{A}{\nu} = \int
\sprobability{A}{x} \, d\nu(x)$ for arbitrary measurable $A$ by the fact that probability measures
are uniquely determined by their values on a generating $\pi$-system
(Lemma \ref{UniquenessOfMeasure}).
\end{proof}

TODO:  This may not be the correct definition of a Markov process to
settle on.   We may want to select the picture of a Markov process as
being a single stochastic process with a family of probability
measures $\probabilityop_x$ for $x \in S$ such that under $\probabilityop_x$ the
stochastic process is Markov (as above) starting at $x$.  This
definition assumes that we have a kernel property (so Lemma
\ref{MarkovMixtures} proves such a kernel property
holds in the ``canonical'' case).  The work we have done to this point
shows that a set of transition kernels gives rise to a Markov process
with on the canonical space $S^T$.  The interpretation as a family of measures
without assuming the probability space is $S^T$ is apparently useful (e.g. when we want to
assume randomization variables exist for some construction).  I still find the variety of
interpretations of what a Markov process is to be very confusing.
Perhaps we should define this latter concept as a Markov family and
keep the current notion as a Markov process (I think Karatzas and
Shreve do this).  In the Karatzas and Shreve definition we wind up
with an interesting new concept which is that the kernel
$\probabilityop_x$ in a Markov family is only assumed to be \emph{universally measurable}
which is a looser condition than Borel measurability (the universal
$\sigma$-algebra being the intersection of the completions of the
Borel $\sigma$-algebra under all probability measures; hence being a
superset of the Borel $\sigma$-algebra).  This loosening
seems to come up as important in the context of stochastic control.  I
am not at all clear on how important it is in the context of Markov
processes as we are likely to develop it; it seems from Karatzas and
Shreve that this loosening comes up in Markov process theory when
trying to find a right-continuous complete filtration with respect to which a
Markov process (in particular Brownian motion) gives us a Markov
family.  So, we have shown that a by Kolmogorov existence we can
construct a Markov family given a set of transition kernels however
the filtration is not right continuous or complete and this
construction results in a Borel measurable kernel $\probabilityop_x$.
However if one tries to modify the construction to get a
right-continuous complete filtration (usual conditions) then one has
to give up Borel measurability in the kernel and make due with
universal measurability.  Perhaps it is worth having a definition of a Markov family 
and a ``relaxed'' or ``complete'' Markov family.  What I don't have any intuition of 
is the circumstances under which we are forced to pass to impose the usual conditions and/or require measurability of $\sprobability{A}{x}$ for arbitrary $A \in \mathcal{A}$.  
The most obvious answer is that we can't consider mixed initial states $\sprobability{A}{\nu}$ to be defined (via $\sprobability{A}{\nu} = \int \sprobability{A}{x}  \, \nu(dx)$) unless we have measurability of $\sprobability{A}{x}$ and therefore we don't even get measures $\probabilityop_{\nu}$ on all of $\mathcal{A}$ until we deal with the measurability issue (but we can always define $\probabilityop_{\nu}$ on $\mathcal{F}^X_\infty$).  Note also that the 
concept of universal measurability comes up in the general theory of processes in which
we have the Debut Theorem that states that every hitting time is universally measurable.  It can also be shown that
analytic sets are universally measurable.  

\begin{defn}\label{MarkovFamilyDefn}
A \emph{time homogeneous Markov family} is a stochastic process $X_t$ with a
probability space $(\Omega, \mathcal{A})$, a time scale $T
\subset \reals_+$, a filtration $\mathcal{F}_t$, a ($\mathcal{F}_t$?) measureable $\theta_t : \Omega \to \Omega$, a state space $(S, \mathcal{S})$ and a family of
probability measures $\probabilityop_x$ on $\Omega$ for $x \in S$ such that 
\begin{itemize}
\item[(i)]$\probabilityop_x$ is a (universally measurable?) kernel
  from $S$ to $\Omega$.  TODO: Is this what we want?  Blumenthal and Getoor say that $\sprobability{X_t \in A}{x}$ is $\mathcal{S}$ measurable for every
$0 \leq t < \infty$ and $A \in \mathcal{S}$; so $\probabilityop_x$ is a kernel to $(\Omega, \mathcal{F}^X_\infty)$ but not necessarily to $(\Omega, \mathcal{A})$ or 
$(\Omega, \mathcal{F}_\infty)$.
\item[(ii)]$\sprobability{X_0 = x}{x} = 1$ for all $x \in S$ (note Blumenthal and Getoor do not assume this).
\item[(iii)]$\cindependent{\mathcal{F}_s}{X_t}{X_s}$ under
  $\probabilityop_x$ for all $s \leq
  t$ and $x \in S$ (i.e. for all $x \in S$, $A \in \mathcal{S}$ and $s \leq t$ we have
  $\csexpectationlong{\mathcal{F}_s}{X_t \in A}{x} =
  \csexpectationlong{X_s}{X_t \in A}{x}$ $\probabilityop_x$-a.s.)  Alternatively do we just say:
\begin{align*}
\cexpectationlong{\mathcal{F}_s}{f (X_t \circ \theta_s)} &= \sexpectation{f(X_t)}{X_s}
\end{align*}
\item[(iv)]there exists a regular version $\mu^x_{s,t} :
  S \times \mathcal{S} \to [0,1]$ of $\csprobability{\mathcal{F}_s}{X_t
    \in \cdot}{x}$ for each $s \leq t$ and $x \in S$ (is there any coherence
  requirement with respect to $x \in S$ here???).
\end{itemize}
\end{defn}

\begin{lem}\label{AugmentingMeasurableSpace}Let $(S, \mathcal{S})$ be a measurable space and take a point $\Delta \notin S$, let $S^{\Delta} = S \cup \lbrace \Delta \rbrace$ and let $\mathcal{S}^\Delta$ be the $\sigma$-algebra on $S^\Delta$ generated by $\mathcal{S}$, then 
\begin{align*}
\mathcal{S}^\Delta &= \mathcal{S} \cup \lbrace A \cup \lbrace \Delta \rbrace \mid A \in \mathcal{S}\rbrace
\end{align*}
\end{lem}
\begin{proof}
Since $S \in \mathcal{S}$ and $\lbrace \Delta \rbrace = S^\Delta \setminus S$ it follows that $\lbrace \Delta \rbrace$ is $\mathcal{S}^\Delta$-measurable and therefore the right hand side is included in $\mathcal{S}^\Delta$.  It suffices to show that the right hand side is a $\sigma$-algebra.  Clearly $\emptyset \in \mathcal{S} \subset \mathcal{S}^\Delta$ and for any $A \in \mathcal{S}$ we have both $S^\Delta \setminus A = (S\setminus A) \cup \lbrace \Delta \rbrace$ and $S^\Delta \setminus (A \cup \lbrace \Delta \rbrace) = S \setminus A$ and therefore the right hand side is closed under set complement.  Given countable index sets $I$ and $J$ and sets $A_i \in \mathcal{S}$ and $B_j \in \mathcal{S}$ we have
\begin{align*}
\cup_{i \in I} A_i \cup \cup_{j \in J} (B_j \cup \lbrace \Delta \rbrace) &=
\begin{cases}
(\cup_{i \in I} A_i \cup \cup_{j \in J} B_j) \cup \lbrace \Delta \rbrace & \text{if $J \neq \emptyset$} \\
\cup_{i \in I} A_i & \text{if $J = \emptyset$} \\
\end{cases}
\end{align*}
which shows that the right hand side is closed under countable unions. 
\end{proof}
Adjoining a point to a measurable space as in the previous lemma will be referred to as \emph{augmenting} the space with the point $\Delta$.

\begin{defn}\label{HomogeneousMarkovFamilyBlumenthalGetoor}
A \emph{Markov family} is a
probability space $(\Omega, \mathcal{A})$ with a distinguished point $\omega_\Delta$, a time scale $T
\subset [0,\infty]$ with $\infty \in T$, a filtration $\mathcal{F}_t$, a measurable $\theta_t : \Omega \to \Omega$ such that $\theta_\infty \equiv \omega_\Delta$, a state space $(S, \mathcal{S})$ augmented with a point $\Delta$, an $\mathcal{F}$-adapted stochastic process $X_t$ with time scale $T$ and state space $S^\Delta$ and a family of
probability measures $\probabilityop_x$ on $(\Omega, \mathcal{A})$ for $x \in S^\Delta$ such that 
\begin{itemize}
\item[(i)] $X_t(\omega_\Delta) \equiv \Delta$ for all $t \in T$
\item[(ii)] $X_t(\omega) = \Delta$ for some $t \in T$ and $\omega \in \Omega$ then $X_u(\omega) = \Delta$ for all $u \geq t$.
\item[(iii)] $X_\infty(\omega) = \Delta$ for all $\omega \in \Omega$.
\item[(iv)] $\sprobability{X_0 = \Delta}{\Delta} = 1$
\item[(v)] For all $t \in T \setminus \lbrace \infty \rbrace$, $A \in \mathcal{S}$, $\sprobability{X_t \in A}{x}$ is $\mathcal{S}$-measurable (i.e. $\pushforward{X_t}{\probabilityop_x}$ is a kernel $S \times \mathcal{S} \to [0,1]$ for all $t \in T \setminus \lbrace \infty \rbrace$).
\item[(vi)] For all $s,t \in T$, $A \in \mathcal{S}^\Delta$ and $x \in S^\Delta$
\begin{align*}
\csprobability{\mathcal{F}_s}{X_{t} \circ \theta_s \in A}{x}&= \sprobability{X_t \in A}{X_s}
\end{align*}
\end{itemize}
If in addition we have $X_t \circ \theta_s = X_{t+s}$ for all $t,s \in T$ then we say that the Markov family is \emph{time-homogeneous}.
\end{defn}

It is worth calling out some subtle points of the defintion.  One of the more significant subtleties is the fact that we do not require that $\probabilityop_x : S^\Delta \times \mathcal{A} \to [0,1]$ is a kernel; while for fixed $x \in S^\Delta$ we know that $\probabilityop_x$ is a probability measure we do not assume that for fixed $A \in \mathcal{A}$ we have $\sprobability{A}{x}$ is $\mathcal{S}^\Delta$-measurable.  This issue will be discussed in some detail later on.  The other most notable issue is that our guiding intuition is that under the probability measure $\probabilityop_x$ $X_t$ is a Markov process starting at $x \in S$, yet we haven't stated that clearly as part of the definition.  At least the fact that $X_t$ is a Markov process under $\probabilityop_x$ $X_t$ can be proven.

\begin{prop} Let $(\Omega, \mathcal{A}, \mathcal{F}_t, X_t, \theta_t, \probabilityop_x)$ be a Markov family then
\begin{itemize}
\item[(i)]For all $t \in T$, $A \in \mathcal{S}^\Delta$, $\sprobability{X_t \in A}{x}$ is $\mathcal{S}^\Delta$-measurable (i.e. $\pushforward{X_t}{\probabilityop_x}$ is a kernel $S^\Delta \times \mathcal{S}^\Delta \to [0,1]$ for all $t \in T$).
\item[(ii)] If $X_t$ is time homogeneous then $\theta_t$ is $\mathcal{F}^X_\infty/\mathcal{F}^X_\infty$-measurable.
\item[(ii)] If $X_t$ is time homogeneous then $\mu_t(x,A) = \probabilityop_x(X_t \in A)$ defines a Markov transition kernel and the pair $X_t$, $\mu_t$ is a Markov process under $\probabilityop_x$.
\end{itemize}
\end{prop}
\begin{proof}
To see (i) note that for $T = \infty$ we have for all $A \in \mathcal{S}^\Delta$ and $x \in S^\Delta$
\begin{align*}
\sprobability{X_\infty \in A}{x} &= \sprobability{\Delta \in A}{x} = \begin{cases}
1 & \text{if $\Delta \in A$} \\
0 & \text{if $\Delta \notin A$} 
\end{cases}
\end{align*}
so is a constant function of $x$ and therefore $\mathcal{S}^\Delta$-measurable.  For $t \in T$ with $t \neq \infty$ and $A \in \mathcal{S}$ then as a function of $S^\Delta$,
\begin{align*}
\sprobability{X_t \in A}{x} &= \characteristic{S}(x) \sprobability{X_t \in A}{x} + \characteristic{\Delta}(x) \sprobability{X_t \in A}{\Delta} 
\end{align*}
which is $\mathcal{S}^\Delta$-measurable since $S, \lbrace \Delta \rbrace \in \mathcal{S}^\Delta$ and $\sprobability{X_t \in A}{x}$ is $\mathcal{S}$-measurable.

For $t \in T$ with $t \neq \infty$ and $B \in \mathcal{S}^\Delta \setminus \mathcal{S}$ by Lemma \ref{AugmentingMeasurableSpace} we know that $B = A \cup \lbrace \Delta \rbrace$ for some $A \in \mathcal{S}$ and thus
\begin{align*}
\sprobability{X_t \in A \cup \lbrace \Delta \rbrace }{x} &=\sprobability{X_t \in A}{x} + \sprobability{X_t = \Delta }{x} = \sprobability{X_t \in A}{x} + 1 - \sprobability{X_t \in S}{x} 
\end{align*}
is seen to be $\mathcal{S}^\Delta$-measurable.

To see (ii) note that by time homogeneity we have for $s, t \in T$ and $A \in \mathcal{S}^\Delta$, $\theta_s^{-1} X_t^{-1}(A) = X_{t+s}^{-1}(A)$ which shows that $\theta_s$ is $\mathcal{F}^X_t/\mathcal{F}^X_{t+s}$-measurable.  In particular, (ii) follows.

To see (iii) we first show that $\mu_t(x,A) = \sprobability{X_t \in A}{x}$ is a Markov transition kernel.  By (i) we have shown that it is a kernel from $S^\Delta$ to $\mathcal{P}(S^\Delta)$ so it remains to show the Chapman Kolmogorov relations.  Let $t,s \in T$, $x \in S^\Delta$ and $A \in \mathcal{S}^\Delta$ then by the tower property of conditional expectation, property (vi) of the defintion of a Markov family and the Expectation Rule Lemma \ref{ExpectationRule} (applied to $\pushforward{X_s}{\probabilityop_X} = \mu_s(x, \cdot)$)
\begin{align*}
\mu_{s+t}(x, A) &= \sprobability{X_{s+t} \in A}{x} \\
&=\sexpectation{\csprobability{\mathcal{F}_s}{X_{t+s} \in A}{x}}{x} \\
&=\sexpectation{\sprobability{X_{t} \in A}{X_s}}{x} \\
&=\sexpectation{\mu_t(X_s, A)}{x} \\
&=\int \mu_t(u, A) \mu_s(x, du) = \mu_s \mu_t (x, A) \\
\end{align*}
showing the Chapman-Kolmogorov relations.  Now property (vi) of the definition of a Markov family shows $\csprobability{\mathcal{F}_s}{X_{t+s} \in A}{x} = \mu_t(X_s,A)$ and
therefore $X_t$ is a Markov process under $\probabilityop_x$ for every $x \in S^\Delta$.
TODO: A non-homogeneous version of this proof and the proper definition of a non-homogeneous Markov family????
\end{proof}

Note that Bass doesn't require that $\probabilityop_x$ is a kernel $S
\to \mathcal{P}(\Omega)$ rather he only requires that
$\pushforward{X}{\probabilityop_x}$ is a kernel from $S$ to
$\mathcal{P}(S^T)$ (equivalently for each $t \in T$ and $A \in \mathcal{S}$ we have
$\sprobability{X_t \in A}{x}$ is a measurable function of $x$ or again
equivalently $\probabilityop_x$ is a kernel on the natural filtration
$\mathcal{F}^X_\infty$ (this is the same as Blumenthal and Getoor as I mention above).  What I don't know if whether a universally
measurable kernel $S \to  \mathcal{P}(\Omega)$ is necessarily Borel
measurable when restricted to $\mathcal{F}^X_\infty$; heck this isn't necessarily a well posed question since $S$ is not assumed to be
a topological space at this point.  It is worth noting that Blumenthal and Getoor do show that every Markov family extends to a completed one with 
the state space given by the universal completion and they assume this completion is in place for much of the subsequent theory.  That said, they
don't modify the defintion of Markov family in doing so.

TODO: Question:  Given a Markov family as above then given an arbitrary
initial distribution $\nu$ on $S$ we can define $\probabilityop_\nu$
by $\sprobability{A}{\nu} = \int \sprobability{A}{x} \, d\nu(x)$.  Is
$X$ a Markov process with initial distribution $\nu$ under
$\probabilityop_\nu$?  Blumenthal and Getoor do this but alas universal measurability arises here as well (not when defining
$\sprobability{A}{\nu}$ for $A \in \sigma(\bigvee_{t \in T} X_t)$ but only when trying to extend to $\mathcal{F}_\infty$)!

\section{Homogeneous Markov Processes}

We have described a relatively general version of Markov processes
compared to what it needed in many applications and the goal of this
section is to define the assumptions that lead to useful
simplifications and to understand how to look at these simplifying
assumptions from a couple of points of view.

\begin{defn}Suppose $(S, \mathcal{S})$ is a measurable Abelian group
  and $\mu : S \times \mathcal{S} \to [0,1]$ is a kernel.  We say
  $\mu$ is \emph{homogeneous} if for every $s \in S$ and
  $A\in \mathcal{S}$ we have $\mu(0, A) = \mu(s, A+s)$.
\end{defn}

A useful observation for computing conditional expectations is that
integrals are invariant under certain changes of variables.
\begin{lem}\label{HomogeneousKernelExpectationRule}Let $(S, \mathcal{S})$ be a measurable Abelian group with a
  homogeneous kernel $\mu : S \times \mathcal{S} \to [0,1]$, then for
  each $y,z \in S$ and integrable $f : S \to \reals$,
\begin{align*}
\int f(x + y) \, \mu(z, dx) &= \int f(x) \, \mu(y+z, dx)
\end{align*}
\end{lem}
\begin{proof}
For $y \in S$, let $t_y : S \to S$ be translation by $y$: $t_y(x) = x
+ y$.  Thinking of the kernel as a measurable measure
valued map (which we denote $\mu(z)$) we compute the pushforward of $\mu(z)$ under $t_y$
using homogeneity
\begin{align*}
\pushforward {t_y}{\mu(z)}(A) &= \mu(z, t_y^{-1}(A)) = \mu(z, A
- y) = \mu(z+y, A)
\end{align*}
thus showing $\pushforward {t_y}{\mu(z)} = \mu(y+z)$.
Now we can apply the Expectation Rule (Lemma \ref{ExpectationRule}) to
see that 
\begin{align*}
\int f(x + y) \, \mu(z, dx) &= \int f(x) \, d\left [ \pushforward {t_y}{\mu(z)} \right] = \int f(x) \, \mu(y+z, dx)
\end{align*}
\end{proof}

A Markov process with homogeneous kernels is said to be
\emph{space-homogeneous}; intuitively the probability of starting out
in a set $A$ at time $s$ and winding up in set $B$ at time $t$ only
depends on the relative positions of $A$ and $B$ (under translations).
\begin{defn}Suppose $(S, \mathcal{S})$ is a measurable Abelian group
  and let $X_t$ be a Markov process with transition kernels
  $\mu_{s,t}$.  Then $X_t$ is \emph{space-homogeneous} if and only if
  $\mu_{s,t}$ is homogeneous for every $s \leq t$.
\end{defn}

\begin{lem}\label{SpaceHomogeneousMarkovDistributions}Let $\mu_{s,t}$
  be a family of space homogeneous transition kernels on a
  measurable Abelian group, then for every $A \in \mathcal{S}^T$ and $x \in S$,
  $\sprobability{A}{x} = \sprobability{A-x}{0}$.
\end{lem}
\begin{proof}
TODO: This proof only seems to require space homogeneity of the
kernels $\mu_{0,t}$; is this a mistake (or does Chapman Kolmogorov
imply the rest of the kernels are space homogeneous as well...)

We begin by establishing the result for sets of the form $\lbrace (X_{t_1},
\dotsc, X_{t_n}) \in A \rbrace$ for $A \in \mathcal{S}^{\otimes n}$ and $t_1 \leq
\cdots \leq t_n$.  The key point is that we know from the proof of
Lemma \ref{MarkovMixtures} that $\sprobability{(X_{t_1},
\dotsc, X_{t_n}) \in A }{x} = \mu_{0, t_1} \otimes \dotsc \otimes
\mu_{t_{n-1}, t_n}(x, A)$, so in particular the case $n=1$ follows
directly from the assumption that each $\mu_{0, t}$ is homogeneous.
To see the result for $n > 1$ we calculate using Lemma \ref{HomogeneousKernelExpectationRule}
\begin{align*}
&\sprobability{(X_{t_1},\dotsc, X_{t_n}) \in A }{x} \\
&= \mu_{0, t_1} \otimes \dotsc \otimes \mu_{t_{n-1}, t_n}(x, A) \\
&= \int \int \characteristic{A}(x_1, x_2, \dotsc, x_n) \mu_{t_1, t_2} \otimes \cdots \otimes \mu_{t_{n-1},
  t_n}(x_1, dx_2, \dotsc, dx_n) \, \mu_{0,t_1}(x, dy) \\
&= \int \int \characteristic{A}(x_1+x, x_2, \dotsc, x_n) \mu_{t_1, t_2} \otimes \cdots \otimes \mu_{t_{n-1},
  t_n}(x_1, dx_2, \dotsc, dx_n) \, \mu_{0,t_1}(0, dy) \\
&= \int \int \characteristic{A-x}(x_1, x_2, \dotsc, x_n) \mu_{t_1, t_2} \otimes \cdots \otimes \mu_{t_{n-1},
  t_n}(x_1, dx_2, \dotsc, dx_n) \, \mu_{0,t_1}(0, dy) \\
&= \mu_{0, t_1} \otimes \dotsc \otimes \mu_{t_{n-1}, t_n}(0, A-x) \\
&=\sprobability{(X_{t_1},\dotsc, X_{t_n}) \in A - x}{0} \\
\end{align*}

Now we complete the result by a monotone class argument.  We know that
sets of the form $\lbrace (X_{t_1}, \dotsc, X_{t_n}) \in A \rbrace$
are a generating $\pi$-system so by the $\pi$-$\lambda$ Theorem
(Theorem \ref{MonotoneClassTheorem}) it suffices to show that $\mathcal{C}
= \lbrace A \mid \sprobability{A}{x} = \sprobability{A-x}{0} \rbrace$ is a
$\lambda$-system. If $A,B \in \mathcal{C}$ with $A \subset B$ then 
\begin{align*}
\sprobability{B\setminus A}{x} &= \sprobability{B}{x} -
\sprobability{A}{x} = \sprobability{B-x}{0} - \sprobability{A-x}{0} =
\sprobability{B\setminus A-x}{0} 
\end{align*}
where we have used the elementary fact that $B\setminus A - x =
(B-x)\setminus(A-x)$ (let $y \in B$ and $y \notin A$ then clearly $y-x
\in B-x$ and $y-x \notin A-x$).  Similarly if $A_n \in \mathcal{C}$
for $n \in \naturals$ with $A_1 \subset A_2 \subset \cdots$ then it is
also true that $A_1 -x  \subset A_2-x \subset \cdots$ and continuity
of measure (Lemma \ref{ContinuityOfMeasure}) shows
\begin{align*}
\sprobability{\cup_n A_n}{x} &=
\lim_{n \to \infty}\sprobability{A_n}{x} = 
\lim_{n \to \infty}\sprobability{A_n-x}{0} = 
\sprobability{\cup_n A_n-x}{0} 
\end{align*}
\end{proof}

There is another way of thinking about the space-homogeneous Markov
processes.  We know that for any $s \leq t$, given the value of $X_s$ the probability
distribution of $X_t$ is independent of the history of $X$ up to
$s$.  Space homogeneity tells us that moreover that the probability
distribution $X_t$ only depends on the \emph{increment} $X_t - X_s$.
Putting these two observations together we should expect that $X_t -
X_s$ is independent (not just conditionally independent) of the
history of $X$ up to $s$.  In fact this provides an equivalent
characterisation of space homogeneous Markov processes as we prove in
the following result.

\begin{defn}Let $(S, \mathcal{S})$ be a measurable Abelian group with a
  time scale $T \subset \reals_+$, a filtration $\mathcal{F}_t$ and an
 $S$-valued $\mathcal{F}$-adapted process $X_t$. We say that $X_t$ has
 $\mathcal{F}$-independent increments if and only if $X_t - X_s$ is
 independent of $\mathcal{F}_s$ for all $s \leq t$.
\end{defn}

\begin{lem}\label{IndependentIncrements}Let $(S, \mathcal{S})$ be a measurable Abelian group with a
  time scale $T \subset \reals_+$, a filtration $\mathcal{F}_t$ and an
 $S$-valued $\mathcal{F}$-adapted process $X_t$.  The $X_t$ has
$\mathcal{F}$-independent increments if and only if $X_t$ is a space-homogeneous
 Markov process.  In this case the transition kernels of $X_t$ are
 given by
\begin{align*}
\mu_{s, t}(x, A) &= \probability{X_t - X_s \in A - x} \text{ for $x
  \in S$, $A \in \mathcal{S}$ and $s \leq t \in T$}
\end{align*}
TODO: The proof actually requires regular versions of
$\cprobability{\mathcal{F}_s}{X_t}$; do we need to assume that
$G$ is Borel or something?  Also we've defined a Markov process as
satisfying the Chapman Kolmogorov relations identically; can that be derived?
\end{lem}
\begin{proof}
Suppose that $X_t$ is a space homogeneous Markov Process with
transition kernels $\mu_{s,t}$.  Then for every $s\leq t$ and $A \in \mathcal{S}$,
\begin{align*}
\cprobability{\mathcal{F}_s}{X_t - X_s \in A} &= 
\int \characteristic{A}(x - X_s) \, \mu_{s,t}(X_s, dx) & & \text{by
  Theorem \ref{Disintegration}} \\
&=\int \characteristic{A}(x) \, \mu_{s,t}(0, dx) & & \text{by Lemma
  \ref{HomogeneousKernelExpectationRule}} \\
&= \mu_{s,t}(0,A)
\end{align*}
which shows that $\cprobability{\mathcal{F}_s}{X_t - X_s \in A}$ is
almost surely constant hence $\cindependent{X_t -
  X_s}{\mathcal{F}_s}{}$.  Moreover by the tower rule we also know
that $\probability{X_t - X_s \in A}=\cprobability{\mathcal{F}_s}{X_t -
  X_s \in A}=\mu_{s,t}(0,A)$ and therefore by another application of
space homogeneity, $\mu_{s, t}(x, A) = \mu_{s,t}(0, A-x) = \probability{X_t - X_s \in A}$.

Suppose that $X_t$ has independent increments.  The key point is that
this property determines the conditional distributions 
\begin{align*}
\mu_{s,t}(x, A) &= \probability{X_t - X_s \in A - x}
\end{align*}
and moreover this form is a regular version.  First note that $\probability{X_t - X_s \in A - x}$ is a probability
kernel since for fixed $A$ it is measurable in $x$ by Lemma
\ref{MeasurableSections} and for fixed $x$ it is just the distribution
of the measurable random element $X_t - X_s -x$.  

Showing that $\probability{X_t - X_s \in A - x}$ is a version of
$\cprobability{\mathcal{F}_s}{X_t \in A}$ is not hard but requires a
bit of care because the random element $X_s$ plays two different roles
in the calculation and it is worth making this fact explicit.
We start by defining 
$\tilde{\mu}_{s,t}(x, A) = \probability{X_t - X_s \in A}$ and observing
that because $\cindependent{X_t - X_s}{\mathcal{F}_s}{}$,
$\tilde{\mu}_{s,t}$ is a kernel for $\cprobability{\mathcal{F}_s}{X_t
  - X_s \in \cdot}$.  With this fact and the $\mathcal{F}$-adaptedness
of $X$, we can apply Theorem
\ref{Disintegration} (using the function $f(x,y) =
\characteristic{A-y}(x)$ evaluated at $(X_t - X_s, X_s)$) to conclude
\begin{align*}
\cprobability{\mathcal{F}_s}{X_t \in A} &= 
\cprobability{\mathcal{F}_s}{X_t - X_s \in A - X_s} \\
&=\int \characteristic{A - X_s}(x) \, \tilde{\mu}_{s,t}(dx) \\
&=\tilde{\mu}_{s,t}(A - X_s) \\
&=\mu_{s,t}(X_s, A)
\end{align*}
Now note that $\mu_{s,t}(X_s, A)$ is $X_s$-measurable hence we have
$\cprobability{\mathcal{F}_s}{X_t \in A} = \cprobability{X_s}{X_t
  \in A} $ for all $A \in \mathcal{S}$ thus the Markov property
holds by Lemma \ref{ConditionalIndependenceDoob}.  
Using the explicit form of the kernel we calculate
\begin{align*}
\mu_{s,t}(x, A) &=  \probability{X_t - X_s \in A-x} = \mu_{s,t}(0, A-x) 
\end{align*}
demonstrating space homogeneity.
\end{proof}

Here is what the proof that space homogeneous Markov implies
independent increments looks like in elementary probability theory
(discrete time countable state space).
\begin{proof}
Space homogeneity means that $\cprobability{X_{n-1} = y}{X_n = x} =
\cprobability{X_{n-1} = 0}{X_n = x-y}$.  This implies that for any
$y\in S$ we have
$\probability{X_n - X_{n-1} = z} = \cprobability{X_{n-1} = y}{X_n =
  z+y}$:
\begin{align*}
\probability{X_n - X_{n-1} = z} &= \sum_x \probability{X_n - X_{n-1} =
  z ; X_{n-1}=x} \\
& =\sum_x \cprobability {X_{n-1}=x}{X_n - X_{n-1} =  z} \probability
{X_{n-1}=x} \\
& =\sum_x \cprobability {X_{n-1}=x}{X_n =  z+x} \probability
{X_{n-1}=x} \\
& =\cprobability {X_{n-1}=y}{X_n =  z+y} \sum_x \probability
{X_{n-1}=x} \\
&= \cprobability {X_{n-1}=y}{X_n =  z+y} 
\end{align*}
Now we use this fact along with the Markov property to see
\begin{align*}
&\probability{X_n - X_{n-1} = z; X_1=x_1 ; \cdots ; X_{n-1}=x_{n-1}} \\
&=\probability{X_n = z+x_{n-1}; X_1=x_1 ; \cdots ; X_{n-1}=x_{n-1}} \\
&=\cprobability{X_1=x_1 ; \cdots ; X_{n-1}=x_{n-1}}{X_n = z+x_{n-1}} 
\probability{X_1=x_1 ; \cdots ; X_{n-1}=x_{n-1}}\\
&=\cprobability{X_{n-1}=x_{n-1}}{X_n = z+x_{n-1}} 
\probability{X_1=x_1 ; \cdots ; X_{n-1}=x_{n-1}}\\
&=\probability{X_n = z} 
\probability{X_1=x_1 ; \cdots ; X_{n-1}=x_{n-1}}\\
\end{align*}
\end{proof}

TODO: Motivate time homogeneity by thinking about discrete time and
the fact that you can generate everything from the unit time
transitions.  Time homogeneity is the property that all of these
transition kernels are the same and therefore the Markov process is
determined by a single kernel (and the initial distribution).

\begin{defn}A family of transition kernels $\mu_{s,t}$  $\integers_+$ or $\reals_+$ is said to
  be \emph{time homogeneous} if and only if there exist a family of
  kernels $\tilde{\mu}_t$ such that $\mu_{s,t}(x, B) =
  \tilde{\mu}_{t-s}(x,B)$ for all $x \in S$ and $B \in \mathcal{A}$.
  A Markov process $X$ is said to be time homogeneous if it has a
  family of time homogeneous transition kernels.
\end{defn}

\begin{prop}If $X$ is a time homogeneous Markov process then
for all $s,t,u \in T$ and $B
  \in \mathcal{S}^T$ we have
  $\cexpectationlong{\mathcal{F}_s}{X_t \in B} = \cexpectationlong{\mathcal{F}_{s+u}}{X_{t+u} \in B}$.
\end{prop}
\begin{proof}
Immediate from the defintions,
\begin{align*}
\cexpectationlong{\mathcal{F}_s}{X_t \in B} = \mu_{t-s}(X_s, B) = \cexpectationlong{\mathcal{F}_{s+u}}{X_{t+u} \in B}
\end{align*}
\end{proof}

\section{Strong Markov Property}

In dealing with Markov processes we make a lot of use of constructions
that involve the following
\begin{defn}If $T$ is equal to $\integers_+$ or $\reals_+$, for each
  $t \in T$ we define
  the \emph{shift operator}  $\theta_t : S^T \to S^T$ by $\theta_t f
  (s) = f(s +t)$.
\end{defn}

It is clear that for a fixed $t \in T$ the shift operator $\theta_t$
is measurable but we often need a stronger property the requires some
more assumptions.
\begin{lem}\label{MeasurabilityOfShiftOperator}For any fixed $t \in T$ the shift operator $\theta_t : S^T
  \to S^T$ is measurable. If $U$ is equal to $S^\infty$, $C(T; S)$ or
  $D(T; S)$, then $\theta_t X$ defines a measurable function $\theta : U \cap S^T \times
  T \to U \cap S^T$.  
\end{lem}
\begin{proof}
First let $t \in T$ be fixed pick $s \in T$ and $A \in \mathcal{S}$.
Then $\theta_t^{-1} \lbrace f(s) \in A \rbrace = \lbrace f(s+t) \in A
\rbrace \in \mathcal{S}^T$.  Therefore since sets of the form $\lbrace f(s) \in A \rbrace$ generate
$\mathcal{S}^T$, we see that $\theta_t$ is measurable by Lemma
\ref{MeasurableByGeneratingSet}.

Now let $U$ be as above.  It is clear that the shift operator
preserves the necessary continuity and limit properties and thus is
well defined as a function $\theta : U \cap S^T \times T \to U \cap S^T$.  To
see measurability of $\theta$, first note that the evaluation map $\pi : U \cap S^T \times
T \to S$ given by $\pi(f,t) = f(t)$ is measurable (e.g. this follows
by considering the process defined by the identity $U
\cap S^T \to U \cap S^T$ and using Lemma
\ref{ContinuityAndProgressiveMeasurability} to see that it is jointly
measurable).
Now let $s \in T$ and $A
\in \mathcal{S}$ as before and calculate
\begin{align*}
\lbrace (f, t) \mid \theta_t f \in \pi_s^{-1} A \rbrace
&=
\lbrace (f, t) \mid \theta_t f(s) \in  A \rbrace 
=\lbrace (f, t) \mid \theta_s f(t) \in  A \rbrace \\
&= (\theta_s, id)^{-1} \lbrace (f, t) \mid f(t) \in  A \rbrace
= (\theta_s, id)^{-1} \pi^{-1} A
\end{align*}
which is measurable by the joint measurability of $\pi$ noted above
and the measurability of $\theta_s$ for fixed $s \in T$.
\end{proof}

When considering Markov processes on the canonical space there is a
very useful construction of time shifting optional times.  Intuitively
the construction is that given two optional times $\sigma$ and $\tau$
one constructs the random time which is ``the first time $\tau$
happens after $\sigma$ happens''.  The following Lemma makes the
construction precise and shows that under some assumption on the path
space that the construction gives us a weak optional time.
\begin{lem}\label{TimeShiftOptionalTimes}Let $S$ be a metric space and
  let $\sigma$ and $\tau$ be weakly optional times on any of the canonical
  spaces $S^\infty$, $C([0,\infty); S)$ or $D([0,\infty); S)$ provided
  with the canonical filtration $\mathcal{F}$.   Then 
\begin{align*}
\gamma &= \begin{cases}
\sigma + \tau  \circ \theta_\sigma & \text{when $\sigma<\infty$} \\
\infty & \text{when $\sigma=\infty$}
\end{cases}
\end{align*}
is also weakly $\mathcal{F}$-optional.
\end{lem}
\begin{proof}
Let $X$ be the canonical process (i.e. $X_t$ is the evaluation
function $\pi_t$).  

First we claim that $\gamma$ is measurable.  This follows by noting
that $\theta_\sigma$ is measurable by writing it as $\theta \circ (id,
\sigma)$ and using by Lemma \ref{MeasurabilityOfShiftOperator}.
Therefore $\gamma$ is measurable by the measurability of
$\theta_\sigma$, $\sigma$
and $\tau$ and application of Lemma \ref{CompositionOfMeasurable} and
Lemma \ref{ArithmeticCombinationsOfMeasurableFunctions}.

Next we claim that if we pull back $\mathcal{F}_t$ by $\theta_\sigma$
then result should only depend on values  $X_s$ for $\sigma \leq s
\leq \sigma + t$ hence should be $\mathcal{F}^+_{\sigma +
  t}$-measurable.  We have to be a bit careful with this claim, because $\sigma$
can be infinite in which case $\theta_\sigma$ isn't defined.  To make
the claim precise and to prove it pick $n \geq 0$ and note that by either discreteness or by continuity of sample
paths together with Lemma
\ref{ContinuityAndProgressiveMeasurability} we know that $X$ is $\mathcal{F}$-progressively measurable. 
By $\mathcal{F}^+$-optionality of $\sigma \wedge n$ and Lemma
\ref{StoppedProgressivelyMeasurableProcess} we know that
$X_{\sigma \wedge n +s} = X_s \circ \theta_{\sigma \wedge n}$ is
$\mathcal{F}^+_{\sigma \wedge n+s}$-measurable for all $s \geq 0$.
Now fix $t \geq 0$ then for $0 \leq s \leq t$, pick a measurable set $B
\in \mathcal{S}$
and let $A = \lbrace X_s \in B \rbrace$; we note that
 $\theta_{\sigma \wedge n}^{-1} A = (X_s \circ \theta_{\sigma \wedge n})^{-1}(B) =
X_{\sigma \wedge n+s}^{-1}(B) \in \mathcal{F}^+_{\sigma \wedge n+s} \subset
\mathcal{F}^+_{\sigma \wedge n+t} $.  
Since $\lbrace A \mid \theta_{\sigma \wedge n}^{-1} A \in
\mathcal{F}^+_{\sigma \wedge n+t} \rbrace$ is a $\sigma$-algebra (Lemma
\ref{SigmaAlgebraPullback}) and sets of the form $\lbrace X_s \in B
\rbrace$ for $0 \leq s \leq t$ generate $\mathcal{F}_t$, we know that
$\theta_{\sigma \wedge n}^{-1} \mathcal{F}_t \subset
\mathcal{F}^+_{\sigma \wedge n +t}$ for all $t \geq 0$ and $n \geq 0$.

Now fix $0 \leq t < \infty$, let $n = \floor{t}+1$
and note that
\begin{align*}
\lbrace \gamma < t \rbrace &= \cup_{\substack{0 < r < t\\r \in
    \rationals}} \lbrace \sigma < r ; \tau \circ \theta_\sigma < t -
r\rbrace \\
&=\cup_{\substack{0 < r < t\\r \in
    \rationals}} \lbrace \sigma  \wedge n < r ; \tau \circ
\theta_{\sigma \wedge n} < t - r\rbrace
\end{align*}
Since $\tau$ is weakly $\mathcal{F}$-optional we know that $\lbrace
\tau < t-r\rbrace \in \mathcal{F}_{t-r}$ hence $\theta_{\sigma \wedge n}^{-1}
\lbrace \tau < t-r \rbrace \in \mathcal{F}^+_{\sigma \wedge n + t -r}$ and
therefore using Lemma \ref{WeaklyOptionalCharacterization} applied to
the stopped $\sigma$-algebra $\mathcal{F}^+_{\sigma  \wedge n + t -r}$ we get
\begin{align*}
\lbrace \sigma \wedge n < r ; \tau \circ \theta_{\sigma \wedge n}< t - r\rbrace &=
\lbrace \sigma \wedge n +t -r < t \rbrace \cap \theta_{\sigma \wedge n}^{-1} \lbrace \tau
< t -r\rbrace \in \mathcal{F}_t
\end{align*}
and therefore $\gamma$ is weakly $\mathcal{F}$-optional.
\end{proof}

Note: Kallenberg's proof of the above Lemma is a little bit different and
from what I can tell has a small error.  He first
proves the result for $\sigma$ bounded, and then claims that
$\gamma_n = \sigma \wedge n + \tau \circ \theta_{\sigma \wedge n} \uparrow \gamma$ enabling us to apply the result for
the bounded case to
$\gamma_n$ and to conclude that $\gamma = \sup_n \gamma_n$ is weakly
$\mathcal{F}$-optional via Lemma
\ref{InfSupStoppedFiltration}.  The problem is that $\gamma_n$ as
defined is not increasing.  To see a counter example let $S = \lbrace
H,T \rbrace$ and consider the result for $S^\infty$ (here time is $\integers_+$).  Define 
\begin{align*}
\tau &= \min \lbrace n \mid n \text{ is even and } X_n = H \rbrace
\end{align*}
It is easy to see that $\tau$ is a stopping time as 
\begin{align*}
\lbrace \tau = n \rbrace  &= 
\begin{cases}
\lbrace X_0 =  H \rbrace & \text{for $n = 0$} \\
\lbrace X_n = H \rbrace \cap \lbrace X_{n-2} = T \rbrace \cap \dotsb
\cap \lbrace X_0 = T \rbrace & \text{if $n$ is even and $n > 0$} \\
\emptyset & \text{if $n$ is odd}
\end{cases} 
\end{align*}
Now let $\sigma$ be
a suitably large deterministic time (say $\sigma = 2$) so that for $n \leq
2$ we have $\gamma_n = n + \tau \circ \theta_n$.  Consider
$\omega = (T,H,H,H,\dotsc) \in S^\infty$.  Note that $\tau(\omega)
= 2$ thus $\gamma_0(\omega) = 2$ but $\tau(\theta_1(\omega)) = 0$ and
therefore $\gamma_1(\omega) = 1 < \gamma_0(\omega)$. 

It is worth noting that even when we are not considering the canonical
case many optional times of interest (in particular hitting times) are
pull backs of optional times on the path space (i.e. are of the form
$\tau \circ X$ where $\tau$ is an optional time defined on $S^T$).  If
we are given a pair of these optional times then we can
apply the time shift construction of the optional times on the path
space and pull back (i.e. forming $\sigma \circ X + \tau \circ
\theta_{\sigma \circ X} \circ X$).  The notation for the non-canonical
case is a bit ugly so sometimes we will simply use the notation $\sigma + \tau \circ
\theta_{\sigma} $ as a shorthand.

\begin{thm}[Strong Markov Property]\label{StrongMarkovPropertyMarkovProcessCountableValues}Let $X$ be a time homogeneous Markov process on
  $\integers_+$ or $\reals_+$ and let $\tau$ be an optional
  time with at most countably many values.  Then for every measurable
  $A \subset S^T$,
\begin{align*}
\cprobability{\mathcal{F}_\tau}{\theta_\tau X \in A}(\omega) &=
\sprobability{A}{X_\tau(\omega)} \text{ for almost all $\omega$ such that $\tau(\omega) < \infty$}
\end{align*}
\end{thm}
\begin{proof}
Before starting on the proof we first need to make some remarks about
the well-definedness of the quantities in the result.  Specifically we
have not defined $\theta_\tau X$ nor $\sprobability{A}{X_\tau}$ when $\tau = \infty$ but neither have
we assumed that $\tau$ is almost surely finite.  The first point is that we
can extend $\sprobability{A}{X_\tau}$ can be defined to be an
arbitrary value on $\lbrace \tau = \infty \rbrace$ without affecting
the values of $\sprobability{A}{X_\tau}$ on $\lbrace \tau < \infty
\rbrace$ hence the assertion of the result.  By locality of
conditional expectation (Lemma \ref{ConditionalExpectationIsLocal})
and the $\mathcal{F}_\tau$-measurability of
$\tau$ (Lemma
\ref{StoppedFiltration}) we can define $\theta_\tau X$ arbitrarily on $\lbrace \tau =
\infty \rbrace$ without affecting the values of
$\cprobability{\mathcal{F}_\tau}{\theta_\tau X \in A}$ on $\lbrace \tau < \infty
\rbrace$ hence the assertion of the result.  Therefore the result
makes sense assuming that such extensions have made and is independent
of the extensions chosen.

We first prove the result for deterministic times and extend to
countably valued optional times.  Note that the content of result is
vacuous for an infinite deterministic time, so pick a finite
deterministic time $t$, $t_1
\leq \cdots \leq t_n$, $B \in \mathcal{S}^{\otimes n}$, $A =
(\pi_{t_1}, \dotsc, \pi_{t_n})^{-1}(B)$ and
calculate using Lemma \ref{MarkovDistributions}, time homogeneity and
the proof of Lemma \ref{MarkovMixtures}
\begin{align*}
\cprobability{\mathcal{F}_t}{\theta_t X \in A} &=
\cprobability{\mathcal{F}_t}{((\theta_t X)_{t_1}, \dotsc,
  (\theta_t X)_{t_n}) \in B} \\
&= \cprobability{\mathcal{F}_t}{(X_{t  + t_1}, \dotsc, X_{t+ t_n}) \in B} \\
&= \mu_{t, t + t_1} \otimes \cdots \otimes \mu_{t+t_{n-1}, t+t_n}(X_t,B) \\
&=\mu_{0, t_1} \otimes \cdots \otimes \mu_{t_{n-1}, t_n}(X_t, B) \\
&= \sprobability {A}{X_t}
\end{align*}
Now we know that sets of the form 
$(\pi_{t_1}, \dotsc, \pi_{t_n})^{-1}(B)$ are a
generating $\pi$-system for the $\sigma$-algebra $\mathcal{S}^T$,
the full result for deterministic times $t$ follows from a monotone
class argument.  Specifically, we simply show that the set of $A$ such that
$\cprobability{\mathcal{F}_t}{\theta_t X \in A} =
\sprobability{A}{X_t}$ a.s. is a $\lambda$-system.  The case for $B
\setminus A$ follows from linearity of conditional expectation and
finite additivity of measure and the case $A_1 \subset A_2 \subset
\cdots$ follows from monotone convergence for conditional expectations
and continuity of measure.

Now we extend to the case of countably valued optional times.  Let $A
\in \mathcal{S}^T$ and $B \in \mathcal{F}_\tau$ and calculate using
Monotone Convergence and the result for deterministic times
\begin{align*}
\expectation{\characteristic{A}(\theta_\tau X) ; B} &= \sum_t
\expectation{\characteristic{A}(\theta_t X) ; \lbrace \tau = t \rbrace
  \cap B} \\
&= \sum_t \expectation{\sprobability{A}{X_t} ; \lbrace \tau = t \rbrace
  \cap B} \\
&= \expectation{\sprobability{A}{X_\tau} ; B} 
\end{align*}
so the result follows by the definition of conditional expectation. 

An alternative argument that extends the case of deterministic times
to countable optional times uses the localization of the stopped
filtration Lemma \ref{LocalizationOfStoppedFiltration} and the local
property of conditional expectations Lemma
\ref{ConditionalExpectationIsLocal}.  Let $t$ be a value in the range
of $\tau$, combining these two results and
using the result for deterministic times we
know that on the set $\lbrace \tau = t \rbrace$ we have
\begin{align*}
\cprobability{\mathcal{F}_{\tau}}{\theta_\tau X \in A} &=
\cprobability{\mathcal{F}_t}{\theta_t X \in A} = \sprobability{A}{X_t}
 = \sprobability{A}{X_\tau} \text{ a.s.}
\end{align*}
Let the set where the above inequality fails be called $N_t$.  Since
we have assumed the set of values of $\tau$ is countable, the union of
the $N_t$ is also a null set and the result holds off of this null
set.

TODO: What about the $\mathcal{F}_\tau$-measurability of
$\sprobability{A}{X_\tau}$?  Note that this is a consequence of result
since we haven't assumed $X$ is progressive(see Lemma
\ref{StrongIndependentIncrements} below where we make this implication explicit).  Double check that we
don't assume it in the proof above (note that Ethier and Kurtz do make the progressive assumption in their
discussion of the strong markov property).
\end{proof}

The key part of the above proof is the computation of finite dimensional distributions as a bridge to
lift the simple Markov property $\cprobability{\mathcal{F}_s}{X_{t+s} \in A} = \mu_{t}(X_s, A) = \sprobability{X_t \in A}{X_s}$ 
on one dimensional distributions to the full $\sigma$-algebra $\mathcal{S}^T$.  For an arbitrary optional time $\tau$ we have an analogous
argument using finite dimensional distributions to show the strong Markov property for the one dimensional case is sufficient to 
prove the full strong Markov property for that optional time.  This result will be used later in the text when we want to show that 
special classes of Markov processes have the strong Markov property.  Note that in this case we are now dealing with arbitrary 
optional times $\tau$ and therefore we must assume the process $X$ is progressive so that $X_\tau$ is $\mathcal{F}_\tau$-measurable 
(Lemma \ref{StoppedProgressivelyMeasurableProcess}).

\begin{prop}\label{StrongMarkovFromOneDimensionalDistribution}Let $X$ be a progressive time homogeneous Markov process on
  $\reals_+$ and let $\tau$ be a finite optional
  time.  If for every $s,t \geq 0$ and $B \in \mathcal{S}$ we have
\begin{align*}
\cprobability{\mathcal{F}_{\tau+s}}{X_{\tau + t + s} \in B} &= \mu_t(X_{\tau+s}, B) \text{ a.s.}
\end{align*}
then for every measurable
  $A \subset S^T$,
\begin{align*}
\cprobability{\mathcal{F}_\tau}{\theta_\tau X \in A} &= \sprobability{A}{X_\tau} \text{ a.s.}
\end{align*}
\end{prop}
\begin{proof}
The crux of the proof is to show the result for finite dimensional distributions.
\begin{clm}Let $n \in \naturals$ and $f_1, \dotsc, f_n$ be bounded measurable functions from $S$ to $\reals$ and
let $t_1 \leq \dotsc \leq t_n$ then 
\begin{align*}
\cexpectationlong{\mathcal{F}_\tau}{\prod_{i=1}^n f_i(X_{\tau + t_i})} 
&= \int \prod_{i=1}^n f_i(u_i) \, \mu_{t_1} \otimes \dotsb \otimes \mu_{t_n - t_{n-1}}(X_\tau, du_1, \dotsc, du_n)
\end{align*}
\end{clm}
The proof is by induction on $n$.  For $n=1$ we have by hypothesis 
\begin{align*}
\cprobability{\mathcal{F}_\tau}{X_{\tau + t} \in \cdot} &= \mu_t(X_\tau, \cdot)
\end{align*} 
and therefore
by Theorem \ref{Disintegration} we have 
\begin{align*}
\cexpectationlong{\mathcal{F}_\tau}{f(X_{\tau + t})} &= \int f(u) \, \mu_t(X_\tau, du)
\end{align*}
which is the claim for $n=1$.  For the induction step, assume the result for all 
$m \in \naturals$ such that $1 \leq m \leq n$ and let $f_1, \dotsc, f_{n+1}$ and $t_1 \leq \dotsb \leq t_{n+1}$ be given then we apply the tower and pullout properties of conditional expectation, the induction 
hypothesis for cases $n=1$ and $n$ and then the definition of kernel products to see
\begin{align*}
&\cexpectationlong{\mathcal{F}_\tau}{\prod_{i=1}^{n+1} f_i(X_{\tau + t_i})}  \\
&=\cexpectationlong{\mathcal{F}_\tau}{\cexpectationlong{\mathcal{F}_{\tau+t_n}}{\prod_{i=1}^{n+1} f_i(X_{\tau + t_i})} } \\
&=\cexpectationlong{\mathcal{F}_\tau}{\prod_{i=1}^n f_i(X_{\tau + t_i}) \cexpectationlong{\mathcal{F}_{\tau+t_n}}{f_{n+1}(X_{\tau + t_{n+1}})} } \\
&=\cexpectationlong{\mathcal{F}_\tau}{\prod_{i=1}^n f_i(X_{\tau + t_i}) \int f_{n+1}(u_{n+1}) \, \mu_{t_{n+1} - t_n}(X_{\tau + t_n}, du_{n+1})} \\
&=\int \prod_{i=1}^n f_i(u_i) \left[\int f_{n+1}(u_{n+1}) \, \mu_{t_{n+1} - t_n}(u_n, du_{n+1}) \right] \, \mu_{t_1} \otimes \dotsb \otimes \mu_{t_n - t_{n-1}}(X_\tau, du_1, \dotsc, du_n)\\
&=\int \prod_{i=1}^{n+1} f_i(u_i) \mu_{t_1} \otimes \dotsb \otimes \mu_{t_{n+1} - t_{n}}(X_\tau, du_1, \dotsc, du_{n+1})\\
\end{align*}
and the claim is proved.

From the claim and the proof of Lemma \ref{MarkovMixtures} we see that for arbitrary $B_1, \dotsc B_n \in \mathcal{S}$ and $t_1 \leq \dotsb \leq t_n$ if we
define $A = (\pi_{t_1}, \dotsc, \pi_{t_n})^{-1}(B_1 \times \dotsb \times B_n)$ (where $\pi_t : S^T \to S$ is the evaluation map $\pi_t f = f(t)$) we have
\begin{align*}
&\cprobability{\mathcal{F}_\tau}{\theta_\tau X \in A} \\
&=\cprobability{\mathcal{F}_\tau}{(X_{\tau+t_1}, \dotsc, X_{\tau+t_n}) \in B_1 \times \dotsb \times B_n} \\
&=\mu_{t_1} \otimes \dotsb \otimes \mu_{t_n - t_{n-1}}(X_\tau, B_1 \times \dotsb \times B_n) \\
&= \sprobability{A}{X_\tau}
\end{align*}
The set of such $A$ is clearly a $\pi$-system and generates $\mathcal{S}^T$ (the latter being generated by the one dimensional $\pi_t^{-1}(B)$ in fact).  By montone classes as in Theorem \ref{StrongMarkovPropertyMarkovProcessCountableValues} we get the result for all $\mathcal{S}^T$.
\end{proof}

The previous result that shows how to establish the Strong Markov property for finite optional times is in fact sufficient to establish the Strong Markov property for all optional times by virtue of the following argument.
\begin{prop}\label{StrongMarkovFromStrongMarkovFiniteOptional}Let $X$ be a progressive time homogeneous Markov process on
  $\reals_+$ and suppose that for all finite optional
  times $\tau$ and all measurable sets
  $A \subset S^T$ we have
\begin{align*}
\cprobability{\mathcal{F}_\tau}{\theta_\tau X \in A} &= \sprobability{A}{X_\tau} \text{ a.s.}
\end{align*}
then for all optional times $\tau$ and measurable sets
  $A \subset S^T$ we have
\begin{align*}
\cprobability{\mathcal{F}_\tau}{\theta_\tau X \in A} &= \sprobability{A}{X_\tau} \text{ a.s. on $\tau < \infty$}
\end{align*}
\end{prop}
\begin{proof}
The fact that the terms in the conclusion of the result are in fact well defined see the discussion in Theorem \ref{StrongMarkovPropertyMarkovProcessCountableValues}.  Given an arbitrary optional time $\tau$, let $n \in \naturals$ and note that $\mathcal{F}_\tau = \mathcal{F}_{\tau \wedge n}$, $\theta_\tau X = \theta_{\tau \wedge n}$ and $\sprobability{A}{X_\tau}=\sprobability{A}{X_{\tau \wedge n}}$ on $\lbrace \tau \leq n \rbrace = \lbrace \tau = \tau \wedge n \rbrace$ (see Proposition \ref{StoppedAlgebraMinOfOptionalTimes} for the first assertion).  Observe that for $t \geq n$, 
\begin{align*}
\lbrace \tau \leq n \rbrace \cap \lbrace \tau \wedge n \leq t \rbrace &= \lbrace \tau \leq n \rbrace \in \mathcal{F}_n
\end{align*} 
and for $t < n$
\begin{align*}
\lbrace \tau \leq n \rbrace \cap \lbrace \tau \wedge n \leq t \rbrace &= \lbrace \tau \leq t \rbrace \in \mathcal{F}_t \subset \mathcal{F}_n
\end{align*} 
Thus $\lbrace \tau \leq n \rbrace  \in \mathcal{F}_\tau \cap \mathcal{F}_{\tau \wedge n} = \mathcal{F}_{\tau \wedge n}$ and we may apply  localization of conditional expectations Lemma \ref{ConditionalExpectationIsLocal} to see that on $\lbrace \tau \leq n \rbrace$,
\begin{align*}
\cprobability{\mathcal{F}_\tau}{\theta_\tau X \in A} &= \cprobability{\mathcal{F}_{\tau \wedge n}}{\theta_{\tau  \wedge n} X \in A} = \sprobability{A}{X_{\tau \wedge n}}  = \sprobability{A}{X_\tau}  \text{ a.s.}
\end{align*}
Now write $\lbrace \tau < \infty \rbrace = \cup_{n=1}^\infty \lbrace \tau \leq n \rbrace$ and the the union of a countable number of null sets.
\end{proof}

In the case of a space homogeneous Markov process the strong Markov
property can be expressed more concisely as an extension of the
independent increments characterization of Lemma
\ref{IndependentIncrements} to optional times.  In many scenarios it
is more convenient to use these properties.  Note that the Lemma does
not require the countable range assumption.
\begin{lem}\label{StrongIndependentIncrements}Let $S$ be a measurable
  Abelian group with a filtration $\mathcal{F}$, $X$ be a time
  homogeneous and space homogeneous $S$-
  valued Markov process and $\tau$ be an almost surely finite optional time.  
Then 
\begin{align*}
\cprobability{\mathcal{F}_\tau}{\theta_\tau X \in A} &= \sprobability{A}{X_\tau}
\end{align*}
if and only if $X_\tau$ is $\mathcal{F}_\tau$-measurable, $\cindependent{\theta_\tau X -
  X_\tau}{\mathcal{F}_\tau}{}$ and $X - X_0 \eqdist \theta_\tau X - X_\tau$
\end{lem}
\begin{proof}
Assume that $X$ satisfies $\cprobability{\mathcal{F}_\tau}{\theta_\tau
  X \in A} = \sprobability{A}{X_\tau}$ for all $A \in \mathcal{S}^T$.  To see that $X_\tau$ is
$\mathcal{F}_\tau$-measurable observe that if we let $\pi_0 : S^T \to
S$ be
evaluation at time $0$, then for any $B \in \mathcal{S}$ and $x \in S$,
\begin{align*}
\sprobability{\pi_0^{-1}B}{x} &= \begin{cases}
1 & \text{if $x \in B$} \\
0 & \text{if $x \notin B$}
\end{cases}
\end{align*}
therefore we have
\begin{align*}
\characteristic{X_\tau \in B} &=
\sprobability{\pi_0^{-1}B}{X_\tau} =
\cprobability{\mathcal{F}_\tau}{\theta_\tau X \in \pi_0^{-1} B}
\end{align*}
which shows that $\lbrace X_\tau \in B \rbrace \in \mathcal{F}_\tau$.

Having established $\mathcal{F}_\tau$-measurability of $X_\tau$ we
know that $P_{X_\tau}$ is a not just a \emph{regular} version for
$\cprobability{\mathcal{F}_\tau}{\theta_\tau X \in \cdot}$ and we can
apply Theorem \ref{Disintegration} and space homogeneity of
$\probabilityop_x$ (Lemma \ref{SpaceHomogeneousMarkovDistributions})
to calculate for $A \in \mathcal{S}^T$ (using $f: S^T \times S \to
\reals_+$ given by $f(x,y) = \characteristic{A+y}(x)$ in the disintegration) 
\begin{align*}
\cprobability{\mathcal{F}_\tau}{\theta_\tau X - X_\tau \in A} &= 
\int \characteristic{A + X_\tau} (x) \,
\probabilityop_{X_\tau}(dx) 
=\sprobability{A + X_\tau}{X_\tau}
=\sprobability{A}{0} \text{ a.s.}
\end{align*}
which is almost surely constant and therefore independence is proven.
This also shows that the distribution of $\theta_\tau X - X_\tau$ is
equal to $\probabilityop_0$ and letting $\tau = 0$ shows $\theta_\tau
X - X_\tau \eqdist X - X_0$.

To prove the converse, note that $X - X_0$ is has
initial distribution $\delta_0$ hence using our independence and
equidistribution assumptions and the definition of the measure
$\probabilityop_0$ we get for any $A \in \mathcal{S}^T$,
\begin{align*}
\cprobability{\mathcal{F}_\tau}{\theta_\tau X - X_\tau \in A} &=
\probability{\theta_\tau X - X_\tau \in A} 
=\probability{X - X_0 \in A} 
=\sprobability{A}{0}
\end{align*}
which provides us with a regular version for
$\cprobability{\mathcal{F}_\tau}{\theta_\tau X - X_\tau \in \cdot}$.
Now by the $\mathcal{F}_\tau$-measurability of $X_\tau$ we can apply
Theorem \ref{Disintegration} and Lemma
\ref{SpaceHomogeneousMarkovDistributions} to get
\begin{align*}
\cprobability{\mathcal{F}_\tau}{\theta_\tau X \in A}
&=\cprobability{\mathcal{F}_\tau}{\theta_\tau X - X_\tau \in A-X_\tau}\\
&= \int \characteristic{A-X_\tau}(x) \, \probabilityop_0(dx) \\
&=\sprobability{0}{A - X_\tau} \\
&=\sprobability{X_\tau}{A }
\end{align*}
and we are done.
\end{proof}

\begin{defn}Let $X$ be a time homogeneous Markov process with
  transition kernel $\mu_t$ we say an initial distribution $\nu$ is
  \emph{invariant} if $\nu \mu_t = \nu$ for all $t \in T$, i.e. we
  have
\begin{align*}
\int \mu_t(x, A) \, \nu(dx) = \nu(A)
\end{align*}
for all $t \in T$ and $A \in \mathcal{S}$.
\end{defn}

\begin{defn}Let $X$ be a stochastic process with time scale $T$ then
  we say $X$ is \emph{stationary} if $\theta_t X \eqdist X$
  for all $t \in T$.
\end{defn}


\begin{lem}\label{InvarianceImpliesStationary}Let $X$ be a time
  homogeneous Markov process with transition kernel $\mu$ and an invariant initial distribution
  $\nu$, then $X$ is stationary.
\end{lem}
\begin{proof}
Fix $t \in T$, $s_1 < \dotsb < s_n$ and $A \in \mathcal{S}^{\otimes
  n}$, then using Lemma
\ref{MarkovDistributions} and time homogeneity we compute
\begin{align*}
\probability{(X_{t + s_1}, \dotsc, X_{t + s_n}) \in A} &= \nu_{t+s_1}
                                                         \otimes
                                                         \mu_{s_2 -
                                                         s_1} \otimes
                                                         \dotsm
                                                         \otimes
                                                         \mu_{s_n -
                                                         s_{n-1}}(A)
  \\
&=\nu_{s_1}
                                                         \otimes
                                                         \mu_{s_2 -
                                                         s_1} \otimes
                                                         \dotsm
                                                         \otimes
                                                         \mu_{s_n -
                                                         s_{n-1}}(A)
= \probability{(X_{ s_1}, \dotsc, X_{s_n}) \in A}
\end{align*}
Since the finite dimensional distributions characterize the
distribution of $\theta_t X$ (Lemma \ref{ProcessLawsAndFDDs}) it follows that $X$ is stationary.
\end{proof}


\section{Discrete Time Markov Chains}

In this section we discuss the theory of Markov processes on a time
scale $\integers_+$.  This part of Markov process theory has many
applications and we'll be able to construct lots of important examples
that both illustrate and motivate the accompanying theory.  Moreover
much of the theory in discrete time illustrates concerns that are also
present in more general cases but with fewer technical distractions.

One of our first concerns is to think about constructing examples of
Markov processes.  The obvious way to approach this is the way we have
done it up until now: specify a transition kernel and an initial
distribution.  As it turns out, it can be surprisingly difficult to
get a handle on the transition kernel of a concrete process and it is
desirable to have alternative ways of constructing and characterizing
Markov processes.  In the discrete time case we can think of a Markov
process as a deterministic system that is perturbed by noise (or
alternatively a  ``transduced'' noise sequence).  We make this precise
in the following theorem.
\begin{thm}\label{RandomMappingRepresentationExistence}Let $X$ be a
  process on time scale $\integers_+$ with a Borel state space
  $S$, then $X$ is Markov if and only if there exist a measurable
  space $(T, \mathcal{T})$, measurable
  functions $f_1, f_2, \dotsc : S \times T \to S$ and
  i.i.d. random elements $\vartheta_1, \vartheta_2, \dotsc
  \Independent X_0$ such that $X_n = f_n(X_{n-1}, \vartheta_n)$
  a.s. for all $n \in \naturals$.   If $X$ is Markov we may find such
  a representation with $T=[0,1]$ and $\vartheta_n$ i.i.d. $U(0,1)$
  random variables.  We may choose $f_1 = f_2 = \dotsb$
  if and only if $X$ is time homogeneous. 
\end{thm}
\begin{proof}
First assume that $X$ has the hypothesized representation.  Let
$\mathcal{F}$ be the filtration generated by $X$.  Let $\nu$ be the
law of $\vartheta_1, \vartheta_2, \dotsc$.  Pick a
random element $\vartheta$ with law $\nu$ and for $A \in
\mathcal{S}$ define $\mu_n(x, A) = \probability{f_n(x, \vartheta) \in
  A}$.  Note that it follows from a simple induction using
$(\vartheta_1, \vartheta_2, \dotsc) \Independent X_0$ and the
expression $X_n = f_n(X_{n-1}, \vartheta_n)$ that $\vartheta_n$ is
independent of $X_m$ for all $m = 0, \dotsc, n-1$.  Therefore $\cprobability{\vartheta_n \in \cdot}
{\mathcal{F}_{n-1}} = \probability{\vartheta \in \cdot} = \nu$ and in particular has a regular
version. Furthermore since $X_{n-1}$ is $\mathcal{F}_{n-1}$-measurable
we can apply Lemma \ref{DisintegrationIndependentLaws} to compute for any $A \in \mathcal{S}$
\begin{align*}
\cprobability{X_n \in A}{\mathcal{F}_{n-1}} &=
\cprobability{f_n(X_{n-1}, \vartheta_n) \in A}{\mathcal{F}_{n-1}} \\
&=\int \characteristic{f_n (X_{n-1}, s)\in A} \, \nu(ds) =
\probability{f_n (X_{n-1}, \vartheta)\in A} = \mu_n(X_{n-1}, A)
\end{align*}
which shows that $X$ is Markov with transition kernel $\mu_n$ (recall
in discrete time the Chapman Kolmogorov relations hold identically for
free).  Note also that $f_1 = f_2 = \dotsc$ if and only if $\mu_1 =
\mu_2 = \dotsc$ which is to say that $X$ is time homogeneous.

Now let $X$ be Markov.  Since $S$ is Borel we may apply Lemma \ref{RandomizationAndKernels}
to each transition kernel $\mu_n$ and construct a measurable function
$f_n : S \times [0,1] \to S$ such that for a $U(0,1)$ random variable
$\vartheta$ we know that $\probability{f_n(s, \vartheta) \in \cdot} = \mu_n(s,
\cdot)$.  Let $\tilde{X}_0$ be a random element such that $\tilde{X}_0
\eqdist X_0$ (e.g. just take the identity on $(S, \mathcal{S})$
provided with the probability measure $\mathcal{L}(X_0)$).  By
extending the probability space of $\tilde{X}_0$ if necessary we
can assume the existence of i.i.d. $U(0,1)$ random variables
$\tilde{\vartheta}_1, \tilde{\vartheta}_2, \dotsc$.  Recursively
define $\tilde{X}_n = f_n(\tilde{X}_{n-1}, \tilde{\vartheta}_n)$ for
$n \in \naturals$ and apply the first part of this theorem to conclude
that $\tilde{X}$ is a Markov process with transition kernels $\mu_n$
and initial distribution $\mathcal{L}(\tilde{X}_0) =
\mathcal{L}(X_0)$.  We now apply Lemma \ref{MarkovDistributions} to
conclude that the of $X \eqfdd \tilde{X}$ and thus $X
\eqdist \tilde{X}$ by Lemma \ref{ProcessLawsAndFDDs}.  Now since
$[0,1]^\infty$ is a Borel space we may
apply Lemma \ref{Transfer} to conclude there are
random variables $\vartheta_1, \vartheta_2, \dotsc$ such that $(X,
(\vartheta_1, \vartheta_2, \dotsc)) \eqdist (\tilde{X},
(\tilde{\vartheta}_1, \tilde{\vartheta}_2, \dotsc))$.  By considering
marginal distributions we conclude that 
$\vartheta_1, \vartheta_2, \dotsc$ are i.i.d. $U(0,1)$ and that
$(\vartheta_1, \vartheta_2, \dotsc) \Independent X_0$. Also using $(X,
(\vartheta_1, \vartheta_2, \dotsc)) \eqdist (\tilde{X},
(\tilde{\vartheta}_1, \tilde{\vartheta}_2, \dotsc))$, the
measurability of the diagonal $\Delta \subset S \times S$ and the
definition of $\tilde{X}$ we conclude
that for each $n \in \naturals$ 
\begin{align*}
\probability{X_n = f_n(X_{n-1}, \vartheta_n)} &= \probability{(X_n,
  f_n(X_{n-1}, \vartheta_n)) \in \Delta} \\
&= \probability{(\tilde{X}_n,
  f_n(\tilde{X}_{n-1}, \tilde{\vartheta}_n)) \in \Delta} = 1
\end{align*}
and we are done.
\end{proof}

The representation of a Markov process as in the preceeding theorem is
refered to as a \emph{random mapping representation} and we'll soon
put it use in constructing examples of Markov processes.

We proceed to study the special subclass of time homogenous Markov processes with time
scale $\integers_+$.  A further important specialization occurs when
the state space $S$ is countable or finite.  We establish
some terminology while recording the definitions.
\begin{defn}A time homogeneous Markov process $X$ with time scale
  $\integers_+$, transition kernels $\mu_n$ and state space $S$ is called a
  \emph{discrete time Markov process}.  Furthermore,
\begin{itemize}
\item[(i)]If  $S$ countable then $X$ is a
  \emph{discrete time Markov chain} 
\item[(ii)]If  $S$ is finite then $X$ is a \emph{finite discrete time
    Markov chain}
\item[(iii)]For each $y \in S$ we let $\tau^+_y
  = \inf \lbrace n \in \naturals \mid X_n = y \rbrace$ and then
  recursively define the \emph{return times}
\begin{align*}
\tau^0_y &= 0 \\
\tau^{k+1}_y &= \tau^k_y + \tau^+_y \circ \theta_{\tau^k_y} \text{ for
  $k \in \integers_+$}
\end{align*}
\item[(iv)]For each $y \in S$ we define the \emph{occupation time} at $y \in S$ as
\begin{align*}
\kappa_y &= \sup \lbrace k \in \integers_+ \mid \tau^k_y < \infty \rbrace
\end{align*}
\item[(v)]For each $x,y \in S$ we define the \emph{hitting probability} 
\begin{align*} 
r_{xy} &= \sprobability{\tau^+_y < \infty}{x} = \sprobability{\kappa_y >
0}{x} 
\end{align*}
\item[(vi)]For each $x,y \in S$ and we define the \emph{transition probabilities}
\begin{align*}
p_{xy} &= \mu_1(x, \lbrace y \rbrace) &
p^n_{xy} &= \mu_n(x, \lbrace y \rbrace) \text{ for $n \in \naturals$}
\end{align*}
\end{itemize}
\end{defn}
For the case of a discrete time Markov chain, the transition
probabilities $p_{xy}$ characterize the transition kernels and recall
from Example \ref{ProbabilityKernelProductFiniteSampleSpace} it is
convenient to interpret the $p_{xy}$ as being the entries of a
\emph{transition matrix}
we shall call $p$.  Moreover
from Example \ref{ProbabilityKernelProductFiniteSampleSpace} and the
Chapman Kolmogorov relations we have 
\begin{align*}
\mu_n(x, \lbrace y \rbrace) &= \mu_1^n(x, \lbrace y \rbrace) =
p_{xy}^n
\end{align*}
where in the last equality we are taking the $(x,y)^{th}$ entry of the
$n$-fold product of the matrix $p$.  This explains the use of the
notation in the above definition of transition probabilites and also shows that for Markov chains
the notation is consistent with transition matrix point of view.  We
emphasize that $p^n_{xy}$ does not signify $p_{xy}$ raised to the
$n^{th}$ power!

We have two initial goals in our study of Markov chains.  The first is
to develop a little macroscopic structure theory of Markov chains.

\begin{prop}\label{RecurrenceTransienceDiscreteTime}Let $X$ be a discrete time Markov process on state space
  $S$.  Then for $y \in S$ we have
\begin{align*}
\kappa_y &= \sum_{n=1}^\infty \characteristic{X_n = y}
\end{align*}
Moreover for all $x,y \in S$ and $n \in \naturals$,
\begin{align*}
\sprobability{\kappa_y \geq n}{x} &= \sprobability{\tau^n_y <
  \infty}{x} = r_{xy} r_{yy}^{n-1} 
\end{align*}
If $r_{xy} = 0$ then $\kappa_y = 0$  $P_x$-almost surely, if $r_{xy} > 0$ and
$r_{yy} = 1$ then $\sprobability{\kappa_y = \infty}{x} = r_{xy} > 0$, otherwise
$\kappa_y$ is integrable with expectation
\begin{align*}
\sexpectation{\kappa_y}{x} &= \frac{r_{xx}}{1 - r_{yy}} =
\sum_{n=1}^\infty p^n_{xy}
\end{align*}
\end{prop}
\begin{proof}
First to see that $\kappa_y = \sum_{n=1}^\infty \characteristic{X_n =
  y}$, simply note that both represent the number of times that $X$
visits $y$.
  
To see that $\sprobability{\kappa_y \geq n}{x} = \sprobability{\tau^n_y <
  \infty}{x}$ simply note that equality holds at the level of events: $\lbrace \kappa_y \geq n \rbrace$ if
and only if $\lbrace \tau^n_y <  \infty \rbrace$.  Since $\tau^{n+1}_y
< \infty$ if and only if $\tau^n_y < \infty$ and $\theta_{\tau^n_y}
\circ \tau^+_y < \infty$ we can use the Strong Markov property to
calculate
\begin{align*}
\sprobability{\tau^{n+1}_y < \infty}{x} &= \sprobability{\tau^{n}_y
  < \infty ; \theta_{\tau^n_y} \circ \tau^+_y < \infty } {x} \\
&=\sexpectation{\tau^{n}_y
  < \infty ; \cprobability{\mathcal{F}_{\tau^n_y}}{\theta_{\tau^n_y} \circ \tau^+_y < \infty}
} {x} \\
&=\sexpectation{\tau^{n}_y
  < \infty ; \sprobability{\tau^+_y < \infty}{y}} {x} = \sprobability{\tau^{n}_y
  < \infty} {x} \sprobability{\tau^+_y < \infty}{y}
\end{align*}
which we use in an induction argument to get $\sprobability{\tau^{n}_y <
  \infty}{x} =  r_{xy} r_{yy}^{n-1}$.

Now we apply this fact along with Lemma \ref{TailsAndExpectations} to
see that
\begin{align*}
\sexpectation{\kappa_y}{x} &= \sum_{n=1}^\infty \sprobability{\kappa_y
  \geq n}{x} = r_{xy} \sum_{n=1}^\infty r_{yy}^{n-1} = \frac{r_{xy}}{1-r_{yy}}
\end{align*}
The rest of the statements in the proposition are trivial consequences
of what we have proven.
\end{proof}
By virtue of this result we can see that for every $x \in S$ there is
a dichotomy: either we have $r_{xx} = 1$ in which case $\kappa_x =
\infty$ $P_x$-a.s. (almost surely $X$ returns to $x$ infinitely many
times) or $0 \leq r_{xx} < 1$ in which case the number of times that
$X$ returns to $x$ has finite expectation $\frac{r_{xx}}{1-r_{xx}}$.
This attribute of states is worthy of a definition.
\begin{defn}Let $X$ be a discrete time Markov process on state space
  $S$, we say a state $x \in S$ is \emph{recurrent} if and only if $X$
  returns to $x$ infinitely many times $P_x$-a.s.  We say $x \in S$ is
  \emph{transient} if and only if $X$  returns to $x$ only finitely many times $P_x$-a.s.
\end{defn}

The theory of Markov processes tends to be concerned with long term
behavior of the process and therefore recurrent states are more
important than transient states (just wait long enough and you'll
never see a transient state again!)  Being able to detect recurrent
states is therefore a useful thing to be able to do.  A simple and
useful criterion can be found when there is an invariant distribution
for $X$.

\begin{prop}\label{RecurrenceFromInvariantDistribution}Let $X$ be a discrete time Markov process with state space
  $S$ and assume that an invariant distribution $\nu$ exists, then for
  every $x \in S$ if $\nu(x) > 0$ it follows that $x$ is recurrent.
\end{prop}
\begin{proof}
Using the invariance of $\nu$ we get for every $n \in \naturals$
\begin{align*}
0 &< \nu(x) = \int p^n_{xy} \, \nu(dy)
\end{align*}
Therefore using the fact that $r_{yx} \leq 1$ for all $x,y \in S$,
Proposition \ref{RecurrenceTransienceDiscreteTime} and Tonelli's Theorem \ref{Fubini} we get
\begin{align*}
\frac{1}{1 - r_{xx}} &\geq \int \frac{r_{yx}}{1-r_{xx}} \, \nu(dy) =
\int \sum_{n=1}^\infty p^n_{yx} \, \nu(dy) = \sum_{n=1}^\infty\int
p^n_{yx} \, \nu(dy) = \infty
\end{align*}
and thus it follows that $r_{xx} = 1$.
\end{proof}

\begin{defn}Let $p^n_{xy}$ be the transition probabilities of a
  discrete time Markov process on $S$.  The \emph{period} of a state
  $x \in S$ is 
\begin{align*}
d_x &= \gcd \lbrace n \in \naturals \mid p^n_{xx} > 0 \rbrace
\end{align*}
If $d_x = 1$ then we say that the state $x$ is \emph{aperiodic}.
\end{defn}

\begin{prop}\label{PeriodicReturnDiscreteTimeMarkov}Let $p^n_{xy}$ be the transition probabilities of a
  discrete time Markov process on $S$, if $x$ has period $d$
  then there exists an $N > 0$ such that $p^{nd}_{xx} > 0$ for all $n
  \geq N$.
\end{prop}
\begin{proof}
We need the following number theoretic fact:

\begin{lem}Let $A \subset \integers_+$ then there exists an integer $m_A$
such that for all $m \geq m_S$ there exist constants $c_1, \dotsc ,
c_n \in \integers_+$ and $x_1, \dotsc, x_n \in A$ such that $m \gcd A = c_1 x_1 + \dotsb +
c_n x_n$.
\end{lem}
\begin{proof}
To prove the lemma we first recall that the greatest common divisor of
a set is an integer linear combination of elements of the set.  

Claim: For any subset $B \subset \integers_+$ there exist elements
$x_1, \dotsc, x_n \in B$ and constants $c_1, \dotsc, c_n \in \integers$
such that $\gcd B = c_1 x_1 + \dotsb + c_n x_n$.

To see this, let $g^*_B$ be
smallest element in the set 
\begin{align*}
C &= \lbrace c_1 x_1 + \dotsb +
c_n x_n > 0 \mid n \in \naturals, c_1, \dotsc, c_n \in \integers \text{ and } x_1,
\dotsc, x_n \in B \rbrace
\end{align*}
  Note that
$g^*_B$ divides every $x \in B$; for if not then there is an $x$ such that
we can write $x = c
g^*_B + r$ with $c \in \integers_+$ and $0 < r < g^*_S$ thus $r = x
- g_B^* \in C$.  Therefore it follows that $\gcd B$ divides $g^*_B$.
On the other hand, since $g^*_B$ is an integer linear combination of a
finite number of
elements of $B$ it follows that $\gcd B$ divides $g^*_B$ and
therefore $\gcd B = g^*_B$.

Claim: For any set $B \subset
\integers_+$ there is a finite subset $F \subset B$ such that $\gcd F
= \gcd B$.  

To see this consider the sequence $g_n = \gcd B \cap \lbrace 0,
\dots, n \rbrace$.  Clearly, $g_n$ is non-increasing and non-negative
so there exists an $N > 0$ such that $g_n = g_N$ for all $n \geq N$.
It is also clear that $g_N$ divides every element of $B$ since every
element of $B$ is in some $B \cap \lbrace 0,
\dots, n \rbrace$ and it follows by a similar argument that $\gcd S
\leq g_N$. Thus $\gcd S = g_N$.

From the previous claim note that it suffices to prove the lemma for
finite sets $A$.  To prove the lemma for finite sets we proceed by
induction on the cardinality of $A$.

The result is vacuous for singleton sets so let $A = \lbrace a, b \rbrace$ and let $g = \gcd A$.  For
every $m \in \naturals$ we can write $m g = ca + db$ for some $c,d \in
\integers$.  By replacing $c$ and $d$ by $c + kb$ and $d - ka$ for
suitable $k \in \integers$ we may assume that $0 \leq c < b$ as well.
Thus in this case, define $m_A = (ab - a -b)/g +1$ and note that for
any $m \geq m_A$ we have
\begin{align*}
m g &=ca + db  \geq (ab - a - b) + g >  ab - a - b
\end{align*}
with $0 \leq c < b$ which implies
\begin{align*}
(d+1)b  &> ab - a - ca \geq 0
\end{align*}
which in turn implies $d \geq 0$.  Thus the result is proven for a two
point set.

Now we do induction on the cardinality of $A$.  Suppose the result is
proven for all $A$ with cardinality less than or equal to $n$.  Let $A$ be a finite
subset of $\integers_+$ with $A = \lbrace a_1, \dotsc, a_n \rbrace$ and $\gcd A = g_A$.  Let $a \in \integers_+
\setminus A$ and note the facts that $\gcd (A \cup \lbrace a \rbrace ) =
\gcd(\gcd A, a)$ and $\gcd(A \cup \lbrace a \rbrace )$ divides $\gcd
A$.  Define $g =  \gcd( A \cup \lbrace a \rbrace)$ and 
\begin{align*}
m_{A \cup  \lbrace a \rbrace} &= (m_{\lbrace a,  g_A \rbrace} g + m_A g_A)/g
\end{align*}
and pick any $m \geq m_{A \cup \lbrace a \rbrace}$ :  trivially we have
$m g \geq m_{\lbrace a,  g_A \rbrace} g + m_A g_A$.  It follows from the
fact that $g = \gcd(\gcd A, a)$ that $g$ divides $g_A$ and therefore
there is a $\tilde{m} \in \integers_+$ such that $mg - m_A g_A = \tilde{m} g \geq m_{\lbrace a,  g_A
  \rbrace} g$.
By the definition of $m_{\lbrace a,  g_A \rbrace}$ we know that
there are integers $c,d \geq 0$ such that $mg - m_A g_A = c a + d
g_A$.  Therefore
\begin{align*}
mg &= ca + (d + m_A) g_A = ca + \sum_{j=1}^n c_j a_j
\end{align*}
with suitable $c_1, \dotsc c_n \in \integers_+$ and the lemma is
proved.
\end{proof}

Now to prove the proposition, let $x \in S$,  let $A = \lbrace n \in \naturals \mid
p^n_{xx} > 0 \rbrace$ assume that $\gcd A = d$.  Applying the lemma we
see that there is an $N > 0$ such that for all $n \geq N$, $nd = c_1 n_1 + \dotsb + c_k
n_k$ for suitable $k \in \naturals$, $n_1, \dotsc, n_k \in A$ and $c_1, \dotsc, c_k \in
\integers_+$.  On the other hand suppose $n, m \in A$ and note that by
the Chapman Kolmogorov relations we have 
\begin{align*}
p^{n+m}_{xx} &= \mu_{n+m}(x, \lbrace x \rbrace) = \mu_{n}\mu_m(x,
\lbrace x \rbrace) = \int \mu_m(y , \lbrace x \rbrace) \, \mu_n(x, dy)
\\
&\geq \int \mu_m(y, \lbrace x \rbrace) \characteristic{x=y} \,
\mu_n(x, dy) 
= \mu_m(x, \lbrace x \rbrace) \mu_n(x, \lbrace x \rbrace) 
> 0
\end{align*}
which shows that $A$ is closed under addition.  It follows that $nd
\in A$ and the result is proven.
\end{proof}

\begin{defn}Let $X$ be a discrete time Markov process with initial
  distribution $\nu$ and transition kernel $\mu$.  We say that $X$ is
  \emph{reversible} if for every non-negative measurable or integrable $f : S \times
  S \to \reals$ we have
\begin{align*}
\int f(x,y) \, (\nu \otimes \mu)(dx,dy) 
&= \iint f(x,y) \, \mu(x, dy) \, \nu(dx) \\
&= \iint f(y,x) \, \mu(x, dy) \, \nu(dx)
= \int f(y,x) \, (\nu \otimes \mu)(dx,dy) 
\end{align*}
\end{defn}

There are a couple of immediate consequences of reversibility that
follow by looking at the finite dimensional distributions of a
reversible $X$.  The first very useful implication is that
reversibility implies stationarity.

\begin{prop}\label{ReversibleImpliesInvariant}Let $X$ be a reversible discrete time Markov process with initial
  distribution $\nu$ and transition kernel $\mu$, then $\nu$ is
  invariant for $X$.
\end{prop}
\begin{proof}
Let $A \in \mathcal{S}$ then using Lemma \ref{MarkovDistributions} and reversibility
\begin{align*}
\pushforward{X}{\probabilityop_{\nu}}
&= \nu \mu (A) 
= (\nu \otimes \mu)(S \times A) 
=  \iint \characteristic{A}(y) \, \mu(x, dy) \, \nu(dx) \\
&=  \iint \characteristic{A}(x) \, \mu(x, dy) \, \nu(dx) 
= \int \characteristic{A}(x) \, \nu(dx) 
= \nu(A)
\end{align*}
\end{proof}

The next implication explains the origin of the term reversible.
Prosaically one says that a reversible Markov process looks the same
if run backwards.
\begin{prop}Let $X$ be a reversible discrete time Markov process then
  for all $n,k \geq 0$ and $A \in \mathcal{S}^{\otimes n}$ 
\begin{align*}
\probability{(X_k, \dotsc, X_{n+k}) \in A} &= 
\probability{(X_{n+k}, \dotsc, X_k) \in A} 
\end{align*}
\end{prop}
\begin{proof}
Because $\nu$ is invariant, it follows that $X$ is a stationary
process (Lemma \ref{InvarianceImpliesStationary}) and therefore it
suffices to prove the result for $k=0$.
In fact we prove a bit more; we show that 
\begin{align}
\int f(x_0, \dotsc, x_n)
\nu \otimes \mu^{\otimes n}(dx_0, \dotsc, dx_n) &= \int f(x_n, \dotsc,x_0)
\nu \otimes \mu^ {\otimes n}(dx_0,\dotsc, dx_n)
\end{align}\label{ReversibleIntegrals}
for all $n \in \naturals$ and all non-negative measurable functions $f
: S^{n+1} \to [0,\infty)$.  By Lemma
\ref{MarkovDistributions} the current result follows from \eqref{ReversibleIntegrals}.
The proof is by induction on $n$ with the case $n=1$ being part of the
definition of reversibility.

Now supposing the result is true for $n-1$, we use Lemma
\ref{MarkovDistributions}, Tonelli's Theorem
and two applications of the induction hypothesis
\begin{align*}
&\int f(s_0, \dotsc, s_n) \, \mu(s_{n-1},ds_n) \dotsm
  \mu(s_0, ds_1) \nu(ds_0) \\
&=\int \left [ \int f(s_0, \dotsc, s_n) \,
  \mu(s_{n-1},ds_n) \right ]\mu(s_{n-2},ds_{n-1}) \dotsm
  \mu(s_0, ds_1) \nu(ds_0) \\
&=\int f(s_{n-1}, \dotsc, s_0, s_n) \, \mu(s_0,ds_n)
  \mu(s_{n-2}, ds_{n-1})\dotsm
  \mu(s_0, ds_1) \nu(ds_0) \\
&=\int \left [ \int f(s_{n-1}, \dotsc, s_0, s_n) \, 
  \mu(s_{n-2}, ds_{n-1})\dotsm
  \mu(s_0, ds_1) \right ] \, \mu(s_0,ds_n) \nu(ds_0) \\
&=\int f(s_{n-1}, \dotsc, s_n, s_0) \, 
  \mu(s_{n-2}, ds_{n-1})\dotsm
  \mu(s_n, ds_1) \mu(s_0,ds_n) \nu(ds_0) \\
&=\int f(t_{n}, \dotsc, t_1, t_0) \, 
  \mu(t_{n-1}, dt_{n})\dotsm
  \mu(t_0,t_1) \nu(dt_0) \\
\end{align*}
where in the last line we have defined new integration variables $t_0
= s_0$, $t_1 = s_n$ and 
$t_k = s_{k-1}$ for $2 \leq k \leq n$.
\end{proof}

We now make the transition to discussing discrete time Markov chains
(that is to say we restrict ourselves to countable state spaces.

Recurrence is a somewhat contagious property; if you start with a
recurrent state $x$ and can reach a state $y$ from $x$ with positive
probability then it will follow that $y$ is recurrent.  Intuitively
this can be seen by making the following observations:
\begin{itemize}
\item If $x$ is recurrent and I can reach $y$ from $x$ with positive
  probability then I must be able to reach $x$ from $y$ with positive
  probability; otherwise with positive probability $x$ reaches $y$
  (returning to itself only a finite number of times on the way) and
  then never again returns to itself contradicting recurrence.
\item One way for $y$ to return to itself is to first travel to $x$,
  then return to itself some number of times, then to make the return
  trip from $x$ to $y$; since $x$ is recurrent with positive
  probability this may be done in infinitely many ways hence $y$ is
  also recurrent.
\end{itemize}
These facts and a few more are captured less prosaically in the
following lemma.
\begin{lem}\label{RecurrenceClassesMarkovChains}Let $X$ be a discrete
  time Markov chain with state space $S$, let $x \in S$ be recurrent
  and define $S_x = \lbrace y \in S \mid r_{xy} > 0 \rbrace$.  Then
  for all $y \in S_x$, it follows that $y$ is recurrent
and for every $y,z \in S_x$ we have $r_{yz} = 1$.
\end{lem}
\begin{proof}
We first handle the case of showing that $r_{yx} = 1$.  For this, we
use a union bound, the Strong Markov property and the fact that
$X_{\tau^+_y} = y$ on $\lbrace \tau^+_y < \infty \rbrace$ to see
\begin{align*}
0 &= \sprobability{\tau^+_x = \infty} {x} 
\geq \sprobability{\tau^+_y < \infty ; \theta_{\tau^+_y} \circ
  \tau^+_x = \infty} {x} \\
&= \sexpectation{\tau^+_y < \infty ; \cprobability {\mathcal{F}_{\tau^+_y}}{\theta_{\tau^+_y} \circ
  \tau^+_x = \infty} } {x} \\
&=\sprobability{\tau^+_y < \infty ; \sprobability{\tau^+_x = \infty}
  {y}} {x} = \sprobability{\tau^+_y < \infty } {x}
\sprobability{\tau^+_x = \infty}{y} = r_{xy}(1 -r_{yx})
\end{align*}
which implies $r_{yx} =1$ since we assumed $r_{xy} > 0$.

Now we turn to the task of showing that all $y \in S_x$ are recurrent.
We know that $r_{xy} > 0$ and $r_{yx} > 0$ and therefore there exist
$m, n \in \naturals$ such that $p^n_{xy} > 0$ and $p^m_{yx} > 0$.
Thus, by Proposition \ref{RecurrenceTransienceDiscreteTime} and two
applications of the 
Chapman Kolmogorov relations and the recurrence of $x$  we get
\begin{align*}
\sexpectation{\kappa_y}{y} &= \sum_{j=1}^\infty p^j_{yy} \geq
\sum_{j=1}^\infty p^{j+m+n}_{yy} =\sum_{j=1}^\infty \sum_{z \in S} \sum_{w \in S} p^m_{yz}p^j_{zw}
p^n_{wy} \\
&\geq \sum_{j=1}^\infty p^m_{yx}p^j_{xx}
p^n_{xy} =\infty
\end{align*}
which implies that $y$ is recurrent.  Knowing that $y$ is recurrent
and having already shown that $r_{yx} =1 >0$, we know that $x \in S_y$
and we can apply the first
argument in the proof to conclude that $r_{xy} =1$ as well.

Lastly let $y,z \in S_x$.  We use the fact that one way for $X$ to get
from $y$ to $z$ is by passing through $x$ first.  Formally we use a
union bound and the Strong Markov Property to see
\begin{align*}
r_{yz} &= \sprobability{\tau^+_z < \infty}{y} \geq
\sprobability{\tau^+_x < \infty ; \theta_{\tau^+_x} \circ \tau^+_z < \infty}{y}
\\
&=\sexpectation{\tau^+_x < \infty ;
  \cprobability{\mathcal{F}_{\tau^+_x}}{\theta_{\tau^+_x} \circ
    \tau^+_z < \infty}}{y} \\
&= \sprobability{\tau^+_x < \infty }{y}
\sprobability{\tau^+_z < \infty}{x} = r_{yx} r_{xz} = 1
\end{align*}
which shows us that $r_{yz} = 1$.
\end{proof}

\begin{defn}Let $X$ be a discrete time Markov chain with state space
  $S$ then we say that $X$ is \emph{irreducible} if $r_{xy} > 0$ for
  all $x,y \in S$.  If $X$ is not irreducible we say that $X$ is \emph{reducible}.
\end{defn}

There are generalizations of the notion of irreducibility to the
general discrete time Markov process case but they will be dealt with
later; the countable state space case is historically the first to be
handled and provides important motivation while avoid some subtle
points.  The first thing is to record some alternative
characterizations of irreducibility; in the sequel we'll fell free to
use these equivalences without explicit mention.  They are all just
slightly different ways of capturing the notion that a Markov chain is
irreducible if it is possible for the chain to reach any part of state
space regardless of the starting point.

\begin{prop}\label{IrreducibleEquivalencesDiscreteTimeMarkovChain}Let
  $X$ be a discrete time Markov chain with state space $S$ then $X$ is
  irreducible if and only if for every $x,y \in S$ there exists $n \in
  \naturals$ such that $p^n_{xy} > 0$.
\end{prop}
\begin{proof}
Suppose $X$ is irreducible and let $x,y \in S$; it follows that
$\sprobability{\tau^+_y < \infty}{x} > 0$.  Writing
$\sprobability{\tau^+_y < \infty}{x} = \cup_{n=1}^\infty
\sprobability{\tau^+_y = n}{x}$ we conclude there is an $n \in
\naturals$ such that $\sprobability{\tau^+_y = n}{x} > 0$.  Now
observe that by a union bound
\begin{align*}
0 &< \sprobability{\tau^+_y = n}{x} \leq \sprobability{X_n = y}{x} = p^n_{xy}
\end{align*}

On the other hand suppose that $p^n_{xy} > 0$.  Then we know that
\begin{align*}
\lbrace X_n = y \rbrace &\subset \lbrace \tau^+_y \leq n \rbrace
                          \subset \lbrace \tau^+_y < \infty \rbrace
\end{align*}
and therefore $ 0 < p^n_{xy} \leq \sprobability{\tau^+_y < \infty}{x}$.
\end{proof}

\begin{prop}\label{IrreducibleChainProperties}Let $X$ be an irreducible discrete time Markov chain, then 
\begin{itemize}
\item[(i)]Either every $x \in S$ is transient or every $x \in S$ is
  recurrent.  Moreover $r_{xy} = 1$ for every $x,y \in S$.
\item[(ii)]Every $x \in S$ has the same period
\item[(iii)]If $\nu$ is an invariant distribution then $\nu(x) > 0$
  for every $x \in S$.
\end{itemize}
\end{prop}
\begin{proof}
Property (i) is an immediate consequence of Lemma
\ref{RecurrenceClassesMarkovChains} since for irreducible $X$ we have $S = S_x$ for any $x \in S$.

To see (ii), let $x,y \in S$ and pick $m,n \in \naturals$ such that
$p^n_{xy} > 0$ and $p^m_{yx} > 0$.  Now by the Chapman Kolmogorov
relations we see that for all $j \in \integers_+$
\begin{align*}
p^{j+m+n}_{yy} &= \sum_{z \in S} \sum_{w \in S} p^m_{yz} p^j_{zw}
p^n_{wy} \geq 
p^m_{yx} p^j_{xx} p^n_{xy} 
\end{align*}
If we choose $j = 0$ then we get inequality $p^{m+n}_{yy} \geq p^m_{yx} p^n_{xy} > 0$ which
implies that $d_y$ divides $m+n$.  With this fact in hand, we see that
for $j > 0$ for which $p^j_{xx} > 0$ it follows that $p^{j+m+n}_{yy} >
0$ and therefore $d_y$ divides $j$ as well.  By definition of the
period we then get $d_y \leq d_x$.  The argument we just made is
symmetric in $x$ and $y$ so the opposite inequality holds as well and
we conclude that $d_x = d_y$.

To see (iii), suppose that $\nu$ is an invariant distribution and
pick an $x \in S$ such that $\nu(x) > 0$.  If we let $y \in S$ by
irreducibility we find $n > 0$ such that $p^n_{xy} > 0$ and by
invariance of $\nu$ we get
\begin{align*}
\nu(y) &= \sum_{x \in S} p^n_{xy} \nu(x) \geq \nu(x) p^n_{xy} > 0
\end{align*}
\end{proof}

We now move to the theorem that gives us a useful criterion for the
existence of an invariant distribution for a discrete time Markov
chain and also shows that in a strong sense any initial distribution
converges to that invariant distribution.

\begin{thm}\label{ErgodicTheoremDiscreteTimeMarkovChains}Let $X$ be an
  irreducible and aperiodic discrete time Markov chain with countable state
  space $(S, \mathcal{S})$.  Then exactly
  one of the following holds
\begin{itemize}
\item[(i)]There exists a unique invariant distribution $\nu$ for which
  $\nu(x) > 0$ for all $x \in S$ and moreover for every initial
  distribution $\mu$ we have
\begin{align}
\lim_{n \to \infty} \sup_{A \in \mathcal{S}^\infty}
  \abs{\pushforward{\theta_n}{\probabilityop_\mu}\lbrace A \rbrace -
  \sprobability{A}{\nu}} = 0
\end{align}\label{ConvergenceToInvariantDiscreteTimeMarkovChain}
\item[(ii)]An invariant distribution does not exist and 
\begin{align*}
\lim_{n \to \infty} p^n_{xy} = 0 & \text{ for all $x,y \in S$}
\end{align*}
\end{itemize}
\end{thm}
The proof breaks down is a few different lemmas.  The proof technique used here is referred to as a
\emph{coupling} argument; it will reappear with increasing levels of
sophistication later in this book.  The common thread in coupling
arguments is the construction of a joint distribution on a product
space (called a \emph{coupling}) and its use to compare a process under study to one with simpler properties.

The first part of the coupling argument is the construction of the
process on the product space.  In this case a pair of independent
Markov chains suffices but we need a few details of about such
products of Markov chains to execute the coupling argument.

\begin{lem}\label{CouplingIndependentMarkovChainsDiscreteTime}Let $X$ and $Y$ be independent discrete time
  Markov chains with state space $S$ and $T$ and transition matrices $p_{xy}$
  and $q_{xy}$ respectively.  Then $(X,Y)$ is an irreducible discrete
  Markov chain with state space $S \times T$ and transition matrix
  $r_{xz,yw} = p_{xy}q_{zw}$.  If $X$ and $Y$ are both irreducible
    and aperiodic then $(X,Y)$ is as well.  If in addition invariant
    distributions exists for both $X$ and $Y$ then it follows that
    $(X,Y)$ is recurrent.
\end{lem}
\begin{proof}
The fact that $(X,Y)$ is a discrete time Markov chain with transition
matrix $p_{xy}q_{zw}$ is a special case of Exercise
\ref{ProductOfIndependentMarkov}.  If we assume that $X$ is
irreducible and aperiodic then for all $x,y \in S$ we know that there
exists $n \in \naturals$ such that $p_{xy}^n > 0$ by irreducibility
and furthermore by aperiodicity we know that $p_{yy}^m > 0$ for all by
finitely many $m \in \naturals$ (Proposition
\ref{PeriodicReturnDiscreteTimeMarkov}) and therefore $p^{m+n}_{xy}
\geq p^n_{xy}p^m{yy} > 0$ for all by finitely many $m \in \naturals$.
Applying the same argument to $Y$ we see that for each $x,y \in S$ and
$z,w \in T$ we have $r_{xz,yw}^n =
p^n_{xy}q^n_{zw} > 0$ for all  but finitely many $n \in \naturals$.
Thus $(X,Y)$ is irreducible and aperiodic.

If we assume that $\nu$ and $\mu$ are invariant distributions for $X$
and $Y$ respectively then it follows the fact that the transition
kernel of $(X,Y)$ is a product measure that the product measure $\nu
\otimes \mu$ is invariant for $(X,Y)$.  Now apply Proposition
\ref{RecurrenceFromInvariantDistribution} to see that $(X,Y)$ has a
recurrent state $(x,y) \in S \times T$ and Proposition
\ref{IrreducibleChainProperties} to see that $(X,Y)$ is recurrent.
\end{proof}

We now apply the coupling to compare the behavior of a pair Markov
chains with the same transition matrix but different initial
distributions.  

\begin{lem}\label{StrongErgodicityMarkovChainsDiscreteTime}Let $X$ and $Y$ be independent discrete time Markov chains
  both with state space $S$ and transition matrix $p_{xy}$ but with
  initial distributions $\nu$ and $\mu$ respectively.  If $(X,Y)$ is
  irreducible, aperiodic and recurrent then
\begin{align*}
\lim_{n \to \infty} \sup_{A \in \mathcal{S}^\infty}
  \abs{\pushforward{\theta_n}{\probabilityop_\nu}\lbrace A \rbrace -
  \pushforward{\theta_n}{\probabilityop_\mu}\lbrace A \rbrace} = 0
\end{align*}
\end{lem}
\begin{proof}
By Lemma \ref{CouplingIndependentMarkovChainsDiscreteTime} we know
that $(X,Y)$ is a Markov chain with transition matrix $p_{xy}p_{zw}$.
Let $\mathcal{F}$ be the induced filtration of $(X,Y)$ and note that
by the independence of $X$ and $Y$ 
each of $X$ and $Y$ is
Markov with respect to $\mathcal{F}$.
Consider the optional time $\tau = \min \lbrace n \in \naturals \mid
X_n = Y_n \rbrace$ and note that by recurrence of $(X,Y)$ we can apply
Lemma \ref{RecurrenceClassesMarkovChains} see that $\tau$ is almost
surely finite (in fact for every $x \in S$,  $\min \lbrace n \in \naturals \mid
X_n = Y_n = x\rbrace < \infty$ almost surely).  Let $A \in
\mathcal{S}^\infty$, then since $\tau$ is
countably valued and almost surely finite by the Strong Markov
Property applied to $X$ and $Y$, 
\begin{align*}
\cprobability {\mathcal{F}_\tau} {\theta_\tau X \in A}
&=\sprobability{A}{X_\tau}
=\sprobability{A}{Y_\tau}
=\cprobability {\mathcal{F}_\tau} {\theta_\tau Y \in A}
\end{align*}
From this and the $\mathcal{F}_\tau$-measurability of $X^\tau$ and
$\tau$ it follows that $(X^\tau, \tau, \theta_\tau X) \eqdist (X^\tau,
\tau, \theta_\tau Y)$.  Define 
$\psi : S^\infty \times \integers_+ \times S^\infty \to S^\infty$
by 
\begin{align*}
\psi(s,n,t)_m = \begin{cases}
s_m & \text{if $m < n$} \\
t_{m-n} & \text{if $m \geq n$} 
\end{cases}
\end{align*}
and note that for $A \in \mathcal{S}$, 
\begin{align*}
\lbrace \psi_m \in A \rbrace = 
\cup_{n < m} \lbrace s_m \in A \rbrace \times \lbrace n \rbrace \times
  S^\infty 
\cup 
S^\infty \times \lbrace n \rbrace \times
\cup_{n \geq m} \lbrace s_{m-n} \in A \rbrace
\end{align*}
and therefore $\psi$ is measurable.  Define $\tilde{X} = \psi(X^\tau,
\tau, \theta_\tau Y)$ so that
\begin{align*}
\tilde{X}_n &= \begin{cases}
X_n & \text{if $n < \tau$} \\
Y_n & \text{if $n \geq \tau$} \\
\end{cases}
\end{align*}
and also note that $X = \psi(X^\tau, \tau, \theta_\tau X)$.  It
follows from the Expectation Rule that $X \eqdist \tilde{X}$ and
therefore for any $A \in \mathcal{S}^{\infty}$
\begin{align*}
\abs{\probability{\theta_n X \in A} - \probability{\theta_n Y \in A} } 
&=\abs{\probability{\theta_n \tilde{X} \in A} - \probability{\theta_n Y \in A} } \\
&=\abs{\probability{\theta_n \tilde{X} \in A; \tau > n} -
  \probability{\theta_n Y \in A; \tau > n} } 
\leq 2 \probability{\tau > n}
\end{align*}
and therefore since $\tau$ is almost surely finite we have
\begin{align*}
\lim_{n \to \infty} \sup_{A \in \mathcal{S}^{\infty}}
  \abs{\probability{\theta_n X \in A} - \probability{\theta_n Y \in A}
  } \leq 2 \lim_{n \to \infty} \probability{\tau > n} = 0
\end{align*}
\end{proof}

The proof of the existence of an invariant distribution also benefits
from the coupling argument of the previous lemma.
\begin{lem}\label{TransitionMatrixConvergeToZeroNoInvariantDistribution}Let $X$ be an irreducible aperiodic Markov chain with state
  space $S$ and transition matrix $p_{xy}$ such that
  there exists $x_0,y_0 \in S$ for which $\limsup_{n \to \infty}
  p^n_{x_0 y_0} > 0$, then an invariant distribution for $X$ exists.
\end{lem}
\begin{proof}
Take a subsequence $N$ such that $\lim_{n \to \infty} p^n_{x_0 y_0}$
exists and is positive.  By countability of $S$ we can use a diagonal
argument to pass to a further subsequence if necessary and assume that
there are non-negative constants $c_y$ for $y \in S$ with $c_{y_0} > 0$ such that 
$\lim_{n\to \infty} p^n_{x_0 y} = c_y$ along $N$ for all $y \in S$.
Note that by Fatou's Lemma 
\begin{align*}
0 < \sum_{y \in S} c_y &\leq \liminf_{n \to \infty} \sum_{y \in S}
                            p^n_{x_0 y} = 1
\end{align*}

Claim: $\lim_{n \to \infty} p^n_{xy} = c_y$ along $N$ for all $x,y \in
S$.

The proof of the claim uses the coupling argument.  Pick an $x \in S$
and let $Y$ be an
Markov chain  independent of $X$ with transition matrix $p_{xy}$ and
initial distribution $\delta_x$, then
$Y$ is
also irreducible and aperiodic thus it follows from Lemma
\ref{CouplingIndependentMarkovChainsDiscreteTime}
that $(X,Y)$ is an irreducible and aperiodic Markov chain with
transition matrix $r_{xz,yw} = p_{xy} p_{zw}$.  Suppose
that $(X,Y)$ is transient then it follows from Proposition
\ref{RecurrenceTransienceDiscreteTime} that 
\begin{align*}
\sum_{n=1}^\infty (p^n_{x_0y_0})^2 &= \sum_{n=1}^\infty r^n_{x_0x_0,y_0y_0} < \infty
\end{align*}
which would imply $\lim_{n \to \infty} p^n_{x_0y_0} = 0$ which is a
contradiction. Thus we know that $(X,Y)$ is recurrent and we may apply
Lemma \ref{StrongErgodicityMarkovChainsDiscreteTime} to conclude that
$\lim_{n \to \infty} (p^n_{xy} - p^n_{x_0 y}) = 0$ for all $y \in S$
and therefore the claim follows.

Now note from the Chapman Kolomogorov relation that for each $x,y \in
S$ and $n \in \naturals$
\begin{align*}
\sum_{z \in S} p^n_{xz}p_{zy} &= p^{n+1}_{xy} = \sum_{z \in S} p_{xz}p^n_{zy}
\end{align*}
Note that $p_{xz}p^n_{zy} \leq p_{xz}$ and $\sum_{z \in S} p_{xz} = 1$
and so we may use Dominated Convergence when taking limits in the
second sum.  In the first sum we can only use Fatou so we get
\begin{align*}
\sum_{z \in S} c_z p_{zy} &\leq \lim_{n \to \infty} \sum_{z \in S}
                           p^n_{xz}p_{zy} 
= \lim_{n \to \infty} \sum_{z \in S}p_{xz}p^n_{zy} 
= c_y \sum_{z \in S}p_{xz}  = c_y
\end{align*}
where all of the limits are taken along the subsequence $N$.
Now suppose we have a strict inequality for some $y \in S$, then
summing over $y$ and using Tonelli's Theorem 
and the finiteness of $\sum_{z \in S} c_z$ we get
\begin{align*}
\sum_{z \in S} c_z &= \sum_{y \in S}\sum_{z \in S} p^n_{xz}p_{zy} =
                     \sum_{z \in S} \sum_{y \in S} p^n_{xz}p_{zy} < \sum_{z \in S} c_z
\end{align*}
which is a contradiction.  Thus we in fact have $\sum_{z \in S} c_z
p_{zy} = c_y$ for all $y \in S$.  We have observed that $\sum_{z \in S}
c_z > 0$ and therefore we may define $\nu(x) = c_x / \sum_{z \in S}
c_z$ to get an invariant distribution.
\end{proof}

It remains to assemble the pieces into the proof of the theorem.
\begin{proof}
By Lemma \ref{TransitionMatrixConvergeToZeroNoInvariantDistribution}
if $X$ has no invariant distribution then $\limsup_{n \to \infty}
p^n_{xy} = 0 \leq \liminf_{n \to \infty} p^n_{xy}$ for all $x,y \in
S$; thus $\lim_{n \to \infty} p^n_{xy} = 0$ for all $x,y \in S$.  Now
suppose that an invariant distribution $\nu$ exists.  Since $X$ is
irreducible we know that
$\nu(x) > 0$ for all $x \in S$ by Proposition
\ref{IrreducibleChainProperties}.  Furthermore by the existence of
$\nu$ and Lemma \ref{CouplingIndependentMarkovChainsDiscreteTime}, if
we let $Y$ be an independent discrete time chain with transition
matrix $p_{xy}$ and initial distribution $\nu$ we
know that $(X,Y)$ is irreducible, aperiodic and recurrent.  Thus we
may apply Lemma \ref{StrongErgodicityMarkovChainsDiscreteTime} and the
fact that $\pushforward{\theta_n}{\probabilityop_\nu} = \nu$ to
conclude that 
\begin{align*}
\lim_{n \to \infty} \sup_{A \in \mathcal{S}^\infty} \abs{\pushforward{\theta_n}{\probabilityop_\mu}\lbrace A \rbrace -
  \sprobability{A}{\nu}} = 0
\end{align*}

To see uniqueness of $\nu$ suppose that we have a second invariant
distribution $\tilde{\nu}$ and note that by invariance of
$\tilde{\nu}$ and the convergence property \eqref{ConvergenceToInvariantDiscreteTimeMarkovChain}
$\sup_{A \in \mathcal{S}^\infty} \abs{\sprobability{A}{\tilde{\nu}}-
  \sprobability{A}{\nu}} = 0$ which implies $\nu =\tilde{\nu}$.
\end{proof}

\begin{prop}Let $X$ be a discrete time Markov chain with state space
  $S$ and let $x,y \in S$ with $y$ aperiodic then it follows that
\begin{align*}
\lim_{n \to \infty} p^n_{xy} &= \frac{\sprobability{\tau^+_y < \infty}{x}}{\sexpectation{\tau^+_y}{y}}
\end{align*}
\end{prop}
\begin{proof}
Let's first consider the case in which $x=y$.  Suppose that $x$ is
transient.  In that case Proposition
\ref{RecurrenceTransienceDiscreteTime} implies $\sum_{n=1}^\infty
p^n_{xx} = \sexpectation{\kappa_x}{x} < \infty$ and thus
$\lim_{n\to \infty} p^n_{xx} = 0$.  Moreover when $x$ is transient we
know that $\sprobability{\tau^+_x = \infty}{x} = 1-r_{xx} > 0$ and
therefore $\sexpectation{\tau^+_x}{x} = \infty$ and therefore the
result holds in this case.  So we now suppose that $x$ is recurrent.
Let $S_x = \lbrace y \in S \mid r_{xy} > 0 \rbrace$ be the irreducible
component containing $x$.  We may restrict $X$ to $S_x$ and then by
Proposition \ref{RecurrenceClassesMarkovChains} it follows that the
restriction is irreducible and recurrent and by Proposition
\ref{IrreducibleChainProperties} it follows that the restriction is
aperiodic.  Now we may apply Theorem
\ref{StrongErgodicityMarkovChainsDiscreteTime} to conclude that $\lim
p^n_{xx}$ exists.

Note that if we let $\xi_1 = \tau^+_x$ and $\xi_{n+1} = \tau^{n+1}_x -
\tau^{n}_x$ for $n \in \naturals$ then by the Strong Markov property
the $\xi_n$ are an i.i.d. sequence with respect to
$\probabilityop_x$.  Moreover $\sexpectation{\tau^+_x} < \infty$
(TODO: I don't believe I've shown this...)

TODO:  Finish
\end{proof}

\begin{defn}Let $P$ be a finite discrete time Markov chain on $S$, we say a function
  $h : S \to \reals$ is \emph{harmonic} if for all $x \in S$, $\sum_{y
    \in S} P(x,y) h(y) = h(x)$.
\end{defn}

\begin{lem}\label{MarkovChainHarmonicFunctionConstant}Let $P$ be an irreducible finite Markov chain on $S$ and let $h :
  S \to \reals$ be harmonic, then $h$ is constant.
\end{lem}
\begin{proof}
Let $M$ be the maximum value of $h$ and let $x \in S$ be such that
$h(x) = M$.  Suppose there exists $y \in S$ such that $P(x,y) > 0$ and
$h(y) < M$.  It would then follow that
\begin{align*}
M &= h(x) = \sum_{y \in S} h(x) P(x,y) < M \sum_{y \in S} P(x,y) = M
\end{align*}
which is a contradiction.  Thus we know that $h(y) = M$ for all $y \in
S$ such that $P(x,y) > 0$.  Now we do an induction.  Suppose $h(y) =
M$ for all $y \in S$ such that $P^{n-1}(x, y) > 0$ and suppose $z \in
S$ is such that $P^n(x,z) > 0$.  It follows from the expression of
matrix multiplication $P^n(x,z) = \sum_{y \in S} P^{n-1}(x,y) P(y,z)$
that there exists a $y \in S$ such that $P^{n-1}(x,y) > 0$ and $P(y,z)
> 0$.  So by the induction hypothesis we know that $h(y) = M$ and by
replaying the case of $n=1$ with $y$ we get that $h(z) = M$.  

By irreducibility we know that for every $y \in S$, there exists $n
\geq 0$ such that $P^n(x,y) > 0$ and thus we have
$h(y) = M$  for every $y \in S$.
\end{proof}

\begin{lem}Let $P$ be an irreducible finite Markov chain, if the
  invariant distribution exists, then is unique.
\end{lem}
\begin{proof}
Let $I$ denote the $\card(S) \times \card(S)$ identity matrix.
By Lemma \ref{MarkovChainHarmonicFunctionConstant} we know that the
matrix $P -I$ has a one dimensional null space given by the constant
functions.  Thus column rank of $P - I$ is $\card(S) - 1$ and the same
is true for the row rank; thus there is a unique solution of $\pi (P
-I) = 0$ that satisfies $\sum_{x \in S} \pi(x) = 1$.  Note that this
does not guarantee the existence of a invariant distribution as that
requires that the entries of $\pi$ be non-negative.
\end{proof}

\begin{lem}\label{DetailBalanceEquationImpliesInvariance}If $\pi(x) P(x,y) = \pi(y) P(y,x)$ for all $x,y \in S$ then
  $\pi \cdot P = \pi$.
\end{lem}
\begin{proof}
This is a simple computation for each $y \in S$,
\begin{align*}
(\pi \cdot P)(y) &=\sum_{x \in S} \pi(x) P (x,y)  = \sum_{x \in S} \pi(y) P(y,x) =
\pi(y) \sum_{x \in S} P(y,x) = \pi(y)
\end{align*}
\end{proof}

The detail balance equation says ``the probability of starting at $x$
and making a transition to $y$ is equal to the probability of starting
at $y$ and making a transition to $x$''.  To be more concise we may
say that with starting distribution $\pi$, the probability of a trajectory $x \to y$ is the same as the
probability of a trajectory $y \to x$.  This is a type of symmetry
that is sometime described as the equivalence running the chain
forward and running the chain backward.  By induction it is not hard
to see that this symmetry extends to reversing trajectories of
arbitrary finite length.  We shall prove something more general by
showing how to ``reverse'' a Markov chain that doesn't necessarily
satisfy the detail balance equations.

\begin{defn}The \emph{time reversal} of an irreducible Markov chain
  with transition matrix $P$ and invariant distribution $\pi$ is
  given by 
\begin{align*}
\hat{P}(x,y) &= \frac{\pi(y) P(y,x)}{\pi(x)}
\end{align*}
\end{defn}

\begin{lem}The time reversal is a stochastic matrix and $\pi$ is
  invariant for $\hat{P}$.  Moreover, for every $x_0, \dotsc, x_n \in
  S$, we have
\begin{align*}
\sprobability{X_0 = x_0; \dotsb ; X_n = x_n}{\pi} &=
\sprobability{\hat{X}_0 = x_n ; \dotsb ; \hat{X}_n = x_0}{\pi}
\end{align*}
\end{lem}
\begin{proof}
By stationarity of $\pi$ with respect to $P$ for all $x \in S$,
\begin{align*}
\sum_{y \in S} \hat{P}(x,y) &=\sum_{y \in S} \frac{\pi(y)
  P(y,x)}{pi(x)} \frac{1}{\pi(x)} \sum_{y \in S} \pi(y)  P(y,x) = 1
\end{align*}
To see $\pi$ is invariant for $\hat{P}$, compute for all $y \in S$,
\begin{align*}
(\pi \cdot \hat{P})(y) &= \sum_{x \in S} \pi(x) \hat{P}(x,y) = \sum_{x
  \in S} \pi(y) P(y,x) =\pi(y)
\end{align*}
The last fact follows from an induction argument where the case $n=1$
is the definition of the time reversal matrix $\hat{P}$.  If we assume
that the result holds for $n-1$ then
\begin{align*}
\sprobability{X_0 = x_0; \dotsb ; X_n = x_n}{\pi}  &= \pi(x_0) P(x_0,
x_1) \dotsb P(x_{n-1}, x_n) \\
&= \hat{P}(x_1, x_0) \pi(x_1) P(x_1, x_2) \dotsb P(x_{n-1},x_n) \\
&= \hat{P}(x_1, x_0) \pi(x_n) \hat{P}(x_n, x_{n-1}) \dotsb \hat{P}
(x_2,x_1) \\
&= \sprobability{\hat{X}_0 = x_n ; \dotsb ; \hat{X}_n = x_0}{\pi}
\end{align*}
\end{proof}

\section{Poisson Process}
The Poisson process is the standard example of a continuous time
stochastic process that has discontinuous sample paths.  It is a
Markov process and is (almost) a martingale.  

\subsection{Exponential Random Variables}
The standard construction of the Poisson process uses sums of a sequence of
i.i.d. exponential random variables so it is therefore useful to
discuss such random variables first.  As explained below exponential
random variables will figure prominently in subsequent theory of
Markov processes as well so it will be a good investment of time to
get familiar with them.

\begin{defn}Given a parameter $\lambda > 0$, the probability measure
  on $\reals_+$ given by $\mu(A) = \lambda \int_A e^{-\lambda x} \,
  dx$ is called the \emph{exponential distribution with rate
    $\lambda$}.  A random variable $\xi$ whose law is an exponential
  distribution is said to be a \emph{exponential random variable}.
\end{defn}

The reader may have learned at some point that incandescent lightbulbs
have peculiar property; the probability that such a light bulb will
fail does not depend on the age of the light bulb.  Expressed using
our notation, if we let $\xi$ be age of a light bulb when it fails we
are saying that for all $t > s$ we have $\cprobability{\xi > s}{\xi >
  t} = \probability{\xi > t -s}$ or equivalently 
\begin{align*}
\probability{\xi > t} &= \probability{\xi > t ; \xi > s} = \cprobability{\xi > s}{\xi >
  t} \probability{\xi > s}= \probability{\xi > t -s}\probability{\xi > s}
\end{align*}

While the stated fact about light bulbs is only
approximately true, it is a concrete illustration of a property
that we call memorylessness.
The reason that exponential random variables figure so prominently in
subsequent theory is that they are precisely the random variables that
have the property of being memoryless.
\begin{prop}\label{ExponentialMemoryless}Let $\gamma$ be an
  exponential random variable then for each $t,s \geq 0$ we have the
  functional equation
\begin{align}
\probability{\gamma > t+s} &= \probability{\gamma > t}\probability{\gamma > s}
\label{ExponentialDistributionFunctionalEquation}\end{align}
Moreover if $\gamma$ is a nonnegative random variable that is not almost surely
equal to $0$ and satisfies \eqref{ExponentialDistributionFunctionalEquation}, it
follows that $\gamma$ is exponential.
\end{prop}
\begin{proof}
The memorylessness property of exponential random variable is a
trivial computation, 
\begin{align*}
\probability{\gamma > t+s} &= e^{-\lambda(t+s)} = e^{-\lambda
  t}e^{-\lambda s} = \probability{\gamma >
  t}\probability{\gamma > s}
\end{align*}

If we let $\probability{\gamma > 1}= e^{-c}$ for some $c \in [0,
\infty]$, then from the functional equation
\eqref{ExponentialDistributionFunctionalEquation}
we immediately see that for every $n \in
\naturals$, $\probability{\gamma > n}=\probability{\gamma >
  1}^n= e^{-cn}$ and then for all
positive rationals $p/q \in \rationals_+$ we have
$\probability{\gamma > p/q} = e^{-cp/q}$.  Now since
$\probability{\gamma > t}$ is right continuous we can conclude that
$\probability{\gamma > t}= e^{-ct}$ for all $0 \leq t < \infty$.
By our assumption that there exists some $t \geq 0$ such that
$\probability{\gamma > t} > 0$
it follows that $c < \infty$ and we have shown that $\gamma$ is
exponentially distributed.
\end{proof}

\begin{prop}\label{SumOfExponentialMemoryless}Let $\gamma_1, \dotsc, \gamma_n$ be a sequence of
  i.i.d. exponential random variables with rate $\lambda$ then for all
  $t > s$ we have
\begin{align*}
\probability{\gamma_1 +\dotsb + \gamma_n > t; \gamma_1 > s} &=
\probability{\gamma_1 +\dotsb + \gamma_n > t-s} \probability{\gamma_1 > s} 
\end{align*}
\end{prop}
\begin{proof}
We proceed by induction.  The initial case is just the memorylessness
of a single exponential random variable.  For $n \geq 2$ we compute using Fubini's theorem (specifically Lemma
\ref{DisintegrationIndependentLaws}) and the non-negativity of
exponential random variables
\begin{align*}
&\probability{\gamma_1 + \dotsb + \gamma_n > t ; \gamma_1 > s} \\
&= \probability{\gamma_1 + \dotsb + \gamma_n > t ; \gamma_1 > s;
  \gamma_n < t-s} + \probability{\gamma_1 + \dotsb + \gamma_n > t ; \gamma_1 > s;
  \gamma_n \geq t-s}  \\
&= \expectation{\probability{\gamma_1 + \dotsb + \gamma_{n-1} > t-u ;
    \gamma_1 > s} \mid_{u = \gamma_n} ;
  \gamma_n < t-s} \\
&+ \probability{\gamma_1 + \dotsb + \gamma_n > t ; \gamma_1 > s;
  \gamma_n \geq t-s}  \\
&= \expectation{\probability{\gamma_1 + \dotsb + \gamma_{n-1} > t-u-s}
  \mid_{u = \gamma_n} ; 
  \gamma_n < t-s} \probability{\gamma_1 > s} \\
&+ \probability{\gamma_n \geq t-s} \probability{\gamma_1 > s} \\
&= \left(\probability{\gamma_1 + \dotsb + \gamma_{n} > t-s;  \gamma_n < t-s}
+ \probability{\gamma_n \geq t-s} \right) \probability{\gamma_1 > s} \\
&= \probability{\gamma_1 + \dotsb + \gamma_{n} > t-s}\probability{\gamma_1 > s} 
\end{align*}
\end{proof}

It is also worth having the density and cumulative distribution of a
sum of i.i.d. exponential random variables handy
\begin{prop}Let $\gamma_1, \dotsc, \gamma_n$ be i.i.d. exponential
  random variables with rate $\lambda$, then the density of $\gamma_1
  + \dotsb + \gamma_n$ is $\lambda^n e^{-\lambda t}
  \frac{t^{n-1}}{(n-1)!}$ and 
\begin{align*}
\probability{\gamma_1 + \dotsb + \gamma_n > t} = e^{-\lambda t}
\sum_{j=1}^{n-1} \frac{\lambda^k t^k}{k!}
\end{align*}
\end{prop}
\begin{proof}
Straightforward induction calculation using the convolution formula.
\end{proof}

We are now in a position to show that Poisson processes exist.  
\begin{thm}\label{HomogeneousPoissonProcessExistence}Let $\gamma_1,
  \gamma_2, \dotsc$ be i.i.d. exponential random variables with rate
  $\lambda > 0$.  For each $n \in \naturals$ define $S_n = \gamma_1 +
  \dotsc + \gamma_n$ and for each $t \in \reals_+$ let 
\begin{align*}
N_t = \max  \lbrace n \in \naturals \mid S_n \leq t \rbrace
\end{align*}
where the maximum of the empty set is taken to be $0$.  
Then $N$ is a homogeneous Poisson process with rate $\lambda$.
\end{thm}
\begin{proof}
Since $\lbrace N_t \geq m \rbrace = \lbrace S_m \leq t \rbrace$ the
measurability of $N_t$ follows from the measurability of the
$\gamma_n$ and thus $N$ is a stochastic process.  

It remains to
show that $N$ has independent increments.  Let $0 \leq s < t < \infty$
and consider the computation of $\cprobability{\mathcal{F}_s}{N_t
  - N_s \in \cdot}$. 
Let $\mathcal{F}$ be the filtration generated by $N$, let
$\mathcal{G}_0 = \lbrace \emptyset, \Omega \rbrace$ and for each $n
\in \naturals$ let $\mathcal{G}_n = \sigma(\gamma_1, \dotsc,
\gamma_n)$.   We wish to do this computation locally on events of the
form $\lbrace N_s = n \rbrace$ for $n \in \integers_+$ by reducing to
a conditional expectation with respect to $\mathcal{G}_n$.  

Rather than appealing to the general Lemma
\ref{ConditionalExpectationIsLocal} we use the following simple version.

Claim: Let $0 \leq s < \infty$, $n \in \integers_+$ and $A \in
\mathcal{F}_s$.  There exists a $B \in \mathcal{G}_n$ such that $A
\cap \lbrace N_s = n \rbrace = B \cap \lbrace N_s = n \rbrace$.

Note first that the set $\mathcal{C}$ of all $A \in \mathcal{F}_s$ for which an
appropriate $B \in \mathcal{G}_n$ exists is a $\sigma$-algebra.  This
is elementary since if $A \cap \lbrace N_s = n \rbrace = B \cap \lbrace
N_s = n \rbrace$ then it follows that $A^c \cap \lbrace N_s = n \rbrace = B^c \cap \lbrace
N_s = n \rbrace$ and morever if $A_m \cap \lbrace N_s = n \rbrace = B_m \cap \lbrace
N_s = n \rbrace$ for all $m \in \naturals$ then 
\begin{align*}
  \left(\cup_{m=1}^\infty A_m \right) \cap \lbrace N_s = n \rbrace
  &=
    \cup_{m=1}^\infty
    \left(A_m
    \cap
    \lbrace
    N_s
    = n
    \rbrace\right)
    =
    \cup_{m=1}^\infty
    \left(B_m
    \cap
    \lbrace
    N_s
    = n
    \rbrace\right) \\
  &=
    \left
    (\cup_{m=1}^\infty
    B_m\right)
    \cap
    \lbrace
    N_s = n \rbrace
\end{align*}
Since $\mathcal{C}$ is a $\sigma$-algebra is suffices to that that
$\lbrace N_u = m \rbrace \in \mathcal{C}$ for all $0 \leq u \leq s$
and $m \in \integers_+$ since such sets generate $\mathcal{F}_s$.  To
see this note that $\lbrace N_u = m \rbrace
\cap \lbrace N_s = n \rbrace = \emptyset$ for $m > n$, $\lbrace N_u = m \rbrace
\cap \lbrace N_s = n \rbrace = \lbrace S_m \leq u < S_{m+1} \rbrace
\cap \lbrace N_s = n \rbrace$ for $m < n$ and $\lbrace N_u = n \rbrace
\cap \lbrace N_s = n \rbrace = \lbrace S_n \leq u \rbrace
\cap \lbrace N_s = n \rbrace$.

We now use the claim to calculate  $\cprobability{\mathcal{F}_s}{N_t
  - N_s \in \cdot}$.  Let $A \in \mathcal{F}_s$ and for each $n \in \integers_+$
we pick $B_n \in \mathcal{G}_n$ such that $A \cap \lbrace N_s = n
\rbrace = B_n \cap \lbrace N_s = n \rbrace$.  We let $k \in
\integers_+$ and use the definition of $N_t$, the independence of the
$\gamma$, Lemma \ref{DisintegrationIndependentLaws} and Proposition \ref{SumOfExponentialMemoryless}
\begin{align*}
  &\probability{N_t - N_s \leq k; A} 
    = \sum_{n=0}^\infty \probability{N_t - N_s \leq k; N_s = n ; A} 
    =\sum_{n=0}^\infty \probability{N_t- N_s \leq k; N_s = n ; B_n}  \\
  &=\sum_{n=0}^\infty \probability{S_{n+k+1} > t; S_{n+1} > s; s \geq S_n ; B_n}  \\
  &=\sum_{n=0}^\infty \expectation{\probability{\gamma_{n+1} + \dotsb
    + \gamma_{n+k+1} > t-u; \gamma_{n+1} > s-u}\mid_{u=S_n}; s \geq S_n ; B_n}  \\
  &=\sum_{n=0}^\infty \expectation{\probability{\gamma_{n+1} + \dotsb
    + \gamma_{n+k+1} > t-s}\probability{\gamma_{n+1} > s-u}\mid_{u=S_n}; s \geq S_n ; B_n}  \\
  &=\probability{\gamma_{1} + \dotsb
    + \gamma_{k+1} > t-s} \sum_{n=0}^\infty \probability{S_{n+1} > s; s \geq S_n ; B_n}  \\
  &=\probability{\gamma_{1} + \dotsb
    + \gamma_{k+1} > t-s} \sum_{n=0}^\infty \probability{N_s=n ; B_n}  \\
  &=\probability{\gamma_{1} + \dotsb
    + \gamma_{k+1} > t-s} \sum_{n=0}^\infty \probability{N_s=n ; A}  \\
  &=\probability{\gamma_{1} + \dotsb
    + \gamma_{k+1} > t-s} \probability{A}  \\
\end{align*}
which shows that 
\begin{align*}
\cprobability{\mathcal{F}_s}{N_t  - N_s \leq k} 
&= \probability{\gamma_{1} + \dotsb  + \gamma_{k+1} > t-s}
= e^{-\lambda(t-s)} \sum_{j=0}^{k} \frac{\lambda^j(t-s)^j}{j!}
\end{align*}
Since the conditional probability is a constant it follows that $N_t -
N_s \Independent \mathcal{F}_s$ and moreover by taking expectations it
follows that $N_t - N_s$ is
Poisson distributed with rate $\lambda(t-s)$.
\end{proof}

A homogeneous Poisson process provides us with another important
example of a continuous time martingale.
\begin{prop}\label{CompensatedHomogeneousPoissonProcessMartingale}Let $N$ be a homogeneous Poisson process with rate
  $\lambda$ then $N_t - \lambda t$ is a cadlag martingale.
\end{prop}
\begin{proof}
It is clear that $N_t - \lambda t$ is a cadlag process, moreover since
$N_t$ is Poisson distributed with rate $\lambda t$ it follows that
$N_t - \lambda t$ is integrable and has mean zero.  The martingale
property follows from the independent increments property
\begin{align*}
\cexpectationlong{\mathcal{F}_s}{N_t} 
&=\cexpectationlong{\mathcal{F}_s}{N_t - N_s}  + N_s = \lambda(t-s) + N_s
\end{align*}
\end{proof}

\section{Pure Jump-Type Markov Processes}
In this section we discuss a simple subclass of time homogeneous
Markov Processes on $\reals_+$.
\begin{defn}A time homogenous Markov process on $\reals_+$ with values
  in a metric (topological?) space $(S, \mathcal{B}(S))$ is said to be \emph{pure
    jump-type} if almost surely its sample paths are piecewise
  constant with isolated jump discontinuities.
\end{defn}

The first goal is to get a more constructive description of the class
of pure jump-type Markov processes.  The key idea in achieving that goal
is to study the random time to the jumps of the process; in fact these
random times are optional with respect to the right continuous
filtration generated by the process.

\begin{defn}Let $X$ be a pure jump-type Markov process then the
  \emph{first jump time} is the random time
\begin{align*}
\tau_1 &=\inf \lbrace t \geq 0 \mid X_t \neq X_0 \rbrace
\end{align*}
 the \emph{$n^{th}$ jump time} is defined to be 
\begin{align*}
\tau_n &= \tau_{n-1} + \tau_1 \circ \theta_{\tau_{n-1}} = \inf \lbrace
t \geq \tau_{n-1} \mid X_t \neq X_{\tau_{n-1}} \rbrace \text{ for $n > 1$}
\end{align*}
and the \emph{$0^{th}$ jump time} is $\tau_0 = 0$.
\end{defn}

\begin{lem}Let $X$ be a pure jump-type Markov process then
$\tau_n$ is a weakly $\mathcal{F}$-optional time for all $n \geq 0$.
\end{lem}
\begin{proof}
The case $\tau_0$ is trivial as it is a deterministic time.
For each $n \in \naturals$, define $\sigma_n = \min \lbrace k/2^n \mid
X_{k/2^n} \neq X_0 \rbrace$.  Note that because of the right
continuity of sample paths of $X$ we have $\sigma_n \downarrow
\tau_1$.  Moreover we have
\begin{align*}
\lbrace \sigma_n \leq t \rbrace &= \cup_{k=0}^{\floor{2^n t}}
\lbrace X_{k/2^n} \neq X_0 \rbrace 
\in \mathcal{F}_{\floor{2^n t}/2^n} \subset \mathcal{F}_t
\end{align*}
and therefore $\sigma_n$ is $\mathcal{F}$-optional.  Therefore by
Lemma \ref{InfSupStoppedFiltration} we see that $\tau_1 = \lim_{n
  \to\infty} \sigma_n = \inf_n \sigma_n$ is weakly
$\mathcal{F}$-optional.

The fact that $\tau_{n}$ is weakly optional follows by induction using
Lemma
\ref{TimeShiftOptionalTimes} applied to the expression $\tau_n =
\tau_{n-1} + \tau_1 \circ \theta_{\tau_{n-1}}$.
\end{proof}

The definition of the optional time $\tau_1$ allows us to define an
important property of elements of $S$.
\begin{defn}A state $x \in S$ is said to be \emph{absorbing} if
  $\probability{X_t \equiv x} = \sprobability{\tau_1 = \infty}{x} =
  1$.  If $x$ is not absorbing we say it is \emph{non-absorbing}.
\end{defn}
By the Markov property we see that if a pure jump-type Markov process
$X$ reaches an absorbing state $x$ it remains there indefinitely
almost surely.  If $X$ is in a non-absorbing state one might ask
whether there is a positive probability that it remains there forever
(i.e. the state is ``partially absorbing'').  In fact in a
non-absorbing state it is almost sure that a jump to a new state will
occur (a kind of 0-1 law).  This fact is a corollary of the following
result that describes the distribution to the next jump from a
non-absorbing state.
\begin{lem}\label{PureJumpFirstJumpTime}Let $X$ be a pure jump-type Markov process and let $x \in
  S$ be nonabsorbing, then under $\probabilityop_x$ the optional time
  $\tau_1$ is exponentially distributed and independent of $\theta_{\tau_1}X$.
\end{lem}
\begin{proof}
To see that $\tau_1$ is exponentially distributed note that
\begin{align*}
\sprobability{\tau_1 > t+s}{x} &= \sprobability{\tau_1 > s; \tau_1
  \circ \theta_s > t}{x} = \sprobability{\tau_1 > s}{x} \sprobability{\tau_1 > t}{x} 
\end{align*}
By our assumption that $x$ is nonabsorbing we know that $\tau_1 > 0$
with positive probability and therefore we can apply Proposition
\ref{ExponentialMemoryless} to conclude that $\tau_1$ is
exponentially distributed.

Recall from Lemma
\ref{HittingTimesContinuous} that when restricted to
$D([0,\infty);S)$, one can think of $\tau_1$
as being a composition of the process $X$ with a measurable function on
$S^{[0,\infty)}$ which we call $\tilde{\tau}_1$
(of course if $X$ is the canonical process $\tau_1=\tilde{\tau}_1$).  
Let $B$ be a measurable set in $S^{[0,\infty)}$ and define the set
\begin{align*}
\tilde{B} &= \lbrace f \in D([0,\infty) \mid \theta_{\tilde{\tau}_1(f)} f \in
B \rbrace
\end{align*}
By writing the indicator of $\tilde{B}$
as the composition 
\begin{align*}
D([0,\infty) ; S) \overset{(id, \tilde{\tau}_1)}  \to D([0,\infty) ; S) \times
[0, \infty)
\overset{\theta} \to D([0,\infty) ; S) \overset{\characteristic{B}}
\to \reals
\end{align*}
we see that $\tilde{B}$ is also measurable
(recall that $\theta$ as above is measurable by Lemma
\ref{MeasurabilityOfShiftOperator}).  It is also noted that we have
the equality $\lbrace X \in \tilde{B} \rbrace = \lbrace
\theta_{\tau_1} X \in B \rbrace$.

Let $\tau^t_1 = \inf \lbrace s \geq
t \mid X_s \neq X_t \rbrace$ and note that $\tau^t_1(X) =
\tilde{\tau}_1(\theta_t X) + t$.   From this we get
\begin{align*}
\left( \theta_{\tau^t_1}X \right)_s &=
X(\tilde{\tau}_1(\theta_tX) + t + s) = \left(\theta_{\tilde{\tau}_1(\theta_t X)} \theta_tX
\right)_s
\end{align*} 
Now we can compute (in rather excruciating detail I might add) using
the fact that $\tau^t_1 = \tau_1$ on the set $\lbrace \tau_1 >
t \rbrace$, the Markov Property of $X$, the definition of $\tilde{B}$ and the fact that $X_t = x$ on
$\lbrace \tau_1 > t \rbrace$ to see
\begin{align*}
\probability{\tau_1 > t ; \theta_{\tau_1}X \in B}
&=
\probability{\tau_1 > t ; \theta_{\tau^t_1}X \in B}\\
&=
\probability{\tau_1 > t ;
  \cexpectationlong{\mathcal{F}_t}{\theta_{\tilde{\tau}_1(\theta_t X)} \theta_t X \in B}}\\
&=
\probability{\tau_1 > t ;
  \cprobability{\mathcal{F}_t}{\theta_t X \in \tilde{B}}}\\
&=
\probability{\tau_1 > t ;
  \sprobability{\tilde{B}}{X_t}}\\
&= \probability{\tau_1 > t } \sprobability{\tilde{B}}{x}\\
&= \probability{\tau_1 > t } \probability{\theta_{\tau_1} X \in B}\\
\end{align*}
\end{proof}

With the distribution of first jump time available we can now see that
a the first jump time is either almost surely finite or almost surely
infinite depending on whether the process starts in a non-absorbing or
absorbing state.
\begin{cor}\label{AbsorbingDichotomy}Let $\probabilityop_x$ be a
  Markov family for pure jump-type Markov process and let $\tau_1$ be
  the first jump time then
\begin{align*}
\sprobability{\tau_1 < \infty} {x} &= \begin{cases}
0 & \text{when $x$ is non-absorbing} \\
1 & \text{when $x$ is absorbing}
\end{cases}
\end{align*}
\end{cor}
\begin{proof}
By Lemma \ref{PureJumpFirstJumpTime} we know that for $x$
non-absorbing $\sexpectation{\tau_1}{x} < \infty$ which implies
$\sprobability{\tau_1 < \infty}{x} < \infty$.  
\end{proof}

It should be noted that in the literature it is very uncommon to make
the subtle distinction between the interpretation of $\tau_1$ as
either a random variable or a function on $D([0, \infty); \reals)$.
On the one hand, authors may deal with the issue by glossing over the
distinction and abusing notation through the use of $\tau_1$ to denote
both functions.  On the other hand authors may try to define the
problem away by restricting attention to the canonical case; this
restriction later biting the reader when results proven in the
canonical case are implicitly extended to the non-canonical case.  At
some point we will start to take the abuse of notation approach
but we want to have some examples in which all of the fine
distinctions are made so that the reader can refer back to them in
times of confusion.

Based on the previous result we see that the distribution of the first
jump of a pure
jump type Markov process boils down to two independent distributions:
the first being an exponential distribution that describes when a jump
happens and the second being a general distribution that describes
where the jump goes to.  This observation can be used to give us a
nice description of the entire process.  Before providing the
construction we settle on some terminology.


\begin{defn}Given a pure jump Markov process $X$ with a first jump
  time $\tau_1$ we define the \emph{rate function} to be 
\begin{align*}
c(x) &=\begin{cases}
  1/\sexpectation{\tau_1}{x} & \text{if $x$ is non-absorbing}\\
0 & \text{if $x$ is absorbing}
\end{cases}
\end{align*}
 the \emph{jump transition kernel} to be
 \begin{align*}
\mu(x,B) &= \begin{cases}
\sprobability{\theta_{\tau_1}X \in B}{x} & \text{ if $x$ is non-absorbing} \\
\delta_x(B) & \text{ if $x$ is absorbing}
\end{cases}
\end{align*} 
and the \emph{rate
    kernel} to be $\alpha(x,B) = c(x) \mu(x,B)$.  
\end{defn}
Note that in the above definition we are thinking of the Markov
process as the family of measures $\probabilityop_x$ on
$S^{[0,\infty)}$ and interpreting $\tau_1$ as a
function from $S^{[0,\infty)}$ to $\reals_+$.

Before proceeeding to our structure theory for pure jump type Markov
processes we establish the basic measurability properties of the
functions just defined.
\begin{lem}\label{MeasurabilityRateKernel}The rate function $c(x)$ is a measurable function on $S$
  and the jump transition kernel and rate kernel are both kernels from
  $S$ to $S^{[0,\infty)}$.  The rate kernel is a measurable function
  of the jump transition kernel.
\end{lem}
\begin{proof}
We know that $\probabilityop_x$ is a kernel by Lemma
\ref{MarkovMixtures} and therefore $\sexpectation{\tau_1}{x}$ is a
measurable function of $x$ by Lemma
\ref{KernelTensorProductMeasurability}.  Lastly we see that 
\begin{align*}
\lbrace x \text{ is non-absorbing} \rbrace &= \lbrace \sprobability{\tau_1 <
  \infty}{x} = 1 \rbrace
\end{align*}
is measurable because $\probabilityop_x$ is a kernel; thus $c(x)$ is measurable.

The fact that $\mu(x,B)$ is a measurable function of $x$ for fixed $B$
follows from the fact $\probabilityop_x$ is a kernel.  The fact that
for fixed $\mu(x,B)$ is a probability measure for fixed $x$ follows
from measurability of the mapping taking $X$ to $\theta_{\tau_1}X$ and
Lemma \ref{PushforwardMeasure}.

To see that $\mu(x,B)$ is a measurable function of $\alpha(x,B)$ just
observe that 
\begin{align*}
\mu(x,B) &= \begin{cases}
\alpha(x,B)/\alpha(x, S) & \text{if $\alpha(x, S)
  \neq 0$} \\
\delta_x(B) & \text{if $\alpha(x, S) = 0$}
\end{cases}
\end{align*}
\end{proof}

Extending these ideas further we will see that every pure jump-type Markov
process decomposes into a discrete time Markov chain that describes
the state transition of the jumps that occur and a sequence of independent exponential random
variables that describe the time between jumps.   This make intuitive sense given the
last lemma and the Strong Markov property: our process begins by
waiting for an exponentially distributed time then makes an
independent jump to a new state; by the Strong Markov property the
process starts afresh in the new state waits for another independent
exponentially distributed time and makes another independent jump and
so on. 
One subtlety arises because the heuristic argument just given
ignores the fact that our process may jump into an absorbing state.
The other subtlety is that the mean time to the next jump depends on
the current state.  If we normalize by the rate function of the
current state then the means are all unity and we might be able
``integrate'' the waiting times into the single source of randomness
that a sequence of i.i.d. exponential random variables would provide.
Handling these problems and making things precise is the job of the
next theorem.

\begin{thm}Let $X$ be a pure jump Markov process with rate kernel
  $\alpha=c\mu$ and jump times $\tau_0, \tau_1, \dotsc$, then there is a Markov process $Y$ on $\integers_+$ with transition kernel $\mu$
  and a
  sequence of i.i.d. exponential random variables $\gamma_0, \gamma_1,
  \dotsc$ of rate $1$ that are independent of $Y$ such that for all $n
  \geq 1$ 
\begin{align*}
\tau_n &= \begin{cases}
\sum_{k=0}^{n-1} \frac{\gamma_k}{c(Y_k)} &\text{when $c(Y_k)
  \neq 0$ for all $k=0, \dotsc, n-1$} \\
\infty & \text{when $c(Y_k) = 0$ for some $k=0, \dotsc, n-1$}
\end{cases}
\end{align*}
and
\begin{align*}
X_t &= Y_n  \text{ a.s. for $\tau_n \leq t < \tau_{n+1}$}
\end{align*} 
when $\tau_n < \infty$.  If $\tau_n = \infty$ for some $n$ then let
$N = \max \lbrace n \mid \tau_n < \infty \rbrace$, then we have $Y_n =
Y_{N-1} = X_{\tau_{N}}$ for all $n > N$.
\end{thm}
\begin{proof}
To simply notation, in the case in which $\tau_n = \infty$ for some
$n$, let $X_\infty = X_{\tau_N}$ where $N$ is defined in the statement
of the Theorem (it is the position of $X$ after its last jump). With
that definition in hand we know that the result of the Theorem
requires that we define $Y_n = X_{\tau_n}$.  The work is in
constructing the $\gamma_n$ and validating the Markov property.

Our first real task is to understand the relationship between the condition
$\lbrace \tau_n < \infty \rbrace$ and the condition $\lbrace c(Y_{n-1})
\neq 0 \rbrace$ in order to make proper sense of the expression for
$\tau_n$.  

Claim: $\tau_n < \infty$ almost surely when $c(Y_{n-1}) \neq 0$ and
$\tau_{n-1} < \infty$
(i.e. $\probability{\tau_n < \infty;c(Y_{n-1}) \neq 0; \tau_{n-1} < \infty} =
\probability{c(Y_{n-1}) \neq 0; \tau_{n-1} < \infty}$).

First note that for any $x \in S$, by definition $c(x) \neq 0$ implies
that $\sexpectation{\tau_1}{x} < \infty$ which certainly implies that
$\sprobability{\tau_1 < \infty}{x} =1$.  Now for all $n \geq 1$ we can
calculate using the tower property and pullout property of conditional
expectations and the Strong Markov property
\begin{align*}
&\probability{\tau_n < \infty;c(Y_{n-1}) \neq 0; \tau_{n-1} < \infty} \\
&=\expectation{\cprobability{\mathcal{F}_{\tau_{n-1}}}{\tau_n <
    \infty;c(Y_{n-1}) \neq 0; \tau_{n-1} < \infty}} \\
&=\expectation{\cprobability{\mathcal{F}_{\tau_{n-1}}}{\tau_1(\theta_{\tau_{n-1}(X)}X) <
    \infty};c(Y_{n-1}) \neq 0; \tau_{n-1} < \infty} \\
&=\expectation{\sprobability{\tau_1 <
    \infty}{Y_{n-1}};c(Y_{n-1}) \neq 0; \tau_{n-1} < \infty} \\
&=\probability{c(Y_{n-1}) \neq 0; \tau_{n-1} < \infty} \\
\end{align*}
and the claim is proved.

Claim: $\lbrace c(Y_{n-1}) = 0 \rbrace$ = $\lbrace \tau_n = \infty
\rbrace$ a.s.

What does this mean?  I think $\probability{\lbrace c(Y_{n-1}) = 0
  \rbrace \triangle \lbrace \tau_n = \infty\rbrace} = 0$.
Calculate
\begin{align*}
&\probability{c(Y_{n-1}) = 0; \tau_n < \infty} \\
&=
\probability{c(Y_{n-1}) = 0; \tau_{n-1} < \infty;
  \tau_1(\theta_{\tau_{n-1}(X)}(X)) < \infty } \\
&= \probability{c(Y_{n-1}) = 0; \tau_{n-1} < \infty;
  \cprobability{\mathcal{F}_{\tau_{n-1}}}{\tau_1(\theta_{\tau_{n-1}(X)}(X)) < \infty }} \\
&= \probability{c(Y_{n-1}) = 0; \tau_{n-1} < \infty;
  \sprobability{\tau_1 < \infty }{Y_{n-1}}} = 0 \text{ by Corollary \ref{AbsorbingDichotomy}}\\
\end{align*}

TODO: Here is the crux of where I get confused.  Kallenberg says the
following:
let $\gamma_1^\prime, \gamma_2^\prime, \dotsc$ be i.i.d. exponentially
distributed of mean 1 and such that $\gamma_n^\prime \Independent X$
which means we must be willing to break out of the canonical case (this is technically not
true of Kallenberg's proof since he states that all randomization variables are assumed to exist
in the canonical process setup).  Define 
\begin{align*}
\gamma_n &= (\tau_n - \tau_{n-1})c(Y_{n-1}) \characteristic{\tau_{n} <
  \infty} + \gamma_n^\prime \characteristic{\tau_{n} =
  \infty} 
\end{align*}
and we claim that if $c(x) > 0$ then we have
\begin{align*}
\sprobability{\gamma_1 > t; Y_1 \in B}{x} &= \sprobability{\tau_1 c(x)
  > t; Y_1 \in B}{x} = e^{-t} \mu(x, B)
\end{align*}
and that if $c(x) = 0$ then 
\begin{align*}
\sprobability{\gamma_1 > t; Y_1 \in B}{x} &= \sprobability{\gamma^\prime_1 
  > t; Y_1 \in B}{x} = e^{-t} \mu(x, B)
\end{align*} 
and this is where I get hung up on a subtlety.  The measure
$\probabilityop_x$ was defined to be on the path space but $\gamma_1$
is not defined on the path space but on an extension.  Probably the
right way to make sense of this is to consider a Markov family as in
Definition \ref{MarkovFamilyDefn} and then consider $\probabilityop_x$
in that context.  Of course, we have not proven that Markov families
exist (though I believe that is implicit in the proof of Markov
processes and the proof of Daniell-Kolmogorov) nor have we proven that
Markov families are preserved under extension (see Blumenthal and Getoor for an exercise that
is sufficient for the case of extending by $[0,1]$).  In any case if we
succeed in doing that then $\probabilityop_x$ is the probability
measure on $\Omega$ under which $X$ is a Markov process with $X_0 = x$
almost surely and computation is a straightforward application of
Lemma \ref{PureJumpFirstJumpTime} and the independence of
$\gamma_n^\prime$ and $X$.  The thing that I am unsatisfied with in this context
is the fact that the statement of the result does not involve or
require Markov
families.  Also, what if $X$ has a non-point mass initial
distribution?  Of course the other issue is that Kallenberg's formula for
$\gamma_n$ is wrong!!!!  He writes
\begin{align*}
\gamma_n &= (\tau_n - \tau_{n-1})c(Y_n) \characteristic{\tau_{n-1} <
  \infty} + \gamma_n^\prime \characteristic{\tau_{n-1} =
  \infty} 
\end{align*}
\end{proof}

It is useful to turn this description of  a pure jump-type Markov
around and use it to construct a pure jump-type Markov process.  

\begin{thm}Let $\alpha = c \mu$ be a kernel on $S$ such that
  $\alpha(x, \lbrace x \rbrace) \equiv 0$, let $Y$ be a Markov chain
  with transition kernel $\mu$ and let $\gamma_1, \gamma_2, \dotsc$ be
  i.i.d. exponential random variables of mean 1 such that $\gamma_1,
  \gamma_2, \dotsc \Independent Y$.  Pick an arbitrary element $s_0
  \in S$ and define $\tau_0 = 0$ and for $n
  \in \naturals$ we define
\begin{align*}
\tau_n &= \sum_{j=1}^n  \frac{\gamma_j}{c(Y_{j-1})} \\
\intertext{and}
X_t &= \begin{cases}
Y_n \text{ for $\tau_n \leq t < \tau_{n+1}$} \\
s_0 \text{ for $t \geq \sup_n \tau_n$} \\
\end{cases}
\end{align*}
If $\lim_{n \to \infty} \tau_n = \infty$ a.s. for every initial distribution
  of $Y$ then $X$ is a pure jump-type Markov process with rate kernel $\alpha$.
\end{thm}
\begin{proof}
We consider $(Y, \gamma)$ as a Markov chain on the state space $S
\times \reals_+$.  (TODO: Show that independence of $Y$ and $\gamma_1,
\gamma_2, \dotsc$ implies this is valid).  Define $\tau_n$ and $X$ as
in the statement of the theorem, let $\mathcal{G}_n$ be the filtration
generated by $(Y, \gamma)$ and let $\mathcal{F}_t$ be the filtration
generated by $X$.

We need leverage our knowledge that $(Y,\gamma)$ is a Markov process
to show that $X$ has the Markov property.  In order to do this we want
to be able use information about conditional expectations with respect
to $\mathcal{G}$ in order to compute conditional expectations with
respect to $\mathcal{F}$; thus we first clarify the relationship
between the two filtrations.  The trick is that in general a given
$X_t$ can be equal to any $Y_n$ and therefore to restrict the set of
possible $Y_n$ we must restrict the number of jumps that occur before
$t$.  In other words we must restrict the possible values of some
$\tau_n$; in this way the random variables $\gamma_1, \gamma_2,
\dotsc$ enter the picture.

Claim: Let $t \geq 0$ and $n \in \naturals$ be fixed, then
$\mathcal{G}_n \vee \lbrace \tau_{n+1} > t \rbrace$ and
$\mathcal{F}_t$ agree on $\lbrace \tau_n \leq t < \tau_{n+1}\rbrace$.
Furthermore $\lbrace \tau_n \leq t < \tau_{n+1}\rbrace$ is
$\mathcal{G}_n \vee \lbrace \tau_{n+1} > t\rbrace\cap \mathcal{F}_t$-measurable.

The $\mathcal{G}_n \vee \lbrace \tau_{n+1} > t \rbrace$ of $\lbrace
\tau_n \leq t < \tau_{n+1}$ is immediate as $\tau_n$ is a function of
$\gamma_1, \dotsc, \gamma_n$ and $Y_0, \dotsc, Y_{n-1}$ and therefore is
$\mathcal{G}_n$ measurable.  To see $\mathcal{F}_t$-measurability
first note that, by construction, $\tau_n$ is the $n^{th}$ jumping
time of $X$ (TODO: What about the fact that we have probability zero
event that $Y_m = Y_{m+1}$?  This seems like a real issue since $X$
cannot detect $\tau_n$ unless the value of $X$ changes there.  What we
do know is that if $X$ sees $n$ jumps then at least $n$ of the timers
$\gamma$ have gone off; maybe this is enough...)
TODO:

Note that even here we have $Y$ assumed to be a Markov family and we
are constructing $X$ as a Markov family.
\end{proof}

TODO: Kolmogorov Backward equation.

Let $X$ be a pure jump-type Markov process on state space $S$, $\tau$
be the first jump time and let $\sigma = \tau \wedge t$ for some $0
\leq t < \infty$.  Now by the Strong Markov Property, the independence
of $X_\tau$ and $\tau$ and the disintegration Lemma \ref{DisintegrationIndependentLaws}
\begin{align*}
T_tf(x) 
&=\sexpectation{f(X_t)}{x} 
=\sexpectation{f((\theta_\sigma X)_{t - \sigma})}{x} 
=\sexpectation{\cexpectationlong{\mathcal{F}_\sigma}{f((\theta_\sigma  X)_{t - \sigma})}}{x} \\
&=\sexpectation{\sexpectation{f(X_{t - \sigma})}{X_\sigma}}{x} 
=\sexpectation{T_{t - \sigma}f(X_\sigma)}{x} \\
&=\sexpectation{T_{t - \sigma}f(X_\sigma) ; \tau > t}{x} +
  \sexpectation{T_{t - \sigma}f(X_\sigma); \tau \leq t}{x} \\
&= \sexpectation{T_{0}f(X_0) ; \tau > t}{x}  +
  \sexpectation{T_{t - \tau}f(X_\tau); \tau \leq t}{x} \\
&=f(x) \sprobability{\tau > t}{x} +  \int_0^t \int c(x) e^{-sc(x)} T_{t - s}f(y) \,  \mu(x, dy) ds \\
&=e^{-tc(x)}   \left( f(x) +  \int_0^t \int c(x) e^{-sc(x)} T_{s}f(y) \,  \mu(x, dy) ds \right )\\
\end{align*}
TODO: There is some measurability subtlety in applying disintegration; we need to know that $T_tf(x)$ is jointly measurable in $(t,x)$
and then apply it to $(\tau, X_\tau)$ and use the fact that $X_\tau$ is progressively measurable by right continuity and thus $X_\tau$ is measurable
by Lemma \ref{StoppedProgressivelyMeasurableProcess}.  TODO: How do the measurability considerations in the definition of Markov family enter into the picture here.

\begin{clm}Let $(\Omega, \mathcal{A}, \mathcal{F}_t, X_t, \theta_t, P_x)$ be a Markov family with $X$ jointly measurable, let $f$ be a bounded or non-negative measurable function and define $T_tf(x) = \sexpectation{f(X_t)}{x}$ then $T_tf(x)$ is a jointly measurable function $[0,\infty) \times S \to \reals$.
\end{clm}
To see the claim we just use the kernel property of $P_x$ and write $T_tf(x) = \sexpectation{f(X_t)}{x} = \int X(t, \omega) \, P_x(d\omega)$ and apply Exercise \ref{MeasurabilityKernelExtraParameter}.  TODO:  Since $X$ is $\mathcal{B}([0,\infty)) \otimes \mathcal{F}^X_\infty$ measurable I don't think the universal measurability issues enter into the discussion here.




\section{Stochastic Integration}

\subsection{Local Martingales}

\begin{defn}Let $M_t$ be an $\mathcal{F}$-adapted process, we say $M$ is a \emph{local martingale} if there exists a sequence of optional times $\tau_n$ such that $\tau_n \uparrow \infty$ a.s. and $M^{\tau_n} - M_0$ is an $\mathcal{F}$-martingale for all $n$.  We say that $\tau_n$ is a \emph{localizing sequence} for $M$.
\end{defn}


A useful fact is that continuous local martingales can always be localized to bounded martingales.
\begin{lem}\label{ContinuousLocalMartingaleLocalizeToBounded}Let $M$ be a continuous local martingale and for each $n \in \integers_+$ let $\tau_n = \inf \lbrace t \geq 0 \mid \abs{M_t} \geq n \rbrace$ then $\tau_n$ is a localizing sequence for $M$.
\end{lem}
\begin{proof}
By continuity and $\mathcal{F}$-adaptedness of $M$ and the fact that $[t, \infty)$ is closed we know that $\tau_n$ is an optional time (Lemma \ref{HittingTimesContinuous}). It is clear that $\abs{M_t} \geq n$ implies $\abs{M_t} \geq n-1$ and therefore $\tau_n$ is an increasing sequence.  By continuity of $M$ we know that $M$ is bounded on bounded intervals and therefore $\tau_n \uparrow \infty$ a.s.  

It remains to show that $(M - M_0)^{\tau_n}$ is a martingale for every $n \in \integers_+$.   Let $\sigma_m$ be a localizing sequence for $M$.  From Optional Stopping we know that $(M - M_0)^{\tau_n \wedge \sigma_m} = ((M-M_0)^{\sigma_m})^{\tau_n}$ is a martingale for every $m,n \in \integers_+$.  Furthermore for fixed $n$ and every $m \in \integers_+$ since $\sigma_m \uparrow \infty$ a.s. we know that $(M-M_0)^{\tau_n \wedge \sigma_m}_t \toas (M-M_0)^{\tau_n}_t$.  Morever $\abs{(M-M_0)^{\tau_n \wedge \sigma_m}_t } = \abs{M_{\tau_n \wedge \sigma_m \wedge t} - M_0} \leq \abs{M_0} + n$.  Since $M_0$ is integrable (TODO: Do we really know this with the Kallenberg definition of a local martingale?; if not what replaces it do we define $\tau_n = \inf \lbrace t \geq 0 \mid \abs{M_t - M_0} \geq n \rbrace$?) so that by Dominated Convergence we get $(M-M_0)^{\tau_n \wedge \sigma_m}_t \tolp{1} (M-M_0)^{\tau_n}_t$ as well. Using both forms of convergence and the martingale property of $M^{\tau_n \wedge \sigma_m}$, for each $s < t$ we get the equality
\begin{align*}
\cexpectationlong{\mathcal{F}_s}{(M - M_0)^{\tau_n}_t} &= \lim_{m \to \infty} \cexpectationlong{\mathcal{F}_s}{(M - M_0)^{\tau_n \wedge \sigma_m }_t} \\
&= \lim_{m \to \infty} (M - M_0)^{\tau_n \wedge \sigma_m }_s =  (M - M_0)^{\tau_n }_s \text{ a.s.}
\end{align*}
which shows that $M^{\tau_n}$ is a martingale.
\end{proof}

\begin{lem}Let $\mathcal{F}$ be a right continuous filtration, $M$ be a cadlag $\mathcal{F}$-local martingale with localizing sequence $\tau_n$ and let $\sigma_n$ be an arbitrary sequence of bounded optional times such that $\sigma_n \uparrow \infty$, then $\tau_n \wedge \sigma_n$ is a localizing sequence for $M$.  In particular the space of cadlag $\mathcal{F}$-local martingales is a linear space.
\end{lem}
\begin{proof}
First we claim that every local martingale $M$ has localizing sequence of bounded optional times.  This follows from picking an arbitrary localizing sequence $\tau_n$ and then noting that $\tau_n \wedge n$ is also a localizing sequence as $\tau_n \wedge n \uparrow \infty$ a.s. and $M^{\tau_n \wedge n} -M_0 = (M^{\tau_n})^n - M_0$ is a martingale from Optional Stopping (Theorem \ref{OptionalStoppingContinuous}) since $M^{\tau_n}$ is a cadlag martingale, $\mathcal{F}$ is right continuous and $n$ is a bounded optional time.

Given $\tau_n$ and $\sigma_n$ as in the hypothesis and by our first claim we assume that each $\tau_n$ and $\sigma_n$ is bounded.  It is clear that $\tau_n \wedge \sigma_n$ is a sequence of optional times such that $\tau_n \wedge \sigma_n \uparrow \infty$ and again applying Optional Stopping we see that $M^{\tau_n \wedge \sigma_n} -M_0 = (M^{\tau_n})^{\sigma_n} - M_0$ is a martingale.  

Lastly if we are given $M$ and $N$ local martingales, take $\tau_n$ and $\sigma_n$ to be bounded localizing sequences for $M$ and $N$ respectively and by the previous claim, we know that $\tau_n \wedge \sigma_n$ is a joint localizing sequence for $M$ and $N$.   Therefore $(aM + bN)^{\tau_n \wedge \sigma_n} - a M_0 - b N_0$ is a martingale for all $n \geq 0$.
\end{proof}

\begin{lem}\label{LocalMartingaleLocalProperty}Let $\tau_n$ be a sequence of optional times such that $\tau_n \uparrow \infty$ a.s.  and let $M$ be an $\mathcal{F}$-adapted process.  Then $M$ is a local martingale if an only if $M^{\tau_n}$ is for all $n\geq 0$.
\end{lem}
\begin{proof}
TODO:
\end{proof}

\begin{lem}\label{ContinuousLocalMartingaleBoundedVariation}Let $M$ be a continuous local martingale with locally bounded variation then $M = M_0$ a.s.
\end{lem}
\begin{proof}
We first reduce to the case in which $M$ is a martingale with locally bounded variation and $M_0=0$.  Let $\tau_n$ be a localizing sequence for $M$ then if we can show that $M_{\tau_n \wedge t} - M_0=0$ a.s. for all $n \geq 0$ and $t \geq 0$ then as $\tau_n \to \infty$ we can conclude that $M_t = M_0$ a.s. for all $t \geq 0$.

Next note that since $M$ is locally of bounded variation we have optional times $\tau_n$ such that $\tau_n \uparrow \infty$ such that $M^{\tau_n}$ is of bounded variation.  This implies that $M$ is of bounded variation on every interval $[0,t]$.  Therefore we can define the total variation process $V_t = TV_0^t(M)$.  Since $M$ is continuous, $V_t$ is continuous (Lemma \ref{ContinuityOfTotalVariation}) and by defintion of total variation it is clear that $V_t$ is $\mathcal{F}$-adapted.  Now define $\sigma_n = \inf \lbrace t \geq 0 \mid V_t = n \rbrace$; we know by continuity of $V_t$ that $\sigma_n$ is an optional time (Lemma \ref{HittingTimesContinuous}) and that $M_{\sigma_n \wedge t}$ is a continuous martingale.  Since $M$ is of locally finite variation we know that $\sigma_n \uparrow \infty$ and as before if we can show that $M_{\sigma_n \wedge t}=0$ a.s. for all $n \geq 0$ and $t \geq 0$ then it will follow that $M_t = 0$ for all $t \geq 0$.

Now we have reduced to the case in which $M$ is a continuous martingale with $M_0=0$ and bounded variation.  So fix $t > 0$ and define the partition $t_{n,k} = kt/n$ for all $n>0$ and $k=0, 1, \dotsc, n$.  If we define
\begin{align*}
\zeta_n &= \sum_{k=1}^n \left(M_{t_{n,k}} - M_{t_{n,k-1}}\right)^2 \leq V_t \max_{1 \leq k \leq n} \abs{M_{t_{n,k}} - M_{t_{n,k-1}}}
\end{align*}
then using the continuity of $M$ we know that $M$ is uniformly continuous on $[0,t]$ and therefore we have $\lim_{n \to \infty} \zeta_n = 0$ a.s.  Moreover we have
\begin{align*}
\zeta_n &\leq \sum_{k=1}^n \sum_{j=1}^n \abs{M_{t_{n,k}} - M_{t_{n,k-1}}} \abs{M_{t_{n,j}} - M_{t_{n,j-1}}} = V_t^2
\end{align*}
Since $V_t$ is bounded we can apply Dominated Convergence, the martingale property of $M_t$ and the fact that $M_0=0$ to conclude
\begin{align*}
0 &= \lim_{n \to \infty} \expectation{\zeta_n} = \sum_{k=1}^n \expectation{M_{t_{n,k}}^2} - \expectation{M_{t_{n,k-1}}^2} = \expectation{M_t^2}
\end{align*}
and from this we conclude that $M_t = 0$ a.s.  Taking the union of a countable number of sets of probability zero we see that almost surely $M_q = 0$ for all $q \in \rationals_+$.  Since $M_t$ is continuous we conclude that almost surely $M_t = 0$ for all $t \in \reals_+$.
\end{proof}

\subsection{Stieltjes Integrals}

There are a few simple facts about Stieltjes integrals that we want to describe in the stochastic setting as they will play a part in the general theory of stochastic integration.  First we record the formula for the restriction of a Lebesgue-Stieltjes measure to an interval.
\begin{lem}\label{RestrictionOfLebesgueStieltjesMeasure}Let $F$ be a right continuous function of bounded variation on $[a,b]$, let $[c,d] \subset [a,b]$.  If we let $\mu_F$ denote the signed Lebesgue-Stieltjes measure associated with $F$ and we let
\begin{align*}
F^{[c,d]}(s) &=
F((s \wedge d) \vee c) = 
\begin{cases}
F(c) & \text{if $s < c$}\\
F(s) & \text{if $c \leq s \leq d$} \\
F(d) & \text{if $d < s$} \\
\end{cases}
\end{align*}
then $F^{[c,d]}$ is right continuous of bounded variation on $[a,b]$ and $\mu_F \mid_{[c,d]} = \mu_{F^{[c,d]}}$.  
\end{lem}
\begin{proof}
First suppose that $F$ is non-decreasing and right continuous.  It is elementary that $F^{[c,d]}$ is also non-decreasing and right continuous.  For any half open interval $(x,y] \subset [a,b]$ we have 
\begin{align*}
\mu_F\mid_{[c,d]}((x,y]) &= \mu_F ([c,d] \cap (x,y]) = \mu_F ((d \wedge x) \vee c, (d \wedge y) \vee c]) \\
&= F((d \wedge y) \vee c) - F((d \wedge x) \vee c) = F^{[c,d]}(y) - F^{[c,d]}(x) = \mu_{F^{[c,d]}}((x,y])
\end{align*}
and as we know that $\mu_F \mid_{[c,d]}$ is locally finite, by Lemma \ref{LebesgueStieltjesMeasure} we get $\mu_F\mid_{[c,d]} = \mu_{F^{[c,d]}}$.

In the case of $F$ is right continuous  of bounded variation, then if we write $F = F_+ - F_-$ as a difference of right continuous non-decreasing functions then it is also true $F^{[c,d]} = F^{[c,d]}_+ - F^{[c,d]}_-$ and clearly each $F^{[c,d]}_\pm$ is non-descreasing which show us that $F^{[c,d]}$ is of bounded variation.  Moreover, using the result for non-descreasing functions
\begin{align*}
\mu_F\mid_{[c,d]} &= \mu_{F_+}\mid_{[c,d]} - \mu_{F_-} \mid_{[c,d]} = \mu_{F^{[c,d]}_+} - \mu_{F^{[c,d]}_-} = \mu_{F^{[c,d]}}
\end{align*}
and we are done.
\end{proof}

The simplest type of stochastic integral arises for a process that has right continuous paths with locally finite variation.  In this case, we can just apply the ordinary theory of Lebesgue-Stieltjes integrals pointwise to the process.  
\begin{defn}Let $F$ be an cadlag adapted process and locally finite variation and let $V$ be a jointly measurable process then we define a new process $\int V_s \, dF_s$ by
\begin{align*}
\left(\int V_s \, dF_s\right)_t(\omega) &= \int_0^t V_s(\omega) \, dF(\omega)_s && \text{for all $t \geq 0$ and $\omega \in \Omega$}
\end{align*}
We usually write $\left(\int V_s \, dF_s\right)_t = \int_0^t V_s \, dF_s$.
\end{defn}

The fact that the integral defined as above is actually a process requires verification.  In addition we show that when $V$ is progressive then the resulting process is adapted.
\begin{lem}\label{StochasticStieltjesIntegral}If $F$ is a cadlag process of locally finite variation (not necessarily adapted) and $V$ is a jointly measurable process then $\int_0^t V_s \, dF_s$ is a cadlag process of locally finite variation.  If in addition $F$ is $\mathcal{F}$-adapted and $V$ is $\mathcal{F}$ - progressively measurable then $\int_0^t V_s \, dF_s$ is $\mathcal{F}$-adapted.
\end{lem}
\begin{proof}
If we denote by $\mu_F$ the signed Lebesgue-Stieltjes measure constructed from $F$ and let $\cup_{j=1}^n (a_j, b_j]$ be a disjoint union of intervals, then we have by finite additivity $\mu_F(\cup_{j=1}^n (a_j, b_j]) = \sum_{j=1}^n (F(b_j) - F(a_j))$ which measurable by the measurability of $F$.  As the set of disjoint unions of half open intervals is a ring (Example \ref{RingOfDisjointUnionHalfOpenIntervals}) and therefore a $\pi$-system that generates the Borel $\sigma$-algebra we know $\mu_F$ is a kernel by monotone classes (specifically Lemma \ref{KernelMeasurability}).  If $V$ is jointly measurable then the same is true of $\characteristic{[0,t]} V$ for every $t \geq 0$ and therefore $\int_0^t V_s \, dF_s$ is measurable by Lemma \ref{KernelTensorProductMeasurability}.  The fact that $\int_0^t V_s \, dF_s$ is cadlag and has locally finite variation follow pointwise from Corollary \ref{StieltjesIntegralBoundedVariationAndContinuous}.

Note also that for any $t \geq 0$ we have by Lemma \ref{ChainRuleDensity} and Lemma \ref{RestrictionOfLebesgueStieltjesMeasure} 
\begin{align*}
\int_0^t V_s \, dF_s &= \int_0^\infty \characteristic{[0,t]} V_s \, dF_s = \int_0^\infty V^t_s \, dF\mid_{[0,t]}(s) = \int_0^\infty V^t_s \, dF^t_s  
\end{align*} 
where $F^t(s) = F(t \wedge s)$ and $V^t_s = V_{t \wedge s}$. If we assume that $F$ is adapted it follows that $F_s^t$ is $\mathcal{F}_t$ measurable for all $s \geq 0$ and by the argument above we see that $\mu_{F^t}$ is an $\mathcal{F}_t$-measurable kernel.  If $V$ is progressive then by writing $V^t(\omega, s) = V \mid_{\Omega \times [0,t]}(\omega, s \wedge t)$ which shows that $V^t$ is $\mathcal{F}_t \otimes \mathcal{B}([0,\infty))$-measurable.  Now applying Lemma \ref{KernelTensorProductMeasurability} we get $\mathcal{F}_t$-measurability of $\int_0^t V_s \, dF_s$.
\end{proof}

Because of the previous result we make the following definition for the space of integrands that we'll initially concern ourselves with.
\begin{defn}If $F$ is a cadlag process of locally finite variation then let $L(F)$ be the space of progressive processes $V$ that are pointwise integrable with respect to $F$.
\end{defn}

Because we use stochastic Stieltjes integrals in defining general stochastic integrals we record the following simple facts.  Both of these facts have analogues for general stochastic integrals as well.
\begin{lem}\label{ChainRuleStieltjes}Let $F$ be a cadlag process of locally finite variation, let $V \in L(F)$ and let $U$ be a progressive process.  $U \in L(\int V \, dF)$ if and only if $UV \in L(F)$ and moreover
\begin{align*}
\int_0^t U_s V_s \, dF_s &= \int_0^t U_s \, d\int V_s \, dF_s
\end{align*}
\end{lem}
\begin{proof}
Initially assume that $U$ and $V$ are both positive.  Note that by definition of the Lebesgue-Stieltjes measure we have pointwise for any finite interval $(a,b]$,
\begin{align*}
\mu_{\int V_s \, dF_s}((a,b]) &= \int_0^b V_s \, dF_s - \int_0^a V_s \, dF_s = \int_0^\infty \characteristic{(a,b]} V_s \, dF_s
\end{align*}
and therefore we have $\mu_{\int V_s \, dF_s} = V \cdot \mu_F$ (i.e. $V$ is a $\mu_F$-density of $\mu_{\int V_s \, dF_s}$); the result now follows from Lemma \ref{ChainRuleDensity}.  The rest of the result follows from writing $U = U_+ - U_-$ and $V = V_+ - V_-$ and using linearity.
\end{proof}

We also want to record the behavior of a stochastic Stieltjes integral under stopping.
\begin{lem}\label{StoppingStieltjes}Let $F$ be a cadlag process of locally finite variation, let $V \in L(F)$ and let $\tau$ be an optional time then
\begin{align*}
\int_0^{t \wedge \tau} V_s \, dF_s &= \int_0^t \characteristic{[0,\tau]} V_s \, dF_s = \int_0^t V_s \, F^{\tau}_s
\end{align*}
\end{lem}
\begin{proof}
This follows immediately by writing $\int_0^\infty \characteristic{[0,t]} \characteristic{[0,\tau]} V_s \, dF_s$ and pointwise using the fact that $\mu_F \mid_{[0, \tau]} = \mu_{F^\tau}$ (Lemma \ref{RestrictionOfLebesgueStieltjesMeasure}).
\end{proof}

\subsection{Stochastic Integrals}

The process of defining stochastic integrals follows the standard path of defining integrals for a subclass of integrands for which the definition and existence of the associated integral is easy to see.  Then one uses approximations to extend the class of integrands.  We begin by defining that initial subclass of integrands and define integrals of them with respect to an arbitrary martingale.
\begin{defn}Let $\tau_1 \leq \tau_2 \leq \dotsb$ be optional times, let $\xi_1, \xi_2, \dotsc$ be bounded random variables and assume $\xi_k$ is  $\mathcal{F}_{\tau_k}$-measurable.  Then we say that 
\begin{align*}
V_t &= \sum_{k=1}^\infty \xi_k \characteristic{\tau_{k} > t}
\end{align*}
is a \emph{predictable step process}.  Given a predictable step process and a process $M$ we define the \emph{elementary stochastic integral}
\begin{align*}
\int_0^t V \, dM &= \sum_{k=1}^\infty \xi_k \left (M_t - M_{\tau_k \wedge t} \right)
\end{align*}
In case $\tau_n = \tau_{n+1} = \dotsb$ and $\xi_n = \xi_{n+1} = \dotsb$ we say that $V$ is a \emph{finite predictable step process}.
\end{defn}
Note that in the definition of a stochastic integral for a predictable step process there is no need to consider convergence questions since for each $t \geq 0$ the sum that defines the integral has only finitely many non-zero terms.

TODO: The definition of the elementary stochastic integral isn't quite justified as we haven't shown that it only depends on $V$ and not a particular representation of $V_t = \sum_{k=1}^\infty \xi_k \characteristic{\tau_{k} > t}$.  To show this it seems like it would be helpful to have a canonical representation for a predictable step process.  At some point we also may need the fact that the space of such processes (at least the finite linear combinations) is a vector space or algebra (as per Rogers and Williams).

If one defines the vector space spanned by $\xi \characteristic{(\sigma, \tau]}$ then there is a standard (but not unique) form $\sum_{j=1}^n \xi_j \characteristic{(\sigma_j, \tau_j]}$ where $\sigma_j$ and $\tau_j$ are optional times satisfying $\sigma_1 \leq \tau_1 \leq \sigma_2 \leq \tau_2 \leq \dotsb \leq \sigma_n \leq \tau_n$.  To see this we first need a simple preliminary fact.  If $\sigma$ and $\tau$ are optional times and $\xi$ is either $\mathcal{F}_\sigma$-measurable then $\xi \characteristic{\sigma < \tau}$ is $\mathcal{F}_{\sigma \wedge \tau}$-measurable.  This follows from noting that for all $t \in \reals$, 
\begin{align*}
\lbrace \xi \characteristic{\sigma < \tau} \leq t \rbrace = 
\begin{cases}
\lbrace \sigma \geq \tau \rbrace \cup ( \lbrace \xi \leq t \rbrace \cap \lbrace \sigma < \tau \rbrace) & \text{if $t \geq 0$} \\
\lbrace \xi \leq t \rbrace \cap \lbrace \sigma < \tau \rbrace & \text{if $t < 0$} \\
\end{cases}
\end{align*}
and since $\lbrace \sigma \geq \tau \rbrace$ is $\mathcal{F}_{\sigma \wedge \tau}$-measurable it suffices to show that $\lbrace \xi \leq t \rbrace \cap \lbrace \sigma < \tau \rbrace \in \mathcal{F}_{\sigma \wedge \tau}$ for all $t \in \reals$.  Thus pick $s \in \reals$ and using the $\mathcal{F}_\sigma$-measurability of $\xi$ and the $\mathcal{F}_{\sigma \wedge \tau}$-measurability of $\lbrace \sigma < \tau \rbrace$ we get
\begin{align*}
\lbrace \xi \leq t \rbrace \cap \lbrace \sigma < \tau \rbrace \cap \lbrace \sigma \wedge \tau \leq s \rbrace = 
\lbrace \xi \leq t \rbrace \cap \lbrace \sigma \leq s \rbrace \cap \lbrace \sigma < \tau \rbrace \cap \lbrace \sigma \wedge \tau \leq s \rbrace \in \mathcal{F}_s
\end{align*}

Now considering the decomposition of the intersection of two half open intervals in $\reals$ into 3 disjoint parts we see
\begin{align*}
&\xi_1 \characteristic{(\sigma_1, \tau_1]} + \xi_2 \characteristic{(\sigma_2, \tau_2]}  = \\
&(\xi_1 \characteristic{\sigma_1 < \sigma_2} + \xi_2 \characteristic{\sigma_2 < \sigma_1} ) \characteristic{(\sigma_1 \wedge \sigma_2 , (\sigma_1 \vee \sigma_2) \wedge \tau_1 \wedge \tau_2]} + \\
&(\xi_1 + \xi_2) \characteristic{(\sigma_1 \vee \sigma_2, \tau_1 \wedge \tau_2 \vee \sigma_1 \vee \sigma_2]} + \\
&(\xi_1 \characteristic{\tau_1 > \tau_2} + \xi_2 \characteristic{\tau_2 > \tau_1}) \characteristic{(\sigma_1 \vee \sigma_2 \vee (\tau_1 \wedge \tau_2), \tau_1 \vee \tau_2 ]}
\end{align*}
By our claim above get that $\xi_1 \characteristic{\sigma_1 < \sigma_2} + \xi_2 \characteristic{\sigma_2 < \sigma_1}$ is $\mathcal{F}_{\sigma \wedge \tau}$-measurable.  By $\mathcal{F}_{\sigma_1}$-measurability of $\xi_1$ and $\mathcal{F}_{\sigma_2}$-measurability of $\xi_2$  we get $\mathcal{F}_{\sigma_1 \vee \sigma_2}$-measurability of $\xi_1 + \xi_2$.  Lastly we know also that $\lbrace \tau_1 > \tau_2 \rbrace$ and  $\lbrace \tau_2 > \tau_1 \rbrace$ are $\mathcal{F}_{\tau_1 \wedge \tau_2}$-measurable so $\xi_1 \characteristic{\tau_1 > \tau_2} + \xi_2 \characteristic{\tau_2 > \tau_1}$ is $\mathcal{F}_{\sigma_1 \vee \sigma_2 \vee (\tau_1 \wedge \tau_2)}$-measurable.  Moreover it is clear that we have the inequalities
\begin{align*}
\sigma_1 \wedge \sigma_2 &\leq (\sigma_1 \vee \sigma_2) \wedge \tau_1 \wedge \tau_2 \leq 
\sigma_1 \vee \sigma_2 \leq \tau_1 \wedge \tau_2 \vee \sigma_1 \vee \sigma_2 \leq 
\sigma_1 \vee \sigma_2 \vee (\tau_1 \wedge \tau_2) \leq \tau_1 \vee \tau_2
\end{align*}
and therefore the result is shown.

The representation for a predictable step process we have given in the definition is occasionally not the most convenient one.  Given $V_t = \sum_{k=1}^\infty \xi_k \characteristic{\tau_{k} > t}$ if we define $\eta_n = \sum_{k=1}^n \xi_k$ and therefore 
\begin{align*}
V_t &= \sum_{k=1}^\infty \xi_k \characteristic{t > \tau_k} = \sum_{k=1}^\infty \xi_k \sum_{j=k}^\infty \characteristic{(\tau_j, \tau_{j+1}]}(t)  \\
&= \sum_{j=1}^\infty \sum_{k=1}^j \xi_k \characteristic{(\tau_j, \tau_{j+1}]}(t)  = \sum_{j=1}^\infty \eta_j \characteristic{(\tau_j, \tau_{j+1}]}(t)  \\
\end{align*}
and 
\begin{align*}
\int_0^t V \, dM &= \sum_{k=1}^\infty \xi_k \left(M_{t} - M_{\tau_k \wedge t} \right) = \sum_{k=1}^\infty \xi_k \sum_{j=k}^\infty \left(M_{\tau_{j+1} \wedge t} - M_{\tau_j \wedge t} \right)  \\
&= \sum_{j=1}^\infty \sum_{k=1}^j \xi_k \left(M_{\tau_{j+1} \wedge t} - M_{\tau_j \wedge t} \right) = \sum_{j=1}^\infty \eta_j \left(M_{\tau_{j+1} \wedge t} - M_{\tau_j \wedge t} \right)
\end{align*}
In what follows we will feel free to switch between these representations without comment.

The first order of business is to establish conditions under which an elementary stochastic integral is a martingale.  To do this we need the following characterization of the martingale property.  
\begin{lem}\label{MartingaleOptionalTimeCriterion}Let $M_t$ be an integrable adapted process on an index set $T$.  Then $M$ is a martingale if and only if $\expectation{M_\sigma} = \expectation{M_\tau}$ for all $T$-valued optional times $\sigma$ and $\tau$ that take at most two values.
\end{lem}
\begin{proof}
Restricting $M_t$ to the union of the ranges of $\tau$ and $\sigma$ we can apply Lemma \ref{ExpectationStoppedMartingaleDiscrete} to conclude $\expectation{M_\sigma} = M_0 = \expectation{M_\tau}$.  In the other direction, let $s,t \in T$ with $s < t$.  Let $A \in \mathcal{F}_s$ and define $\sigma = s \characteristic{A^c} + t \characteristic{A}$ and note that $\sigma$ is an optional time.  Now, applying our hypothesis to the optional time $\sigma$ and the deterministic optional time $s$,  we get $\expectation{M_t ; A} = \expectation{M_\sigma} - \expectation{M_s; A^c}  = \expectation{M_s} - \expectation{M_s; A^c}  = \expectation{M_s; A}$ which shows $\cexpectationlong{\mathcal{F}_s}{M_t} = M_s$ a.s.
\end{proof}

\begin{lem}\label{StochasticIntegralFinitePredictableStepProcess}Suppose $\mathcal{F}$ is a filtration,  $\tau_1 \leq \tau_2 \leq \dotsb \leq \tau_n$ are bounded $\mathcal{F}$-optional times, $M_t$ is a martingale
and either 
\begin{itemize}
\item[(i)] each $\tau_k$ is countably valued
\item[(ii)]$\mathcal{F}$ and $M$ are right continuous
\end{itemize}
Then if
\begin{align*}
V_t &= \sum_{k=1}^n \xi_k \characteristic{\tau_{k} > t}
\end{align*}
is a finite predictable step process then $\int_0^t V \, dM$ is a martingale.  If we assume that $M$ is a local martingale then $\int_0^t V \, dM$ is a local martingale.
\end{lem}
\begin{proof}
By definition of elementary stochastic integral and linearity, it suffices to show that $N_t = \xi \left (M_t - M_{\tau \wedge t} \right)$ is a martingale whenever either $\tau$ is a countably valued optional time or $\mathcal{F}$ and $M$ are right continuous and $\xi$ is a bounded $\mathcal{F}_\tau$-measurable random variable.  In the first case, by restricting $M_t$ to the range of $\tau$ we can apply the Optional Stopping Theorem \ref{OptionalStoppingDiscrete} to the bounded optional time $\tau \wedge t$ to conclude that $M_{\tau \wedge t}$ is integrable and in the second case we can apply the continuous time Optional Stopping Theorem \ref{OptionalStoppingContinuous} to conclude that $M_{\tau \wedge t}$ is integrable.  This together with the integrability of $M_t$ and boundedness of $\xi$ shows that $N_t$ is integrable.  If we note that $N_t = \xi \characteristic{ \tau \leq t} \left (M_t - M_{\tau \wedge t} \right)$ then because $ \xi \characteristic{ \tau \leq t}$ and $M_t$ are $\mathcal{F}_t$-measurable and $M_{\tau \wedge t}$ is $\mathcal{F}_{\tau \wedge t}$-measurable (hence $\mathcal{F}_t$-measurable) we see that $N_t$ is adapted.  Lastly let $\sigma$ be an countably valued optional time then by the $\mathcal{F}_\tau$-measurability of $\xi$ we have and either the Optional Stopping Theorem \ref{OptionalStoppingDiscrete} or the Optional Stopping Theorem \ref{OptionalStoppingContinuous} we get
\begin{align*}
\cexpectationlong{\mathcal{F}_\tau}{N_\sigma} &= \xi \cexpectationlong{\mathcal{F}_\tau}{M_\sigma - M_{\tau \wedge \sigma}} = \xi \left(M_{\tau \wedge \sigma} - M_{\tau \wedge \sigma} \right) = 0
\end{align*}
and by the tower property of conditional expectations we get $\expectation{N_\sigma} = 0$.  Now by Lemma \ref{MartingaleOptionalTimeCriterion} we see that $N_t$ is a martingale.

Now let us assume that $M$ is a local martingale.  To see that $\int_0^t V \, dM$ is a local martingale let $\sigma_n$ be a localizing sequence and note that
\begin{align*}
\left( \int_0^t V \, dM\right)^{\sigma_n} &= \sum_{k=1}^n \xi_k \left( M_{\sigma_n \wedge t} - M_{\sigma_n \wedge\tau_k \wedge t} \right) = \int_0^t V \, dM^{\sigma_n}
\end{align*}
is a martingale with localizing sequence $\sigma_n$  by the first part of the Lemma.
\end{proof}

\begin{lem}\label{StochasticIntegralPredictableStepProcess}Suppose $\mathcal{F}$ is a filtration,  $\tau_1 \leq \tau_2 \leq \dotsb \leq \dotsb$ are bounded $\mathcal{F}$-optional times with $\tau_n \uparrow \infty$, $M_t$ is an $L^2$ martingale with $M_0$, $V_t$ is a predictable step process with $\abs{V_t} \leq 1$ and either
\begin{itemize}
\item[(i)] each $\tau_k$ is countably valued
\item[(ii)]$\mathcal{F}$ and $M$ are right continuous
\end{itemize}
then $\int_0^t V \, dM$ is an $L^2$-martingale and $\expectation{\left(\int_0^t V \, dM\right)^2} \leq \expectation{M^2_t}$.
\end{lem}
\begin{proof}
We let $V_t = \sum_{k=1}^\infty \eta_k \characteristic{(\tau_k, \tau_{k+1}]}$.  We start by taking an arbitrary $n >0$ and defining $V^n_t =  \sum_{k=1}^n \eta_k \characteristic{(\tau_k, \tau_{k+1}]}$ so that $V^n$ is a finite predictable step process.  By Lemma \ref{StochasticIntegralFinitePredictableStepProcess} shows that $\int_0^t V^n \, dM$ is a martingale.  The $L^2$ bound for $V^n$ follows from Optional Stopping (Theorem \ref{OptionalStoppingDiscrete} or Theorem \ref{OptionalStoppingContinuous} depending on which hypothesis we choose).  The critical point is that for any $1 \leq k < j \leq n$ we have for each cross term term of the stochastic integral
\begin{align*}
&\expectation{\eta_j \eta_k \left (M_{\tau_{j+1} \wedge t} - M_{\tau_j \wedge t} \right)  \left (M_{\tau_{k+1} \wedge t} - M_{\tau_k \wedge t} \right)} \\
&=\expectation{\eta_j \eta_k \left (M_{\tau_{j+1} \wedge t} - M_{\tau_j} \right)  \left (M_{\tau_{k+1} } - M_{\tau_k} \right) ; t > \tau_j} \\
&=\expectation{ \eta_j \eta_k \cexpectationlong{\mathcal{F}_{\tau_j}}{M_{\tau_{j+1} \wedge t} - M_{\tau_j }}  \left(M_{\tau_{k+1} } - M_{\tau_k} \right) ; t > \tau_j} =0
\end{align*}
and 
\begin{align*}
\expectation{ \left (M_{\tau_{k+1} \wedge t} - M_{\tau_k \wedge t} \right)^2} &= \expectation{M^2_{\tau_{k+1} \wedge t}} - 2\expectation{M_{\tau_{k+1} \wedge t}M_{\tau_{k} \wedge t}} + 
\expectation{M^2_{\tau_k \wedge t}} \\
&=\expectation{M^2_{\tau_{k+1} \wedge t}} - 2\expectation{\cexpectationlong{\mathcal{F}_{\tau_k \wedge t}}{M_{\tau_{k+1} \wedge t}}M_{\tau_{k} \wedge t}} + 
\expectation{M^2_{\tau_k \wedge t}} \\
&= \expectation{M^2_{\tau_{k+1} \wedge t}} - \expectation{M^2_{\tau_k \wedge t}}
\end{align*}
Using the above facts, the fact that $M_0 = 0$ and the bound on $V$ 
\begin{align*}
\expectationop \left( \int_0^t V^n \, dM \right)^2 &= \expectationop \sum_{k=1}^n \eta^2_k \left (M_{\tau_{k+1} \wedge t} - M_{\tau_k \wedge t} \right)^2 \\
&\leq \expectationop \sum_{k=1}^n \left (M_{\tau_{k+1} \wedge t} - M_{\tau_k \wedge t} \right)^2 \\
&= \sum_{k=1}^n \expectation{M^2_{\tau_{k+1} \wedge t}} - \expectation{M^2_{\tau_k \wedge t}} \\
&\leq \sum_{k=1}^\infty \expectation{M^2_{\tau_{k+1} \wedge t}} - \expectation{M^2_{\tau_k \wedge t}} \\
&=  \lim_{n \to \infty} \expectation{M^2_{\tau_n \wedge t}} = \expectation{M^2_t}
\end{align*}

Now in the general case, we get the $L^2$ bound by Fatou's Lemma Theorem \ref{Fatou}
\begin{align*}
\expectationop \left( \int_0^t V \, dM \right)^2 &= \liminf_{n \to \infty} \expectationop  \left( \int_0^t V^n \, dM \right)^2 \leq \expectation{M^2_t}
\end{align*}
In addition, the $L^2$ bound shows that the family $\int_0^t V \, dM, \int_0^t V^1 \, dM, \int_0^t V^2 \, dM, \dotsc$ is uniformly integrable (Lemma \ref{BoundedLpImpliesUniformlyIntegrable}) and therefore for every $t \geq 0$ and the martingale property of $\int_0^t V^n \, dM$ we get for $u < t$,
\begin{align*}
\cexpectationlong{\mathcal{F}_u}{\int_0^t V \, dM} &= \cexpectationlong{\mathcal{F}_u}{\lim_{n \to \infty} \int_0^t V^n \, dM} \\
&= \lim_{n \to \infty} \cexpectationlong{\mathcal{F}_u}{ \int_0^t V^n \, dM} \\
&= \lim_{n \to \infty} \int_0^u V^n \, dM = \int_0^u V \, dM 
\end{align*}
(to exchange the limits with the conditional expectation, use the fact that for each $A \in \mathcal{F}_u$ we can see that $\characteristic{A} \int_0^t V^n \, dM$ is uniformly integrable then use Theorem \ref{LpConvergenceUniformIntegrability})
showing $\int_0^t V \, dM$ is an $L^2$-martingale.
\end{proof}

\begin{defn}Fix a probability space $(\Omega, \mathcal{A}, P)$ and suppose $\mathcal{F}$ is a right continuous and complete filtration.  Let $\mathcal{M}^2$ be the space of $L^2$ bounded continuous $\mathcal{F}$-martingales such that $M_0 = 0$ a.s.  That is to say that for all $M \in \mathcal{M}^2$ there exists $C \geq 0$ such that for all $0 \leq t < \infty$ we have $\norm{M_t}_2 \leq C$.  For $M, N \in \mathcal{M}^2$, define $\langle M, N \rangle = \langle M_\infty, N_\infty \rangle = \expectation{M_\infty N_\infty }$.
\end{defn}
\begin{lem}\label{ContinuousL2MartingalesHilbert}The space $\mathcal{M}^2$ is a Hilbert space.
\end{lem}
\begin{proof}
The fact that $\mathcal{M}^2$ is a vector space follows immediately from linearity of conditional expectation, the linearity of the space $C([0,\infty); \reals)$ and the triangle inequality of the $L^2$ norm on $C([0,\infty); \reals)$. 
 
To  see that we have an inner product on $\mathcal{M}^2$, first observe that if $\langle M, M \rangle = \norm{M_\infty}^2_2 = 0$ then $M_\infty = 0$ a.s. hence since $M$ is closable it follows that $M_t = \cexpectationlong{\mathcal{F}_t}{M_\infty} = 0$ a.s.  for all $0 \leq t < \infty$.  Since $M$ is continuous it follows that $M=0$ a.s.  Symmetry of $\langle \cdot,\cdot \rangle$ follows immediately from symmetry of the $L^2$ inner product on $C([0,\infty); \reals)$.  Supposing $M$ and $N$ are both $L^2$ bounded continuous martingales we know they are closable hence $M_t = \cexpectationlong{\mathcal{F}_t}{M_\infty}$ and similarly for $N$.  It then follows from linearity of conditional expectation that $(aM + bN)_\infty = aM_\infty + bN_\infty$ for any $a,b \in \reals$.  From this fact we see that  for all $M,N,R \in \mathcal{M}^2$ and $a,b \in \reals$ we have $\langle aM + bN, R \rangle = \langle aM_\infty + bN_\infty, R_\infty \rangle = a \langle M, R \rangle + b \langle N, R \rangle$. 

We now show that $\mathcal{M}^2$ is complete.  Suppose $M^1, M^2, \dotsc$ is Cauchy in $\mathcal{M}^2$, then $M^1_\infty, M^2_\infty, \dotsc$ is Cauchy in $L^2$ and therefore has a limit $\xi$ in $L^2$ that is $\mathcal{F}_\infty$-measurable.  Define $M_t = \cexpectationlong{\mathcal{F}_t}{\xi}$ so we know that $M_t$ is a martingale and $M_\infty = \xi$ a.s.  TODO: Do we know at this point that $M$ is $L^2$ bounded?  Now by the Doob $L^2$ inequality Lemma \ref{DoobLpInequalityContinuous} applied to the closed martingale $M_t$ on $[0,\infty]$ we have $\norm{\sup_{0 \leq s \leq \infty} (M_s^n - M_s)}_2 \leq 2 \norm{M^n_\infty - M_\infty}_2$.  From this we get that $\lim_{n \to \infty} \norm{\sup_{0 \leq s \leq \infty} (M_s^n - M_s)}_2 = 0$ hence $\sup_{0 \leq s \leq \infty} (M_s^n - M_s) \toprob 0$ (Lemma \ref{ConvergenceInMeanImpliesInProbability}) and therefore $\sup_{0 \leq s \leq \infty} (M_s^n - M_s) \toas 0$ along a subsequence (Lemma \ref{ConvergenceInProbabilityAlmostSureSubsequence}) which shows that $M$ is has almost surely continuous sample paths (Lemma \ref{UniformLimitContinuousFunctionsIsContinuous}).
\end{proof}

\subsection{Quadratic Variation}
The crux of the problem in defining stochastic integrals is the fact that sample paths of continuous martingales almost surely have infinite total variation and therefore Lebesgue-Stieltjes integrals cannot be defined.   

\begin{lem}Let $M$ and $N$ be continuous local martingales and let $\tau$ be an optional time, then $M^\tau \left( N - N^\tau \right)$ is a local martingale.
\end{lem}
\begin{proof}
First let us assume that $N$ is a martingale, $\tau$ is an optional time and $\eta$ is an $\mathcal{F}_\tau$-measurable bounded random variable.  We claim that $\eta(N_t - N^\tau_t)$ is a martingale.  By the Optional Stopping Theorem \ref{OptionalStoppingContinuous} we know that $\tau \wedge t$ is a bounded optional time hence $N_{\tau \wedge t}$ is integrable and therefore by boundedness of $\eta$ we know that $\eta(N_t - N_{\tau \wedge t})$ is integrable.  To see adaptedness, note that $\eta(N_t - N_{\tau \wedge t}) = \eta \characteristic{\tau \leq t} (N_t - N_{\tau \wedge t})$ so that by $\mathcal{F}_\tau$-measurability of $\eta$ we also have $\mathcal{F}_t$-measurability of $\eta \characteristic{\tau \leq t}$.  Furthermore, $N_{\tau \wedge t}$ is $\mathcal{F}_{\tau \wedge t}$-measurable and since $\tau \wedge t \leq t$ we see that it is also $\mathcal{F}_t$-measurable.  To see that $\eta (N_t - N_{\tau \wedge t})$ is a martingale we let $\sigma$ be any bounded optional time and then note by Optional Sampling, the tower and pullout properties of conditional expectation 
\begin{align*}
\expectation{\eta (N_\sigma - N_{\tau \wedge \sigma})} &= \expectation{\eta \cexpectationlong{\mathcal{F}_\tau}{ (N_\sigma - N_{\tau \wedge \sigma})}} = \expectation{\eta  (N_{\tau \wedge \sigma} - N_{\tau \wedge \sigma})} = 0
\end{align*}
which independent of $\sigma$ hence we can apply Lemma \ref{MartingaleOptionalTimeCriterion} to conclude that $\eta(N - N^\tau)$ is a martingale.

TODO: Finish
\end{proof}

TODO: We used continuity of the local martingale $M$ to reduce ourselves to the case of bounded martingales which was used to obtain integrability.  Do general local martingales localize to $L^2$ bounded martingales or something else that would allow us to get integrability?

\begin{thm}[Quadratic Covariation]\label{OptionalQuadraticCovariation}Let $M$ and $N$ be continuous local martingales, there exists an almost surely unique continuous process $[M,N]$ of locally finite variation such that $[M,N]_0 = 0$ and $MN - [M,N]$ is a local martingale.  The pairing $[M,N]$ is bilinear and symmetric and for every optional time $\tau$ satisfies
\begin{align*}
[M,N]^\tau = [M^\tau, N^\tau] = [M^\tau, N] \text{ a.s.}
\end{align*}
If we define $[M] = [M,M]$ then it is the case that $[M]$ is almost surely non-decreasing.  The process $[M,N]$ is called the \emph{quadratic covariation} of $M$ and $N$ and the process $[M]$ is called the \emph{quadratic variation} of $M$.
\end{thm}
\begin{proof}
We first consider the case when $M=N$ and we first assume that $M$ is a bounded martingale such that $M_0=0$ and $\abs{M_t} \leq C$ for some deterministic constant $C > 0$.  To motivate the construction recall the basic fact that for a function $f$ of bounded variation we have the Lebesgue-Stieltjes integral $f^2 = 2 \int f \, df$.  We suspect that in a stochastic setting such an identity won't quite work (because the Stieltjes integral doesn't work).  That suspicion is correct and what does turn out to be true is that once we have defined a stochastic integral, $M^2 - 2 \int M \, dM = [M]$.  Of course our plan is to use the quadratic variation to define stochastic integrals so this reasoning is getting pretty circular here; nonetheless if we suspend belief for moment and define something that \emph{looks like} it could be $\int M \, dM$ then we might get the right definition for quadratic variation.  Motivated by these observations, our first step is to come up with an approximation of $M$ by predictable step processes so we can create an approximation of $\int M \, dM$.  For each $n > 0$ define the sequence of optional times $\tau^n_0, \tau^n_1, \dotsc$ by $\tau^n_0 = 0$ and 
\begin{align*} 
\tau^n_k &= \inf \lbrace t > \tau^n_{k-1} \mid \abs{M_t - M_{\tau^n_{k-1}}} = 2^{-n} \rbrace \text{ for $k > 0$} \\
&=\tau^n_{k-1} + \tau^n_1 \circ \theta_{\tau^n_{k-1}}
\end{align*}

Claim: Either $\tau^n_k = \infty$ or $M_{\tau^n_k} = j/2^n$ for some random $j \in \integers$.

This is a simple induction on $k$ for each $n$ using continuity of $M_t$, the fact that $M_0 = 0$ and $\tau^n_0=0$.

Claim: Suppose $M_t = j/2^n$ for some $n \geq 0$ and $k \geq 0$ and let $K = \max \lbrace k \mid \tau^n_k \leq t$, then $M_{\tau^n_K} = j/2^n$.

The claim is trivially true if $\tau^n_K = t$.  If this is not true by the previous claim we have $i \in \integers$ such that $M_{\tau^n_k} = i/2^n$.  By definition of $K$, we have $\tau^n_K < t < \tau^n_{K+1}$ and by definition of $\tau^n_{K+1}$, continuity of $M_t$ and the intermediate value theorem we know that $\abs{M_t - M_{\tau^n_K}} = \abs{i-j}2^{-n}< 2^{-n}$.  Thus $i=j$ and the claim is verified.

Claim: For every $n \geq 0$ and $k \geq 0$ there exists a random integer $l \geq 0$ such that $\tau^n_k = \tau^{n+1}_l$.

We proceed by induction on $k$ with the case $k=0$ being true because $\tau^n_0 = 0$ for all $n \geq 0$.  Having assumed $M_0 = 0$ we see that we can pick $0 \leq j < \infty$ such that $M_{\tau^n_k} = j/2^n$. (TODO: what to say when $\tau^n_k = \infty$; not clear that we can assert some $\tau^{n+1}_l=\infty$ since we may be oscillating with small enough amplitude?)  Let $i$ be the largest index such that $\tau^{n+1}_i \leq \tau^n_k$.  Since $M_{\tau^n_k} = j/2^n = 2j/2^{n+1}$ we can apply the previous claim to see that $M_{\tau^{n+1}_i}  = M_{\tau^n_k}$.  By the intermediate value theorem we know that $M_{\tau^n_{k-1}} < M_{\tau^{n+1}_i}  \leq M_{\tau^n_k}$. Because $\abs{M_{\tau^{n+1}_i} - M_{\tau^n_{k-1}}} = \abs{M_{\tau^{n}_k} - M_{\tau^n_{k-1}}} = 2^{-n}$ by definition of $\tau^n_k$ we know that $\tau^{n+1}_i \geq \tau^n_k$ and therefore $\tau^{n+1}_i = \tau^n_k$.

Define 
\begin{align*}
V^n_t &= \sum_{k=0}^\infty M_{\tau^n_{k}} \characteristic{(\tau^n_{k}, \tau^n_{k+1}]}(t)
\end{align*}
where despite the fact that we have written an infinite sum we don't have to worry about convergence since for any fixed $t$ the sum is finite.  Clearly, each $V^n$ is a bounded predictable step process and it is also clear that $V^n$ is an approximation of $M$ (though we won't yet belabor the exact sense in which this is true).  Pick $t \geq 0$ and let $K$ be the random index such that $\tau^n_{K} < t \leq \tau^n_{K+1}$ then we can compute using high school algebra and the fact that $M_{\tau^n_0} = M_0 = 0$
\begin{align*}
2 \int_0^t V^n \, dM &= 2 \sum_{k=0}^\infty M_{\tau^n_{k}} \left ( M_{t \wedge \tau^n_{k+1}} - M_{t \wedge \tau^n_k} \right ) \\
&=2 M_{\tau^n_K} M_t - 2 M_{\tau^n_K}^2 + 2 \sum_{k=0}^{K-1} M_{\tau^n_k} M_{\tau^n_{k+1}} - 2 \sum_{k=0}^{K-1} M^2_{\tau^n_k} \\
&=2 M_{\tau^n_K} M_t -  M_{\tau^n_K}^2 + 2 \sum_{k=0}^{K-1} M_{\tau^n_k} M_{\tau^n_{k+1}} - \sum_{k=0}^{K-1} M^2_{\tau^n_k}  - \sum_{k=0}^{K-1} M^2_{\tau^n_{k+1}} \\
&=2 M_{\tau^n_K} M_t -  M_{\tau^n_K}^2 - \sum_{k=0}^{K-1} \left (M_{\tau^n_{k+1}}  - M_{\tau^n_k}\right)^2 \\
&=M_t^2 - \left( M_t - M_{\tau^n_K} \right)^2 - \sum_{k=0}^{K-1} \left (M_{t \wedge \tau^n_{k+1}}  - M_{t \wedge \tau^n_k}\right)^2 \\
&=M_t^2 - \sum_{k=0}^\infty \left (M_{t \wedge \tau^n_{k+1}}  - M_{t \wedge \tau^n_k}\right)^2 \\
\end{align*}
So if we define 
\begin{align*}
Q^n_t = \sum_{k=0}^\infty \left (M_{t \wedge \tau^n_{k+1}}  - M_{t \wedge \tau^n_k}\right)^2 
\end{align*}
we have the identity
\begin{align*}
M^2_t &= 2 \int_0^t V^n \, dM + Q^n_t
\end{align*}
Since $V^n$ is a bounded predictable step process and $M$ is an $L^2$ continuous martingale, we know that $\int V^n \, dM$ is a continuous $L^2$ martingale (Lemma \ref{StochasticIntegralPredictableStepProcess}).   Furthermore by construction we have $\sup_{0 \leq t < \infty} \abs{V^n_t - M_t} < 2^{-n}$ and therefore $\sup_{0 \leq t < \infty} \abs{V^n_t - V^m_t} < 2^{-n+1}$ for all $n \leq m$.  
\begin{align*}
\norm{\int V^n \, dM - \int V^m \, dM}_2 &=\norm{\int (V^n - V^m )\, dM }_2 \\
&=\norm{\lim_{t \to \infty} \int_0^t (V^n - V^m )\, dM }_2\\
&\leq \lim_{t \to \infty} \norm{\int_0^t (V^n - V^m )\, dM }_2 && \text{ by Fatou's Lemma}\\
&\leq \lim_{t \to \infty} 2^{-n+1} \norm{M_t}_2 && \text{ by Lemma \ref{StochasticIntegralPredictableStepProcess}}\\
&= 2^{-n + 1} \norm{M_\infty}_2 = 2^{-n+1} \norm{M}_2 && \text{ since $M_t \tolp{2} M_\infty$}
\end{align*}
which shows that $\int V^n \, dM_s$ is a Cauchy sequence in $\mathcal{M}^2$.  (TODO: I am almost certain that we know $\int (V^n - V^m) \, dM$ is a bounded $L^2$ martingale so that in fact we have $\int_0^t (V^n - V^m) \, dM \tolp{2} \int_0^\infty (V^n - V^m) \, dM$ and we don't need Fatou above, we have equality).  By completeness of $\mathcal{M}^2$ (Lemma \ref{ContinuousL2MartingalesHilbert}) there is $N \in \mathcal{M}^2$ such that $\int V_s^n \, dM_s$ converges to $N$.  Define $[M] = M^2 - 2N$ and use the Doob $L^2$ inequality $\sup_{0 \leq t \leq \infty} \abs{N_t - \int_0^t V^n\, dM} \leq 2 \norm{N - \int V^n\, dM} \to 0$ to get
\begin{align*}
\sup_{0 \leq t < \infty} \abs{Q^n_t - [M]_t} &=\sup_{0 \leq t < \infty} \abs{Q^n_t - M^2_t + 2N_t} \\
&= 2 \sup_{0 \leq t < \infty} \abs{N_t - \int_0^t V^n \, dM} \toprob 0
\end{align*}
Therefore $\sup_{0 \leq t < \infty} \abs{Q^n_t - [M]_t} \toas 0$ along a subsequence (Lemma \ref{ConvergenceInProbabilityAlmostSureSubsequence}).  Define the random set $T = \lbrace \tau^n_k \mid n,k \in \naturals \rbrace$.  We have shown above that for any two elements $s < t \in T$ for sufficiently large $n$ such that $s = \tau^n_k$ and $t = \tau^n_j$ for appropriate $k,j \in \integers_+$ (where $k$ and $j$ depend on $n$ of course).  From the definition of $Q^n_t$ it follows that $Q^n_s \leq Q^n_t$ for all such $n$; thus $[M]$ is almost surely non-decreasing on $T$.  By continuity of $[M]$ we can extend this to conclude that almost surely $[M]$ is non-decreasing on the closure $\overline{T}$.  To see that $[M]$ is non-decreasing everywhere, we know that $\reals_+ \setminus \overline{T}$ is a countable union of open intervals so it suffices to show that $[M]$ is constant on any open interval $(a,b) \subset \reals_+ \setminus \overline{T}$.  If $[M]$ is not constant on $(a,b)$ then we can find suitable $s,t$ such that $a < s < t < b$ and $X_s = k/2^n$ and $X_t = (k+1)/2^n$ or $X_t = (k-1)/2^n$ for some $k,n \in \integers$.  Pick the largest $i$ such that $\tau^n_i \leq s$.  As $(a,b) \cap \overline{T} = \emptyset$ we know that $\tau^n_i < s$.  By our previous claim we know that $X_{\tau^n_i} = X_s$ and therefore $\abs{X_t - X_{\tau^n_i}} = \abs{X_t - X_s} = 2^{-n}$ which implies $\tau^n_i < s < \tau^n_{i+1} \leq t$ which contradicts $(a,b) \cap \overline{T} = \emptyset$.

Now we need to extend the definition of the quadratic variation to unbounded martingales $M$.  Let $\tau_n = \inf \lbrace t \geq 0 \mid \abs{M_t} = n \rbrace$ which is an optional time by continuity of sample paths of $M$ and Lemma \ref{HittingTimesContinuous}.  By what we have proven, we know that $[M^{\tau_n}]$ exists and is almost surely non-decreasing.
TODO: Finish the result by extending to the unbounded case.

Having defined the quadratic variation $[M]$, we now extend it to the quadratic covariation $[M,N]$ for general local martingales $M$ and $N$.  First we establish the uniqueness.  Note that if we are given processes of locally bounded variation $Q$ and $R$ such that $Q_0 = R_0 = 0$ and $MN - Q$ and $MN - R$ are local martingales, then $Q-R = (MN -R) - (MN -Q)$ is a local martingale of locally bounded variation and Lemma \ref{ContinuousLocalMartingaleBoundedVariation} implies that $Q - R = Q_0 - R_0 = 0$ a.s.  From the uniqueness we immediately see that $[M,N]$ is bilinear and symmetric.

Now we reduce the definition of $[M,N]$ to the case $[M]$ with $M_0 = 0$ by a pair of reductions.

Claim: $[M-M_0,N-N_0] = [M,N]$ a.s.

Simply note that 
\begin{align*}
MN - (M-M_0)(N-N_0) &= M_0 N_0 + M_0 N + N_0 M
\end{align*}
is a local martingale and therefore $(M-M_0)(N-N_0) - [M,N] = -(MN - (M-M_0)(N-N_0) ) + MN - [M,N]$ is a local martingale.

Claim: $[M,N] = \frac{1}{4} ([M+N] - [M-N])$

Note that 
\begin{align*}
4 MN - [M+N] + [M-N] &= ((M+N)^2 - [M+N]) - ((M-N)^2 - [M-N])
\end{align*}
is a local martingale.

Lastly we prove the behavior of localization under optional times.
Claim: Let $\tau$ be an optional time, then $[M,N]^\tau = [M^\tau, N^\tau] = [M^\tau,N]$ a.s.

For the first reduction, suppose that $\tau$ is an optional time then we know that
\begin{align*}
(MN - [M,N])^\tau &= M^\tau N^\tau - [M,N]^\tau
\end{align*}
which is a local martingale by Lemma \ref{LocalMartingaleLocalProperty} and moreover $[M,N]^\tau$ is of locally finite variation therefore we see $[M^\tau,N^\tau] = [M,N]^\tau$ a.s.  We also know that $M^\tau (N^\tau -N)$ is a local martingale (TODO:this is supposed to follow for martingales from optional stopping (doesn't that actually show $M_\tau(N - N^\tau)$ is a martingale) then given a localizing sequence $\tau_n$ for $M$ and $\sigma_n$ for $N$ we know that $\tau_n \wedge \sigma_n$ is a localizing sequence for both $M$ and $N$) and therefore 
\begin{align*}
M^\tau N - [M,N]^\tau &= M^\tau (N - N^\tau)  + M^\tau N^\tau - [M,N]^\tau
\end{align*}
is a local martingale which shows that $[M,N]= [M^\tau,N]$ a.s.

TODO: This proof does not make it clear that when $M$ is a martingale we know $M^2 - [M]$ is in fact a martingale (is that always true?  According to Rogers and Williams we know that $[M]$ is a uniformly integrable martingale whenever $M$ is $L^2$-bounded).  Here is an argument that may too complicated but shows that if $[M]$ exists for bounded martingales $M$ then $[M]$ exists for $L^2$ bounded martingales $M$ and $M^2 - [M]$ is a uniformly integrable martingale.

By $L^2$ boundedness we know that for every optional time $\tau$ we have $\abs{M^2_\tau} \leq (M^*)^2$ and moreover by Doob's inequality $\expectation{(M^*)^2} \leq 2 \norm{M} < \infty$ so $\lbrace M^2_\tau \mid \tau \text{ is an optional time}\rbrace$ is a uniformly integrable family (Example \ref{DominatedImpliesUniformlyIntegrable}).  Therefore by Lemma \ref{LpConvergenceUniformIntegrability} for any sequence of optional times $\tau_n$ such that $\tau_n \uparrow \infty$ a.s. we have not only $M^2_{\tau_n} \toas M^2_\infty$ but also $M_{\tau_n} \tolp{2} M_\infty$.

Now, define $\tau_n = \inf \lbrace t \geq 0 \mid M_t = n \rbrace$ which is an optional time by Lemma \ref{HittingTimesContinuous}.  TODO: Show $\tau_n \uparrow \infty$ a.s.  As in the proof above we know that $[M^{\tau_m}] = [M^{\tau_n}]$ on $[0,\tau_m]$ for any $m \leq n$ and therefore we can define $[M] = \lim_{n \to \infty} [M^{\tau_n}]$ and we have $[M]^{\tau_n} = [M^{\tau_n}]$ on $[0,\tau_n]$.  Moreover, since each $[M^{\tau_n}]$ is increasing we know that $[M^{\tau_n}]_\infty = [M^{\tau_n}]_{\tau_n} \uparrow [M]_\infty$ and therefore we can apply Monotone Convergence to conclude $[M^{\tau_n}]_\infty \tolp{1} [M]_\infty$.  TODO: Finish.
\end{proof}

\begin{lem}\label{QuadraticCovariationAndContinuity}Let $M_n$ be a sequence of continuous local martingales, then $M_n^* \toprob 0$ if and only if $[M_n]_\infty \toprob 0$.
\end{lem}
\begin{proof}
First we assume that $M_n^* \toprob 0$.  Let $\epsilon > 0$ be given and define $\tau_n = \inf \lbrace t \geq 0 \mid (M_n)_t > \epsilon \rbrace$ which is an optional time because of the continuity of $M_n$.  Moreover, we know that $M^{\tau_n}_n$ is a bounded continuous martingale and therefore $(M^2_n - [M_n])^{\tau_n} = (M^{\tau_n}_n)^2 - [M^{\tau_n}_n]$ is a martingale starting at zero which shows that for all $t \geq 0$,
\begin{align*}
\expectation{[M^{\tau_n}_n]_t} &=  \expectation{(M^{\tau_n}_n)_t^2} \leq \epsilon^2
\end{align*}
Now we can use a Markov bound to see that
\begin{align*}
\probability{[M_n]_\infty > \epsilon} &\leq \probability{[M_n]_\infty > \epsilon; \tau_n < \infty } + \probability{[M_n]_\infty > \epsilon; \tau_n = \infty } \\
&\leq \probability{ \tau_n < \infty } + \probability{[M_n]_{\tau_n} > \epsilon} \\
&\leq \probability{ M^*_n > \epsilon } + \epsilon^{-1} \expectation{[M_n]_{\tau_n}} \\
&\leq \probability{ M^*_n > \epsilon } + \epsilon
\end{align*}
To see that this shows convergence in probability, first note that by our assumption that $M^*_n \toprob 0$ we have $\lim_{n \to \infty} \probability{[M_n]_\infty > \epsilon} \leq \epsilon$.  But now note that the left hand limit is a decreasing function of $\epsilon$ and therefore 
\begin{align*}
\lim_{n \to \infty} \probability{[M_n]_\infty > \epsilon} &\leq \lim_{\epsilon \to 0^+} \lim_{n \to \infty} \probability{[M_n]_\infty > \epsilon} \leq \lim_{\epsilon \to 0^+} \epsilon
\end{align*}
thus as $\epsilon > 0$ was arbitrary we have shown $[M_n]_\infty \toprob 0$.

Now we assume that $[M_n]_\infty \toprob 0$.  As before let $\epsilon > 0$ be given and this time define $\tau_n = \inf \lbrace t \geq 0 \mid [M_n]_t > \epsilon^2$ which is an optional time by continuity of $[M_n]$.  

Claim: Let $N$ be a continuous local martingale with $N_0 = 0$ and $\expectation{[N]_\infty} < \infty$, the $N$ is in fact an $L^2$ bounded martingale.

To see the claim, pick $\sigma_n = \inf \lbrace t \geq 0 \mid \abs{N_t} > n \rbrace$ and we have seen that $\sigma_n$ is a localizing sequence for $N$ such that $N^{\sigma_n}$ is a bounded martingale.  Therefore $(N^{\sigma_n})_t^2 - [N^{\sigma_n}]$ is a martingale starting at zero and for all $t \geq 0$ because $[N]_t$ is increasing
\begin{align*}
\expectation {(N^{\sigma_n})_t^2} &= \expectation{[N^{\sigma_n}]_t} \leq \expectation{[N]_\infty} < \infty
\end{align*}
Therefore for fixed $t \geq 0$, the sequence $(N^{\sigma_n})^2_t$ is $L^2$ bounded and therefore the sequence $N^{\sigma_n}_t$ is uniformly integrable (Lemma \ref{BoundedLpImpliesUniformlyIntegrable}) which shows us that 
\begin{align*}
\cexpectation{\mathcal{F}_s}{N_t}   &= \lim_{n \to \infty} \cexpectation{\mathcal{F}_s}{N^{\sigma_n}_t} = \lim_{n \to \infty} N^{\sigma_n}_s = N_s
\end{align*}
and by Fatou's Lemma we have 
\begin{align*}
\expectation{N_t^2} &\leq \liminf_{n \to \infty} \expectation { (N^{\sigma_n})_t^2} \leq \expectation{[N]_\infty}
\end{align*}
which shows that $N$ is an $L^2$ bounded martingale.

Now we can apply the claim to the local martingale $M^{\tau_n}_n$ for which by definition of $\tau_n$ we have $[M^{\tau_n}_n]_\infty = [M_n]_{\tau_n} \leq \epsilon^2$ and therefore a fortiori $\expectation {[M^{\tau_n}_n]_\infty} < \infty$.  Thus we conclude that $M^{\tau_n}_n$ is an $L^2$-bounded martingale and therefore $(M^{\tau_n}_n)^2 - [M^{\tau_n}_n]$ is a uniformly integrable martingale starting at zero.  We are now in a position to mimic the first part of the proof.   By the martingale property and the defintion of $\tau_n$ we have for all $0 \leq t \leq \infty$,
\begin{align*}
\expectation{(M^{\tau_n}_n)_t^2} &= \expectation{ [M^{\tau_n}_n]_t } = \expectation{ [M_n]_{\tau_n \wedge t} } \leq \epsilon^2
\end{align*}
and by a Markov bound and Doob's $L^2$ inequality applied to the $L^2$ bounded martingale $M^{\tau_n}_n$ we get,
\begin{align*}
\probability{M^*_n \geq \epsilon} &\leq \probability{M^*_n \geq \epsilon; \tau_n < \infty }  + \probability{M^*_n \geq \epsilon ; \tau_n = \infty} \\
&\leq \probability{\tau_n < \infty }  + \probability{(M^{\tau_n}_n)^* \geq \epsilon} \\
&\leq \probability{\tau_n < \infty }  + \epsilon^{-1} \expectation{(M^{\tau_n}_n)^*} \\
&\leq \probability{[M_n]_\infty > \epsilon^2 }  + 2 \epsilon^{-1} \expectation{(M^{\tau_n}_n)^2_\infty} \\
&\leq \probability{[M_n]_\infty > \epsilon^2 }  + 2 \epsilon \\
\end{align*}
and as before take the limit as $n \to \infty$ and then as $\epsilon \to 0$ to see that $M^*_n \toprob 0$.
\end{proof}

Because the covariation process $[M,N]$ is of finite variation we can define a pointwise Lebesgue-Stieltjes integral $\int f(\omega,s) \, d[M,N]_s$ for any progressive process $f(\omega,t)$ (TODO: is jointly measurable enough?  If we assume progressive then I guess we get a local martingale out of this).  Note that there is the potential for ambiguity in interpreting an integral with respect to a process of finite variation when the integrand is a step process as we could also consider using the definition as an elementary stochastic integral.  It does turn out that these two possible definitions agree but we'll defer addressing the question and instead we will always explicitly denote the integration variable when considering a pointwise Stieltjes integral as in the expression $\int_0^t U_s \, dM_s$.  TODO: Validate that the elementary stochastic integral defined above is consistent with the definition of the pointwise Stieltjes integral; it is worth understand the point at which we need this fact as Rogers and Williams indicate that they don't require it for quite some time.  Actually the consistency when integrands are step processes is trivial to see.  The fact that Rogers and Williams defer is the deeper fact that once one has defined a stochastic integral for not necessarily continuous local martingales one has the possibility for a stochastic integral with an integrator of finite variation.  This integral can be shown to agree with the pointwise Stieltjes integral.

Before we begin we record the following simple fact about Lebesgue-Stieltjes integrals.

TODO: Remove as we moved this into a separate section.
\begin{lem}\label{StoppingStieltjesIntegrals}Let $F$ be a function of finite variation and for each $t \geq 0$ define $F^t(s) = F(t \wedge s)$, then for all measurable $g$ we have $\int_0^t g \, dF = \int g \, dF^t$.
\end{lem}
\begin{proof}
First suppose that $F$ is a non-decreasing right continuous function, and consider the Stieltjes measure $\mu_t$ defined by $F^t$ (see \ref{LebesgueStieltjesMeasure}).  For any interval $[a,b]$ we have
\begin{align*}
\mu_t([a,b]) &= F^t(b) - F^t(a) = F(b \wedge t) - F(a \wedge t) = \int_0^\infty \characteristic{[a,b]} \characteristic{[0,t]} \, dF 
\end{align*}
and therefore $\mu_t$ obtained by applying the density function $\characteristic{[0,t]}$ to the Stieltjes measure for $F$.  Now by Lemma \ref{ChainRuleDensity} we see that $\int g \, dF^t = \int g \characteristic{[0,t]} \, dF = \int_0^t g \, dF$.  To finish the result, write a function of finite variation as a difference of two montone functions.
\end{proof}

\begin{lem}\label{CovariationOfPredictableStepProcessesContinuousLocalMartingale}Let $M$ and $N$ be continuous local martingales and let $U$ and $V$ be finite predictable step processes with deterministic jump times, then 
\begin{align*}
[ \int U \, dM, \int V \, dN] &= \int U_s V_s \, d[M,N]_s \text{ a.s.}
\end{align*}
\end{lem}
\begin{proof}
We know that each of $\int U \, dM$ and $\int U \, dM$ is a continuous local martingale by Lemma \ref{StochasticIntegralPredictableStepProcess}.  In addition each of the expressions in the results is invariant under centering thus we may assume $M_0 = N_0 = 0$.  Furthermore for any optional time $\tau$ we have by Theorem \ref{OptionalQuadraticCovariation}
\begin{align*}
[ \int U \, dM, \int V \, dN]^\tau &= [ \left(\int U \, dM \right)^\tau, \left(\int V \, dN \right)^\tau] = [ \int U \, dM^\tau, \int V \, dN^\tau]
\intertext{and  by Lemma \ref{StoppingStieltjes}}
\left(\int U_s V_s \, d[M,N]_s\right)^\tau &= \int U_s V_s \, d[M,N]^\tau_s
\end{align*}
so if we choose a common localizing sequence $\tau_n \uparrow \infty$ it suffices prove the result for $M^{\tau_n}$, $N^{\tau_n}$ and $[M,N]^{\tau_n}$.  Thus, we may assume that $M$, $N$ and $[M,N]$ is each bounded.  Thus each of $M$, $N$ and $MN - [M,N]$ is a bounded martingale hence each is closable and we may in fact assume each is a bounded martingale on $[0,\infty]$.

Now we first assume that $V = 1$ and let $U = \sum_{k=1}^n \eta_k \characteristic{(t_{k-1}, t_k]}$.  By appending an extra term with $\eta_n = 0$ we may assume that $t_n=\infty$.  Now we compute using the definitions and the martingale property of $M$, $N$ and $MN-[M,N]$ to see
\begin{align*}
\expectation{N_\infty \int_0^\infty U \, dM } &= \expectation{\sum_{k=1}^n \eta_k \left(M_{t_k} - M_{t_{k-1}}\right) \sum_{k=1}^n \left( N_{t_k} - N_{t_{k-1}}\right)} \\
&= \expectation{\sum_{k=1}^n \eta_k \left(M_{t_k} N_{t_k} - M_{t_{k-1}} N_{t_{k-1}}\right)} \\
&= \expectation{\sum_{k=1}^n \eta_k \left([M, N]_{t_k} - [M, N]_{t_{k-1}}\right)} \\
&= \expectation{\int_0^\infty U_s \, d[M,N]_s}
\end{align*}
For an arbitrary optional time $\tau$ we can also apply this argument to $M^\tau$ and $N^\tau$ to see that
\begin{align*}
\expectation{N_\tau \int_0^\tau U \, dM } &= \expectation{N^\tau_\infty\int_0^\infty U \, dM^\tau } \\
&= \expectation{\int_0^\infty U_s \, d[M^\tau,N^\tau]_s} = \expectation{\int_0^\tau U_s \, d[M,N]_s}
\end{align*}
From Lemma \ref{MartingaleOptionalTimeCriterion} we see that $N_t \int_0^t U \, dM - \int_0^t U_s \, d[M,N]_s$ is a martingale
and therefore $[\int U \, dM, N] = \int_0^t U_s \, d[M,N]_s$ a.s. by uniqueness of the quadratic covariation.

Now we finish by assuming a general $V = \sum_{k=1}^n \xi_k \characteristic{(t_{k-1},t_k]}$.  Note that we can assume by redefining $\xi_k$ and $\eta_k$ appropriately that both $U$ and $V$ are defined with respect to the same sequence of deterministic jump times $0=t_0 < t_1 < \dotsb < t_n$ so in particular $UV = \sum_{k=1}^n \eta_k \xi_k \characteristic{(t_{k-1},t_k]}$.  We can compute directly twice using the special case just proven
\begin{align*}
[\int U \, dM , \int V \, dN]_t &= \int_0^t U_s \, d[M, \int V \, dN]_s \\
&= \sum_{k=1}^n \eta_k \left( [M, \int V \, dN]_{t_k \wedge t} - [M, \int V \, dN]_{t_{k-1} \wedge t} \right) \\
&= \sum_{k=1}^n \eta_k \left( \int_0 ^{t_k \wedge t} V_u \, d [M, N]_u - \int_0^{t_{k-1} \wedge t} V_u \,d [M, N]_u \right) \\
&= \sum_{k=1}^n \eta_k \sum_{j=0}^n  \xi_j \left( [M, N]_{t_j \wedge t_k\wedge t} - [M, N]_{t_{j-1} \wedge t_k \wedge t} - [M, N]_{t_j \wedge t_{k-1}\wedge t} + [M, N]_{t_{j-1} \wedge t_{k-1} \wedge t} \right) \\
&= \sum_{k=1}^n \eta_k  \xi_k \left( [M, N]_{t_k\wedge t} - [M, N]_{t_{k-1} \wedge t} \right) \\
&= \int_0^t U_s V_s \, d[M,N]_s
\end{align*}
and the full result is proven.
\end{proof}

We have the following bounds on ruin probabilities as a corollary of Optional Stopping for continuous martingales.
\begin{lem}\label{GamblersRuinContinuousMartingale}Let $M$ be a continuous martingale with $M_0 = 0$ and such that $\probability{M^* > 0} > 0$.  If we define $\tau_x = \inf \lbrace t>0 \mid M_t = x\rbrace$ then for every $a < 0 < b$ we have
\begin{align*}
\cprobability{M^* > 0}{\tau_a < \tau_b} &\leq \frac{b}{b-a} \leq \cprobability{M^* > 0}{\tau_a \leq \tau_b} 
\end{align*}
\end{lem}
\begin{proof}We know that $\tau_a$ and $\tau_b$ are optional by continuity of $M$ and Lemma \ref{HittingTimesContinuous}.  Define $\tau = \tau_a \wedge \tau_b$ which we know is optional as well.  For every $t \geq 0$, by Optional Stopping we know that $\expectation{M_{\tau \wedge t}} = M_0 = 0$.  Clearly $\lim_{t \to \infty} M_{\tau \wedge t} = M_\tau$ and by the definition of $\tau$ we know that $\abs{M_{\tau \wedge t}} \leq -a \vee b < \infty$ and therefore we can apply Dominated Convergence to conclude that $\expectation{M_\tau} = 0$.  Now we can establish bounds using two simple facts.   First by continuity of $M$, we know that $\tau_a = \tau_b$ if and only if $\tau_a = \tau_b = \tau = \infty$.   Secondly $\tau_a \neq \tau_b$ implies $M^* > 0$.  With these observations in hand,
\begin{align*}
0 &=\expectation{M_\tau; \tau_a < \tau_b} + \expectation{M_\tau; \tau_b < \tau_a} + \expectation{M_\infty; \tau_a = \tau_b =\infty} \\
&\leq a \probability{\tau_a < \tau_b} + b \probability{\tau_b < \tau_a} + b \probability{M^* > 0 ; \tau_a = \tau_b =\infty} \\
&= a \probability{\tau_a < \tau_b} + b \probability{M^* > 0 ; \tau_b \leq \tau_a} \\
&= a \probability{\tau_a < \tau_b} + b \probability{M^* > 0} - b \probability{M^* > 0 ; \tau_a < \tau_b} \\
&= b \probability{M^* > 0} - (b - a)\probability{M^* > 0 ; \tau_a < \tau_b} \\
\end{align*}
which gives the first inequality. The second inequality is demonstrated in the same way but using a lower bound for $M_\infty$ on $\tau_a = \tau_b = \infty$, 
\begin{align*}
0 &\geq a \probability{\tau_a < \tau_b} + b \probability{\tau_b < \tau_a} + a \probability{M^* > 0 ; \tau_a = \tau_b =\infty} \\
&= a \probability{M^* > 0; \tau_a \leq \tau_b} + b \probability{M^* > 0 ; \tau_b < \tau_a} \\
&= a \probability{M^* > 0; \tau_a \leq \tau_b} + b \probability{M^* > 0} - b \probability{M^* > 0 ; \tau_a \leq \tau_b} \\
&= b \probability{M^* > 0} - (b - a)\probability{M^* > 0; \tau_a \leq \tau_b} \\
\end{align*}
\end{proof}

\begin{thm}[Burkholder-Davis-Gundy Inequalities]\label{BDGInequalities}For every $p > 0$ there exist a constant $0 < c_p < \infty$ such that for every continuous local martingale $M$ with $M_0 = 0$ we have
\begin{align*}
c_p^{-1} \expectation{[M]^{p/2}_\infty} &\leq \expectation{(M^*)^p} \leq c_p \expectation{[M]^{p/2}_\infty}
\end{align*}
\end{thm}
\begin{proof}
TODO: Perform reduction to the bounded martingale case via localization and optional stopping (Kallenberg indicates that we may also assume $[M]$ is bounded).

The following argument is quite elementary in each of its steps but is not entirely obvious so we spell it out in great detail.  To derive the inequalities for expectations we'll use Lemma \ref{TailsAndExpectations} and therefore we proceed by creating tail bounds for the random variables in question.  We first work on the right hand inequality of the result.  Let $r > 0$ be fixed and define $\tau = \inf \lbrace t \geq 0 \mid M_t^2 = r \rbrace$ (which is an optional time by continuity and Lemma \ref{HittingTimesContinuous}) and define $\tilde{M} = M - M^\tau$ and $N = \tilde{M}^2 - [\tilde{M}]$.  Pick any $0 < c < 1$ (we'll later refine the required bounds on $c$) and write
\begin{align*}
\probability{(M^*)^2 \geq 4r} &= \probability{(M^*)^2 \geq 4r ; [M]_\infty \geq cr} + \probability{(M^*)^2 \geq 4r; [M]_\infty < cr} \\
&\leq \probability{[M]_\infty \geq cr} + \probability{(M^*)^2 \geq 4r; [M]_\infty < cr} 
\end{align*}
we get
\begin{align*}
\probability{(M^*)^2 \geq 4r} - \probability{[M]_\infty \geq cr} &\leq \probability{(M^*)^2 \geq 4r; [M]_\infty < cr} 
\end{align*}
Since $[\tilde{M}] = [M] - [M]^\tau$ and $[M]$ is non-decreasing it follows that $[\tilde{M}] \leq [M]$ and therefore $[M]_\infty < cr$ implies $[\tilde{M}]_\infty < cr$.  Since trivially $\tilde{M}^2 \geq 0$ we know that $[M]_\infty < cr$ implies $N > -cr$.  On $\lbrace (M^*)^2 \geq 4r \rbrace$ we know that $\tau < \infty$ and therefore $\abs{M_\tau} = \sqrt{r}$ and for any $\epsilon > 0$ we can find $t \geq 0$ such that $\abs{M_t} \geq 2\sqrt{r} - \epsilon$, thus $\abs{M_t - M_\tau} > \sqrt{r} - \epsilon$ which implies $\sup_t \tilde{M}^2_t \geq r$.  Putting these observations together we see that $\lbrace (M^*)^2 \geq 4r; [M]_\infty < cr \rbrace \subset \lbrace N > -cr; \sup_t N_t > r - cr \rbrace$ and we get 
\begin{align*}
\probability{(M^*)^2 \geq 4r} - \probability{[M]_\infty \geq cr} &\leq \probability{(M^*)^2 \geq 4r; [M]_\infty < cr} \\
&\leq \probability{N > -cr; \sup_t N_t > r - cr}
\end{align*}
Now since $N$ is a martingale with $N_0 = 0$, we can apply the Gambler's Ruin Lemma \ref{GamblersRuinContinuousMartingale} with $-cr < 0 < r - cr$ to and use the fact that $\lbrace N > -cr; \sup_t N_t > r - cr \rbrace \subset \lbrace \tau_{-cr} > \tau_{r - cr} ; N^* > 0 \rbrace$ to conclude that 
\begin{align*}
\probability {N > -cr; \sup_t N_t > r - cr} &\leq \probability{\tau_{-cr} > \tau_{r - cr} ; N^* > 0} \\
&= 1 - \probability{\tau_{-cr} \leq \tau_{r - cr} ; N^* > 0} \\
&\leq (1 - \frac{r - cr}{r -cr + cr})\probability{N^* > 0} = c \probability{N^* > 0}
\end{align*}
It is clear from the nonnegativity of $[\tilde{M}]$ and the definition of $N$ that $N^* > 0$ implies $\tilde{M}^*= (M-M^\tau)^* > 0$ which $\tau < \infty$ and therefore $(M^*)^2 > r$.  Thus
\begin{align*}
\probability{(M^*)^2 \geq 4r} - \probability{[M]_\infty \geq cr} &\leq \probability{(M^*)^2 \geq 4r; [M]_\infty < cr} \\
&\leq \probability{N > -cr; \sup_t N_t > r - cr} \\
&\leq c \probability{N^* > 0} \leq c \probability{(M^*)^2 > r}
\end{align*}

Now we multiply by $\frac{p}{2}r^{p/2-1}$ and integrate to get 
\begin{align*}
&\frac{p}{2}\int_0^\infty r^{p/2-1}\probability{(M^*)^2 \geq 4r} \, dr - \frac{p}{2}\int_0^\infty r^{p/2-1}\probability{[M]_\infty \geq cr} \, dr \\
&\leq \frac{cp}{2}\int_0^\infty r^{p/2-1} \probability{(M^*)^2 > r} \, dr \\
\end{align*}
which yields upon changing integration variables and applying Lemma \ref{TailsAndExpectations}
\begin{align*}
2^{-p} \expectation{\abs{M^*}^{p}} - c^{-p/2} \expectation{\abs{[M]_\infty}^{p/2}} &\leq c \expectation{\abs{M^*}^{p}}
\end{align*}
Thus we get the right hand inequality for $c_p = c^{-p/2}/(2^{-p} -c)$ which is a positive constant for any $0 < c < 2^{-p}$.

The proof of the left hand inequality follows the same pattern but this time we define the optional time $\tau = \inf \lbrace t \geq 0 \mid [M_t] = r \rbrace$ and as before $\tilde{M} = M - M^\tau$ and $N = \tilde{M}^2 - [\tilde{M}]$.  We let $r > 0$ be arbitrary, assuming that $0 < c < 1/4$.  We give the entire computation at once and them make some comments about the details of the justification:
\begin{align*}
\probability{[M]_\infty \geq 2r } - \probability{(M^*)^2 \geq cr} &\leq \probability{[M]_\infty \geq 2r ; (M^*)^2 < cr} \\
&\leq \probability{N < 4cr ; \inf_t N_t < 4cr -r} \\
&\leq 4c \probability{N^* > 0} \\
&\leq 4c \probability{[M]_\infty \geq r}
\end{align*}
The first inequality follows as before by a simple union bound.  To see the second inequality, note first that on $\lbrace (M^*)^2 < cr \rbrace$ by non-negativity of $[\tilde{M}]$ we have 
\begin{align*}
N &\leq \tilde{M}^2 \leq (\abs{M} + \abs{M^\tau})^2 \leq (2 M^*)^2 < 4 cr
\end{align*}
and also on $\lbrace [M]_\infty \geq 2r \rbrace$ we have $\tau < \infty$ and 
\begin{align*}
[\tilde{M}]_\infty = [M]_\infty - [M]_\tau \geq 2r - r = r
\end{align*}
To see the third inequality we again apply Gambler's Ruin Lemma \ref{GamblersRuinContinuousMartingale} to $N$  this time on $4cr - r < 0 < 4 cr$ noting that $\probability{N < 4cr ; \inf_t N_t < 4cr -r} \leq \probability{ \tau_{4cr - r} <  \tau_{4c} ; N^* > 0} \leq 4c \probability{N^* > 0}$.  The final inequality again follows from noting that $\tau < \infty$ on $N^* > 0$ and therefore because $[M]$ is non-decreasing we have $[M]_\infty \geq [M]_\tau = r$.

Again we multiply by $(p/2) r^{p/2 -1}$ and integrate to get 
\begin{align*}
&\frac{p}{2} \int_0^\infty r^{p/2 -1} \probability{[M]_\infty \geq 2r } \, dr - \frac{p}{2} \int_0^\infty r^{p/2 -1} \probability{(M^*)^2 \geq cr} \, dr\\
&\leq 4c \frac{p}{2} \int_0^\infty r^{p/2 -1} \probability{[M]_\infty \geq r} \, dr
\end{align*}
which upon changing variables and applying Lemma \ref{TailsAndExpectations}
\begin{align*}
2^{p/2} \expectation{\abs{[M]_\infty}^{p/2}} - c^{-p/2} \expectation{\abs{M^*}^p} &\leq 4c \expectation{\abs{[M]_\infty}^{p/2}} 
\end{align*}
which yields the left hand inequality with $c_p = c^{-p/2}/(2^{-p/2} -4c)$ which is positive for any $0 < c < 2^{-p/2 - 2}$.
\end{proof}

In the following Lemma we remind the reader of the notation $\int g \, \abs{dF}$ to denote integration with respect to the Lebesgue-Stieltjes measure determined by the total variation function of $F$.
\begin{lem}\label{CourregeCauchySchwartz}Let $M$ and $N$ be continuous local martingales, then almost surely for every $t \geq 0$, 
\begin{align}
\abs{[M,N]_t} &\leq \int_0^t \abs{d[M,N]}_s \leq [M]_t^{1/2} [N]_t^{1/2}
\label{CourregeCauchySchwartz1}\end{align}

Furthermore almost surely for any jointly measurable processes $U$ and $V$ we have 
\begin{align*}
\int_0^t \abs{U_sV_s} \, \abs{d[M,N]}_s &\leq \left(\int_0^t U_s^2 \, d[M]_s\right)^{1/2} \left(\int_0^t V_s^2 \, d[N]_s\right)^{1/2} 
\end{align*}
(TODO: Confirm that almost sure this holds for all $U,V$ not that for each pair $U,V$ this holds a.s.)
\end{lem}
\begin{proof}
First we can use positivity and bilinearity of quadratic covariation to see that for a fixed $t \geq 0$ and $\lambda \in \reals$ we have 
\begin{align*}
0 &\leq [M + \lambda N]_t = [M]_t + 2\lambda[M,N]_t + \lambda^2[N]_t \text{ a.s.}
\end{align*}
It follows that $\probability {\cap_{\lambda \in \rationals} \lbrace 0 \leq  [M]_t + 2\lambda[M,N]_t + \lambda^2[N]_t  \rbrace} = 1$ and by continuity of the quadratic polynomial we get that for fixed $t \geq 0$, almost surely $0 \leq [M]_t + 2\lambda[M,N]_t + \lambda^2[N]_t$ for all $\lambda \in \reals$.  Taking the discriminant of the quadratic polynomial and using the fact that it must be non-negative we see that for every $t \geq 0$ we have $[M,N]_t^2 \leq [M]_t [N]_t$ almost surely.  Again, taking the intersection of a countable number of almost sure events we see that almost surely we have $[M,N]_q^2 \leq [M]_q [N]_q$ for all $q \in \rationals$ with $q \geq 0$ and by continuity of the quadratic variation this implies that almost surely $[M,N]_t^2 \leq [M]_t [N]_t$ for all $t \geq 0$.

Now fix an $s \geq 0$ and consider the processes $M - M^s$ and $N - N^s$.  Replaying our continuity argument once more we see that almost surely the inequality just proven will hold almost surely over all the processes $M - M^s$, $N - N^s$ and all $t \geq 0$.   Using this fact and Theorem \ref{OptionalQuadraticCovariation} we conclude that almost surely for all $s \geq 0$ and $s < t$ we have
\begin{align*}
\abs{[M,N]_t - [M,N]_s } &=\abs{ [M - M^s, N-N^s]_t} \leq \left([M]_t - [M]_s\right)^{1/2} \left([N]_t - [N]_s\right)^{1/2}
\end{align*}
Suppose we are given a partition $s=t_0 < \dotsb < t_n=t$ and use the triangle inequality, the Cauchy-Schwartz inequality for sequences and the above inequality gives us 
\begin{align*}
\abs{[M,N]_t - [M,N]_s} &\leq \sum_{j=1}^n \abs{[M,N]_{t_j} - [M,N]_{t_{j-1}}} \\
&\leq \sum_{j=1}^n \left([M]_{t_j} - [M]_{t_{j-1}}\right)^{1/2} \left([N]_{t_j} - [N]_{t_{j-1}}\right)^{1/2} \\
&\leq  \left(\sum_{j=1}^n [M]_{t_j} - [M]_{t_{j-1}}\right)^{1/2} \left(\sum_{j=1}^n [N]_{t_j} - [N]_{t_{j-1}}\right)^{1/2} \\
&= ([M]_t - [M]_s)^{1/2} ([N]_t-[N]_s)^{1/2}
\end{align*}
Again, note that this holds almost sure simultaneously for all $0 \leq s < t$, all $n \geq 0$ and all partitions $s=t_0 < \dotsb < t_n=t$.  We may then take the supremum over all partitions to get 
\begin{align*}
\abs{[M,N]_t - [M,N]_s} &\leq \int_s^t \abs{d[M,N]_s} \leq ([M]_t - [M]_s)^{1/2} ([N]_t-[N]_s)^{1/2}
\end{align*}
and substituting $s=0$ we get \eqref{CourregeCauchySchwartz1}.

Before proceeding further it is helpful to name all of the random Lebesgue-Stieltjes measures floating around: let $\mu = d[M]$, $\nu = d[N]$ and $\rho = \abs{d[M,N]}$.  Note that we have shown that almost surely for every closed interval $I \subset \reals$ we have $\rho(I)^2 \leq \mu(I) \nu(I)$.  By continuity of $[M]$, $[N]$ and $[M,N]$ the measures above have no atoms and therefore this inequality also holds for open intervals.  Now if we let $G$ be an arbitrary open set then we can write it as a disjoint union of open intervals (Lemma \ref{OpenSetsOfReals}) $G = \cup_{n=1}^\infty I_n$.  Then by countable additivity and Cauchy-Schwartz for sequences
\begin{align*}
\rho(G) &= \sum_{n=1}^\infty \rho(I_n) \leq \sum_{n=1}^\infty \mu(I_n)^{1/2}\nu(I_n)^{1/2} \\
&\leq \left( \sum_{n=1}^\infty \mu(I_n) \right)^{1/2}\left( \sum_{n=1}^\infty \nu(I_n) \right)^{1/2} = \mu(G)^{1/2}\nu(G)^{1/2}
\end{align*}

TODO: Extend to general Borel sets by monotone classes: I think we needed boundedness of the measures here.

Now let $f = \sum_{i=1}^n a_i \characteristic{A_i}$ and $g = \sum_{i=1}^n b_i \characteristic{A_i}$ be positive simple functions.  Then once again applying Cauchy-Schwartz for sequences we get
\begin{align*}
\int f(s) g(s) \, \abs{d[M,N]_s} &= \sum_{i=1}^n a_i b_i \rho(A_i) \\
&\leq \sum_{i=1}^n a_ib_i \mu(A_i)^{1/2} \nu(A_i)^{1/2} \\
&\leq \left(\sum_{i=1}^n a_i^2 \mu(A_i) \right)^{1/2} \left( \sum_{i=1}^n b_i^2 \nu(A_i)\right)^{1/2} \\
&= \left( \int f^2(s) \, d[M]_s \right)^{1/2} \left( \int g^2(s) \, d[N]_s \right)^{1/2} 
\end{align*}


For general positive measurable functions $f$ and $g$ we take positive simple approximations $f_n \uparrow f$ and $g_n \uparrow g$ and we get by Monotone Convergence
\begin{align*}
\int f(s) g(s) \, \abs{d[M,N]_s} &= \lim_{n \to \infty} \int f_n(s) g_n(s) \, \abs{d[M,N]_s} \\
&\leq \lim_{n \to \infty}  \left( \int f_n^2(s) \, d[M]_s \right)^{1/2} \lim_{n \to \infty}\left( \int g_n^2(s) \, d[N]_s \right)^{1/2} \\
&= \left( \int f^2(s) \, d[M]_s \right)^{1/2} \left( \int g^2(s) \, d[N]_s \right)^{1/2} 
\end{align*}
noting that this holds almost surely for all $f$ and $g$ positive and measurable.

TODO: Finish, is there anything subtle about applying to the processes?
\end{proof}

\begin{defn}Given a continuous local martingale $M$ we let $L(M)$ denote the set of processes that are progressively measurable and for which $\int_0^t V^2_s \, d[M]_s < \infty$ almost surely for all $t \geq 0$.
\end{defn}
The space $L(M)$ gives the integrands for the extension of the stochastic integral with respect to the integrator $M$.
\begin{thm}\label{StochasticIntegralContinuousLocalMartingaleIntegrator}Let $M$ be a continuous local martingale and $V \in L(M)$, there exists an almost surely unique continuous local martingale $\int V \, dM$ starting at zero and for which almost surely
$[\int V \, dM, N]_t = \int_0^t V_s \, d[M,N]_s$ for all $t \geq 0$.
\end{thm}
\begin{proof}
First we show uniqueness as we shall use it during the existence argument.  Suppose that $M^\prime$ and $M^{\prime \prime}$ are continuous local martingales starting at zero for which for every continuous local martingale $[M^\prime, N] = [M^{\prime \prime},N] = \int V_s \, d[M,N]_s$ almost surely.  By linearity of quadratic covariation, this tell us that for all $N$ we have $[M^{\prime} - M^{\prime \prime} ,N]=0$ almost surely.  In particular this will be true if we pick $N = M^{\prime} - M^{\prime \prime}$ so we know that $[M^{\prime} - M^{\prime \prime}] = 0$ almost surely.  By definition of the quadratic variation this implies that $(M^{\prime} - M^{\prime \prime})^2$ is a continuous local martingale starting at zero.  Picking a localizing sequence $\tau_n$ and using the martingale property we see that $\expectation{(M_{t \wedge \tau_n}^{\prime} - M_{t \wedge \tau_n}^{\prime \prime})^2} =0$ which shows us that $(M_{t \wedge \tau_n}^{\prime} - M_{t \wedge \tau_n}^{\prime \prime})^2$ almost surely.  Taking the limit as $n \to \infty$ we get that $(M_{t}^{\prime} - M_{t}^{\prime \prime})^2=0$ almost surely for each $t \geq 0$ hence simultaneously for all $t \in \rationals_+$ and then by continuity for all $t \geq 0$ almost surely.

We first assume that $\int_0^\infty V^2_s \, d[M]_s < \infty$ almost surely and we use the notation $\norm{V}^2_M = \int_0^\infty V^2_s \, d[M]_s$ to denote the corresponding value.  Then if $N \in \mathcal{M}^2$ we have
\begin{align*}
\abs{\expectation{\int_0^\infty V_s \, d[M,N]_s}} &\leq \expectation{\abs{\int_0^\infty V_s \, d[M,N]_s}} \\
&\leq  \expectation{\int_0^\infty \abs{V_s} \, \abs{ d[M,N]_s}} && \text{by Lemma \ref{AbsoluteValueOfStieltjes}} \\
&\leq  \expectation{\left(\int_0^\infty V^2_s \, d[M]_s\right)^{1/2}\left(\int_0^\infty d[N]_s\right)^{1/2}} && \text{by Lemma \ref{CourregeCauchySchwartz}} \\
&=  \expectation{\left(\int_0^\infty V^2_s \, d[M]_s\right)^{1/2}[N]_\infty^{1/2}} \\
&\leq \expectation{\int_0^\infty V^2_s \, d[M]_s}^{1/2}\expectation{[N]_\infty}^{1/2} && \text{by Cauchy Schwartz} \\
&=\norm{V}_M \expectation{N^2_\infty}^{1/2} = \norm{V}_M \norm{N}
\end{align*}
which shows that $N \mapsto \expectation{\int_0^\infty V_s \, d[M,N]_s}$ is a continuous linear functional on $\mathcal{M}^2$.  Thus since $\mathcal{M}^2$ is a Hilbert space with inner product given by $\langle M,N \rangle = \expectation{M_\infty N_\infty}$ (Lemma \ref{ContinuousL2MartingalesHilbert}) we know that there exists an $L^2$-bounded martingale $\int V \, dM \in \mathcal{M}^2$ such that $\expectation{\int_0^\infty V_s \, d[M,N]_s } = \expectation{N_\infty \cdot \int_0^\infty V \, dM}$ for all $N \in \mathcal{M}^2$ (we emphasize that the use of the integral sign in the name $\int V \, dM$ we give to this martingale is only meant to be suggestive and the reader should not get confused trying to figure out that this element can be constructed by some kind of generalized sum; at this point it is no more and no less than the element of the Hilbert space corresponding to the linear functional we've defined).

Since $V$ is progressive we know that $\int V_s \, d[M,N]_s$ is $\mathcal{F}$-adapted (Lemma \ref{StochasticStieltjesIntegral}) and we have just shown that it is integrable.  Now let $\tau$ be an arbitrary optional time and apply the above construction to $N^\tau$ (TODO: Remind why $N^\tau \in \mathcal{M}^2$).  in the following computation
\begin{align*}
\expectation{\int_0^\tau V_s \, d[M,N]_s}  &=\expectation{ \int_0^\infty V_s \, d[M,N]^\tau_s} && \text{Lemma \ref{StoppingStieltjes}}\\
&=\expectation{ \int_0^\infty V_s \, d[M,N ^\tau]_s} && \text{Lemma \ref{OptionalQuadraticCovariation}}\\
&=\expectation{ N^\tau_\infty \cdot \int_0^\infty V \, dM} && \text{definition of $\int V \, dM$}\\
&=\expectation{ N_\tau \cdot \int_0^\infty V \, dM} \\
&=\expectation{ N_\tau \cexpectationlong{\mathcal{F}_\tau}{ \int_0^\infty V \, dM} } && \text{Tower Property}\\
&=\expectation{ N_\tau \int_0^\tau V \, dM}  && \text{Optional Stopping}\\
\end{align*}
We apply Lemma \ref{MartingaleOptionalTimeCriterion} to conclude that $N_t \int_0^t V \, dM - \int_0^t V_s \, d[M,N]_s$ is a martingale.  By the continuity of $[M,N]$ we know that $\int_0^t V_s \, d[M,N]_s$ is continuous and has locally finite variation (Corollary \ref{StieltjesIntegralBoundedVariationAndContinuous}); thus uniqueness and the defining property of quadratic covariation implies $\int V_s \, d[M,N]_s = [N, \int V \, dM]$ almost surely.

The next step is to extend the defining property of the integral to arbitrary continuous local martingales.  For this we take a localizing sequence $\tau_n$ such that $N^{\tau_n}$ is bounded (hence in $\mathcal{M}^2$).  Let $A$ be the event that $\tau_n \uparrow \infty$ and for each $n$, let $A_n$ be the event that $[N^{\tau_n}, \int V \, dM] = \int V_s \, d[M,N]_s$.  For all  $\omega \in A \cap \left(\cap_{n=1}^\infty A_n \right)$ and $t \geq 0$ we have 
\begin{align*}
[N, \int V \, dM]_t(\omega) &= \lim_{n \to \infty} [N, \int V \, dM]^{\tau_n}_t(\omega) \\
&= \lim_{n \to \infty} [N ^{\tau_n}, \int V \, dM]_t(\omega) \\
&= \lim_{n \to \infty} \int_0^t V_s(\omega) \, d[M,N^{\tau_n}]_s(\omega) \\
&= \lim_{n \to \infty} \int_0^{t \wedge \tau_n} V_s(\omega) \, d[M,N]_s(\omega) \\
&= \int_0^{t} V_s(\omega) \, d[M,N]_s(\omega) \\
\end{align*}
and as $\probability{A \cap \left(\cap_{n=1}^\infty A_n \right)} = 1$ we have $[N, \int V \, dM] = \int V_s \, d[M,N]_s$ almost surely.

Lastly we must remove the assumption that $\int_0^\infty V_s^2 \, d[M]_s < \infty$.  We know that $\int_0^t V_s^2 \, d[M]_s$ is a continuous process (Lemma \ref{StieltjesIntegralBoundedVariationAndContinuous}) and therefore for every $n >0$ we can define an optional time $\tau_n = \inf \lbrace t \geq 0 \mid \int_0^t V_s^2 \, d[M]_s = n \rbrace$.  We have
\begin{align*}
\int_0^\infty V_s^2 \, d[M^{\tau_n}]_s &= \int_0^{\tau_n} V^2_s \, d[M]_s = n < \infty
\end{align*}
and by our assumption that $\int_0^t V_s^2 \, d[M]_s < \infty$ for all $t \geq 0$ we know that $\tau_n \uparrow \infty$.  We apply the existing construction to define $\int V \, dM^{\tau_n}$ and it satisfies 
\begin{align*}
[N, \int V \, dM^{\tau_n}]_t  &= \int_0^t V_s \, d[M,N]^{\tau_n}_s = \int_0^{t\wedge\tau_n} V_s \, d[M,N]_s
\end{align*}
for every continuous local martingale $N$.  Moreover for $m < n$, from the above fact and Lemma \ref{OptionalQuadraticCovariation} we have
\begin{align*}
[N, \left(\int V \, dM^{\tau_n}\right)^{\tau_m}]_t &= [N, \int V \, dM^{\tau_n}]_{t \wedge \tau_m} =\int_0^{t\wedge\tau_m} V_s \, d[M,N]_s
\end{align*}
for all continuous local martingales $N$ which by uniqueness of the stochastic integral shows $\left(\int V \, dM^{\tau_n}\right)^{\tau_m} = \int V \, dM^{\tau_m}$ so that $\int V \, dM^{\tau_m}$ and $\int V \, dM^{\tau_n}$ agree on the interval $[0,\tau_m]$.  Therefore we can define $\int_0^t V \, dM$ as the limit of $\int_0^t V \, dM^{\tau_n}$ for any $\tau_n \geq t$.  The fact that this defines an adapted process follows from writing $\int_0^t V \, dM = \sum_{n=1}^\infty \characteristic{\lbrace \tau_{n-1} \leq t < \tau_n\rbrace} \int_0^t V \, dM^{\tau_n}$ together with the facts that $\tau_n$ is optional and $\int V \, dM^{\tau_n}$ is adapted.  Continuity at $t \geq 0$ follows by picking $\tau_n > t$ and noting that $\int_0^t V \, dM = \int_0^t V \, dM^{\tau_n}$ and continuity of $\int_0^t V \, dM^{\tau_n}$ at $t$.  By Lemma \ref{LocalMartingaleLocalProperty} we know that $\int V \, dM$ is a continuous local martingale.  Lastly by construction, for all $n \geq 0$ and each continuous local martingale there is a set $A_n$ with $\probability{A_n} =1$ such that
\begin{align*}
[N, \int V \, dM]_t &=[N, \int V \, dM]^{\tau_n}_t = [N, \left(\int V \, dM\right)^{\tau_n}]_t = [N, \int V \, dM^{\tau_n}]_t \\
&= \int_0^t V_s \, d[M^{\tau_n}, N]_s = \int_0^{t\wedge \tau_n} V_s \, d[M, N]_s = \int_0^{t} V_s \, d[M, N]_s
\end{align*}
for all $0 \leq t \leq \tau_n$ on $A_n$.  Thus taking the intersection of $A_n$ we see that $[N, \int V \, dM] = \int V_s \, d[M,N]_s$ almost surely.
\end{proof}

With this Theorem proven we know that the following defintion makes sense.
\begin{defn}Given a continuous local martingale $M$ and a progressive process $V$ such that $\int_0^t V^2_s \, d[M]_s < \infty$ for all $t \geq 0$, the \emph{stochastic integral} $\int V \, dM$ is the almost surely unique continuous local martingale for which $[\int V \, dM, N]_t = \int_0^t V_s \, d[M,N]_s$ for all $t \geq 0$ almost surely for every continuous local martingale $N$.
\end{defn}

Here we collect a few of the most elementary facts about the stochastic integral.  In particular we call attention to the Ito Isometry which doesn't figure prominently in our presentation but is a critical step in others; we shall have more to say about this later.  
\begin{lem}\label{BasicPropertiesStochasticIntegralContinuousMartingale}Let $M$ be a continuous local martingales.  If $U,V \in L(M)$ such that $U_t=V_t$ for all $t \geq 0$ almost surely then $\int U \, dM = \int V \, dM$.  The stochastic integral is bilinear in both the integrand and integrator.
TODO: Be very precise about assumptions here!  E.g. is $V \in L(aM + bN)$ equivalent to $V \in L(M)$ and $V \in L(N)$?  Clearly the latter is at least as strong as the former.
If $M$ is a continuous local martingale and $V \in L(M)$ then we have $[\int V \, dM]_t = \int_0^t V_s \, d[M]_s$ for all $t \geq 0$ almost surely.  In particular, if $M$ is a continuous martingale with $M_0 = 0$ we have the \emph{Ito Isometry} 
\begin{align*}
\expectation{\left(\int_0^t V \, dM\right)^2} &= \int_0^t V_s^2 \, d[M]_s \text{ for all $t \geq 0$}
\end{align*}
\end{lem}
\begin{proof}
With the assumption that $U = V$ almost surely we see that for all continuous local martingales $N$ we have
\begin{align*}
[\int U \, dM, N]_t &= \int_0^t U_s \, d[M,N]_s = \int_0^t V_s \, d[M,N]_s = [\int V \, dM, N]_t
\end{align*}
for all $t \geq 0$ almost surely.  By the uniqueness property of the stochastic integral we have $\int U \, dM = \int V \, dM$.

Bilinearity boils down to a couple of simple computations using bilinearity of the Lebesgue-Stieltjes integral and the quadratic covariation
\begin{align*}
[\int (aV + bU) \, dM, N]_t &= \int_0^t (aV_s + bW_s) \, d[M,N]_s \\
&= a\int_0^t V_s \, d[M,N]_s + b\int_0^t W_s \, d[M,N]_s \\
&= a[\int V \, dM, N]_t + b[\int W \, dM, N]_t \\
&= [a\int V \, dM + b\int W \, dM, N]_t
\end{align*}
and
\begin{align*}
[\int V \, d[aM + bN], R]_t &= \int_0^t V_s \, d[aM+bN, R]_s \\
&= a\int_0^t V_s \, d[M, R]_s + b\int_0^t V_s \, d[N, R]_s \\
&= a[\int V \, dM, R] + b[\int V \, dN, R]_t \\
&= [a\int V \, dM + b\int V \, dN, R]_t
\end{align*}
Now apply the uniqueness criteria for stochastic integrals.

Using the defining property of the stochastic integral twice and Lemma \ref{ChainRuleStieltjes} once we see
\begin{align*}
[\int V \, dM]_t &= \int_0^t V_s \, d[M, \int V \, dM]_s = \int_0^t V_s \, d\int_0^s V_u \, d[M](u) \\
&= \int_0^t V^2(s) \, d[M]_s
\end{align*}
In the special case that $M$ is a martingale starting at zero we know that $M^2 - [M]$ is a martingale starting at zero and thus taking expectations we get
\begin{align*}
\expectation{\left( \int_0^t V \, dM\right)^2} &= \expectation{[\int V \, dM]_t} = \expectation{\int_0^t V^2(s) \, d[M]_s}
\end{align*}
\end{proof}

It is common for the details of defining the stochastic integral to unfold a bit differently than our presentation.  The alternative presentation begins just as we have defines the stochastic integral for predicatable step process integrands but then notes the property of the Ito Isometry holds for such integrands.  The basic idea is to show that predicatable step processes are dense in an $L^2$ space and the use the Ito isometry to extend the definition of the stochastic integral by a completion argument.  There is a subtlety to deal with.  We note that the isometry holds for every \emph{fixed} $t \geq 0$ and thus is a family of isometries between an $L^2$ space of integrands (predictable step processes on $[0,t]$) and an $L^2$ space of random variables; it is not a single isometry between a spaces of processes.  There are two ways to proceed.  In the first case (Steele, Peres and Morters, others) one stays with the \emph{one $t$ at a time} approach and shows that step processes are dense in the progressive processes in $L^2(\Omega \times [0,t])$ and then extends the stochastic integral pointwise in $t \geq 0$.  An extra step is necessary at this point to show that one may find a version of the resulting stochastic integral process that is indeed a continuous martingale.  In the second case (e.g. Karatzas and Shreve), one defines a norm on the space of $L^2$ continuous martingales (different from the Hilbert space structure we have used) and shows that the Ito isometries can be assembled into a single isometry between the space of integrands and this space of martingales; again one extends by completion.  What about Rogers and Williams; they use the Ito isometry approach but I think the details are slightly different.
 
The basic continuity property of the stochastic integral is
\begin{lem}\label{LimitsOfStochasticIntegralContinuousLocalMartingale}Let $M_n$ be a sequence of continuous local martingales and let $V_n \in L(M_n)$, then 
$\left( \int V_n \, dM_n \right)^* \toprob 0$ if and only if $\int_0^\infty V_n^2(s) \, d[M_n](s) \toprob 0$.
\end{lem}
\begin{proof}
Lemma \ref{QuadraticCovariationAndContinuity} says that $\left( \int V_n \, dM_n \right)^* \toprob 0$ if and only if $[\int V_n \, dM_n]_\infty \toprob 0$ but Lemma \ref{BasicPropertiesStochasticIntegralContinuousMartingale} tells us that $[\int V_n \, dM_n]_\infty = \int_0^\infty V^2_n(s) \, d[M_n](s)$. 
\end{proof}

Before proceeding further we extend the class of integrators in what initially seems like a very ad-hoc manner.  Indeed this extension follows the historical path of the development of stochastic integration which broadened the scope of definitions in exactly these ways.  The reader is encouraged not to spend too much time trying to find the method in the madness as later we will prove a theorem that shows that the only continuous stochastic processes that make sense as integrators are the ones we define here.
\begin{defn}A \emph{continuous  semimartingale} $X$ is a cadlag adapted process in $\reals$ such that there is a continuous local martingale $M$ and a continuous, adapted process of locally finite variation $A$ with $A_0 = 0$ such that $X = M + A$.  A cadlag adapted process $X=(X_1, \dotsc, X_d)$ in $\reals^d$ is said to be a continuous semimartingale if and only each $X_i$ is.  Given a continuous semimartingale $X = M + A$ we let 
\begin{align*}
L(X) &= \lbrace V \mid V^2 \in L([M]) \text{ and } V \in L(A) \rbrace
\end{align*}
that is to say $L(X) = L(M) \cap L(A)$ and for any $V \in L(X)$ we define $\int V \, dX = \int V \, dM + \int V_s \, dA_s$.
\end{defn} 
Note that the decomposition $X = M + A$ is almost surely unique as if $M + A = \tilde{M} + \tilde{A}$ then $M - \tilde{M} = \tilde{A} - A$ is a continuous local martingale of locally finite variation and is therefore $0$ almost surely by Lemma \ref{ContinuousLocalMartingaleBoundedVariation}.  As such, we refer to this as the \emph{canonical decomposition}.

We want to develop the primary properties of the stochastic integral with a continuous semimartingale integrator.  Note that by the definition $\int V \, dX = \int V \, dM + \int V \, dA$ we can see that a stochastic integral with respect to a continuous semimartingale integrator is itself a continuous semimartingale.  Thus we can consider a stochastic integral as an integrator and the first result is a generalization of the ``chain rule'' proven in Lemma \ref{ChainRuleStieltjes}.
\begin{lem}\label{ChainRuleContinuousSemimartingale}Let $X$ be a continuous semimartingale and let $V \in L(X)$ then $U \in L(\int V \, dX)$ if and only if $UV \in L(X)$ and $\int U \, d\int V \, dX = \int UV \, dX$ a.s.
\end{lem}
\begin{proof}
For $X$ an adapted process of locally finite variation, this is proven in Lemma \ref{ChainRuleStieltjes}.  Now suppose that $X =M$ is a continuous local martingale.  In this case from the proof of Lemma \ref{LimitsOfStochasticIntegralContinuousLocalMartingale} and Lemma \ref{ChainRuleStieltjes} we have 
\begin{align*}
\int_0^t U_s^2 \, d[\int V \, dM]_s &= \int_0^t U_s^2 d \int_0^s V_u^2 d[M]_u = \int_0^t U_s^2 V_s^2 \, d[M]_s
\end{align*}
which shows us that $U \in L(\int V \, dM)$ if and only if $UV \in L(M)$.  Moreover, for any continuous local martingale $N$, we have
\begin{align*}
[\int U \, d\int V \, dM, N]_t &= \int_0^t U_s \, d[\int V \, dM, N]_s = \int_0^t U_s \, d \int_0^s V_u \, d[M, N]_u \\
&= \int_0^t U_s V_s \, d[M,N]_s = [\int UV \, dM, N]_t
\end{align*}
almost surely.  Thus by the defining property of stochastic integrals with a continuous local martingale integrator we know that $\int U \, d\int V \, dM = \int UV \, dM$.

Lastly let $X$ be a continuous semimartingale and let $X = M + A$ be the canonical decompostion of $X$.  Since the canonical decomposition of $\int V \, dX$ is $\int V \, dM + \int V_s \, dA_s$ we have 
\begin{align*}
L(\int V \, dX) &= L(\int V \, dM) \cap L(\int V_s \, dA_s) 
\end{align*}
hence combining results for Stieltjes integrals and continuous local martingale we have $U \in L(\int V \, dX)$ if and only if $UV \in L(M)$ and $UV \in L(A)$ (i.e. $UV \in L(X)$).  Furthermore,
\begin{align*}
\int_0^t U \, d\int V \, dX &= \int_0^t U \, d\int V \, dM + \int_0^t U \, d \int V_s \, dA_s \\
&= \int_0^t UV \, dM + \int_0^t U_s V_s \, dA_s = \int_0^t UV \, dX
\end{align*}
and the result is proven.
\end{proof}

The other useful result is the behavior of stochastic integrals under stopping (a generalizaiton of Lemma \ref{StoppingStieltjes}).
\begin{lem}\label{StoppingIntegralsContinuousSemimartingale}Let $X$ be a continuous semimartingale, $V \in L(X)$ and $\tau$ an optional time then
\begin{align*}
\left(\int V \, dX \right)^\tau &= \int V \, dX^\tau = \int \characteristic{[0,\tau]} V \, dX
\end{align*}
\end{lem}
\begin{proof}
The result is proven for Stieltjes integrals in Lemma \ref{StoppingStieltjes}, so consider next the case in which $X = M$ is a continuous local martingale.  Suppose $N$ is another continuous local martingale and compute
\begin{align*}
[\left(\int V \, dM \right)^\tau, N]_t &= [\int V \, dM, N]^\tau_t = \int_0^{t \wedge \tau} V_s \, d[M,N]_s = \int_0^t V_s \, d[M,N]^\tau_s \\
&= \int_0^t V_s \, d[M^\tau,N]_s = [\int V \, M^\tau, N]_t
\end{align*}
and similarly
\begin{align*}
[\left(\int V \, dM \right)^\tau, N]_t &= [\int V \, dM, N]^\tau_t = \int_0^t \characteristic{[0,\tau]} V_s \, d[M,N]_s = [\int \characteristic{[0,\tau]} V \, dM, N]_t
\end{align*}
and we appeal to the defining property of stochastic integrals with a continuous local martingale integrator.

For a general continuous semimartingale $X$, let $X = M + A$ be the canonical decomposition and then the fact that $M^\tau$ is a continuous local martingale and $A^\tau$ has locally finite variation to conclude that the canonical decomposition of $X^\tau$ is $M^\tau + A^\tau$ and use the results for the continuous local martingale case and the Stieltjes integral case to see
\begin{align*}
\left(\int V \, dX \right)^\tau &= \left(\int V \, dM\right)^\tau + \left(\int V_s \, dA_s\right)^\tau = \int V \, dM^\tau + \int V_s \, dA^\tau_s = \int V \, dX^\tau
\end{align*}
The second equality is equally trivial.
\end{proof}

The following Lemma will be a useful for exchanging limits and stochastic integrals and represents the fundamental continuity property of stochastic integrals.
\begin{lem}\label{DominatedConvergenceContinuousSemimartingale}Let $X$ be a continuous semimartingale and let $U, V, V_1, V_2, \dotsc \in L(X)$ with $\abs{V_n} \leq U$ and $V_n \toas V$ (TODO: Make precise what this means) then $\sup_{0 \leq s \leq t} \abs{\int_0^s V_n \, dX - \int_0^s V \, dX} \toprob 0$ for all $t \geq 0$.
\end{lem}
\begin{proof}
Write $X = M + A$ so that $U^2 \in L([M])$ and $U \in L(A)$.  By ordinary Dominated Convergence applied pointwise in $\Omega$ we know that almost surely $\int_0^t V_n(u) \, dA(u) \to \int_0^t V(u) \, dA(u)$ and $\int_0^t V_n^2(u) \, d[M](u) \to \int_0^t V^2(u) \, d[M](u)$ for every $t \geq 0$.  Because $\abs{V_n} \leq U$ we have
\begin{align*}
\abs{\int_0^t V_n(u) \, dA(u)} &\leq \int_0^t \abs{V_n(u)} \, d\abs{A}(u) \leq \int_0^t U(u) \, d\abs{A}(u)
\end{align*}
and the uniform continuity of $\int_0^t U(u) \, d\abs{A}(u)$ on every bounded interval we know that the family $\int_0^t V_n(u) \, dA(u)$ is uniformly equicontinuous on every bounded interval.  Therefore the pointwise convergence $\int_0^t V_n(u) \, dA(u) \to \int_0^t V(u) \, dA(u)$ can be extended to uniform convergence on bounded intervals $\sup_{0 \leq s \leq t} \abs{\int_0^s V_n(u) \, dA(u) - \int_0^s V(u) \, dA(u)} \toas 0$ and so it follows that $\sup_{0 \leq s \leq t} \abs{\int_0^s V_n(u) \, dA(u) - \int_0^s V(u) \, dA(u)} \toprob 0$.

From $\int_0^t V_n^2(u) \, d[M](u) \toas \int_0^t V^2(u) \, d[M](u)$ we get $\int_0^\infty V_n^2(u) \, d[M^t](u) \toas \int_0^\infty V^2(u) \, d[M^t](u)$ (Lemma \ref{StoppingStieltjes}).  By Lemma \ref{LimitsOfStochasticIntegralContinuousLocalMartingale} the latter convergence statement implies $\left(\int V_n \, dM^t  - \int V \, dM^t\right)^* \toprob 0$ and the Lemma follows since $\left(\int V_n \, dM^t  - \int V \, dM^t\right)^* = \left(\int V_n \, dM  - \int V \, dM\right)_t^*$ (TODO: Is this obvious from earlier or do we need to reference the general stopping property of stochastic integral) from Lemma \ref{StoppingIntegralsContinuousSemimartingale}.
\end{proof}

Recall that in the proof of Theorem \ref{OptionalQuadraticCovariation} we motivated the construction of the quadratic variation $[M]$ by pointing out that in the case of a bounded martingale starting at zero what we were doing was defining $[M] = M^2 - \int M \, dM$; the stochastic integral had not been defined at that point so the comment served the pedagogical purpose of motivating the formulae but wasn't mathematically justified.  Now that we have defined the stochastic integral are in a position to state and prove a proper Theorem.
\begin{thm}[Integration by parts]\label{IntegrationByPartsContinuousSemimartingale}Let $X$ and $Y$ be continuous semimartingales then 
\begin{align*}
X Y &= X_0 Y_0 + \int X \, dY + \int Y \, dX + [X,Y]
\end{align*}
\end{thm}
\begin{proof}
First let us assume $X =Y$ (we will later use polarization to extend to the general case).  Furthermore, let us assume that $X = M$ where $M \in \mathcal{M}^2$ is bounded and starts at zero.  Recall that from the proof of Theorem \ref{OptionalQuadraticCovariation}, if we define for $n \geq 0$,
\begin{align*} 
\tau^n_k &= \inf \lbrace t > \tau^n_{k-1} \mid \abs{M_t - M_{\tau^n_{k-1}}} = 2^{-n} \rbrace \text{ for $k > 0$} \\
V^n_t &= \sum_{k=0}^\infty M_{\tau^n_{k}} \characteristic{(\tau^n_{k}, \tau^n_{k+1}]}(t) \\ 
Q^n_t &= \sum_{k=0}^\infty \left (M_{t \wedge \tau^n_{k+1}}  - M_{t \wedge \tau^n_k}\right)^2 
\end{align*}
then we have the identity
\begin{align*}
M^2_t &= 2 \int_0^t V^n \, dM + Q^n_t
\end{align*}
and the convergence results that $V^n \toas M$ and $\sup_{0 \leq t < \infty} \abs{Q^n_t - [M]_t} \toprob 0$.  While in the proof of Theorem \ref{OptionalQuadraticCovariation} we weren't in a position to discuss the convergence of $\int_0^t V^n \, dM$ we now note that in addition we have $\abs{V^n_t} \leq \sup_{0 \leq s \leq t} \abs{M_s} < \infty$ so we can apply Lemma \ref{DominatedConvergenceContinuousSemimartingale} to conclude that 
\begin{align*}
\sup_{0 \leq s \leq t} \abs{\int_0^s V^n \, dM - \int_0^t M \, dM} \toprob 0
\end{align*}
for all $t \geq 0$.  So we have $Q^n_t \toas [M]_t$ and $\int_0^s V^n \, dM  \toas \int_0^t M \, dM$ along a common subsequence and therefore $M^2_t = 2 \int_0^t M \, dM + [M]_t$ almost surely.  For an arbitrary continuous local martingale $M$ we take a localizing sequence $\tau_n$ such that each $M^{\tau_n}$ is bounded (Lemma \ref{ContinuousLocalMartingaleLocalizeToBounded}) then using the result for bounded $M$, Lemma \ref{StoppingIntegralsContinuousSemimartingale} and Theorem \ref{OptionalQuadraticCovariation} we have for each $t \geq 0$, almost surely
\begin{align*}
M_t^2 &= \lim_{n \to \infty} M^2_{t \wedge \tau_n} = \lim_{n \to \infty} 2 \int_0^{t} M^{\tau_n} \, dM^{\tau_n} + [M^{\tau_n}]_t \\
&=\lim_{n \to \infty} \left( 2 \int_0^{t \wedge \tau_n} M \, dM + [M]_{t \wedge \tau_n} \right) = 2 \int_0^{t} M \, dM + [M]_{t} 
\end{align*}

Note that by Tonelli's Theorem we know that for any measurable space $S$, any $\sigma$-finite measure $\mu$ and any positive measurable function $f : S \times S \to \reals_+$ we have
\begin{align*}
\iint f(x,y) \, d\mu(x) \otimes d\mu(y) &= \int \left [ \int f(x,y) \, d\mu(y) \right] \, d\mu(x) \\
&= \int \left [ \int f(y,x) \, d\mu(x) \right] \, d\mu(y)  = \iint f(y,x) \, d\mu(x) \otimes d\mu(y)
\end{align*}
so in particular the product measure is invariant under reflection along the diagonal. Using this fact, for $X = A$ with $A$ of locally finite variation and $A_0 = 0$, we have by definition $[A] = 0$ and 
\begin{align*}
A_t^2 &= \int_0^t \int_0^t dA(u) \otimes dA(v) =2\int_0^t \left[\int_0^u \, dA(v)\right] \, dA(u) \ = 2 \int_0^t A(u) \, dA(u)
\end{align*}
so the result holds for Stieltjes integrals.

Now assume that $X = M+A$ is a continuous semimartingale with $X_0 = 0$.  Using the results for the continuous local martingale case and the Stieltjes integral case we have
\begin{align*}
X^2 &= M^2 + A^2 + 2MA = 2 \int M \, dM + 2 \int A_s \, dA_s + [M] + 2MA\\
&=2 \int X \, dX - 2 \int A \, dM - 2 \int M_s \, dA_s + [X] + 2MA 
\end{align*}
so the result will follow if we can show that $MA = \int A \, dM + \int M_s \, dA_s$ almost surely.  For this we can proceed by defining approximations.  Fix a $t \geq 0$ and for each $n > 0$ define processes $A^n_s = A_{(k-1)t/n}$ and $M^n_s = M_{tk/n}$ for $s \in (t(k-1)/n, tk/n]$.  Note $A^n$ is a predictable step process by construction and that
\begin{align*}
&\int_0^t A^n \, dM + \int_0^t M^n_s \, dA_s \\
&= \sum_{k=1}^n A_{t(k-1)/n} \left(M_{tk/n} - M_{t(k-1)/n}\right) + \sum_{k=1}^n M_{kt/n} \left(A_{tk/n} - A_{t(k-1)/n}\right) \\
&=A_t M_t
\end{align*}
for every $n > 0$.  We have $A^n \toas A$ by continuity of $A$ and therefore $\sup_{0 \leq s \leq t} \abs{\int_0^s A^n \, dM - \int A \, dM} \toprob 0$ by Lemma \ref{LimitsOfStochasticIntegralContinuousLocalMartingale} (TODO: we need domination!) and $M^n \toas M$ and therefore $\int_0^t M^n_s \, dA_s \to \int_0^t M_s \, dA_s$ by Dominated Convergence applied pointwise (TODO: We need domination!).

Now we remove the assumption $X_0 = 0$.  Applying the result proven to $X - X_0$, we have
\begin{align*}
X^2 &= (X-X_0)^2 + 2X_0X - X_0^2 = 2 \int (X - X_0) \, d(X - X_0) + [X-X_0] + 2X_0 X - X_0^2 \\
&= 2 \int X \, dX - 2X_0 (X-X_0) + [X] + 2X_0 X - X_0^2 = X_0^2 + 2 \int X \, dX + [X]
\end{align*}

Lastly, we perform the polarization to extend to general $X$ and $Y$, using bilinearity of the stochastic integral and bilinearity and symmetry of the quadratic covariation,
\begin{align*}
XY &= \frac{1}{4} \left( (X + Y)^2 - (X-Y)^2 \right) \\
&=\frac{1}{4} \bigl( (X_0 + Y_0)^2 + 2\int (X+Y) \, d(X+Y) + [X+Y] \\
&- (X_0-Y_0)^2 - 2\int (X-Y) d(X-Y) - [X-Y] \bigr) \\
&=X_0 Y_0 + \int X \, dY + \int Y \, dX + [X,Y]
\end{align*}
\end{proof}

The following Theorem shows that the class of continuous semimartingales is closed under composition sufficiently smooth functions and provides a means of computing many stochastic integrals.  It is probably the most important theorem in stochastic calculus.
\begin{thm}[Ito's Lemma]\label{ItoLemmaContinuousSemimartingale}Let $X$ be a continuous semimartingale and let $f \in C^2(\reals)$ then almost surely
\begin{align*}
f(X) &= f(X_0) + \int f^\prime(X) \, dX + \frac{1}{2} \int f^{\prime \prime}(X) (s) \, d[X](s)
\end{align*}
\end{thm}
\begin{proof}
Let $\mathcal{C}$ be set of all functions for which the result holds.
First we show that $\mathcal{C}$ contains all polynomials and then extend to smooth functions via an approximation argument.  It is trivial that it is true for $f=c$ a constant and for $f(x) = x$ the result is simply the fact that $\int_0^t dX = X - X_0$.  To see that $\mathcal{C}$ contains all polynomials, if suffices to show that $\mathcal{C}$ is an algebra.  Suppose that $f,g \in \mathcal{C}$, using integration by parts Theorem \ref{IntegrationByPartsContinuousSemimartingale}, the Chain Rule Lemma \ref{ChainRuleContinuousSemimartingale} and the defining property of stochastic integrals, we get almost surely
\begin{align*}
&f (X) g(X) - f(X_0) g(X_0) = \int f(X) \, dg(X) + \int g(X) \, df(X) + [f(X), g(X)] \\
&=\int f(X) \, d\int g^{\prime}(X) \, dX + \frac{1}{2} \int f(X) \, d\int g^{\prime \prime}(X)(s) \, d[X](s) \\
&+ \int g(X) \, d\int f^{\prime}(X) \, dX + \frac{1}{2} \int g(X) \, d\int f^{\prime \prime}(X)(s) \, d[X](s) \\
&+ [\int f^{\prime}(X)\, dX + \frac{1}{2}\int f^{\prime \prime}(X)(s) \, d[X](s) , \int g^{\prime} (X) \, dX + \frac{1}{2} \int g^{\prime\prime}(X)(s) \, d[X](s)] \\
&=\int f(X)  g^{\prime}(X) \, dX + \frac{1}{2} \int f(X) g^{\prime \prime}(X) (s) \, d[X](s) + \int g(X)  f^{\prime}(X) \, dX\\
&+ \frac{1}{2} \int g(X) f^{\prime \prime}(X)(s) \, d[X](s) + [\int f^{\prime}(X)\, dX , \int g^{\prime} (X) \, dX] \\
&=\int (fg)^{\prime}(X) \, dX + \frac{1}{2} \int f(X) g^{\prime \prime}(X) (s) \, d[X](s) + \frac{1}{2} \int g(X) f^{\prime \prime}(X)(s) \, d[X](s) \\
&+ \int f^{\prime}(X) g^{\prime} (X) (s) \, d[X](s) \\
&=\int (fg)^{\prime}(X) \, dX  + \frac{1}{2} \int (fg)^{\prime \prime}(X)(s) \, d[X](s) 
\end{align*}

Now suppose that we have $f \in C^2(\reals)$.  Let $t \geq 0$ be fixed and by the Weierstrass Approximation Theorem (Corollary \ref{WeierstrassApproximation}) (TODO: We actually need approximation in $C((-\infty, \infty); \reals)$; i.e. uniform approximation on compact sets) find a polynomials $q_n(x)$ such that $q_n$ uniformly approximates $f^{\prime \prime}(x)$ on every interval $[-c,c]$.  Taking two antiderivatives of each $q_n(x)$ we get polynomials $p_n(x)$ such that 
\begin{align*}
\lim_{n \to \infty} \sup_{-c \leq x \leq c} \abs{f(x) - p_n(x)} \vee \abs{f^{\prime}(x) - p^{\prime}_n(x)}  \vee \abs{f^{\prime\prime}(x) - p^{\prime\prime}_n(x)} &= 0
\end{align*}
for every $t \geq 0$.  In particular, $p_n(X_t(\omega)) \to f(X_t(\omega))$ for every $t \geq 0$ and $\omega \in \Omega$.
TODO: Finish
\end{proof}

\begin{lem}\label{ApproximateOptionalQuadraticCovariationContinuousSemimartingale}Let $X$ and $Y$ be continuous semimartingales, let $t \geq 0$ be fixed and suppose that we have a sequence of partitions $0=t_{n,0} < t_{n,1} < \dotsb < t_{n, k_n}=t$ such that $\lim_{n \to \infty} \max_{1 \leq k \leq k_n} (t_{n,k} - t_{n, k-1}) = 0$, then 
\begin{align*}
\sum_{k=1}^{k_n} (X_{n,k} - X_{n,k-1}) (Y_{n,k} - Y_{n,k-1}) \toprob [X,Y]_t
\end{align*}
\end{lem}
\begin{proof}
Using $[X,Y] = [X-X_0, Y-Y_0]$ it is immediate that we may assume $X_0 = Y_0 = 0$.  Given the partition $0=t_{n,0} < t_{n,1} < \dotsb < t_{n, k_n}=t$ we define predicatable step processes $X^n_s = \sum_{k=1}^{k_n} X_{t_{k-1}} \characteristic{(t_{k-1}, t_k]}(s)$ and $Y^n_s = \sum_{k=1}^{k_n} Y_{t_{k-1}} \characteristic{(t_{k-1}, t_k]}(s)$.  By a little algebra using the fact that the integrals $\int X^n \, dY$ and $\int Y^n \, dX$ are given by Riemann sums we see
\begin{align*}
&\sum_{k=1}^{k_n} (X_{n,k} - X_{n,k-1}) (Y_{n,k} - Y_{n,k-1}) \\
&= \sum_{k=1}^{k_n} X_{n,k} (Y_{n,k} - Y_{n,k-1})  - \int_0^t X^n \, dY \\
&= \sum_{k=1}^{k_n}( X_{n,k} Y_{n,k} - X_{n,k-1} Y_{n,k-1})  - \int_0^t X^n \, dY - \int_0^t Y^n \, dX \\
&= X_t Y_t - \int_0^t X^n \, dY - \int_0^t Y^n \, dX \\
\end{align*}
By continuity of $X$ and $Y$ we see that $X^n \toas X$ and $X^n_t\leq X^*_t< \infty$.  Since $X$ is continuous the same is true of $X^*$ hence $X^* \in L(Y)$, therefore by may apply Lemma \ref{DominatedConvergenceContinuousSemimartingale} to conclude that $\int_0^t X^n \, dY \toprob \int_0^t X \, dY$.  In exactly the same way we see that $\int_0^t Y^n \, dX \toprob \int_0^t Y \, dX$.  Now we can apply integration by parts Lemma \ref{IntegrationByPartsContinuousSemimartingale} to conclude that 
\begin{align*}
\int_0^t X^n \, dY \toprob \int_0^t X \, dY \toprob X_t Y_t - \int_0^t X \, dY - \int_0^t Y \, dX 
\end{align*}
\end{proof}

\subsection{Approximation By Step Processes}

We defined the stochastic integral in an elegant but somewhat abstract way as the representative of a linear functional on a Hilbert space.  The uniqueness property of the integral showed us that this definition was consistent with intuitively clear defintion of the stochastic integral for step process integrands as Riemann sums.  The uniqueness property of the stochastic integral has shown itself to be a very useful technical tool but is lacking somewhat in intuitive appeal.  We repair this deficiency by showing that the continuity properties of the stochastic integral also characterize the extension from step process integrands.  To see this requires that we understand the approximation by step processes in the spaces $L(M)$.  We note that these approximation results also lead to an alternative path to defining the stochastic integral in the first place.  

\begin{lem}Let $X$ be a continuous semimartingale with canonical decomposition $X = M+A$ and let $V \in L(X)$.  Then there exists processes $V_1, V_2, \dotsc \in \mathcal{E}$ such that almost surely $\lim_{n \to \infty} \int_0^t (V_n -V)^2(s) \, d[M](s) = 0$ and $\lim_{n \to \infty} \sup_{0 \leq s \leq t} \abs{\int_0^s (V_n-V)(u) \, dA(u) } = 0$ for all $t \geq 0$.
\end{lem}
\begin{proof}
TODO: A bunch of stuff

Now suppose that $A$ is a strictly increasing, continuous and adapted process with $A_0=0$.  If one thinks for a moment about the case in which $A_t = t$ then it is more or less clear how to approximate any integrable function $f$ by a continuous one: just define $f^h(t) = \frac{1}{h}\int_{t-h}^t f(s) \, ds$ for $h > 0$ and note that by the Fundamental Theorem of Calculus for almost all $t$ we have $\lim_{h \to 0^+} f^h(t) = f(t)$.  If we treat a general Stieltjes integral then we just have to use the fact that every Lebesgue-Stieltjes measure is of the form $\pushforward{G}{\lambda}$.  Specifically, from the proof of Lemma \ref{LebesgueStieltjesMeasure} recall that if $F$ is nondecreasing and right continuous then the Lebesgue-Stieltjes measure associated with $F$ is given by $\pushforward{G}{\lambda}$ where
$G(t) = \sup\lbrace s \mid F(s) < t \rbrace$.
Let us apply this to our process $A$ pointwise by defining the process, $T_t = \sup \lbrace s \geq 0 \mid A_s < t \rbrace$ for $t \geq 0$.  TODO: Show $T$ is a process.  Because we have assumed that $A$ is strictly increasing, $T$ is strictly increasing and is an actual inverse satisfying $T(A(t)) = A(T(t)) = t$.  We can now define the approximation for $h > 0$ and $t > 0$,
\begin{align*}
V^h_t &= \frac{1}{h} \int_{T((A_t-h) \vee 0)}^t V(s) \, dA(s) = \frac{1}{h} \int_{(A_t-h) \vee 0}^{A_t} V(T(s)) \, ds
\end{align*}
where we have used the change of variables Lemma \ref{ChangeOfVariables} and the fact that $T((A_t - h) \vee 0)\leq T(s) \leq t$ if and only if $(A_t - h) \vee 0 \leq s \leq A(t)$.  TODO: What about $t=0$?  Having expressed the definition of $V^h_t$ in terms of an ordinary Lebesgue integral, we can apply the Fundamental Theorem of Calculus to see that 
\begin{align*}
\lim_{h \to 0} V^h (T(t)) &= \lim_{h \to 0} \frac{1}{h} \int_{(t-h) \vee 0}^{t} V(T(s)) \, ds = V(T(t))
\end{align*} 
for almost all $0 \leq t \leq A_1$.  Now we can apply the Dominated Convergence Theorem to conclude 
\begin{align*}
\lim_{h \to 0} \int_0^1 \abs{V^h_s - V_s} \, dA_s &= \lim_{h \to 0} \int_0^{A_1} \abs{V^h(T(s)) - V(T(s))} \, ds = 0
\end{align*}
\end{proof}

\begin{lem}\label{SimpleProcessApproximationBoundedLebesgue}Let $V$ be a bounded $\mathcal{F}$-adapted process then there exist $V^n \in \mathcal{E}$ such that 
\begin{align*}
\sup_{0 \leq T < \infty} \lim_{n \to \infty} \expectation{\int_0^T \abs{V_s - V_s^n}^2 \, ds} = 0
\end{align*}
\end{lem}
\begin{proof}
First fix a $T \geq 0$ and we will approximate on the interval $[0,T]$.  It is also notationally convenient to set $V_t = 0$ for all $t < 0$ in what follows.  Set up the following family of approximations; for every $s \geq 0$ and $n \in \naturals$ define
\begin{align*}
V^{(n,s)}_t(\omega)  &= \sum_{j=0}^{\ceil{2^n T}} V_{j/2^n + s}(\omega) \characteristic{(j/2^n + s, (j+1)/2^n + s]}(t) \characteristic{[0,T]}(t)
\end{align*}
Note that $V^{(n,s)} \in \mathcal{E}$ and moreover it is jointly measurable in $(s,t,\omega)$.  Also note that $V^{(n,s)}_t = V^{(n, s+1/2^n)}_t$ for all $s \geq 0$ and all $t \geq 0$.

Claim: Let $f \in L^2([0,T])$ then $\lim_{h \downarrow 0} \int_0^T (f(s) - f((s - h) \vee 0))^2 \, ds= 0$.

By Lemma \ref{LpApproximationByContinuous} we can find bounded continuous $f_n$ such that $f_n \tolp{2} f$.  By the triangle inequality, continuity of $f_n$, Dominated Convergence and the translation invariance of Lebesgue measure we get for every $n$
\begin{align*}
&\lim_{h \downarrow 0} \left(\int_0^T (f(s) - f((s - h) \vee 0))^2 \, ds\right)^{1/2} \\
&\leq \left(\int_0^T (f(s) -f_n(s))^2 \, ds\right)^{1/2} + \\
&\lim_{h \downarrow 0} \left(\int_0^T (f_n(s) - f_n((s - h) \vee 0))^2 \, ds\right)^{1/2} + \\
&\lim_{h \downarrow 0} \left(\int_0^T (f_n((s-h) \vee 0) - f((s - h) \vee 0))^2 \, ds\right)^{1/2} \\
&=\left(\int_0^T (f(s) -f_n(s))^2 \, ds\right)^{1/2} + \lim_{h \downarrow 0} \left(\int_0^{T-h} (f_n(s) - f(s))^2 \, ds\right)^{1/2}\\
&\leq 2 \norm{f - f_n}_2
\end{align*}
so we now take the limit as $n \to \infty$.

It is a simple matter to extend this result to a bounded adapted process $V$.  In this case we know that $\int_0^T (V_s - V_{(s-h) \vee 0})^2 \, ds$ is bounded and therefore we conclude from Dominated Convergence and the result on $L^2([0,T])$ that
\begin{align*}
\lim_{h \downarrow 0} \expectation{\int_0^T (V_s - V_{(s-h) \vee 0})^2 \, ds} &= \expectation{\lim_{h \downarrow 0} \int_0^T (V_s - V_{(s-h) \vee 0})^2 \, ds} = 0
\end{align*}

Claim: $\lim_{n \to \infty } \expectation {\int_0^T \int_0^1 (V^{(n,s)}_t - V_t)^2 \, ds \, dt} = 0$.

First off, from $V^{(n,s)}_t = V^{(n, s+1/2^n)}_t$, the definition of $V^{(n,s)}_t$ and a change of integration variable we write
\begin{align*}
\int_0^1 (V^{(n,s)}_t - V_t)^2 \, ds &= 2^n \int_0^{2^{-n}}  (V^{(n,s)}_t - V_t)^2 \, ds = 2^n \int_{t-2^{-n}}^t  (V_s - V_t)^2 \, ds = 2^n \int_0^{2^{-n}}  (V_t - V_{t - h})^2 \, dh
\end{align*}
Now using this fact and Tonelli's Theorem
\begin{align*}
\expectation {\int_0^T \int_0^1 (V^{(n,s)}_t - V_t)^2 \, ds \, dt}  &= 2^n \int_0^{2^{-n}} \expectation {\int_0^T (V_t - V_{t -h})^2 \, dt}  \, dh
\end{align*}
By the previous claim for any $\epsilon > 0$ we can find $N>0$ such that $\expectation {\int_0^T (V_t - V_{t -h})^2 \, dt} < \epsilon$ for all $0 \leq h \leq 2^{-N}$ and therefore for all $0 \leq h \leq 2^{-n}$ for any $n \geq N$.  Thus for any $n \geq N$ we have $\expectation {\int_0^T \int_0^1 (V^{(n,s)}_t - V_t)^2 \, ds \, dt} < \epsilon$ and the claim is shown by letting $\epsilon \to 0$.

TODO: Make sure we deal with the boundary at $0$ consistently (we're not at the moment).

Viewing $\expectation {\int_0^n (V^{n}_t - V_t)^2 \, dt}$ as a random variable on the probability space $([0,1], \mathcal{B}([0,1]), \lambda)$  and applying Tonelli's Theorem to previous claim, conclude $\expectation {\int_0^T (V^{(n,s)}_t - V_t)^2 \, dt} \tolp{1} 0$ which implies $\expectation {\int_0^T (V^{(n,s)}_t - V_t)^2 \, dt} \toas 0$ along some subsequence $N \subset \naturals$ (Lemma \ref{ConvergenceInMeanImpliesInProbability}  and Lemma \ref{ConvergenceInProbabilityAlmostSureSubsequence}).  Pick any $s \in [0,1]$ where the subsequence converges.

To finish the proof, for each $n \in \naturals$ we apply the result for fixed $T=n$ and find an element $V^n \in \mathcal{E}$ such that $\expectation {\int_0^n (V^{n}_t - V_t)^2 \, dt} < 1/n$.  Then given $T > 0$ and any $\epsilon > 0$ it holds for any $n > \epsilon^{-1} \vee T$ that 
\begin{align*}
\expectation {\int_0^T (V^{n}_t - V_t)^2 \, dt} &< \expectation {\int_0^n (V^{n}_t - V_t)^2 \, dt} < 1/n < \epsilon
\end{align*}
and the result is proven.
\end{proof}

\begin{lem}\label{SimpleProcessApproximationPredictableStepProcess}Let $A$ be a non-decreasing, continuous and $\mathcal{F}$-adapted process such that $A_0 = 0$ and $\expectation{A_t} < \infty$ for all $t \geq 0$.  Let $\sigma$ and $\tau$ be bounded $\mathcal{F}$-optional times such that $\sigma \leq \tau$ and $\xi$ be an $\mathcal{F}_\sigma$-measurable bounded random variable.
Then there exist $V^n \in \mathcal{E}$ such that 
\begin{align*}
\sup_{0 \leq T < \infty} \lim_{n \to \infty} \expectation{\int_0^T \abs{\xi \characteristic{(\sigma, \tau]}(s) - V^n(s)}^2 \, dA(s)} = 0
\end{align*}
\end{lem}
\begin{proof}
Take the standard discrete approximation of optional times $\tau_n = \frac{1}{2^n} \floor{2^n \tau + 1}$ and $\sigma_n = \frac{1}{2^n} \floor{2^n \sigma + 1}$ (Lemma \ref{DiscreteApproximationOptionalTimes}) so that $\tau_n \downarrow \tau$ and $\sigma_n \downarrow \sigma$.  Note that $s \in (\sigma_n, \tau_n]$ if and only if there exists a $k$ such that $\tau_n \geq k/2^n$, $\sigma_n \leq (k-1)/2^n$ and $(k-1)/2^n < s \leq k/2^n$.  As $\tau_n = k/2^n$ when $(k-1)/2^n \leq \tau < k/2^n$ and likewise for $\sigma_n$ we see that $\tau_n \geq k/2^n$ is equivalent to $\tau_n \geq (k-1)/2^n$ and $\sigma_n \leq (k-1)/2^n$ is equivalent to $\sigma < (k-1)/2^n$. From these facts and the boundedness of $\tau$ we see that 
\begin{align*}
\characteristic{(\sigma_n, \tau_n]}(s) &= \sum_{k=1}^N \characteristic{\lbrace \sigma < (k-1)/2^n \leq \tau \rbrace} \characteristic{ ((k-1)/2^n, k/2^n]}(s)
\end{align*}
for some large $N$.  Now we define 
\begin{align*}
V^n &= \xi \characteristic{(\sigma_n, \tau_n]}(s) = \sum_{k=1}^N \xi \characteristic{\lbrace \sigma < (k-1)/2^n \leq \tau \rbrace} \characteristic{ ((k-1)/2^n, k/2^n]}(s)
\end{align*}
and claim that $V^n \in \mathcal{E}$.  

Lastly we note that because $\sigma < \sigma_n \leq \tau < \tau_n$ and $\xi$ is bounded (say $\abs{\xi} \leq K$) we get
\begin{align*}
\expectation{\int_0^T \abs{\xi \characteristic{(\sigma, \tau]}(s) - V^n(s)}^2 \, dA(s)} &=\expectation{\xi^2 \int_0^T (\characteristic{(\sigma, \tau]}(s) - \characteristic{(\sigma_n, \tau_n]}(s))^2 \, dA(s)} \\
&=\expectation{\xi^2 (A_{\tau_n} - A_\tau)} + \expectation{\xi^2 (A_{\sigma_n} - A_\sigma)} \\
&\leq K^2 \expectation{ (A_{\tau_n} - A_\tau)} + \expectation{(A_{\sigma_n} - A_\sigma)} \\
\end{align*}
If we let $C$ be a bound for $\tau$, it follows that $\tau_n$ is bounded by $C+1$ for all $n$ and by the non-decreasingness of $A$ we have $\abs{A_{\tau_n} - A_\tau} \leq 2A_{C+1}$ and similarly with $\sigma$, therefore by Dominated Convergence we get $\lim_{n \to \infty} \expectation{\int_0^T \abs{\xi \characteristic{(\sigma, \tau]}(s) - V^n(s)}^2 \, dA(s)}=0$.

TODO: If we need the sup over $T$ then we have that argument elsewhere; check if we really use it.
\end{proof}

\begin{lem}Let $A$ be a non-decreasing, continuous and $\mathcal{F}$-adapted process with $A_0=0$ and $\expectation{A_t} < \infty$ for all $t \geq 0$.  Let $V$ be an $\mathcal{F}$-progressively measurable process such that 
\begin{align*}
\expectation{\int_0^t V^2_s \, dA_s} < \infty
\end{align*}
for every $t \geq 0$, then 
then there exist $V^n \in \mathcal{E}$ such that 
\begin{align*}
\sup_{0 \leq T < \infty} \lim_{m \to \infty} \expectation{\int_0^T \abs{V(s) - V^n(s)}^2 \, dA(s)} = 0
\end{align*}
\end{lem}
\begin{proof}
Pick a $T \geq 0$ fixed and assume that $V_t = 0$ for all $t > T$ and that $V_t(\omega) \leq C$ for all $t \geq 0$ and $\omega \in \Omega$.  Now we want to use the fact that a Lebesgue-Stieltjes integral can be reduced to an ordinary Lebesgue integral via change of variables: this will allow us to use Lemma \ref{SimpleProcessApproximationBoundedLebesgue}.  To make dealing with the change of variables a bit easier, consider $A_s + s$ which is a strictly increasing function; in this case we have genuine inverse $T_s$ that is increasing.  Moreover since $A_{T_s}+T_s = s$ and $A_s \geq 0$ we have $T_s \leq s$ and from the increasingness of $T_s$ we have $\lbrace T_s \leq t \rbrace = \lbrace s \leq A_t + t\rbrace \in \mathcal{F}_t$; so in particular, each $T_s$ is a bounded $\mathcal{F}$-optional time.  Now define the process $W_s = V_{T_s}$ and the filtration $\mathcal{G}_s = \mathcal{F}_{T_s}$ and note that by $\mathcal{F}$-progressive measurability of $V$ and Lemma \ref{StoppedProgressivelyMeasurableProcess} we know that $W_s$ is $\mathcal{G}$-adapted.  Also we compute
\begin{align*}
\expectation { \int_0^\infty W_s^2 \, ds} &= \expectation { \int_0^\infty \characteristic{T_s \leq T} (s) V_{T_s}^2 \, ds} = \expectation { \int_0^{A_T + T} V_{T_s}^2 \, ds} \leq C ( \expectation{A_T}+T) < \infty
\end{align*}
so that in particular $\lim_{R \to \infty} \expectation { \int_R^\infty W_s^2 \, ds} = 0$.  
By our boundedness assumption and Lemma  \ref{SimpleProcessApproximationBoundedLebesgue} we know that we can approximate $W$ by $\mathcal{G}$-predicatable step processes with deterministic jump times.  Thus if we let $\epsilon > 0$ then we can find $R>0$ such that $\expectation { \int_R^\infty W_s^2 \, ds} < \epsilon/2$ and $W^\epsilon_s = \xi_0 \characteristic{\lbrace 0 \rbrace}(s) + \sum_{j=1}^n \xi_j \characteristic{(s_{j-1}, s_j]}(s)$ such that 
$\expectation{\int_0^R \abs{W_s - W_s^\epsilon}^2 \, ds} < \epsilon/2$ and by defining $W^\epsilon_s = 0$ for $s > R$ we have
\begin{align*}
\expectation{\int_0^\infty \abs{W_s - W_s^\epsilon}^2 \, ds} &= \expectation{\int_0^R \abs{W_s - W_s^\epsilon}^2 \, ds}  + \expectation{\int_R^\infty W_s^2 \, ds} < \epsilon
\end{align*}
Now we undo our change of variables to see what type of approximation we have of $V$.  Let 
\begin{align*}
V^\epsilon_s &= W^\epsilon_{A_s+s} = \xi_0 \characteristic{\lbrace 0 \rbrace}(A_s+s) + \sum_{j=1}^n \xi_j \characteristic{(s_{j-1}, s_j]}(A_s+s) \\
&=\xi_0 \characteristic{\lbrace 0 \rbrace}(s) + \sum_{j=1}^n \xi_j \characteristic{(T_{s_{j-1}}, T_{s_j}]}(s) 
\end{align*}
we claim that $V^\epsilon$ is $\mathcal{F}$-adapted.  This follows from the fact that $\xi_j$ is $\mathcal{F}_{s_{j-1}}$-measurable and for any $u > 0$ and $j \geq 1$, 
\begin{align*}
\lbrace \xi_j \characteristic{(T_{s_{j-1}}, T_{s_j}]}(s) \leq u \rbrace &= \lbrace \xi_j \leq u \rbrace \cap \lbrace T_{s_{j-1}} < s \rbrace \cap \lbrace s \leq T_{s_j}  \rbrace \in \mathcal{F}_s
\end{align*}
TODO: Why is $\lbrace s \leq T_{s_j}  \rbrace \in \mathcal{F}_s$?
Moreover, by the construction of Stieltjes integral we have
\begin{align*}
\expectation{\int_0^T \abs{V_s - V^\epsilon_s}^2 \, dA_s} &\leq \expectation{\int_0^\infty \abs{V_s - V^\epsilon_s}^2 \, d(A_s + s)} \\
&=\expectation{\int_0^\infty \abs{W_s - W^\epsilon_s}^2 \, ds} < \epsilon
\end{align*}
 We are not quite done as $V^\epsilon$ has random jump times.  However, we can apply Lemma \ref{SimpleProcessApproximationPredictableStepProcess} to find $V^{(m,n)} \in \mathcal{E}$ such that $\lim_{m \to \infty} \expectation{\int_0^T \abs{V^{1/n}_s -V^{(m.n)}_s}^2 \, dA_s} = 0$ and then we find a $V^{(m_n,n)}$ such that $\lim_{n \to \infty} \expectation{\int_0^T \abs{V_s -V^{(m_n.n)}_s}^2 \, dA_s} = 0$.

Now we remove the assumption that $V$ is bounded.  For a general $V_s$ such that $\expectation { \int_0^T V_s^2 \, dA_s} < \infty$, let $V^n_s = V_s \characteristic{\abs{V_s} \leq n}$ where by the Dominated Convergence Theorem we know that $\expectation { \int_0^T \abs{V_s - V_s^n}^2 \, dA_s} = 0$.  Since each $V_s^n$ is bounded we can find a sequence $V^{(n,m)}_s$ such that $\lim_{m \to \infty} \expectation { \int_0^T \abs{V^n_s - V_s^{(n,m)}}^2 \, dA_s} = 0$ and now an array argument shows we get a subsequence $V^{(n,m_n)}$ such that $\lim_{n \to \infty} \expectation { \int_0^T \abs{V^{(n,m_n)}_s - V_s}^2 \, dA_s} = 0$.


Lastly it remains to remove the assumption that we are dealing with a fixed $T \geq 0$.  By what we have proven thus far, if $V$ is such that $\expectation { \int_0^t V_s^2 \, dA_s} < \infty$ for all $t \geq 0$, then for each $m > 0$ we have a sequence $V^{(n,m)} \in \mathcal{E}$ such that $\lim_{n \to \infty} \expectation{\int_0^m \abs{V_s - V^{(n,m)}_s}^2 \, dA_s} = 0$, so in particular there is $n_m$ such that $\expectation{\int_0^m \abs{V_s - V^{(n_m,m)}_s}^2 \, dA_s} < \frac{1}{m}$.  If we let $V_s^m = V_s^{(n_m,m)}$ then
for every $T > 0$, 
\begin{align*}
\lim_{m \to \infty} \expectation{\int_0^T \abs{V_s - V^{(n_m,m)}_s}^2 \, dA_s} &\leq \lim_{m \to \infty} \expectation{\int_0^m \abs{V_s - V^{(n_m,m)}_s}^2 \, dA_s} = 0
\end{align*}
and thus $\sup_{0 \leq T < \infty} \lim_{m \to \infty} \expectation{\int_0^T \abs{V_s - V^{(n_m,m)}_s}^2 \, dA_s} = 0$ and we are finally done.
\end{proof}

\chapter{More Real Analysis}
Holding area for more advanced topics in real analysis that are
eventually required (and in some cases there may be some topics that I
am just interested in).
\section{Topological Spaces}
\begin{lem}\label{OpenAlternative}A set $U \subset X$ is open if and only if for every $x \in
  U$ there is an open set $V \subset U$ such that $x \in V$.
\end{lem}
\begin{proof}
Suppose $U$ is open and $x \in U$, then let $V = U$.

Suppose for every $x \in U$ there exist an open set $V_x$ such that $x
\in V_x \subset U$.  Note that $\cup_x V_x \subset U$ because each
$V_x \subset U$ and on the other hand $\cup_x V_x \supset U$ since
every $x \in U$ satisfies $x \in V_x$.  Thus $U = \cup_x V_x$ which
shows that $U$ is open.
\end{proof}
\begin{defn}A mapping $f : X \to Y$ between topological spaces is said
  to be \emph{continuous} if and only if $f^{-1}(V)$ is open in $X$
  for every $V$ open in $Y$.
\end{defn}
\begin{defn}A mapping $f : X \to Y$ between topological spaces is said
  to be \emph{continuous at x} if and only if for every $V$ open in
  $Y$ such that $f(x) \in V$, there exists an open set $U$ in $X$ with $x \in U$ and $f(U)
  \subset V$.
\end{defn}
\begin{lem}A mapping $f : X \to Y$ between topological spaces is
  continuous if and only if it is continuous at $x$ for every $x \in X$.
\end{lem}
\begin{proof}
Suppose $f$ is continuous and let $x \in X$ and $V$ be open in $Y$
with $f(x) \in V$.  By continuity of $f$, we know that $f^{-1}(V)$ is
open in $X$ and $x \in f^{-1}(V)$.  By Lemma \ref{OpenAlternative} we
can pick an open set $U$ such that $x \in U$ and $U \subset
f^{-1}(V)$.  It follows that $f(U) \subset V$.

Now suppose $f$ is continuous at every $x \in X$ and let $V$ be open
in $Y$.  If $x \in f^{-1}(V)$ then $f$ is continuous at $x$ hence
there exists and open $U$ such that $x \in U$ and $f(U) \subset V$.
It follows that $U \subset f^{-1}(V)$ and by Lemma
\ref{OpenAlternative}  we have shown that $f^{-1}(V)$ is open.
\end{proof}

\begin{defn}A \emph{base} of a topology $\mathcal{T}$ at a point $x
  \in X$ is a collection
  of sets $\mathcal{B}$ such that for every open set $U \in
  \mathcal{T}$ such that $x \in U$ there exists a $B \in \mathcal{B}$ such
  that $x \in B \subset U$.  A base of a topology is a collection of
  sets that is a base at all points $x \in X$.
\end{defn}

\begin{lem}A set $\mathcal{B}$ of sets $B \subset X$ is a base of a
  topology if and only if for every $x \in X$ there exists $B \in
  \mathcal{B}$ such that $x \in B$ and for every $A, B \in
  \mathcal{B}$ and $x \in A \cap B$ there exists $C \in \mathcal{B}$
  such that $x \in C \subset A \cap B$.
\end{lem}
\begin{proof}
Suppose $\mathcal{B}$ satisfies the hypothesized conditions and let
\begin{align*}
\tau &= \lbrace U \subset X \mid \text { for every } x \in U \text{
  there exists } B \in \mathcal{B} \text{ such that } x \in B \subset
U \rbrace
\end{align*}
It is certainly the case that $\mathcal{B}\subset \tau$ and we claim that $\tau$ is a topology.  Certainly $\emptyset \in \tau$.
Let $U_\alpha$ for $\alpha \in \Lambda$ are sets in $\tau$.
Then if $x \in \cup_{\alpha \in \Lambda} U_\alpha$ there exists an
$\alpha \in \Lambda$ such that $x \in U_\alpha$ and by hypothesis we
pick $B$ such that $x \in B \subset U_\alpha \subset  \cup_{\alpha \in
  \Lambda} U_\alpha$.  If $U_1, \dotsc, U_n \in \tau$ and $x \in U_1
\cap \dotsc \cap U_n$ then there exists $B_1, \dotsc, B_n$ such that
$x \in B_j \subset U_j$ for $j = 1, \dotsc, n$ and therefore $x \in
B_1 \cap \dotsc \cap B_n \subset U_1 \cap \dotsc \cap U_n$.  A simple
induction on the hypothesis shows that $B_1 \cap \dotsc \cap B_n \in \mathcal{B}$.
Because $\mathcal{B}$ is cover of $X$ we have $X = \cup_{B \in
  \mathcal{B}} B \in \tau$ and therefore $\tau$ is a topology.  By the
definition of $\tau$ it is immediate that $\mathcal{B}$ is a base of
the topology.
\end{proof}


\begin{defn}
\begin{itemize}
\item[(i)]A topological space is said to be \emph{separable} if and
  only if it has a countable dense subset.
\item[(ii)]A topological space is said to be \emph{first countable} if and
  only if every point has a countable local base.
\item[(ii)]A topological space is said to be \emph{second countable} if and
  only if every the topology has a countable base.
\end{itemize}
\end{defn}
\begin{lem}A metric space is separable if and only if it is second countable.
\end{lem}
\begin{proof}
TODO:
outline of proof is to pick a countable dense subset $\lbrace x_n
\rbrace$ and then pick the open balls $B(x_n; \frac{1}{m})$ for $m \in
\naturals$.  Show this is a base of the topology.
\end{proof}

TODO: The goal of the next set of results is to show that separable
complete metric spaces are Borel.


The following appears in Royden as Theorem 8.11 (with proof delgated
to exercises)
\begin{lem}Let $X$ be a Hausdorff topological space, $Y$ be a
  complete metric space and $Z \subset X$ be a dense subset.  If $f :
  Z \to X$ is a homeomorphism then $Z$ is a countable intersection of
  open sets.
\end{lem}
\begin{proof}
For each $n$ let 
\begin{align*}
O_n &= \lbrace x \in X \mid \text{there exists $U$
  open with $x \in U$ and $\diam(f(U \cap Z)) < \frac{1}{n}$} \rbrace
\end{align*}
Note that $O_n$ is open because for any $x \in O_n$ by definition we
have the open set $U$ that provides the evidence that $x \in O_n$;
$U$ also provides the evidence that proves that every $y \in U$
belongs to $O_n$.  Also
note that $Z \subset O_n$ since for any $n$, by continuity of $f$ at $x \in Z$ and Lemma
\ref{OpenAlternative}  we
can find an open $U \subset X$ such that $x \in U \cap Z$ and $f(U \cap Z) \subset B(f(x),
\frac{1}{2n})$ (sets of the form $U \cap Z$ being precisely the open
sets in $Z$).

Now define $E = \cap_n O_n$.  As noted we know $Z \subset E$ so we
will be done if we can show $E
\subset Z$ as well.  Let $x \in E$; we will construct $z \in Z$
such that $x = z$.  For each $n$ pick $U_n$ such $x \in U_n$ and $\diam(f(U_n \cap Z)) <
\frac{1}{n}$ and let $x_n$ be an arbitrary point in $\cap_{j=1}^n
U_j \cap Z$ (the intersection is non-empty because $Z$ is dense in
$X$).  
For every $n$ and $m \geq n$ we have by construction that $x_n
\in U_n$ and $x_m \in U_n$ hence $d(f(x_n), f(x_m)) < \frac{1}{n}$.
Therefore $f(x_n)$ is Cauchy in
$Y$ and by completeness of $Y$ we know that $f(x_n)$ converges to a
value $y \in Y$ with $d(y, f(x_n)) \leq \frac{1}{n}$.  
Because $f$ is a homeomorphism we know that 
there is a unique $z \in Z$ such that $f(z) = y$; we claim that $x =
z$.  Suppose that $x
\neq z$, then by the Hausdorff property on $X$ we can pick open sets $U$ and
$V$ such that $U \cap V = \emptyset$, $x \in U$ and $z \in V$.  Since
$f$ is a homeomorphism, we know $f(Z \cap V)$ is open and contains
$f(z)$ hence for sufficiently large $n$, $f^{-1}(B(f(z), \frac{1}{n}))
\subset Z \cap V \subset V$.  On
the other hand, by the definition of $x$ we have $U_{2n}$ open such that
$x \in U_{2n}$ and $\diam(f(Z \cap U_{2n})) < \frac{1}{2n}$.  By openness of
$U \cap U_{2n}$ and density of $Z$ we know there is a $w \in U \cap
U_{2n} \cap Z$.  Putting these observations together we have
\begin{align*}
d(f(w), f(z)) &\leq  d(f(w), f(x_{2n})) + d(f(x_{2n}), f(z)) 
< \frac{1}{2n} + \frac{1}{2n} = \frac{1}{n}
\end{align*}
which implies $w \in V$ providing a contradiction of $U \cap V =
\emptyset$ hence we conclude $x = z$.
\end{proof}

\begin{thm}[Tychonoff's Theorem]\label{Tychonoff}Let $I$ be index set
  and let $(X_i,
  \mathcal{T}_i)$ be a topological space for each $i \in I$, the
  cartesian product $\prod_{i \in I} X_i$ with the product topology is
  compact.
\end{thm}
\begin{proof}
TODO:
\end{proof}

Separation axioms tells us that we have enough open sets in a topology
to distinguish features of the the underlying set (e.g. distinguishing
points from points or closed sets from closed sets).  Another way of
thinking about the size of a topology is by considering the number of
continuous functions that the topology allows.  The following theorem
shows that in normal topological spaces we have enough continuous
functions to approximate indicator functions of closed sets.

\begin{thm}[Uryshon's Lemma]\label{UrysohnsLemma}Let $X$ be a
  topological space, then following are equivalent
\begin{itemize}
\item[(i)]$X$ is normal
\item[(ii)]Given a closed set $F \subset X$ and an open neighborhood
  $F \subset U$ there is an open set $V$ such that $F \subset V
  \subset \overline{V} \subset U$.
\item[(iii)]Given disjoint closed sets $F$ and $G$ there exists a
  continuous function $f : X \to [0,1]$ such that $f \equiv 1$ on $F$
  and $f \equiv 0$ on $G$.
\item[(iv)]Given a closed set $F$ with an open neighborhood $U$ there
  is a continuous function $f$ such that $\characteristic{F}(x) \leq
  f(x) \leq \characteristic{U}(x)$ for all $x \in X$.
\end{itemize}
\end{thm}
\begin{proof}
(i) $\implies$ (ii): Since $U^c$ is and $F \cap U^c = \emptyset$ we
use normality to find disjoint open sets $V$ and $O$ such that $F \subset V$
and $U^c \subset O$.  Note that $\overline{V} \cap U^c = \emptyset$; if $x \in U^c$ then $O$ is an open neighborhood $x$ such that
$O \cap V$ which implies $x \notin \overline{V}$. Therefore we have $F
\subset V \subset \overline{V} \subset U$.

(ii) $\implies$ (i): Let $F$ and $G$ be closed subsets of $X$, it
follows that $G^c$ is open and $F \subset G^c$.  Find an open set $V$
such that $F \subset V \subset \overline{V} \subset G^c$ and observe
that if we define $U = \overline{V}^c$ then we have $V \cap U =
\emptyset$ and $F \subset V$ and $G \subset U$.

(iii) $\implies$ (iv): Construct continuous $f : X \to [0,1]$ such
that $f$ equals $1$ on $F$ and $f$ equals 0 on $U^c$.  Clearly
$\characteristic{F} \leq f$ and $\characteristic{U^c} \leq 1 -f$.  The
latter is equivalent to $f \leq \characteristic{U}$ since
$\characteristic{U^c} = 1 - \characteristic{U}$.

(iv) $\implies$ (iii):  Note that $F \subset G^c$ and construct $f$
such that $\characteristic{F} \leq f \leq \characteristic{G^c}$.  The
first inequality implies that $f \equiv 1$ on $F$ while the second
implies that $f \equiv 0$ on $(G^c)^c = G$.

(iii) $\implies$ (i):  Given $F$ and $G$ and a continuous function $f
: X \to [0,1]$ such that $F \subset f^{-1}(1)$ and $G \subset
f^{-1}(0)$, simply define $U =  f^{-1}(2/3,1]$ and $V = f^{-1}[0,1/3)$
and note that by continuity of $f$ both $U$ and $V$ are open.  

(ii) $\implies$ (iv):  We construct $f$ as a limit of (discontinuous)
indicator functions.  Suppose that $F$ and $U$ are given as in the
hypothesis in (iv).  Define $F_1 = F$ and $U_0 = U$.  Using (ii) we
find an open neighborhood $V$ such that $F_1 \subset V \subset
\overline{V} \subset U$.  Define $F_{1/2} = \overline{V}$ and $U_{1/2}
= V$ so we may rewrite our inclusions as 
\begin{align*}
F_1 &\subset U_{1/2} \subset F_{1/2} \subset U_{0}
\end{align*}
Now we iterate this construction.  To make it clear and to set the
notation for the iteration we turn the crank one more time we apply
(ii) to the pair $F_1 \subset U_{1/2}$ to construct an open set $U_{3/4}$
and closed set $F_{3/4}$ and to the pair $F_{1/2} \subset U_{0}$ to
construct an open set $U_{1/4}$
and closed set $F_{1/4}$ yielding the inclusions
\begin{align*}
F_1 &\subset U_{3/4} \subset F_{3/4} \subset U_{1/2} \subset F_{1/2} \subset U_{1/4} \subset F_{1/4} \subset U_{0}
\end{align*}
Now we induct over the dyadic rationals $\mathcal{D} = \lbrace a/2^n
\mid a \in \naturals \text{ and } n \in \naturals \rbrace \cap (0,1)$ so that we create a sequence
of open and closed sets $U_q$ and $F_q$ satisfying
\begin{itemize}
\item[(i)] $U_q \subset F_q$ for all $q \in \mathcal{D}$
\item[(i)] $F_r \subset U_q$ for all $r,q \in \mathcal{D}$ with $r > q$.
\end{itemize}
Now let $f(x) = \inf \lbrace q \mid x \in U_q \rbrace$.  
TODO: Show that $f$ works...
\end{proof}

\begin{thm}[Tietze's Extension
  Theorem]\label{TietzeExtensionTheorem}Let $F$ be a closed subset of
  a normal topological space, let $a < b$ be real numbers and let $f :
  F \to [a,b]$ be a continuous function.  There exists a continuous
  function $g : X \to [a,b]$ such that $g\mid_F = f$.  If $f : F \to
  \reals$ is a continuous function then there exists a continuous
  function $g: X \to \reals$ such that $g \mid_F = f$.
\end{thm}
\begin{proof}
We begin with the case of $f$ with bounded range.  We construct $g$
via an iterative procedure.  
TODO:
\end{proof}

\begin{defn}Given a topological space $(X, \mathcal{T})$ the Baire
  $\sigma$-algebra is smallest $\sigma$-algebra for which all bounded
  continuous functions are measurable.  Equivalently 
\begin{align*}
Ba(X,\mathcal{T}) &= \sigma(\lbrace f^{-1}(U) \mid U \subset \reals
\text{ is open; } f \in C_b(X,\reals)\rbrace)
\end{align*}
\end{defn}
\begin{lem}For every topological space $(X, \mathcal{T})$, $Ba(X)
  \subset \mathcal{B}(X)$.  For a metric space $(S,d)$, $Ba(S) = \mathcal{B}(S)$.
\end{lem}
\begin{proof}
To see the inclusion $Ba(X)
  \subset \mathcal{B}(X)$, note that by continuity of $f \in
  C_b(X;\reals)$, every set $f^{-1}(U)$ is open.

Now suppose $(S,d)$ is a metric space.  To show $\mathcal{B}(S)
\subset Ba(S)$, it suffices if we show every closed set $F \subset S$
can be written as $f^{-1}(G)$ where $G \subset \reals$ is closed and
$f \in C_b(S; \reals)$.  By the triangle inequality (see e.g. Lemma
\ref{DistanceToSetLipschitz}) we know
that $g(x) = d(x, F)$ is continuous (in fact Lipschitz) and by Lemma
\ref{MaxMinOfLipschitz} we know that $f(x) = d(x, F) \wedge 1$ is also
Lipschitz and therefore $f(x) \in C_b(S; \reals)$.  Because $F$ is
closed we also know that $F = f^{-1}(\lbrace 0 \rbrace)$ and we are done.
\end{proof}

The theory of probability measures on separable metric spaces is simpler in many ways than its general counterpart 
for non-separable metric spaces. The simplicity derives from the fact that a separable metric space is not too far being compact
so in a sense doesn't have too many unbounded continuous functions with respect to which probability measures can misbehave.

One way in which to understand the way in which a separable metric space is close to being compact is to consider the real line.  Through any number of homeomorphisms, the real line is homeomorhpic to the open unit interval $(0,1)$.  In this way, distances on the real line may be rescaled so as to make the real line bounded.  Then by completing the open interval we see that real line is a couple of points away from being compact.

\begin{defn}Let $(S,d)$ be a metric space then we denote by $U^d(S)$ the set of uniformly continuous functions from $S$ to $\reals$ and by $U_b^d(S)$ the set of bounded uniformly continuous functions from $S$ to $\reals$.
\end{defn}

\begin{lem}\label{SeparabilityOfBoundedUniformlyContinuous}Let $(S,d)$ be a separable metric space, then $X$ is
  homeomorphic to a subset of $[0,1]^{\integers_+}$ and furthermore
\begin{itemize}
\item[(i)]$S$ has a metric making it totally bounded
\item[(ii)]If $S$ is compact then $C(S; \reals)$ with the uniform
  topology is separable.
\item[(iii)]If $\hat{d}$ is a totally bounded metric on $S$ then $U^{\hat{d}}(S) = U^{\hat{d}}_b(S)$ and 
  $U^{\hat{d}}_b(S)$ is separable
\end{itemize}
\end{lem}
\begin{proof}
Let $\rho$ be the product metric $\rho(x,y) = \sum_{n=1}^\infty
\frac{\abs{x_n-y_n}}{2^n}$ on the space $[0,1]^{\integers_+}$. 
 Pick a countable dense subset $x_1, x_2, \dotsc$ of $S$ and define 
$f : S \in [0,1]^{\integers_+}$ by 
\begin{align*}
f(x) &= \left ( \frac{d(x_1, x)}{1 + d(x_1, x)}, \frac{d(x_2, x)}{1 +
    d(x_2, x)}, \dotsc \right )
\end{align*}
 
\begin{clm}$f(x)$ is continuous.
\end{clm}

By definition of the product topology $f(x)$ is continuous if and only
if each coordinate is.  For any given fixed $x_j$, we know that
$d(x_j, x)$ is continuous (in fact Lipschitz by Lemma
\ref{DistanceToSetLipschitz}) and thus the result follows from the
continuity of $x/(1+x)$ on $\reals_+$.

\begin{clm}$f(x)$ is injective.
\end{clm}

For any $z \neq y$ we find $\epsilon > 0$ such that $B(z ; \epsilon)
\cap B( y ; \epsilon) = \emptyset$ and then using density of $x_1,
x_2, \dotsc$ to pick an $x_n$ such that $d(z,x_n) < \epsilon$ and
$d(y, x_n) \geq \epsilon$ showing $f(z) \neq f(y)$.  

\begin{clm}The inverse of $f(x)$ is continuous.
\end{clm}

Fix an $x \in S$ and let $\epsilon >0$ be given.  Pick $x_n$ such that
$d(x_n, x) < \epsilon/2$.  If we let $g(x) : [0,1) \to \reals_+$ be
defined by $g(x) = x/(1-x)$ then $g(x)$ is the inverse of $x/(1+x)$ 
and by continuity of $g(x)$ at the point $\frac{d(x_n, x)}{1+d(x_n,x)}$ we know that there exists a $\delta > 0$
such that $\abs{\frac{d(x_n, x)}{1+d(x_n,x)} - 
\frac{d(x_n,  y)}{1+d(x_n,y)}}< \delta$ implies $\abs{d(x_n,x) - d(x_n,y)} <
  \epsilon/2$.
Then if $f(y) \in B(f(x), \frac{\delta}{2^n})$ we have
\begin{align*}
\abs{\frac{d(x_n, x)}{1+d(x_n,x)} - 
\frac{d(x_n,  y)}{1+d(x_n,y)}} &\leq 2^n \rho(f(x), f(y)) < \delta
\end{align*}
$d(x,y) \leq d(x_n,x) + \abs{d(x_n,x) - d(x_n,y)} < \epsilon$.

Now to see (i) we simply pull back the metric $\rho$ via the embedding
$f(x)$ and use the facts that $\rho$ generates the product topology,
$[0,1]^{\integers_+}$ is compact in product topology
(by Tychonoff's Theorem \ref{Tychonoff}; alternatively one can avoid
the use of Tychonoff's Theorem for it is easy to
see with a diagonal subsequence argument that a countable product of
sequentially compact metric spaces is sequentially compact) hence totally bounded (Theorem
\ref{CompactnessInMetricSpaces}).

Here is the argument that $\rho$ generates the product topology; TODO:
put this in a separate lemma.  To see that the topology generated by
$\rho$ is finer than the product topology, suppose $U$ is open in the
topology generated by $\rho$.  Pick $x \in U$ and select $N > 0$ such
that $B(x,\epsilon) \subset U$.  Then pick $N > 0$ such that $2^{-N-1} <
\epsilon$ and consider $B=B(x_1, \epsilon/2)
\times \dotsb \times B(x_{2^N}, \epsilon/2) \times S \times \dotsb$
which is open in the product topology.  If $y \in B$ then 
\begin{align*}
\rho(x,y) &=
\sum_{n=1}^\infty \frac{\abs{x_n-y_n}}{2^n} = \sum_{n=1}^{2^N}
\frac{\abs{x_n-y_n}}{2^n} + \sum_{n=2^N+1}^\infty \frac{\abs{x_n-y_n}}{2^n} \leq
\frac{\epsilon}{2} \sum_{n=1}^{2^N} \frac{1}{2^n} +
\sum_{n=2^N+1}^\infty \frac{1}{2^n} < \epsilon
\end{align*}
To see that the product topology is finer than the metric topology,
suppose $n >0$ is an integer, $U\subset[0,1]$ is open and consider $\pi_n^{-1}(U)$.  Let $x
\in \pi_n^{-1}(U)$ and find an $\epsilon > 0$ such that $B(x_n,
\epsilon) \subset U$.  Note that if $y \in B(x, \frac{\epsilon}{2^n})$
then $\abs{x_n - y_n} < 2^n \rho(x,y) \leq \epsilon$ and therefore
$B(x, \frac{\epsilon}{2^n}) \subset \pi_n^{-1}(B(x_n, \epsilon))
\subset U$.

To see (ii), if $S$ is compact then $f(S) \subset [0,1]^{\integers_+}$
is compact (Lemma \ref{ContinuousImageOfCompact}).  Observe that 
\begin{align*}
A &= \lbrace \Pi_{i=1}^np_i(x_i) \mid n \in \naturals \text{ and } p_i
\in \rationals[x] \rbrace
\end{align*}
is a subalgebra of $C([0,1]^{\integers_+} ; \reals)$ and $A$ separates points (
given $x \neq y \in [0,1]$, pick
$n$ such that $x_n \neq y_n$ and pick the function $g(x) = x_n$).  By
the Stone-Weierstrass Theorem \ref{StoneWeierstrassApproximation} we know that $A$ is dense in $C([0,1]^{\integers_+};
\reals)$; now pullback $A$ under $f(x)$ to a countable dense subset of
$C(S;\reals)$. 

To see (iii), suppose $\hat{\rho}$ is a totally bounded metric on
$S$.  Let $\hat{S}$ be the completion of $S$ with respect to
this metric.  

\begin{clm}$\hat{\rho}$ extends to a totally bounded
metric on $\hat{S}$.  
\end{clm}

Let $\epsilon>0$ be given and cover $S$ by ball
$B(x_i, \epsilon/2)$; we show that $B(x_i, \epsilon)$ covers
$\hat{S}$.  Biven $y \in \hat{S}$ we can find $x \in
S$ such that $\hat{\rho}(x,y) < \epsilon/2$.  Since $x \in S$ there
exists an $x_i$ such that $x \in B(x_i, \epsilon/2)$ and therefore
$\hat{\rho}(x_i,y) \leq \hat{\rho}(x,x_i) + \hat{\rho}(x_i,y) <
\epsilon$.

Because $(\hat{S},\hat{\rho})$ is complete and totally bounded we know
it is compact (Theorem \ref{CompactnessInMetricSpaces}) and we have
just shown that $C(\hat{S} ; \reals)$ has a countable dense subset.

\begin{clm}$f\vert_S : C(\hat{S} ; \reals) \to U_b^{\hat{\rho}}(S ;
\reals)$ is a well defined, continuous and surjective.
\end{clm}

Being well defined in this context means that restriction to $S$
results in a bounded uniformly continuous function.  This follows
from the fact that any continuous function of a compact set is bounded
and uniformly continuous (Theorem \ref{ContinuousImageOfCompact} and
Theorem \ref{UniformContinuityOnCompactSets} respectively) and these
properties are preserved upon restriction.  To see surjectivity, let
$g : S \to \reals$ be uniformly continuous.  We may apply Proposition
\ref{ExtensionOfUniformlyContinuousMapCompleteRange} to see that $g$ has a unique extension
to a continuous function from the closure of $S$ to $\reals$.  Since the closure of $S$ in $\hat{S}$ is
$\hat{S}$ be are done with the claim.  Note that we did not need boundedness of $g$ in order to prove the existence of the extension;
therefore we have shown $U^{\hat{d}}(S) = U^{\hat{d}}_b(S)$.

Now the continuous image of a dense set under a surjective map is also
dense.  This is easily seen by picking a point $f(x)$ in the image;
picking a sequence $x_n$ such that $x_n \to x$ and then considering
the image $f(x_n) \to f(x)$.  Thus the result is proven.
\end{proof}

\begin{lem}[Dini's Theorem]\label{DinisTheorem}Let $K$ be a compact
  topological space and let $f_n: K \to \reals$ be a sequence  of continuous
  functions such that $f_n \downarrow 0$ pointwise on $K$, then $f_n
  \to 0$ uniformly.
\end{lem}
\begin{proof}
Given $\epsilon > 0$ define $U_n = f_n^{-1}((-\infty,\epsilon))$.
Then each
$U_n$ is open, $U_1 \subset U_2 \subset \dotsb$ (since the $f_n$ are
decreasing) and the $U_n$ form an open cover of $K$.  We can extract a
finite subcover which since the $U_n$ are nested implies that $K =
U_N$ for some $N > 0$.  This is exactly the statement that $\sup_{x
  \in K} \abs{f_n(x)} < \epsilon$ for all $n \geq N$ hence the result proven.
\end{proof}

\begin{lem}\label{StoneDaniellProbability}Let $(S,d)$ be a separable metric space and let $\Lambda :
  U_b^d(S; \reals) \to \reals$ be a linear map such that 
\begin{itemize}
\item[(i)] $\Lambda$ is non-negative (i.e. if
  $f \geq 0$ then $\Lambda(f) \geq 0$) 
\item[(ii)] $\Lambda(1) = 1$
\item[(iii)] for all $\epsilon > 0$ there exists a compact set $K
  \subset S$ such that for all $f \in U_b^d(S; \reals)$,
\begin{align*}
\abs{\Lambda(f)} &\leq \sup_{x \in K} \abs{f(x)} + \epsilon \norm{f}_u
\end{align*}
\end{itemize}
then there exists a Borel probability measure $\mu$ on $S$ such that
$\Lambda(f) = \int f \, d\mu$.  Whenever such a probability measure
exists it is unique.
\end{lem}
\begin{proof}
We construct $\mu$ by use of the Daniell-Stone Theorem
\ref{DaniellStoneTheorem}.  It is clear that $U_b^d(S; \reals)$ is
closed under max and min and contains the constant functions so
$U_b^d(S; \reals)$ is a Stone Lattice.  It remains to show that
$\Lambda$ obeys the ``montone convergence'' property: if $f_n
\downarrow 0$ pointwise then $\Lambda(f_n) \downarrow 0$.  This
property is a corollary of Dini's Theorem \ref{DinisTheorem} since by that result,
if $f_n$ are continuous and $f_n \downarrow 0$ pointwise on a compact
set then the converge uniformly to $0$ on the compact set.  In
particular, pick an $\epsilon > 0$ and let $K \subset S$ be compact
as in the hypothesis.  By Dini's Theorem there exists $N > 0$ such
that $\sup_{x \in K} f_n(x) < \epsilon$ for all $n \geq N$.  Therefore
for all $N > 0$,
\begin{align*}
\abs{\Lambda(f_n)} &\leq \sup_{x \in K} \abs{f_n(x)} + \epsilon
\norm{f_n}_\infty \\
&\leq \epsilon(1 + \norm{f_1}_\infty)
\end{align*}
thus $\lim_{n \to \infty} \Lambda(f_n) = 0$ and we can apply Theorem \ref{DaniellStoneTheorem}. 

Uniqueness follows because a probability measure is determined by its
integrals over $U_b^d(S;\reals)$ (in fact over the subset of bounded
Lipschitz functions).  This follows because for any closed $F \subset
S$ we can define $f_n(x) = n d(x, F) \wedge 1$ so that $f_n
\downarrow \characteristic{F}$ and apply Montone Convergence (see the
proof of the Portmanteau Theorem \ref{PortmanteauTheorem} for complete
details on this argument).
\end{proof}

\begin{thm}[Prohorov's Theorem]\label{Prohorov}Let $(S,d)$ be a
  separable metric space, then a tight set of probability measures on
  $S$ is weakly relatively compact.  If $S$ is also complete then a
  weakly relatively compact set is tight.
\end{thm}
\begin{proof}
By the Portmanteau Theorem \ref{PortmanteauTheorem} we know that a
set of measures is tight if and only if its weak closure is tight
(compact sets are closed hence can only gain mass in a weak limit).  Thus it suffices to assume
that we have a closed tight set $M$ of measures.  Put a totally
bounded metric $\hat{d}$ on $S$ so that $U_b^{\hat{d}}(S;\reals)$ is separable
(Lemma \ref{SeparabilityOfBoundedUniformlyContinuous}); let $f_1, f_2,
\dotsc$ be a countable uniformly dense subset.  

Pick a sequence $\mu_n$ from $M$; we must show that it has a weakly
convergent subsequence.  For every fixed $f_m$
we know that $\abs{\int f_m \, d\mu_n} \leq \norm{f_m}_u < \infty$ so
there is a subsequence $N \subset \naturals$ such that $\int f_m \,
d\mu_n$ converges along $N$.  Since is true for every $m>0$ by a
diagonalization argument we know there is a subsequence $\hat{\mu}_{k}$ such that
$\lim_{k \to \infty} \int f_m \, d\hat{\mu}_{k}$ exists for every
$m>0$.  Define $\Lambda(f_m) =   \lim_{k \to \infty} \int f_m \,
d\hat{\mu}_{k}$ for every such $f_m$.  Our next goal is to extend
$\Lambda$ to all of $U_b^\rho(S;\reals)$.  Since $\Lambda$ is
uniformly continuous on a dense subset we know that a continuous
extension is defined; however we need a little bit more information.

\begin{clm}$\lim_{k \to \infty} \int f \, d\hat{\mu}_{k}$ exists for every
$f \in U_b^\rho(S;\reals)$; moreover $\lim_{k \to \infty} \int f_m \,
d\hat{\mu}_{k} = lim_{m \to \infty} \Lambda(\hat{f}_m)$ where
$\hat{f}_m$ is any subsequence of $f_m$ that converges uniformly to $f$.
\end{clm}

Pick a subsequence of the $f_m$ that converges to $f$.  Let that
subsequence be donoted $\hat{f}_m$ so that $\lim_{m \to \infty} \norm{\hat{f}_m -
  f}_\infty = 0$.  For every $m > 0$ we have
\begin{align*}
\int \hat{f}_m \, d \hat{\mu}_k -\norm{\hat{f}_m - f}_\infty &\leq 
\int f \, d\hat{\mu}_k \leq \int \hat{f}_m \, d \hat{\mu}_k + \norm{\hat{f}_m - f}_\infty
\end{align*}
and therefore taking limits in $k$ and using the definition of
$\Lambda$ at the points $f_m$,
\begin{align*}
\Lambda(\hat{f}_m) -\norm{\hat{f}_m - f}_\infty 
&\leq \liminf_{k \to \infty} \int f \, d\hat{\mu}_k 
\leq \limsup_{k \to \infty} \int f \, d\hat{\mu}_k 
\leq \Lambda(\hat{f}_m) + \norm{\hat{f}_m - f}_\infty
\end{align*}
Now letting $m$ go to infinity we get $\lim_{m \to \infty}
\Lambda(\hat{f}_m) = \lim_{k \to \infty} \int f \, d\hat{\mu}_k$.

As a result of the claim, we now define $\Lambda(f) = \lim_{k \to
  \infty} \int f \, d\hat{\mu}_{k}$ for every $f$  and it is clearly
linear (by linearity of integral and limits), nonnegative (by
monotonicity of integral) and satisfies $\Lambda(1) = 1$ (by direct
computation).  

To show that $\Lambda$ defines a probability measure, we bring the
tightness hypothesis to the table.  Pick $\epsilon > 0$ and by
tightness take a compact set $K \subset S$ such that $\sup_{\mu \in M}
\mu(K) > 1 - \epsilon$.  For any $f \in U_b^{\hat{d}}(S ; \reals)$ we
have
\begin{align*}
\abs{\Lambda(f)} 
&= \lim_{k \to \infty} \abs{\int f \, d\hat{\mu}_k}
= \lim_{k \to \infty} \abs{\int f \characteristic{K} \, d\hat{\mu}_k +
\int f \characteristic{S \setminus K} \, d\hat{\mu}_k} \leq \sup_{x
\in K} \abs{f(x)} + \epsilon \norm{f}_\infty
\end{align*}
so we may apply Lemma \ref{StoneDaniellProbability} to conclude there
exists a probability measure $\mu$ such that for all $f \in U_b^{\hat{d}}(S ; \reals)$ we
have $\Lambda(f) =  \lim_{k \to \infty} \abs{\int f \, d\hat{\mu}_k} =
\int f \, d\mu$.  Since $U_b^{\hat{d}}(S ; \reals)$ contains all
bounded Lipschitz functions by the Portmanteau Theorem
\ref{PortmanteauTheorem} we conclude $\mu_n$ converges weakly to
$\mu$.

Now assume that $S$ is complete and separable and let $M$ be a weakly
relatively compact set of measures.  Let $x_1, x_2, \dotsc$ be a countable dense
subset of $S$.  For every integer $n > 0$ we have $S =
\cup_{k=1}^\infty B(x_k, 1/n)$.  Thus $\cap_{N=1}^\infty \cap_{k=1}^N
B(x_k, 1/n)^c = \emptyset$ so by continuity of measure (Lemma
\ref{ContinuityOfMeasure}) for any fixed probability measure $\mu$ we
can find an $N_{n, \mu} > 0$ such that $\mu(\cap_{k=1}^{N_{n, \mu}}
B(x_k, 1/n)^c ) < \epsilon/2^n$.  We claim that, because $M$ is
compact, we can find an $N_n$ for
which this is true uniformly over the measures in $M$.

\begin{clm}\label{ProhorovClaim2}For every $n > 0$ there exists $N_n > 0$ such that $\mu(\cap_{k=1}^{N_{n}}
B(x_k, 1/n)^c ) < \epsilon/2^n$ for all $\mu \in M$.
 \end{clm}

We argue by contraction by reducing the case where $M$ is a singleton
set (where we have already shown the claim holds).  If Claim \ref{ProhorovClaim2} is not
true then there exists $n$ such that for every integer $N>0$ we have
some $\mu_N \in M$ such that $\mu_N(\cap_{k=1}^{N} B(x_k, 1/n)^c )
\geq \epsilon/2^n$.  By sequential compactness of $M$ we know that
there is a weakly convergent subsequence $\mu_{N_j}$ such that
$\mu_{N_j} \toweak \mu$ for some probability measure $\mu$.  For every
$N > 0$ we have $\cap_{k=1}^{N} B(x_k, 1/n)^c$ is closed and therefore
by the Portmanteau Theorem \ref {PortmanteauTheorem}
\begin{align*}
\epsilon/2^n &\leq \limsup_{j \to \infty} \mu_{N_j}(\cap_{k=1}^{N_j}
B(x_k, 1/n)^c) \\
&\leq \limsup_{j \to \infty} \mu_{N_j}(\cap_{k=1}^{N} B(x_k, 1/n)^c)
\\
&\leq \mu(\cap_{k=1}^{N} B(x_k, 1/n)^c)
\end{align*}
where in the second inequality we have used the fact that the limit
only depends on the tail of the sequence of sets $\cap_{k=1}^{N_j}
B(x_k, 1/n)^c$ and by a union bound for sufficiently large $N_j$ we
have $\mu_{N_j}(\cap_{k=1}^{N_j} B(x_k, 1/n)^c)  \leq
\mu_{N_j}(\cap_{k=1}^{N} B(x_k, 1/n)^c)$.  To finish we get a
contradiction by taking the 
limit and using continuity of measure
\begin{align*}
0 < \epsilon/2^n \leq \lim_{N \to \infty} \mu(\cap_{k=1}^{N} B(x_k,
1/n)^c) = 0
\end{align*}

With Claim \ref{ProhorovClaim2} proven we mimic the proof of Ulam's Theorem.  Let 
\begin{align*}
K &=
\cap_{m=1}^\infty \cup_{j=1}^{N_m} \overline{B}(x_j,\frac{1}{m})
\end{align*}
which is easily seen to be closed (hence complete) and by construction
is totally bounded thus is compact (Theorem
\ref{CompactnessInMetricSpaces})
and furthermore for all $\mu \in M$,
\begin{align*}
\mu(K^c) &\leq \mu((\cap_{m=1}^\infty
\cup_{j=1}^{N_m} B(x_j,\frac{1}{m}))^c) \\
&=\mu(\cup_{m=1}^\infty 
\cap_{j=1}^{N_m} B(x_j,\frac{1}{m})^c) \\
&=\sum_{m=1}^\infty \mu(
\cap_{j=1}^{N_m} B(x_j,\frac{1}{m})^c) \\
&\leq \sum_{m=1}^\infty \frac{\epsilon}{2^m} = \epsilon
\end{align*}

\end{proof}

\begin{lem}For $f,g \in C([0,\infty) ; \reals)$ define
\begin{align*}
\rho(f,g) &= \sum_{n=1}^\infty \frac{1}{2^n} \sup_{0 \leq t \leq n}
(\abs{f(t) - g(t)} \wedge 1)
\end{align*}
then $\rho$ is a metric on $C([0,\infty) ; \reals)$ and $C([0,\infty);
\reals)$ is complete and separable with respect to this metric.
\end{lem}
\begin{proof}
It is clear that $\rho(f,f) = 0$ and furthermore if $\rho(f,g) = 0$
then $f = g$ on every interval $[0,n]$ and therefore $f = g$.
Symmetry and the triangle inequality of $\rho$ is immediate from the
corresponding properties of the absolute value (TODO: OK the triangle
inequality may need a bit more of an argument).

We claim that the set of polynomials with rational coefficients is
dense in $C([0,\infty); \reals)$.  Pick $f \in C([0,\infty); \reals)$
and let $\epsilon > 0$ be given.  Now take $m > 0$ sufficiently large
so that $1/2^m < \epsilon / 2$ and by the Stone Weierstrass Theorem \ref{StoneWeierstrassApproximation} we
pick a polynomial with rational coefficients $p$ such that $\sup_{0
  \leq t \leq m} \abs{f(t) - p(t)} < \epsilon/2$ then we have
\begin{align*}
\rho(f,p) &\leq \sum_{n=1}^m\frac{1}{2^n} \sup_{0 \leq t \leq n}
\abs{f(t) - p(t)} + \sum_{n=m+1}^\infty \frac{1}{2^n} \\
&\leq \sup_{0 \leq t \leq m}
\abs{f(t) - p(t)} \sum_{n=1}^m\frac{1}{2^n} + \epsilon/2 <\epsilon
\end{align*}

Completeness follows from arguing over intervals $[0,n]$.  Suppose
$f_n$ is a Cauchy sequence in $C([0,\infty); \reals)$.  Given
$\epsilon > 0$ and $n > 0$ we can find $N > 0$ such that $\rho(f_m,
f_N) < \epsilon/2^n$ for all $m \geq N$.  Thus $\sup_{0 \leq t \leq n}
\abs{f_m(t) - f_N(t)} < \epsilon$ for all $m \geq N$ so we see that
$f_n$ is uniformly  Cauchy on every interval $[0,n]$.  By completeness
of $C([0,n];\reals)$ we know that the pointwise limit of $f_n$ exists
on every $[0,n]$ and is a continuous function.  Therefore we have a
limit $f$ defined on $[0,\infty)$ and since continuity is a local
property $f \in C([0,\infty); \reals)$.  It remains to show that $f_n$
converges to $f$ in the metric $\rho$.  This follows arguing as we
have above.  Let $\epsilon > 0$ be given and choose $n > 0$ such that
$\frac{1}{2^n} < \epsilon/2$ and choose $N > 0$ such that $\sup_{0 \leq t \leq n}
\abs{f_m(t) - f_N(t)} < \epsilon/2$ and then observe
\begin{align*}
\rho(f_m, f_N) &\leq \sum_{k=1}^n \frac{1}{2^k} \sup_{0 \leq t \leq k}
\abs{f_m(t) - f_N(t)} + \sum_{k=n+1}^\infty \frac{1}{2^k}  < \epsilon
\end{align*}
\end{proof}

The topology defined by $\rho$ is often refered to as the topology of
uniform convergence on compact sets by virtue of the following lemma.
\begin{lem}\label{UniformConvergenceOnCompacts}A sequence $f_n$
  converges to $f$ in $C^\infty([0,\infty), \reals)$ if and only if
  $f_n$ converges to $f$ uniformly on every interval $[0,T]$ for $T > 0$.
\end{lem}
\begin{proof}
TODO:  This is elementary.
\end{proof}

\begin{defn}Given a function $f : [0,T] \to \reals$ the \emph{modulus
    of continuity} is the function
\begin{align*}
m(T, f, \delta) &= \sup_{\substack{\abs{s - t} < \delta \\
0 \leq s,t \leq T}} \abs{f(s) - f(t)}
\end{align*}
\end{defn}

\begin{lem}For fixed $T > 0$ and $\delta > 0$, $m(T, f, \delta)$ is a
  continuous function on $C([0,\infty); \reals)$.  For fixed $T > 0$
  and function $f : \reals \to \reals$, $m(T,f,\delta)$ is
  nonincreasing in $\delta$ and 
\begin{align*}
\lim_{\delta \to 0} m(T, f, \delta) = 0
\end{align*}
provided $f \in C([0,\infty); \reals)$.
\end{lem}
\begin{proof}
To see continuity on $C([0,\infty); \reals)$ let $f \in C([0,\infty);
\reals)$, $T > 0$, $\delta > 0$ and $\epsilon > 0$ be given and pick $g$ that $\rho(f,g) <
\epsilon/2^{\ceil{T}+1}$.
From the definition of the metric $\rho$ for any $n > 0$, $\sup_{0 \leq t \leq n} \abs{f(t) - g(t)} \wedge
1 \leq 2^n \epsilon$, so for any $T > 0$, 
\begin{align*}
\sup_{0 \leq t \leq T} \abs{f(t) - g(t)} \wedge
1 &\leq \sup_{0 \leq t \leq \ceil{T}} \abs{f(t) - g(t)} \wedge
1 \leq \epsilon/2
\end{align*}  
Therefore by the triangle inequality,
\begin{align*}
\sup_{\substack{
\abs{s -t} < \delta \\
0 \leq s,t \leq T}} \abs{g(s) - g(t)} \wedge 1 
&\leq 
\sup_{\substack{
\abs{s -t} < \delta \\
0 \leq s,t \leq T}} \left ( \abs{g(s) - f(s)} + \abs{f(s) - f(t)} + \abs{f(t)
- g(t)} \right ) \wedge 1 \\
&\leq \epsilon/2 + 
\sup_{\substack{
\abs{s -t} < \delta \\
0 \leq s,t \leq T}} \abs{f(s) - f(t)} \wedge 1 + \epsilon/2
\end{align*}
and therefore arguing with the roles of $f$ and $g$ reversed shows 
$\abs{m(T, f, \delta) - m(T, g, \delta) } \leq \epsilon$.

The fact that $m(T, f, \delta)$ is decreasing in $\delta$ is clear
because the definition shows that for $\delta_1 \leq \delta_2$ we
have 
\begin{align*}
\lbrace \abs{f(t) - f(s) } \mid 0 \leq s,t \leq T \text{ and }
  \abs{s-t} < \delta_1 \rbrace 
&\subset 
\lbrace \abs{f(t) - f(s) } \mid 0 \leq s,t \leq T \text{ and }
  \abs{s-t} < \delta_2 \rbrace
\end{align*} and therefore $m(T, f, \delta_2) \leq
  m(T, f, \delta_1)$.

Lastly if we suppose $f \in C([0,\infty); \reals)$ then $f$ is
uniformly continuous on $[0,T]$ for every $T > 0$ (Theorem
\ref{UniformContinuityOnCompactSets}).  Thus given an $\epsilon > 0$
there exists $\delta>0$ such that 
\begin{align*}
\sup_{\substack{
\abs{s -t} < \delta \\
0 \leq s,t \leq T}} \abs{f(s) - f(t)} < \epsilon
\end{align*}
which shows $\lim_{\delta \to 0} m(T, f, \delta) = 0$.
\end{proof}
The following Theorem is a version of the Arzela-Ascoli Theorem of
real analysis.
\begin{thm}[Arzela-Ascoli Theorem]\label{ArzelaAscoliTheorem}A set $A
  \subset C([0,\infty); \reals)$ is relatively compact if and only if 
\begin{itemize}
\item[(i)]$\sup_{f \in A} \abs{f(0)} < \infty$
\item[(ii)]$\lim_{\delta \to 0} \sup_{f \in A} m(T, f, \delta) = 0$
  for all $T > 0$.
\end{itemize}
\end{thm}
\begin{proof}
To see the necessity of condition (i), observe that $\overline{A}$ is
compact and by completeness of $C([0,\infty); \reals)$ we know that
$\overline{A}$ comprises continuous functions.  Therefore we know that
$A \subset \overline{A} \subset \cup_{n=1}^\infty \lbrace f \in
C([0,\infty) \mid
\abs{f(0)} < n\rbrace$.  Since each $\lbrace f \in
C([0,\infty) \mid
\abs{f(0)} < n\rbrace$ is easily seen to be an open set, by
compactness of $\overline{A}$ we have a finite subcover which implies
there exists an $N$ such that $A \subset \overline{A} \subset \lbrace f \in
C([0,\infty) \mid
\abs{f(0)} < N\rbrace$.

To see the necessity of condition (ii), fix $\epsilon > 0$, $T > 0$
and define for each $\delta > 0$ the set 
\begin{align*}
F_\delta &= \lbrace f \in \overline{A} \mid m(T, f, \delta) \geq
\epsilon \rbrace
\end{align*}
By continuity of $m(T, f, \delta)$ we know that $F_\delta$ is closed.
Since $F_\delta \subset \overline{A}$ with $\overline{A}$ compact we
conclude that $F_\delta$ is compact.  Furthermore since for fixed $f
\in \overline{A}$ continuity (more specifically uniform continuity on compact
sets) implies $\lim_{\delta \to 0} m(T,f,\delta) = 0$, we know that
$\cap_{\delta > 0} F_\delta = \emptyset$.  By nestedness and
compactness of the
$F_\delta$ we know that there is some specific $\delta>0$ for which $F_\delta =
\emptyset$ (Lemma \ref{IntersectionOfNestedCompactSets}) and (ii) is established.

To see the sufficiency of conditions (i) and (ii), we first construct
the limiting subsequence on a the set of rationals $\rationals_+
\subset [0,\infty)$.  To do this, we first claim that for any $T \in
\rationals_+$, (in fact any $T \in [0,\infty)$, the set $\lbrace
\abs{f(x)} \mid f \in A \rbrace$ is bounded.  The claim follows for
$T>0$ by using (ii) to select a $\delta > 0$ such that $\sup_{f \in A} m(T, f,
\delta) < 1$.  Picking the integer $m \geq 0$ such that $m \delta < T \leq
(m+1)\delta$ and considering  the grid $0, \delta, 2\delta, \dotsc,
m\delta, T$ we can write the telescoping sum
\begin{align*}
f(T) - f(0) = f(T) - f(m\delta) + \sum_{k=1}^m f(k \delta) - f((k-1)\delta)
\end{align*}
and use the triangle inequality to conclude that $\abs{f(T)}
\leq \abs{f(0)} + m+1$ for every $f \in A$.  Coupled with (i) this shows that
$\sup_{f \in A} \abs{f(T)} < \infty$.

We now enumerate the rationals $\rationals_+$ and use
compactness in $\reals$ and a diagonal
subsequence argument to pick a sequence $f_n$ with $f \in A$ such that
$f_n(T)$ converges for every $T \in \rationals_+$.  Define $f :
\rationals_+ \to \reals$ by $f(T) = \lim_{n \to \infty} f_n(T)$.

Having selected a convergent subsequence $f_n$ and defined $f$ on
$\rationals_+$ we proceed to see that $f$ is uniformly continuous.
This follows by using (ii) to see that for every $f_n$, $T > 0$ and 
$\epsilon > 0$ there is $\delta > 0$ such that $\abs{f_n(s) - f_n(t)} <
\epsilon$ when $0 \leq s,t \leq T$ and $\abs{s - t} <\delta$.  From
this we have for every $n>0$, and $s,t \in \rationals$, $0 \leq s,t
\leq T$ and $\abs{s-t} < \delta$
\begin{align*}
\abs{f(s) -f(t)} &\leq \abs{f(s) -f_n(s)} + \abs{f_n(s) -f_n(t)} +
\abs{f_n(t) - f(t)} \\
&\leq \abs{f(s) -f_n(s)} + \epsilon +
\abs{f_n(t) - f(t)}
\end{align*}
Taking the limit as $n \to \infty$ using pointwise convergence of
$f_n$ to $f$ shows uniform continuity on every
$[0,T] \cap \rationals$ hence on $\rationals_+$.  Since $f$ is uniformly continuous on
$\rationals_+$ it follows that $f$ has a continuous extension to $f :
[0,\infty) \to \reals$.  Moreover we have shown that $\abs{f(s) -f(t)}
< \epsilon$ when $\abs{s -t} < \delta$.

It remains to prove that $f_n \to f$ in $C([0,\infty); \reals)$.  It
suffices (Lemma \ref{UniformConvergenceOnCompacts}) to show that $f_n
\to f$ uniformly on every interval $[0,T]$.  Let $T > 0$ be given.
Pick $\epsilon > 0$ and let
$\delta > 0$ be such that $m(T,f_n,\delta) < \epsilon$ (hence $m(T,f,\delta)
< \epsilon$ by the above comment).   Pick $N > 0$ such that
$\abs{f_n(k\delta) - f(k\delta)} < \epsilon/3$ for all $k=0,1, \dotsc,
\ceil{T/\delta}$ and $n \geq N$.  Then for every $0 \leq t \leq T$ and
$n \geq N$ let $k\geq 0$ be such that $k\delta \leq t < (k+1)\delta$
\begin{align*}
\abs{f_n(t) - f(t)} &\leq \abs{f_n(t) - f_n(k\delta)}
+\abs{f_n(k\delta) - f(k\delta)} +\abs{f(k\delta) - f(t)} < \epsilon
\end{align*}
and we are done.
\end{proof}

Provided with a characterization of compact sets in
$C^\infty([0,\infty); \reals)$ we can now state the probabilistic
analogue.
\begin{lem}\label{TightnessOfContinuousFunctions}A sequence of Borel probability measures $\mu_n$ on $C^\infty([0,\infty);
  \reals)$ is tight if and only if 
\begin{itemize}
\item[(i)]$\lim_{\lambda \to \infty} \sup_{n \geq 1} \sprobability{\abs{f(0)}
  \geq \lambda}{\mu_n} = 0$.
\item[(ii)] $\lim_{\delta \to 0} \sup_{n \geq 1} \sprobability{m(T, f,
  \delta) \geq \lambda}{\mu_n} = 0$ for all $\lambda > 0$ and $T > 0$.
\end{itemize}
\end{lem}
\begin{proof}
Let $\mu_n$ be a tight sequence.  Let $\epsilon > 0$ be given and pick
$K \subset C^\infty([0,\infty); \reals)$ compact with $\mu_n(K) >
1-\epsilon$ for all $n$.  Then by Theorem \ref{ArzelaAscoliTheorem} we know that $\sup_{f \in K}
\abs{f(0)} < \infty$ and therefore $\sprobability{\abs{f(0)}\geq
\lambda}{\mu_n} \leq \mu_n(K^c) < \epsilon$ for any $\lambda > \sup_{f
\in K} \abs{f(0)}$.  Thus (i) is shown.  Similarly applying Theorem \ref{ArzelaAscoliTheorem} we know that for
every $T > 0$ and $\lambda>0$
there exists $\delta>0$ such that $\sup_{f \in K} m(T, f, \delta) <
\lambda$.  Therefore $\lbrace f \mid m(T,f,\delta) \geq \lambda \rbrace
\subset K^c$ and by a union bound, for every $n>0$ we have $\sprobability{m(T,f,\delta) \geq
  \lambda}{\mu_n} \leq \sprobability{K^c}{\mu_n} < \epsilon$.
Therefore we have shown (ii).

Now assume that (i) and (ii) hold and suppose that $\epsilon > 0$ is
given.  By (i) there exists $\lambda > 0$ such that $\sup_{n \geq 1}
\sprobability{\abs{f(0)} \geq \lambda}{\mu_n} < \epsilon/2$.  By (ii)
for every integer $T > 0$ and $k > 0$, there exists a $\delta_{T,k}$
such that $\sup_{n \geq 1} \sprobability{m(T, f, \delta_{T,k}) \geq
  1/k}{\mu_n} < \epsilon/2^{T+k+1}$.  If we define 
\begin{align*}
A_T &= \lbrace f \mid 
m(T,f,\delta_{T,k}) < 1/k \text{ for all } k \geq 1\rbrace
\end{align*}
so that $A^c_T \subset \cup_{k=1}^\infty \lbrace f \mid m(T, f, \delta_{T,k}) \geq
  1/k \rbrace$ then by a union bound
\begin{align*}
\sup_{n \geq 1} \mu_n(A_T) &= \sup_{n \geq 1} (1 - \mu_n(A^c_T))\\
&\geq \sup_{n \geq 1} \left(1 - \sum_{k=1}^\infty \sprobability{m(T, f, \delta_{T,k}) \geq
  1/k}{\mu_n}\right ) \\
&\geq 1 - \epsilon/2^{T+1}
\end{align*}
If we define $K = \lbrace f \mid \abs{f(0)} < \lambda \rbrace \cap
\cap_{T=1}^\infty A_T$ then another union bound shows $\sup_{n \geq 1}
\mu_n(K) > 1 - \epsilon$ and by construction the set $K$ satisfies the
conditions of Theorem \ref{ArzelaAscoliTheorem} so is proven compact.  
\end{proof}

To prove that the rescaled and linearly interpolated random walk converges we need
prove tightness.  To prove tightness we need to show equicontinuity.
The following Lemma begins the process by demonstrating equicontinuity
at $0$.  Keep in mind the picture of the scaling of the random walk at
level $n$
which places the value of $S_j$ at the point $j/n$ scaled by the
factor $1/\sigma\sqrt{n}$.  With this geometry in mind note that what
we are proving is a bound for each of the sequence of rescaled random
walks on the interval $[0,\delta]$.

TODO: Replace $\epsilon$ by $\lambda$ in the following Lemma?
\begin{lem}\label{RandomWalkEquicontinuityAt0} Let $\xi_n$ be i.i.d. with mean $0$ and finite variance
  $\sigma^2$ and define $S_n = \sum_{k=1}^n \xi_k$.  Then for all
  $\epsilon > 0$ 
\begin{align*}
\lim_{\delta \to 0} \limsup_{n \to \infty} \frac{1}{\delta}
\probability{\max_{1 \leq j \leq \floor{n\delta}+1} \frac{\abs{S_j}}{\sigma\sqrt{n}} \geq
  \epsilon} = 0
\end{align*}
\end{lem}
\begin{proof}
The idea of the proof is to leverage the Central Limit Theorem and
Gaussian tail bounds to control behavior at the right endpoint of the
interval under consideration.  Then independence of increments and
finite variance can be used to control the behavior over the entire
interval.

The sequence of random variables $\frac{1}{\sigma
  \sqrt{\floor{n\delta}+1}}S_{\floor{n\delta}+1}$ is a subsequence of
$\frac{1}{\sigma\sqrt{n}}S_n$ and therefore converges in distribution
to $N(0,1)$ by the Central Limit Theorem.  Furthermore, $\lim_{n \to
  \infty} \frac{\sqrt{\floor{n\delta}+1}}{\sqrt{n\delta}} = 1$ so by
Slutsky's Lemma we also have $\frac{1}{\sigma\sqrt{n\delta}}
S_{\floor{n\delta}+1} \todist Z$ where $Z$ is an $N(0,1)$ Gaussian
random variable.  By the Portmanteau Theorem (Theorem
\ref{PortmanteauTheorem}) and
a Markov bound (Lemma \ref{MarkovInequality}) we have
\begin{align*}
\limsup_{n \to \infty}
\probability{\abs{\frac{1}{\sigma\sqrt{n\delta}}S_{\floor{n\delta}+1}}
  \geq \lambda} &\leq \probability{\abs{Z}  \geq \lambda} \leq \frac{\expectation{\abs{Z}^3}}{\lambda^3}
\end{align*}

We want to leverage this bound to create a maximal inequality that controls the entire interval of
values of the rescaled random walk the approach being to leverage the
fact that either the final point is in the tail (in which
case the Central Limit Theorem bound just proven applies) or the final
point is outside the tail and some interior point is in the tail
providing us with an amount of variation whose probability can be
controlled by use of a second moment bound.  With $\epsilon > 0$ fixed as in
the hypothesis of the Lemma, define the random variable $\tau
= \min \lbrace j \geq 1 \mid \abs{\frac{S_j}{\sigma\sqrt{n}} } >
\epsilon\rbrace$ (this is a stopping time though we make no use of the
concept here).  Pick $\delta > 0$ satisfying $0 < \delta <
\epsilon^2/2$.

\begin{align*}
&\probability{\max_{1 \leq j \leq \floor{n\delta} + 1} \abs{\frac{S_j}{\sigma \sqrt{n}}} \geq \epsilon
  } \\
&=\probability{\max_{1 \leq j \leq \floor{n\delta} + 1}
  \abs{\frac{S_j}{\sigma \sqrt{n}}} \geq \epsilon; 
\abs{\frac{S_{\floor{n\delta}+1}}{\sigma \sqrt{n}}}  \geq \epsilon -
\sqrt{2\delta}} \\
&+ \probability{\max_{1 \leq j \leq \floor{n\delta} + 1}
  \abs{\frac{S_j}{\sigma \sqrt{n}}} \geq \epsilon; 
\abs{\frac{S_{\floor{n\delta}+1}}{\sigma \sqrt{n}}} < 
  \epsilon - \sqrt{2\delta}} \\
&\leq \probability{\abs{\frac{S_{\floor{n\delta}+1}}{\sigma \sqrt{n}}}
  \geq \epsilon - \sqrt{2\delta}} + 
\sum_{j=1}^{\floor{n\delta}}
\probability{\abs{\frac{S_{\floor{n\delta}+1}}{\sigma \sqrt{n}}} <
  \epsilon - \sqrt{2\delta}; \tau = j} \\
&= \probability{\abs{\frac{S_{\floor{n\delta}+1}}{\sigma \sqrt{n}}}
  \geq \epsilon - \sqrt{2\delta}} + 
\sum_{j=1}^{\floor{n\delta}}
\probability{\abs{\frac{S_{\floor{n\delta}+1}}{\sigma \sqrt{n}} -
    \frac{S_j}{\sigma \sqrt{n}}} > \sqrt{2\delta}; \tau = j} \\
&\leq \probability{\abs{\frac{S_{\floor{n\delta}+1}}{\sigma \sqrt{n}}}
  \geq \epsilon - \sqrt{2\delta}} + 
\frac{1}{2\delta}\sum_{j=1}^{\floor{n\delta}}
\expectation{\left(
    \frac{S_{\floor{n\delta}+1}}{\sigma \sqrt{n}} - \frac{S_j}{\sigma
      \sqrt{n}} \right)^2 \characteristic{\tau = j}} \\
&= \probability{\abs{\frac{S_{\floor{n\delta}+1}}{\sigma \sqrt{n}}}
  \geq \epsilon - \sqrt{2\delta}} + 
\frac{1}{2\delta}\sum_{j=1}^{\floor{n\delta}}
\expectation{\left( \sum_{i=j+1}^{\floor{n\delta}+1}
    \frac{\xi_i}{\sigma\sqrt{n}} \right)^2 } \probability{\tau = j} \\
&= \probability{\abs{\frac{S_{\floor{n\delta}+1}}{\sigma \sqrt{n}}}
  \geq \epsilon - \sqrt{2\delta}} + 
\frac{\floor{n\delta}}{2 n \delta}\sum_{j=1}^{\floor{n\delta}}
\probability{\tau = j} \\
&= \probability{\abs{\frac{S_{\floor{n\delta}+1}}{\sigma \sqrt{n}}}
  \geq \epsilon - \sqrt{2\delta}} + 
\frac{1}{2}\probability{\max_{1 \leq j \leq \floor{n\delta} + 1} \abs{\frac{S_j}{\sigma \sqrt{n}}} \geq \epsilon}
\end{align*}
Therefore we have shown that 
\begin{align*}
\probability{\max_{1 \leq j \leq \floor{n\delta} + 1} \abs{\frac{S_j}{\sigma \sqrt{n}}} \geq \epsilon}
&\leq 2 \probability{\abs{\frac{S_{\floor{n\delta}+1}}{\sigma \sqrt{n}}}
  \geq \epsilon - \sqrt{2\delta}} 
\end{align*}
and we can use our tail bound derived from the Central Limit Theorem
(with $\lambda = \frac{\epsilon - \sqrt{2\delta}}{\sqrt{\delta}}$)
to see that 
\begin{align*}
\lim_{\delta \to 0} \limsup_{n \to \infty} \frac{1}{\delta} \probability{\max_{1 \leq j \leq \floor{n\delta} + 1} \abs{\frac{S_j}{\sigma \sqrt{n}}} \geq \epsilon}
&\leq \lim_{\delta \to 0 } \frac{2}{\delta} \expectation{\abs{Z}^3}
\left(\frac {\sqrt{\delta}}{\epsilon - \sqrt{2\delta}}\right)^3 = 0
\end{align*}
\end{proof}

The next step is to extend the estimate that provides equicontinuity
at $0$ to prove equicontinuity of the random walk on all finite intervals.  
\begin{lem}\label{RandomWalkEquicontinuity}
Let $\xi_n$ be i.i.d. with mean $0$ and finite variance
  $\sigma^2$ and define $S_n = \sum_{k=1}^n \xi_k$.  Then for all
  $\epsilon > 0$ and $T > 0$ 
\begin{align*}
\lim_{\delta \to 0} \limsup_{n \to \infty} 
\probability{\max_{
\substack{1 \leq j \leq \floor{n\delta}+1 \\
0 \leq k \leq \floor{nT}+1}} \frac{\abs{S_{j+k}
    - S_k}}{\sigma\sqrt{n}} \geq
  \epsilon} = 0
\end{align*}
\end{lem}
\begin{proof}
Pick $0 \leq \delta \leq T$ and let $m \geq 2$ be the integer such that
$T/m < \delta \leq T/(m-1)$.  Since
\begin{align*}
\lim_{n \to \infty} \frac{\floor{nT}+1}{\floor{n\delta}+1} &=
\frac{T}{\delta} < m
\end{align*}
we know that for sufficiently large $n$ we have $\floor{nT}+1  <
(\floor{n\delta}+1)m$.  For any such $n$, suppose $\frac{\abs{S_{j+k}
    - S_k}}{\sigma\sqrt{n}} > \epsilon$ for some $k$ with $0 \leq k
\leq \floor{nT}+1$ and some $j$ with $0 \leq j \leq
\floor{n\delta}+1$.  Now let $p$ be the integer such that $0 \leq p
\leq m -1$ and 
\begin{align*}
(\floor{n\delta}+1)p \leq k < (\floor{n\delta}+1)(p+1)
\end{align*}
Since $0 \leq j \leq \floor{n\delta}+1$ either 
\begin{align*}
(\floor{n\delta}+1)p \leq k+j < (\floor{n\delta}+1)(p+1)
\end{align*}
or
\begin{align*}
(\floor{n\delta}+1)(p+1) \leq k+j < (\floor{n\delta}+1)(p+2)
\end{align*}
In the first case by the triangle inequality we have
\begin{align*}
\abs{S_{j+k} - S_k} \leq \abs{S_{k} - S_{(\floor{n\delta}+1)p}}+ \abs{S_{j+k} - S_{(\floor{n\delta}+1)p}}
\end{align*}
and therefore we know that either $\frac{\abs{S_{k}
    - S_{(\floor{n\delta}+1)p}}}{\sigma\sqrt{n}} \geq \epsilon/2 > \epsilon/3$ or $\frac{\abs{S_{k+j}
    - S_{(\floor{n\delta}+1)p}}}{\sigma\sqrt{n}} \geq \epsilon/2 > \epsilon/3$.  In
the second case by the triangle inequality we have
\begin{align*}
\abs{S_{j+k} - S_k} \leq \abs{S_{k} - S_{(\floor{n\delta}+1)p}}+ \abs{S_{(\floor{n\delta}+1)(p+1)} - S_{(\floor{n\delta}+1)p}}+ \abs{S_{j+k} - S_{(\floor{n\delta}+1)(p+1)}}
\end{align*}
and therefore we know that either  
$\frac{\abs{S_{k} - S_{(\floor{n\delta}+1)p}}}{\sigma\sqrt{n}} \geq
\epsilon/3$,  
$\frac{\abs{S_{(\floor{n\delta}+1)(p+1)} -
    S_{(\floor{n\delta}+1)p}}}{\sigma\sqrt{n}} \geq \epsilon/3$ or 
$\frac{\abs{S_{k+j} - S_{(\floor{n\delta}+1)(p+1)}}}{\sigma\sqrt{n}}
\geq \epsilon/3$.  Therefore we have the inclusion of events
\begin{align*}
\left \lbrace\max_{
\substack{1 \leq j \leq \floor{n\delta}+1 \\
0 \leq k \leq \floor{nT}+1}} \frac{\abs{S_{j+k}
    - S_k}}{\sigma\sqrt{n}} \geq
  \epsilon\right \rbrace 
&\subset 
\bigcup_{p=0}^{m} \left \lbrace \max_{1 \leq j \leq \floor{n\delta}+1}
\frac{\abs{S_{j + (\floor{n\delta}+1)p} - S_{(\floor{n\delta}+1)p}}}{\sigma\sqrt{n}} \geq
\epsilon/3 \right \rbrace
\end{align*}
By the i.i.d. nature of $\xi_n$ and the fact that $S_0 = 0$ we know that 
\begin{align*}
\probability{\max_{1 \leq j \leq \floor{n\delta}+1}
\frac{\abs{S_{j + (\floor{n\delta}+1)p} - S_{(\floor{n\delta}+1)p}}}{\sigma\sqrt{n}} \geq
\epsilon/3 } &= \probability{\max_{1 \leq j \leq \floor{n\delta}+1}
\frac{\abs{S_{j}}}{\sigma\sqrt{n}} \geq \epsilon/3 }
\end{align*}
and therefore 
\begin{align*}
\probability{\max_{
\substack{1 \leq j \leq \floor{n\delta}+1 \\
0 \leq k \leq \floor{nT}+1}} \frac{\abs{S_{j+k}
    - S_k}}{\sigma\sqrt{n}}} \leq (m+1) \probability{\max_{1 \leq j \leq \floor{n\delta}+1}
\frac{\abs{S_{j}}}{\sigma\sqrt{n}} \geq \epsilon/3 }
\end{align*}
Since $\lim_{\delta \to 0} (m+1)\delta < \lim_{\delta \to 0} (T/\delta
+ 2) \delta = T < \infty$ we can apply Lemma \ref{RandomWalkEquicontinuityAt0} to get the result.
\end{proof}

By Prohorov's Theorem \ref{Prohorov} we know that a tight sequence of probability
measures on a separable metric space has a convergent subsequence.  What is often required is some
way of proving that a particular measure is indeed the limit of that
subsequence.  Recalling Lemma \ref{ProcessLawsAndFDDs} we know that
finite dimensional distributions characterize the laws of stochastic
processes which leads one to the following general procedure for
proving convergence of a sequence of processes.

TODO: Kallenberg (Chapter 16) has general results here for $C(T ; S)$ with $T$ a
$lcscH$-space and $S$ metric.  Of course there are also results for
spaces of discontinuous functions for use in proving convergence of
empirical distribution functions.  Kallenberg also has results for
point process/spaces of measures.

We are taking the point of view of Brownian motion and the linearly
interpolated random walk as being a random element in $C([0,\infty) ;
\reals)$.  On the other hand we have thus far treated a stochastic
process as a random element in a subset of a path space $(S^T,
\mathcal{S}^{\otimes T})$ \emph{equipped with the product
  $\sigma$-algebra}.  It is tempting to gloss over this
point, however to tie in the general definition of stochastic
processes with the random elements of $C([0,\infty); \reals)$ we are
dealing with it is
important to understand the relationship between the Borel
$\sigma$-algebra on $C([0,\infty); \reals)$ and the product
$\sigma$-algebra $\mathcal{B}(\reals)^{\otimes [0,\infty)}$ used in the definition of
processes.
\begin{lem}\label{BorelGeneratedByProjections}For every $t \in [0,\infty)$ let $\pi_t : C([0,\infty); \reals) \to
  \reals$ be the evaluation map $\pi_t(f) = f(t)$.  The Borel $\sigma$-algebra on $C([0,\infty); \reals)$ is
  equal to $\sigma(\lbrace \pi_t \mid t \in [0,\infty) \rbrace)$ and
  therefore $\mathcal{B}(C([0,\infty); \reals)) = C([0,\infty);
  \reals) \cap \mathcal{B}(\reals)^{\otimes [0,\infty)}$.
\end{lem}
\begin{proof}Since each $\pi_t$ is a continuous function, it is Borel
  measurable and therefore the Borel $\sigma$-algebra contains
  $\sigma(\lbrace \pi_t \mid t \in [0,\infty) \rbrace)$.

On the other hand, we know that $C([0,\infty) ; \reals)$ is separable
so we may pick a countable dense set $f_1, f_2, \dotsc$.
If we let $U \subset C([0,\infty) ; \reals)$ be open then for every
$f_j \in U$ there exists $r_j > 0$ such that $B(f_j, r_j) \subset U$
and $U$ is the union of such $B(f_j, r_j)$ (indeed, any $y \in U$ not in the
union of balls can't be the limit of the $f_j$ that are in $U$; on the
other hand it can't be the limit of the $f_j$ that are in $U^c$ since
the latter set is closed; thus the existence of such a $y$ would
contradict the density of $f_1, f_2, \dotsc$).  To show $U \in
\sigma(\lbrace \pi_t \mid t \in [0,\infty) \rbrace)$
it suffices to show that $B(f, r) \in \sigma(\lbrace \pi_t \mid t \in
[0,\infty) \rbrace)$ for every $f \in C([0,\infty) ; \reals)$ and $r > 0$.

Let $B(f, r)$ be given and note that by continuity of the
elements of $C([0,\infty) ; \reals)$ the closed ball
\begin{align*}
\overline{B(f, r)} &= \lbrace g \mid \sup_{x \in [0,\infty)} \abs{f(x)
  - g(x)} \leq r \rbrace \\
&= \lbrace g \mid \sup_{\substack{x \in
    [0,\infty) \\ x \in \rationals}} \abs{f(x)
  - g(x)} \leq r \rbrace \\
&= \cap_{\substack{x \in
    [0,\infty) \\ x \in \rationals}} \pi_x^{-1} ([f(x)-r,f(x)+r]) 
\end{align*}
which shows that $\overline{B(f, r)} \in \sigma(\lbrace \pi_t \mid t \in
[0,\infty) \rbrace)$ and $B(f, r) = \cap_{n=1}^\infty \overline{B(f, r+1/n)}$
which shows that $B(f, r) \in \sigma(\lbrace \pi_t \mid t \in
[0,\infty) \rbrace)$.
\end{proof}

\begin{thm}\label{ConvergenceInDistributionOfContinuousAsTightnessAndFDDs}Let $X_n$ be a tight sequence of continuous processes such
  that for all $d > 0$ and $0 \leq t_1 < \dotsb < t_d < \infty$ the
  sequence $(X_{n, t_1}, \dotsc , X_{n, t_d})$ converges in
  distribution, then the laws $X_n$ converge to a Borel probability
  distribution $\mu$ on $C([0,\infty); \reals)$ for which the
  canonical process $W_t(\omega) = \omega(t)$ satisfies
\begin{align*}
(X_{n, t_1}, \dotsc , X_{n, t_d}) \todist (W_{t_1}, \dotsc, W_{t_d})
\end{align*}
\end{thm}
\begin{proof}
By tightness and Prohorov's Theorem \ref{Prohorov} we know that $X_n$
has a weakly convergent subsequence.  Our first claim is that any two weakly convergent subsequences of
$X_n$ have the same limiting distribution.  Let $\check{X}_n$ and $\hat{X}_n$ be two
such subsequences and suppose that $\pushforward{\check{X}_n}{P} \to
\check{\mu}$ and $\pushforward{\hat{X}_n}{P} \to\hat{\mu}$
respectively.  Fix $0 \leq t_1 < \dotsb < t_d < \infty$ and note that
by the Continuous Mapping Theorem \ref{ContinuousMappingTheorem} we
know that $\pushforward{(\check{X}_{n,t_1}, \dotsc,
  \check{X}_{n,t_d})}{P} \todist \pushforward{(\pi_{t_1}, \dotsc,
  \pi_{t_d})}{\check{\mu}}$ and $\pushforward{(\hat{X}_{n,t_1}, \dotsc,
  \hat{X}_{n,t_d})}{P} \todist \pushforward{(\pi_{t_1}, \dotsc,
  \pi_{t_d})}{\hat{\mu}}$.  By hypothesis we conclude that $\pushforward{(\pi_{t_1}, \dotsc,
  \pi_{t_d})}{\check{\mu}} = \pushforward{(\pi_{t_1}, \dotsc,
  \pi_{t_d})}{\hat{\mu}}$ and therefore by
Lemma \ref{BorelGeneratedByProjections} we can apply Lemma
\ref{ProcessLawsAndFDDs} to conclude $\check{\mu}=\hat{\mu}$ which we
now refer to as $\mu$.

Now suppose that the distributions of $X_n$ do not converge weakly to
$\mu$.  Then there exists a bounded continuous $f$ such that either
$\lim_{n \to \infty} \expectation{f(X_n)}$ does not exist or exists
and is different from $\int f \, d\mu$.  In either case by the
boundedness of $f$ we know that 
\begin{align*}
-\infty &< -\norm{f}_\infty \leq \liminf_{n \to \infty}
\expectation{f(X_n)} \leq \limsup_{n \to \infty} \expectation{f(X_n)}
\leq \norm{f}_\infty  < \infty
\end{align*}
and we can extract
a subsequence $\check{X}_n$ such that $\lim_{n \to \infty}
\expectation{f(\check{X}_n)}$ exists and $\lim_{n \to \infty}
\expectation{f(\check{X}_n)} \neq \int f \, d\mu$.  This is a
contradiction since by tightness we know that $\check{X}_n$ has a weakly
convergent subsequence and we have already just shown that the limiting
distribution is $\mu$.
\end{proof}

The power of this Theorem is that it is often not too difficult to
prove weak convergence of finite dimensional distributions because we
have the power of a rich theory available (e.g. the Central Limit
Theorem, Slutsky's Theorem, characteristic functions).

\begin{lem}\label{ConvergenceOfRandomWalkFDD}Let $\xi_n$ be i.i.d. with mean $0$ and finite variance
  $\sigma^2$, define $S_n = \sum_{k=1}^n \xi_k$, $S_n^*(t) =
  S_{\floor{t}} + (t - \floor{t})\xi_{\floor{t}+1}$ and
  $X_n(t) = \frac{1}{\sigma \sqrt{n}} S_n^*(nt)$ where the latter are
  interpreted as random elements of the Borel measurable space
  $C([0,\infty);\reals)$.  For every $d > 0$ and real numbers $0 \leq t_1 < \cdots < t_d
  < \infty$ we have 
\begin{align*}
(X_n(t_1), \dotsc, X_n(t_d)) \todist (B_{t_1}, \dotsc, B_{t_d})
\end{align*}
where $B_t$ is a standard Brownian motion.
\end{lem}
\begin{proof}
Let $0 \leq t_1 < \dotsb < t_n < \infty$ be given.  The basic point is
that the result follows by the Central Limit Theorem; however due to
the linear interpolation there is a bit of extra work to do.

First note that by definition 
\begin{align*}
\abs{X_n(t) - \frac{1}{\sigma \sqrt{n}} S_{\floor{nt}} }
&\leq  \frac{1}{\sigma \sqrt{n}}\abs{\xi_{\floor{nt}+1}}
\end{align*}
so by a Chebyshev bound (Lemma \ref{ChebInequality}) we have
\begin{align*}
\lim_{n \to \infty} \probability{\abs{X_n(t) - \frac{1}{\sigma
      \sqrt{n}} S_{\floor{nt}} } > \epsilon}
&\leq  \lim_{n \to \infty} \frac{1}{n\epsilon^2} = 0
\end{align*}
thus $X_n(t)  \toprob \frac{1}{\sigma\sqrt{n}} S_{\floor{nt}}$ and by
Lemma \ref{ConvergenceInProbabilityInProductSpaces}
we have $(X_n(t_1), \dotsc, X_n(t_d)) \toprob (\frac{1}{\sigma\sqrt{n}}
S_{\floor{nt_1}}, \dotsc, \frac{1}{\sigma\sqrt{n}}
S_{\floor{nt_d}})$.  Our result will follow by Slutsky's Theorem
\ref{Slutsky} if we can show that
\begin{align*}
(\frac{1}{\sigma\sqrt{n}}
S_{\floor{nt_1}}, \dotsc, \frac{1}{\sigma\sqrt{n}}S_{\floor{nt_d}}) \todist (B_{t_1}, \dotsc, B_{t_d})
\end{align*}
Application of the Continuous Mapping Theorem
\ref{ContinuousMappingTheorem} lets us reduce further to showing that 
\begin{align*}
(\frac{1}{\sigma\sqrt{n}}(S_{\floor{nt_1}} -
S_{\floor{nt_0}}), \dotsc, \frac{1}{\sigma\sqrt{n}}
(S_{\floor{nt_d}}- S_{\floor{nt_{d-1}}}))\todist (B_{t_1} - B_{t_0}, \dotsc, B_{t_d} - B_{t_{d-1}})
\end{align*}
where for uniformity of notation we have defined $t_0 = 0$.  Since the
$\xi_n$ are independent this implies that the $S_{\floor{nt_j}}-
S_{\floor{nt_{j-1}}}$ are independent for $j=1, \dotsc, d$ and by
definition of independent increments property of Brownian motion we
know that $B_{t_j} - B_{t_{j-1}}$ are independent, thus by Lemma \ref{IndependenceProductMeasures} it suffices to
show that $\frac{1}{\sigma\sqrt{n}} (S_{\floor{nt_j}}-S_{\floor{nt_{j-1}}}) \todist N(0, t_j -
t_{j-1})$.  We shall prove this fact for an arbitrary $0 \leq s < t < \infty$.

By the definition of $S_n$ we write $\frac{1}{\sigma\sqrt{n}}
(S_{\floor{nt}}-S_{\floor{ns}}) =
\frac{1}{\sigma\sqrt{n}}\sum_{i=\floor{ns}+1}^{\floor{nt}}
\xi_i$.  For every $\epsilon > 0$ we have by another Chebyshev bound 
\begin{align*}
&\lim_{n \to \infty} \probability{\abs{\frac{1}{\sigma\sqrt{n}}
\sum_{i=\floor{ns}+1}^{\floor{nt}}\xi_i
- \frac{\sqrt{t-s}}{\sigma\sqrt{\floor{nt}-\floor{ns}}}
\sum_{i=\floor{ns}+1}^{\floor{nt}}\xi_i } > \epsilon} \\
&\leq \lim_{n \to \infty}\frac{1}{\epsilon^2}\variance{\left(\frac{1}{\sigma\sqrt{n}}
-\frac{\sqrt{t-s}}{\sigma\sqrt{\floor{nt}-\floor{ns}}} \right)
\sum_{i=\floor{ns}+1}^{\floor{nt}}\xi_i} \\
&= \lim_{n \to \infty}\frac{1}{\epsilon^2}\left(\frac{1}{\sigma\sqrt{n}}
-\frac{\sqrt{t-s}}{\sigma\sqrt{\floor{nt}-\floor{ns}}} \right)^2
(\floor{nt}-\floor{ns}) \sigma^2\\
&= \lim_{n \to \infty} \frac{1}{\epsilon^2} \left(
  \frac{\sqrt{\floor{nt}-\floor{ns}}}{\sqrt{n}} -
  \sqrt{t-s}\right)^2 = 0
\end{align*}
Therefore we have $\frac{1}{\sigma\sqrt{n}}
\sum_{i=\floor{ns}+1}^{\floor{nt}}\xi_i \toprob \frac{\sqrt{t-s}}{\sigma\sqrt{\floor{nt}-\floor{ns}}}
\sum_{i=\floor{ns}+1}^{\floor{nt}}\xi_i $ and one last appeal to
Slutsky's Theorem \ref{Slutsky} implies that it suffices to show
\begin{align*}
\frac{\sqrt{t-s}}{\sigma\sqrt{\floor{nt}-\floor{ns}}}
\sum_{i=\floor{ns}+1}^{\floor{nt}}\xi_i \todist N(0, t-s)
\end{align*}
which is just the Central Limit Theorem (and to be precise the
Continuous Mapping Theorem \ref{ContinuousMappingTheorem} to account for the multiplication by $\sqrt{t-s}$).
\end{proof}

The last step we make is in extending the equicontinuity of the random
walk to equicontinuity of the linearly interpolated random walk which
are honest elements of $C([0, \infty); \reals)$.  This equicontinuity
will prove tightness and weak convergence of the linearly interpolated
random walk.  One of the elements of proving the equicontinuity of the
linearly interpolated random walk is a general fact about the modulus
of continuity of a class of piecewise linear functions which we prove
as a separate lemma.

\begin{lem}\label{ModulusOfContinuityOfPL}Let $f(t)$ be a continuous function that is linear on every
  interval $[j,j+1]$ for $j=0, 1, \dotsc$.  For every integer $M
  > 0$ and $N > 0$, we have
\begin{align*}
\sup_{\substack{\abs{s -t} \leq M \\ 0 \leq s,t
    \leq N} } \abs{f(s) - f(t)}
&\leq
\sup_{\substack{1 \leq j \leq M \\ 0 \leq k
    \leq N}} \abs{f(k+j) - f(k)}
\end{align*}
\end{lem}
\begin{proof}
Pick $0 \leq s<t \leq M$.  If there exists $j < N$ such that $j \leq s
< t \leq j+1$ then it is clear from linearity  that $\abs{f(s) - f(t)}
\leq \abs{f(j) - f(j+1)}$ so it suffices to consider the case in which 
$j \leq s < j +1 < \dotsb < j+k < t \leq j+k+1$ for some $j \geq 0$ and $k
> 0$.  If we let $f(t)$ has slope $a_j$ on the interval $[j,j+1]$
then we can write $f(t) - f(s) = a_{j}(j+1 -s) + \dotsb + a_{j+k}(t -
j -k)$.  Note that
if $f(t) - f(s)$ has a different sign  than $a_j$ then $\abs{f(t) -
  f(s)} \leq \abs{f(t) -  f(j+1)}$ and similarly with $a_{j+k}$ so if
suffices to assume that $a_j$ and $a_{j+k}$ have the same sign as
$f(t)-f(s)$.  Now if $\abs{a_j} \leq \abs{a_{j+k}}$ then we slide the
pair $(s,t)$ to the right until either $s$ or $t$ hits an integer.
More formally if $j+1 - s \leq j+k+1 -t$ then we get
$\abs{f(t) -
  f(s)} \leq \abs{f(t + j+1-s) - f(j+1)}$ and if $j+k+1 -t \leq j+1 -
  s$
we get the bound $\abs{f(t) -
  f(s)} \leq \abs{f(j+k+1) - f(s + j+k+1 -t)}$.  If we $\abs{a_j} \geq
\abs{a_{j+k}}$ we slide to the left in an analogous way.  The point is
that we are reduced to the case in which either $s=j-1$ or $t=j+k+1$.

Once we know that either $s=j-1$ or $t=j+k+1$ , because $M$ is integer
we know that in fact $k \leq M$ and therefore we get a final bound
$\abs{f(t) - f(s)} \leq \abs{f(j+k+1) - f(j-1)}$ which proves the
result.

TODO: This proof is grotesque.  Try to do better!
\end{proof}

We are finally ready to put all of the pieces together to prove
Donsker's Theorem on the convergence of random walks to Brownian
motion.  Note that we have not used the existence of Brownian motion
anywhere in the proof so this Theorem is among other things an
existence proof for Brownian motion.
\begin{thm}[Donsker's Invariance Principle for Random Walks]\label{Donsker2}Let $\xi_n$ be i.i.d. with mean $0$ and finite variance
  $\sigma^2$, define $S_n = \sum_{k=1}^n \xi_k$, $S_n^*(t) =
  S_{\floor{t}} + (t - \floor{t})\xi_{\floor{t}+1}$ and
  $X_n(t) = \frac{1}{\sigma \sqrt{n}} S_n^*(nt)$ where the latter are
  interpreted as random elements of the Borel measurable space
  $C([0,\infty);\reals)$.  
Then the law of $X_n$  converges weakly to a probability measure under
which the coordinate mapping $(f,t) \to f(t)$ is a standard Brownian motion.
\end{thm}
\begin{proof}
Lemma \ref{ConvergenceOfRandomWalkFDD} shows that finite dimensional
distributions of the linearly interpolated and rescaled random walk
converge to the finite dimensional distributions of Brownian motion.
Therefore by Theorem
\ref{ConvergenceInDistributionOfContinuousAsTightnessAndFDDs} it
remains to show that $X_n$ is a tight sequence of processes.
By Lemma \ref{TightnessOfContinuousFunctions} we must show for all $X_n(t)$,
\begin{itemize}
\item[(i)]$\lim_{\lambda \to \infty} \sup_{n \geq 1} \probability{\abs{X_n(0)}
  \geq \lambda}= 0$.
\item[(ii)] $\lim_{\delta \to 0} \sup_{n \geq 1} \probability{m(T, X_n,
  \delta) \geq \lambda} = 0$ for all $\lambda > 0$ and $T > 0$.
\end{itemize}
Since $X_n(0) = 0$ the condition (i) holds trivially.  As for
condition (ii) we first argue that it suffices to show $\lim_{\delta \to 0} \limsup_{n \geq 1} \probability{m(T, X_n,
  \delta) \geq \lambda}= 0$.  This follows from the fact that for
fixed $n > 0$,
$\lim_{\delta \to 0} \probability{m(T, X_n,  \delta) \geq \lambda} =
0$ (continuity of $X_n$) and $\probability{m(T, X_n,  \delta) \geq \lambda}$ is a decreasing
function of $\delta$.  Indeed, if we let $\epsilon > 0$ be given pick
$\Delta > 0$ such that $\limsup_{n \geq 1} \probability{m(T, X_n,
  \delta) \geq \lambda} < \epsilon$ for all $\delta \leq \Delta$.  Then pick $N > 0$ is such that $\sup_{n \geq N} \probability{m(T, X_n,
  \Delta) \geq \lambda} < \epsilon$ and note that because $\probability{m(T, X_n,
  \delta) \geq \lambda}$ is decreasing in fact we have $\sup_{n \geq N} \probability{m(T, X_n,
  \delta) \geq \lambda} < \epsilon$ for all $\delta \leq \Delta$.
Since $\lim_{\delta \to 0} \probability{m(T, X_n,
  \delta) \geq \lambda} = 0 $ for every $n>0$ we can find
$\hat{\Delta} < \Delta$ such that $\probability{m(T, X_n,
  \delta) \geq \lambda}  < \epsilon$ for all $n=1, \dotsc, N-1$ and
$\delta \leq \hat{\Delta}$ and
thus $\sup_{n \geq 1} \probability{m(T, X_n,  \Delta) \geq \lambda} < \epsilon$ for all $\delta < \hat{\Delta}$.

With this reduction in hand, we can estimate
\begin{align*}
\probability{m(T, X_n,  \delta) \geq \lambda} &=
\probability{\sup_{\substack{\abs{s -t} \leq \delta \\ 0 \leq s,t
    \leq T}} \abs{X_n(s) - X_n(t)} \geq \lambda} \\
&\leq
\probability{\sup_{\substack{\abs{s -t} \leq \floor{n \delta} + 1 \\ 0 \leq s,t
    \leq \floor{T \delta}+1}} \abs{S^*_n(s) - S^*_n(t)} \geq \sigma
\sqrt{n} \lambda} \\
&\leq 
\probability{\sup_{\substack{1 \leq j \leq \floor{n \delta} + 1 \\ 0 \leq k
    \leq \floor{T \delta}+1}} \abs{S_n(k+j) - S_n(k)} \geq \sigma
\sqrt{n} \lambda} 
\end{align*}
where the last inequality follows Lemma \ref
{ModulusOfContinuityOfPL}.  Now we can apply Lemma
\ref{RandomWalkEquicontinuity} to conclude $\lim_{\delta \to
  0}\limsup_{n \to \infty} \probability{m(T, X_n,  \delta) \geq
  \lambda}$ and tightness is shown.
\end{proof}

\section{Skorohod Space}
TODO: Currently going through this.  

TODO:  The development below is using the absolute value notation even though
paths take values in an arbitrary metric space.  Clean this up (which is actually pretty barfy because
there are a bunch of different metrics floating around).

TODO: Show that for $S$ complete, the sup norm makes $D([0,T];S)$ into a complete non-separable metric space.
We actually use the completeness in showing that the metric $d$ is complete.

Question 1:  In the definition of the $J_1$ topology on $D([0,\infty);
S)$ given a time shift $\lambda(t)$ we define $d(f,g,\lambda, u) = \sup_{t \geq 0} q(f(t \wedge u),
g(\lambda(t) \wedge u))$ and take the distance given the time shift as
$\int_0^\infty e^{-u} d(f,g,\lambda,u) \, du$.  Why is $d$ defined
this way and not as  $d(f,g,\lambda, u) = \sup_{0 \leq t \leq u} q(f(t),
g(\lambda(t)))$?  Would the latter fail to define a metric or would it
fail to be complete?

Question 2: Given a cadlag function $f : [0,1] \to S$, we know that
$f$ has only countably many jump discontinuities; is there some notion
of uniform continuity that can be preserved?  E.g. can we say that
given $\epsilon > 0$ for
all points of continuity $x$ of $f$  there exists a uniform $\delta > 0$
such that $\abs{x-y} < \delta$ implies $q(f(x), f(y)) < \epsilon$?

\begin{defn}Let $S$ be a topological space, then for every $0 < T < \infty$ we let $D([0,T]; S)$ denote the set of functions
$f : [0,T] \to S$ such that for every $0 \leq t < T$ we have $f(t) = \lim_{s \to t^+} f(s)$ and for every $0 < t \leq T$ the limit 
$\lim_{s \to t^{-}} f(s)$ exists and is finite.  The space $D([0,\infty); S)$ is the set of functions $f : [0,\infty)$ such that for all 
$t \geq 0$ we have $f(t) = \lim_{s \to t^+} f(s)$ and for all $t >0$ we have $\lim_{s \to t^{-}} f(s)$ exists and is finite.
\end{defn}

\begin{lem}\label{CadlagCountableDiscontinuitySet}If $x \in D([0,T];
  S)$ or $x \in D([0,\infty); S)$ then $x$ is continuous at all but a
  countable number of points.
\end{lem}
\begin{proof}
We begin by considering the case of $x \in D([0,t]; S)$.  Pick an
$\epsilon > 0$ and define
\begin{align*}
A_\epsilon &= \lbrace 0 \leq t \leq T \mid r(x(t-), x(t)) \geq
\epsilon \rbrace
\end{align*}

\begin{clm} $A_\epsilon$ is finite.
\end{clm}

Suppose otherwise, then by compactness of $[0,T]$ there is an
accumulation point $t$ of $A_\epsilon$.  By passing to a further
subsequence we can assume that we have a sequence $t_n$ such that $t_n
\in A_\epsilon$ and
either $t_n \downarrow t$ or $t_n \uparrow t$.  First consider the
case $t_n \downarrow t$.  For every $n$ by the existence of the left
limit $x(t_n-)$ we can find $t_n^\prime$ such that $t_{n+1} >
t_n^\prime > t_n$ and $r(x(t_n), x(t_n^\prime)) > \epsilon/2$.  Now by
construction we
know that $t_n^\prime \downarrow t$ and by right continuity we get
$\lim_{n \to \infty} x(t_n) = \lim_{n \to \infty} x(t_n) = x(t)$.  
However this is a contradiction since we can find $N > 0$ such that
$r(x(t), x(t_N)) < \epsilon/4$ and $r(x(t), x(t_N^\prime) <
\epsilon/4$ which yields $r(x(t_N), x(t_N^\prime)) < \epsilon/2$.  If
$t_n \uparrow t$ we argue similarly construction a sequence
$t_n^\prime$ such that $t_{n-1} < t_n^\prime < t_n$ and $r(x(t_n),
x(t_n^\prime)) > \epsilon/2$.  By existence of left limits, we know that $\lim_{n \to \infty}
x(t_n^\prime) = \lim_{n \to \infty} x(t_n) = x(t-)$ and this gives a
contradiction as before.

Now simply note that the set of
discontinuities of $x$ is $\cup_{n=1}^\infty A_{1/n}$ and is therefore
countable.  In a similar way we see that the set of discontinuities
for $x \in D([0,\infty); S)$ is countable since it is equal to the
union of the discontinuities of $x$ restricted to $[0,n]$ for $n \in \naturals$.
\end{proof}

\begin{defn}Let $(S,r)$ be a metric space, define $\Lambda$ denote the set of all $\lambda : [0,T] \to
  [0,T]$ such that $\lambda$ is continuous, strictly increasing and
  bijective.  Then for each $\lambda \in \Lambda$ we define 
\begin{align*}
\rho(x,y,\lambda) &= \abs{\lambda(t) - t} \vee \sup_{t \in [0,T]} r(x(t), y(\lambda(t))
\end{align*}
and define $\rho : D([0,T]; S) \times D([0,T]; S) \to  \reals$ by 
\begin{align*}
\rho(x,y) &= \inf_{\lambda \in \Lambda} \rho(x,y,\lambda)
= \inf_{\lambda \in \Lambda} \sup_{t \in [0,T]}
\sup_{t \in [0,T]}\abs{\lambda(t) - t} \vee \sup_{t \in [0,T]} r(x(t), y(\lambda(t))
\end{align*}
\end{defn}

\begin{lem}\label{SkorohodJ1RhoMetric}$\rho$ is a metric on $D([0,T];S)$.
\end{lem}
\begin{proof}
It is clear that $\rho(x,y) \geq 0$, now suppose that $\rho(x,y) =
0$.  By definition we can find a sequence $\lambda_n \in \Lambda$ such
that $\sup_{t \in [0,T]} \abs{\lambda_n(t) - t} < 1/n$ and $\sup_{t
  \in [0,T]} r(x(t), y(\lambda_n(t)) < 1/n$.  From the former
inequality we see that $\lim_{n \to \infty} \lambda_n(t) = t$ and the second
inequality we see that $\lim_{n \to \infty} y(\lambda_n(t) = x(t)$.
By the cadlag nature of $y$ shows that either $x(t) = y(t)$ or $x(t) =
y(t-)$; so in particular, $x(t) = y(t)$ at all continuity points of
$y(t)$.  However, since the set of discontinuity points is countable
it follows that the set of continuity points is dense in $[0,T]$ and
therefore for every $t \in T$ we can find a set of continuity points
$t_n$ such that $t_n \downarrow t$ and therefore by right continuity
of $y$ we conclude $x(t) = y(t)$.

To see symmetry of $\rho$ we first note that $\lambda \in \Lambda$
implies $\lambda^{-1} \in \Lambda$.   To see this, it is first off
clear that $\lambda^{-1}$ exists because $\lambda$ is a bijection.
The fact that $\lambda^{-1}$ is strictly increasing follows because
if $0 \leq t < s \leq T$ and $0 \leq \lambda^{-1}(s) \leq lambda^{-1}(t) \leq T$ then strictly
increasing and bijective nature of $\lambda$ tells $s \leq t$ which is
contradiction.  To see that $\lambda^{-1}$ is continuous, pick $0 < t
< T$ and let $\epsilon > 0$ be given such that $0 < \lambda^{-1}(t) -
\epsilon < \lambda^{-1}(t)  < \lambda^{-1}(t) + \epsilon < T$.  By strict increasingness
and bijectivity 
of $\lambda$ we know that $0 < \lambda(\lambda^{-1}(t) -
\epsilon) < t < \lambda(\lambda^{-1}(t) + \epsilon) < T$.  Let 
\begin{align*}
\delta &< ( t -  \lambda(\lambda^{-1}(t) -\epsilon)) \wedge (
\lambda(\lambda^{-1}(t) +\epsilon) -t)
\end{align*}
and note by the strict increasingness of $\lambda^{-1}$ we have
\begin{align*}
0 &< \lambda^{-1}(t) -\epsilon < \lambda^{-1}(t - \delta) <
\lambda^{-1}(t) < \lambda^{-1}(t + \delta) < \lambda^{-1}(t) + \epsilon
< T
\end{align*}
Now by the bijectivity of $\lambda$ we know that by a change of variables
\begin{align*}
\sup_{0 \leq t \leq T} \abs{\lambda(t) - t} &= \sup_{0 \leq s \leq T} \abs{s - \lambda^{-1}(s)} \\
\sup_{0 \leq t \leq T} r(x(t), y(\lambda(t))) &= \sup_{0 \leq s \leq
  T} r(x(\lambda^{-1}(s)), y(s)) = 
\sup_{0 \leq s \leq  T} r(y(s), x(\lambda^{-1}(s)))\\
\end{align*}
and therefore $\rho(x,y,\lambda) = \rho(y,x,\lambda^{-1}$.  Because
inversion is a bijection on $\Lambda$ we then get
\begin{align*}
\rho(x,y) &= \inf_{\lambda \in \Lambda} \rho(x,y,\lambda) =
\inf_{\lambda \in \Lambda} \rho(y,x,\lambda^{-1}) = \inf_{\lambda^{-1}
  \in \Lambda} \rho(y,x,\lambda^{-1}) = \rho(y,x)
\end{align*}
\end{proof}

The metric $\rho$ defines the Skorohod $J_1$ topology on the space
$D([0,T];S)$.  We emphasize here that we are actually interested in
the underlying topology as much as the metric space structure itself
since $\rho$ is not a complete metric.

\begin{examp}\label{NoncompletenessSkorohod}Let $f_n =
  \characteristic{[1/2, 1/2 + 1/(n+2))}$ for $n >0$ be a sequence in
  $D([0,1];\reals)$.  We show that $f_n$ is a Cauchy sequence with
  respect to $\rho$ but $f_n$ does not converge in the $J_1$ topology.
To see that $f_n$ is Cauchy, let $n > 0$ be given and suppose $m \geq
n$.  Define 
\begin{align*}
\lambda_{n,m}(t) &= \begin{cases}
t & \text{if $0 \leq t \leq 1/2$} \\
\frac{n+m+2}{n+2}(t - 1/2) + 1/2 & \text{if $1/2 \leq t < 1/2 +
  1/(n+m+2)$} \\
\frac{\frac{1}{2} - \frac{1}{n}}{\frac{1}{2} - \frac{1}{n+m+2}}(t - \frac{1}{2} - \frac{1}{n+m+2}) + \frac{1}{2} + \frac{1}{n} & \text{if $1/2 +
  1/(n+m+2) \leq t \leq 1$} \\
\end{cases}
\end{align*}
so that $f_{n+m} (t) = f_{n}(\lambda_{m+n}(t))$ for all $t \in [0,1]$
and $\sup_{0 \leq t \leq 1} \abs{\lambda_{m+n}(t) - t} = \frac{1}{n} -
\frac{1}{n+m+2} < \frac{1}{n}$ which shows $\rho(f_n, f_{n+m}) < \frac{1}{n}$.

\begin{clm}If $f_n$ converges in then it must converge to $0$.
\end{clm}

Suppose that $f_n$ converges to some $f \in D([0,1];\reals)$.  Then
there exist $\lambda_n \in \Lambda$ such that $\lim_{n \to \infty} \sup_{0 \leq t \leq 1}
\abs{\lambda_n(t) - t} = 0$ and 
$\lim_{n \to \infty} \sup_{0 \leq t \leq 1}\abs{f_n(t) -
  f(\lambda_n(t))} = 0$.  Therefore for each $0 \leq t \leq 1$ that is
a point of continuity of $f$ we have 
$\lim_{n \to \infty} f_n(t) = \lim_{n \to \infty} f(\lambda_n(t)) =
f(t)$.  By definition of $f_n(t)$ and Lemma
\ref{CadlagCountableDiscontinuitySet} we see that $f(t) = 0$ for all
but a countable number of $0 \leq t \leq 1$.  Therefore by right
continuity and the existence of left limits we conclude $f(t) = 0$ for
all $0 \leq t \leq 1$.  Since $f(\lambda(t))$ is identically zero for
all $\lambda \in \Lambda$ we conclude that $\rho(f_n, 0) = 1$ hence
$f_n$ does not converge.
\end{examp}

\begin{defn}Given $f \in D([0,T];S)$ the function
\begin{align*}
w(f,\delta) &= \inf_{\substack{0=t_0 < t_1 < \dotsb < t_n = T \\
  \min_{1 \leq i \leq n} (t_i - t_{i-1}) > \delta \\ n \in \naturals}}
\max_{1 \leq i \leq n} \sup_{t_{i-1} \leq s < t < t_i} r(f(s), f(t))
\end{align*}
is called the modulus of continuity.
\end{defn}

\begin{lem}\label{SkorohodJ1ModulusOfContinuity}If $f \in D([0,T];S)$ then $\lim_{\delta \to 0} w(f,\delta)
  = 0$.
\end{lem}
\begin{proof}
First note that for fixed $f$ the function $w(f, \delta)$ is a
non-decreasing function of $\delta$.  This is simply because any
candidate partition $0 = t_0 < t_1 < \dotsb < t_n=T$ with $\min_{1
  \leq i \leq n} (t_i - t_{i-1}) > \delta$ is also a candidate for any
smaller value of $\delta$.  Thus the set of candidate partitions gets
larger as $\delta$ shrinks and the infimum over the set of candiates
shrinks.

Let $\epsilon > 0$ be given.  Define $t_0 = 0$ then so long as
$t_{i-1} < T$ we inductively define
$t_i = \inf \lbrace t > t_{i-1} \mid r(f(t), f(t_{i-1})) > \epsilon \rbrace \wedge T$.  We claim that
there exists $n$ such $t_n = T$.  First, note that the sequence $t_i$
is strictly increasing while $t_i < T$ by the right continuity of
$f$.  If there are an infinite number of $t_i < T$ then by compactness
of $[0,T]$ there is a limit point $0 \leq t \leq T$.  However the
existence of the left limit $f(t-)$ says
exists $\delta > 0$ such that for all $0 < t - s < \delta$ we have
$r(f(s), f(t-)) < \epsilon/3$.
This is a contradiction since we can find an $n > 0$ such that for all
$i \geq n$ we have $t - t_i < \delta$.  By definition of the
$t_i$ for any $i \geq n+1$ we can pick
$t_i \leq s < t$ such that $r(f(s), f(t_{i-1})) >
\epsilon$ which provides us with $0 < t -s < \delta$ and 
\begin{align*}
r(f(s),f(t-)) &> r(f(s), f(t_{i-1})) - r(f(t_{i-1}), f(t-)) > \epsilon -
\epsilon/2 = \epsilon/2
\end{align*}

Thus we have constructed a sequence $0 =t_0 < t_1 < \dotsb < t_n = T$
such that $\max_{1 \leq i \leq n} \sup_{t_{i-1} < s < t < t_i}
r(f(s),f(t)) < 2 \epsilon$ so it we define $\delta = \frac{1}{2} \min_{1 \leq i
  \leq n} (t_i - t_{i-1})$ we have shown $w(f, \delta) \leq 2
\epsilon$.  Since $\epsilon$ was arbitrary and $w(f,\delta)$ is a
non-decreasing function of $\delta$ we are done.
\end{proof}

Even though the metric $\rho$ is not complete, the underlying topology
is Polish because we can define an equivalent metric that is
complete.  To repair the incompleteness of $\rho$ we have to be a bit
more strict about the types of time changes that are allowed; more
specifically we have to prevent time changes are asymptotically flat
(or by considering taking the inverse of a time change prevent time
changes that are asymptotically vertical).  The following is a way of
quantifying such a requirement.
\begin{defn}For every $\lambda \in \Lambda$ define 
\begin{align*}
\gamma(\lambda) =
  \sup_{0 \leq s < t \leq T} \abs{\log \frac{\lambda(t) -
      \lambda(s)}{t-s}}
\end{align*}
For every $x,y \in D([0,T]; E)$ define 
\begin{align*}
d(x,y) = \inf_{\substack{\lambda \in \Lambda \\ \gamma(\lambda) <
    \infty}} \gamma(\lambda) \vee \sup_{0 \leq t \leq T} r(x(t) ,y(\lambda(t)))
\end{align*}
\end{defn}

The main goal is to prove that $d$ is a metric that is equivalent to $\rho$.
Before proving that we need a few facts about $\gamma$.
\begin{lem}\label{SkorohodJ1GammaFacts}$\gamma(\lambda) = \gamma(\lambda^{-1})$ and
  $\gamma(\lambda_1 \circ \lambda_2) \leq \gamma(\lambda_1) +
  \gamma(\lambda_2)$.
\end{lem} 
\begin{proof}
These both follow from reparameterizations using the fact that
$\lambda^{-1}$ is a strictly increasing bijection.  For the first
\begin{align*}
\gamma(\lambda) &= 
\sup_{0 \leq s < t \leq T} \abs{\log  \frac{\lambda(t) -
    \lambda(s)}{t-s}} \\
&=\sup_{0 \leq \lambda^{-1}(s) < \lambda^{-1}(t) \leq T} \abs{\log
  \frac{\lambda(\lambda^{-1}(t)) -
    \lambda(\lambda^{-1}(s))}{\lambda^{-1}(t)-\lambda^{-1}(s)}} \\
&= \sup_{0 \leq s < t \leq T} \abs{\log  \frac{\lambda^{-1}(t) - \lambda^{-1}(s)}{t-s}} 
\end{align*}
and for the second
\begin{align*}
\gamma(\lambda) &= 
\sup_{0 \leq s < t \leq T} \abs{\log  \frac{\lambda_2(\lambda_1(t)) -
    \lambda_2(\lambda_1(s))}{t-s}} \\
&\leq \sup_{0 \leq s < t \leq T} \abs{\log  \frac{\lambda_2(\lambda_1(t)) -
    \lambda_2(\lambda_1(s))}{\lambda_1(t)-\lambda_1(s)}} +
\sup_{0 \leq s < t \leq T} \abs{\log  \frac{\lambda_1(t) -
    \lambda_1(s)}{t-s}} \\
&\leq \sup_{0 \leq s < t \leq T} \abs{\log  \frac{\lambda_2(t) -
    \lambda_2(s)}{t-s}} +
\sup_{0 \leq s < t \leq T} \abs{\log  \frac{\lambda_1(t) -
    \lambda_1(s)}{t-s}} \\
&=\gamma(\lambda_2) + \gamma(\lambda_1)
\end{align*}
\end{proof}

\begin{lem}\label{SkorohodEquivalenceA}For all $\lambda$ such that $\gamma(\lambda) < 1/2$ we have
  $\sup_{0 \leq t \leq T} \abs{\lambda(t) - t} \leq
  2T\gamma(\lambda)$.  For all $f,g \in D([0,T]; S)$ such that $d(f,g) < 1/2$ we
  have $\rho(f,g) \leq 2Td(f,g)$.
\end{lem}
\begin{proof}
From the inequality $1+x \leq e^x$ we have
$\log(1+2x) \leq 2x$ for all $x > -1/2$ and therefore for $0 < x <
1/2$ we have $\log(1-2x) \leq -2x < -x < 0$.  
Similarly we have
$\log(1-2x) \leq -2x$ for all $x < 1/2$ and therefore for $0 < x <
1/2$ we have $\log(1-2x) \leq -2x < -x < 0$ for $0 < x < 1/2$.  On the
other hand, we see that $\frac{d}{dx} \left ( \log(1+2x) - x \right) =
\frac{2}{1+2x} - 1$ is positive for $0 < x < 1/2$ and therefore we
conclude 
\begin{align*}
\log(1-2x) &< -x < 0 < x < \log(1+2x) \text{ for } 0 < x < 1/2
\end{align*}

Suppose $\gamma(\lambda) < 1/2$ and let $0 < t \leq T$.  By definition and the
fact that $\lambda(0) = 0$ we have
\begin{align*}
\abs{\log \frac{\lambda(t)}{t}} &\leq \sup_{0 \leq s < t \leq
  T}\abs{\log \frac{\lambda(t) - \lambda(s)}{t-s}} = \gamma(\lambda) 
\end{align*}
and therefore we get
\begin{align*}
\log(1-2\gamma(\lambda)) &< -\gamma(\lambda) < \log \frac{\lambda(t)}{t} < \gamma(\lambda) < \log(1+2\gamma(\lambda))
\end{align*}
and exponentiating
\begin{align*}
1 - 2\gamma(\lambda) < \frac{\lambda(t)}{t} < 1 + 2 \gamma(\lambda)
\end{align*}
and therefore $\abs{\lambda(t) - t} < 2 T \gamma(\lambda)$ for $0 < t \leq T$.  Since $\lambda(0) - 0 = 0$ 
it follows that $\sup_{0 \leq t \leq T} \abs{\lambda(t) - t} \leq 2 T \gamma(\lambda)$.

Now suppose we have $d(f,g) < 1/2$.  Let $0 < \epsilon < 1/2 - d(f,g)$ be given and
select $\lambda \in \Lambda$ such that $\gamma(\lambda) <
d(f,g) + \epsilon$ and $\sup_{0 \leq t \leq T} r(f(t),
g(\lambda(t))) < d(f,g) + \epsilon$.  By what we have just shown, we
get that $\sup_{0 \leq t \leq T} \abs{\lambda(t) - t} < 2 T
\gamma(\lambda) < 2T(d(f,g) + \epsilon)$ and therefore $\rho(f,g) <
2T(d(f,g) + \epsilon)$.  Now let $\epsilon \to 0$.
\end{proof}

\begin{lem}\label{SkorohodEquivalenceB}For any $\delta>1/4$ if $\rho(f,g) < \delta^2$ then $d(f,g) \leq 4\delta + w(f,\delta)$.
\end{lem}
\begin{proof}
Let $\delta > 0$ be given, by definition of $w(f,\delta)$ choose a partition $0=t_0 < t_1 < \dotsb < t_n=T$ such that 
$t_i - t_{i-1} > \delta$ and $\sup_{t_{i-1} \leq s \leq t < t_i} \abs{f(t) - f(s)} < w(f,\delta) + \delta$ 
for all $i=1, \dotsc, n$.  Using the fact that $\rho(f,g) < \delta^2$ to pick a $\mu \in \Lambda$ such that 
$\sup_{0 \leq t \leq T} r(f(t),g(\mu(t))) < \delta^2$  and $\sup_{0 \leq t \leq T} \abs{\mu(t) - t} < \delta^2$.  Note that 
by the properties of $\mu$ we also have $\sup_{0 \leq t \leq T} r(f(\mu^{-1}(t)), g(t)) < \delta^2$.

Now we construct an appropriate $\lambda$ with which to bound $d(f,g)$.  
Define $\lambda(t_i) = \mu(t_i)$ for each $i=0,\dotsc, n$ and extend by linear
interpolation
\begin{align*}
\lambda(t) &= \frac{t-t_{i-1}}{t_i-t_{i-1}} \mu(t_i) + \frac{t-t_{i}}{t_{i-1}-t_{i}} \mu(t_{i-1}) \text{ for $t_{i-1} \leq t \leq t_i$}
\end{align*}
From the fact that $\mu(t_{i-1}) < \mu(t_i)$ it follows that $\lambda(t)$ is strictly increasing, $\lambda(0) = \mu(0) = 0$, 
$\lambda(T) = \mu(T) = T$ and $\lambda$ is piecewise linear hence continuous; thus $\lambda \in \Lambda$.  Moreover by
the increasingness of $\lambda$ and $\mu$ (and $\mu^{-1}$) we have $t_{i-1} \leq t \leq t_i$ is equivalent to $\lambda(t_{i-1}) \leq \lambda(t) \leq \lambda(t_i)$
which is in turn equivalent to $\mu^{-1}(\lambda((t_{i-1})) = t_{i-1} \leq \mu^{-1}(\lambda(t)) \leq t_i = \mu^{-1}(\lambda( t_i))$.
Thus
\begin{align*}
r(f(t), g(\lambda(t))) &\leq r(f(t), f(\mu^{-1}(\lambda(t))))  + r(f(\mu^{-1}(\lambda(t)))- g(\lambda(t))) \\
&\leq w(f,\delta) + \sup_{0 \leq t \leq T}  f(\mu^{-1}(\lambda(t))) \leq w(f,\delta) + \delta^2
\end{align*}
As for bounding $\gamma(\lambda)$ we have
\begin{align*}
\abs{\lambda(t_i) - \lambda(t_{i-1}) - (t_i - t_{i-1})} &= \abs{\mu(t_i) - \mu(t_{i-1}) - (t_i - t_{i-1})} \\
&\leq \abs{\mu(t_i) -  t_{i}}  + \abs{\mu(t_{i-1}) - t_{i-1}} \leq 2 \delta^2 < 2 \delta (t_i - t_{i-1})
\end{align*}
Recalling that $\lambda$ is linear on each interval $[t_{i-1},t_i]$ we note that this inequality simply says that the
slope of $\lambda$ on the linear segment $[t_{i-1},t_i]$ is in the interval $(1 - 2\delta, 1-2\delta)$.  Thus the inequality trivial extends 
to any $t_{i-1} \leq s \leq t \leq t_i$.  For other $s<t$  pick $i<j$ with $t_{i-1} \leq s \leq t_i$ and $t_{j-1} \leq t \leq t_j$.  Then we have
\begin{align*}
\abs{\lambda(t) - \lambda(s) - (t-s)} 
&\leq \abs{\lambda(t) - \lambda(t_{j-1}) - (t-t_{j-1})} + \sum_{k=i}^{j-2} \abs{\lambda(t_{k+1}) - \lambda(t_{k}) - (t_{k+1}-t_{k})}  + \abs{\lambda(t_i) - \lambda(s) - (t_i-s)} \\
&\leq 2 \delta (t-t_{j-1}) + 2 \delta \sum_{k=i}^{j-2} (t_{k+1}-t_{k})  + 2\delta (t_i-s) = 2\delta (t-s) \\
\end{align*}
and therefore 
\begin{align*}
\log(1 - 2\delta) \leq \log \left( \frac{\lambda(t) - \lambda(s)}{t-s} \right) \leq \log(1+2\delta)
\end{align*}
For arbitrary $\delta$ we have $\log(1+2\delta) \leq 2\delta < 4 \delta$ (Theorem \ref{BasicExponentialInequalities}) and by 
Taylor's Theorem (Lemma \ref{LagrangeFormRemainder}) we have for any $0 < \delta < 1/4$ there is a $0 < c < \delta < 1/4$ such that
\begin{align*}
\log(1 - 2\delta) &= \frac{-2\delta}{1 - 2c} > -4\delta
\end{align*}
and therefore $\gamma(\lambda) < 4\delta$.
\end{proof}
Now we are ready to show that $d$ is a metric and is equivalent to $\rho$.
\begin{thm}\label{SkorohodJ1Metric}$d$ is a metric on $D([0,T];S)$ that is equivalent to
  $\rho$.
\end{thm}
\begin{proof}
The fact that $d(f,g) \geq 0$ is immediate. Suppose $d(f,g)$ and pick $\lambda_n$ such that
$\lim_{n \to \infty} \gamma(\lambda_n) = 0$ and $\lim_{n \to \infty} \sup_{0 \leq t \leq
  T} r(f(t), g(\lambda_n(t))) = 0$.  By Lemma
\ref{SkorohodEquivalenceA}  we know that $\lim_{n \to \infty} \sup_{0
  \leq t \leq T} \abs{\lambda_n(t) - t} = 0$ as well and therefore we
can repeat the argument of Lemma \ref{SkorohodJ1RhoMetric} to conclude $f=g$.

To see symmetry just note that by reparametrizing and Lemma \ref{SkorohodJ1GammaFacts}
\begin{align*}
d(f,g) &= \inf_{\substack{\lambda \in \Lambda \\ \gamma(\lambda) <
    \infty}} \gamma(\lambda) \vee \sup_{0 \leq t \leq T} r(f(t),
g(\lambda(t))) \\
&= \inf_{\substack{\lambda \in \Lambda \\ \gamma(\lambda) <
    \infty}} \gamma(\lambda^{-1}) \vee \sup_{0 \leq \lambda^{-1}(t) \leq T}
r(g(t), f(\lambda^{-1}(t))) = d(g,f)
\end{align*}
and similarly with the triangle inequality.

TODO: Write out the triangle inequality part.

To see that $\rho$ and $d$ define the same topology we first show that every ball
in the $\rho$ metric contains a ball in the $d$ metric and vice versa.  For general notation
let $B_\rho(f,\epsilon)$ and $B_d(f,\epsilon)$ denote balls of radius $\epsilon$ centered at $f$ in the $\rho$ and $d$ metric
respectively.  Let $f$ and $r > 0$ be given.  Define $\delta < \frac{\epsilon}{2T} \wedge \frac{1}{4}$ and then apply Lemma \ref{SkorohodEquivalenceA}
to conclude that $d(f,g) \leq \delta$ implies $\rho(f,g) \leq 2T d(f,g) < \epsilon$; thus $B_d(f,\delta) \subset B_\rho(f,\epsilon)$.  On the other hand for a given $\epsilon>0$
because $\lim_{\delta \to 0} 4 \delta + w(f,\delta) = 0$ we can find $0 < \delta < 1/4$ such that $4 \delta + w(f,\delta) < \epsilon$ and therefore
by Lemma \ref{SkorohodEquivalenceB} we have $B_\rho(f, \delta^2) \subset B_d(f, \epsilon)$.

Now let $U$ be an open set in the topology defined by $d$.  For every $f \in U$, by openness of $U$ we find $\epsilon_f>0$ such that $B_d(f,\epsilon_f) \subset U$.
By the above argument we may find $\delta_f>0$ such that $B_\rho(f, \delta_f) \subset B_d(f,\epsilon_f) \subset U$.  Therefore we can write $U = \cup_{f \in U} B_\rho(f, \delta_f)$
which shows that $U$ is an open set in the topology defined by $\rho$ as well.  It is clear that the argument is symmetric in the role of $\rho$ and $d$ and therefore we
have shown that $d$ and $\rho$ are equivalent metrics.
\end{proof}

The goal in introducing $d$ was to provide a complete metric; a useful
thing to check first is that $d$ fixes the example which showed $\rho$
was not a complete metric.
\begin{examp}Here we continue the Example
  \ref{NoncompletenessSkorohod} by showing directly that $f_n$ is not
  Cauchy in the metric $d$.  Because $f_n$ are indicator functions it
  follows that $\sup_{0 \leq t \leq 1} r(f_{n+m}(t), f_{n}(\lambda(t)))$ is
  either $0$ or $1$.  Therefore if $f_n$ is Cauchy then we can find
  $\lambda_{nm}(t)$ such that $\sup_{0 \leq t \leq 1} r(f_{n+m}(t),
    f_{n}(\lambda_{nm}(t))) = 0$.  By definition this tells us that
  $\lambda_{nm}([1/2,1/2 + 1/n+m+2]) =  [1/2,1/2+1/n]$ 
(of course $\lambda_{nm}([0,1/2]) = [0,1/2]$ and
$\lambda_{nm}([1/2 + 1/n+m+2, 1]) = [1/2 + 1/n,1]$ as well).  
From this fact we see that $\gamma(\lambda_{mn}) \geq
\frac{n+m+2}{n+2} > 1$ which shows that $d(f_n, f_{n+m}) \geq 1$ so
$f_n$ is not Cauchy with respect to $d$.
\end{examp}

Now we show that the metric $d$ is indeed complete.  In addition we show
that the $J_1$ topology is separable which shows us that it defines a Polish space.
This will allow us to apply our theory of weak convergence.
\begin{thm}\label{SkorohodJ1MetricPolish}Let $S$ be a complete metric space, then the metric $d$ on $D([0,T]; S)$
is complete.  Morever if $S$ is separable then $D([0,T];S)$ is separable in the $J_1$ topology.
\end{thm}
\begin{proof}
Let $f_n$ be a Cauchy sequence in $D([0,T]; S)$ with the metric $d$.  As a general principle of metric spaces it
suffices to show that $f_n$ has a convergence subsequence $f_{n_j}$.  Suppose that such a subsequence exists and converges
to $f$.  Then we may find $n_j$ such that $d(f,f_{n_j}) < \epsilon/2$ and $d(f_m, f_{n_j}) < \epsilon/2$ for all $m \geq n_j$; it follows
that in fact $d(f,f_m)<\epsilon$ for all $m \geq n_j$ and thus $f_n \to f$.

Using the Cauchy property we can find a subsequence $n_j$ such that $d(f{n_j}, f_{n_{j+1}}) < 2^{-j}$.  Therefore we have $\lambda_j$ 
such that $\gamma{\lambda_j} < 2^{-j}$ and $\sup_{0 \leq t \leq T} r(f_{n_j}(t), f_{n_{j+1}}(\lambda_j(t))) < 2^{-j}$.  Moreover from 
Lemma \ref{SkorohodEquivalenceA} we have $\sup_{0 \leq t \leq T} \abs{\lambda_j(t) -t } < T 2^{-j+1}$ for all $j \in \naturals$.  
Therefore for every $m,n \in \naturals$,
\begin{align*}
&\sup_{0 \leq t \leq T} \abs{\lambda_{m+n+1} \circ \lambda_{m+n} \circ \dotsb \circ \lambda_n(t) -  \lambda_{m+n} \circ \dotsb \circ \lambda_n(t)} \\
&= \sup_{0 \leq t \leq T} \abs{\lambda_{m+n+1}(t) - t} \leq T 2^{-(m+n)}
\end{align*}
which shows us that for fixed $n \in \naturals$ the sequence $\lambda_{m+n} \circ \dotsc \circ \lambda_n$ is Cauchy in $C([0,T], \reals)$ with
the sup norm.  Since this latter space is complete we know that there is  a limit $\nu_n$.  

\begin{clm} $\nu_n \in \Lambda$ and $\gamma(\nu_n) \leq 2^{-n+1}$.
\end{clm}

It is clear that $\nu_n$ is continuous as the uniform limit of continuous functions.  Moreover $\nu_n(0) = \lim_{m \to \infty} \lambda_{m+n} \circ \dotsb \lambda_{n}(0) = 0$
and similarly $\nu_n(T) = T$.  As a uniform limit of strictly increasing functions we also know that $\nu_n$ is non-decreasing.  To estimate $\gamma(\nu_n)$ we compute
\begin{align*}
&\abs{\log \frac{\lambda_{m+n} \circ \dotsb \circ \lambda_n(t) - \lambda_{m+n} \circ \dotsb \circ \lambda_n(s)}{t-s}} \\
&\leq \gamma(\lambda_{m+n} \circ \dotsb \circ \lambda_n) \\
&\leq \gamma(\lambda_{m+n}) + \dotsm +\gamma(\lambda_n) < 2^{-(m+n)} + \dotsm + 2^{-n} < 2^{-n+1}
\end{align*}
and therefore taking the limit as $m \to \infty$ we conclude 
\begin{align*}
\abs{\log \frac{\nu_n(t) - \nu_n(s)}{t-s}} &\leq 2^{-n+1}
\end{align*}
which show both that $\nu_n$ is strictly increasing (hence $\nu_n \in \Lambda$) and moreover that $\gamma(\nu_n) < 2^{-n+1}$ and the claim is shown.

Now note that $\nu_n = \nu_{n+1} \circ \lambda_n$.  From this it follows that
\begin{align*}
\sup_{0 \leq t \leq T} r(f_{n_j}(\nu_j^{-1}(t)), f_{n_{j+1}}(\nu_{j+1}^{-1}(t))) &= \sup_{0 \leq t \leq T} r(f_{n_j}(t),  f_{n_{j+1}}(\lambda_j(t))) < 2^{-j}
\end{align*}
and therefore for $j,m \in \naturals$ we have
\begin{align*}
\sup_{0 \leq t \leq T} r(f_{n_j}(\nu_j^{-1}(t)), f_{n_{j+m}}(\nu_{j+m}^{-1}(t))) &\leq 2^{-j} + \dotsm + 2^{-j-m+1} < 2^{-j+1}
\end{align*}
which shows that $f_{n_j} \circ \nu_j^{-1}$ is a Cauchy sequence in $D([0,T]; S)$ with respect to the sup norm.  Recalling that the sup norm
makes $D([0,T];S)$ into a complete (but not separable) space we can find a limit $f \in D([0,T]; S)$.  Thus $\sup_{0 \leq t \leq T} r(f_{n_j}(\nu_j^{-1}(t)), f(t)) \to 0$
and moreover $\gamma(\nu_j) \to 0$ which shows us that $d(f_{n_j}, f) \to 0$.

TODO: Show separability
\end{proof}

\subsection{Compactness and Tightness in Skorohod Space}

The entire point behind constructing the $J_1$ topology on $D([0,T]; S)$ was to
make it a Polish space so that Prohorov's Theorem can be applied to understand
weak convergence of cadlag stochastic processes.  The other pieces of the puzzle
in applying Prohorov's Theorem are being able to prove tightness and being able to
characterize limits (e.g. by understanding the finite dimensional distributions).  The latter
problem has little to do with topological aspects of path space (and to be honest in many
cases is an impossibly difficult nut to crack).  The former problem is deeply tied into the
nature of compactness in path space and we now turn to the consideration of such matters.

The first thing to do is to prove an analogue of the Arzela-Ascoli theorem that characterizes
relatively compact sets in $D([0,T]; S)$.  The proof of such a theorem ultimately rests on
approximation by step functions so we first prove that certain collections of step functions
are compact.

\begin{lem}\label{SkorohodJ1CompactSetsOfStepFunctions}Let $(S,r)$ be a metric space and let $K \subset S$.  Let $\delta > 0$ and define
$A(K, \delta) \subset D([0,T];S)$ be the set of functions $f$ for which there is a partition
$0=t_0 < t_1 < \dotsb < t_n=T$ with $t_i-t_{i-1} >  \delta$ for $i=1, \dotsc, n$, points $x_1, \dotsc, x_n \in K$ with $x_j \neq x_{j-1}$ for $j=2, \dotsc, n$ such that $f(t) = x_i$ for $t_{i-1} \leq t < t_i$.  If $K$ is compact in $S$ then $A(K, \delta)$ is relatively compact in the $J_1$ topology.
\end{lem}
\begin{proof}
It suffices to show that every sequence in $A(K,\delta)$ has a convergent subsequence. 
For each $f_n$ we by the definition of $A(K,\delta)$, let $0=t_{0,n} < \dotsm < t_{m_n, n}=T$   be a partition with $t_{j,n}-t_{j-1,n}>\delta$ for $j=1, \dotsc, m_n$ and $x_{1, n}, \dotsc, x_{m_n, n}$ be elements of $K$ such that $f_n(t) = x_{j,n}$ for $t_{j-1,n} \leq t < t_{j,n}$.  Let $m = \liminf_{n \to \infty} m_n$; that is to say $m$ is the smallest $m_n$ for which there are an infinite number of $f_n$ whose partitions have length $m$.
Note that $m \leq T/\delta < \infty$.  Since we are only looking for a convergence subsequence we can pass to the subsequence of $f_n$ for which $m_n = m$ and therefore we assume that all partitions have length $m$.

Consider the sequence $t_{1,n}$ and the sequence $x_{1,n}$.  Let $t_1 = \limsup_{n \to \infty} t_{1,n}$.  We know that $\delta < t_{1,n} \leq T$ thus $t_1 \in [\delta,T]$; pick a subsequence $N^1$ such that $t_{1,n} \to t_1$ along $N^1$. 
We also have $x_{1,n} \in K$ along $N^1$ so by compactness of  $K$ there 
is $x_1 \in K$ and a subsequence $N^2 \subset N^1$ such that $t_{1,n} \to t_1$ and $x_{1,n} \to x_1$ along $N^2$.  
In fact we can ask for another property of the sequence of $t_{1,n}$.  The function $g(t) = \abs{\frac{t_1}{t}}$ is continuous at $t_1$ 
and equals $g(t_1) = 0$ thus $g(t_{1,n}) \to 0$ along $N^2$ and we may pass to a further subsequence $N^3 \subset N^2$ (which we now
denote by $n_j$) to arrange that
\begin{itemize}
\item[(i)] $\lim_{j \to \infty} t_{1,n_j} = t_1$ with $\delta \leq t_1 \leq T$
\item[(ii)] $\abs{\frac{t_1}{t_{1, n_j}}} < \frac{1}{j}$ for $j \in \naturals$
\item[(iii)] $\lim_{j \to \infty} x_{1,n_j} = x_1$ with $x_1 \in K$
\end{itemize}

TODO: Get rid of the $k=1$ step since the induction step is clear once we define the trivial $k=0$ step.

If $m = 1$ we stop here (and note that we must have $t_1 = T$), otherwise we iterate to find further subsequences. 
To see the induction step we suppose that we have run the procedure $k<m$ times 
so that we have a subsequence $N^k$ and for each
$1 \leq i \leq k$ we have $i\delta \leq t_i < T$, $x_i \in K$ such that
\begin{itemize}
\item[(i)] $\lim_{j \to \infty} t_{i,n_j} = t_1$ 
\item[(ii)] $\abs{\frac{t_i - t_{i-1}}{t_{i, n_j} - t_{i-1, n_j}}} < \frac{1}{j}$ for $j \in \naturals$
\item[(iii)] $\lim_{j \to \infty} x_{i,n_j} = x_1$ 
\end{itemize}
We now let $t_{k+1} = \limsup_{j \to \infty} t_{k+1, n_j}$.  We know 
that $(k+1) \delta \leq t_{k+1} \leq T - (m - k)\delta$ and in fact $t_{k+1} = T$ if and only if $k+1 = m$.  
Now we replay the argument we used for $k=1$.  We extract a subsequence $N^{1,k} \subset N^k$ such that $t_{k+1, n_j} \to t_k$ along $N^{1,k}$.  Note also
that $m_{n_j} \geq k+1$ along this subsequence.  We use
compactness of $K$ to the get a further subsequence $N^{2,k}$ such that there is $x_{k+1} \in K$ with $x_{k+1, n_j} \to x_{k+1}$ along $N^{2,k}$.  The we use
continuity of $\abs{\frac{t_{k+1} - t_{k}}{t - t_{k, n_j}}}$ at $t_k$ to arrange for a final subsequence $N^{3,k}$ so that if we redefine $n_j$ to be the subsequence
$N^{3,k}$ we have $\abs{\frac{t_{k+1}- t_{k}}{t_{k+1, n_j} - t_{k, n_j}}} < \frac{1}{j}$ for all $j \in \naturals$.

Define $f(t)$ to be equal to $x_k$ on $t_{k-1} \leq t < t_k$ for $k=1, \dotsc, m$ (clearly $f \in D([0,T]; S)$).  We claim $f_n \to f$ in the $J_1$ topology along the subsequence $N^m$.
To see this, let $\lambda_j(t_{k, n_j}) = t_k$ for $k=1, \dotsc, m$ and extend by linear interpolation.  This is well defined because of the fact that $t_m = T$.  
Moreover from the property (ii) we have $\gamma(\lambda_j) \leq \frac{1}{j}$ so that
$\lim_{j \to \infty} \gamma(\lambda_j) = 0$.  We also have $r(f_{n_j}(t), f(\lambda_j(t))) = r(x_{n_j, k}, x_k)$ for $t_{k-1, n_j} \leq t < t_{k, n_j}$.  Thus because there are only finitely many
$x_k$ we can conclude that $\lim_{j \to \infty} \sup_{0 \leq t \leq T} r(f_{n_j}(t), f(\lambda_j(t))) = 0$ and therefore $\lim_{j \to \infty} d(f_{n_j}, f) = 0$.

TODO: Can the proof be simplified if we use the metric $\rho$ to demostrate convergence?
\end{proof}

\begin{thm}\label{ArzelaAscoliTheoremJ1Topology}Let $(S,r)$ be a complete metric space.  A set $A \subset D([0,T]; S)$ is relatively compact in the $J_1$ topology if and only if 
\begin{itemize}
\item[(i)]for each rational number
$t \in [0,T] \cap \rationals$ there exists a compact set $K_t \subset S$ such that $\cup_{f \in A} f(t) \subset K_t$
\item[(ii)]$\lim_{\delta \to 0} \sup_{f \in A} w(f, \delta) = 0$
\end{itemize}
\end{thm}
\begin{proof}
We consider $D([0,T];S)$ as a metric space with the complete metric $d$.  We first suppose that $A$ is a set satisfying (i) and (ii).
Since a set $A$ is totally bounded if and only if 
its closure is totally bounded and a closed set of  a complete metric space is complete, it suffices to
show that $A$ is totally bounded with respect to $d$ (Theorem \ref{CompactnessInMetricSpaces}).  By (ii) every $k \in \naturals$ we
pick $0 < \delta_k < 1$ such that $\sup_{f \in A} w(f, \delta_k) < \frac{1}{k}$.  Pick $m_k \in \naturals$ with $\frac{1}{m_k} < \delta_k$ and 
define 
\begin{align*}
K^{(k)}  &= \cup_{i=1}^{T m_k} K_{i/m_k}
\end{align*}
so that $K^{(k)}$ is compact and $A_k = A(K^{(k)}, \delta_k)$ is relatively compact by Lemma \ref{SkorohodJ1CompactSetsOfStepFunctions}.
Pick $f \in A$ and since $w(f,\delta_k) < \frac{1}{k}$ we may choose a partition $0=t_0 < t_1 < \dotsb < t_n=T$ such that  for $i=1, \dotsc, n$ we have 
$t_i-t_{i-1} > \delta_k$ and $r(f(s), f(t)) < \frac{1}{k}$ for all $s,t \in [t_{i-1}, t_i)$.  Since we have chose $\frac{1}{m_k} < \delta_k < t_i - t_{i-1}$ note that every
interval $[\frac{j-1}{m_k}, \frac{j}{m_k}]$ has either $0$ or $1$ element of $t_i$ in it.  Define $g \in D([0,T]; S)$ by
\begin{align*}
g(t) &= f(\frac{\floor{m_k t_{i-1}} + 1}{m_k}) \text{ for $t_{i-1} \leq t < t_i$ and $i=1, \dotsc, n$}
\end{align*}
(this says that we define $g(t)$ on $[t_{i-1}, t_i)$ to be the value $f(j/m_k)$ where $j$ is the smallest integer such that $t_{i-1} < j/m_k$).
It is clear that $g \in A_k$ since $g$ is a step function on the partition $\lbrace t_j \rbrace$ and takes values in $K_{j/m_k}$ for appropriate $0 \leq j \leq T m_k$.
Furthermore we have for $t_{i-1} \leq t < t_i$ we have
\begin{align*}
r(f(t), g(t)) &\leq r(f(t), f(\frac{\floor{m_k t_{i-1}} + 1}{m_k})) < \frac{1}{k}
\end{align*}
which shows that $d(f,g) < \frac{1}{k}$.  

Now let $\epsilon > 0$ be given and pick $\frac{1}{k} < \epsilon /2$.  Since $A_k$ is relatively compact it
is totally bounded and thus there exist $g_1, \dotsc, g_n$ such that the balls $B(g_i, \epsilon/2)$ cover
$A_k$.  By the above argument for any $f \in A$ we can find $g \in A_k$ such that $d(f,g) < \frac{1}{k} < \epsilon/2$ 
and therefore by the triangle inequality the balls $B(g_i, \epsilon)$ cover $A$ and $A$ is totally bounded.

TODO: Do the converse.
\end{proof}

\section{Riesz Representation}

We saw in the Daniell-Stone Theorem \ref{DaniellStoneTheorem} that one
may recapture a part of integration theory by considering certain
linear functionals on a space of functions.  There is a analogue to
that result that applies in the case of measure on topological
spaces and allows one to bring the machinery of functional analysis to
bear on problems of measure theory.  

The Riesz representation theorem is actually a class of different
theorems with different hypotheses made about the measures involved
and the topology on the
underlying space.  Here we concentrate the reasonable general case of
Hausdorff locally compact spaces.  Other presentations may treat the
slightly simpler cases in which either second countability,
compactness or
$\sigma$-compactness are added as hypotheses on the topological space.  More general
presentations may drop the the assumption of local compactness and treat
arbitrary Hausdorff spaces.  

\begin{defn}A topological space $X$ is said to be \emph{locally
    compact} if every point in $X$ has a compact neighborhood
  (i.e. for every $x \in X$ there exists an open set $U$ and a compact
  set $K$ such that $x \in U \subset K$).
\end{defn}

\begin{lem}\label{LocallyCompactEquivalences}Let $X$ be a Hausdorff topological space then the following
  are equivalent
\begin{itemize}
\item[(i)]$X$ is locally compact
\item[(ii)]Every point in $X$ has an open neighborhood with compact closure
\item[(iii)]$X$ has a base of relatively compact neighborhoods
\end{itemize}
\end{lem}
\begin{proof}
(i) implies (ii):  If $X$ is Hausdorff then a closed subset of a compact set is compact
and therefore if $X$ is locally compact and $x \in X$ we take $U$ open
and $K$ compact such that $x \in U \subset K$ and then it follows that
$\overline{U}$ is compact hence (ii) follows.  

The fact that (ii) implies (i) is immediate.

(ii) implies (iii): For each $x \in X$ pick a relatively compact
neighborhood $U_x$, let $\mathcal{B}_x = \{ U \subset U_x \mid U
\in \mathcal{T} \}$ and let $\mathcal{B} = \cup_{x \in X}
\mathcal{B}_x$.  It is clear that $\mathcal{B}$ is a base for the
topology $\mathcal{T}$.  Moreover for each $U
\in \mathcal{B}$ there exists $x \in X$ such that $U \subset U_x$ with
$\overline{U}_x$ compact and then since $X$ is Hausdorff we know that
$\overline{U}$ is compact.

(iii) implies (ii) is immediate.
\end{proof}

\begin{prop}\label{CompleteRegularityLCH}A locally compact Hausdorff space X is completely regular
  (i.e. for every $x \in X$ and closed set $F \subset X$ such that $x
  \notin F$ there is are disjoint open sets $U$ and $V$ such that $x
  \in U$ and $F \subset V$).
\end{prop}
\begin{proof}
Let $F$ be a closed set and pick $x \in X \setminus F$.  By Lemma
\ref{LocallyCompactEquivalences} and the openness of $X \setminus F$
we can find a relatively compact neighborhood $U_0$ of $x$ such that $x
\in U_0 \subset X \setminus F$.  The set $\overline{U_0} \cap F$ is a closed subset of a compact
set hence is compact.  For each $y \in \overline{U_0} \cap F$ by the
Hausdorff property we may find open neighborhoods $x \in U_y$ and $y
\in V_y$ such that $U_y \cap V_y = \emptyset$.  By compactness of
$\overline{U} \cap F$ we get a finite subcover $V_{y_1}, \dotsc,
V_{y_n}$ of $\overline{U_0} \cap F$.  Now define $U = U_0 \cap U_{y_1}
\cap \dotsb U_{y_n}$.  This is an open neighborhood of $x$ and
moreover $\overline{U} \cap F = \emptyset$.  Define $V = X \setminus \overline{U}$.
\end{proof}

\begin{defn}Let $X$ be a topological space, then a subset $A \subset
  X$ is said to be \emph{bounded} if there exists a compact set $K$
  such that 
  $A \subset K \subset X$.  A subset $A \subset
  X$ is said to be \emph{$\sigma$-bounded} if there exists a sequence
  of compact sets $K_1, K_2, \dotsc$ such that
$A \subset  \cup_{i=1}^\infty K_i \subset X$.
\end{defn}

\begin{prop}\label{SigmaBoundedEquivalence}A set $A$ is $\sigma$-bounded Borel set if and only if
  there exist disjoint bounded Borel sets $A_1, A_2, \dotsc$ such that $A =
  \cup_{i=1}^\infty A_i$.
\end{prop}
\begin{proof}
Suppose $A$ is a $\sigma$-bounded Borel set and let $K_1, K_2, \dotsc$
be compact sets such that $A \subset \cup_{i=1}^\infty K_i$.  Define
$A_1= A \cap K_1$ and for $n>1$ let $A_n = A \cap K_n \setminus
\cup_{j=1}^{n-1} A_j$  Trivially each $A_n$ is bounded (it is contained in
$K_n$), $A = \cup_{i=1}^\infty A_i$ (by construction $A_n \subset A$
and for any $x \in A$ we can find $n$ such that $x \in K_n$; it
follows that $x \in A_n$).  Moreover by construction it is clear that
the $A_n$ are Borel.  On the other hand, if $A =
\cup_{i=1}^\infty A_i$ with $A_i$ bounded and Borel and disjoint, then take $K_i$ compact
such that $A_i \subset K_i$ and it follows that $A \subset
\cup_{i=1}^\infty K_i$.  $A$ is clearly Borel as it is a countable
union of Borel sets.
\end{proof}

\begin{lem}\label{BoundedNeighborhoodsOfCompactSets}Let $K$ be a
  compact set in a locally compact Hausdorff topological
  space $X$, then there exists a bounded open set $U$ such that $K
  \subset U$.  Moreover if $V$ is a open set such that $K \subset V$
  then there is a bounded open set $U$ such that $K \subset U \subset
  \overline{U} \subset V$.
\end{lem}
\begin{proof}
By taking $V = X$ we see the second assertion implies the first so it
suffices to prove the second assertion.  By complete regularity of $X$
(Proposition \ref{CompleteRegularityLCH}) and local compactness of $X$
for each $x \in K$ we may
find a relatively compact open neighborhood $x \in U_x$ such that
$\overline{U}_x \cap V^c = \emptyset$.  By compactness of $K$ we may
take a finite subcover $U_{x_1}, \dotsc, U_{x_n}$.  Then
$U = U_{x_1} \cup \dotsb \cup U_{x_n}$ is an open set with $K \subset
U$ and
$\overline{U} = \overline{U}_{x_1} \cup \dotsb \cup \overline{U}_{x_n}$ is
a finite union of
compact sets and is therefore compact.  Lastly $\overline{U} \cap
V^c = (\overline{U}_{x_1} \cap V^c) \cup \dotsb \cup
(\overline{U}_{x_n} \cap V^c) = \emptyset$ and therefore $K \subset U
\subset \overline{U} \subset V$.
\end{proof}

For our purposes the reason for bringing up $\sigma$-bounded sets is
the fact that the properties of inner and outer regularity are
essentially equivalent on them.

\begin{lem}\label{InnerOuterRegularityEquivalence}Let $X$ be a locally compact Hausdorff topological space and let $\mu$ be a measure
  that is finite on compact sets.  Then $\mu$ is inner regular on
  $\sigma$-bounded Borel sets if and only if $\mu$ is outer regular on
  $\sigma$-bounded Borel sets.
\end{lem}
\begin{proof}
Suppose that $\mu$ is inner regular on $\sigma$-bounded sets.  Let $A$
be a bounded Borel set and suppose $\epsilon > 0$ is given.   First,
note that $\overline{A}$ is compact so may apply
Lemma \ref{BoundedNeighborhoodsOfCompactSets} to find a bounded
open set $U$ such that $\overline{A} \subset U$.  Therefore
$\overline{U} \setminus A$ is a bounded Borel set so by inner
regularity we may find a
compact set $K \subset \overline{U} \setminus A$ such that 
\begin{align*}
\mu(\overline{U} \setminus A) - \epsilon &< \mu(K) \leq \mu(\overline{U} \setminus A)
\end{align*}
Let $V = U \cap K^c$.  Then $V$ is an open set and $A \subset V$.
Moreover,
\begin{align*}
\mu(V) &= \mu(U) - \mu(K) \leq \mu(\overline{U}) - \mu(\overline{U}
\setminus A) + \epsilon = \mu(A) + \epsilon
\end{align*}
Since $\epsilon > 0$ was arbitrary we see that $\mu$ is outer regular
on bounded Borel sets.  Now we need to extend to outer regularity on
$\sigma$-bounded sets.  Let $A$ be a $\sigma$-bounded Borel set and
let $\epsilon > 0$ be given.  Apply Lemma
\ref{SigmaBoundedEquivalence}
to find disjoint bounded Borel sets $A_i$ such that $A = \cup_{i=1}^\infty
A_i$.  By the just proven outer regularity on bounded Borel sets we
may find open sets $U_i$ such that $\mu(U_i) \leq \mu(A_i) +
\epsilon/2^i$.  Then clearly $A \subset U$, $U$ is open and 
\begin{align*}
\mu(U) &\leq \sum_{i=1}^\infty \mu(U_i) \leq \epsilon + \sum_{i=1}^\infty
\mu(A_i)  = \epsilon + \mu(A)
\end{align*}
Again, as $\epsilon > 0$ is arbitrary we see that $\mu$ is outer
regular on $\sigma$-bounded Borel sets.

Now we assume that $\mu$ is outer regular on $\sigma$-bounded Borel
sets.  As before we start with the bounded case.  Let $A$ be a bounded
Borel set and suppose that $\epsilon > 0$ is given.  Let $L$ be a
compact set such that $A \subset L$.  Since $L \setminus A$ is also a
bounded Borel set, we may apply outer regularity to find an open set
$U$ such that $L \setminus A \subset U$ and 
\begin{align*}
\mu(U) - \epsilon < \mu(L \setminus A) \leq \mu(U)
\end{align*}
Define $K = L \setminus U = L \cap U^c$. As $K$ is a closed subset of the compact set $L$
it is compact.  Also
\begin{align*}
\mu(K) &= \mu(L) - \mu(L \cap U) \geq \mu(L) - \mu(U) = \mu(A) + \mu(L
\setminus A) - \mu(U) > \mu(A) - \epsilon
\end{align*}
As $\epsilon >0$ was arbitrary we see that $\mu$ is inner regular on
bounded Borel sets.  

Lastly we extend inner regularity to
$\sigma$-bounded Borel sets.  Let $A$ be $\sigma$-bounded Borel and
write $A = \cup_{i=1}^\infty A_i$ with the $A_i$ disjoint and each
$A_i$ bounded Borel (Lemma \ref{SigmaBoundedEquivalence}).  Let
$\epsilon > 0$ be given.  As each
$A_i$ is bounded and $\mu$ is finite on compact sets it follows that
$\mu(A_i) < \infty$ for all $i \in \naturals$.  Now by the just proven
inner regularity on bounded Borel sets we find $L_i \subset A_i$ with
$L_i$ compact and
\begin{align*}
\mu(A_i) - \epsilon/2^i < \mu(L_i) \leq \mu(A_i)
\end{align*}
The disjointness of the $A_i$ implies that the $L_i$ are disjoint as well.
Let $K_n = L_1 \cup \dotsb \cup L_n$ and note that
\begin{align*}
\mu(K_n) &= \sum_{i=1}^n \mu(L_i) > \sum_{i=1}^n \left( \mu(A_i) -
\epsilon/2^i \right ) > \sum_{i=1}^n \mu(A_i) - \epsilon
\end{align*}
Now take the limit as $n \to \infty$ to conclude that 
\begin{align*}
\sup \lbrace \mu(K) \mid K \subset A \text{ and $K$ is compact}
\rbrace &\geq \sup_n \mu(K_n) \geq \mu(A) - \epsilon
\end{align*}
and as $\epsilon > 0$ was arbitrary inner regularity of $\mu$ on
$\sigma$-bounded Borel sets is proven.
\end{proof}

\begin{defn}Let $X$ be a topological space the $C_c(X)$ is the set of
  all continuous function $f : X \to \reals$ with compact support
  (i.e. $supp(f) = \overline{\{x \in X \mid f(x) \neq 0 \}}$ is
  compact).
\end{defn}

\begin{defn}Let $X$ be a topological space the $C_0(X)$ is the set of
  all continuous function $f : X \to \reals$ which vanish at infinity
  in the sense that for every $\lambda >0$ the set $\{ x \in X \mid
  \abs{f(x)} \geq \lambda \}$ is  compact.
\end{defn}

\begin{prop}\label{BanachSpaceOfFunctionsVanishingAtInfinity}If $X$ is a topological space, then  for each $f \in C_0(X)$
  define $\norm{f} = \sup_{x \in X} \abs{f(x)}$ then $\norm{f}$ is a
  norm on $C_0(X)$ and $C_0(X)$ is a Banach space.  Furthermore
  $C_c(X)$ is dense in $C_0(X)$.
\end{prop}
\begin{proof}
TODO:
\end{proof}

The difficult part of the Riesz-Markov Theorem is the construction of
a Radon measure that corresponds to a positive functional.  The
tradition is to break that construction into two pieces: first the
construction of a set function on a smaller class of sets than the
full $\sigma$-algebra  and secondly the extension of that set function
to a full blown Radon measure. In many developments the set function
is defined on the compact subsets of the  locally compact
Hausdorff space $X$ and are called \emph{contents}.  Following
Arveson, we choose a set function is one
that is defined on just the open subsets of $X$.

The description of the desireable properties of the set function and
the process of extending the set function to a Radon measure is dealt with in the following Lemma.
\begin{lem}\label{ExtensionToRadonMeasure}Let $X$ be a locally compact Hausdorff space and let $m$ be
  a function from the open set of $X$ to $[0,\infty]$ satisfying:
\begin{itemize}
\item[(i)]$m(U) < \infty$ if $\overline{U}$ is compact
\item[(ii)]if $U \subset V$ then $m(U) \leq m(V)$
\item[(iii)]$m(\cup_{i=1}^\infty U_i) \leq \sum_{i=1}^\infty m(U_i)$
  for all open sets $U_1, U_2, \dotsc$.
\item[(iv)]if $U \cap V = \emptyset$ then $m(U \cup V) = m(U) + m(V)$
\item[(v)]$m(U) = \sup \lbrace m(V) \mid V \text{ is open, } \overline{V} \subset U \text{ and }
  \overline{V} \text{ is compact} \rbrace$
\end{itemize}
then there is a unique Radon measure $\mu$ such that $\mu(U) = m(U)$
for all open sets $U$.  Moreover every Radon measure satisfies
properties (i) through (v) when restricted to the open subsets of $X$.
\end{lem}
\begin{proof}
First we show that a Radon measure satisfies properties (i) through
(v) on the open sets of $X$.  In fact, properties (ii), (iii) and (iv)
follow for all measures and (i) follows from the fact that $\mu$ is
finite on compact subsets and monotonicity of measure.  Property (v)
requires a bit more justification.  If we let $U$ is an open set and
$\epsilon > 0$ is given then by inner regularity of $\mu$ we may find
a compact set $K$ such that $K \subset U$ and $\mu(U) \geq \mu(K) > \mu(U) -
\epsilon$.  By Lemma \ref{BoundedNeighborhoodsOfCompactSets} we may
find a relatively compact open set $V$ such that $K \subset V \subset
\overline{V} \subset U$.  Then by monotonicity we have $\mu(U) \geq \mu(V) > \mu(U) -
\epsilon$ and since $\epsilon$ was arbitrary (v) follows.

Next we prove uniqueness of the extension of $m$ to a Radon measure
$\mu$.  Since a Radon measure is inner regular on all Borel sets Lemma
\ref{InnerOuterRegularityEquivalence} implies that any extension $\mu$
is outer regular on all $\sigma$-bounded Borel sets.  Since the values
of $\mu$ are determined on all open sets this implies that the values
of $\mu$ are determined on all $\sigma$-bounded Borel sets; in
particular the values of $\mu$ are determined on all compact
sets. Clearly a Radon measure is determined uniquely by its values on
compact sets.

Now we turn to proving existence of the extension $\mu$.  The proof
goes in a few steps.  First we define an outer measure from $m$ and
observe that Borel sets are measurable with respect to it; though the
Caratheordory restriction of the outer
measure is outer regular it is not necessarily inner regular.  The
second step is to 
modify the Caratheodory restriction to make it inner regular.  

We begin by defining the outer measure in a standard way.  Let $A$ be
an arbitrary subset of $X$ and define
\begin{align*}
\mu^*(A) &= \inf \lbrace m(U) \mid A \subset U \text{ and $U$
  is open} \rbrace
\end{align*}
Note that $\mu^*(U) = m(U)$ for all open sets.

\begin{clm}$\mu^*$ is an outer measure
\end{clm}

Note that because the emptyset is relatively compact we know from (i)
that $m(\emptyset) < \infty$ and thus from (iv) we see that
$m(\emptyset) = 2m(\emptyset)$.  Thus $m(\emptyset) = 0$ and it
follows that $\mu^*(\emptyset) = 0$.  If $A \subset B$ then it is
trival that 
\begin{align*}
\lbrace m(U) \mid A \subset U \text{ and $U$
  is open } \rbrace &\subset \lbrace m(U) \mid B \subset U \text{ and $U$
  is open } \rbrace
\end{align*}
which implies $\mu^*(A) \leq \mu^*(B)$.  If we let $A_1, A_2, \dotsc$
be given and define $A = \cup_{i=1}^\infty A_i$.  If any
$\mu^*(A_i) = \infty$ it follows that $\mu(A) \leq \sum_{i=1}^\infty
\mu^*(A_i) = \infty$.  If on the other hand every $\mu^*(A_i) <
\infty$ then let $\epsilon > 0$ be given and find an open set $U_i
\subset A_i$ such that $m(U_i) \leq \mu^*(A_i) + \epsilon / 2^i$.
Clearly $\cup_{i=1}^\infty U_i$ is an open subset of $A$ and it
follows from (iii) and the definition of $\mu^*$ that
\begin{align*}
\mu^*(A) &\leq m(\cup_{i=1}^\infty U_i) \leq \sum_{i=1}^\infty m(U_i)
\leq \sum_{i=1}^\infty m(A_i) + \epsilon
\end{align*}
Since $\epsilon > 0$ is arbitrary we see that $\mu^*$ is countably
subadditive and is therefore proven to be an outer measure.

\begin{clm}Borel sets are $\mu^*$-measurable.
\end{clm}

The $\mu^*$-measurable sets form a $\sigma$-algebra by Lemma
\ref{CaratheodoryRestriction} and therefore it suffices to show that
open sets are $\mu^*$-measurable.  Let $U$ be open subset and $A$ be an
arbitrary subset of $X$, by subadditivity of $\mu^*$ we only have to
show the inequality 
\begin{align*}
\mu^*(A) &\geq \mu^*(A \cap U) + \mu^*(A \cap U^c)
\end{align*}
Obviously we may assume that $\mu^*(A) < \infty$ since otherwise the
inequality is trivially satisfied.  
We first assume that $A$ is an open set.  Since $\mu^*$ and $m$ agree
on open sets we have to show
\begin{align*}
m(A) &\geq m(A \cap U) + \mu^*(A \cap U^c)
\end{align*}
Let $\epsilon > 0$ be given and use property (v) so we can find an relatively compact open set $V$ such
that $\overline{V} \subset A \cap U$ and $m(V) \geq m(A \cap U) -
\epsilon$.  Then $A \cap \overline{V}^c$ is an open set containing $A
\cap U^c$ disjoint from $V$ and it follows from (ii), (iv)  and the
definition of $\mu^*$ that
\begin{align*}
m(A) &\geq m(V \cup A \cap \overline{V}^c) = m(V) + m(A \cap
\overline{V}^c) \geq m(A \cap U) - \epsilon + \mu^*(A \cap U^c) 
\end{align*}
As $\epsilon > 0$ was arbitrary we are done with the case of open
sets $A$.  Now suppose that $A$ is an arbitrary set with $\mu^*(A) <
\infty$ and let $\epsilon
> 0$ be given.  We find an open set $V$ such that $A \subset V$ and
$m(V) \leq \mu^*(A) + \epsilon$.  From what we have just proven of
open sets and the monotonicity of $\mu^*$
\begin{align*}
\mu^*(A) + \epsilon &\geq \mu^*(V) \geq \mu^*(V \cap U) + \mu^*(V \cap
U^c) \geq \mu^*(A \cap U) + \mu^*(A \cap U^c) 
\end{align*}
The claim follows by observing that $\epsilon >0$ was arbitrary.

Now by Caratheodory Restriction (Lemma \ref{CaratheodoryRestriction}) we may restrict $\mu^*$ to a Borel measure
$\overline{\mu}$ that is outer regular by definition and that
satisfies $\overline{\mu}(U) = m(U)$ for all open sets $U$.  Moreover
$\overline{\mu}(K) < \infty$ for all compact sets since by Lemma
\ref{BoundedNeighborhoodsOfCompactSets} we may find a relatively
compact open neighborhood $U$ such that $K \subset U$; monotonicity
and (i) tell us that 
\begin{align*}
\overline{\mu}(K) \leq \overline{\mu}(U) = m(U) <
\infty
\end{align*}  
Since $\overline{\mu}$ is outer regular on all Borel sets \emph{a fortiori}
it is outer regular on all $\sigma$-bounded Borel sets.  By  Lemma
\ref{InnerOuterRegularityEquivalence} it follows that
$\overline{\mu}$ is inner regular on all $\sigma$-bounded Borel sets.
Note that if we assume that $X$ is $\sigma$-compact (i.e. all Borel
sets are $\sigma$-bounded) then we already know that $\overline{\mu}$
is a Radon measure.  In the general case it is not necessarily true
and we must make a further modification to $\overline{\mu}$ to make it
inner regular.

For an arbitrary Borel set $A$ we define
\begin{align*}
\mu(A) &= \sup \lbrace \overline{\mu}(B) \mid B \subset A \text{ and
  $B$ is a $\sigma$-bounded Borel set } \rbrace
\end{align*}
Clearly, $\mu(\emptyset) = \overline{\mu}(\emptyset) = 0$.
It is also immediate from the definition that $\mu(A) = \overline{\mu}(A)$
for all $\sigma$-bounded Borel sets $A$ and therefore that $\mu(U) =
m(U)$ for all $\sigma$-bounded open sets $U$.  In fact more is true.

\begin{clm}$\mu(U) = m(U)$ for all open sets $U$.
\end{clm}

Let $V$ be a relatively compact open set with $\overline{V} \subset U$.  We have
\begin{align*}
m(U) &= \overline{\mu}(U) \geq \mu(U) \geq \mu(V) = m(V)
\end{align*}
Now we take the supremum over all such $V$ and by property (v) 
\begin{align*}
m(U) &\geq \mu(U) \sup \lbrace m(V) \mid V \text{ is relatively
  compact and $\overline{V} \subset U$} \rbrace = m(U)
\end{align*}
and therefore $\mu(U) = m(U)$.

\begin{clm}$\mu$ is a measure.
\end{clm}

To see that $\mu$ is a measure it remains to show countable
additivity.  Let $A_1, A_2, \dotsc$ be disjoint Borel sets.  First we
show countable subadditivity.  Let $B$ be a $\sigma$-bounded Borel
subset of $\cup_{i=1}^\infty A_i$ and define $B_i = B \cap A_i$.
Clearly the $B_i$ are disjoint $\sigma$-bounded Borel measures, thus
using the countable additivity of $\overline{\mu}$ we get
\begin{align*}
\overline{\mu{B}} &= \sum_{i=1}^\infty \overline{\mu}(B_i) \leq \sum_{i=1}^\infty \mu(A_i)
\end{align*}
Taking the supremum over all such $B$ subadditivity follows.

We need to show the opposite inequality.  Suppose that some $\mu(A_j) = \infty$ for some $j$.  Then we may find
a sequence of $\sigma$-bounded Borel sets $B_n$ such that
$\overline{\mu}(B_n) \geq n$.  Since $B_n \subset \cup_{i=1}^\infty
A_i$ we also see that $\mu(\cup_{i=1}^\infty A_i) = \infty$.  Thus we
may now assume that $\mu(A_i) < \infty$ for all $i$.  Let $\epsilon >
0$ be given and for each $i$ find a $\sigma$-bounded Borel set $B_i$
such that $B_i \subset A_i$ and $\overline{\mu}(B_i) \geq \mu(A_i) -
\epsilon/2^i$.  For each $n$ define $C_n = \cup_{j=1}^n B_j$ and note
that $C_n$ is a $\sigma$-bounded Borel set such that $C_n \subset
\cup_{i=1}^\infty A_i$.  Also, for every $n$, 
\begin{align*}
\mu(\cup_{i=1}^\infty A_i) \geq \mu(C_n) = \overline{\mu}(C_n) =
\sum_{j=1}^n \overline{\mu}(B_j) \geq \sum_{j=1}^n \mu(A_j)  -
\epsilon/2^j \geq \sum_{j=1}^n \mu(A_j)  -\epsilon
\end{align*}
Now take the limit as $n \to \infty$ and using the fact that $\epsilon
> 0$ was arbitrary, we get $\sum_{j=1}^\infty \mu(A_j) \leq \mu(\cup_{j=1}^\infty A_j)$.

\begin{clm}$\mu$ is a Radon measure.
\end{clm}

The fact that $\mu(K) < \infty$ for all compact sets follows from the
fact that $\mu$ and $\overline{\mu}$ agree on $\sigma$-bounded sets
and the fact that $\overline{\mu}(K) < \infty$.  To see inner
regularity, let $A$ be a Borel set and let $\epsilon > 0$ be given.
By the definition of $\mu$ we find a $\sigma$-bounded Borel set $B
\subset A$ such that $\overline{\mu}(B) \geq \mu(A) - \epsilon/2$.  Then by
the fact that $\overline{\mu}$ is inner regular on $\sigma$-bounded
sets we find a compact set $K$ such that $\overline{\mu}(K) \geq
\overline{\mu}(B) - \epsilon/2$.  Combining the two inequalities and
using the fact that $\mu$ and $\overline{\mu}$ agree on compact sets
we get $\mu(K) \geq \mu(A) - \epsilon$.  Since $\epsilon > 0$ was
arbitrary we are done.
\end{proof}

Given a Radon measure on a locally compact Hausdorff space, all
compactly supported continuous functions are integrable: $\int \abs{f}
\, d\mu \leq \norm{f}_\infty \mu(supp(f)) < \infty$.  Thus such a
measure yields a linear functional on $C_c(X)$.  Such functionals
share another simple property.
\begin{defn}A linear functional $\Lambda$ on $C_c(X)$ is said to be
  \emph{positive} if $f \geq 0$ implies $\Lambda(f) \geq 0$.
\end{defn}

It is clear that the linear functional defined by integration with
respect to a Radon measure is positive.  The Riesz-Markov Theorem
tells us that the positive linear functionals are precisely those
generated by integration with respect to a Radon measure.  To prove
the result we will need to figure out how to define a measure from a
positive linear functional.  As a warm up let's first answer that
question in the case of integration with respect to a Radon measure.

\begin{lem}\label{RieszMarkovUniquenessOnOpen}Let $X$ be a locally compact Hausdorff space and let $\mu$
  be a Radon measure on $X$, then for every open set $U$ we have
\begin{align*}
\mu(U) &= \sup \lbrace \int f \, d\mu \mid 0 \leq f \leq 1, f \in
C_c(X), supp(f) \subset U \rbrace
\end{align*}
\end{lem}
\begin{proof}
For the inequality $\geq$, suppose that $f \in C_c(X)$ satisfies $0
\leq f \leq 1$ and $supp(f) \subset U$, then observe the hypotheses
imply that $f \leq \characteristic{U}$ so that
\begin{align*}
\int f(x) \, d\mu(x) &\leq \int \characteristic{U}(x) \, d\mu(x) = \mu(U)
\end{align*}
and the inequality follows by taking the supremum over all such $f$.

For the inequality $\leq$ we leverage the inner regularity of $\mu$.
Let $K \subset U$ be a compact set.  By Lemma
\ref{BoundedNeighborhoodsOfCompactSets} we find an relatively compact
open set $V$ with $K \subset V \subset \overline{V} \subset U$.  Since
$\overline{V}$ is a compact Hausdorff space, it is normal and we may
apply the Tietze Extension Theorem \ref{TietzeExtensionTheorem} to
find a continuous function $g : \overline{V} \to [0,1]$ such that $g
\equiv 1$ on $K$.  Applying Urysohn's Lemma \ref{UrysohnsLemma} we
construct a continuous function $h : X \to [0,1]$ such that $h = 1$ on
$K$ and $h=0$ on $V^c$.  We define
\begin{align*}
f(x) &= \begin{cases}
h(x) g(x) & \text{if $x \in \overline{V}$} \\
0 & \text{if $x \notin \overline{V}$}
\end{cases}
\end{align*}
By the corresponding properties of $g$ and $h$, it is clear that $0 \leq f \leq 1$ and that $f=1$ on $K$.  We claim that $f$ is continuous on all of $X$.  Since $h$ restricts to a continuous function
on $\overline{V}$ it is clear that the restriction of $f$ to
$\overline{V}$ is continuous.  Let $O \subset \reals$ be an open
set.  If $0 \notin O$ then it follows that $f^{-1}(O) \subset V$ and
is therefore open by the continuity of $f$ restricted to $V$.  If on
the other hand, $0 \in O$ then $f^{-1}(O) \cap \overline{V}$ is open
in $\overline{V}$ hence is of the form $Z \cap \overline{V}$ for some
open subset $Z \subset X$.  Because $0 \in O$ if follows that $Z
\subset f^{-1}(O)$ and therefore by the definition of $f$ we we may
write $f^{-1}(O) = Z \cup \overline{V}^c$ which is an open set.

TODO: This is a locally compact Hausdorff version of Tietze, factor it
out into a separate result.

With the extension $f$ in hand we see that 
\begin{align*}
\mu(K) &\leq \int f(x) \, d\mu(x) \leq \sup \lbrace \int f \, d\mu \mid 0 \leq f \leq 1, f \in
C_c(X), supp(f) \subset U \rbrace
\end{align*}
Now taking the supremum over all compact subsets $K \subset U$ and
using the inner regularity of $\mu$ the result follows.
\end{proof}


Before we state an prove the Riesz-Markov theorem we need the
existence of finite partitions of unity on compact sets in an LCH: a standard
bit of general topology.

\begin{lem}\label{PartitionOfUnity}Let $X$ be a locally compact Hausdorff space, $K$ be a
  compact subset of $X$ and $\lbrace U_\alpha \rbrace$ an open
  covering of $K$.  There exists a finite subset $\alpha_1, \dotsc,
  \alpha_n$ and continuous functions with compact support $f_{\alpha_1}, \dotsc,
  f_{\alpha_n}$ such that $supp(f_{\alpha_j}) \subset U_{\alpha_j}$
  and $f_{\alpha_1} + \dotsb + f_{\alpha_n} = 1$ on $K$.
\end{lem}
\begin{proof}
Pick an $x \in K$, pick an $U_{\alpha_x}$ such that $x \in U_{\alpha_x}$ and
using complete regularity of $X$, construct a continuous function
$g_x$ from $X$ to $[0,1]$ such that $g_x(x) = 1$ and
$g_x \equiv 0$ on $U_{\alpha_x}^c$.  Thus $g_x^{-1}(0,1] \subset
U_{\alpha_x}$ and the $g_x^{-1}(0,1]$ form an open cover of $K$.  By
compactness of $K$ we extract a finite subcover $g_{x_1}^{-1}(0,1],
\dotsc, g_{x_n}^{-1}(0,1]$.  If we clean up notation by denoting $U_{\alpha_{x_j}} =
U_{\alpha_j}$ and $g_{\alpha_j} = g_{\alpha_{x_j}}$, it follows that
$U_{\alpha_1}, \dotsc, U_{\alpha_n}$ is an open cover of $K$ and $g =
\sum_{j=1}^n g_{\alpha_j}$ is strictly positive on $K$.  Moreover by
compactness of $K$ we know that $g$ has a minimum value $C > 0$ on
$K$.  Define $h = g \vee C$ so that $h$ is continuous, $h = g$ on $K$ and $h
\geq C > 0$ everywhere on $X$.  By continuity and strict positivity of
$h$ we can define $f_{\alpha_j} = g_{\alpha_j}/h$ so that
$f_{\alpha_j}$ is continuous and moreover $f_{\alpha_1} + \dotsb +
f_{\alpha_n} = g/h = 1$ on $K$.
\end{proof}

\begin{thm}[Riesz-Markov Theorem]\label{RieszMarkov}Let $X$ be a
  locally compact Hausdorff space and let $\Lambda : C_c(X) \to
  \reals$ be a positive linear functional then there exists a unique
  Radon measure $\mu$ such that $\Lambda(f) = \int f \, d\mu$ for all
  $f \in C_c(X)$.
\end{thm}
\begin{proof}
The uniqueness part of the result is straightforward.  By Lemma
\ref{RieszMarkovUniquenessOnOpen} we know that the values of $\mu$ on open sets
are determined by $\Lambda$.  By Lemma
\ref{InnerOuterRegularityEquivalence} we conclude that the values of
$\mu$ on $\sigma$-bounded Borel sets are determined by $\Lambda$, in
particular the values on compact sets are determined by $\Lambda$.
The inner regularity of $\mu$ implies that the values on all Borel
sets are determined by $\Lambda$.

For existence we follow the lead of Lemma
\ref{RieszMarkovUniquenessOnOpen} and define the set function on the
open sets of $X$
\begin{align*}
m(U) &=  \sup \lbrace \Lambda( f ) \mid 0 \leq f \leq 1, f \in C_c(X), supp(f) \subset U \rbrace
\end{align*}
We proceed by showing that $m(U)$ satisfies properties (i) through (v)
from Lemma \ref{ExtensionToRadonMeasure} and that if $\mu$ is the
Radon measure constructed by that result that we indeed have
$\Lambda(f) = \int f \, d\mu$.

\begin{clm}$m$ satisfies (i)
\end{clm}

Let $U$ be a relatively compact open set.  By Lemma
\ref{BoundedNeighborhoodsOfCompactSets} we can find another relatively
compact open set $V$ such that $\overline{U} \subset V$.  By the
Tietze Extension Theorem argument of Lemma \ref{ExtensionToRadonMeasure} we can find a continuous function $g: X \to
[0,1]$ such that $g = 1$ on $\overline{U}$ and $g = 0$ on $V^c$.
Since $g \in C_c(X)$ we have $\Lambda(g) < \infty$.  Now suppose that
$f \in C_c(X)$ satisfies $0 \leq f \leq 1$ and  $f \in C_c(X), supp(f)
\subset U$.  It follows that $0 \leq f \leq g$ and linearity and
positivity of $\Lambda$ we know that $\Lambda(f) \leq \Lambda(g)$.
Taking the supremum over all such $f$ we get
\begin{align*}
m(U) &= \sup \lbrace \Lambda( f ) \mid 0 \leq f \leq 1, f \in C_c(X),
supp(f) \subset U \rbrace \leq \Lambda(g) < \infty
\end{align*}

\begin{clm}$m$ satsifies (ii)
\end{clm}

This is immediate since $U \subset V$ implies 
\begin{align*}
\lbrace \Lambda( f ) \mid 0 \leq f \leq 1, f \in C_c(X),
supp(f) \subset U \rbrace \subset \lbrace \Lambda( f ) \mid 0 \leq f \leq 1, f \in C_c(X),
supp(f) \subset V \rbrace
\end{align*}

\begin{clm}$m$ satisfies (iii)
\end{clm}

Let $U_1, U_2, \dotsc$ be open sets and let $f \in C_c(X)$ satisfy $0
\leq f \leq 1$ and $supp(f) \subset \cup_{n=1}^\infty U_n$.  By
compactness of $supp(f)$ and Lemma
\ref{PartitionOfUnity} we may find an $N$ and continuous functions
$g_i$ for $i=1, \dotsc, N$ such that $0 \leq g_i \leq 1$, $supp(g_i) \subset U_i$ and
$\sum_{i=1}^N g_i = 1$ on $supp(f)$ and therefore $f = f \cdot
\sum_{i=1}^N g_i$.  It also follows that $0 \leq f g_i
\leq 1$ and $supp(f g_i) \subset U_i$ for $i=1, \dotsc, N$ and thus
\begin{align*}
\Lambda(f) &= \sum_{i=1}^N \Lambda( f g_i) \leq \sum_{i=1}^N m(U_i)
\leq \sum_{i=1}^\infty m(U_i)
\end{align*}
Now we take the supremum over all such $f$ to conclude that
$m(\cup_{i=1}^\infty U_i) \leq \sum_{i=1}^\infty m(U_i)$.

\begin{clm}$m$ satisfies (iv)
\end{clm}

Suppose $U$ and $V$ are disjoint open sets.  We only need to show that
$m(U \cup V) \geq m(U) + m(V)$ since the opposite inequality follows
from (iii).  Let $f, g \in C_c(X)$ such that $0 \leq f,g \leq 1$,
$supp(f) \subset U$ and $supp(g) \subset V$.  By disjointness of $U$
and $V$ it follows that $f+g \in C_c(X)$, $0 \leq f+g \leq 1$ and
$supp(f+g) \subset supp(f) \cup supp(g) \subset U \cup V$.  Therefore
\begin{align*}
\Lambda(f) + \Lambda(g) &= \Lambda(f+g) \leq m(U \cup V)
\end{align*}
Now take the supremum over all $f$ and $g$ to get the result.

\begin{clm}$m$ satisfies (v)
\end{clm}

Let $U$ be an open set and let $f \in C_c(X)$ such that $0 \leq f \leq
1$ and $supp(f) \subset U$.  By compactness of $supp(f)$ and Lemma
\ref{BoundedNeighborhoodsOfCompactSets} we may find a relatively
compact open set $V$ such that 
$supp(f) \subset V \subset \overline{V} \subset U$.  
It follows that 
\begin{align*}
\Lambda(f) &\leq m(V) \leq \sup \lbrace m(V) \mid V \text{ is open, } \overline{V} \subset U \text{ and }
  \overline{V} \text{ is compact} \rbrace
\end{align*}
Now take the supremum over all such $f$.

We may now apply Lemma \ref{ExtensionToRadonMeasure} to construct a
Radon measure $\mu$ such that $\mu(U) = m(U)$.  We need to show that
for every $f \in C_c(X)$ we have $\Lambda(f) = \int f \, d\mu$. By
linearity we know that $\Lambda(0) = \int 0 \, d\mu = 0$ so we may
assume that $f \neq 0$.  We may write $f = f_+
- f_-$ with $f_+ = f \vee 0 \in C_c(X)$ and $f_- = (-f) \vee 0 \in
C_c(X)$.  Since both $\Lambda$ and the integral are linear it suffices
to show the result for $f \geq 0$.  Since $f$ is continuous with
compact support, it follows that $f$ is bounded and since $f \neq 0$
we have $0 < \norm{f}_\infty < \infty$.  Again, by linearity of
$\Lambda$ and integration it suffices to prove the result of
$f/\norm{f}_\infty$ and thus we may assume that $0 \leq f \leq 1$.

We proceed by constructing a generalized upper and lower sum
approximation to the integral of $f$.  Once again apply Lemma
\ref{BoundedNeighborhoodsOfCompactSets} to find a relatively compact
open neighborhood $U_0$ with $supp(f) \subset U_0$.  Let $\epsilon >
0$ be given and choose $n \in \naturals$ such that $\mu(U_0) <
\epsilon n$.  For $j=1, \dotsc, n$ define $U_j = f^{-1} (j/n,
\infty)$.  Because $f$ is continuous and of compact support, each
$U_j$ is a relatively compact open set and it is trivial from the
definitions that we have $\emptyset = U_n \subset U_{n-1} \subset \dotsb \subset
U_0$.  In fact by the continuity of $f$ it is also true that $\overline{U}_j \subset U_{j+1}$.
Define the lower and upper approximations to $f$
\begin{align*}
u(x) &= \begin{cases}
\frac{j}{n} & \text{if $x \in U_j \setminus U_{j+1}$ for some $j=1, \dotsc,  n-1$} \\
0 & \text{if $x \notin U_1$}
\end{cases}
\end{align*}
 and similarly 
\begin{align*}
v(x) &= \begin{cases}
\frac{j}{n} & \text{if $x \in U_{j-1} \setminus U_{j}$ for some
  $j=1,  \dotsc,  n$} \\
0 & \text{if $x \notin U_0$}
\end{cases}
\end{align*}
Note that we have the property that $u \leq f \leq v$ and moreover we
have the useful alternative defintion of $u$ and $v$
\begin{align*}
u(x) &= \frac{1}{n} \sum_{j=1}^n \characteristic{U_j}(x) \\
v(x) &= \frac{1}{n} \sum_{j=1}^{n} \characteristic{U_{j-1}}(x) \\
\end{align*}
which shows that $u,v \in C_c(X)$.

\begin{clm}$\int (v -u) \, d\mu < \epsilon$
\end{clm}

This follows by observing that $v - u = \frac{1}{n}
(\characteristic{U_0} - \characteristic{U_n}) = \frac{1}{n}
\characteristic{U_0}$ since $U_n = \emptyset$.

\begin{clm}$\int u \, d\mu \leq \Lambda(f) \leq \int v \, d\mu + \epsilon$
\end{clm}

To see this claim we decompose $f$ into a representation that is
adapted to the nested sequence $U_n \subset \dotsb \subset U_0$.  For
$j=1, \dotsc, n$ define 
\begin{align*}
\phi_j(x) &= \begin{cases}
1/n & \text{if $x \in U_j$} \\
f(x) - \frac{j-1}{n} & \text{if $x \in U_{j-1} \setminus U_{j}$} \\
0 & \text{if $x \notin U_{j-1}$}
\end{cases} \\
&=[(f(x) - \frac{j-1}{n}) \vee 0] \wedge \frac{1}{n}\\
\end{align*}
where the second representation shows that $\phi_j \in C_c(X)$ and $0 \leq \phi_j \leq \frac{1}{n}$.
Note also
that if we are given $x \in U_{j-1} \setminus U_j$ for some $j=1,
\dotsc, n$ then $\phi_i(x) = 0$ for $j < i \leq n$ and $\phi_i(x) =
\frac{1}{n}$ for $1 \leq i < j$.  Therefore we have 
\begin{align*}
\phi_1(x) + \dotsb + \phi_n(x) &= \phi_1(x) + \dotsb + \phi_j(x) \\
&= \frac{j-1}{n} + f(x) - \frac{j-1}{n} = f(x)
\end{align*}
It is clear that for $x \notin U_0$ we have $f(x) = 0$ and
$\phi_j(x) = 0$ for all $j=1, \dotsc, n$ so we have
 $f = \phi_1 + \dotsb + \phi_n$ on all of $X$.

We now need to bound $\Lambda(\phi_j)$ in terms of $\mu(U_k) = m(U_k)$ for
suitable $k=1, \dotsc, n$.  First we get a lower bound on
$\Lambda(\phi_j)$.  Let $1 \leq j \leq n$ be given.  Suppose that we have a $g \in C_c(X)$ with
$0 \leq g \leq 1$ and $supp(g) \subset U_j$.  Then by positivity of
$\phi_j$ and the fact that $\phi_j(x) = \frac{1}{n}$ on $U_j$ we see
that $g \leq \characteristic{U_j} \leq n \phi_j$ and therefore
$\Lambda(g) \leq n \Lambda(\phi_j)$.    Taking the supremum over all
such $g$ we see that $\frac{1}{n} \mu(U_j) \leq \Lambda(\phi_j)$.  Taking the
sum over all $j=1, \dotsc, n$ and using linearity of $\Lambda$ we get
\begin{align*}
\int u \, d\mu &= \frac{1}{n} \sum_{j=1}^n \mu(U_j) \leq \sum_{j=1}^n
\Lambda(\phi_j) = \Lambda(f)
\end{align*}

Now we get an upper bound on $\Lambda(\phi_j)$.   For $j=2, \dotsc, n$ we that
$n phi_j \in C_c(X)$, $0 \leq n \phi_j \leq 1$ and $supp(n \phi_j)
\subset \overline{U_{j-1}} \subset U_{j-2}$.  From the definition of
$\mu(U_{j-2}) =m(U_{j-2})$ it follows that $\Lambda(\phi_j) \leq
\frac{1}{n} \mu(U_{j-2})$.  As for $\phi_1$, we have
$n \phi_1 \in C_c(X)$ and $0 \leq n\phi_1 \leq 1$ by exactly the same
argument as for $j \geq 2$.  We also have $supp(n \phi_1) \subset
supp(f) \subset U_0$ so that $\Lambda(\phi_1) \leq \frac{1}{n}
\mu(U_0)$.  If we define $U_{-1} = U_0$ the we get
have $\Lambda(\phi_j) \leq \frac{1}{n} \Lambda(U_{j-2})$ for $j=1,
\dotsc, n$.  Again we sum and use linearity of $\Lambda$,
\begin{align*}
\Lambda(f) &= \sum_{j=1}^n \Lambda(\phi_j) \leq \frac{1}{n}
\sum_{j=1}^n \mu(U_{j-2}) = \frac{1}{n} \mu(U_{-1}) +  \frac{1}{n}
\sum_{j=1}^{n-1} \mu(U_{j-1}) \\
&\leq \frac{1}{n} \mu(U_{0}) +  \frac{1}{n}
\sum_{j=1}^{n} \mu(U_{j-1}) \leq \epsilon + \int v \, d\mu
\end{align*}

It remains to stitch together the previous claims to
show that $\Lambda(f) = \int f \, d\mu$.  Integrating the inequality
$u \leq f \leq v$ we get $\int u \, d\mu \leq f \leq \int v \, d\mu$.
Now using this fact and previous two claims we get
\begin{align*}
\Lambda(f) - \int f \, d\mu &\leq \Lambda(f) - \int v \, d\mu \leq
\int u \, d\mu  - \int u \, d\mu + \epsilon \leq 2 \epsilon
\end{align*}
and
\begin{align*}
\Lambda(f) - \int f \, d\mu &\geq \int u \, d\mu - \int f \, d\mu \geq
\int u \, d\mu - \int v \, d\mu \geq -\epsilon
\end{align*}
from which we conclude that $\abs{\Lambda(f) - \int f \, d\mu} \leq
2\epsilon$.  Since $\epsilon > 0$ was arbitrary we are done.
\end{proof}

\begin{defn}Let $\mu$ be a measure on the Borel $\sigma$-algebra of a
Hausdorff topological space $S$.  
\begin{itemize}
\item[(i)] A Borel set $B$ is \emph{inner regular} if for
 $\mu(B) = \sup_{K \subset B} \mu(K)$ where $K$
  is compact. $\mu$ is inner regular if every Borel set is inner regular.
\item[(ii)]A Borel set $B$ is \emph{outer regular} if $\mu(B) = \inf_{U \supset B} \mu(U)$ where $U$
  is open.  A measure $\mu$ is outer regular if every Borel set
  $B$ is outer regular.
\item[(iii)] $\mu$ is \emph{locally finite} if every $x \in S$ has an
  open neighborhood $x \in U$ such that $\mu(U) < \infty$.
\item[(iv)] $\mu$ is a \emph{Radon measure} it is inner regular and
  locally finite.
\item[(v)] $\mu$ is a \emph{Borel measure} when?????  In some cases
  I've seen it required that $\mu(B) < \infty$ for all Borel sets $B$
  (reference?) and in other cases just that the Borel sets are measurable.
\item[(vi)]A Borel set  $B$ is \emph{closed regular} if $\mu(B) = \inf_{F \subset B} \mu(F)$ where $F$
  is closed (e.g. Dudley pg. 224).  A measure $\mu$ is closed regular
  if every Borel set $B$ is closed regular.
\item[(vii)] If $\mu$ is finite, then we say \emph{tight} if and only if
  X is inner regular (e.g. Dudley pg. 224).
\end{itemize}
\end{defn}

\begin{prop}Let $\mu$ be a measure on the Borel $\sigma$-algebra of a
  locally compact Hausdorff space $S$.  Then $\mu$ is locally finite
  if and only if $\mu(K) < \infty$ for all compact sets $K \subset S$.
\end{prop}
\begin{proof}
If $\mu(K) < \infty$ for all compact sets $K$ we let $x \in S$ and
pick a relatively compact neighborhood $U$ of $x$.  Then $\mu(U) \leq
\mu(\overline{U}) < \infty$ which shows $\mu$ is locally finite.  On
the other hand, suppose $\mu$ is locally finite and let $K$ be a
compact set.  For each $x \in K$ we take an open neighborhood $U_x$
such that $\mu(U_x) < \infty$ and then extract a finite subcover $U_{x_1},
\dotsc,U_{x_n}$.  By subadditivity, we have $\mu(K) \leq \mu(U_{x_1}) + \dotsb
+ \mu(U_{x_n}) < \infty$.
\end{proof}

\begin{defn}Let $\mu$ be a Borel measure on a Hausdorff topological space. A set measurable set $A$ is called \emph{regular} if 
\begin{itemize}
\item[(i)]$\mu(A) = \inf_{U \supset A} \mu(A)$ where $U$ are open
\item[(ii)]$\mu(A) = \sup_{F \subset A} \mu(A)$ where $F$ are closed 
\end{itemize}
TODO: Alternative def assumes that $F$ are compact (see inner
regularity above).  If every measurable set is regular then $\mu$ is
said to be regular.  Note that if we assume the definition of
regularity uses compact inner approximations then regular measures are
inner and outer regular (although inner and outer regularity refer to
only Borel sets; is that a meaningful distinction?)  I think this use
of closed inner regularity is a bit non-standard should probably get
rid of it.
\end{defn}


TODO: Regularity of outer measures and the relationship to regularity
of measures as defined above (see Evans and Gariepy).  Note that
regularity of outer measure implies that if we take an outer measure $\mu$
and the measure on the $\mu$-measurable sets and then take the induced
outer measure we get $\mu$ back if and only $\mu$ is a regular outer
measure.  Evans and Gariepy show that Radon outer measures on
$\reals^n$ are inner
regular as measures on the $\mu$-measurable sets (I think we prove this more
generally above in the context of LCH spaces; note that every set in
$\reals^n$ is $\sigma$-bounded).  Note that inner
regular is part of the most common definition of Radon measure so
their result can be taken as showing a weaker definition of Radon
measure holds on $\reals^n$ (but also they phrase everything in terms
of outer measures...).

TODO: How much this stuff on regularity can be extended to outer
measures????  I want to understand the overlap with the results in
Evans and Gariepy.

\begin{lem}\label{InnerRegularSetsSigmaAlgebra}Let $X$ be a Hausdorff topological space, $\mathcal{A}$
  a $\sigma$-algebra on $X$ and $\mu$ a finite tight measure.  Then
\begin{align*}
\mathcal{R} &= \lbrace A \in \mathcal{A} \mid A \text { and } A^c
\text{ are $\mu$-inner regular} \rbrace
\end{align*}
is a $\sigma$-algebra.  The same is true if the condition is replaced
by sets that are $\mu$-closed inner regular (without the requirement
that $\mu$ is tight).
\end{lem}
\begin{proof}
By definition, $\mathcal{R}$ is closed under complement.  By
assumption that $\mu$ is tight we have $X \in \mathcal{R}$ so all that
needs to be shown is closure under countable union.

Assume $A_1, A_2, \dots \in \mathcal{R}$ and let $\epsilon>0$ be
given.  By finiteness of $\mu$, $\mu(\cup_{n=1}^\infty A_n) < \infty$ and
continuity of measure (Lemma \ref{ContinuityOfMeasure}) there exists $M>0$ such that $\mu(\cup_{n=1}^M
A_n) > \mu(\cup_{n=1}^\infty A_n) - \epsilon$.
 By assumption that $A_n \in \mathcal{R}$ and finiteness of $\mu$, for each
$A_n$ there exists a compact $K_n$ such that $\mu(A_n \setminus K_n) <
\frac{\epsilon}{2^n}$ and there exists compact $L_n$ such that $\mu(A_n^c \setminus L_n) <
\frac{\epsilon}{2^n}$. Let
\begin{align*}
K &= \cup_{n=1}^M K_n \\
L &= \cap_{n=1}^\infty L_n
\end{align*}
and note that both $K$ and $L$ are compact (in the latter case,
because X is Hausdorff we know that each $L$ is closed hence the
intersection is a closed subset of a compact set hence compact).
Furthermore we can compute
\begin{align*}
\mu(\cup_{n=1}^\infty A_n \setminus K) &= \mu(\cup_{n=1}^\infty A_n
\setminus \cup_{n=1}^M K_n)  \\
&= \mu(\cup_{n=1}^M A_n
\setminus \cup_{n=1}^M K_n)  + \mu(\cup_{n=1}^\infty A_n \setminus \cup_{n=1}^M A_n
\setminus \cup_{n=1}^M K_n)\\
&\leq \mu(\cup_{n=1}^M A_n \setminus K_n)  + \mu(\cup_{n=1}^\infty A_n
\setminus \cup_{n=1}^M A_n)\\
&\leq \sum_{n=1}^M(A_n \setminus K_n)  + \epsilon \\
&\leq 3 \epsilon
\end{align*}
and
\begin{align*}
\mu((\cup_{n=1}^\infty A_n)^c \setminus L) &=\mu(\cap_{n=1}^\infty
A_n^c \setminus \cap_{n=1}^\infty L_n) \\
 &=\mu(\cap_{n=1}^\infty
A_n^c \cap \cup_{n=1}^\infty L_n^c) \\
 &=\mu(\cup_{n=1}^\infty \cap_{m=1}^\infty
A_m^c \cap L_n^c) \\
 &\leq \mu(\cup_{n=1}^\infty 
A_n^c \cap L_n^c) \\
 &\leq \sum_{n=1}^\infty (
A_n^c \setminus L_n) \\
&\leq 2 \epsilon
\end{align*}

TODO: The closed inner regular case...
\end{proof}

TODO:  In metric space, tightness is equivalent to inner regularity.
Then Ulam's Theorem that finite measures on separable metric spaces
are automatically inner regular.  Also finite measures on arbitrary
metric spaces are closed inner regular as well as outer regular.

\begin{lem}\label{FiniteMeasuresOnMetricSpacesAreClosedInnerRegular}
Let $(S,d)$ be a metric space and $\mu$ be a Borel measure on $(S,
\mathcal{B}(S))$, then $\mu$ is closed inner regular.  If in addition
$\mu$ is a finite measure then it is outer regular.
\end{lem}
\begin{proof}
Let $U$ be an open set in $S$.  Then $U^c$ is closed and the
function $f(x) = d(x, U^c)$ is continuous.  If we define 
\begin{align*}
F_n &= f^{-1}([1/n, \infty))
\end{align*}
then each $F_n$ is closed, $F_1 \subset F_2 \subset \cdots$ and
$\cup_{n=1}^\infty F_n = U$.  By continuity of measure (Lemma
\ref{ContinuityOfMeasure}) we know that $\lim_{n \to \infty} \mu(F_n)
= \mu(U)$.  So this shows that every open set is inner closed
regular.  Furthermore it is trivial to note that $U^c$ is inner closed
regular because it is closed.  

By Lemma \ref{InnerRegularSetsSigmaAlgebra} we know know that 
\begin{align*}
\mathcal{B}(S) &\subset \mathcal{R} = \lbrace A \subset S \mid A
\text{ and } A^c
\text{ are inner closed regular}  \rbrace
\end{align*}

Outer regularity follows from taking complements and using the
finiteness of $\mu$.
\end{proof}

If we add the criterion that the metric space is separable, then we
can upgrade the closed inner regularity to inner regularity.
\begin{lem}\label{SeparableInnerRegularTight}Let $(S,d)$ be a
  separable metric space and $\mu$ be a finite Borel measure on $(S,
\mathcal{B}(S))$, then $\mu$ is inner regular if and only if it is tight.
\end{lem}
\begin{proof}
Clearly inner regularity implies tightness (which is just inner
regularity of the set $S$), so it suffices to show
that tightness implies inner regularity.

Suppose that $\mu$ is a tight measure.  By Lemma
\ref{InnerRegularSetsSigmaAlgebra} it suffices to show that both open
and closed sets are inner regular.

Pick $\epsilon >0$ and select $K \subset S$ a compact set such that $\mu(S \setminus K) < \frac{\epsilon}{2}$.
By Lemma \ref{FiniteMeasuresOnMetricSpacesAreClosedInnerRegular} we
know that for any Borel set $B$ there exists a closed set $F \subset
B$ such that $\mu(B \setminus F) < \frac{\epsilon}{2}$.  Note that $F
\cap K$ is compact.   We have
\begin{align*}
\mu(B \setminus (F \cap K)) &\leq \mu(B \cap F^c) + \mu(B \cap K^c) \leq \mu(B \cap F^c) + \mu(S \cap K^c) < \epsilon
\end{align*}
\end{proof}
\begin{thm}[Ulam's Theorem]\label{UlamsTheorem}Let $(S,d)$ be a
  complete separable metric space and $\mu$ be a finite Borel measure on $(S,
\mathcal{B}(S))$, then $\mu$ is inner regular.
\end{thm}
\begin{proof}
By Lemma \ref{SeparableInnerRegularTight} it suffices to show that $\mu$ is tight.  Pick
$\epsilon > 0$ and we construct a compact set $K \subset S$ such that
$\mu(S \setminus K) < \epsilon$.  Let
$\overline{B}(x,r)$ denote the closed ball of radius $r$ around $x \in
S$.  Pick
a countable dense subset $x_1, x_2, \dotsc \in S$.  For each $m \in
\naturals$, by density of $\lbrace x_n \rbrace$, we know $\cap_{n=1}^\infty \left ( S
\setminus \cup_{j=1}^n \overline{B}(x_j, \frac{1}{m}) \right ) =
\emptyset$, thus by
continuity of measure (Lemma \ref{ContinuityOfMeasure}) there exists
$N_m > 0$ such that $\mu(S
\setminus \cup_{j=1}^n \overline{B}(x_j, \frac{1}{m}) < \frac{\epsilon}{2^m}$ for
all $n \geq N_m$.
If we define
\begin{align*}
K &= \cap_{m=1}^\infty \cup_{j=1}^{N_m} \overline{B}(x_j, \frac{1}{m})
\end{align*}
we claim that $K$ is compact.  Note that $K$ is easily seen to be
closed as it is an intersection of a finite union of closed balls.
Since $S$ is complete this implies that $K$ is also complete.  Also it
is easy to see that $K$ is totally bounded since by construction we
have demonstrated a cover by a finite number of balls of radius
$\frac{1}{m}$ for each $m \in \naturals$.  So by Theorem
\ref{CompactnessInMetricSpaces} we know $K$ is compact.

To finish the result we claim $\mu(S \setminus K) < \epsilon$:
\begin{align*}
\mu(S \setminus K) 
&= \mu(S \cap \left(\cap_{m=1}^\infty
  \cup_{j=1}^{N_m} \overline{B}(x_j, \frac{1}{m})\right)^c) \\
&= \mu(S \cap \cup_{m=1}^\infty \left(
  \cup_{j=1}^{N_m} \overline{B}(x_j, \frac{1}{m})\right)^c) \\
&= \mu(\cup_{m=1}^\infty S \setminus 
  \cup_{j=1}^{N_m} \overline{B}(x_j, \frac{1}{m})) \\
&\leq \sum_{m=1}^\infty \mu( S \setminus 
  \cup_{j=1}^{N_m} \overline{B}(x_j, \frac{1}{m})) \\
&< \epsilon
\end{align*}
\end{proof}

\begin{thm}Let $\mu$ be a finite Borel measure on a metric space $S$,
  then $\mu$ is closed regular.  If $\mu$ is tight then $\mu$ is regular.
\end{thm}
TODO: Specialize the definition of Radon measure in the presence of
more assumptions on $X$ (in particular local compactness,
$\sigma$-compactness, second countability).

TODO: Are Radon measures automatically outer regular?

Tao proves Riesz representation under assumption of local compactness
and $\sigma$-compactness.

Kallenberg proves Riesz representation under assumption of LCH and second countability (this is more general than the Tao
result as $\sigma$-compactness implies second countability (I think))
and targets Radon measures.  Our results taken from Arveson are more
general as the remove the second countability assumption.

Evans and Gareipy prove Riesz representation only on $\reals^n$ using
Radon outer measures.  This is probably subsumed by our results taken
from Arveson but I need to understand whether the use of outer
measures adds anything to the picture.

Arveson has some well known lecture notes that prove Riesz on general
LCH spaces
and emphasizes Radon measures (it also explores how Baire measures figure in the
picture).  I have chosen to follow these notes.

Fremlin probably has some very general account of Reisz representation
(of course).

Dudley proves Riesz representation of compact Hausdorff spaces and
phrases things in terms of Baire measures.  Dudley does not really
discuss Radon measures.  Arveson discusses the relationship between
the use of Baire and Radon measures.

\section{Covering Theorems in $\reals^n$}

Since our purposes have been to understand probability theory we have
hitherto avoided making assumptions that we are dealing with
$\reals^n$.  While this decision has benefits, it has drawbacks as
well.  Among those drawbacks are that we lose sight of some history and also some very
beautiful and deep understanding of the measure theory of the reals.
TODO: Vitali and Besicovich.

\section{Hausdorff Measure}

\subsection{Introduction}

In this section we discuss the construction of a family of outer
measures on $\reals^n$ called \emph{Hausdorff measures}.  Note the
construction can be generalized to metric spaces.  The following is
motivation why a tool like Hausdorff measure may be useful.  Suppose
very specifically that we are
in $\reals^3$, then the Lebesgue product measure essentially
corresponds to a notion of volume.  What about the surface area of a
$2$-dimensional object or the length of a $1$-dimensional object?  As
you may have learned in advanced calculus these ideas can indeed be
describe in great generality by the notion of differential forms.
However, the formalism of forms usually has some notion of smoothness
associated with it (hence the adjective differential); a natural question to ask is whether one can fine
a purely measure theoretic approach to the problem.  Hausdorff measures
provide one answer to this question.   The broad form of the theory
is perhaps a bit more general than one might expect; for any space
there is a Hausdorff outer measure for every real number $s$.  The
case of integers
$s=1$ corresponds to arclength, $s=2$ surface area, $s=3$ volume and so
on.  Measures with $s$ non-integral are
\emph{fractal}.  On $\reals^n$, the Hausdorff measure with $s=n$ is equal to
Lebesgue measure and any Hausdorff measure with $s > n$ is trivial
(gives $0$ measure to all sets).  We'll prove all of this and more in
what follows.

\subsection{Construction of Hausdorff Measure}

The following technical Lemma is useful (we'll use it when
discussing Hausdorff outer measures).  If the reader is in a hurry,
no harm will come from skipping over this result and returning to it
when the need arises.  Note that if the user is only interested in
probability theory this result may never come up.
\begin{lem}[Caratheodory Criterion]\label{CaratheodoryCriterion}Let $(S,d)$ be a metric space with an outer measure $\mu^*$.
  Then $\mu^*$ is a Borel outer measure (i.e. all Borel sets are
  $\mu^*$-measurable) if and only if $\mu^*(A \cup B) = \mu^*(A) +
  \mu^*(B)$ for all $A,B$ such that $d(A, B) > 0$.
\end{lem}
\begin{proof}
We begin with the only if direction.  Let $A$ be a closed set in $S$
and let $B \subset S$.  To show $A$ is $\mu^*$-measurable it suffices
to show $\mu^*(B) \geq \mu^*(A \cap B) + \mu^*(A^c \cap B)$.  Since
the inequality is trivially satisfied when $\mu^*(B) = \infty$ we
assume that $\mu^*(B) < \infty$.  For
every $n \in \naturals$, let $A_n =
\lbrace x \in S \mid d(x, A) \leq \frac{1}{n} \rbrace$.  By definition
of $A_n$, we have $d(A,
A_n^c) > \frac{1}{n} > 0$ and therefore $d(A \cap B, A_n^c \cap B)
> \frac{1}{n} > 0$.  Now by our assumption, we can conclude $\mu^*((A
\cap B) \cup (A_n^c \cap B)) = \mu^*(A \cap B) +
\mu^*(A_n^c \cap B)$.

We claim that $\lim_{n \to \infty} \mu^*(A_n^c \cap B) = \mu^*(A^c
\cap B)$.  Note that if we prove the claim the Lemma is proven because then we have
\begin{align*}
\mu^*(B) &\geq \mu^*((A
\cap B) \cup (A_n^c \cap B)) & & \text{by monotonicity}\\
&= \mu^*(A \cap B) +
\mu^*(A_n^c \cap B)
\end{align*}
and taking limits we have 
\begin{align*}
\mu^*(B) \geq \lim_{n\to \infty} \mu^*(A \cap B) +
\mu^*(A_n^c \cap B) &= \mu^*(A \cap B) +
\mu^*(A^c \cap B)
\end{align*}
To prove the claim we observe that monotonicity of outer measure
implies that $\lim_{n \to \infty} \mu^*(A_n^c \cap B) \leq \mu^*(A^c
\cap B)$ so we just need to
work on the opposite inequality.  To see it first define the rings
around $A$
\begin{align*}
R_n &= \lbrace x \mid \frac{1}{n+1} < d(x, A) \leq \frac{1}{n} \rbrace
\end{align*}
and note that because $A$ is closed, for each $n$,
\begin{align*}
A^c &= \lbrace x \in S \mid d(x, A) > 0 \rbrace \\
&=\lbrace x \in S \mid d(x, A) > n \rbrace \cup \bigcup_{m=n}^\infty \lbrace
x \in S \mid \frac{1}{m+1} < d(x, A) \leq \frac{1}{m} \rbrace \\
&=A_n^c \cup \bigcup_{m=n}^\infty R_m
\end{align*}
It follows that
$A^c \cap B = A_n^c \cap B \cup \cup_{m=n}^\infty
R_m \cap B$ and therefore by subadditivity of outer measure 
\begin{align*}
\mu^*(A^c \cap B) \leq \mu^*(A_n^c \cap B) + \sum_{m=n}^\infty
\mu^*(R_m \cap B)
\end{align*}
The claim will follow if we can show $\lim_{n \to \infty} \sum_{m=n}^\infty
\mu^*(R_m \cap B)=0$ which in turn will follow if we can show that $\sum_{m=1}^\infty
\mu^*(R_m \cap B)$ converges.  By construction, $d(R_{2m}, R_{2n})
> 0$ and therefore $d(R_{2m} \cap B, R_{2n} \cap B)
> 0$ for any $m \neq n$.  So if we consider only the even terms of the
series we can use our hypothesis to show that for any $n$
\begin{align*}
\sum_{m=1}^n \mu^*(R_{2m} \cap B) &= \mu^*(\cup_{m=1}^n
R_{2m} \cap B) \leq \mu^*(B) < \infty
\end{align*}
and by taking limits $\sum_{m=1}^\infty \mu^*(R_{2m} \cap B) \leq \mu^*(B)$
The same argument applies to the odd indexed terms and we get
\begin{align*}
\sum_{m=1}^\infty \mu^*(R_{m} \cap B) &\leq 2\mu^*(B) < \infty
\end{align*}
The claim and the Lemma follow.
\end{proof}


TODO:  Here I am taking the path of Evans and Gariepy and normalizing
Hausdorff measure so that $\mathcal{H}^n = \lambda_n$.  I am not sure
if this winds up being inconvenient when one considers Hausdorff
measure in arbitrary metric spaces (nor do I know whether we'll bother
considering Hausdorff measures in metric spaces).

\begin{lem}Let $\lambda_n$ be Lebesgue measure on $\reals^n$, then
  $\lambda_n(B(0, 1)) = \frac{\pi^{n/2}}{\Gamma(\frac{n}{2} + 1)}$.
\end{lem}
\begin{proof}
TODO
\end{proof}

\begin{defn}Let $(S,d)$ be a metric space and $A \subset S$, the
  \emph{diameter} of $A$ is 
\begin{align*}
\diam(A) &= \sup \lbrace d(x,y) \mid x,y \in A \rbrace
\end{align*}
\end{defn}

\begin{defn}Let $(S,d)$ be a metric space, $0 \leq s < \infty$ and $0
  < \delta$.  Then for $A \subset S$,
\begin{align*}
\mathcal{H}^s_\delta(A) &= \inf \lbrace \sum_{n=1}^\infty \alpha(s)
\left ( \frac{\diam(C_n)}{2}\right )^s \mid A \subset
\cup_{n=1}^\infty C_n \text{ where } \diam(C_n) \leq \delta \text{ for
  all } n\rbrace
\end{align*}
where 
\begin{align*}
\alpha(s) &= \frac{\pi^{n/2}}{\Gamma(\frac{n}{2} + 1)}
\end{align*}
For $A$ and $s$ as above define
\begin{align*}
\mathcal{H}^s(A) &= \lim_{\delta \to 0} \mathcal{H}_\delta^s(A) = \sup_{\delta>0} \mathcal{H}_\delta^s(A)
\end{align*}
\end{defn}

\section{Integration in Banach Spaces}

Our prior development of measure and integration theory made use of the
order structure on the reals in various places and as a
result the theory does not hold for functions with values in arbitrary
vector spaces.  As we shall soon see it is useful to be able to
integrate functions with vector space values (in particular Banach
spaces) so we need an integration theory.  As it turns out there are a
couple of directions that one can go.  In the simplest case that will
suffice for many of our needs, we simply develop the theory of Riemann
integrals.  The primary loss of generality is
that the domains of functions in the Riemann integral case must be 
functions of a real variable.  For our purposes we shall only be
requiring the Riemann theory for a single real variable so that shall
suit us fine.  For problems in which the domain in an arbitrary
measurable space we need a Lebesgue-like theory that was developed by
Bochner.  The reader may want to be made aware that in addition to these integrals there is also an
integral due to Gelfand and Pettis that we shall not discuss.

\subsection{Riemann Integrals}
As mentioned we shall only bother to develop the Riemann integral for
a single real variable.
\begin{defn}Let $a \leq b$ be real numbers then a \emph{partition} of
  the interval $[a,b]$ is a finite sequence of real numbers $a=a_0
  \leq a_1 \leq \dotsb \leq a_n = b$.  Let $X$ be a Banach space then
  a map $f : [a,b] \to X$ is said to be a \emph{step map with repsect
    to P} if there
  exists a partition $P=\lbrace a_j \rbrace_{j=0}^n$ and elements $w_1,
  \dotsc, w_n \in X$ such that $f(t) = w_j$ for $a_{j-1} < t < a_j$.
  A \emph{step map} is any map $f$ such that for which there exists a partition
  $P$ for which $f$ is a step map with respect to $P$.
  The \emph{integral} of a step map with repsect to $P$ is
\begin{align*}
I_P(f) &= 
\sum_{j=1}^n (a_j - a_{j-1}) w_j
\end{align*}
\end{defn}
Note that a step map has it values constrained on the open intervals
$(a_{j-1},a_j)$ but not at the points $a_j$.

With all of these elementary definitions in hand we come to our first
task which is to show that the integral of a step map is well defined.
\begin{prop}Let $X$ be a Banach space and let $f : [a,b] \to X$ be a
  step map with respect to partitions $P$ and $Q$ then it follows that
  $I_P(f) = I_Q(f)$.
\end{prop}
\begin{proof}
Given a partition $P$ of the form $a=a_0 \leq \dotsb \leq a_n=b$ let
$c \in [a,b]$ and let the refinement $P_c$ represent the partition
obtained by adding $c$ to the set of $a_j$.  It is clear that $f$ is
still a step map with respect to $P_c$ and that $I_P(f)
= I_{P_c}(f)$.  A partition $R$ is said to be a refinement of $P$ if
it is a subset of $P$; by induction we see that $I_P(f) = I_R(f)$
whenever $R$ is a refinement of $P$.  Now given arbitrary partitions
$P$ and $Q$ as in the hypotheses we simply find a common refinement
(e.g. take the union of $P$ and $Q$) and the result follows.
\end{proof}

Now we extend the integral by a limiting procedure.  To do this we use 
somewhat abstract language of Banach space theory.  First let us set
up the Banach space in which we operate.

\begin{prop}Let $X$ be a normed vector space, let $S$ be an arbitrary
  set and let $\mathfrak{B}(S, X)$
  represent the set of bounded functions $f : S \to X$.  Let
  $\abs{x}$ denote the norm on $X$.  If we
  define $\norm{f} = \sup_{s \in S} \abs{f(s)}$ then $\norm{f}$
  makes $\mathfrak{B}(S, X)$ into a normed vector space.
\end{prop}
\begin{proof}
We first observe that $\mathfrak{B}(S, X)$ is a vector space.  
This follows from the fact that if $f$ is bounded by $C$ then for all
$a \in \reals$ we have $af$ is bounded by $\abs{a} C$ if both $f$ and
$g$ are bounded by $C_1$ and $C_2$ respectively then using the
triangle inequality in $X$ we see that $f+g$ is bounded by $C_1 + C_2$.

Next we prove that we have defined a norm.  The fact that $\norm{f}
\geq 0$ and $\norm{0} =0$ follow immediately from the definition and
the fact that $\abs{\cdot}$ is a norm on $X$.   If $\norm{f}
= 0$ the it follows that $\abs{f(s)} = 0$ for all $s \in S$
and therefore $f=0$.  Let $c \in \reals$ then since $\abs{cf(s)} =
\abs{c} \abs{f(s)}$ it follows that $\norm{cf} \leq \abs{c}\norm{f}$.
On the other hand, let $\epsilon > 0$ be given then we may find an $s
\in S$
such that $\norm{f} - \epsilon < \abs{f(s)}$.  It follows that 
\begin{align*}
\abs{c} \norm{f} - \abs{c} \epsilon < \abs{c} \abs{f(s)} = \abs{cf(s)}
\end{align*}
Now $\epsilon$ was chosen arbitrarily so we may let $\epsilon \to 0$
and we get the inequality $\abs{c} \norm{f} \leq
\abs{cf(s)}$.  Now take the supremum over $s \in S$ to get opposite
inequality $\abs{c} \norm{f} \leq \norm{cf}$ and it follows that
$\abs{c} \norm{f} = \norm{cf}$.  The triangle inequality follows in a
similar way.  Given an $f$ and $g$ we see using the triangle
inequality in $X$ that for all $s \in S$ we
have $\abs{f(s) + g(s)} \leq \abs{f(s)} + \abs{g(s)} \leq \norm{f}
+\norm{g}$; taking the supremum over $s \in S$ we get $\norm{f+g} \leq
\norm{f} + \norm{g}$.
\end{proof}

Now we have the following extension result
\begin{lem}Let $X$ be a normed vector space and let $Y$ be a Banach
  space.  Suppose that $V \subset X$ is a subspace and $A : V \to Y$
  is a bounded linear map, then $A$ has a unique extension
  $\overline{A} : \overline{V} \to Y$ from the closure of $V$ into
  $Y$.  Moreover if $C$ is a bound on $A$ the $C$ is also a bound on $\overline{A}$.
\end{lem}
\begin{proof}
\end{proof}

As is usual to compute with integrals it is imperative to connect the
integration with differentiation.  Since we are dealing with the
Riemann integral we must use relatively strong hypotheses however
these results will suffice for our applications and the proof are very
simple.  We start with the Fundamental Theorem of Calculus.

\begin{thm}[Fundamental Theorem of
  Calculus]\label{FundamentalTheoremOfCalculusForBanachSpaceRiemannIntegrals}Let
 $X$ be a Banach space and let $f : [a,b] \to X$ be continuously
 differentiable then 
\begin{align*}
f(b) - f(a) &= \int_a^b Df(t) dt
\end{align*}
\end{thm}
\begin{proof}
First we let $g(t)$ be a regulated function from $[a,b]$ to $X$ and
consider the integral $\int_a^s g(t) \, dt$.  Suppose that $g$ is
continuous at $c \in [a,b]$ and let $\epsilon > 0$ be given.  By
right continuity we may find $\delta > 0$ such that $\abs{g(c+h) -
  g(c)} < \epsilon$ for all $\abs{h} < \delta$.  If we let $G(s) = \int_a^s g(t) \,
dt$ then if $\abs{h} < \delta$ we have
\begin{align*}
\abs{ \frac{G(c + h) - G(c)}{h} - g(c)} &= \abs{\frac{1}{h}
                                          \int_c^{h+c} g(t)\, dt  -
                                          \frac{1}{h} \int_c^{h+c}
                                          g(c) \, dt} \\
&= \abs{\frac{1}{h}  \int_c^{h+c} (g(t) - g(c)) \, dt} \\
&\leq \frac{1}{\abs{h}} \abs{h} \sup_{c \leq s \leq c+h} \abs{g(t) -
  g(c)} 
=\sup_{c \leq s \leq c+h} \abs{g(t) -  g(c)} 
\end{align*}
By continuity
TODO
\end{proof}

\begin{prop}\label{RiemannIntegralOfContinuousMaps}Let $X$ and $Y$ be
  Banach spaces and let $f : [a,b] \to L(X,Y)$ be regulated then it
  follows that for every $x \in X$ we have
\begin{align*}
\int_a^b f(t) x \, dt &= \int_a^b f(t) \, dt \cdot x
\end{align*}
\end{prop}
\begin{proof}
TODO
\end{proof}

\subsection{Bochner Integrals}

TODO:

\section{Differentiation in Banach Spaces}

TODO:
\begin{itemize}
\item Absolute convergence of a infinite series in a Banach space
\item Define space of linear maps with operator norm
\item Show that Frechet deriviative is equal to Jacobian matrix on
  finite dimensional spaces
\end{itemize}

\begin{prop}Let $X$ be a Banach space then if $\sum_{j=0}^\infty a_j$
  converges absolutely then $\sum_{j=0}^\infty a_j$ converges in $X$.
\end{prop}
\begin{proof}
By completeness of $X$ it suffices to show that $S_n = \sum_{j=0}^n
a_j$ is a Cauchy sequence.  Let $\epsilon > 0$ be given and pick $n >
0$ such that $\sum_{j=n}^\infty \norm{a_j} < \epsilon$.  Then for all
$m \geq n$ we have
\begin{align*}
\norm{S_m - S_n} &\leq \norm{\sum_{j=n}^{m-1} a_j} \\
&\leq \sum_{j=n}^{m-1}\norm{ a_j} \leq \sum_{j=n}^{infty}\norm{ a_j} < \epsilon
\end{align*}
and we are done.
\end{proof}

\begin{prop}Let $X$ be a Banach space.  The set of invertible maps in $L(X)$ is open, moreover
  for any invertible map $A \in L(X)$ and any $\norm{A-B} <
  \norm{A}^{-1}$ we have 
\begin{align*}
B^{-1} &= \sum_{n=0}^\infty A^{-n-1} (A-B)^n 
\end{align*}
In particular, the inversion map is continuously differentiable on its domain.
\end{prop}
\begin{proof}
We first assume that $A= I$ is the identity map.  If we let $\norm{B}
< 1$ then note that 
\begin{align*}
\norm{\sum_{n=0}^m B^n} &\leq \sum_{n=0}^m \norm{ B}^n <
                          \sum_{n=0}^\infty \norm{ B}^n =
                          \frac{1}{1-\norm{B}} < \infty
\end{align*}
which shows that $\sum_{n=0}^\infty B^n$ converges absolutely and
is well defined in $L(X)$.  Moreover we have
\begin{align*}
\norm{(1- B) \sum_{n=0}^\infty B^n }
\end{align*}
TODO: Finish....
\end{proof}

We present some of the basic results on differentiation in Banach
spaces.

\begin{defn}Let $X$ and $Y$ be Banach spaces, let $U \subset X$ be
  open and let $f : U \to Y$ be
  a map.  We say that $f$ is differentiable at $x \in U$ if there
  exists a bounded linear map $L : X \to Y$ such that
\begin{align*}
\lim_{h \to 0} \frac{f(x + h) - f(x) -Lh}{\norm{h}} = 0
\end{align*}
We call the linear map $L$ the \emph{Frechet derivative} of $f$ at $x$
and denote it $Df(x)$.
\end{defn}

As it stand, we have been a little loose in defining \emph{the}
Frechet derivative as we have not ruled out the possibility that
multiple linear maps may satisfy the defining property.   The first
task is to show that in fact the Frechet derivative is uniquely
defined provided it exists.

\begin{prop}Suppose $A$ and $B$ are bounded linear maps satisfying the
  defining property of the Frechet derivative then $A = B$.
\end{prop}
\begin{proof}
Let $\epsilon > 0$ be given and pick $\delta > 0$ so that we have
$\norm{f(x+h) - f(x) -Ah} < \epsilon \norm{h}$ for $\norm{h} < \delta$
and similarly for $B$.  It then follows that
\begin{align*}
\norm{Ah - Bh} &\leq \norm{f(x+h) - f(x) -Ah} + \norm{f(x+h) - f(x)
                 -Bh} < 2 \epsilon \norm{h}
\end{align*}
so by linearity we see that $\norm{A - B} < 2 \epsilon$.  Since
$\epsilon>0$ was arbitrary it follows that $\norm{A -B} = 0$ and
therefore $A = B$.
\end{proof}

There are weaker forms of derivative that one can consider.  For the
most part we shall be concerned with only the Frechet derivative but
it can be helpful to be aware of the alternatives if for no other
reason than to refine one's understanding of the nature of the Frechet derivative.
\begin{defn}Let $X$ and $Y$ be Banach spaces, let $U \subset X$ be
  open and let $f : U \to Y$ be
  a map.  Let $v \in X$, the we say that $f$ has a directional
  derivative at $x$ in the direction of $v$ if the limit 
\begin{align*}
df(x,v) &= \lim_{t \to
    0} \frac{f(x + tv) - f(x)}{t}
\end{align*} 
exists.  We say that $f$ is
  \emph{G\^{a}teaux differentiable at $x$} if it has directional
  derivatives at all $v \in X$.
\end{defn}

We first observe that Frechet differentiability implies G\^{a}teaux differentiability.
\begin{prop}\label{FrechetDifferentiableImpliesGateauxDifferentiable}Let $X$ and $Y$ be Banach spaces, let $U \subset X$ be
  open and let $f : U \to Y$ be
  a Fr\'{e}chet differentiable at $x \in U$.  Then $f$ is G\^{a}teaux
  differentiable at $x$ and the directional derivative at $v$ is equal
  to $Df(x)v$.
\end{prop}
\begin{proof}
Let $\epsilon >0$ be given and pick $\delta>0$ such
that $\norm{f(x+h) - f(x) -Df(x)h} \leq \epsilon \norm{h}$ for all
$\norm{h} < \delta$.  With $v \in X$ fixed and suppose that $\norm{v}
= 1$; we note that for all
$\abs{t} < \delta$ we have $\norm{f(x+tv) - f(x) - t Df(x) v}
\leq \epsilon \abs{t} $ and thus 
\begin{align*}
\norm{\frac{f(x+tv) - f(x) }{t} - Df(x) v} < \epsilon
\end{align*}
so that $df(x,v) = Df(x) v$.  Now it is a simple matter to validate
that $df(x, tv) = t df(x,v) = Df(x) \cdot tv$ for all $t \in \reals$.
\end{proof}
In general G\^{a}teaux derivatives need not be linear (i.e. even
though $df(x,tv) = tdf(x,v)$ it is not necessarily the case that
$df(x,v+w) = df(x,v) + df(x,w)$) and even if
linear need not be bounded.  Somewhat more surprising is that even if
the G\^{a}teaux derivative exists and is bounded and linear the
Fr\'{e}chet derivative may not exist.  What is necessary is that the
limits $\lim_{t \to 0} \frac{f(x+tv) -f(x)}{t}$ converge uniformly for
$v$ in the unit sphere.

We calculate some trivial Frechet derivatives.
\begin{examp}Let $f : X \to Y$ be a constant map $f(x) = y$ for some
  fixed $y \in Y$, then $f$ is differentiable at every point $x \in X$
  and moreover $Df(x) = 0$.
\end{examp}
\begin{examp}Let $A : X \to Y$ be a bounded linear map, then $A$ is differentiable at every point $x \in X$
  and moreover $Df(x) = A$.
\end{examp}

The following example generalizes the product rule of calculus.
\begin{examp}Let $A : X_1  \times \dotsm \times X_n \to Y$ be a bounded
  multilinear map, then $A$ is differentiable at every point $x \in
  X_1 \times \dotsm \times X_n$
  and moreover 
\begin{align*}
Df(x_1, \dotsc, x_n)(h_1, \dotsc, h_n) = A(h_1, x_2, \dotsc, x_n) +
  A(x_1, h_2, x_3, \dotsc, x_n) + \dotsm + A(x_1, x_2, \dotsc, h_n)
\end{align*}
\end{examp}

Another important case is the behavior of deriviative when composing
with a linear map.
\begin{examp}\label{FrechetDerivativeCompositionWithLinearMap}Let $X$, $Y$ and $Z$ be Banach spaces, let $U \subset X$
  be open, let $f : U \to Y$ be differentiable and let $A : Y \to Z$
  be a bounded linear map, then $D (A \circ f)(x) = A \circ Df(x)$.

Note that this would follow from the Chain Rule below (Proposition
\ref{ChainRuleBanachSpaces}) but is worth showing this directly to get
some practice with the definitions.  Let $\epsilon > 0$ be given and
pick $\delta>0$ such that $\norm{f(x+h) - f(x) - Df(x)h} \leq
\frac{\epsilon}{\norm{A}} \norm{h}$ for all $\norm{h} < \delta$.  Note
that
\begin{align*}
\norm{Af(x+h) - Af(x) - A Df(x)h} &\leq \norm{A} \norm{f(x+h) - f(x) -
                                    Df(x)h} \leq \epsilon \norm{h}
\end{align*}
for all $\norm{h} < \delta$ which shows the result.
\end{examp}

\begin{prop}\label{DifferentiabilityImpliesContinuity}Let $X$ and $Y$ be Banach spaces, let $U \subset X$ be
  open and let $f : U \to Y$ be differentiable at $x \in U$ then $f$
  is continuous at $x$.
\end{prop}
\begin{proof}
Let $\epsilon > 0$ be given and define $0 < \delta < \frac{\epsilon}{1 + \norm{Df(x)}}$ small enough so
that $\norm{f(x+h) - f(x) - Df(x)h} \leq \norm{h}$ for all $\norm{h} <
\delta$ then 
\begin{align*}
\norm{f(x + h) - f(x)} &\leq \norm{f(x + h) - f(x) - Df(x)h} +
                         \norm{Df}\norm{h} \\
&\leq \norm{h} + \norm{Df}\norm{h} < \epsilon
\end{align*}
and continuity is proven.
\end{proof}

\begin{prop}[Chain Rule]\label{ChainRuleBanachSpaces}Let $X$, $Y$ and $Z$ be
  Banach spaces, let $U \subset X$ and $f : U \to Y$ be differentiable
  at $x \in U$, let $V \subset Y$ with $f(U) \subset V$, $g : V \to Z$ be
  differentiable at $f(x)$ then $g \circ f : U \to Z$ is
  differentiable at $x$ and moreover
\begin{align*}
D(g \circ f)(x) &= Dg(f(x)) \circ Df(x)
\end{align*}
\end{prop}
\begin{proof}
Let $\epsilon$ be given.  Let $\tilde{\delta} > 0$ be chosen so that
$\norm{g(f(x) + h) - g(f(x)) - Dg(f(x)) h} < \frac{1}{2} \epsilon
\norm{h}$ for all $\norm{h} < \tilde{\delta}$.  By continuity of $f$
at $x$ we can choose $\delta_1>0$ such that $\norm{f(x+h) - f(x)} <
\tilde{\delta}$ for all $\norm{h} < \delta_1$ and by differentiability
we may choose $\delta_2>0$ such that $\norm{f(x+h) - f(x) - Df(x)h} <
\frac{\epsilon}{\epsilon + 2\norm{Dg(f(x))}} \norm{h}$ for all
$\norm{h} < \delta_2$.  Let $\delta = \delta_1 \vee \delta_2$ and then
for $\norm{h} < \delta$ we compute
\begin{align*}
&\norm{g(f(x+h)) - g(f(x)) - Dg(f(x)) Df(x) h} \\
&\leq \norm{g(f(x+h)) -
                                                g(f(x)) - Dg(f(x))
                                                (f(x+h)-f(x)) }  \\
&+\norm{Dg(f(x))(f(x+h)-f(x))
                                                - Dg(f(x)) Df(x) h}\\
&\leq \frac{1}{2} \epsilon\norm{f(x+h)-f(x)}  +\norm{Dg(f(x))}\norm{f(x+h)-f(x) - Df(x) h}\\
&\leq \frac{1}{2} \epsilon \norm{h}  +(\frac{\epsilon}{2} +
  \norm{Dg(f(x))}) \frac{\epsilon}{\epsilon + 2\norm{Dg(f(x))}}
  \norm{h} = \epsilon \norm{h}\\
\end{align*}
and we're done.
\end{proof}

\subsection{Higher Order Derivatives and Taylor's Theorem}

\begin{thm}[Mean Value Theorem]\label{MeanValueTheoremBanachSpaces}Let
  $X$ and $Y$ be  Banach spaces, $U \subset X$ be open and let $f : U
  \to Y$ be continuously differentiable.  Suppose $x \in U$ and $y \in
  X$ such that $x + ty \in U$ for all $0 \leq t \leq 1$ then
\begin{align*}
f(x + y) - f(x) &= \int_0^1 Df(x + ty) y dt = \int_0^1 Df(x + ty) dt
                  \cdot y
\end{align*}
\end{thm}
\begin{proof}
Define $g(t) = f(x + ty)$.  Then by the Chain Rule it follows that
$g(t)$ is continuously differentiable and $Dg(t) = Df(x + ty) y$.
Since $Dg(t)$ is continuous we may apply the Fundamental Theorem of
Calculus (Theorem
\ref{FundamentalTheoremOfCalculusForBanachSpaceRiemannIntegrals}) to
conclude that 
\begin{align*}
f(x+y) - f(x) &= g(1) - g(0) = \int_0^1 Df(x+ty) y dt = \int_0^1
                Df(x+ty) dt \cdot y
\end{align*}
where in the last inequality we have use Proposition \ref{RiemannIntegralOfContinuousMaps}.
\end{proof}

Higher order derivatives are defined by iterating Frechet derivatives.  For
example if we assume that the map $f : U \to Y$ differentiable on all
of $U$ then the second derivative is obtained by
taking the derivative of the map $Df : U \to L(X,Y)$ whereever it
exists.  Thus the second derivative is a map $D^2f : U \to
L(X,L(X,Y))$.  

\begin{examp}Let $A : X \to Y$ be a bounded linear map then $D^2A = 0$.
\end{examp}

Based on the definition via induction we think of $D^nf$ as a map from
$U$ to $L(X, \dotsb ,L(X, Y) \dotsb)$.  The range here actually has a
more convenient representation as the space of multilinear maps $X
\times \dotsm \times X \to Y$.  For example given an element in $f \in
L(X,L(X,Y))$ we may define $\tilde{f}(u,v) = f(u) v$ and note that
\begin{align*}
\tilde{f}(au + bv, w) &= f(au + bv) w = a f(u) w + b f(v) w = a
\tilde{f}(u,w) + b \tilde{f}(v,w)
\end{align*} and 
\begin{align*}
\tilde{f}(u, av + bw) &= f(u)(av+bw) = a f(u) v + b f(u) w = a
                        \tilde{f}(u,v) + b\tilde{f}(u,w)
\end{align*}
so that $\tilde{f}$ is indeed bilinear.  It is easy to see that this
is an isomorphism and that the construction extends to general $n$.

TODO: Do this in the required excruciating detail...

In the sequel, it will be convenient to view higher derivatives as maps
from $U$ to the space of multilinear maps.  It turns out that higher
derivatives are not arbitrary multilinear maps but also have the
property of being symmetric.
\begin{prop}Let $U \subset X$ be open and $f : U \to Y$ be $C^p$ then
  $D^pf(x)$ is multilinear and symmetric for every $x \in U$.
\end{prop}
\begin{proof}
TODO:
We first consider the case $p=2$.  Let $u,v \in X$ and consider
\begin{align*}
D^2f(x) (u,v) = 
\end{align*}
\end{proof}

A more complicated but important example is the computation of the
derivative of the inverse in a Banach algebra.
\begin{prop}\label{FrechetDerivativeOfInverse}The map $\phi(A) = A^{-1}$ on $L(X,X)$ is $C^\infty$ on the open set of
  invertible maps.  In fact we have
\begin{align*}
D^n\phi(A)(h_1, \dotsc, h_n) &= (-1)^n\sum_\sigma A^{-1} h_{\sigma_1}
                               A^{-1} \dotsm h_{\sigma_n} A^{-1}
\end{align*}
where the summation is over all permutations of $\lbrace 1, \dotsc, n \rbrace$.
\end{prop}
\begin{proof}
We first compute the first derivative of $\phi$.  Let $A$ be invertible and
observe that for $\norm{h} < \norm{A^{-1}}^{-1}$ we know that $I +
A^{-1} h$ is invertible and moreover
\begin{align*}
(A + h)^{-1} &= A^{-1} (I + h A^{-1})^{-1} = A^{-1} \sum_{n=0}^\infty
              (-1)^n h^n A^{-n}
\end{align*}
and therefore using the absolute convergence of the series on the
right we get
\begin{align*}
\norm{(A + h)^{-1} - A^{-1} + A^{-1} h A^{-1}} &\leq \sum_{n=2}^\infty
                                                 \norm{h}^n
                                                 \norm{A^{-1}}^n \\
&= \frac{\norm{h}^2\norm{A^{-1}}^2}{1 - \norm{h}\norm{A^{-1}}} < \norm{h}^2\norm{A^{-1}}^2
\end{align*}
which shows us that $D\phi(A)h = -A^{-1} h A^{-1}$ (for $\epsilon > 0$
let $\delta < \epsilon \norm{A^{-1}}^{-2}$).

Now to see that $\phi$ is in fact $C^{\infty}$, we do an induction.
TODO: Finish
\end{proof}

With the defintion of higher derivatives available we are now able to
extend the Mean Value Theorem to Taylor's Theorem in Banach spaces.  
\begin{thm}[Taylor's Theorem]\label{TaylorsTheoremBanachSpaces}Let $X$
  and $Y$ be Banach spaces and let $U \subset X$ be open and of class
  $C^p$.  Suppose that $x \in U$ and $y \in X$ such that $x + ty \in
  U$ for all $0 \leq t \leq 1$ then we have
\begin{align*}
f(x+y) &= f(x) + Df(x) y + \dotsm + \frac{D^{p-1}f(x)
         y^{(p-1)}}{(p-1)!} + \int_0^1 \frac{(1-t)^{p-1}}{(p-1)!}
         D^pf(x + ty) y^{(p)} dt
\end{align*}
where $y^{(k)} = (y, \dotsc, y) \in X^k$.
\end{thm}
\begin{proof}
TODO
\end{proof}

It is worth noting that in the case $Y=\reals$ that Theorem
\ref{TaylorsTheoremBanachSpaces} can be proven using the one
dimensional version Theorem \ref{TaylorsTheorem} and the chain rule
Proposition \ref{ChainRuleBanachSpaces}.  We'll show this in the proof
of the Lagrange form of the remainder term below.

We've presented Taylor's Theorem in Banach spaces with the integral
form of the remainder term.  There are several different versions of
the remainder and estimates derived therefrom that are useful to
note.  The first that we mention is applicable in the important case
in which $Y = \reals$; the Lagrange form of the remainder.
\begin{prop}\label{LagrangeFormRemainderBanachSpaces}There is a number $c \in
  (0,1)$ such that 
\begin{align*}
\int_0^1 \frac{(1-t)^{p-1}}{(p-1)!}
         D^pf(x + ty) y^{(p)} dt &= 
\frac{D^pf(x + cy) y^{(p)}}{p!}
\end{align*}
\end{prop}
\begin{proof}
We derive this from the one dimensional Taylor's Theorem.  Note that
$g(t) = f(x + ty)$ is $C^p$ from $[0,1]$ to $\reals$ and by the chain
rule we have $g^\prime(t) = Df(x + ty) y$.  Now since evaluation $A \to A y$
is a bounded linear map on $L(X,Y)$, an induction argument using either Example \ref
{FrechetDerivativeCompositionWithLinearMap}
or the chain rule shows that $g^{(k)}(t) = D^kf(x+ty) y^{(k)}$.
Now apply Theorem \ref{TaylorsTheorem} to see there is a $0 < c < 1$ such that 
\begin{align*}
f(x+y) &= g(1) = g(0) + g^\prime(0) + \dotsm + \frac{g^{(p-1)} (0)}{(p-1)!} +
         \frac{g^{(p)} (c)}{p!}  \\
&=f(x) + Df(x) y + \dotsm + \frac{D^{p-1}f(x) y^{(p-1)}}{(p-1)!} +
         \frac{D^{p}f(x + cy) y^{(p)}}{p!}
\end{align*}
\end{proof}

\subsection{Inverse and Implicit Function Theorems}

\begin{thm}\label{InverseFunctionTheoremBanachSpaces}Let $X$ and $Y$ be Banach spaces let $U \subset X$ be an
  open subset of $X$ and suppose that $f : U \to Y$ is continuously
  differentiable and $Df(x)$ is invertible at $x \in U$.  There is an
  open set $V \subset U$ containing $x$ and an open set $W \subset Y$
  containing
  $f(x)$ such that $f : V \to W$ is a bijection and $f^{-1}$ is
  continuously differentialble on $W$.
\end{thm}

The general Banach space proof is rather elegant but feels a bit like
magic.  We'll be a bit redundant a  provide both the general proof as
well as a proof for the finite dimensional case that is more verbose
but is very elementary.

For the finite dimensional proof we use the following simple
consequence of the mean value theorem
that shows a continuously differentiable function is Lipschitz
continuous on a bounded domain.
\begin{lem}\label{IFT:BoundedDerivativeImpliesLipschitz}Let $f : \reals^n \to \reals^n$ be differentiable on an
  open rectangle $R = (a_1,b_1) \times \dotsm \times (a_n, b_n)$ such that 
\begin{align*}
\abs{\frac{\partial f_i}{\partial x_j}(x)} \leq M
\end{align*}
for all $1 \leq i,j \leq n$ and $x \in R$ then it follows that
$\norm{f(y)-f(x)} \leq M \cdot n^2 \cdot \norm{y-x}$ for all $x,y \in R$.
\end{lem}
\begin{proof}
By expanding as a telescoping sum and the one
dimensional mean value theorem we
get for every $i=1, \dotsc, n$ 
\begin{align*}
f_i(y) - f_i(x) &= \sum_{j=1}^n f_i(y_1, \dotsc, y_j, x_{j+1},
                     \dotsc, x_n) - 
f_i(y_1, \dotsc, y_{j-1}, x_{j},
                     \dotsc, x_n)\\
&= \sum_{j=1}^n \frac{\partial f_i}{\partial x_j} (y_1, \dotsc, y_{j-1}, y^*_j, x_{j+1},
                     \dotsc, x_n) (y_j - x_j)\\
\end{align*}
where $a_j < y^*_j  < b_j$ (in fact $x_j \leq y^*_j \leq y_j$ when
$x_j \leq y_j$ and similarly when $y_j < x_j$).  Now by the triangle
inequality and the bound on partials of $f$ we get
\begin{align*}
\norm{f(y) - f(x)} &\leq \sum_{i=1}^n \abs{f_i(y) - f_i(x)} \\
&= \sum_{i=1}^n \sum_{j=1}^n \abs{\frac{\partial f_i}{\partial x_j} (y_1, \dotsc, y_{j-1}, y^*_j, x_{j+1},
                     \dotsc, x_n)} \abs{ y_j - x_j } \\
&\leq \sum_{i=1}^n \sum_{j=1}^n M \norm{y - x} = M \cdot n^2 \cdot
  \norm{y -x}
\end{align*}
\end{proof}

Now we can proceed with the proof of the theorem in the finite
dimensional case.
\begin{proof}
We first make a reduction to the case in which $Df(x)$ is the
identity.  If the result is proven in that case then for general $f$
we can define $Df(x)^{-1} \circ f : X \to X$ where from the Chain Rule
it follows that
$D(Df(x)^{-1} \circ f)(x)$ is the identity.  Applying the inverse
function theorem we see there exists open
sets $V$ and $\tilde{W}$ containing $x$ and $Df(x)^{-1}  f(x)$ respectively such
$Df(x)^{-1} \circ f$ is a bijection from $V$ to $\tilde{W}$ with $(Df(x)^{-1} \circ f)^{-1}$
continuously differentiable.  Now we define $W = Df(x)(\tilde{W})$
which is open by continuity of $Df(x)^{-1}$ and contains $f(x)$.
Since $f^{-1} = Df(x) \circ Df(x)^{-1} \circ f$ it follows by the
Chain Rule that $f^{-1}$ is continuously differentiable on $W$.

\begin{clm}There is an open ball $B(x, \delta) \subset U$ such that $f$ is
injective on the closure of $B(x, \delta)$, $Df(y)$ is invertible for all $y \in B(x,
\delta)$ and 
\begin{align}
\abs{\frac{\partial f_i}{\partial x}(y) - \frac{\partial
    f_i}{\partial x}(x)} 
&< \frac{1}{2n^2} \text{ for all $1 \leq i,j \leq n$ and $y \in
  B(x,\delta)$}
\end{align}\label{IFT:Partials}
\end{clm}

By the the openness of $U$, triangle inequality and the fact that $Df(x)$ is the
identity we know that we can find $\delta > 0$
such that $B(x,\delta) \subset U$ and 
\begin{align*}
\norm{f(x+h) - f(x)} &= \norm{f(x+h) - f(x) - h + h} \geq \norm{h} -
                       \norm{f(x+h) - f(x) - h} \\
&\geq \frac{1}{2}\norm{h}
\end{align*}
so injectivity on $B(x, \delta)$ follows; by continuity of $f$ the
bound and hence the injectivity extends to the closure of $B(x,\delta)$.  Since the invertible linear maps are an open
subset of $L(X,Y)$ and $Df$ is continuous we may also assume that
$\delta > 0$ is chosen so that $Df$ is invertible on $B(x,\delta)$.
Similarly continuity of $Df$ implies the continuity of each partial
derivative $\frac{\partial f_i}{\partial x}$ and therefore
\eqref{IFT:Partials} follows for sufficiently small $\delta > 0$.

The next claim should be thought of as asserting that the inverse of
$f$ is Lipschitz.  As it turns out the estimate is useful in showing
that $f^{-1}$ exists.

\begin{clm}$\norm{y - z}
\leq 2 \norm{f(y) - f(z)}$ for all $y,z \in B(x, \delta)$.
\end{clm}

Define $g(x) = f(x) - x$ on $B(x, \delta)$.  Because $Df(x)$ is the
identity we know that $\frac{\partial f_i}{\partial x}(x) =
\delta_{ij}$ and therefore 
\begin{align*}
\abs{\frac{\partial g_i}{\partial x}(y)} &=
\abs{\frac{\partial f_i}{\partial x}(y) - \frac{\partial f_i}{\partial
                                           x}(x)} 
\leq \frac{1}{2n^2}
\end{align*}
Since $y$ and $z$ are contained in some open rectangle that is a
subset of $B(x,\delta)$ we can apply Lemma \ref{IFT:BoundedDerivativeImpliesLipschitz} and the
triangle inequality to
conclude that 
\begin{align*}
\frac{1}{2} \norm{y - z} &\geq \norm{g(y) - g(z)} \\
&=\norm{f(y) - y - g(z) + z} \\
&\geq \norm{y - z} - \norm{f(y) - g(z)}
\end{align*}
and the claim follows by collecting terms.

The next step in the proof is to validate that the image of $B(x,
\delta)$ under $f$ contains an open set (on which we will then have a
bijection).  Consider the function $f(y) - f(x)$.  It is
continuous and by compactness of the boundary $\partial B(x,\delta)$
and the injectivity of $f$ on the closed ball we know that there
exists an $\epsilon > 0$ such that $g(y) \geq \epsilon$ on $\partial
B(x,\delta)$.  Define $W = B(f(x), \epsilon/2)$ and notice that by the
choice of $\epsilon$, the triangle inequality and the previous claim
we have for all $z \in W$ and $y \in \partial B(x, \delta)$
\begin{align}
\norm{z - f(y)} &\geq \norm{f(x) - f(y)} - \norm {f(x) - z} \geq d -
                  \frac{d}{2} \\
&= \frac{d}{2} > \norm{z - f(x)}
\end{align}\label{IFT:InteriorMinimumEstimate}

This estimate is used to construct an open set in the image of $f$.

\begin{clm}For every $z \in W$ there is a unique $y \in B(x, \delta)$ such
that $f(y) = z$.
\end{clm}

To see existence we let $z \in W$ be given and we define the function
$h(y) = \norm{f(y) - z}^2 = \sum_{j=1}^n (f_j(y) - z_j)^2$.  Differentiability of $h$ follows from the
differentiability of $f$ and the chain rule.  In
particular, $h$ is continuous and therefore by compactness of the
closed ball $\overline{B}(x, \delta)$ it attains its minimum.  By the
estimate \eqref{IFT:InteriorMinimumEstimate} we see that the minimum
must occur in the interior of the ball.  Therefore we know that the
derivative of $h$ must vanish at the minimum so by the Chain Rule we
know that for all $v \in X$
\begin{align*}
0 &=D \norm{f - z}^2 (y) \cdot v = 2 \norm{Df(y)\cdot v}{f(y) - z}
\end{align*}
and by the invertibility of $Df(y)$ it follows that we must have $f(y)
= z$ at the minimum.

The uniqueness of $y$ follows from the injectivity of $f$.

Now we define $V = f^{-1}(W) \cap B(x, \delta)$ and it follows that
$f$ is a bijection from $V$ to $W$. Now that $f^{-1}$ is well defined
we immediately get its continuity.  

\begin{clm}$f^{-1}$ is continuous on $W$.
\end{clm}

The second claim proved the bound $\norm{y - z} \leq 2 \norm{f(y) -
  f(z)}$ on $B(x, \delta)$ which certainly shows that $\norm{f^{-1}(z)
  - f^{-1}(w)} \leq 2 \norm{z -w}$ on $W$ so that $f^{-1}$ is
Lipschitz in particular continuous.

It remains to show that $f^{-1}$ is differentiable.

\begin{clm}$f^{-1}$ is continuously differentiable on $W$.
\end{clm}

In fact we show (as would follow from the Chain Rule) that $Df^{-1}
(z) = \left[ Df(f^{-1}(z) )\right]^{-1}$ for all $z \in W$.  Note that
this is well defined since $Df$ is invertible on all of $V$.  To clean
up the notation a bit let $A = Df(f^{-1}(z) )$.  Let $\epsilon > 0$ be
given.  Using differentiability of $f$ at $f^{-1}(z)$ we choose
$\tilde{\eta} > 0$ such that
\begin{align*}
\norm{f(f^{-1}(z) +
  h) -
  f(f^{-1}(z)) - Ah} &< \frac{\epsilon}{2\norm{A^{-1}}} \norm{h}
                       \text{ for all $\norm{h} < \tilde{\eta}$}
\end{align*}
By continuity of $f^{-1}$ at $z$ we choose $\eta > 0$ such that
$\norm{f^{-1}(z+h) - f^{-1}(z)} < \tilde{\eta}$ for
all $\norm{h} < \eta$.  Pick $h \in Y$ with $\norm{h}<\eta$ and
compute using the Lipschitz continuity of $f^{-1}$
\begin{align*}
&\norm{f^{-1}(z + h) - f^{-1}(z) - A^{-1} h} \\
&=\norm{A^{-1} \left(f(f^{-1}(z+h)) - f(f^{-1}(z)) - A(f^{-1}(z + h) -
                                              f^{-1}(z)) \right)} \\
&\leq \norm{A^{-1}} \norm{f(f^{-1}(z+h)) - f(f^{-1}(z)) - A(f^{-1}(z + h) -
                                              f^{-1}(z))} \\
&\leq \frac{1}{2} \epsilon \norm{f^{-1}(z + h) - f^{-1}(z)} \leq \epsilon \norm{h} 
\end{align*}
which gives us $Df^{-1}(z) = \left[ Df(f^{-1}(z) )\right]^{-1}$.
Continuity of $Df^{-1}$ follows from the continuity of $Df$,
continuity of $f^{-1}$ and continuity of inversion of invertible maps
in $L(X,Y)$.
\end{proof}

The proof of the Inverse Function Theorem in general Banach spaces
rests on a simple result that is of broad applicability.  The result
actually doesn't use the vector space structure and is valid in
general complete metric spaces; it provides a
very general mechanism for solving equations in such spaces.

\begin{prop}[Contraction Mapping
  Principle]\label{ContractionMappingPrinciple}Let $(S,d)$ be a complete
  metric space
  space, let $F \subset S$ be a closed subset and let $g : F \to F$ be
  a mapping such that there exists a constant $0 < K < 1$ such that 
\begin{align*}
d(g(x), g(y)) &\leq K d(x,y) \text{ for all $x,y \in F$}
\end{align*}
then there exists a unique $x_0 \in F$ such that $g(x_0) = x_0$ and
moreover given any $x \in F$ the sequence $\lbrace g^n(x) \rbrace$ is
Cauchy and converges to $x_0$.
\end{prop}
\begin{proof}
First we prove uniqueness.  Suppose there are two points $x$ and $y$
satisfying $g(x) = x$ and $g(y) = y$ then we know that
\begin{align*}
d(x,y) &= d(g(x), g(y)) \leq K d(x,y)
\end{align*}
and since $0 < K < 1$ this shows that $d(x,y) = 0$.

Now we let $x \in F$ be arbitrary and show that $g^n(x)$ is Cauchy.
Suppose that $m > n$ and observe that by a simple induction
\begin{align*}
d(g^n(x), g^m(x)) \leq K^n d(x, g^{m-n}(x))
\end{align*}
In particular, we have that $d(g^n(x), g^{n+1}(x)) \leq K^n d(x,g(x))$
and therefore by the triangle inequality 
\begin{align*}
d(x, g^n(x)) &\leq d(x, g(x)) + \dotsm + d(g^{n-1}(x), g^n(x)) \leq (1
               + \dotsm + K^{n-1}) d(x,g(x)) < \frac{d(x,g(x))}{1-K}
\end{align*}
Putting these two bounds together we see that $d(g^n(x), g^m(x)) \leq  \frac{K^n
  d(x,g(x))}{1-K}$ hence $g^n(x)$ is Cauchy.  

Since $F$ is
a closed subset of a complete metric space, it is 
complete and we know that $g^n(x)$ converges to some $x_0 \in F$; it remains to show that $x_0$
is a fixed point of $g$.  Let $\epsilon > 0$ be given and chose $N >
0$ such that $d(x_0, g^n(x)) < \epsilon$ for all $n \geq N$.  Then we
know that 
\begin{align*}
d(g(x_0), g^n(x)) &\leq K d(x_0, g^{n-1}(x)) \leq K\epsilon < \epsilon
\end{align*}
for all $n \geq N+1$ which shows that $g^n(x)$ converges to $g(x_0)$.  It
follows that $x_0 = g(x_0)$.
\end{proof}

Note that in a Banach space finding fixed points $f(x) = x$ is
equivalent to finding roots $g(x) = 0$ (just find a fixed point of
$g(x) + x$ to find a root of $g$ and find a root of $f(x) - x$ to find
a fixed point of $f$). 

The Inverse Function Theorem has the following equally important
consequence that is known at the Implicit Function Theorem.

\begin{thm}[Implicit Function
  Theorem]\label{ImplicitFunctionTheorem}Let $X$, $Y$ and $Z$ be a Banach
  spaces, let $U \subset X \times Y$ be an open set  and let $f : U
  \to Z$ be $C^p$.  
Suppose $(x_0,y_0) \in U$, that $f(x_0,y_0) = 0$
  and $Df(x_0,y_0)(0,v)$ defines an invertible map from $Y \to Z$, then
  there exists an open set $V \subset X$ such that $x_0 \in V$, an
  open set $W \subset Y$ such that $y_0 \in W$ and a function $g : V \to
  W$ such that $g$ is continuously differentiable, $f(x, g(x)) = 0$
  for all $x \in V$ and $f(x,y) = 0$ if and only if $y = g(x)$ for all
  $(x,y) \in V \times W$.
\end{thm}
\begin{proof}
First define the map $g : U \to X \times Z$ by $g(x,y) = (x,
f(x,y))$.  We claim that $g$ has a local inverse at $(x_0,y_0)$.  To
see this we compute
\begin{align*}
Dg(x_0,y_0) 
\end{align*}
TODO:
\end{proof}

\subsection{Optimization in Banach Spaces}
As one will undoubtedly remember from one's first calculus class the
derivative is an extraordinarily useful tool for finding maxima and
minima of functions of a real variable.  Essentially all of that
theory carries over to the case of general Banach space domains.  One
of the goals in this subsection is to develop that basic theory.  

In a multivariate calculus course the reader almost certainly
encounterd problems of constrained optimization as well: learning the
tool of the Lagrange multiplier for solving such problems. This theory
also carries over to the Banach space setting and we develop it here.

What one has learned up to this point is the theory of equality
constrained optimization.  Though it tends not to be taught in the
introductory
calculus curriculum, in applications it is equally important to be able to
solve optimization with both equality and inequality constraints.
Having the tools we have developed it is no harder to develop such
theory in the general Banach space setting and we do so here.

We will discuss optimization problems in terms of minimization; as a
general rule there is no loss of generality in doing so as
maximization of a function $f$ may be performed by minimizing the
function $-f$.  

First we distinguish the different kinds of minima that we may
characterize.  The primary distinction is the dichotomy between local
and global minimization.  There are subtler distinctions to be made
between different type of local minima.  The definitions make sense
for arbitrary topological spaces.

\begin{defn}Let $X$ be a topological space and let $f : X \to \reals$
  be a function.  We say that $x^* \in X$ is a \emph{global minimizer}
  of $f$ if $f(x^*) \leq f(x)$ for all $x \in X$.  We say that $x^*
  \in X$ is a \emph{local minimizer} if there exists an open set $U
  \subset X$ such that $x^* \in U$ and $f(x^*) \leq f(x)$ for all $x
  \in U$.  We say that $x^*
  \in X$ is a \emph{strict local minimizer} if there exists an open set $U
  \subset X$ such that $x^* \in U$ and $f(x^*) < f(x)$ for all $x
  \in U$ with $x^* \neq x$.  We say that $x^* \in X$ is an
  \emph{isolated local minimizer} if there exists and open set $U
  \subset X$ such that $x^* \in U$ and $x^*$ is the only local
  minimizer in $U$.
\end{defn}

\begin{examp}Let 
\begin{align*}
f(x) &= \begin{cases}
x^4 \cos(1/x) + 2x^4 & \text{if $x \neq 0$} \\
0 & \text{if $x = 0$}
\end{cases}
\end{align*}
then $0$ is a strict local minimizer that is not isolated.  TODO: Show this
\end{examp}

Note that minimizers are not guaranteed to exist since functions may
be unbounded below.  More subtely may have a lower bound but may never
take the value of its greatest lower bound.  We already know a case in
which both of these problems are avoided: namely continuous images of
compact sets are compact in $\reals$ and this guarantees the exists of
a miminum (Theorem \ref{ContinuousImageOfCompact}).  As it turns out
this fact can be generalized a bit by relaxing the property of
continuity.

\begin{defn}Let $X$ be a topological space and let $f : X \to \reals$
  be a function, we say that $f$ is \emph{lower semicontinuous}
  (resp. \emph{upper semicontinuous}) if for every
  $\epsilon > 0$ there exists an open set $U$ containing $x$ such that
  $f(y) \geq f(x) - \epsilon$ (resp. $f(y) \leq f(x) + \epsilon$) for
  every $y \in U$. We say that $f$ is \emph{sequentially lower
    semicontinuous} (resp. \emph{sequentially upper semicontinuous})
  at $x$ if for every sequence $x_n$ such that $\lim_{n \to \infty}
  x_n = x$ we have $f(x) \leq \liminf_{n \to \infty} f(x_n)$ (resp. $f(x) \geq \limsup_{n \to \infty} f(x_n)$).
\end{defn}

It is simple to see that $f$ is lower semicontinuous
(resp. sequentially lower semicontinuous) if and only if $-f$ is upper
semicontinuous (resp. sequentially upper semicontinuous).

A function is lower (resp. upper) semicontinuous at $x$ if its values near $x$ are
either close to $f(x)$ or larger (resp. smaller) than $f(x)$.  In
general sequential semicontinuity is a weaker property than
semicontinuity (since sequences do not characterize convergence in
general topological spaces); however in metric spaces the two concepts
are equivalent.

\begin{prop}Let $X$ be a topological space and let $f$ be a lower
  (resp. upper) semicontinuous at $x$, then $f$ is sequentially
  lower (resp. upper) semicontinuous at $x$.  If $X$ is a metric space and
  $f$ is sequentially lower (resp. upper) semicontinuous at $x$ then $f$ is
  lower (resp. upper) semicontinuous at $x$.
\end{prop}
\begin{proof}If suffices to handle the cases of lower semicontinuity
  since the upper semicontinuity results follow by applying the lower
  semicontinuity case to $-f$.

If $f$ is lower semicontinuous and $x_n \to x$.  Let $\epsilon > 0$ be
given an find an open neighborhood $U$ of $x$ such that $f(y) \geq
f(x) - \epsilon$ for all $y \in U$.  Since $x_n \to x$ we know that
there exists $N > 0$  such that $x_n \in U$ for $n \geq N$ and thus
$\inf_{m \geq n} f(x_m) \geq f(x) - \epsilon$ for every $n \geq N$.
We take the limit at $n \to \infty$ to get $\liminf_{n \to \infty}
f(x_n) \geq f(x) - \epsilon$.  Since $\epsilon > 0$ was arbitrary we
conclude that $f$ is sequentially lower semicontinuous at $x$.

Now let $X$ be a metric space and suppose that $f$ is not
lower semicontinuous at $x$.  Then there exists an $\epsilon > 0$ such
that for every $n \in \naturals$ there exists $x_n$ with $d(x,x_n) <
1/n$ such that $f(x_n) < f(x) - \epsilon$.  Clearly $x_n \to x$ and
moreover $\liminf_{n \to \infty} f(x_n) \leq f(x) - \epsilon < f(x)$
which shows that $f$ is not sequentially lower semicontinuous at $x$.
\end{proof}

Compactness and sequential lower semicontinuity suffice to show that
a function has a global minimizer.
\begin{thm}Let $X$ be a topological space, let $f : X \to [-\infty,
  \infty]$ be a sequentially lower
  semicontinuous function and suppose that there exists $M \in \reals$
  such that $\lbrace x \in X \mid f(x) \leq M \rbrace$ is non-empty
  and compact then
  $f$ has a global minimizer.  Moreover if $f$ is lower
  semicontinuous, the set of global minimizers
  is compact.
\end{thm}
\begin{proof}
We first show the existence of a global minimizer.  Let $\alpha = \inf_{x \in X} f(x)$ and note that we know $\alpha \leq
M < \infty$.  If $\alpha = M$ then in fact $\lbrace x \in X \mid f(x)
\leq M \rbrace  = \lbrace x \in X \mid f(x) = M \rbrace $ and is assumed
non-empty and we are done.  Therefore we assume that $\alpha < M$.  Let $x_n$ be chosen so that
$\lim_{n \to \infty} f(x_n) = \alpha$ (if $\alpha > -\infty$ then
choose $x_n$ such that $f(x_n) < \alpha + 1/n$ otherwise choose $x_n$
so that $f(x_n) \leq -n$).  As $\alpha < M$ we know that there exists
$N>0$ such that $f(x_n) \leq M$ for all $n \geq N$ and therefore by
compactness there exists a convergent subsequence $x_{n_j}$.  Let $x =
\lim_{j \to \infty} x_{n_j}$ and note that by sequential lower semicontinuity
at $x$
\begin{align*}
\alpha \leq f(x) \leq \liminf_{j \to \infty} f(x_{n_j}) = \alpha
\end{align*}
which shows that $f(x) = \alpha$.

Let $G = \lbrace x \in X \mid f(x) = \alpha \rbrace$.  Now we know
that since $\alpha \leq M$  that $G \subset \lbrace x \in X \mid f(x)
\leq M \rbrace$ hence it suffices to show that the set of global
minimizers is closed (Corollary \ref{ClosedSubsetsCompact}).  Let $x$
be in $\overline{G}$ and let $\epsilon > 0$ be given.  Since $f$ is
lower semicontinuous we can find an open set $U$ containing $x$ such
that $f(y) \geq f(x) - \epsilon$.  However we know that $G \cap U \neq
\emptyset$ therefore $\alpha \geq f(x) - \epsilon$.  Since $\epsilon$
is arbitrary we conclude $\alpha \geq f(x)$ and thus $x \in G$.  
\end{proof}


Given that derivatives are determined by the behavior of functions on
arbitrarily small neighborhoods of a point it is clear that they have
little to say about when a point is a global minimizer.  On the other
hand derivatives are rather informative about local minimizers and we
turn our attention to this.

\subsubsection{Unconstrained Optimization}

The first thing to do is to note that there are \emph{necessary}
conditions for a point being a local minimizer that are described by
derivatives.  The first such is the vanishing of the first derivative.

\begin{thm}\label{VanishingFirstDerivativeAtLocalMinimum}Let $X$ be a
  Banach space, let $f : X \to \reals$ be a function and let $x^*$ be
  a local minimum.  If $f$ is $C^1$ on an open neighborhood of $x^*$
  then $Df(x^*) = 0$.
\end{thm}
\begin{proof}
The proof is by contradiction.  Suppose that $Df(x^*) \neq 0$.  Thus
there exists $y \in X$ such that $Df(x^*)y > 0$.  Let $f$ be $C^1$ on
an open neighborhood $U$ of $x^*$.  By continuity of $Df(x)$ on $U$ we
may also find a $\delta > 0$ such that $Df(x) y > 0$ for all $x \in
B(x^*, \delta) \subset U$.  By multiplying $y$ by an
appropriate positive constant we may assume that $\norm{y} < \delta$.  Now we can
apply Taylor's Theorem to conclude that 
\begin{align*}
f(x^* - y) &= f(x^*) - \int_0^1 Df(x^*-ty) y \, dt < f(x^*)
\end{align*}
which shows that $x^*$ is not a local minimizer.

Here is an alternative proof that avoid appealing to Taylor's Theorem.
Find an open ball $\delta > 0$ such that $f(x^*) \leq f(y)$
for all $y$ with $\norm{y - x} < \delta$.  Pick an arbitrary $y \in X$
then for all $0 < t < \delta/\norm{y}$ we have $f(x^* + t y) - f(x^*)
\geq 0$ which implies 
\begin{align*}
Df(x) y &= \lim_{t \to 0} \frac{f(x^* + ty) - f(x^*)}{t} \geq 0
\end{align*}
On the other hand applying the argument to $-y$ and using linearity of
$Df(x)$ show that $Df(x) y = -Df(x) (-y) \leq 0$ and therefore $Df(x)y
= 0$.
\end{proof}

When $f$ has two derivatives then we can say even more.
\begin{thm}\label{PositiveSemidefiniteSecondDerivativeAtLocalMinimum}Let $X$ be a
  Banach space, let $f : X \to \reals$ be a function and let $x^*$ be
  a local minimum.  If $f$ is $C^2$ on an open neighborhood of $x^*$
  then $Df(x^*) = 0$ and $D^2f(x^*)$ is positive semidefinite
  (i.e. $D^2f(x^*) (v,v) \geq 0$ for all $v \in X$).
\end{thm}
\begin{proof}
Again we proceed by contradiction.  Suppose that $D^2f(x^*)(v,v) < 0$.
By continuity of $D^2f$ we may find a $\delta > 0$ such that $D^2f(x)(v,v) < 0$ for
all $x \in B(x^*,\delta) \subset U$.  If necessary multiply $v$ be a
small positive constant to guarantee that $\norm{v} < \delta$.  By Theorem \ref{VanishingFirstDerivativeAtLocalMinimum} we know that
$Df(x^*) = 0$ so Taylor's Theorem says
\begin{align*}
f(x^* + v) &= f(x^*) + \int_0^1 (1 - t) D^2f(x^* + tv) (v,v) \, dt < f(x^*)
\end{align*}
which is a contradiction.
\end{proof}

When $f$ has two derivatives there also exists sufficient conditions
that a point be a local minimizer.
\begin{thm}\label{LocalMinimumAtPositiveDefiniteSecondDerivative}Let $X$ be a
  Banach space, let $f : X \to \reals$ be a function and suppose $f$
  is $C^2$ on an open neighborhood $U$ of $x^*$.
  If $Df(x^*) = 0$ and $D^2f(x^*)$ is positive definite
  (i.e. there exists an $\alpha > 0$ such that $D^2f(x^*) (v,v) >
  \alpha \norm{v}^2$ for all $v \in X$ with $v \neq 0$) then
  $x^*$ is a strict local minimizer of $f$.
\end{thm}
\begin{proof}
Using continuity of $D^2f$ at $x^*$ we may find a $\delta > 0$ such
that $B(x^*,
\delta) \subset U$ and $\norm{D^2f(x^* + y) - D^2f(x^*)}
< \frac{\alpha}{2}$ for all $\norm{y} < \delta$.  Note in particular
that 
\begin{align*}
(D^2f(x^* + y) - D^2f(x^*) )(v,w) &> -\frac{\alpha \norm{v}
  \norm{w}}{2} \text{ for all $\norm{y} < \delta$ and $v,w \in X$}
\end{align*}  
By Taylor's Theorem, 
for all $y$ with $\norm{y} < \delta$
\begin{align*}
f(x^* + y) - f(x) &= 
\int_0^t (1-t) D^2f(x^* + ty)(y,y) \, dt \\
&=\frac{1}{2} D^2f(x^*)(y,y) + \int_0^t (1-t) (D^2f(x^* + ty)
  -D^2f(x^*)) (y,y) \, dt  \\
&\geq \frac{\alpha \norm{y}^2}{2} - \frac{\sup_{0 \leq t \leq 1}
  \norm{D^2f(x^* + ty) - D^2f(x^*)} \norm{y}^2}{2} \\
&\geq  \frac{\alpha
  \norm{y}^2}{4} > 0\\
\end{align*}
which shows that $x^*$ is a local minimizer.
\end{proof}

Note that in finite dimensions the condition of positive definiteness
is equivalent to the apparently weaker condition $D^2f(x^*) (v,v) > 0$
for all $v \neq 0$.

\subsubsection{Constrained Optimization}

We first consider an abstract version of constrained
optimization.  Let $X$ be a Banach space and consider a function $f :
X \to \reals$ then given a closed set $F \subset X$ we can consider
the problem of finding a minimizer of $f$ restricted to $F$.  Note
that the meaning of finding a constrained minimizer is captured by
using our existing definitions of minimizers on the space $F$ with the
relative topology.

In order
to apply derivatives to the problem of characterizing minimizers on
$F$ we need to restrict them to directions that don't leave $F$;
if we have a point $x \in F$ and $f$ is decreasing at $x$ in a direction that immediately takes one out of
$F$ then that alone won't means that $x$ isn't a minimizer when
restricted to $F$.  This leads us to a definition of direction tangent
to a closed set $F$.  Note that if a direction is tangent to a set at
a point then any positive multiple should be tangent (though if the
set has corners then negative multiples may fail to be tangents;
consider the behavior of $\abs{x}$ at the origin).  As a result of
this observation we should be seeking to characterize a cone of
tangent directions.  
\begin{defn}Let $X$ be a Banach space, let $F$ be a closed subset and
  let $x \in F$ then we say that $v \in X$ is a \emph{tangent vector to $F$
  at $x$} if there is a sequence $x_n$ such that $x_n \in F$ and
  $\lim_{n \to \infty} x_n = x$ and a sequence of positive real numbers
  $t_n$ such that $\lim_{n \to \infty} t_n = 0$ that together satisfy
\begin{align*}
\lim_{n \to \infty} \frac{x_n - x}{t_n} = v
\end{align*}
The set $T_F(x)$ of all tangent vectors to $F$ at $x$ is called the
\emph{tangent cone to $F$ at $x$}.
\end{defn}

We call out the fact that the tangent cone is in fact a cone.
\begin{prop}The tangent cone $T_F(x)$ is a cone (i.e. for every
  $\alpha \geq 0$ and $v \in T_F(x)$ we have $\alpha v \in T_F(x)$).
\end{prop}
\begin{proof}
It is trivial to see that $0 \in T_F(x)$ since we can just pick the
$x_n \equiv x$.  Let $v \in T_F(x)$, $\alpha > 0$ and pick sequences $x_n$ and $t_n$ such that $x_n \to x$, $t_n \to 0$ and
$\frac{x_n - x}{t_n} = v$.  Then let $\tilde{t}_n = t_n/\alpha$ and note that
$\tilde{t}_n \to 0$ and $\frac{x_n -x}{\tilde{t}_n} = \alpha v$.
\end{proof}

The first hint that we have the correct notion of tangent vector is
the following necessary condition for a local minimizer to exist.

A few facts about Landau notation.
\begin{defn}Let $X$, $Y$ and $Z$ be Banach spaces.  Let $x_n$ be a sequence in $X$ and let
  $y_n$ be a sequence in  $Y$ we say that $x_n = o(y_n)$
  as $n \to \infty$ if $\lim_{n \to \infty} \frac{x_n}{\norm{y_n}} =
  0$ (equivalently $\lim_{n \to \infty} \frac{\norm{x_n}}{\norm{y_n}} =
  0$).  We say that $x_n = O(y_n)$ if there exists $M > 0$ and $N \geq
  0$ such that
  $\frac{\norm{x_n}}{\norm{y_n}}  \leq M$ for all $n \geq N$.  
Given functions $f : X \to Y$, $g : X \to Z$ and $x_0 \in X$ we say that
  $f(x)$ is $o(g(x))$ as $x \to x_0$ if $\lim_{x \to x_0}
  \frac{f(x)}{\norm{g(x)}} = 0$ (equivalently $\lim_{x \to x_0}
  \frac{\norm{f(x)}}{\norm{g(x)}} = 0$) and we say that $f(x)$ is
  $O(g(x))$ if there exists $M > 0$ and $\delta > 0$ such that
  $\frac{\norm{f(x)}}{\norm{g(x)}} \leq M$ for all $\norm{x - x_0} < \delta$.
\end{defn}
Because the definitions above really only depend on the norms of the
sequences and functions in question, it is often useful to say that a
sequence $x_n \in X$ is $o(\norm{y_n})$ or a function $f(x)$ is
$o(\norm{g(x)})$.  It is also worth pointing out that Landau notation
is confusing for uninitiated in large part because of its abuse of the
equality sign.  

\begin{prop}\label{LandauNotationIdentities}The following are true:
\begin{itemize}
\item[(i)] $o(y_n) + o(y_n) = o(y_n)$.
\item[(ii)] Suppose $z_n = O(y_n)$ then if $x_n = o(z_n)$ it follows
  that $x_n = o(y_n)$.  In shorthand we say that $o(O(y_n)) = o(y_n)$.
\end{itemize}
\end{prop}
\begin{proof}
(i) follows from linearity: if $x_n = o(y_n)$ and $z_n = o(y_n)$ then
it follows that 
\begin{align*}
\lim_{n \to \infty} \frac{x_n + z_n}{\norm{y_n}} &= 
\lim_{n \to \infty} \frac{x_n}{\norm{y_n}}
+ \lim_{n \to \infty} \frac{z_n}{\norm{y_n}}
= 0
\end{align*}
To see (ii), we know that $\lim_{n \to \infty} \frac{\norm{x_n}}{\norm{z_n}} =
0$ and there exist $M,N \geq 0$ such that $\norm{z_n} \leq M
\norm{y_n}$ for all $n \geq N$, therefore 
\begin{align*}
0 &\leq \lim_{n \to \infty} \frac{\norm{x_n}}{\norm{y_n}} \leq M \lim_{n \to \infty}
\frac{\norm{x_n}}{\norm{z_n}} = 0
\end{align*}
\end{proof}


\begin{thm}\label{LocalConstrainedMinimizerFirstDerivative}Let $X$ be a Banach space, $F \subset X$ be closed and let
  $f : U  \to \reals$ be $C^1$ on an open set $U \supset F$.  Then if
  $x^*$ is a local minimizer of $f$ on $F$ we have $Df(x^*) v \leq 0$
  for all $v \in T_F(x^*)$.  
\end{thm}
\begin{proof}
Suppose that we have $x_n \in F$ with $x_n \to x$, $t_n>0$ with $t_n \to 0$
and $(x_n - x)/t_n \to v$.  By Taylor's Theorem and the fact that
$x^*$ is a local minimizer we know that we
can find a neighborhood $x^* \subset V \subset U$ such that
\begin{align*}
f(y)  - f(x^*) = Df(x^*) (y - x) + o(\norm{y - x^*}) \leq 0
\end{align*}
and for $y \in V$.  From the fact that $x_n \to x$ we can find an $N \in \naturals$ such
that $x_n \in V$ for all $n \geq N$.  Since $(x_n - x)/t_n \to v$ we
know that $\norm{x_n -x^*}$ is $O(t_n)$ and therefore $o(\norm{x_n -
  x})$ is $o(t_n)$ (Proposition \label{LandauNotationIdentities}) and since $x_n - x -t_nv$ is $o(t_n)$ we have
\begin{align*}
t_n Df(x^*) v + o(t_n) &= f(x_n) - f(x^*) \leq 0
\end{align*}
which implies that $Df(x^*) v \leq 0$ (divide by $t_n>0$ and let $n
\to \infty$).
\end{proof}

TODO: Define the normal cone(s)...
\begin{defn}Let $X$ be a Hilbert space, $F \subset X$ be closed and let
  $x \in F$ then we say that $v \in X$ is a \emph{regular normal
    vector to $F$ at $x$} if $\langle v, w \rangle \leq 0$ for all $w
  \in T_F(x)$.  The set of all regular normal vectors to $F$ at $x$ is
 called the \emph{regular normal cone} and is denoted
 $\widehat{N}_F(x)$.  We say that $v \in X$ is a \emph{limiting normal vector to $F$
    at $x$} or simply a \emph{normal vector to $F$
    at $x$} if there are sequences $x_n$, $v_n$ with $v_n$ a regular
  normal vector to $F$ at $x_n$ and $\lim_{n \to \infty} x_n = x$ and $\lim_{n \to \infty}
  v_n =v$.
\end{defn}

Facts:

The proximal normal cone is a convex cone but may not be closed
The limiting normal cone is a closed cone but may not be convex.
The limiting normal cone may be defined as the limit of proximal
normal vectors as well as by using regular normal vectors.

\begin{defn}Let $X$ be a Hilbert space, $F \subset X$ be closed and let
  $x \in F$ we say that $F$ is \emph{Clarke regular at $x$} if $F$ is locally closed at $x$ and
$\widehat{N}_F(x) = N_F(x)$.
\end{defn}

\begin{prop}\label{RegularNormalLittleO}$\widehat{N}_F(x)$ is a closed convex cone.  Moreover, $v$ is a regular normal to $F$ at $x$ if and only if
  $\langle v, y-x \rangle \leq o(y -x)$ for $y \in F$ and $x \neq y$.  That is to say
  for every sequence $x_n \in F$ with $x_n \neq x$ and $\lim_{n \to \infty} x_n = x$, we have 
\begin{align*}
\limsup_{n \to \infty} \frac{\langle v, x_n-x \rangle}{\norm{x_n - x}}
&\leq 0
\end{align*}
\end{prop}
\begin{proof}
The fact that $\widehat{N}_F(x)$ is a closed convex cone follows from
the fact that is is an intersection of closed halfspaces.

Suppose that there exists an sequence $x_n \in F$ with $x_n \neq x$,
$x_n \to x$ and 
\begin{align*}
\limsup_{n \to \infty} \frac{\langle v, x_n-x \rangle}{\norm{x_n - x}}
&> 0
\end{align*}
By passing to a subsequence we may assume that the $\limsup$ may be
replaced by a limit.  Let $w_n = \frac{x_n - x}{\norm{x_n - x}}$ and
  by compactness we may pass to a further
subsequence and assume that $w_n$ converges
  to a unit vector $w$ and $\langle v,w \rangle = \lim_{n \to \infty} \langle v,w_n \rangle > 0$. 
On the other hand, $w$ is seen to be a tangent vector to $F$ at $x$ because it is
the limit $\frac{x_n - x}{\norm{x_n - x}}$ and $\norm{x_n -x} \to 0$.

Now suppose that $\langle v, y-x \rangle \leq o(y -x)$ and let $w \in T_F(x)$.  Pick a defining
sequence $x_n \to x$ with $x_n \in F$ and $t_n \downarrow 0$ such that $\lim_{n \to \infty} \frac{x_n - x}{t_n} = w$.
\begin{align*}
\langle v,w \rangle &= \lim_{n \to \infty} \langle v, \frac{x_n - x}{t_n} \rangle \\
&\leq \limsup_{n \to \infty} \langle v, \frac{x_n - x}{\norm{x_n- x}} \rangle \lim_{n \to \infty} \frac{\norm{x_n - x}}{t_n} \\
&= \norm{w} \limsup_{n \to \infty} \langle v, \frac{x_n - x}{\norm{x_n- x}} \rangle \leq 0\\
\end{align*}
\end{proof}

\begin{thm}$v$ is a regular normal to $F$ at $x$ if and only if there exists a function $f$ differentiable at $x$ such that 
$f$ has a local minimum on $F$ at $x$ and $\nabla f (x) = v$.  In fact, we can choose $f$ be differentiable on all of $\reals^n$ 
with a global minimum at $x$.
\end{thm}
\begin{proof}
TODO:  
Let $v \in \widehat{N}_F(x)$ be given.  The first step to building $f$ is to define
\begin{align*}
\theta_0(r) = \sup \lbrace \langle v, y - x \rangle \mid y \in F \text{ and } \norm{y-x} \leq r \rbrace
\end{align*}
It is simple to see that $\theta_0$ is a non-decreasing function of $r$,  $\theta_0(0) = 0$ and
$\theta_0(r) \leq \norm{v} r$ so $\lim_{r \downarrow 0} \theta_0(r) = 0$. 
Moreover from Proposition \ref{RegularNormalLittleO} we get for any $\epsilon > 0$ there exists a $\delta > 0$ such that
$\langle v, y-x \rangle \leq \epsilon \norm{y-x}$ for $y \in F$ and $\norm{y-x} \leq \delta$.    Thus for $0 \leq r \leq \delta$, 
\begin{align*}
\theta_0(r)&= \sup \lbrace \langle v, y - x \rangle \mid y \in F \text{ and } \norm{y-x} \leq r \rbrace \\
&\leq \sup \lbrace \epsilon \norm{y-x} \mid y \in F \text{ and } \norm{y-x} \leq r \rbrace \\
&\leq \epsilon r \\
\end{align*}
and thus $\theta_0(r) = o(r)$.
Now let 
\begin{align*}
h_0(y) &= \langle v, y-x\rangle - \theta_0(\norm{y-x})
\end{align*}
and note since $\theta_0(r) = o(r)$,
\begin{align*}
\lim_{w \to 0} \frac{h_0(x+w) - h_0(x) - \langle v,w \rangle}{\norm{w}} =\lim_{w \to 0} \frac{-\theta_0(\norm{w})}{\norm{w}} =0\\
\end{align*}
which shows $h_0$ is differentiable at $x$ and moreover $\nabla h_0(x) = v$.
\end{proof}

\begin{defn}Let $X$ be a Hilbert space, $F \subset X$ be closed and let
  $x \in F$ then we say that $v \in X$ is a \emph{proximal normal
    vector to $F$ at $x$} if there exists an $M \geq 0$ such that
  $\langle v, y - x \rangle \leq M \norm{y - x}$ for all $y \in F$.
\end{defn}

The following is (supposed to be) the interpretation of a proximal
normal vector: $v$ is a proximal normal vector $F$ at $x$ if there
exists a $y \in X$ such that $x$ is the closest point in $F$ to $y$
and there exists $c > 0$ such that $v = c(y -x)$.  Also an interpretation is that
$v$ is a proximal normal when there exists $y$ and $c \geq 0$ with $v = cy$ and
a closed ball $B(y,r)$ with $B(y,r) \cap F = \lbrace x \rbrace$.

For computational purposes (and in particular for numerical
optimization problems) we are given a constraint set in some concrete
form rather than the abstract formulation we've used.  In practice it
is useful to formulate a constraint set using a combination of
equalities and inequalities.  For the moment we specialize to the case
of finite dimensions.  Let $X$ be a finite dimensional Banach space
(i.e. $\reals^n$) and suppose we are given finite sets $\mathcal{E}$
(the \emph{equality constraints})
and $\mathcal{I}$ (the \emph{inequality constraints}) and for each $i
\in \mathcal{E} \cup \mathcal{I}$ we have a $C^1$ 
function $c_i : X \to \reals$.  Let $f : X \to \reals$ be a $C^1$
function and we consider the constrained minimization problem for $f$
with constraint set 
\begin{align*}
F &= \lbrace x \in X \mid c_i(x) = 0 \text{ for all $i \in
    \mathcal{E}$ and } c_i(x) \geq 0 \text{ for all $i \in
    \mathcal{I}$} \rbrace
\end{align*}
It is clear from the continuity of the $c_i(x)$ that $F$ is closed and
therefore we have the first order necessary condition of Theorem
\ref{LocalConstrainedMinimizerFirstDerivative} for local minimizers of
$f$ restricted to $F$.  What we seek are
conditions in terms of $f$ and the $c_i$ that are implied by the
conditions in Theorem \ref{LocalConstrainedMinimizerFirstDerivative};
for that we need to understand how $T_F(x)$ might be expressed in
terms of $f$ and the $c_i$.  To that end, we first have the following
definitions.
\begin{defn}Given a Banach space $X$, disjoint sets $\mathcal{E}$ and
  $\mathcal{I}$, functions $c_i : X \to \reals$ for each $i \in
  \mathcal{E} \cup \mathcal{I}$ and the set
\begin{align*}
F &= \lbrace x \in X \mid c_i(x) = 0 \text{ for all $i \in
    \mathcal{E}$ and } c_i(x) \geq 0 \text{ for all $i \in
    \mathcal{I}$} \rbrace
\end{align*}
we say that a constraint $c_i$ is \emph{active} at $x \in F$ if either
$i \in \mathcal{E}$ or $i \in \mathcal{I}$ and $c_i(x) = 0$.  For each
$x \in F$ we let the \emph{active constraint set} be 
\begin{align*}
\mathcal{A}(x) &= \lbrace i \in \mathcal{E} \cup \mathcal{I} \mid i
                 \text{ is active at $x$} \rbrace
\end{align*}
Assume that the $c_i$ are continuously differentiable, then the set of \emph{linearized feasible directions at $x$} is defined to
be 
\begin{equation*}
\mathcal{F}(x) =
\left \{ v \in X \mid
\begin{aligned}
Dc_i(x) v = 0 & \text{ for all $i \in \mathcal{E}$} \\
Dc_i(x) v \geq 0 & \text{ for all $i \in \mathcal{A}(x) \cap \mathcal{I}$} \\
\end{aligned}
\right \}
\end{equation*}
\end{defn}
Note that it is trivial to see that $\mathcal{F}(x)$ is a cone.  The
first thing is to note that every tangent vector is a linearized
feasible direction.

\begin{prop}$T_F(x) \subset \mathcal{F}(x)$.
\end{prop}
\begin{proof}
Let $v \in T_F(x)$ and pick a feasible sequence $x_n \to x$ and
sequence of positive numbers $t_n
\to 0$ such that $x - x_n = t_nv + o(t_n)$.  Applying Taylor's Theorem
we can conclude that 
\begin{align*}
c_i(x_n) &= c_i(x) + Dc_i(x) (x_n - x) + o(\norm{x_n-x}) \\
&= c_i(x) + t_n Dc_i(x) v + o(t_n) 
\end{align*}
so if $i \in \mathcal{A}(x)$ we have $c_i(x_n) = t_n Dc_i(x) v +
o(t_n)$. Dividing by $t_n$ and taking the limit as $n \to
\infty$ we get $Dc_i(x) v = \lim_{n \to \infty}
\frac{c_i(x_n)}{t_n}$.  Thus it follows that  $i \in \mathcal{E}$
implies $Df(x) v = 0$ and $i \in \mathcal{A}(x) \cap \mathcal{I}$
implies $Df(x) v \geq 0$.
\end{proof}
It is not true in general that $T_F(x) = \mathcal{F}(x)$ yet the
result that we want to demonstrate requires that this equality holds.
A set of conditions that we place on the $c_i$ that guarantees such an
equality is called a \emph{constraint qualification}; more generally a
constraint qualification may a bit weaker than that and simply imply
that $T_F(x)$ and $\mathcal{F}(x)$ aren't too different.  There are a
variety of choices of constraint qualifications we state a
conceptually straightforward and useful one.
\begin{defn}A set of constraints $c_i$ satisfies the \emph{linearly
    independent constraint qualification (LICQ)} at $x$ if the set of
  derivatives $\lbrace Dc_i(x) \rbrace$ for $i \in \mathcal{A}(x)$ is
  linearly independent in $X^*$.
\end{defn}
The LICQ is a sufficient criterion for the equality of the tangent
cone and the linearized feasible set.

\begin{examp}Consider the set $F \subset \reals^2$ defined by the
  constraints $c_1(x,y) = 1 - x^2 - (y-1)^2 \geq 0$ and $c_2(x,y) =
  -y \geq 0$.

TODO: Show that $T_F(x)$ is a strict subset of $\mathcal{F}(x)$.
\end{examp}

\begin{prop}Let $F$ be defined by a set of constraints $c_i(x)$ which
  satisfy the LICQ at $x$ then $T_F(x) = \mathcal{F}(x)$.
\end{prop}
\begin{proof}
Let $v \in \mathcal{F}(x)$, we need to show that $v$ is a tangent
vector producing a sequence $x_n \in F$ and $t_n > 0$ such that $v = x_n + o(t_k)$. 
By assumption the set of derivatives $Dc_i(x)$ for $i \in
\mathcal{A}(x)$ is linearly independent hence is a basis for the linear 
span $V = \lbrace Dc_i(x) \rbrace_{i \in \mathcal{A}(x)}$.  Let $m$ be the cardinality of $\mathcal{A}(x)$ and $c : X \to \reals^m$ 
be defined by $c(y) = (c_{i_1}(y), \dotsc, c_{i_m}(y))$ where $\lbrace i_1, \dotsc, i_m \rbrace = \mathcal{A}(x)$.
Take the orthogonal complement $W$ of $V$ 
in $X^*$ and pick a basis $w_j$ for $W$ and let $w : X \to \reals^{n-m}$ be defined by $w(y) = (w_1(y), \dotsc, w_{n-m}(y))$.  Now define $R : X \times \reals \to X$ by
\begin{align*}
R(y,t) &= \begin{bmatrix}
c(y) - t Dc(x) v \\
w(y - x - tv)
\end{bmatrix}
\end{align*}
and note that $R(x,0) = 0$.  Moreover 
\begin{align*}
DR(x,0)(u, 0) &= \begin{bmatrix}
Dc(x) u \\
w(u)
\end{bmatrix}
\end{align*}
which is invertible by construction of $w$.  Now we can apply the Implicit Function Theorem \ref{ImplicitFunctionTheorem} to 
conclude that there exists and $\epsilon > 0$ and a function $f : (-\epsilon, \epsilon) \to X$ such that $f(0) = x$, $R(f(t), t) = 0$ 
for all $-\epsilon < t < \epsilon$ and moreover $f(t)$ is the unique solution to the equation $R(x,t) = 0$.  In addition note that
since we have assumed that $v \in \mathcal{F}(x)$ we have from $R(f(t), t) = 0$,
\begin{align*}
c_i(f(t)) &= t Dc_i(x) v = 0 \text{ for $i \in \mathcal{E}$} \\
c_i(f(t)) &= t Dc_i(x) v >= 0 \text{ for $t > 0$ and $i \in \mathcal{A}(x)\cap \mathcal{I}$} \\
\end{align*}
and moreover by continuity we have $c_i(f(t)) > 0$ for $i \in \mathcal{I} \setminus \mathcal{A}(x)$ (here we may need to shrink $\epsilon$ for this to be true.  Thus we have $f(t) \in F$.  



Now pick any sequence $0 < t_n < \epsilon$ with $\lim_{n \to \infty} t_n = 0$ and define $x_n = f(t_n)$; by continuity of $f$ and and the 
fact that $f(0) = x$ we have $x_n \to x$.  If we Taylor expand $R(y,t)$ around $(x,0)$ we get
\begin{align*}
0 &= f(x_n, t_n) = \begin{bmatrix}
Dc(x) (x_n - x - t_n v) \\
w(x_n - x - t_n v)
\end{bmatrix}
+ o(\norm{(x_n,t_n) - (x,0)})
\end{align*}
Since $x_n = f(t_n)$ and $f$ is differentiable it follows that $x_n - x = O(t_n)$ and therefore $o(\norm{(x_n,t_n) - (x,0)}) = o(t_n)$.  Thus by
considering the first component of the vector we have $Dc(x)  (x_n - x - t_n v)  = o(t_n)$ and since 
$Dc(x)$ is invertible we get that $x_n - x -t_n v = o(t_n)$ which shows that $v \in T_F(x)$.
\end{proof}

\subsubsection{Algorithms for Unconstrained Optimization}

We have developed criteria for detecting minimizers (mostly local)
however we have not yet addressed the issue of how we might find one.
There are two basic paradigms to consider: line search and trust
region methods.  We first consider line search.

For motivation we give an interpretation of the Frechet derivative of
a real valued function on a Hilbert space $X$.  Since $Df(x)$ is a
bounded linear functional on $X$, we know by Reisz
representation that there is a unique element of $X$ representing the
functional.

\begin{defn}Let $X$ be a Hilbert space, $U \subset X$ be open and let
  $f : U \to \reals$ be differentiable at $x \in U$.  The
  \emph{gradient of $f$ at $x$} is the unique element $\nabla f(x)$ of
  $X$ such that $\langle \nabla f(x), v \rangle = Df(x) v$ for all $v
  \in X$.
\end{defn}

We now proceed to interpret the vector $-\nabla f(x)$ as the
direction of steepest decrease of the function $f$.  Suppose
that we are at a point $x$ for which $\nabla f(x) \neq 0$.  To see
this, let $v \in X$ be an arbitrary unit vector in $X$ and consider the function of a single
real variable $g_v(t) = f(x + tv)$.  The question we ask is what is the
direction $v$ along which $f$ is decreasing the fastest at $x$.  By the Chain Rule, the definition of the
gradient and Taylor's Theorem we can write
\begin{align*}
f(x + tv) = f(x) + t \langle \nabla f(x), v \rangle + o(t)
\end{align*}
which implies that $g_v^\prime(0) = \langle \nabla f(x), v \rangle$.  So
what we want is to find the unit vector $v$ which minimizes the value
of $g_v^\prime(0)$.  We can write $v = \alpha \nabla f(x)/\norm{\nabla
  f(x)} + w$ where $\langle \nabla f(x), w \rangle = 0$.  Note that on
the one hand $\langle \nabla f(x), v \rangle = \alpha/\norm{\nabla
  f(x)}$ and on the other hand from $\norm{v} = 1$ we see that $-1
\leq \alpha \leq 1$.  Therefore it is clear that the minimum of
$g_v^\prime(0)$ occurs for $\alpha = -1$ which implies that $v = -
\nabla f(x) /\norm {\nabla f(x)}$. It is colloquial to say that the direction of
the gradient is the \emph{direction of steepest descent of $f$}.  Note
that the computation above shows that $g_v^\prime(0) < 0$ precisely
when $\langle \nabla f(x), v \rangle < 0$ which motivates the
following defintion
\begin{defn}Let $X$ be a Hilbert space, $U \subset X$ be open and let
  $f : U \to \reals$ be differentiable at $x \in U$ we say that $v \in
  X$ is a \emph{descent direction for $f$ at $x$} if $\langle \nabla
  f(x) , v\rangle < 0$.
\end{defn}


TODO: Discuss gradient flow in the Hilbert space and observe how the
solutions of the differential equation have limit points equal to the
stationary points of $f$.

Armed with the idea that when we are in possesion of derivatives we can
find directions along which the values of a function decreases, we
seek find an iterative algorithm for minimization.  The obvious idea
is that if at a given point $x_k$ we can find a descent direction
(e.g. the gradient $\nabla f(x_k)$ then we should move in that
direction and thereby expect that the function decreases.  There are
three problems to address about such an algorithm.  The first issue is
that the descent direction is characterized by an infinitessimal
condition and therefore there is no guarantee that a finite step in
that direction will result in a decrease in the function value.  The
second issue is that if our step sizes in the descent direction are
too small asymptotically we may never reach the minimum.  The third
issue is that if we choose a variable descent direction, the descent
direction may get increasing close to being orthogonal to the gradient
in which case function values may not decrease enough to converge
(note this is a non-issue is we choose the steepest descent direction).  We
seek conditions on the choice of step sizes and descent directions
that give us convergence to a stationary point of $f$.



\section{Linear Algebra}

This is a little refresher on linear algebra with more of a focus on matrix factorizations.

\begin{defn}Let $\mathds{F}$ be a field, then a \emph{vector space} over $\mathds{F}$ is an abelian group $(V,+,0)$ together with 
a multiplication operator $\mathds{F} \times V \to V$ such that $a (v + w) = av + aw$.
\end{defn}

\begin{defn}If $W \subset V$ is a vector space then we say that $W$ is a \emph{subspace} of $V$.  We say that a set of elements  $v_1, \dotsc, v_n$ of $V$ is \emph{linearly independent} if and only if $a_1 v_1 + \dotsb + a_n v_n = 0$ implies
$a_1 = \dotsb = a_n = 0$.  If a set is not linearly independent then we say it is  \emph{linearly dependent}.  The \emph{dimension} of a vector space $V$ is the supremum of cardinalities of linearly independent sets in $V$.  Given elements $v_\alpha$ in $V$ the \emph{linear span} is the intersection of all subspaces of $V$ containing the $v_\alpha$.  
\end{defn}

\begin{prop}Let $V$ be a vector space of $\mathds{F}$. 
\begin{itemize}
\item[(i)]Let $W_\alpha$ be a collection of subspaces of $V$, then $W = \cap_\alpha W_\alpha$ is a subspace.  
\item[(ii)] Let $v_\alpha$ be elements in $V$ then the span of $v_\alpha$ is a subspace and moreover the span is precisely the set of finite linear combinations of elements of $v_\alpha$.  
\item[(iii)]The dimension of the span of $\lbrace v_1, \dotsc, v_n \rbrace$ is less than or equal to $n$.  It is equal to $n$ if and only if $v_1, \dotsc, v_n$ are linearly independent.
\item[(iv)]$\dim \lbrace v_1, \dotsc, v_{n+1} \rbrace \leq \dim \lbrace v_1, \dotsc, v_n \rbrace + 1$.
\end{itemize}
\end{prop}
\begin{proof}
The fact that $W$ is a subspace is simple and left to the read.  The fact that a linear span is a subspaces follows from the first assertion as it was defined as an intersection of subspaces.  

To see that the set of finite linear combinations is indeed the span, first note that it is clear all finite linear combinations belong to every subspace containing the $v_\alpha$.  It suffices to show that the set of finite linear combinations is a vector space.  Given $u$ and $w$ which are finite linear combinations with index sets $A \subset \Lambda$ and $B \subset \Lambda$ respectively.  By use of zero coefficients we express both $u$ and $w$ using the index set $A \cup B$.  So to be concrete we write $u = \sum_{i=1}^n u_i v_{\alpha_i}$ and $w = \sum_{i=1}^n w_i v_{\alpha_i}$.  Let $a,b \in \mathds{F}$ and we compute $au + bw = \sum_{i=1}^n (au_i + bw_i) v_{\alpha_i}$.

To see (iii) we use induction.  For $n=1$ it is clear that $\dim\lbrace v_1 \rbrace = 1$ if $v_1 \neq 0$ and $0$ if $v_1 = 0$ and that $v_1 \neq 0$ if and only if $\lbrace v_1 \rbrace$ is linearly independent.  Suppose the result is true for $n-1$ and consider $v_1, \dots, v_n$.  Suppose $w_1, \dotsc, w_m$ is a linearly independent set in the span of the $v_i$.  Write
$w_i = \sum_{j=1}^n a_{ij} v_j$ for $1 \leq i \leq m$.  If $a_{in}=0$ for all $1 \leq i \leq n$

TODO: Finish
\end{proof}


\begin{defn}Let $V$ and $W$ be vector spaces over $\mathds{F}$ then a function $A : V \to W$ is said to be a \emph{linear map} if $A (av + bw) = aAv + bAw$ for all $v,w \in V$ and $a,b \in \mathds{F}$.  In the special case that $W = \mathds{F}$ we may say that $A$ is a \emph{linear functional}.  The set of linear functionals on $V$ is denoted $V^*$ and is called the \emph{dual space} to $V$.  Given any linear map $A : V \to W$ we define the \emph{dual map} $A^* : W^* \to V^*$ by $A^*(\lambda)(v)= \lambda(Av)$.  
\end{defn}
Note that the dual map is well defined since 
\begin{align*}
A^*(\lambda)(av + bw) &= \lambda(A(av+bw)) = \lambda(aAv + b A w) = a \lambda(Av) + b\lambda(Aw) = aA^*(\lambda)(v) + b A^*(\lambda)(w)
\end{align*}
shows that $A^*(\lambda)$ is a linear functional.  The dual space is easily seen to be a vector space and the dual map is easily shown to be a linear map.

\begin{prop}$V^*$ is a vector space over $\mathds{F}$ with addition and scalar multiplication defined pointwise as $(a\lambda + b\mu)(v) = a\lambda v + b \mu(v)$.  With repsect to this vector space structure the dual map $A^*$ is linear.  If $V$ is finite dimensional then $V^*$ is finite dimensional and $\dim V = \dim V^*$.
\end{prop}
\begin{proof}
The proof that $V^*$ is a vector space is elementary and left to the reader.  To see that $A^*$  is a linear map we just compute using the definitions
\begin{align*}
A^*( a \lambda + b \mu)(v) &= (a \lambda + b\mu)(Av) = a \lambda (Av) + b \mu (Av) = (a A^*(\lambda) + b A^*(\mu)) (v) \text{ for all $v \in V$}
\end{align*}

Now if $V$ is finite dimensional we can select a basis $v_1, \dotsc, v_n$.  Define $v_i^* \in V^*$ by $v_i^*(v_j) = \delta_{ij}$ for $1 \leq i,j \leq n$.  We claim that $v^*_i$ is a basis for $V^*$.  Clearly $v^*_i$ spans $V^*$ since if we are given an arbitrary $\lambda$ then by linearity
\begin{align*}
\left( \sum_{i=1}^n \lambda(v_i) v^*_i \right) v &= \sum_{i=1}^n \lambda(v_i) v^*_i (\sum_{j=1}^n a_j v_j) = \sum_{i=1}^n a_i \lambda(v_i) = \lambda(v)
\end{align*}
Moreover the $v^*_i$ are seen to be linearly independent since if $\sum_{i=1}^n a_i v^*_i = 0$ then for each $1 \leq j \leq n$ we have $0 = \left(\sum_{i=1}^n a_i v^*_i \right)v_j = a_j$.
Thus $v^*_i$ is a basis hence $\dim V  = \dim V^* = n$.
\end{proof}

The basis $v^*_i$ constructed from the basis $v_i$ in the above proof is referred to as the \emph{dual basis}.  

\begin{defn}Let $A$ be an $m \times n$ real matrix, then a triple comprising an $m \times m$ orthogonal matrix $U$, an $n \times n$ orthogonal matrix $V$ and a $m \times n$ diagonal matrix $\Sigma$ with $\Sigma_{11} \geq \dotsb \Sigma_{pp}$ where $p = m \wedge n$ such that $A = U \Sigma V^T$ is called a \emph{singular value decomposition}.
\end{defn}

\begin{thm}\label{SingularValueDecomposition}Singular value decompositions exist.
\end{thm}
\begin{proof}
The result is trivially true if $A = 0$ (let $\Sigma=0$, $U$ and $V$ be identity matrices), so assume that $A \neq 0$.  Let $\sigma_1$ be the $L^2$ operator norm of $A$.   By compactness of the unit sphere we can find a unit vector $x_1 \in \reals^n$ such that $0 \neq \sigma_1 = \norm{Ax}$.  Define $y_1 = \sigma^{-1}_1 Ax$ so that $y$ is a unit vector in $\reals^m$.  We can now find an orthonormal basis $\lbrace x_2, \dotsc, x_n \rbrace$ of $x_1^{\perp}$ and $\lbrace y_2, \dotsc, y_m \rbrace$ of $y_1^{\perp}$.  Define the orthogonal matrices $U_1 = [y_1, \dotsc, y_m]$ and $V = [x_1, \dotsc, x_n]$.  We may write
\begin{align*}
U_1^T A V_1 &=
\begin{bmatrix}
\sigma_1 & w^T \\
0 & B
\end{bmatrix}
\end{align*}
where $w \in \reals^{n-1}$ and $B$ is an $(m-1) \times (n-1)$ matrix.
Observe that
\begin{align*}
U_1^T A V_1 \begin{bmatrix} \sigma_1 \\ w\end{bmatrix}
&= \begin{bmatrix} \sigma_1^2 + w^T w \\ 
B \begin{bmatrix} \sigma_1 \\ w\end{bmatrix}
 \end{bmatrix}
\end{align*}
so that $\norm{U_1^T A V_1}^2 \geq (\sigma_1^2 + w^T w )$.  On the other hand, by orthogonality of $U_1$ and $V_1$ we know that
$\norm{U_1^T A V_1} = \norm{A} = \sigma_1$ and therefore we see that $w = 0$.    Now we can use the induction hypothesis to conclude that there exist $U_2$ and $V_2$ such that
$U_2^T B V_2 = \Sigma_2$ is diagonal and note that if we define 
\begin{align*}
U &= 
U_1 \begin{bmatrix}
1 & 0 \\
0 & U_2
\end{bmatrix}, &
V &= V_1 
\begin{bmatrix}
1 & 0 \\
0 & V_2
\end{bmatrix}
\end{align*}
Then 
\begin{align*}
U^T A V 
&=
\begin{bmatrix}
1 & 0 \\
0 & U_2^T
\end{bmatrix} U_1^T A V_1 
\begin{bmatrix}
1 & 0 \\
0 & V_2
\end{bmatrix} 
= 
\begin{bmatrix}
1 & 0 \\
0 & U_2^T
\end{bmatrix}
\begin{bmatrix}
\sigma_1 & 0 \\
0 & B
\end{bmatrix}
\begin{bmatrix}
1 & 0 \\
0 & V_2
\end{bmatrix} \\
&= 
\begin{bmatrix}
\sigma_1 & 0 \\
0 & \Sigma_2
\end{bmatrix} 
\end{align*}
\end{proof}

The fact that we may perform matrix factorizations in linear algebra in a Borel measurable way is easy to verify with the following elegant method relying on the ``principle of measurable choice''.
\begin{thm}\label{PrincipleOfMeasurableChoice}Let $S$ and $T$ be separable complete metric spaces and let $A \subset S \times T$ be closed and $\sigma$-compact.  Then $\pi_1(A)$ is Borel and there exists a Borel measurable function $f : \pi_1(A) \to T$ such that the graph of $f$ is contained in $A$.
\end{thm}
\begin{proof}
We start with the following
\begin{clm}Let $F$ be a closed set in $T$ then $\pi_1(A \cap S\times F)$ is Borel measurable.
\end{clm}
Write $A = \cup_n K_n$ with $K_n$ compact.  Since $F$ be a closed set it follows that $A \cap S\times F$ is closed $A \cap S\times F =  \cup_n K_n \cap S \times F$ where each $K_n \cap S \times F$ is compact.  Since $\pi_1$ is continuous each $\pi_1 (K_n \cap S \times F)$ is compact in $S$ and $\pi_1 (A \cap S\times F) = \cup_n \pi_1(K_n \cap S \times F)$ is therefore Borel.  

Applying the claim with $F = T$ we see that $\pi_1(A)$ is Borel.

Take a countable dense subset $\lbrace y_n \rbrace$ of $T$.  We define $f$ by an iterative approximation scheme.  
For $x \in \pi_1(A)$ define $f_1(x)$ to be the first $y_n$ (in index order) such that $A \cap \lbrace x \rbrace \times \overline{B}(y_n, 1/2) \neq \emptyset$ where $\overline{B}(z,r)$ represents the closed ball of radius $r$ centered at $z$.  Clearly $f_1$ is well defined since there exists an $(x,y) \in A$ and by density of $\lbrace y_n \rbrace$ in $T$ for any $r > 0$ there exists a $y_n$ with $(x,y) \in \lbrace x \rbrace \times \overline{B}(y_n,r)$.

Next observe that $f_1(x)$ is Borel measurable.  To see this, for each $r > 0$ and $n \in \naturals$ we define 
\begin{align*}
C_{n,r} &= \pi_1(A \cap S \times \overline{B}(y_n,r) )\\
&= \lbrace x \in S \mid A \cap \lbrace x \rbrace \times \overline{B}(y_n,r) \neq \emptyset \rbrace
\end{align*}
which is Borel by the claim.  Moreover it follows that $f_1^{-1}(y_n) = C_{1, 1/2}^c \cap \dotsb \cap C_{n-1, 1/2}^c \cap C_{{n}, 1/2}$ is Borel.  Since $f_1$ is countably valued it follows that $f_1$ is Borel measurable.

Now define $f_k(x)$ be the first $y_n$ such that $A \cap \lbrace x \rbrace \times \overline{B}(y_n, 1/2^k) \neq \emptyset$ and $d(f_{k-1}(x), y_n) \leq 1/2^{k-2}$; again density of the $\lbrace y_n \rbrace$ shows that $f_k(x)$ is well defined.  Borel measurability of $f_k$ follows by a simple induction as the Borel measurability of $f_{k-1}$ and Lipschitz continuity of $d$ imply that $D_{n,r} = \lbrace x \mid d(f_{k-1}(x), y_n) \leq r \rbrace$ is Borel measurable for every $n \in \naturals$ and $r > 0$.  Since 
\begin{align*}
&f_k^{-1}(y_n) = \\
&(C_{1,1/2^k} \cap D_{1, 1/2^{k-2}})^c \cap \dotsb \cap (C_{n-1,1/2^k} \cap D_{n-1, 1/2^{k-2}})^c \cap  C_{n,1/2^k} \cap D_{n, 1/2^{k-2}}
\end{align*}
and $f_k$ is countably valued it follows that $f_k$ is Borel measurable.

For a fixed $x$ note that for $j > k$ we have the triangle inequality
\begin{align*}
d(f_k(x), f_j(x)) &\leq \sum_{i=k}^{j-1} d(f_i(x), f_{i+1}(x)) \leq \sum_{i=k}^{j-1} 1/2^{i-1} \leq \sum_{i=k}^\infty 1/2^{i-1} = 1/2^{k-2}
\end{align*}
which shows that $f_k(x)$ is Cauchy.  By completeness of $T$, we can take 
the limit of $f_k$ which is a Borel measurable function (Lemma \ref{LimitsOfMeasurableMetricSpace}).  Since $A$ is closed and $\lim_{k \to \infty} d(f_k(x), A) = 0$ it follows that $(x,f(x)) \in A$ and we are done.

TODO: Do we ever use Polishness of $S$?  Note the reference Azoff ``Borel Measurability in Linear Algebra'' in the Proceedings of the AMS who in turn references Bourbaki.
\end{proof}

We now illustrate how matrix factorizations (and canonical forms) may be shown to be Borel measurable by using the singular value decomposition as an example.  
\begin{cor}\label{BorelMeasurabilitySVD}There exists a Borel measurable function $f (A) = (U,\Sigma,V)$ from  $\reals^{m\times n}$ to $\reals^{m \times m} \times \reals^{m \times n} \times \reals^{n \times n}$ such that $A = U \Sigma V^T$ is a singular value decomposition.
\end{cor}
\begin{proof}
Let 
\begin{align*}
F &= \lbrace (A, U,\Sigma, V) \mid \text{$U, V$ are orthogonal, $\Sigma$ is diagonal and $A = U \Sigma V^T$} \rbrace
\end{align*}
and note that $F$ is closed (because the spaces of orthognal and diagonal matrices are closed and matrix multiplication is continuous).  $F$ is also $\sigma$-compact because the ambient space is (just write $F = \cup_n F \cap \overline{B}(0,n)$).  Since singular value decompositions exist (Theorem \ref{SingularValueDecomposition}) we know that $\pi_1(F) = \reals^{m \times n}$ and therefore the result follows immediately from Theorem \ref{PrincipleOfMeasurableChoice}.
\end{proof}

B\'{e}la Sz Nagy also seems to prove measurability of eigenvalues and eigenvectors using the minimax criterion in ``Harmonic Analysis of Operators on Hilbert Space''.  From this one can bootstrap up to the measurability of the Schur decomposition and then get to the measurability of the SVD by considering Schur decompositions of $A^TA$ and $AA^T$.  There are also some more powerful section theorems that may have some relevance.

\chapter{Skorohod Space}
TODO: Currently going through this.  

TODO:  The development below is using the absolute value notation even though
paths take values in an arbitrary metric space.  Clean this up (which is actually pretty barfy because
there are a bunch of different metrics floating around).

TODO: Show that for $S$ complete, the sup norm makes $D([0,T];S)$ into a complete non-separable metric space.
We actually use the completeness in showing that the metric $d$ is complete.

Question 1:  In the definition of the $J_1$ topology on $D([0,\infty);
S)$ given a time shift $\lambda(t)$ we define $d(f,g,\lambda, u) = \sup_{t \geq 0} q(f(t \wedge u),
g(\lambda(t) \wedge u))$ and take the distance given the time shift as
$\int_0^\infty e^{-u} d(f,g,\lambda,u) \, du$.  Why is $d$ defined
this way and not as  $d(f,g,\lambda, u) = \sup_{0 \leq t \leq u} q(f(t),
g(\lambda(t)))$?  Would the latter fail to define a metric or would it
fail to be complete?

Question 2: Given a cadlag function $f : [0,1] \to S$, we know that
$f$ has only countably many jump discontinuities; is there some notion
of uniform continuity that can be preserved?  E.g. can we say that
given $\epsilon > 0$ for
all points of continuity $x$ of $f$  there exists a uniform $\delta > 0$
such that $\abs{x-y} < \delta$ implies $q(f(x), f(y)) < \epsilon$?

TODO: It would be convenient to treat the case of $[0,\infty]$ below

\begin{defn}Let $S$ be a topological space, then for every $0 < T < \infty$ we let $D([0,T]; S)$ denote the set of functions
$f : [0,T] \to S$ such that for every $0 \leq t < T$ we have $f(t) = \lim_{s \to t^+} f(s)$ and for every $0 < t \leq T$ the limit 
$\lim_{s \to t^{-}} f(s)$ exists and is finite.  The space $D([0,\infty); S)$ is the set of functions $f : [0,\infty)$ such that for all 
$t \geq 0$ we have $f(t) = \lim_{s \to t^+} f(s)$ and for all $t >0$ we have $\lim_{s \to t^{-}} f(s)$ exists and is finite.
\end{defn}

In what follows we will often use the notation $f(t-)$ to denote the limit $\lim_{s \to t^{-}} f(s)$.

\begin{lem}\label{CadlagCountableDiscontinuitySet}If $x \in D([0,T];
  S)$ or $x \in D([0,\infty); S)$ then $x$ is continuous at all but a
  countable number of points.
\end{lem}
\begin{proof}
We begin by considering the case of $x \in D([0,T]; S)$.  Pick an
$\epsilon > 0$ and define
\begin{align*}
A_\epsilon &= \lbrace 0 \leq t \leq T \mid r(x(t-), x(t)) \geq
\epsilon \rbrace
\end{align*}

\begin{clm} $A_\epsilon$ is finite.
\end{clm}

Suppose otherwise, then by compactness of $[0,T]$ there is an
accumulation point $t$ of $A_\epsilon$.  By passing to a further
subsequence we can assume that we have a sequence $t_n$ such that $t_n
\in A_\epsilon$ and
either $t_n \downarrow t$ or $t_n \uparrow t$.  First consider the
case $t_n \downarrow t$.  For every $n$ by the existence of the left
limit $x(t_n-)$ we can find $t_n^\prime$ such that $t_{n+1} >
t_n^\prime > t_n$ and $r(x(t_n), x(t_n^\prime)) > \epsilon/2$.  Now by
construction we
know that $t_n^\prime \downarrow t$ and by right continuity we get
$\lim_{n \to \infty} x(t_n) = \lim_{n \to \infty} x(t_n) = x(t)$.  
However this is a contradiction since we can find $N > 0$ such that
$r(x(t), x(t_N)) < \epsilon/4$ and $r(x(t), x(t_N^\prime) <
\epsilon/4$ which yields $r(x(t_N), x(t_N^\prime)) < \epsilon/2$.  If
$t_n \uparrow t$ we argue similarly construction a sequence
$t_n^\prime$ such that $t_{n-1} < t_n^\prime < t_n$ and $r(x(t_n),
x(t_n^\prime)) > \epsilon/2$.  By existence of left limits, we know that $\lim_{n \to \infty}
x(t_n^\prime) = \lim_{n \to \infty} x(t_n) = x(t-)$ and this gives a
contradiction as before.

Now simply note that the set of
discontinuities of $x$ is $\cup_{n=1}^\infty A_{1/n}$ and is therefore
countable.  In a similar way we see that the set of discontinuities
for $x \in D([0,\infty); S)$ is countable since it is equal to the
union of the discontinuities of $x$ restricted to $[0,n]$ for $n \in \naturals$.
\end{proof}

\begin{defn}Let $(S,r)$ be a metric space, define $\Lambda$ denote the set of all $\lambda : [0,T] \to
  [0,T]$ such that $\lambda$ is continuous, strictly increasing and
  bijective.  Then for each $\lambda \in \Lambda$ we define 
\begin{align*}
\rho(x,y,\lambda) &= \sup_{t \in [0,T]}\abs{\lambda(t) - t} \vee \sup_{t \in [0,T]} r(x(t), y(\lambda(t))
\end{align*}
and define $\rho : D([0,T]; S) \times D([0,T]; S) \to  \reals$ by 
\begin{align*}
\rho(x,y) &= \inf_{\lambda \in \Lambda} \rho(x,y,\lambda)
= \inf_{\lambda \in \Lambda} 
\sup_{t \in [0,T]}\abs{\lambda(t) - t} \vee \sup_{t \in [0,T]} r(x(t), y(\lambda(t))
\end{align*}
\end{defn}

\begin{lem}\label{SkorohodJ1RhoMetric}$\rho$ is a metric on $D([0,T];S)$.
\end{lem}
\begin{proof}
It is clear that $\rho(x,y) \geq 0$, now suppose that $\rho(x,y) =
0$.  By definition we can find a sequence $\lambda_n \in \Lambda$ such
that $\sup_{t \in [0,T]} \abs{\lambda_n(t) - t} < 1/n$ and $\sup_{t
  \in [0,T]} r(x(t), y(\lambda_n(t)) < 1/n$.  From the former
inequality we see that $\lim_{n \to \infty} \lambda_n(t) = t$ and the second
inequality we see that $\lim_{n \to \infty} y(\lambda_n(t)) = x(t)$.
The sequence $\lambda_n(t)$ has either a decreasing subsequence or a increasing subsequence
therefore passing to that subsequence and using the fact that $y$ is cadlag we see that either $x(t) = y(t)$ or $x(t) =
y(t-)$.  In particular, $x(t) = y(t)$ at all continuity points of
$y(t)$ and since the set of discontinuity points of $y$ is countable
it follows that the set of continuity points is dense in $[0,T]$.
Therefore for every $0 \leq t < T$ we can find a sequence $t_n$ of continuity points of $y$
such that $t_n \downarrow t$ and therefore by right continuity
of $y$ we conclude $x(t) = y(t)$.  The fact that $x(T) = y(T)$ follows from
the fact that $\lambda_n(T)=T$ for all $n \in \naturals$ thus 
\begin{align*}
y(T) &= \lim_{n \to \infty} y(\lambda_n(T)) = x(T)
\end{align*}

To see symmetry of $\rho$ we first note that $\lambda \in \Lambda$
implies $\lambda^{-1} \in \Lambda$.   To see this, it is first off
clear that $\lambda^{-1}$ exists because $\lambda$ is a bijection.
The fact that $\lambda^{-1}$ is strictly increasing follows because
if $0 \leq t < s \leq T$ and $0 \leq \lambda^{-1}(s) \leq \lambda^{-1}(t) \leq T$ then strictly
increasing and bijective nature of $\lambda$ tells $s \leq t$ which is
contradiction.  To see that $\lambda^{-1}$ is continuous, pick $0 < t
< T$ and let $\epsilon > 0$ be given such that $0 < \lambda^{-1}(t) -
\epsilon < \lambda^{-1}(t)  < \lambda^{-1}(t) + \epsilon < T$.  By strict increasingness
and bijectivity 
of $\lambda$ we know that $0 < \lambda(\lambda^{-1}(t) -
\epsilon) < t < \lambda(\lambda^{-1}(t) + \epsilon) < T$.  Let 
\begin{align*}
\delta &< ( t -  \lambda(\lambda^{-1}(t) -\epsilon)) \wedge (
\lambda(\lambda^{-1}(t) +\epsilon) -t)
\end{align*}
and note by the strict increasingness of $\lambda^{-1}$ we have
\begin{align*}
0 &< \lambda^{-1}(t) -\epsilon < \lambda^{-1}(t - \delta) <
\lambda^{-1}(t) < \lambda^{-1}(t + \delta) < \lambda^{-1}(t) + \epsilon
< T
\end{align*}
Now by the bijectivity of $\lambda$ we know that by a change of variables
\begin{align*}
\sup_{0 \leq t \leq T} \abs{\lambda(t) - t} &= \sup_{0 \leq s \leq T} \abs{s - \lambda^{-1}(s)} \\
\sup_{0 \leq t \leq T} r(x(t), y(\lambda(t))) &= \sup_{0 \leq s \leq
  T} r(x(\lambda^{-1}(s)), y(s)) = 
\sup_{0 \leq s \leq  T} r(y(s), x(\lambda^{-1}(s)))\\
\end{align*}
and therefore $\rho(x,y,\lambda) = \rho(y,x,\lambda^{-1}$.  Because
inversion is a bijection on $\Lambda$ we then get
\begin{align*}
\rho(x,y) &= \inf_{\lambda \in \Lambda} \rho(x,y,\lambda) =
\inf_{\lambda \in \Lambda} \rho(y,x,\lambda^{-1}) = \inf_{\lambda^{-1}
  \in \Lambda} \rho(y,x,\lambda^{-1}) = \rho(y,x)
\end{align*}
\end{proof}

The metric $\rho$ defines the Skorohod $J_1$ topology on the space
$D([0,T];S)$.  We emphasize here that we are actually interested in
the underlying topology as much as the metric space structure itself
since $\rho$ is not a complete metric.

\begin{examp}\label{NoncompletenessSkorohod}Let $f_n =
  \characteristic{[1/2, 1/2 + 1/(n+2))}$ for $n >0$ be a sequence in
  $D([0,1];\reals)$.  We show that $f_n$ is a Cauchy sequence with
  respect to $\rho$ but $f_n$ does not converge in the $J_1$ topology.
To see that $f_n$ is Cauchy, let $n > 0$ be given and suppose $m \geq
n$.  Define 
\begin{align*}
\lambda_{n,m}(t) &= \begin{cases}
t & \text{if $0 \leq t \leq 1/2$} \\
\frac{n+m+2}{n+2}(t - 1/2) + 1/2 & \text{if $1/2 \leq t < 1/2 +
  1/(n+m+2)$} \\
\frac{\frac{1}{2} - \frac{1}{n}}{\frac{1}{2} - \frac{1}{n+m+2}}(t - \frac{1}{2} - \frac{1}{n+m+2}) + \frac{1}{2} + \frac{1}{n} & \text{if $1/2 +
  1/(n+m+2) \leq t \leq 1$} \\
\end{cases}
\end{align*}
so that $f_{n+m} (t) = f_{n}(\lambda_{m+n}(t))$ for all $t \in [0,1]$
and $\sup_{0 \leq t \leq 1} \abs{\lambda_{m+n}(t) - t} = \frac{1}{n} -
\frac{1}{n+m+2} < \frac{1}{n}$ which shows $\rho(f_n, f_{n+m}) < \frac{1}{n}$.

\begin{clm}If $f_n$ converges in then it must converge to $0$.
\end{clm}

Suppose that $f_n$ converges to some $f \in D([0,1];\reals)$.  Then
there exist $\lambda_n \in \Lambda$ such that $\lim_{n \to \infty} \sup_{0 \leq t \leq 1}
\abs{\lambda_n(t) - t} = 0$ and 
$\lim_{n \to \infty} \sup_{0 \leq t \leq 1}\abs{f_n(t) -
  f(\lambda_n(t))} = 0$.  Therefore for each $0 \leq t \leq 1$ that is
a point of continuity of $f$ we have 
$\lim_{n \to \infty} f_n(t) = \lim_{n \to \infty} f(\lambda_n(t)) =
f(t)$.  By definition of $f_n(t)$ and Lemma
\ref{CadlagCountableDiscontinuitySet} we see that $f(t) = 0$ for all
but a countable number of $0 \leq t \leq 1$.  Therefore by right
continuity and the existence of left limits we conclude $f(t) = 0$ for
all $0 \leq t \leq 1$.  Since $f(\lambda(t))$ is identically zero for
all $\lambda \in \Lambda$ we conclude that $\rho(f_n, 0) = 1$ hence
$f_n$ does not converge.
\end{examp}

\begin{defn}Given $f \in D([0,T];S)$ the function
\begin{align*}
w(f,\delta) &= \inf_{\substack{0=t_0 < t_1 < \dotsb < t_n = T \\
  \min_{1 \leq i \leq n} (t_i - t_{i-1}) > \delta \\ n \in \naturals}}
\max_{1 \leq i \leq n} \sup_{t_{i-1} \leq s < t < t_i} r(f(s), f(t))
\end{align*}
is called the modulus of continuity.
\end{defn}

\begin{lem}\label{SkorohodJ1ModulusOfContinuity}If $f \in D([0,T];S)$ then $\lim_{\delta \to 0} w(f,\delta)
  = 0$.
\end{lem}
\begin{proof}
First note that for fixed $f$ the function $w(f, \delta)$ is a
non-decreasing function of $\delta$.  This is simply because any
candidate partition $0 = t_0 < t_1 < \dotsb < t_n=T$ with $\min_{1
  \leq i \leq n} (t_i - t_{i-1}) > \delta$ is also a candidate for any
smaller value of $\delta$.  Thus the set of candidate partitions gets
larger as $\delta$ shrinks and the infimum over the set of candiates
shrinks.

Let $\epsilon > 0$ be given.  Define $t_0 = 0$ then so long as
$t_{i-1} < T$ we inductively define
$t_i = \inf \lbrace t > t_{i-1} \mid r(f(t), f(t_{i-1})) > \epsilon \rbrace \wedge T$.  We claim that
there exists $n$ such $t_n = T$.  First, note that the sequence $t_i$
is strictly increasing while $t_i < T$ by the right continuity of
$f$.  If there are an infinite number of $t_i < T$ then by compactness
of $[0,T]$ there is a limit point $0 \leq t \leq T$.  However the
existence of the left limit $f(t-)$ says
exists $\delta > 0$ such that for all $0 < t - s < \delta$ we have
$r(f(s), f(t-)) < \epsilon/3$.
This is a contradiction since we can find an $n > 0$ such that for all
$i \geq n$ we have $t - t_i < \delta$.  By definition of the
$t_i$ for any $i \geq n+1$ we can pick
$t_i \leq s < t$ such that $r(f(s), f(t_{i-1})) >
\epsilon$ which provides us with $0 < t -s < \delta$ and 
\begin{align*}
r(f(s),f(t-)) &> r(f(s), f(t_{i-1})) - r(f(t_{i-1}), f(t-)) > \epsilon -
\epsilon/2 = \epsilon/2
\end{align*}

Thus we have constructed a sequence $0 =t_0 < t_1 < \dotsb < t_n = T$
such that $\max_{1 \leq i \leq n} \sup_{t_{i-1} < s < t < t_i}
r(f(s),f(t)) < 2 \epsilon$ so it we define $\delta = \frac{1}{2} \min_{1 \leq i
  \leq n} (t_i - t_{i-1})$ we have shown $w(f, \delta) \leq 2
\epsilon$.  Since $\epsilon$ was arbitrary and $w(f,\delta)$ is a
non-decreasing function of $\delta$ we are done.
\end{proof}

Even though the metric $\rho$ is not complete, the underlying topology
is Polish because we can define a topologically equivalent metric that is
complete.  To repair the incompleteness of $\rho$ we have to be a bit
more strict about the types of time changes that are allowed; more
specifically we have to prevent time changes are asymptotically flat
(or by considering taking the inverse of a time change prevent time
changes that are asymptotically vertical).  The following is a way of
quantifying such a requirement.
\begin{defn}For every $\lambda \in \Lambda$ define 
\begin{align*}
\gamma(\lambda) =
  \sup_{0 \leq s < t \leq T} \abs{\log \frac{\lambda(t) -
      \lambda(s)}{t-s}}
\end{align*}
For every $x,y \in D([0,T]; E)$ define 
\begin{align*}
d(x,y) = \inf_{\substack{\lambda \in \Lambda \\ \gamma(\lambda) <
    \infty}} \gamma(\lambda) \vee \sup_{0 \leq t \leq T} r(x(t) ,y(\lambda(t)))
\end{align*}
\end{defn}

The main goal is to prove that $d$ is a metric that is topologically equivalent to $\rho$.
Before proving that we need a few facts about $\gamma$.
\begin{lem}\label{SkorohodJ1GammaFacts}$\gamma(\lambda) = \gamma(\lambda^{-1})$ and
  $\gamma(\lambda_1 \circ \lambda_2) \leq \gamma(\lambda_1) +
  \gamma(\lambda_2)$.
\end{lem} 
\begin{proof}
These both follow from reparameterizations using the fact that
$\lambda^{-1}$ is a strictly increasing bijection.  For the first
\begin{align*}
\gamma(\lambda) &= 
\sup_{0 \leq s < t \leq T} \abs{\log  \frac{\lambda(t) -
    \lambda(s)}{t-s}} \\
&=\sup_{0 \leq \lambda^{-1}(s) < \lambda^{-1}(t) \leq T} \abs{\log
  \frac{\lambda(\lambda^{-1}(t)) -
    \lambda(\lambda^{-1}(s))}{\lambda^{-1}(t)-\lambda^{-1}(s)}} \\
&= \sup_{0 \leq s < t \leq T} \abs{\log  \frac{\lambda^{-1}(t) - \lambda^{-1}(s)}{t-s}} 
\end{align*}
and for the second
\begin{align*}
\gamma(\lambda) &= 
\sup_{0 \leq s < t \leq T} \abs{\log  \frac{\lambda_2(\lambda_1(t)) -
    \lambda_2(\lambda_1(s))}{t-s}} \\
&\leq \sup_{0 \leq s < t \leq T} \abs{\log  \frac{\lambda_2(\lambda_1(t)) -
    \lambda_2(\lambda_1(s))}{\lambda_1(t)-\lambda_1(s)}} +
\sup_{0 \leq s < t \leq T} \abs{\log  \frac{\lambda_1(t) -
    \lambda_1(s)}{t-s}} \\
&\leq \sup_{0 \leq s < t \leq T} \abs{\log  \frac{\lambda_2(t) -
    \lambda_2(s)}{t-s}} +
\sup_{0 \leq s < t \leq T} \abs{\log  \frac{\lambda_1(t) -
    \lambda_1(s)}{t-s}} \\
&=\gamma(\lambda_2) + \gamma(\lambda_1)
\end{align*}
\end{proof}

\begin{lem}\label{SkorohodEquivalenceA}For all $\lambda \in \Lambda$ such that $\gamma(\lambda) < 1/2$ we have
  $\sup_{0 \leq t \leq T} \abs{\lambda(t) - t} \leq
  2T\gamma(\lambda)$.  For all $f,g \in D([0,T]; S)$ such that $d(f,g) < 1/2$ we
  have $\rho(f,g) \leq 2Td(f,g)$.
\end{lem}
\begin{proof}
From the inequality $1+x \leq e^x$ we have
$\log(1+2x) \leq 2x$ for all $x > -1/2$ and therefore for $0 < x <
1/2$ we have $\log(1-2x) \leq -2x < -x < 0$.  
Similarly we have
$\log(1-2x) \leq -2x$ for all $x < 1/2$ and therefore for $0 < x <
1/2$ we have $\log(1-2x) \leq -2x < -x < 0$ for $0 < x < 1/2$.  On the
other hand, we see that $\frac{d}{dx} \left ( \log(1+2x) - x \right) =
\frac{2}{1+2x} - 1$ is positive for $0 < x < 1/2$ and therefore we
conclude 
\begin{align*}
\log(1-2x) &< -x < 0 < x < \log(1+2x) \text{ for } 0 < x < 1/2
\end{align*}

Suppose $\gamma(\lambda) < 1/2$ and let $0 < t \leq T$.  By definition and the
fact that $\lambda(0) = 0$ we have
\begin{align*}
\abs{\log \frac{\lambda(t)}{t}} &\leq \sup_{0 \leq s < t \leq
  T}\abs{\log \frac{\lambda(t) - \lambda(s)}{t-s}} = \gamma(\lambda) 
\end{align*}
and therefore we get
\begin{align*}
\log(1-2\gamma(\lambda)) &< -\gamma(\lambda) < \log \frac{\lambda(t)}{t} < \gamma(\lambda) < \log(1+2\gamma(\lambda))
\end{align*}
and exponentiating
\begin{align*}
1 - 2\gamma(\lambda) < \frac{\lambda(t)}{t} < 1 + 2 \gamma(\lambda)
\end{align*}
and therefore $\abs{\lambda(t) - t} < 2 T \gamma(\lambda)$ for $0 < t \leq T$.  Since $\lambda(0) - 0 = 0$ 
it follows that $\sup_{0 \leq t \leq T} \abs{\lambda(t) - t} \leq 2 T \gamma(\lambda)$.

Now suppose we have $d(f,g) < 1/2$.  Let $0 < \epsilon < 1/2 - d(f,g)$ be given and
select $\lambda \in \Lambda$ such that $\gamma(\lambda) <
d(f,g) + \epsilon$ and $\sup_{0 \leq t \leq T} r(f(t),
g(\lambda(t))) < d(f,g) + \epsilon$.  By what we have just shown, we
get that $\sup_{0 \leq t \leq T} \abs{\lambda(t) - t} < 2 T
\gamma(\lambda) < 2T(d(f,g) + \epsilon)$ and therefore $\rho(f,g) <
2T(d(f,g) + \epsilon)$.  Now let $\epsilon \to 0$.
\end{proof}

\begin{lem}\label{SkorohodEquivalenceB}For any $\delta>1/4$ if $\rho(f,g) < \delta^2$ then $d(f,g) \leq 4\delta + w(f,\delta)$.
\end{lem}
\begin{proof}
Let $\delta > 0$ be given, by definition of $w(f,\delta)$ choose a partition $0=t_0 < t_1 < \dotsb < t_n=T$ such that 
$t_i - t_{i-1} > \delta$ and $\sup_{t_{i-1} \leq s \leq t < t_i} \abs{f(t) - f(s)} < w(f,\delta) + \delta$ 
for all $i=1, \dotsc, n$.  Using the fact that $\rho(f,g) < \delta^2$ to pick a $\mu \in \Lambda$ such that 
$\sup_{0 \leq t \leq T} r(f(t),g(\mu(t))) < \delta^2$  and $\sup_{0 \leq t \leq T} \abs{\mu(t) - t} < \delta^2$.  Note that 
by the properties of $\mu$ we also have $\sup_{0 \leq t \leq T} r(f(\mu^{-1}(t)), g(t)) < \delta^2$.

Now we construct an appropriate $\lambda$ with which to bound $d(f,g)$.  
Define $\lambda(t_i) = \mu(t_i)$ for each $i=0,\dotsc, n$ and extend by linear
interpolation
\begin{align*}
\lambda(t) &= \frac{t-t_{i-1}}{t_i-t_{i-1}} \mu(t_i) + \frac{t-t_{i}}{t_{i-1}-t_{i}} \mu(t_{i-1}) \text{ for $t_{i-1} \leq t \leq t_i$}
\end{align*}
From the fact that $\mu(t_{i-1}) < \mu(t_i)$ it follows that $\lambda(t)$ is strictly increasing, $\lambda(0) = \mu(0) = 0$, 
$\lambda(T) = \mu(T) = T$ and $\lambda$ is piecewise linear hence continuous; thus $\lambda \in \Lambda$.  Moreover by
the increasingness of $\lambda$ and $\mu$ (and $\mu^{-1}$) we have $t_{i-1} \leq t \leq t_i$ is equivalent to $\lambda(t_{i-1}) \leq \lambda(t) \leq \lambda(t_i)$
which is in turn equivalent to $\mu^{-1}(\lambda((t_{i-1})) = t_{i-1} \leq \mu^{-1}(\lambda(t)) \leq t_i = \mu^{-1}(\lambda( t_i))$.
Thus
\begin{align*}
r(f(t), g(\lambda(t))) &\leq r(f(t), f(\mu^{-1}(\lambda(t))))  + r(f(\mu^{-1}(\lambda(t)))- g(\lambda(t))) \\
&\leq w(f,\delta) + \sup_{0 \leq t \leq T}  f(\mu^{-1}(\lambda(t))) \leq w(f,\delta) + \delta^2
\end{align*}
As for bounding $\gamma(\lambda)$ we have
\begin{align*}
\abs{\lambda(t_i) - \lambda(t_{i-1}) - (t_i - t_{i-1})} &= \abs{\mu(t_i) - \mu(t_{i-1}) - (t_i - t_{i-1})} \\
&\leq \abs{\mu(t_i) -  t_{i}}  + \abs{\mu(t_{i-1}) - t_{i-1}} \leq 2 \delta^2 < 2 \delta (t_i - t_{i-1})
\end{align*}
Recalling that $\lambda$ is linear on each interval $[t_{i-1},t_i]$ we note that this inequality simply says that the
slope of $\lambda$ on the linear segment $[t_{i-1},t_i]$ is in the interval $(1 - 2\delta, 1-2\delta)$.  Thus the inequality trivially extends 
to any $t_{i-1} \leq s \leq t \leq t_i$.  For other $s<t$  pick $i<j$ with $t_{i-1} \leq s \leq t_i$ and $t_{j-1} \leq t \leq t_j$.  Then we have
\begin{align*}
\abs{\lambda(t) - \lambda(s) - (t-s)} 
&\leq \abs{\lambda(t) - \lambda(t_{j-1}) - (t-t_{j-1})} + \sum_{k=i}^{j-2} \abs{\lambda(t_{k+1}) - \lambda(t_{k}) - (t_{k+1}-t_{k})}  + \\
&\abs{\lambda(t_i) - \lambda(s) - (t_i-s)} \\
&\leq 2 \delta (t-t_{j-1}) + 2 \delta \sum_{k=i}^{j-2} (t_{k+1}-t_{k})  + 2\delta (t_i-s) \\
&= 2\delta (t-s) \\
\end{align*}
and therefore 
\begin{align*}
\log(1 - 2\delta) \leq \log \left( \frac{\lambda(t) - \lambda(s)}{t-s} \right) \leq \log(1+2\delta)
\end{align*}
For arbitrary $\delta$ we have $\log(1+2\delta) \leq 2\delta < 4 \delta$ (Theorem \ref{BasicExponentialInequalities}) and by 
Taylor's Theorem (Lemma \ref{LagrangeFormRemainder}) we have for any $0 < \delta < 1/4$ there is a $0 < c < \delta < 1/4$ such that
\begin{align*}
\log(1 - 2\delta) &= \frac{-2\delta}{1 - 2c} > -4\delta
\end{align*}
and therefore $\gamma(\lambda) < 4\delta$.
\end{proof}
Now we are ready to show that $d$ is a metric and generates the same topology as $\rho$.
\begin{thm}\label{SkorohodJ1Metric}$d$ is a metric on $D([0,T];S)$ that is topologically equivalent to
  $\rho$.
\end{thm}
\begin{proof}
The fact that $d(f,g) \geq 0$ is immediate. Suppose $d(f,g)$ and pick $\lambda_n$ such that
$\lim_{n \to \infty} \gamma(\lambda_n) = 0$ and $\lim_{n \to \infty} \sup_{0 \leq t \leq
  T} r(f(t), g(\lambda_n(t))) = 0$.  By Lemma
\ref{SkorohodEquivalenceA}  we know that $\lim_{n \to \infty} \sup_{0
  \leq t \leq T} \abs{\lambda_n(t) - t} = 0$ as well and therefore we
can repeat the argument of Lemma \ref{SkorohodJ1RhoMetric} to conclude $f=g$.

To see symmetry just note that by reparametrizing and Lemma \ref{SkorohodJ1GammaFacts}
\begin{align*}
d(f,g) &= \inf_{\substack{\lambda \in \Lambda \\ \gamma(\lambda) <
    \infty}} \gamma(\lambda) \vee \sup_{0 \leq t \leq T} r(f(t),
g(\lambda(t))) \\
&= \inf_{\substack{\lambda \in \Lambda \\ \gamma(\lambda) <
    \infty}} \gamma(\lambda^{-1}) \vee \sup_{0 \leq \lambda^{-1}(t) \leq T}
r(g(t), f(\lambda^{-1}(t))) = d(g,f)
\end{align*}
and similarly with the triangle inequality.

TODO: Write out the triangle inequality part.

To see that $\rho$ and $d$ define the same topology we first show that every ball
in the $\rho$ metric contains a ball in the $d$ metric and vice versa.  For general notation
let $B_\rho(f,\epsilon)$ and $B_d(f,\epsilon)$ denote balls of radius $\epsilon$ centered at $f$ in the $\rho$ and $d$ metric
respectively.  Let $f$ and $r > 0$ be given.  Define $\delta < \frac{\epsilon}{2T} \wedge \frac{1}{4}$ and then apply Lemma \ref{SkorohodEquivalenceA}
to conclude that $d(f,g) \leq \delta$ implies $\rho(f,g) \leq 2T d(f,g) < \epsilon$; thus $B_d(f,\delta) \subset B_\rho(f,\epsilon)$.  On the other hand for a given $\epsilon>0$
because $\lim_{\delta \to 0} 4 \delta + w(f,\delta) = 0$ we can find $0 < \delta < 1/4$ such that $4 \delta + w(f,\delta) < \epsilon$ and therefore
by Lemma \ref{SkorohodEquivalenceB} we have $B_\rho(f, \delta^2) \subset B_d(f, \epsilon)$.

Now let $U$ be an open set in the topology defined by $d$.  For every $f \in U$, by openness of $U$ we find $\epsilon_f>0$ such that $B_d(f,\epsilon_f) \subset U$.
By the above argument we may find $\delta_f>0$ such that $B_\rho(f, \delta_f) \subset B_d(f,\epsilon_f) \subset U$.  Therefore we can write $U = \cup_{f \in U} B_\rho(f, \delta_f)$
which shows that $U$ is an open set in the topology defined by $\rho$ as well.  It is clear that the argument is symmetric in the role of $\rho$ and $d$ and therefore we
have shown that $d$ and $\rho$ are topologically equivalent metrics.
\end{proof}

The goal in introducing $d$ was to provide a complete metric; a useful
thing to check first is that $d$ fixes the example which showed $\rho$
was not a complete metric.
\begin{examp}Here we continue the Example
  \ref{NoncompletenessSkorohod} by showing directly that $f_n$ is not
  Cauchy in the metric $d$.  Because $f_n$ are indicator functions it
  follows that $\sup_{0 \leq t \leq 1} r(f_{n+m}(t), f_{n}(\lambda(t)))$ is
  either $0$ or $1$.  Therefore if $f_n$ is Cauchy then we can find
  $\lambda_{nm}(t)$ such that $\sup_{0 \leq t \leq 1} r(f_{n+m}(t),
    f_{n}(\lambda_{nm}(t))) = 0$.  By definition this tells us that
  $\lambda_{nm}([1/2,1/2 + 1/n+m+2]) =  [1/2,1/2+1/n]$ 
(of course $\lambda_{nm}([0,1/2]) = [0,1/2]$ and
$\lambda_{nm}([1/2 + 1/n+m+2, 1]) = [1/2 + 1/n,1]$ as well).  
From this fact we see that $\gamma(\lambda_{mn}) \geq
\frac{n+m+2}{n+2} > 1$ which shows that $d(f_n, f_{n+m}) \geq 1$ so
$f_n$ is not Cauchy with respect to $d$.
\end{examp}

Now we show that the metric $d$ is indeed complete.  In addition we show
that the $J_1$ topology is separable which shows us that it defines a Polish space.
This will allow us to apply our theory of weak convergence.
\begin{thm}\label{SkorohodJ1MetricPolish}Let $S$ be a complete metric space, then the metric $d$ on $D([0,T]; S)$
is complete.  Morever if $S$ is separable then $D([0,T];S)$ is separable in the $J_1$ topology.
\end{thm}
\begin{proof}
Let $f_n$ be a Cauchy sequence in $D([0,T]; S)$ with the metric $d$.  As a general principle of metric spaces it
suffices to show that $f_n$ has a convergent subsequence $f_{n_j}$.  Suppose that such a subsequence exists and converges
to $f$.  Then we may find $n_j$ such that $d(f,f_{n_j}) < \epsilon/2$ and $d(f_m, f_{n_j}) < \epsilon/2$ for all $m \geq n_j$; it follows
that in fact $d(f,f_m)<\epsilon$ for all $m \geq n_j$ and thus $f_n \to f$.

Using the Cauchy property we can find a subsequence $n_j$ such that $d(f{n_j}, f_{n_{j+1}}) < 2^{-j}$.  Therefore we have $\lambda_j$ 
such that $\gamma{\lambda_j} < 2^{-j}$ and $\sup_{0 \leq t \leq T} r(f_{n_j}(t), f_{n_{j+1}}(\lambda_j(t))) < 2^{-j}$.  Moreover from 
Lemma \ref{SkorohodEquivalenceA} we have $\sup_{0 \leq t \leq T} \abs{\lambda_j(t) -t } < T 2^{-j+1}$ for all $j \in \naturals$.  
Therefore for every $m,n \in \naturals$,
\begin{align*}
&\sup_{0 \leq t \leq T} \abs{\lambda_{m+n+1} \circ \lambda_{m+n} \circ \dotsb \circ \lambda_n(t) -  \lambda_{m+n} \circ \dotsb \circ \lambda_n(t)} \\
&= \sup_{0 \leq t \leq T} \abs{\lambda_{m+n+1}(t) - t} \leq T 2^{-(m+n)}
\end{align*}
which shows us that for fixed $n \in \naturals$ the sequence $\lambda_{m+n} \circ \dotsc \circ \lambda_n$ is Cauchy in $C([0,T], \reals)$ with
the sup norm.  Since this latter space is complete we know that there is  a limit $\nu_n$.  

\begin{clm} $\nu_n \in \Lambda$ and $\gamma(\nu_n) \leq 2^{-n+1}$.
\end{clm}

It is clear that $\nu_n$ is continuous as the uniform limit of continuous functions.  Moreover $\nu_n(0) = \lim_{m \to \infty} \lambda_{m+n} \circ \dotsb \lambda_{n}(0) = 0$
and similarly $\nu_n(T) = T$.  As a uniform limit of strictly increasing functions we also know that $\nu_n$ is non-decreasing.  To estimate $\gamma(\nu_n)$ we compute
\begin{align*}
&\abs{\log \frac{\lambda_{m+n} \circ \dotsb \circ \lambda_n(t) - \lambda_{m+n} \circ \dotsb \circ \lambda_n(s)}{t-s}} \\
&\leq \gamma(\lambda_{m+n} \circ \dotsb \circ \lambda_n) \\
&\leq \gamma(\lambda_{m+n}) + \dotsm +\gamma(\lambda_n) < 2^{-(m+n)} + \dotsm + 2^{-n} < 2^{-n+1}
\end{align*}
and therefore taking the limit as $m \to \infty$ we conclude 
\begin{align*}
\abs{\log \frac{\nu_n(t) - \nu_n(s)}{t-s}} &\leq 2^{-n+1}
\end{align*}
which show both that $\nu_n$ is strictly increasing (hence $\nu_n \in \Lambda$) and moreover that $\gamma(\nu_n) < 2^{-n+1}$ and the claim is shown.

Now note that $\nu_n = \nu_{n+1} \circ \lambda_n$.  From this it follows that
\begin{align*}
\sup_{0 \leq t \leq T} r(f_{n_j}(\nu_j^{-1}(t)), f_{n_{j+1}}(\nu_{j+1}^{-1}(t))) &= \sup_{0 \leq t \leq T} r(f_{n_j}(t),  f_{n_{j+1}}(\lambda_j(t))) < 2^{-j}
\end{align*}
and therefore for $j,m \in \naturals$ we have
\begin{align*}
\sup_{0 \leq t \leq T} r(f_{n_j}(\nu_j^{-1}(t)), f_{n_{j+m}}(\nu_{j+m}^{-1}(t))) &\leq 2^{-j} + \dotsm + 2^{-j-m+1} < 2^{-j+1}
\end{align*}
which shows that $f_{n_j} \circ \nu_j^{-1}$ is a Cauchy sequence in $D([0,T]; S)$ with respect to the sup norm.  Recalling that the sup norm
makes $D([0,T];S)$ into a complete (but not separable) space we can find a limit $f \in D([0,T]; S)$.  Thus $\sup_{0 \leq t \leq T} r(f_{n_j}(\nu_j^{-1}(t)), f(t)) \to 0$
and moreover $\gamma(\nu_j) \to 0$ which shows us that $d(f_{n_j}, f) \to 0$.

TODO: Show separability
\end{proof}

\section{Compactness and Tightness in Skorohod Space}

The entire point behind constructing the $J_1$ topology on $D([0,T]; S)$ was to
make it a Polish space so that Prohorov's Theorem can be applied to understand
weak convergence of cadlag stochastic processes.  The other pieces of the puzzle
in applying Prohorov's Theorem are being able to prove tightness and being able to
characterize limits (e.g. by understanding the finite dimensional distributions).  The latter
problem has little to do with topological aspects of path space (and to be honest in many
cases is an impossibly difficult nut to crack).  The former problem is deeply tied into the
nature of compactness in path space and we now turn to the consideration of such matters.

The first thing to do is to prove an analogue of the Arzela-Ascoli theorem that characterizes
relatively compact sets in $D([0,T]; S)$.  The proof of such a theorem ultimately rests on
approximation by step functions so we first prove that certain collections of step functions
are compact.

\begin{lem}\label{SkorohodJ1CompactSetsOfStepFunctions}Let $(S,r)$ be a metric space and let $K \subset S$.  Let $\delta > 0$ and define
$A(K, \delta) \subset D([0,T];S)$ be the set of functions $f$ for which there is a partition
$0=t_0 < t_1 < \dotsb < t_n=T$ with $t_i-t_{i-1} >  \delta$ for $i=1, \dotsc, n$, points $x_1, \dotsc, x_n \in K$ with $x_j \neq x_{j-1}$ for $j=2, \dotsc, n$ such that $f(t) = x_i$ for $t_{i-1} \leq t < t_i$.  If $K$ is compact in $S$ then $A(K, \delta)$ is relatively compact in the $J_1$ topology.
\end{lem}
\begin{proof}
It suffices to show that every sequence in $A(K,\delta)$ has a convergent subsequence. 
For each $f_n$ we by the definition of $A(K,\delta)$, let $0=t_{0,n} < \dotsm < t_{m_n, n}=T$   be a partition with $t_{j,n}-t_{j-1,n}>\delta$ for $j=1, \dotsc, m_n$ and $x_{1, n}, \dotsc, x_{m_n, n}$ be elements of $K$ such that $f_n(t) = x_{j,n}$ for $t_{j-1,n} \leq t < t_{j,n}$.  Let $m = \liminf_{n \to \infty} m_n$; that is to say $m$ is the smallest $m_n$ for which there are an infinite number of $f_n$ whose partitions have length $m$.
Note that $m \leq T/\delta < \infty$.  Since we are only looking for a convergence subsequence we can pass to the subsequence of $f_n$ for which $m_n = m$ and therefore we assume that all partitions have length $m$.

Consider the sequence $t_{1,n}$ and the sequence $x_{1,n}$.  Let $t_1 = \limsup_{n \to \infty} t_{1,n}$.  We know that $\delta < t_{1,n} \leq T$ thus $t_1 \in [\delta,T]$; pick a subsequence $N^1$ such that $t_{1,n} \to t_1$ along $N^1$. 
We also have $x_{1,n} \in K$ along $N^1$ so by compactness of  $K$ there 
is $x_1 \in K$ and a subsequence $N^2 \subset N^1$ such that $t_{1,n} \to t_1$ and $x_{1,n} \to x_1$ along $N^2$.  
In fact we can ask for another property of the sequence of $t_{1,n}$.  The function $g(t) = \abs{\frac{t_1}{t}}$ is continuous at $t_1$ 
and equals $g(t_1) = 0$ thus $g(t_{1,n}) \to 0$ along $N^2$ and we may pass to a further subsequence $N^3 \subset N^2$ (which we now
denote by $n_j$) to arrange that
\begin{itemize}
\item[(i)] $\lim_{j \to \infty} t_{1,n_j} = t_1$ with $\delta \leq t_1 \leq T$
\item[(ii)] $\abs{\frac{t_1}{t_{1, n_j}}} < \frac{1}{j}$ for $j \in \naturals$
\item[(iii)] $\lim_{j \to \infty} x_{1,n_j} = x_1$ with $x_1 \in K$
\end{itemize}

TODO: Get rid of the $k=1$ step since the induction step is clear once we define the trivial $k=0$ step.

If $m = 1$ we stop here (and note that we must have $t_1 = T$), otherwise we iterate to find further subsequences. 
To see the induction step we suppose that we have run the procedure $k<m$ times 
so that we have a subsequence $N^k$ and for each
$1 \leq i \leq k$ we have $i\delta \leq t_i < T$, $x_i \in K$ such that
\begin{itemize}
\item[(i)] $\lim_{j \to \infty} t_{i,n_j} = t_1$ 
\item[(ii)] $\abs{\frac{t_i - t_{i-1}}{t_{i, n_j} - t_{i-1, n_j}}} < \frac{1}{j}$ for $j \in \naturals$
\item[(iii)] $\lim_{j \to \infty} x_{i,n_j} = x_1$ 
\end{itemize}
We now let $t_{k+1} = \limsup_{j \to \infty} t_{k+1, n_j}$.  We know 
that $(k+1) \delta \leq t_{k+1} \leq T - (m - k)\delta$ and in fact $t_{k+1} = T$ if and only if $k+1 = m$.  
Now we replay the argument we used for $k=1$.  We extract a subsequence $N^{1,k} \subset N^k$ such that $t_{k+1, n_j} \to t_k$ along $N^{1,k}$.  Note also
that $m_{n_j} \geq k+1$ along this subsequence.  We use
compactness of $K$ to the get a further subsequence $N^{2,k}$ such that there is $x_{k+1} \in K$ with $x_{k+1, n_j} \to x_{k+1}$ along $N^{2,k}$.  The we use
continuity of $\abs{\frac{t_{k+1} - t_{k}}{t - t_{k, n_j}}}$ at $t_k$ to arrange for a final subsequence $N^{3,k}$ so that if we redefine $n_j$ to be the subsequence
$N^{3,k}$ we have $\abs{\frac{t_{k+1}- t_{k}}{t_{k+1, n_j} - t_{k, n_j}}} < \frac{1}{j}$ for all $j \in \naturals$.

Define $f(t)$ to be equal to $x_k$ on $t_{k-1} \leq t < t_k$ for $k=1, \dotsc, m$ (clearly $f \in D([0,T]; S)$).  We claim $f_n \to f$ in the $J_1$ topology along the subsequence $N^m$.
To see this, let $\lambda_j(t_{k, n_j}) = t_k$ for $k=1, \dotsc, m$ and extend by linear interpolation.  This is well defined because of the fact that $t_m = T$.  
Moreover from the property (ii) we have $\gamma(\lambda_j) \leq \frac{1}{j}$ so that
$\lim_{j \to \infty} \gamma(\lambda_j) = 0$.  We also have $r(f_{n_j}(t), f(\lambda_j(t))) = r(x_{n_j, k}, x_k)$ for $t_{k-1, n_j} \leq t < t_{k, n_j}$.  Thus because there are only finitely many
$x_k$ we can conclude that $\lim_{j \to \infty} \sup_{0 \leq t \leq T} r(f_{n_j}(t), f(\lambda_j(t))) = 0$ and therefore $\lim_{j \to \infty} d(f_{n_j}, f) = 0$.

TODO: Can the proof be simplified if we use the metric $\rho$ to demostrate convergence?
\end{proof}

\begin{thm}\label{ArzelaAscoliTheoremJ1Topology}Let $(S,r)$ be a complete metric space.  A set $A \subset D([0,T]; S)$ is relatively compact in the $J_1$ topology if and only if 
\begin{itemize}
\item[(i)]for each rational number
$t \in [0,T] \cap \rationals$ there exists a compact set $K_t \subset S$ such that $\cup_{f \in A} f(t) \subset K_t$
\item[(ii)]$\lim_{\delta \to 0} \sup_{f \in A} w(f, \delta) = 0$
\end{itemize}
In fact when $A \subset D([0,T]; S)$ is relatively compact in the $J_1$ topology then there is a compact set $K \subset S$ such that $\cup_{f \in A} \cup_{0 \leq t \leq T} f(t) \subset K$.
\end{thm}
\begin{proof}
We consider $D([0,T];S)$ as a metric space with the complete metric $d$.  We first suppose that $A$ is a set satisfying (i) and (ii).
Since a set $A$ is totally bounded if and only if 
its closure is totally bounded and a closed set of  a complete metric space is complete, it suffices to
show that $A$ is totally bounded with respect to $d$ (Theorem \ref{CompactnessInMetricSpaces}).  By (ii) every $k \in \naturals$ we
pick $0 < \delta_k < 1$ such that $\sup_{f \in A} w(f, \delta_k) < \frac{1}{k}$.  Pick $m_k \in \naturals$ with $\frac{1}{m_k} < \delta_k$ and 
define 
\begin{align*}
K^{(k)}  &= \cup_{i=1}^{T m_k} K_{i/m_k}
\end{align*}
so that $K^{(k)}$ is compact and $A_k = A(K^{(k)}, \delta_k)$ is relatively compact by Lemma \ref{SkorohodJ1CompactSetsOfStepFunctions}.
Pick $f \in A$ and since $w(f,\delta_k) < \frac{1}{k}$ we may choose a partition $0=t_0 < t_1 < \dotsb < t_n=T$ such that  for $i=1, \dotsc, n$ we have 
$t_i-t_{i-1} > \delta_k$ and $r(f(s), f(t)) < \frac{1}{k}$ for all $s,t \in [t_{i-1}, t_i)$.  Since we have chose $\frac{1}{m_k} < \delta_k < t_i - t_{i-1}$ note that every
interval $[\frac{j-1}{m_k}, \frac{j}{m_k}]$ has either $0$ or $1$ element of $t_i$ in it.  Define $g \in D([0,T]; S)$ by
\begin{align*}
g(t) &= f(\frac{\floor{m_k t_{i-1}} + 1}{m_k}) \text{ for $t_{i-1} \leq t < t_i$ and $i=1, \dotsc, n$}
\end{align*}
(this says that we define $g(t)$ on $[t_{i-1}, t_i)$ to be the value $f(j/m_k)$ where $j$ is the smallest integer such that $t_{i-1} < j/m_k$).
It is clear that $g \in A_k$ since $g$ is a step function on the partition $\lbrace t_j \rbrace$ and takes values in $K_{j/m_k}$ for appropriate $0 \leq j \leq T m_k$.
Furthermore we have for $t_{i-1} \leq t < t_i$ we have
\begin{align*}
r(f(t), g(t)) &\leq r(f(t), f(\frac{\floor{m_k t_{i-1}} + 1}{m_k})) < \frac{1}{k}
\end{align*}
which shows that $d(f,g) < \frac{1}{k}$.  

Now let $\epsilon > 0$ be given and pick $\frac{1}{k} < \epsilon /2$.  Since $A_k$ is relatively compact it
is totally bounded and thus there exist $g_1, \dotsc, g_n$ such that the balls $B(g_i, \epsilon/2)$ cover
$A_k$.  By the above argument for any $f \in A$ we can find $g \in A_k$ such that $d(f,g) < \frac{1}{k} < \epsilon/2$ 
and therefore by the triangle inequality the balls $B(g_i, \epsilon)$ cover $A$ and $A$ is totally bounded.
 
Now assume that $A$ is compact in $D([0,T];S)$.  To show (i) we actually show the stronger criteria that there exists a compact set $K$ such that $\cup_{f \in A} \cup_{0 \leq t \leq T} f(t) \subset K$.  It suffices to show that $\cup_{f \in A} \cup_{0 \leq t \leq T} f(t)$ is totally bounded.  Let $\delta > 0$ given and use the fact that $A$ is totally bounded to get a set $f_1, \dotsc, f_n \in A$ such that $B(f_j, \delta/2)$ cover $A$.  For each $f_j$ we pick a partition $0=t^j_0 < t^j_1 < \dotsb < t^j_{n_j} = T$ with $\max_{1 \leq i \leq n_j} \sup_{t^j_{i-1} \leq s < t < t^j_i} r(f_j(s), f_j(t)) < \delta/2$.  We claim that the ball $B(f_j(t^j_i); \delta)$ for $0 \leq i \leq n_j$ cover $\cup_{f \in A} \cup_{0 \leq t \leq T} f(t)$.  Now let $f \in A$ and $0 \leq t \leq T$ be given and pick $1 \leq j \leq n$ such that $f \in B(f_j, \delta/2)$ and then pick $\lambda \in \Lambda$ such that $\sup_{0 \leq s \leq T} r(f(s), f_j(\lambda(s))) < \delta/2$; in particular $r(f(t), f_j(\lambda(t))) < \delta/2$.  Pick $1 \leq i \leq n_j$ such that $t^j_{i-1} \leq \lambda(t) < t^j_i$ (TODO: What if $t = T$?)  and then it follows that 
\begin{align*}
r(f(t), f_j(t^j_i)) &\leq r(f(t), f_j(\lambda(t))) + r( f_j(\lambda(t)), f_j(t^j_{i-1})) < \delta
\end{align*}

To see (ii) we argue by contradiction.  Suppose that there exists $\epsilon$ such that $\sup_{f \in A} w(f, \delta) \geq \epsilon$ for all $\delta > 0$.  In particular we can find a sequence $f_n \in A$ such that $w(f_n, 1/n) \geq \epsilon$.  By relative compactness of $A$ we can pass to a convergent subsequence $f_{n_k} \to f$ with $f \in D([0,T];S)$.  But then
we have from Lemma \ref{SkorohodJ1ModulusOfContinuity} (actually we have only proven the relevant part  in Lemma \ref{SkorohodInfiniteJ1ModulusOfContinuity})
\begin{align*}
\epsilon &\leq \limsup_{k \to \infty} w(f_{n_k}, \delta) \leq w(f, \delta)
\end{align*}
for all $\delta > 0$ which is a contradicts $\lim_{\delta \to 0}  w(f, \delta) = 0$.
 
Here is another attempt at an argument to show (i) that I didn't finish.  It suffices to show that $\cup_{f \in A} \cup_{0 \leq t \leq T} f(t) \subset K$ is relatively compact so let $f_n(t_n)$ be a sequence with $f_n \in A$ and $0 \leq t_n \leq T$.  We first make some simple reductions.  By compactness of $[0,T]$ we know there is a subsequence such that $t_n$ converges to a value $0 \leq t \leq T$ and by passing to a further subsequence we may assume that $t_n < t$ or $t_n \geq t$.  By passing to a third subsequence using the compactness of $A$ we may assume that $f_n$ converges to some $f \in D([0,T];S)$ in the $J_1$ topology.  For notational cleanliness we assume that $f_n(t_n)$ represents this final subsequence.  By the fact that $f_n \to f$ in the $J_1$ topology we may assume that there exist $\lambda_n \in \Lambda$ such that $\lim_{n \to \infty} \sup_{0 \leq s \leq T} \abs{\lambda(s) - s} = 0$.  In particular, $\lim_{n \to \infty} \lambda_n(t) = t$ and by passing to a subsequence we may assume that $\lambda_n(t) < t$ or $\lambda_n(t) \geq t$ for all $n$ along the subsequence.   TODO: How to finish this off; it is clear how this shows that $\cup_{f \in A} f(t)$ is relatively compact for a fixed $t$????
\end{proof}

\begin{lem}\label{SkorohodBorelGeneratedByProjections}For every $t \in [0,T]$ let $\pi_t : D([0,T];S) \to
  S$ be the evaluation map $\pi_t(f) = f(t)$.  The Borel $\sigma$-algebra on $D([0,T]; S)$ is
  equal to $\sigma(\lbrace \pi_t \mid t \in [0,T] \rbrace)$ and
  therefore $\mathcal{B}(D([0,T]; S)) = D([0,T];
  S) \cap \mathcal{B}(\reals)^{\otimes [0,\infty)}$.
\end{lem}
\begin{proof}
TODO:
\end{proof}


\section{The space $D([0,\infty); S)$}

For applications it is equally or perhaps more important to deal with the space $D([0,\infty); S)$ of cadlag functions on the half infinite interval.  
Essentially all of the results we have proven for $D([0,T]; S)$ hold but there is some subtlety in getting the definitions right so that the topology
on $D([0,\infty); S)$ gives the right behavior to the restriction maps $D([0,\infty); S) \to D([0,T]; S)$.  The problem that has to be dealt with is best 
illustrated with an example.  Consider the function 
\begin{align*}
f(x) &= \begin{cases}
0 & \text{if $0 \leq x < 1$} \\
1 & \text{if $1 \leq x$}
\end{cases}
\end{align*}
and the approximating sequence
\begin{align*}
f_n(x) &= \begin{cases}
0 & \text{if $0 \leq x < 1+1/n$} \\
1 & \text{if $1+1/n \leq x$}
\end{cases}
\end{align*}
Our intuition is the $J_1$ topology on $D([0,\infty); \reals)$ should make functions close if there is a small time change that makes them
uniformly close; thus we should expect that $f_n$ converges to $f$.  However consider the restriction of these functions to $D([0,1]; \reals)$.  The restriction of
$f$ has a jump discontinuity of size $1$ at the endpoint $x=1$ while the restrictions of $f_n$ are all identically zero.  Because the endpoint of the domain
$[0,1]$ cannot be moved by a time change it is easy to see that $f_n$ does not converge to $f$ in the $J_1$ topology on $D([0,1]; \reals)$.  Another way of 
looking at the example is to observe that is shows the restriction map $D([0,T]; S)$ to $D([0,1]; S)$ for $T > 1$ is not continuous in the $J_1$ topology.  As a
side effect of this one cannot simply glue the spaces $D([0,n];S)$ for $n \in \naturals$ together (formally to create a projective limit) to get a topology on $D([0,\infty); S)$ in the 
same way that one can do so with $C([0,T];S)$ and $C([0,\infty);S)$.  

The good news is that the example we have given illustrates the only problem that has to be
dealt with: namely restricting to a point of discontinuity of an element $f \in D([0,\infty), S)$.  For a given $f$ we have already seen that there are only a countable
number of discontinuities of $f$ so in particular the set of discontinuities has Lebesgue measure zero.  The trick is that if we integrate the restrictions to $D([0,T];S)$ 
over $0 \leq T < \infty$ then these discontinuities won't matter and we can create a metric on $D([0,\infty);S)$.  While one can proceed in this fashion by using the
existing metric on $D([0,T];S)$ to create a metric on $D([0,\infty);S)$ it turns out to be about as much work to just start from scratch.  The advantage in doing so
is that the development can be formally independent of the case of a finite interval (though we do refer to some proofs for details that are left out).  

\begin{defn}Let $(S,r)$ be a metric space, define $\Lambda_\infty$ denote the set of all $\lambda : [0,\infty) \to
  [0,\infty)$ such that $\lambda$ is continuous, strictly increasing and
  bijective.  For $\lambda \in \Lambda_\infty$ define
\begin{align*}
\gamma(\lambda) =
  \sup_{0 \leq s < t \leq T} \abs{\log \frac{\lambda(t) -
      \lambda(s)}{t-s}}
\end{align*}
Then for each $\lambda \in \Lambda_\infty$ we define 
\begin{align*}
\psi(x,y,\lambda, T) &= \sup_{0 \leq t < \infty} r(x(t \wedge T), y(\lambda(t) \wedge T)) \wedge 1
\end{align*}
and define $\rho : D([0,\infty); S) \times D([0,\infty); S) \to  \reals$ by 
\begin{align*}
\rho(x,y) &= \inf_{\lambda \in \Lambda_\infty} 
\int_0^\infty \gamma(\lambda) \vee e^{-T} \left[ \psi(x,y,\lambda,T) \vee (\sup_{0 \leq u \leq T}\abs{\lambda(t) - t} \wedge 1) \right] \, dT 
\end{align*}
$d : D([0,\infty); S) \times D([0,infty); S) \to  \reals$ by 
\begin{align*}
d(x,y) &= \inf_{\substack{\lambda \in \Lambda_\infty \\
\gamma(\lambda) < \infty}} 
\left[ \gamma(\lambda) \vee \int_0^\infty e^{-T} \psi(x,y,\lambda,T) \, dT \right]
\end{align*}
\end{defn}
So note with this definition for each $T$ we are restricting each pair of cadlag functions $x,y$ to the interval $[0,T]$ but also
implicitly thinking of $D([0,T])$ as being embedded in $D([0,\infty))$ by extending as a constant function.

TODO: Note the measurability of $\psi(x,y,\lambda,T)$ as a function of $T$.

Note that $\psi(x,y,\lambda,T)$ is not a continuous function of $T$ for fixed $x,y,\lambda$ (however I believe it is cadlag?)
\begin{examp}
Let $f = \characteristic{[1, \infty)}$ and $g = \characteristic{[2, \infty)}$ and $\lambda(1) = 2$, then $\psi(f,g,\lambda,t) = \characteristic{[0,1) \cup [2,\infty)}$.
\end{examp}

In general when thinking of convergence relative to $d$ (which we have not yet shown is a metric of course) we usually use 
the following rendering of the definition.  We warn the reader that this result is used so frequently that we will rarely make explicit note
of it.

\begin{lem}\label{SkorohodInfinityConvergenceToZero}Let $f_n$ and $g_n$ be sequences in $D([0,\infty); S)$ then $\lim_{n \to \infty} d(f_n, g_n) = 0$ if and only if there exist $\lambda_n \in \Lambda_\infty$ such
that $\gamma(\lambda_n) < \infty$ for all $n \in \naturals$, $\lim_{n \to \infty} \gamma(\lambda_n) = 0$ and 
\begin{align*}
\lim_{n \to \infty} m \lbrace t \in [0, T] \mid \sup_{0 \leq s < \infty} r(f_n(s \wedge t), g_n(\lambda_n(s) \wedge t)) > \epsilon \rbrace = 0
\end{align*} 
for all $0 < T < \infty$ and $0 < \epsilon < 1$.  Moreover if $\lim_{n \to \infty} \gamma(\lambda_n) = 0$ then we also have $\lim_{n \to \infty}   \sup_{0 \leq t \leq T} \abs{\lambda_n(t) - t} = 0$ for every $T > 0$.
\end{lem}
\begin{proof}
Suppose that such $\lambda_n$ exist then for every $0 < T < \infty$ and $0 < \epsilon < 1$, then it follows from the definition of $\psi$ that $\sup_{0 \leq s < \infty} r(f_n(s \wedge t), g_n(\lambda_n(s) \wedge t)) > \epsilon$ if and only if $\psi(f_n,g_n, \lambda_n, t) > \epsilon$
\begin{align*}
\lim_{n \to \infty} \int_0^\infty e^{-t} \psi(f_n,g_n, \lambda_n, t) \, dt
&\leq \lim_{n \to \infty} \int_0^T e^{-t} \psi(f_n,g_n, \lambda_n, t) \, dt + \int_T^\infty e^{-t} dt \\
&\leq \epsilon \int_0^T e^{-t} \, dt + \lim_{n \to \infty} m \lbrace t \in [0, T] \mid \psi(f_n,g_n,\lambda_n,t) > \epsilon \rbrace + e^{-T} \\
&\leq  \epsilon + e^{-T}
\end{align*}
now let $T \to \infty$ and $\epsilon \to 0$ to see that $\lim_{n \to \infty} \int_0^\infty e^{-t} \psi(f_n,g_n, \lambda_n, t) \, dt = 0$.

On the other hand, suppose that $\lim_{n \to \infty} d(f_n, g_n) = 0$ and pick $\lambda_n$ such that $\lim_{n \to \infty} \gamma(\lambda_n) = 0$ and 
$\lim_{n \to \infty} \int_0^\infty e^{-t} \psi(f_n,g_n, \lambda_n, t) \, dt = 0$.  By a Markov bound we know that $\psi(f_n,g_n, \lambda_n, t)  \toprob 0$ with respect to
the exponential distribution with rate $1$ on $[0,\infty)$.  Thus we have for every $T > 0$ and $\epsilon > 0$,
\begin{align*}
m \lbrace t \in [0, T] \mid \psi(f_n,g_n,\lambda_n,t) > \epsilon \rbrace 
&= \int_0^T \characteristic{\psi(f_n,g_n,\lambda_n,t) > \epsilon} \, dt \\
&\leq e^{T} \int_0^\infty e^{-t} \characteristic{\psi(f_n,g_n,\lambda_n,t) > \epsilon} \, dt = e^T \probability{\psi(f_n,g_n,\lambda_n,t) > \epsilon} \\
\end{align*}
and let $n \to \infty$.

By the proof of Lemma \ref{SkorohodEquivalenceA} we know that for every $T > 0$ and for $n \in \naturals$ such that $\gamma(\lambda_n) < 1/2$ we have $\sup_{0 \leq t \leq T} \abs{\lambda_n(t) - t} \leq 2T \gamma(\lambda_n)$ and therefore $\lim_{n \to \infty} \sup_{0 \leq t \leq T} \abs{\lambda_n(t) - t} = 0$ for every $T > 0$.
\end{proof}

\begin{prop}$d$ is a metric on $D([0,\infty); S)$.
\end{prop}
\begin{proof}
It is immediate from the definition that $d(f,g) \geq 0$.  Suppose $d(f,g) = 0$.  Pick $\lambda_n$ as per Lemma \ref{SkorohodInfinityConvergenceToZero} applied to constant sequences $f_n \equiv f$ and $g_n \equiv g$.  Thus,
$\lim_{n \to \infty}   \sup_{0 \leq t \leq T} \abs{\lambda_n(t) - t} = 0$ for all $0 < T < \infty$ and $\lim_{n \to \infty} m \lbrace 0 \leq t \leq T \mid \sup_{0 \leq s < \infty} r(f(s \wedge t), g(\lambda_n(s) \wedge t)) > \epsilon \rbrace = 0$ for all $0 < T < \infty$ and $0 < \epsilon < 1$.  So for any fixed $0 \leq u < \infty$ we let $0 < \epsilon < 1$ be arbitrary and
pick $T > u+\epsilon$. 
Therefore $m \lbrace 0 \leq t \leq T \mid \sup_{0 \leq s < \infty} r(f(s \wedge t), g(\lambda_n(s) \wedge t)) > \epsilon \rbrace < T -u-\epsilon$ for sufficiently large $n$ thus 
$(u+\epsilon,T) \cap \lbrace 0 \leq t \leq T \mid \sup_{0 \leq s < \infty} r(f(s \wedge t), g(\lambda_n(s) \wedge t)) \leq \epsilon \rbrace \neq \emptyset$ for sufficiently large $n$  
and therefore for sufficiently large $n$ we may pick $u+\epsilon < w_n < T$ such that
\begin{align*}
r(f(u), g(\lambda_n(u) \wedge w_n)) \leq \sup_{0 \leq s < \infty} r(f(s \wedge w_n), g(\lambda_n(s) \wedge w_n)) \leq \epsilon
\end{align*}
We also know that $\lim_{n \to \infty} \lambda_n(u) = u$ and by passing to a subsequence $N$, we may assume that $\lambda_n(u) \geq u$ or $\lambda_n(u) \leq u$ along $N$. 
In particular, $\lambda_n(u) < w_n$ for large $n$ and thus $\lim_{n \to \infty} r(f(u), g(\lambda_n(u) \wedge w_n)) = r(f(u), g(u))$ or $\lim_{n \to \infty} r(f(u), g(\lambda_n(u) \wedge w_n)) = r(f(u), g(u-))$ where the limits are taken along the subsequence $N$.  Now we argue as in Lemma \ref{SkorohodJ1RhoMetric},  $f(u) = g(u)$ at all continuity points of $g$ but the discontinuity points are a countable set thus $f(u) = g(u)$ everywhere by right continuity.

Symmetry of $d$ follows just as for the $D([0,T])$ case Theorem \ref{SkorohodJ1Metric} by using the facts that $\gamma(\lambda) = \gamma(\lambda^{-1})$ and $\psi(f,g,\lambda,t) = \psi(g,f,\lambda^{-1},t)$ for all $f,g \in D([0,\infty),S)$ and $0 \leq t < \infty$.

TODO: Finish the triangle inequality part...
\end{proof}

 Because the form of the Skorohod metric is a bit opaque, some effort will go into developing different criteria for convergence
in the topology.  A first such result follows; note that this result is very close to saying that implies that $\lim_{n \to \infty} d(f_n,f) = 0$ precisely when the restrictions
to $[0,T]$ converge in $D([0,T];S)$ whenever $T$ is a continuity point of $f$.  TODO: What extra work has to be done to massage the $\lambda_n$ so that the equivalence is proven???

\begin{prop}\label{SkorohodInfiniteJ1EquivalenceA}Let $f,f_n \in D([0,\infty); S)$ then $\lim_{n \to \infty} d(f_n,f) = 0$ if and only if there exists $\lambda_n \in \Lambda_\infty$ such that $\lim_{n \to \infty} \gamma(\lambda_n) = 0$ and 
\begin{align*}
\lim_{n \to \infty} \psi(f_n,f,\lambda_n,t) = 0 \text{ for all continuity points of $f$}
\end{align*}
If $\lim_{n \to \infty} d(f_n,f) = 0$ then $\lim_{n \to \infty} f_n(t) = f(t)$ when $t$ is a continuity point of $f$.
\end{prop}
\begin{proof}
Suppose that we have $\lambda_n$ as in the hypotheses.  Then as the the number of discontinuity points of $f$ is countable, $\lim_{n \to \infty} \psi(f_n,f,\lambda_n,t) = 0$ for almost every $0 \leq t < \infty$.   Since
$0 \leq e^{-t} \psi(f_n,f,\lambda_n,t) \leq 1$ we may apply Dominated Convergence to conclude that $\lim_{n \to \infty} \int_0^\infty e^{-t} \psi(f_n,f,\lambda_n,t)  \, dt = 0$.

On the other hand suppose $\lim_{n \to \infty} d(f_n,f) =0$ and let $0 \leq T < \infty$ be fixed.  
\begin{clm}There exists $\lambda_n$ and $T < T_n < \infty$ such that $\lim_{n \to \infty} T_n = \infty$, $\lim_{n \to \infty} \gamma(\lambda_n) = 0$ and 
\begin{align*}
\lim_{n \to \infty} \sup_{0 \leq t < \infty} r(f_n(t \wedge T_n), f(\lambda_n(t) \wedge T_n)) = 0
\end{align*}
\end{clm}
Pick $\lambda_n$ as per Lemma \ref{SkorohodInfinityConvergenceToZero} applied to sequences $f_n$ and $g_n \equiv f$.   So $\lim_{n \to \infty} \gamma(\lambda_n) = 0$ and for each
$k \in \naturals$ we may find a $N_k$ such that 
\begin{align*}
m \lbrace 0 \leq t \leq T+k \mid \sup_{0 \leq s < \infty} r(f_n(s \wedge t), f(\lambda_n(s) \wedge t)) > 1/k \rbrace < 1/k \text{ for all $n \geq N_k$}
\end{align*}
It follows that $(T+k-1, T+k) \not \subset \lbrace 0 \leq t \leq T+k \mid \sup_{0 \leq s < \infty} r(f_n(s \wedge t), f(\lambda_n(s) \wedge t)) > 1/k \rbrace$ for every $n \geq N_k$.
Now working inductively on $k$, for $0 \leq n < N_1$ pick $T_n$ arbitrarily and for $k \in \naturals$ and for each $N_k \leq n < N_{k+1}$, pick a $T_n$ with $T \leq T+k-1 < T_n < T+k$ such that 
\begin{align*}
\sup_{0 \leq s < \infty} r(f_{n}(s \wedge T_n), f(\lambda_{n}(s) \wedge T_n)) \leq 1/k
\end{align*}
By construction, $\lim_{n\to \infty} T_n = \infty$ and $\lim_{n \to \infty} \sup_{0 \leq s < \infty} r(f_{n}(s \wedge T_n), f(\lambda_{n}(s) \wedge T_n)) =0$.

Let $\lambda_n$ and $T_n$ be chosen as in the claim, and by the triangle inequality
\begin{align*}
\sup_{0 \leq t < \infty} r(f_n(t \wedge T), f(\lambda_n(t) \wedge T)) 
&\leq \sup_{0 \leq t < \infty} r(f_n(t \wedge T), f(\lambda_n(t \wedge T) \wedge T_n)) +\\
&\sup_{0 \leq t < \infty} r(f(\lambda_n(t \wedge T) \wedge T_n), f(\lambda_n(t) \wedge T)) 
\end{align*}
We look at each of the terms on the right hand side in detail.  For the first term, since $T < T_n$,
\begin{align*}
\sup_{0 \leq t < \infty} r(f_n(t \wedge T), f(\lambda_n(t \wedge T) \wedge T_n))
&=\sup_{0 \leq t \leq T} r(f_n(t), f(\lambda_n(t) \wedge T_n)) \\
&=\sup_{0 \leq t \leq T} r(f_n(t \wedge T_n), f(\lambda_n(t) \wedge T_n))  \\
&\leq \sup_{0 \leq t} r(f_n(t \wedge T_n), f(\lambda_n(t) \wedge T_n))
\end{align*}
and for the second term, 
\begin{align*}
&\sup_{0 \leq t < \infty} r(f(\lambda_n(t \wedge T) \wedge T_n), f(\lambda_n(t) \wedge T))  \\
&=\sup_{0 \leq t \leq T} r(f(\lambda_n(t) \wedge T_n), f(\lambda_n(t) \wedge T))  \vee
\sup_{T \leq t < \infty} r(f(\lambda_n(T) \wedge T_n), f(\lambda_n(t) \wedge T))  \\
&=\sup_{T \leq t \leq (\lambda_n(T) \wedge T_n) \vee T} r(f(t), f(T))  \vee
\sup_{\lambda_n(T) \wedge T \leq t \leq T} r(f(\lambda_n(T) \wedge T_n), f(t))  \\
&\leq \sup_{T \leq t \leq \lambda_n(T) \vee T} r(f(t), f(T))  \vee
\sup_{\lambda_n(T) \wedge T \leq t \leq T} r(f(T), f(t))  + 
r(f(\lambda_n(T) \wedge T_n), f(T)))\\
&\leq \sup_{\lambda_n(T) \wedge T\leq t \leq \lambda_n(T) \vee T} r(f(t), f(T)) + 
r(f(\lambda_n(T) \wedge T_n), f(T)))\\
\end{align*}
Thus we have 
\begin{align*}
&\lim_{n \to \infty} \sup_{0 \leq t < \infty} r(f_n(t \wedge T), f(\lambda_n(t) \wedge T)) 
\leq \lim_{n \to \infty} \sup_{0 \leq t < \infty} r(f_n(t \wedge T_n), f(\lambda_n(t) \wedge T_n)) + \\
&\lim_{n \to \infty} \sup_{\lambda_n(T) \wedge T\leq t \leq \lambda_n(T) \vee T} r(f(t), f(T)) + 
\lim_{n \to \infty} r(f(\lambda_n(T) \wedge T_n), f(T))) \\
&=0
\end{align*}
where the first limit on the right hand side is zero by the claim and the second two limits are zero
because $\lim_{n \to \infty} \lambda_n(T) = T$, $T < T_n$ and $f$ is continuous at $T$.
\end{proof}

TODO: The main part of the following proposition is (iii) implies (ii).  I think there is a lot of redundancy between the argument provided and the one showing the modulus of continuity converges to zero.  We should probably introduce the modulus of continuity first and then pick a partition to assist with the proof.  Can we phrase the argument in terms
of open balls as done in the $D([0,T])$ case that I culled from Bass?

\begin{prop}\label{SkorohodInfiniteJ1EquivalenceB}Let $f,f_n \in D([0,\infty); S)$ then the following are equivalent
\begin{itemize}
\item[(i)] $\lim_{n \to \infty} d(f_n,f) = 0$ 
\item[(ii)] there exists $\lambda_n \in \Lambda_\infty$ with $\lim_{n \to \infty} \gamma(\lambda_n) = 0$ and 
\begin{align*}
\lim_{n \to \infty} \sup_{0 \leq t \leq T} r(f_n(t), f(\lambda_n(t)) = 0 \text{ for all $T > 0$}
\end{align*}
\item[(iii)] for every $T > 0$ there exists $\lambda_n \in \Lambda_\infty$ with $\lim_{n \to \infty} \sup_{0 \leq t \leq T} \abs{\lambda_n(t) -t} = 0$ and 
\begin{align*}
\lim_{n \to \infty} \sup_{0 \leq t \leq T} r(f_n(t), f(\lambda_n(t)) = 0
\end{align*}
\end{itemize}
\end{prop}
\begin{proof}
TODO: Finish

(ii) $\implies$ (iii) is immediate since any single sequence $\lambda_n$ that satisfies (ii) also satisfies (iii) for all $T > 0$ by Lemma \ref{SkorohodInfinityConvergenceToZero}.

(iii) $\implies$ (ii)  For each $N \in \naturals$, pick $\lambda^N_n \in \Lambda_\infty$ such that $\lim_{n \to \infty} \sup_{0 \leq t \leq N} \abs{\lambda^N_n(t) - t} = 0$ and $\lim_{n \to \infty} \sup_{0 \leq t \leq N} r(f_n(t), f(\lambda^N_n(t))) = 0$.  Since the relevant conditions are independent of the values of $\lambda^N_n(t)$ for $t > N$ we may also assume that
$\lambda^N_n(t) = \lambda^N_n(N) + t - N$ for $t > N$.  Now define $\tau^N_0 = 0$ and inductively define
\begin{align*}
\tau^N_k = \inf \lbrace t > \tau^N_{k-1} \mid r(f(t), f(\tau^N_{k-1})) > 1/N \rbrace
\end{align*}
if $\tau^N_{k-1} < \infty$ and define $\tau^N_k = \infty$ otherwise.

\begin{clm}If $\tau^N_k < \infty$ then $\tau^N_k < \tau^N_{k+1}$.
\end{clm}
By right continuity there exists $\delta > 0$ such that $r(f(\tau^N_k), f(t)) < 1/N$ for all $\tau^N_k < t < \tau^N_{k} + \delta$ and therefore $\tau^N_{k+1} \geq \tau^N_k + \delta$.

\begin{clm}If $\lim_{k \to \infty} \tau^N_k = \infty$.
\end{clm}
If $\lim_{k \to \infty} \tau^N_k < \infty$ then the previous claim and by the existence of left limits we know $f(\lim_{k \to \infty} \tau^N_k) = \lim_{k \to \infty} f(\tau^N_k) < \infty$.  On the other hand we know that $r(f(\tau^N_k), f(\tau^N_{k+1})) \geq 1/N$ which shows that such a limit cannot exist.

For each $n \in \naturals$ and $k \in \integers_+$ define
\begin{align*}
u^N_{k,n} = (\lambda^N_n)^{-1}(\tau^N_k)
\end{align*}
when $\tau^N_k < \infty$ and $u^N_{k,n} = \infty$ if $\tau^N_k = \infty$.  Define
\begin{align*}
\mu^N_n(t) = \begin{cases}
\tau^N_k + \left( \frac{\tau^N_{k+1} - \tau^N_k}{u^N_{k+1,n} - u^N_{k,n}} \right) (t - u^N_{k,n}) & \text{for $t \in [u^N_{k,n}, u^N_{k+1,n}) \cap [0,N]$} \\
\mu^N_n(N) + t - N & \text{for $t > N$}
\end{cases}
\end{align*}
where we use the convention that $\frac{\infty}{\infty} = 1$.  Note that $\mu^N_n(u^N_{k,n}) = \tau^N_k= \lambda^N_n(u^N_{k,n})$.  Note also that for $u^N_{k,n} \leq t < u^N_{k+1,n}$ we have $\tau^N_k \leq \lambda^N_t(t) < \tau^N_{k+1}$ which implies $r(f(\tau^N_k), f(\lambda^N_n(t))) < 1/N$ and similarly $r(f(\tau^N_k), f(\mu^N_n(t))) < 1/N$.  Thus 
$\sup_{0 \leq t \leq T} r(f(\lambda^N_n(t)),  f(\mu^N_n(t))) < 2/N$.

Since we have defined $\mu^N_n$ to be piecewise linear it follows that
\begin{align*}
\gamma(\mu^N_n) &= \max_{0 \leq k \leq ?} \abs{\log \frac{\tau^N_{k+1} - \tau^N_k}{u^N_{k+1,n} - u^N_{k,n}} } < \infty
\end{align*}
(TODO: Note that $\mu^N_n$ only has a finite number of different slopes; but that exact number doesn't seem to have a simple formula)
and moreover
\begin{align*}
\sup_{0 \leq t \leq T} r(f_n(t), f(\mu^N_n(t))) &\leq \sup_{0 \leq t \leq T} r(f_n(t), f(\lambda^N_n(t))) + \sup_{0 \leq t \leq T} r(f_n(\lambda^N_n(t)), f(\mu^N_n(t))) \\
&\leq \sup_{0 \leq t \leq T} r(f_n(t), f(\lambda^N_n(t))) + 2/N
\end{align*}

\begin{clm}$\lim_{n \to \infty} \gamma(\mu^N_n) = 0$.
\end{clm}
Since $\lim_{n \to \infty} \sup_{0 \leq t \leq T} \abs{\lambda_n(t) - t} = 0$ it follows that $\lim_{n \to \infty} u^N_{k,n} = \lim_{n \to \infty}  (\lambda^N_n)^{-1}(\tau^N_k) = \tau^N_k$ hence
\begin{align*}
\lim_{n \to \infty} \gamma(\mu^N_n) &= \max_{0 \leq k \leq ?} \lim_{n \to \infty} \abs{\log \frac{\tau^N_{k+1} - \tau^N_k}{u^N_{k+1,n} - u^N_{k,n}} } = 0
\end{align*}

Let $n_0 =0$ and for each $n \in \naturals$ pick $n_N > n_{N-1}$ such that 
\begin{itemize}
\item[(i)] $\gamma(\mu^N_n) < 1/N$ for all $n \geq n_N$
\item[(ii)] $\sup_{0 \leq t \leq N} r(f_n(t), f(\lambda^N_n(t))) < 1/N$ for all $n \geq n_N$
\end{itemize}

For $1 \leq n < n_1$ define $\hat{\lambda}_n \in \Lambda_\infty$ satisfy $\gamma(\hat{\lambda}_n) < \infty$ but otherwise be arbitrary and for $n_N \leq n < n_{N+1}$ let $\hat{\lambda}_n = \mu^N_n$.  Then for any $T>0$, $N > \ceil{T}$ and $n_N \leq n < n_{N+1}$ we have 
\begin{align*}
\sup_{0\leq t \leq T} r(f_n(t), f(\hat{\lambda}_n(t))) &\leq \sup_{0\leq t \leq N} r(f_n(t), f(\hat{\lambda}_n(t))) \\
&=\sup_{0\leq t \leq N} r(f_n(t), f(\mu^N_n(t)))  \leq \sup_{0\leq t \leq N} r(f_n(t), f(\lambda^N_n(t))) + 2/N < 3/N
\end{align*}
which shows that $\lim_{n \to \infty} \sup_{0\leq t \leq T} r(f_n(t), f(\hat{\lambda}_n(t)))$.
\end{proof}

In order to understand compactness in $D([0,\infty);S)$ we need to have a notion of equicontinuity.  The basic idea is that we
express equicontinuity relative to bounded intervals $[0,T]$ and we must account for the existence jumps on such an interval
and particularly at $T$ itself.
\begin{defn}Let $(S,r)$ be a metric space and let $f \in D([0,\infty); S)$ be given then for each $\delta > 0$ and $T > 0$ we define the \emph{modulus of continuity}
\begin{align*}
w^\prime(f,\delta, T) &= \inf_{\substack{0=t_0 < t_1 < \dotsb < t_{n-1} <  T \leq t_n \\
  \min_{1 \leq i \leq n} (t_i - t_{i-1}) > \delta \\ n \in \naturals}}
\max_{1 \leq i \leq n} \sup_{t_{i-1} \leq s < t < t_i} r(f(s), f(t))
\end{align*}
\end{defn}

TODO: Unify the proofs here with the case of $D([0,T];S)$.

Note that we allow the right hand endpoint of the partition $\lbrace t_i \rbrace$ to extend beyond $T$ in the definition above.  TODO: Presumably this makes dealing with jumps
at $T$ easier; on the other hand we don't do this in the case of $D([0,T];S)$ ; is it necessary to define it this way?
\begin{lem}\label{SkorohodInfiniteJ1ModulusOfContinuity}
Let $f, g\in D([0,\infty); S)$ then 
\begin{itemize}
\item[(i)] $w^\prime(f,\delta, T)$ is a non-decreasing function of $\delta$ and $T$
\item[(ii)] For every $\delta > 0$ and $T > 0$
\begin{align*}
w^\prime(f,\delta, T) &\leq w^\prime(g,\delta, T) + 2 \sup_{0 \leq t \leq T+ \delta} r(f(t), g(t))
\end{align*}
\item[(iii)]For each fixed $T > 0$, $w^\prime(f,\delta, T)$ is a right continuous function of $\delta$ and
\begin{align*}
\lim_{\delta \to 0^+} w^\prime(f,\delta, T) &= 0
\end{align*}
\item[(iv)]If $f_n \in D([0,\infty); S)$ and $\lim_{n \to \infty} d(f_n,g) = 0$ then for every $\delta > 0$, $T > 0$ and $\epsilon > 0$ 
\begin{align*}
\limsup_{n \to \infty}  w^\prime(f_n,\delta, T)  &\leq  w^\prime(f,\delta, T+\epsilon) 
\end{align*}
\item[(v)]For fixed $\delta > 0$ and $T>0$, $ w^\prime(f,\delta, T)$ is a Borel measurable function of $f$.
\end{itemize}
\end{lem}
\begin{proof}
(i) is immediate from the fact that the set of partitions over which the infimum is calculated is non-decreasing with respect to set inclusion as either $\delta$ decreases or $T$ increases.  Suppose $0 < \delta < \delta^\prime$, then if we have a partition $t_0 < \dotsb < t_{n-1} < T \leq t_n$ and $t_i - t_{i-1} > \delta^\prime$ for all $i=1, \dotsc, n$ then it follows that $t_i - t_{i-1} > \delta$ for all $i=1, \dotsc, n$ as well.  Thus 
\begin{align*}
&\lbrace (t_0, \dotsc, t_n) \mid t_0 < \dotsb < t_{n-1} < T \leq t_n \text{ and } t_i - t_{i-1} > \delta^\prime \rbrace \subset  \\
&\lbrace (t_0, \dotsc, t_n) \mid t_0 < \dotsb < t_{n-1} < T \leq t_n \text{ and } t_i - t_{i-1} > \delta \rbrace 
\end{align*}
and taking unions over $n \in \naturals$ and infimums it follows that $w(f,\delta, T) \leq w(f,\delta^\prime, T)$.  Suppose that $0 < T < T^\prime$, given a partition $t_0 < \dotsb < t_{n-1} < T \leq t_n$ with $t_i - t_{i-1} > \delta$ for all $i=1, \dotsc, n$ we may extend it to a partition $t^\prime_0 < \dotsb < t^\prime_{n^\prime-1} < T^\prime \leq t^\prime_{n^\prime}$ with $t^\prime_i - t^\prime_{i-1} > \delta$ for all $i=1, \dotsc, n^\prime$ and $t^\prime_i = t_i$ for all $0 \leq i \leq n$; moreover we have $\max_{1 \leq i \leq n} \sup_{t_{i-1} \leq s < t < t_i} r(f(s),f(t)) \leq \max_{1 \leq i \leq n^\prime} \sup_{t^\prime_{i-1} \leq s < t < t^\prime_i} r(f(s),f(t))$ and therefore
\begin{align*}
w^\prime(f,\delta, T) &= \inf_{\substack{0=t_0 < t_1 < \dotsb < t_{n-1} <  T \leq t_n \\
  \min_{1 \leq i \leq n} (t_i - t_{i-1}) > \delta \\ n \in \naturals}}
\max_{1 \leq i \leq n} \sup_{t_{i-1} \leq s < t < t_i} r(f(s), f(t)) \\
&\leq \inf_{\substack{0=t_0 < t_1 < \dotsb < t_{n-1} <  T \leq t_n \\
  \min_{1 \leq i \leq n} (t_i - t_{i-1}) > \delta \\ n \in \naturals}}
\max_{1 \leq i \leq n^\prime} \sup_{t^\prime_{i-1} \leq s < t < t^\prime_i} r(f(s), f(t)) \\
&\leq \inf_{\substack{0=t_0 < t_1 < \dotsb < t_{n-1} <  T^\prime \leq t_n \\
  \min_{1 \leq i \leq n} (t_i - t_{i-1}) > \delta \\ n \in \naturals}}
\max_{1 \leq i \leq n} \sup_{t_{i-1} \leq s < t < t_i} r(f(s), f(t)) \\
&= w(f, \delta, T^\prime)\\
\end{align*}

(ii) follows from the fact that for fixed $\delta > 0$, $T>0$ and any valid partition $t_1 < \dotsb < t_{n-1} < T \leq t_n$ we have $0 \leq t_i \leq T+\delta$ for all $1 \leq i \leq n$ and therefore by the triangle inequality
\begin{align*}
\sup_{t_{i-1} \leq s < t < t_i} r(f(s), f(t)) &\leq \sup_{t_{i-1} \leq s < t < t_i} r(f(s), g(s)) +\sup_{t_{i-1} \leq s < t < t_i} r(g(s), g(t)) + \sup_{t_{i-1} \leq s < t < t_i} r(g(t), f(t)) \\
&\leq \sup_{t_{i-1} \leq s < t < t_i} r(g(s), g(t))  + 2 \sup_{0 \leq t \leq T+\delta} r(g(t), f(t))
\end{align*}
Now take the maximum over $1 \leq i \leq n$ and the infimum over all partitions.

To see the right continuity in (iii), let $T>0$, $\delta>0$ and $\epsilon > 0$ be given and pick a partition $t_0 < \dotsb < t_{n-1} < T \leq t_n$ such that $\min_{1 \leq i \leq n} (t_i - t_{i-1}) > \delta$ and $\max_{1 \leq i \leq n} \sup_{t_{i-1} \leq s < t < t_i} r(f(s), f(t)) \leq w^\prime(f,\delta, T) + \epsilon$.  Using the fact that $w^\prime(f,\delta, T)$ is a nondecreasing function of $\delta$, the existence of this partition shows that $w^\prime(f,\delta^\prime, T) < w^\prime(f,\delta, T) +\epsilon$ for all $\delta^\prime -\delta < \frac{1}{2}\min_{1 \leq i \leq n} (t_i - t_{i-1})$.  The fact that $\lim_{\delta \to 0^+} w^\prime(f,\delta, T) = 0$ follows by the same argument as in Lemma \ref{SkorohodJ1ModulusOfContinuity}.

To see (iv) we know that there exist $\lambda_n$ such that $\lim_{n \to \infty} \sup_{0 \leq t \leq T+\delta}\abs{\lambda_n(t) - t} = 0$ and $\lim_{n \to \infty} \sup_{0 \leq t \leq T+\delta} r(f_n(t), f(\lambda_n(t))) = 0$.  Define $\delta_n = \sup_{0 \leq t \leq T}\abs{\lambda_n(t+\delta) - \lambda_n(t)}$ and note that $\delta_n \to 0$.  If we let $t_0 < \dotsb t_{m-1} < \lambda(T) \leq t_m$ be a partition with $t_i - t_{i-1} > \delta_n$ for all $1 \leq i \leq m$ then it follows that $\lambda_n^{-1}(t_0) < \dotsb \lambda_n^{-1}(t_{m-1}) < T \leq \lambda_n^{-1}(t_m)$ and
\begin{align*}
\lambda^{-1}_n(t_i) - \lambda^{-1}_n(t_{i-1}) > \delta
\end{align*}
for all $1 \leq i \leq m$.  Since we also have $\sup_{\lambda^{-1}_n(t_{i-1})  \leq s < t < \lambda^{-1}_n(t_i) } r(f(s), f(t)) = \sup_{t_{i-1}  \leq s < t < t_i} r(f(\lambda_n(s)), f(\lambda_n(t)))$ it follows that $w^\prime(f \circ \lambda_n,\delta, T) \leq w^\prime(f,\delta_n, \lambda_n(T))$.  Using this fact, (i), (ii)  and $\lim_{n \to \infty} \lambda_n(T) = T$ we get for any $\epsilon > 0$,
\begin{align*}
\limsup_{n \to \infty}  w^\prime(f_n,\delta, T) 
&\leq \limsup_{n \to \infty} \left[ w^\prime(f \circ \lambda_n,\delta, T) + 2 \sup_{0 \leq t \leq T+\delta} r(f_n(t), f(\lambda_n(t)) \right] \\
&=\limsup_{n \to \infty}  w^\prime(f \circ \lambda_n,\delta, T) \\
&\leq \limsup_{n \to \infty}  w^\prime(f , \delta_n, \lambda_n(T)) \\
&\leq \limsup_{n \to \infty}  w^\prime(f , \delta \vee \delta_n, T+\epsilon) \\
&=w^\prime(f,\delta, T+\epsilon) 
\end{align*}

To see (v) define $w^\prime(f,\delta, T+) = \lim_{\epsilon \to 0^+} w^\prime(f,\delta, T+\epsilon)$, then by the fact that $w^\prime(f_n,\delta, T)$ is non-decreasing in $T$ and (iv) we for every $\epsilon > 0$,
\begin{align*}
\limsup_{n \to \infty}  w^\prime(f_n,\delta, T+) &\leq \limsup_{n \to \infty}  w^\prime(f_n,\delta, T+\epsilon) \leq w^\prime(f,\delta, T+2\epsilon) 
\end{align*}
Now let $\epsilon \to 0^+$ to conclude that  $w^\prime(f,\delta, T+)$ is an upper semicontinuous function of $f$, hence Borel measurable.  Now we claim that 
$w^\prime(f,\delta, T) = \lim_{n \to \infty} w^\prime(f,\delta, (T-1/n)+)$ which shows that $w^\prime(f,\delta, T)$ is Borel measurable.


I believe that this last statement is supported by the following (which we should add to the statement of the Lemma):
\begin{clm}$w^\prime(f,\delta, T)$ is a left continuous function of $T$
\end{clm}
Let $\epsilon > 0$ be given and using the existence of the left limit $\lim_{t \to T^-} f(t)$, pick $\rho > 0$ such that $\sup_{T - \rho \leq s < t < T} r(f(s), f(t)) < \epsilon/2$.  Since 
$w^\prime(f,\delta, T)$ is non-decreasing in $T$, we know that $\lim_{t \to T^-} w^\prime(f,\delta, t)$ exists and $w^\prime(f,\delta, s) \leq \lim_{t \to T^-} w^\prime(f,\delta, t)$ for all $0 \leq s < T$.  We can pick a partition $t_0 < \dotsb < t_{m -1} < T - \rho \leq t_{m}$ with $\min_{1 \leq i \leq m} (t_i - t_{i-1}) > \delta$ and 
\begin{align*}
\max_{1 \leq i \leq m} \sup_{t_{i-1} \leq s < t < t_i} r(f(s),f(t)) &< w^\prime(f,\delta, T-\rho) + \epsilon/2 \leq \lim_{t \to T^-} w^\prime(f,\delta, t) + \epsilon/2
\end{align*}
If we have $t_{m} \geq T$ then we can conclude 
\begin{align*}
w^\prime(f,\delta,T) &\leq \max_{1 \leq i \leq m} \sup_{t_{i-1} \leq s < t < t_i} r(f(s),f(t)) \leq \lim_{t \to T^-} w^\prime(f,\delta, t) + \epsilon/2
\end{align*}  
It that is not the case then
we modify these chosen partition to make one with which we can bound $w^\prime(f,\delta,T)$.  Specifically supposing $t_{m} < T$, we define $\tilde{t}_j = t_j$ for $0 \leq j < m$ and define $\tilde{t}_{m} = T$.  It is clear from the properties of the partition $\lbrace t_i \rbrace$ and the fact we have only moved the rightmost endpoint of the partition further to the right that $\tilde{t}_0 < \dotsb < \tilde{t}_{m -1} < T = \tilde{t}_{m}$, that
$\min_{1 \leq i \leq m} (\tilde{t}_i - \tilde{t}_{i-1}) \geq \min_{1 \leq i \leq m} (t_i - t_{i-1}) > \delta$ and for $1 \leq i < m$
\begin{align*}
\sup_{\tilde{t}_{i-1} \leq s < t < \tilde{t}_i} r(f(s),f(t))  &= \sup_{t_{i-1} \leq s < t < t_i} r(f(s),f(t))  \leq \sup_{t_{i-1} \leq s < t < t_i} r(f(s),f(t)) + \epsilon/2
\end{align*}
Moreover we may pick a point $T^*$ such that  
\begin{align*}
\tilde{t}_{m-1} &= t_{m-1} < T - \rho < T^*  < t_{m} < T = \tilde{t}_m
\end{align*} 
and use this to break down the supremum into cases:
\begin{align*}
&\sup_{\tilde{t}_{m-1} \leq s < t < \tilde{t}_{m}} r(f(s),f(t)) \\
&=\sup_{\tilde{t}_{m-1} \leq s < t \leq T^*} r(f(s),f(t)) \vee \sup_{T^* \leq s < t < \tilde{t}_{m}} r(f(s),f(t)) \vee \sup_{\tilde{t}_{m-1} \leq s < T^* < t < \tilde{t}_{m}} r(f(s),f(t)) \\
&\leq \sup_{t_{m-1} \leq s < t <  t_{m}} r(f(s),f(t)) \vee \sup_{T - \rho < s < t < T} r(f(s),f(t)) \vee \\
&\sup_{t_{m-1} \leq s < T^* < t < T} \left[ r(f(s),f(T^*)) + r(f(T^*), f(t)) \right] \\
&\leq \sup_{t_{m-1} \leq s < t < t_{m}} r(f(s),f(t)) \vee \sup_{T - \rho < s < t < T} r(f(s),f(t)) \vee \\
&\left(\sup_{t_{m-1} \leq s < t < t_{m}} r(f(s),f(t)) + \sup_{T - \rho < s < t \leq T} r(f(s),f(t)) \right) \\
&\leq \sup_{t_{m-1} \leq s < t < t_{m}} r(f(s),f(t)) + \epsilon/2
\end{align*}
Thus we have 
\begin{align*}
w^\prime(f,\delta, T) &\leq \max_{1 \leq i \leq m} \sup_{\tilde{t}_{i-1} \leq s < t < \tilde{t}_i} r(f(s),f(t))  \\
&\leq \max_{1 \leq i \leq m} \sup_{t_{i-1} \leq s < t < t_i} r(f(s),f(t)) + \epsilon/2 \\
&\leq \lim_{t \to T^-} w^\prime(f,\delta, t) + \epsilon
\end{align*}
Now since $\epsilon > 0$ was arbitrary we let $\epsilon \to 0$ and conclude $w^\prime(f,\delta, T)  \leq \lim_{t \to T^-} w^\prime(f,\delta, t) $.
On the other hand it is clear from the fact that $w^\prime(f,\delta,T)$ is non-decreasing in $T$ that $\lim_{t \to T^-} w^\prime(f,\delta, t)  \leq w^\prime(f,\delta, T)$ and therefore $\lim_{t \to T^-} w^\prime(f,\delta, t)  = w^\prime(f,\delta, T)$.
\end{proof}


\begin{lem}\label{SkorohodInfiniteBorelGeneratedByProjections}For every $t \in [0,\infty)$ let $\pi_t : D([0,\infty); S) \to
  S$ be the evaluation map $\pi_t(f) = f(t)$.  
\begin{itemize}
\item[(i)] For arbitrary $S$ every $\pi_t$ is Borel measurable.  
\item[(ii)] $\sigma(\pi_t ; 0 \leq t < \infty) = \sigma(\pi_t ; t \in D)$ for any dense subset $D \subset [0,\infty)$.
\item[(iii)]If $S$ is separable then the Borel $\sigma$-algebra on $D([0,\infty); S)$ is
  equal to $\sigma(\lbrace \pi_t \mid t \in [0,\infty) \rbrace)$ and
  therefore $\mathcal{B}(D([0,\infty); S)) = D([0,\infty);
  S) \cap \mathcal{B}(S)^{\otimes [0,\infty)}$.
\end{itemize}
\end{lem}
\begin{proof}
Let $\psi : S \to \reals$ be a bounded continuous function and suppose $\epsilon > 0$ and $0 \leq t < \infty$ are fixed.  Define
\begin{align*}
\psi^\epsilon_t(f) &= \frac{1}{\epsilon} \int_t^{t + \epsilon} \psi(\pi_s(f)) \, ds = \frac{1}{\epsilon} \int_t^{t + \epsilon} \psi(f(s)) \, ds
\end{align*}

\begin{clm}$\psi^\epsilon_t : D([0,\infty); S) \to \reals$ is a continuous function 
\end{clm}
Suppose that $f_n \to f$ in the $J_1$ topology.  By Proposition \ref{SkorohodInfiniteJ1EquivalenceA} we know that $f_n(t) \to f(t)$ at every continuity point of $f$.  Since the set of discontinuity points of $f$ is countable it has measure zero and it follows that $f_n \to f$ almost everywhere.  Thus by continuity of $\psi$ we have $\psi \circ f_n \to \psi \circ f$ almost everywhere and thus by Dominated Convergence we conclude that $\psi^\epsilon_t (f_n) \to \psi^\epsilon_t (f)$.  

By the Fundamental Theorem of Calculus we have $\lim_{\epsilon \to 0^+} \psi^\epsilon_t(f) = \psi(\pi_t(f))$ which shows that $\psi \circ \pi_t$ is Borel measurable for every bounded
continuous $\psi : S \to \reals$.  Since every bounded Borel measurable function is a limit of bounded continuous functions (TODO: ) if follows that $\psi \circ \pi_t$ is 
Borel measurable for every bounded measurable $\psi : S \to \reals$.  In particular for $A \in \mathcal{B}(S)$ then $\pi_t^{-1}(A) = (\characteristic{A} \circ \pi_t)^{-1}(\lbrace 1 \rbrace)$ is Borel measurable.

To see (ii) let $0 \leq t < \infty$ be given and pick a sequence $t_1, t_2, \dotsc$ with $t_n \in D \cap [t,\infty)$ such that $\lim_{n \to \infty} t_n = t$.  By right continuity of the elements
of $D([0,\infty); S)$ we see that $\pi_t = \lim_{n \to \infty} \pi_{t_n}$.

To see (iii) we first set about showing that open balls are in $\sigma(\pi_t ; 0 \leq t < \infty)$.  To show that is equivalent to showing that for fixed $g \in D([0,\infty); S)$ the function
$d(\cdot, g) : D([0,\infty);S) \to reals$ is $\sigma(\pi_t ; 0 \leq t < \infty)$-measurable, so we set to work on that.  To create approximations of $d(\cdot, g)$, we assume that
a partition $0=t_0 < t_1 < \dotsb < t_n < t_{n+1}=\infty$ is given and define $\eta : S^{n+1} \to D([0,\infty);S)$ by
\begin{align*}
\eta(x_0, \dotsc, x_n)(t) = \sum_{j=0}^n x_j \characteristic{[t_j,t_{j+1})}(t)
\end{align*}
By considering the $\lambda(t) = t$ we see that $d(\eta(x_0, \dotsc, x_n), \eta(y_0, \dotsc, y_n)) \leq \max_{0 \leq j \leq n} r(x_j,y_j)$ which shows that $\eta$ is continuous.  Since
$S$ is separable we know that $\mathcal{B}(S^{n+1}) = \mathcal{B}(S)^{\otimes n+1}$ and therefore $\eta$ is Borel measurable as well.  Now let $g \in D([0,\infty); S)$ be fixed and consider $d(\eta \circ (\pi_{t_0}, \dotsc, \pi_{t_n}), g) : D([0, \infty) ; S) \to \reals$ which is therefore $\sigma(\pi_t ; 0 \leq t < \infty)$-measurable.  Now apply this construction to
the sequence of partitions $t^m_j = j/m$ for $j=0, \dotsc, m^2$ and $m \in \naturals$, letting $\eta_m$ be the $m^{th}$ constructed embedding.
\begin{clm}$\lim_{m \to \infty} d(\eta_m \circ (\pi_{t^m_0}, \dotsc, \pi_{t^m_{m^2}}), g) = d(\cdot, g)$.
\end{clm}
Suppose $\epsilon > 0$ and $T > 0$ be given.  By Lemma \ref{SkorohodInfiniteJ1ModulusOfContinuity} we may find a partition $0=t_0 < t_1 < \dotsb < t_{m-1} < T \leq t_m$ such that
$\max_{1 \leq i \leq m} \sup_{t_{i-1} \leq s < t < t_i} r(f(s), f(t)) < \epsilon$.  Pick $N \in \naturals$ such that $1/N < \min_{1 \leq i \leq m} (t_i - t_{i-1}) \wedge 1/t_m$ and for every $n \geq N$ and $1 \leq i \leq m$ we let $j(i,n)$ be the integer such that $\frac{j(i,n) -1}{n} < t_i \leq \frac{j(i,n)}{n}$.  Define $\lambda_n(j(i,n)/n) = t_i$ for $1 \leq i \leq m$, linearly interpolate in between and $\lambda_n(t) = t_m + (t - j(m,n)/n)$ for $t >  j(m,n)/n$.  For any $0 \leq t \leq T$, we pick $k \in \naturals$ such that $k-1/n \leq t < k/n$ and we have by definition $f_n(t) = f((k-1)/n)$.  Furthermore, since $0 \leq t \leq T$,  we may pick $1 \leq i \leq m$ such that $j(i-1,n) \leq k-1 < j(i,n)$ and it follows that $j(i-1,n)/n \leq t < j(i,n)/n$ hence $t_{i-1} \leq \lambda_n(t)) < t_i$ and also 
\begin{align*}
t_{i-1} \leq \frac{j(i-1,n)}{n} \leq \frac{k-1}{n} \leq  \frac{j(i,n)-1}{n} < t_i
\end{align*} 
Therefore we conclude $r(f_n(t), f(\lambda_n(t))) \leq \sup_{t_{i-1} \leq s < t < t_i} r(f(s), f(t)) < \epsilon$ for all $n \geq N$ and we have shown that $\lim_{n \to \infty} \sup_{0 \leq t \leq T} r(f_n(t), f(\lambda_n(t)) = 0$.
Since $\lambda_n$ is defined to be piecewise linear it follows that $\sup_{0 \leq t \leq T} \abs{\lambda_n(t) - t} = \max_{0 \leq m} \abs{\lambda_n(j(i,n)/n) - j(i,n)/n}= \max_{0 \leq m} \abs{t_i - j(i,n)/n}$.  It is clear by the definition of $j(i,n)$ that $\lim_{n\to \infty} j(i,n)/n = t_i$ and thus it follows that $\sup_{0 \leq t \leq T} \abs{\lambda_n(t) - t} \to 0$.  Proposition \ref{SkorohodInfiniteJ1EquivalenceB} shows this is sufficient to show convergence (recall that we may pick $\lambda_n$ depending on $T$ according to that Lemma).

By the claim, we see that for every $g \in D([0,\infty); S)$ the function $d(\cdot, g)$ is $\sigma(\pi_t ; 0 \leq t < \infty)$-measurable from which it follows that every open
ball $B(g, r) \in \sigma(\pi_t ; 0 \leq t < \infty)$.  Since $S$ is separable, every open set is a countable union of open balls and therefore every open set belongs to $\sigma(\pi_t ; 0 \leq t < \infty)$ and we are done.
\end{proof}

\begin{thm}\label{SkorohodInfiniteArzelaAscoliTheoremJ1Topology}Let $(S,r)$ be a complete metric space.  A set $A \subset D([0,\infty); S)$ is relatively compact in the $J_1$ topology if and only if 
\begin{itemize}
\item[(i)]for each rational number
$t \in [0,\infty) \cap \rationals$ there exists a compact set $K_t \subset S$ such that $\cup_{f \in A} f(t) \subset K_t$
\item[(ii)] For all $T>0$, $\lim_{\delta \to 0} \sup_{f \in A} w^\prime(f, \delta, T) = 0$
\end{itemize}
\end{thm}
\begin{proof}
TODO: Not significantly different than the case of $D([0,T];S)$.
\end{proof}

\begin{lem}\label{SkorohodInfiniteTightnessOfJ1Topology}A set of Borel probability measures $\mu_\alpha$ on $D([0,\infty); S)$ is tight if and only if 
\begin{itemize}
\item[(i)]For every $\epsilon >0$ and $t \in [0,\infty) \cap \rationals$ there exists a compact set $K_{\epsilon, t} \subset S$ such that $\sup_\alpha \mu_\alpha(f(t) \in K_{\epsilon, t}) > 1 - \epsilon$.
\item[(ii)] For every $\lambda > 0$ and $T > 0$ 
\begin{align*}
\lim_{\delta \to 0} \sup_\alpha \mu_\alpha(w^\prime(f, \delta, T) \geq \lambda) &= 0
\end{align*}
\end{itemize}
\end{lem}
\begin{proof}
Let $\mu_\alpha$ be tight.  Let $\epsilon > 0$ be given and pick
$K \subset D^\infty([0,\infty); S)$ compact with $\mu_\alpha(K) >
1-\epsilon/2$ for all $\alpha$.  Then by Theorem \ref{SkorohodInfiniteArzelaAscoliTheoremJ1Topology} we know that 
for every $t \in [0,\infty) \cap \rationals$ there exists a compact set $K_{\epsilon,t} \subset S$ such that
$f(t) \in K_{\epsilon,t}$ for every $f \in K$ and therefore by a union bound
\begin{align*}
\sup_\alpha \mu_\alpha(f(t) \in K_{\epsilon,t}) \geq \sup_\alpha \mu_\alpha(K) \geq 1 - \epsilon/2 > 1 - \epsilon
\end{align*}

Similarly applying Theorem \ref{SkorohodInfiniteArzelaAscoliTheoremJ1Topology} we know that for
every $T > 0$ and $\lambda>0$
there exists $\delta>0$ such that $\sup_{f \in K} w^\prime(f, \delta, T) <
\lambda$.  Therefore $\lbrace f \mid w^\prime(f, \delta, T) \geq \lambda \rbrace
\subset K^c$ and by a union bound applied for every $\alpha$ we have $\sup_\alpha \mu_\alpha(w^\prime(f, \delta, T) \geq  \lambda) 
\leq \mu_\alpha(K^c) < \epsilon$.  Since $w^\prime(f, \delta, T)$ is a non-decreasing function of $\delta$ this it follows that for all $0 < \rho \leq \delta$,
\begin{align*}
\sup_\alpha \mu_\alpha(w^\prime(f, \rho, T) \geq  \lambda)  \leq \sup_\alpha \mu_\alpha(w^\prime(f, \delta, T) \geq  \lambda)  < \epsilon
\end{align*}
and we have shown (ii).

Now assume that (i) and (ii) hold and suppose that $\epsilon > 0$ is
given.  Let $q_1, q_2, \dotsc$ be an enumeration of $t \in [0,\infty) \cap \rationals$.  By (i) for every $q_M$ there exists compact $K_{\epsilon, M} \subset S$ such that $\sup_{\alpha}
\mu_{\alpha} (f(q_M) \notin K_{\epsilon, M}) < \epsilon/2^{M+1}$.  By (ii)
for every $N,k \in \naturals$, there exists a $\delta_{N,k}$
such that $\sup_{\alpha} \mu_\alpha(w^\prime(f, \delta_{N,k}, N)\geq 1/k)< \epsilon/2^{N+k+1}$.  If we define 
\begin{align*}
A_N &= \lbrace f \mid 
w^\prime(f,\delta_{N,k}, N) < 1/k \text{ for all } k \geq 1\rbrace
\end{align*}
so that $A^c_N \subset \cup_{k=1}^\infty \lbrace f \mid w^\prime(f, \delta_{N,k}, N) \geq 1/k \rbrace$ 
then by a union bound
\begin{align*}
\sup_{\alpha} \mu_\alpha(A_N) &= \sup_{\alpha} (1 - \mu_\alpha(A^c_N))\\
&\geq \sup_{\alpha} \left(1 - \sum_{k=1}^\infty \mu_\alpha(w^\prime(f, \delta_{N,k}, N) \geq  1/k) \right ) \\
&\geq 1 - \epsilon/2^{T+1}
\end{align*}
If we define $K = \cap_{M=1}^\infty \lbrace f(q_M) \in K_{\epsilon, M} \rbrace \cap
\cap_{N=1}^\infty A_N$ then another union bound shows 
\begin{align*}
\sup_{\alpha} \mu_\alpha(K^c) &=\sup_{\alpha} \mu_\alpha(\cup_{M=1}^\infty \lbrace f(q_M) \notin K_{\epsilon, M} \rbrace 
\cup
\cup_{N=1}^\infty A^c_N) \\
&\leq \sup_{\alpha} \left[ \sum_{M=1}^\infty \mu_\alpha( f(q_M) \notin K_{\epsilon, M})
+
\sum_{N=1}^\infty \mu_\alpha(A^c_N) \right] \\
&\leq \sum_{M=1}^\infty \epsilon/2^{M+1} + \sum_{N=1}^\infty \epsilon/2^{T+1} = \epsilon
\end{align*} 
and by construction the set $K$ satisfies the conditions of Theorem \ref{SkorohodInfiniteArzelaAscoliTheoremJ1Topology} so is proven compact.  
\end{proof}
 
TODO: Understand the Ethier and Kurtz construction and its relationship with Lindvall's.  One proposal is that the only difference is the manner in which we are embedding $D([0,\infty))$ in $D_0^\infty$ (accounting for the lack of linearity).  Here we define for each $n \in \naturals$ the map $c_n : D([0,\infty)) \to D([0,\infty])$ by $c_n(f)(t) = f(t \wedge n)$ and then map $f$ to $\Psi(f) = (c_1(f), c_2(f), \dotsc)$. The key things that we need to have be true are that
\begin{itemize}
\item[(i)] $\Psi$ is an injection
\item[(ii)] $\Psi(D([0,\infty)))$ is closed in $D_0[0,\infty]^\infty$.
\end{itemize}

It is obvious that $\Psi$ is linear and furthermore if $f \neq 0$ the we pick $0 \leq t < \infty$ such that $f(t) \neq 0$ and then $n > t$ and it follows that $c_n(f)(t) = f(t) \neq 0$ thus $\Psi(f) \neq 0$.  

\begin{lem}For each $m \in \naturals$, $c_m$ is continuous on the set 
\begin{align*}
D_0 = \lbrace f \in D([0,\infty);S) \mid \lim_{t \to \infty} f(t) \text{ exists and is finite} \rbrace
\end{align*}  
Furthermore, $\Psi(D)$ is closed.
\end{lem}
\begin{proof}
Suppose $f_n \to f$ in $D_0$.  Then there exist $\lambda_n$ such that $\lim_{n \to \infty} \sup_{0 \leq t < \infty} r(f_n(\lambda_n(t)) , f(t)) = 0$ and $\lim_{n \to \infty} \sup_{0 \leq t < \infty} \abs{\lambda_n(t) - t} = 0$.  We apply the sequence $c_m(f_n)(t) = f_n(t \wedge m)$.  
\begin{align*}
\sup_{0 \leq t < \infty} r(c_m(f_n)(\lambda_n(t)) , c_m(f)(t)) 
&=\sup_{0 \leq t < \infty} r(f_n(\lambda_n(t) \wedge m) , f(t \wedge m)) \\
&=\sup_{0 \leq t < \infty} r(f_n(\lambda_n(t) \wedge m) , f_n(\lambda_n(t \wedge m)))+ r( f_n(\lambda_n(t \wedge m)) , f(t \wedge m)) \\
\end{align*}

Note that 
\begin{align*}
\sup_{0 \leq t < \infty} r( f_n(\lambda_n(t \wedge m)) , f(t \wedge m)) = \sup_{0 \leq t \leq m} r( f_n(\lambda_n(t)) , f(t)) 
\leq \sup_{0 \leq t < \infty} r( f_n(\lambda_n(t)) , f(t)) \to 0
\end{align*}

Oops.  It is not true that $\sup_{0 \leq t < \infty} r(f_n(\lambda_n(t) \wedge m) , f_n(\lambda_n(t \wedge m))) \to 0$ for a counterexample
let $0 < x < \infty$ and define 
\begin{align*}
\lambda_n(t) = \begin{cases}
(1-1/nx) t & 0 \leq t \leq x \\
2t - x - 1/n & x < t < x+1/n \\
t & t \geq x+1/n
\end{cases}
\end{align*}
so that $f_n(\lambda_n(x) \wedge x) = f_n((x - 1/n) \wedge x) = f_n(x-1/n)$ and $f_n(\lambda_n(x \wedge x) = f_n(x)$.  If $f_n$ all have a jump of the same size $\epsilon>0$ at 
$t=x$ it follows that $\sup_{0 \leq t < \infty} r(f_n(\lambda_n(t) \wedge m) , f_n(\lambda_n(t \wedge m))) \geq \epsilon > 0$.

Note that this doesn't imply that $f \to f(\cdot \wedge n)$ isn't a continuous map necessarily we might just need to be more creative in finding the sequence $\lambda_n$.
In fact the example $f = \characteristic{[x, \infty)}$ and $f_n = \characteristic{[x+1/n)}$ provide an example showing that $c_x$ is not continuous.  Can it still be true that $\Psi(D)$ is closed?
\end{proof}

TODO: Lindvall's approach only works with $S$ a linear space since we use multiplication to tamp things down.  Maybe there are some worthwhile exercises to be culled from that.
For each $n \in \naturals$ we define 
\begin{align*}
g_n(t) &= \begin{cases}
1 & 0 \leq t \leq n \\
1 - t - n & n < t < n + 1 \\
0 & t \geq n+1
\end{cases}
\end{align*}
and for every $f \in D([0,\infty)$.

\section{Aldous Criterion}

TODO: How to tie this in with what Ethier and Kurtz claim to be the Aldous criterion.
\begin{thm}Let $X^n$ be a sequence of processes in $D([0,T];S)$ with $(S,r)$ a metric space and suppose for every $\epsilon > 0$ and $t \in [0,T] \cap \rationals$ there exists a compact set $K_{\epsilon,t}$ such that $\probability{X^n_t \in K_{\epsilon,t}} > 1 - \epsilon$, then for every $\lambda > 0$ $X^n$ satisfies $\lim_{\delta \to 0} \sup_n \probability{w(X^n, \delta) \geq \lambda} =0$ if and only if for any sequence of $\mathcal{F}^{X^n}$-optional times $\tau_n$ and any sequence $\delta_n > 0$ with $\lim_{n \to \infty} \delta_n =0$ we have
\begin{align*}
r(X^n_{\tau_n}, X^n_{\tau_n + \delta_n}) \toprob 0
\end{align*}
\end{thm}
\begin{proof}
TODO:
\end{proof}

In practice it turns out that it is difficult to apply the criterion $\lim_{\delta \to 0} \sup_{\alpha} \mu_\alpha( w^\prime(f, \delta, T) \geq \lambda) = 0$ to show tightness of a family of
measures $\mu_\alpha$ on Skorohod space.  In some sense this is not surprising as, being an infimum over a set of partitions, the modulus of continuity is a rather complicated object.  What we need to develop are tools for estimating $w^\prime(f, \delta, T)$ that are strong enough to imply the tightness condition.  One technique for finding upper bounds of $w^\prime(f, \delta, T)$ is clear; one simply needs to find a particular partition $0=t_0 < \dotsb < t_{n-1} < T \leq t_n$ for which we can calculate (or upper bound) each term
$\sup_{t_{i-1} \leq s < t < t_i} r(f(s), f(t))$.  

Our first step is to work pointwise in $D([0,\infty); S)$ and show how create a useful partition for an arbitrary cadlag function $f$.  For the construction we assume that $\epsilon > 0$ and $f \in D([0,\infty); S)$ are both given.  First define inductively $\tau_0=0$ and $n \in \naturals$
\begin{align*}
\tau_{n} &= \begin{cases}
\inf \lbrace t > \tau_{n-1} \mid r(f(t), f(\tau_{n-1})) > \epsilon/2 \rbrace & \text{if $\tau_{n-1} < \infty$} \\
\infty & \text{if $\tau_{n-1} = \infty$}
\end{cases}
\end{align*}
and then define for $n \in \integers_+$
\begin{align*}
\sigma_{n} &= \begin{cases}
\sup \lbrace t \leq \tau_{n} \mid r(f(t), f(\tau_{n})) \vee r(f(t-), f(\tau_{n}))   \geq \epsilon/2 \rbrace & \text{if $\tau_n < \infty$} \\
\infty & \text{if $\tau_n = \infty$}
\end{cases}
\end{align*}

Note that by right continuity of $f$ we have $r(f(\tau_n), f(\tau_{n-1})) \geq \epsilon/2$ whenever $\tau_n < \infty$ (in particular $\tau_{n-1} < \tau_n$ whenever $\tau_{n-1} < \infty$).

\begin{clm}Let $\delta > 0, T>0$ be given. If $w^{\prime}(f,\delta, T) < \epsilon/2$  then $\min \lbrace \tau_{n+1} - \sigma_n \mid \tau_n < T \rbrace > \delta$.  
\end{clm}
The claim is verified by contradiction, so suppose that we have $\tau_n < T$ and $\tau_{n+1} - \sigma_n \leq \delta$ for some $n \in \integers_+$.  If we are given a partition
$0=t_0 < \dotsb t_{m-1} < T \leq t_m$ with $\min_{1 \leq i \leq m} (t_i-t_{i-1}) > \delta$ it follows that there is some $1 \leq i \leq m$ such that $t_{i-1} \leq \tau_n < \tau_i$.  If $\sigma_n \leq t_{i-1} < t_i \leq \tau_{n+1}$ then $\tau_{n+1} - \sigma_n \geq t_i - t_{i-1} > \delta$ which is a contradiction therefore either $t_{i-1} < \sigma_n \leq \tau_n < t_i$ or 
$t_{i-1} \leq \tau_n < \tau_{n+1} < t_i$ or both.  In the first case by definition of $\sigma_n$ we can find a $t_{i-1} < u \leq \sigma_n$ such that $r(f(u), f(\tau_n)) \geq \epsilon/2$ and in the second case we have already observed $r(f(\tau_n), f(\tau_{n-1})) \geq \epsilon/2$; thus $\max_{1 \leq i \leq m} \sup_{t_{i-1} \leq s < t < t_{i}} r(f(s), f(t)) \geq \sup_{t_{i-1} \leq s < t < t_{i}} r(f(s), f(t)) \geq \epsilon/2$.  Now take the infimum over all partitions.

Next we define the partition that will generate our upper bound on the modulus of continuity.
\begin{align*}
s_n &= \frac{\sigma_n + \tau_n}{2}
\end{align*}
and note that
\begin{align*}
\sigma_n &\leq s_n \leq \tau_n \leq \sigma_{n+1} \leq s_{n+1} \leq \tau_{n+1}
\intertext{and}
s_{n+1} - s_n &= 
\frac{\sigma_{n+1} + \tau_{n+1}}{2} - \frac{\sigma_n + \tau_n}{2} \geq 
\frac{\tau_{n} + \tau_{n+1}}{2} - \frac{\sigma_n + \tau_n}{2} = 
\frac{\tau_{n+1} - \sigma_n}{2} 
\end{align*}

\begin{clm}For every $\delta >0, T>0$ if $\min \lbrace \tau_{n+1} - \sigma_n \mid \tau_n < T + \delta/2 \rbrace > \delta$ then 
$\min \lbrace s_{n+1} - s_n \mid s_n < T \rbrace > \delta/2$
\end{clm}
We argue by contradiction, so suppose that $s_n < T$ and $s_{n+1} - s_n \leq \delta/2$.  Then $\tau_n \leq s_{n+1} \leq s_n + \delta/2 < T + \delta/2$ and 
\begin{align*}
\tau_{n+1} - \sigma_n &\leq 2(s_{n+1} - s_n) \leq \delta
\end{align*}

\begin{clm}For every $\delta >0, T>0$ if $\min \lbrace s_{n+1} - s_n  \mid s_n < T \rbrace > \delta/2$ then 
$w^{\prime}(f, \delta/2, T)  \leq \epsilon$
\end{clm}
The claim follows if we can show $\sup_{s_n \leq s < t < s_{n+1}} r(f(s), f(t)) \leq \epsilon$ for then $0=s_0 < \dotsb < s_{n} < T \leq s_{n+1}$ is a partition which shows that
$w^{\prime}(f, \delta/2, T)  \leq \epsilon$ (recall that $s_n \to \infty$ so there are only finitely many $s_n < T$).
The property $\sup_{s_n \leq s < t < s_{n+1}} r(f(s), f(t)) \leq \epsilon$ follows from the triangle inequality if we can show that for any $s_n \leq s < s_{n+1}$ we have 
$r(f(s), f(\tau_n)) < \epsilon/2$.  To see this last fact we consider two cases.
First assume $s_n \leq \tau_n \leq s < s_{n+1} \leq \tau_{n+1}$ then by the definition of
$\tau_{n+1}$ we know that $r(f(s), f(\tau_n)) < \epsilon/2$.
If on the other hand, $s_n \leq s < \tau_n$ then this implies that $\sigma_n < \tau_n$ hence $\sigma_n < s_n$ and therefore either $\sigma_n < s < \tau_n$
By the definition of $\sigma_n$ we know that $r(f(s), f(\tau_n)) < \epsilon/2$.  

TODO: What are the steps and examples that lead one to considering the definitions of $\sigma_n$ and $s_n$ (the case for the definition of $\tau_n$ is clear).

\begin{clm}If $S$ is separable the $\tau_n$, $\sigma_n$ and $s_n$ are Borel measurable functions on $D([0,\infty); S)$ to $[0,\infty]$.
\end{clm}
TODO:

\begin{lem}\label{SkorohodInfiniteModulusOfContinuityEquivalences}Let $(S,r)$ be a separable metric space, let $A$ be an arbitrary index set and let $X^\alpha$ for $\alpha \in A$ be a family of stochastic processes with values in $D([0,\infty); S)$.  Define $\tau^{\alpha, \epsilon}_n, \sigma^{\alpha, \epsilon}_n$ and $s^{\alpha, \epsilon}_n$ as above then the following are equivalent
\begin{itemize}
\item[(i)]$\lim_{\delta \to 0} \inf_{\alpha \in A} \probability{w^{\prime}(X^\alpha, \delta, T) < \epsilon} = 1$ for all $\epsilon > 0$ and $T > 0$
\item[(ii)]$\lim_{\delta \to 0} \inf_{\alpha \in A} \probability{\min \lbrace \tau^{\alpha, \epsilon}_{n+1} - \sigma^{\alpha, \epsilon}_n \mid \tau^{\alpha, \epsilon}_n < T \rbrace \geq \delta} = 1$ for all 
$\epsilon > 0$ and $T > 0$
\item[(iii)]$\lim_{\delta \to 0} \inf_{\alpha \in A} \probability{\min \lbrace s^{\alpha, \epsilon}_{n+1} - s^{\alpha, \epsilon}_n \mid s^{\alpha, \epsilon}_n < T \rbrace \geq \delta} = 1$ for all 
$\epsilon > 0$ and $T > 0$
\end{itemize}
\end{lem}
\begin{proof}
Let $\epsilon > 0$ and $T > 0$ be given then by (i) we know that $\lim_{\delta \to \infty} \inf_{\alpha \in A} \probability{w^{\prime}(X^\alpha, \delta, T) < \epsilon/2} = 1$.  On the other hand,
for all $\alpha \in A$ we know that 
\begin{align*}
\probability{w^{\prime}(X^\alpha, \delta, T) < \epsilon/2} &\leq 
\probability{\min \lbrace \tau^{\alpha, \epsilon}_{n+1} - \sigma^{\alpha, \epsilon}_n > \delta \mid \tau^{\alpha, \epsilon}_n < T \rbrace > \delta} \\
&\leq
\probability{\min \lbrace \tau^{\alpha, \epsilon}_{n+1} - \sigma^{\alpha, \epsilon}_n > \delta \mid \tau^{\alpha, \epsilon}_n < T \rbrace \geq \delta} \\
\end{align*}
Now take the infimum over $\alpha \in A$ and let $\delta \to 0$ to conclude (ii).

Assume (ii) holds.  Let $\epsilon > 0$ and $T > 0$ be given.  Now pick $T^\prime > T$, let $\eta > 0$ then by (ii) there exists $0 < \delta < 2(T^\prime - T)$ such that 
\begin{align*}
1 - \eta &< \inf_{\alpha \in A} \probability{\min \lbrace \tau^{\alpha, \epsilon}_{n+1} - \sigma^{\alpha, \epsilon}_n \mid \tau^{\alpha, \epsilon}_n < T^\prime \rbrace \geq \delta} \\
&\leq \inf_{\alpha \in A} \probability{\min \lbrace \tau^{\alpha, \epsilon}_{n+1} - \sigma^{\alpha, \epsilon}_n \mid \tau^{\alpha, \epsilon}_n < T^\prime \rbrace > \rho}  \text{ for all $0 < \rho < \delta$}\\
&\leq \inf_{\alpha \in A} \probability{\min \lbrace s^{\alpha, \epsilon}_{n+1} - s^{\alpha, \epsilon}_n \mid s^{\alpha, \epsilon}_n < T^\prime - \rho/2 \rbrace > \rho/2}  \text{ for all $0 < \rho < \delta$} \\
&\leq \inf_{\alpha \in A} \probability{\min \lbrace s^{\alpha, \epsilon}_{n+1} - s^{\alpha, \epsilon}_n \mid s^{\alpha, \epsilon}_n < T \rbrace \geq \rho/2}  \text{ for all $0 < \rho < \delta$} \\
\end{align*}
which shows (iii).

Assume (iii) holds, let $T > 0$ and $\epsilon > 0$ be given.  Let $\eta>0$ be arbitrary and by (iii) pick a $\delta > 0$ such that
\begin{align*}
1 -\eta &< \inf_{\alpha \in A} \probability{\min \lbrace \tau^{\alpha, \epsilon}_{n+1} - \sigma^{\alpha, \epsilon}_n \mid \tau^{\alpha, \epsilon}_n < T \rbrace \geq \delta} \\
&\leq \inf_{\alpha \in A} \probability{\min \lbrace \tau^{\alpha, \epsilon}_{n+1} - \sigma^{\alpha, \epsilon}_n \mid \tau^{\alpha, \epsilon}_n < T \rbrace > \rho} \text{ for all $0 < \rho < \delta$} \\
&\leq \inf_{\alpha \in A} \probability{w^{\prime}(X^\alpha, \rho, T)  \leq \epsilon} \text{ for all $0 < \rho < \delta$} \\
\end{align*}
which shows $\lim_{\delta \to 0} \inf_{\alpha \in A} \probability{w^{\prime}(X^\prime, \rho, T)  \leq \epsilon} = 1$ for all $T > 0$ and $\epsilon > 0$.  This is equivalent to (i) since for any $\alpha \in A$, $T > 0$ and $0 < \epsilon$ we have 
\begin{align*}
\probability{w^{\prime}(X^\alpha, \rho, T)  \leq \epsilon/2} \leq \probability{w^{\prime}(X^\alpha, \rho, T)  < \epsilon} \leq \probability{w^{\prime}(X^\alpha, \rho, T)  \leq 2\epsilon}
\end{align*}
\end{proof}

\begin{lem}\label{SkorohodInfiniteModulusOfContinuityPullOutMin}Let $A$ be an arbitrary index set and for every $\alpha \in A$ let $0=s^\alpha_0 < s^\alpha_1 < \dotsb$ be a sequence of random variables such that $\lim_{n \to \infty} s^\alpha_n = \infty$.  Let $T > 0$ be arbitrary and define $K(\alpha,T) = \max \lbrace n \in \naturals \mid s^\alpha_n < T \rbrace$ and $F : [0,\infty) \to [0,1]$ by $F(t) = \sup_{\alpha \in A} \sup_{n \geq 0} \probability{(s^\alpha_{n+1} - s^\alpha_n) < t , s^\alpha_n < T}$.  Then for all $\delta > 0$ and $L \in \integers_+$ 
\begin{align*}
F(\delta) &\leq \sup_{\alpha \in A} \probability{\min_{0 \leq n \leq K(\alpha, T)} (s^\alpha_{n+1} - s^\alpha_n) < \delta} &\leq LF(\delta) + e^T \int_0^\infty L e^{-Lt} F(t) \, dt
\end{align*}
Therefore
\begin{align*}
\lim_{\delta \to 0} \sup_{\alpha \in A} \probability{\min_{0 \leq n \leq K(\alpha, T)} (s^\alpha_{n+1} - s^\alpha_n) } &= 0
\end{align*}
in and only if $F(0+) = 0$.
\end{lem}
\begin{proof}
For each fixed $\alpha \in A$ and every $n \in \naturals$, 
\begin{align*}
\lbrace (s^\alpha_{n+1} - s^\alpha_n) < \delta , s^\alpha_n < T \rbrace &\subset \lbrace \min_{0 \leq n \leq K(\alpha, T)} (s^\alpha_{n+1} - s^\alpha_n) < \delta \rbrace
\end{align*}
and therefore a union bound, taking the supremum over $n \in \naturals$ and $\alpha \in A$ yields the first inequality.

To see the second inequality, 
\begin{align*}
&\probability{\min_{0 \leq n \leq K(\alpha, T)} (s^\alpha_{n+1} - s^\alpha_n) < \delta} \\
&=\probability{\min_{0 \leq n \leq K(\alpha,T)} (s^\alpha_{n+1} - s^\alpha_n) < \delta, K(\alpha, T) \leq L-1} + \probability{\min_{0 \leq n \leq K(\alpha,T)} (s^\alpha_{n+1} - s^\alpha_n) < \delta, K(\alpha, T) \geq L} \\
&\leq \sum_{n=0}^{L-1} \probability{(s^\alpha_{n+1} - s^\alpha_n) < \delta, s^\alpha_n < T}  + \probability{K(\alpha, T) \geq L} \\
&\leq L F(\delta) + \expectation{e^{T - \sum_{n=0}^{L-1} (s^\alpha_{n+1} - s^\alpha_n)} ; K(\alpha, T) \geq L} \\
&\leq L F(\delta) + e^T \Pi_{n=0}^{L-1} \expectation{e^{-L(s^\alpha_{n+1} - s^\alpha_n)}; K(\alpha, T) \geq L}^{1/L} \\
&\leq L F(\delta) + e^T \Pi_{n=0}^{L-1} \expectation{e^{-L(s^\alpha_{n+1} - s^\alpha_n)}; s^\alpha_k < T}^{1/L} \\
\end{align*}
Here there is a multivariate generalization of Cauchy Schwartz used I think ($\expectation{\Pi_{k=1}^L f_k}^L \leq \Pi_{k=1}^L \expectation{f^L_k}$ for $f_k \geq 0$; TODO: prove this, instead of Young's inequality use the AMGM inequality).
\end{proof}

The last two lemmas can be combined to yield the following additional equivalent criterion for equicontinuity of a family of stochastic processes.
\begin{prop}Let $(S,r)$ be a separable metric space, let $A$ be an arbitrary index set and let $X^\alpha$ for $\alpha \in A$ be a family of stochastic processes with values in $D([0,\infty); S)$.  Define $\tau^{\alpha, \epsilon}_n, \sigma^{\alpha, \epsilon}_n$ and $s^{\alpha, \epsilon}_n$ as above then the following are equivalent
\begin{itemize}
\item[(i)]$\lim_{\delta \to 0} \inf_{\alpha \in A} \probability{w^{\prime}(X^\alpha, \delta, T) < \epsilon} = 1$ for all $\epsilon > 0$ and $T > 0$
\item[(ii)]$\lim_{\delta \to 0} \sup_{\alpha \in A} \sup_{n \geq 0} \probability{\tau^{\alpha, \epsilon}_{n+1} - \sigma^{\alpha, \epsilon}_n < \delta, \tau^{\alpha, \epsilon}_n < T} = 0$ \text{ for all $\epsilon > 0$ and $T>0$}
\end{itemize}
\end{prop}
\begin{proof}
\begin{clm} For all $\epsilon > 0$, $T > 0$, $\delta > 0$ and $\alpha \in A$,
\begin{align*}
\probability{ s^{\alpha, \epsilon}_{n+1} - s^{\alpha, \epsilon}_n < \delta/2, s^{\alpha, \epsilon}_n < T } 
&\leq \probability{ \tau^{\alpha, \epsilon}_{n+1} - \sigma^{\alpha, \epsilon}_n < \delta, \tau^{\alpha, \epsilon}_n < T + \delta} 
\leq \probability{s^{\alpha, \epsilon}_{n+1} - s^{\alpha, \epsilon}_n < \delta, s^{\alpha, \epsilon}_n < T + \delta} 
\end{align*}
\end{clm}

Recalling the defintion $s^{\alpha, \epsilon}_n = (\tau^{\alpha, \epsilon}_{n} + \sigma^{\alpha, \epsilon}_n)/2$ we see that if $s^{\alpha, \epsilon}_{n+1} - s^{\alpha, \epsilon}_n < \delta/2$ and $s^{\alpha, \epsilon}_n < T$ then it follows from $\tau_n \leq \sigma_{n+1}$ that 
\begin{align*}
\tau^{\alpha, \epsilon}_{n+1} - \sigma^{\alpha, \epsilon}_{n} &\leq  \tau^{\alpha, \epsilon}_{n+1} - \sigma^{\alpha, \epsilon}_{n} + \sigma^{\alpha, \epsilon}_{n+1} - \tau^{\alpha, \epsilon}_{n}  \\
&= 2 (s^{\alpha, \epsilon}_n  - \sigma^{\alpha, \epsilon}_n)  < \delta
\end{align*}
and from $\tau^{\alpha, \epsilon}_n \leq  s^{\alpha, \epsilon}_{n+1}$,
\begin{align*}
\tau^{\alpha, \epsilon}_n &< s^{\alpha, \epsilon}_{n} + \delta/2 < T + \delta
\end{align*}
Further if $\tau^{\alpha, \epsilon}_{n+1} - \sigma^{\alpha, \epsilon}_{n} < \delta$ and $\tau^{\alpha, \epsilon}_n < T + \delta$ then we have 
\begin{align*}
s^{\alpha, \epsilon}_{n+1} - s^{\alpha, \epsilon}_n &\leq \tau^{\alpha, \epsilon}_{n+1} - \sigma^{\alpha, \epsilon}_{n} < \delta
\end{align*}
and $ s^{\alpha, \epsilon}_n \leq  \tau^{\alpha, \epsilon}_n < T + \delta$.
The claim follows from the union bound implied by these set inclusions.

Now if we assume (i) then $\epsilon$ and $T > 0$ be given.  By Lemma \ref{SkorohodInfiniteModulusOfContinuityEquivalences} we know that 
\begin{align*}
1 &= \lim_{\delta \to 0} \inf_{\alpha \in A} \probability{\min \lbrace s^{\alpha, \epsilon}_{n+1} - s^{\alpha, \epsilon}_n \mid s^{\alpha, \epsilon}_n < T \rbrace \geq \delta} \\
&=1 - \lim_{\delta \to 0} \sup_{\alpha \in A} \probability{\min \lbrace s^{\alpha, \epsilon}_{n+1} - s^{\alpha, \epsilon}_n \mid s^{\alpha, \epsilon}_n < T \rbrace < \delta} 
\end{align*}
thus by the claim and Lemma \ref{SkorohodInfiniteModulusOfContinuityPullOutMin} we know that 
\begin{align*}
\lim_{\delta \to 0} \sup_{\alpha \in A} \sup_{n \geq 0} \probability{(\tau^{\alpha, \epsilon}_{n+1} - \sigma^{\alpha, \epsilon}_n) < \delta , \tau^{\alpha, \epsilon}_n < T} 
&\leq \lim_{\delta \to 0} \sup_{\alpha \in A} \sup_{n \geq 0} \probability{(s^{\alpha, \epsilon}_{n+1} - s^{\alpha, \epsilon}_n) < \delta , s^{\alpha, \epsilon}_n < T} = 0\\
\end{align*}

If we have (ii) then by the claim,
\begin{align*}
\lim_{\delta \to 0} \sup_{\alpha \in A} \sup_{n \geq 0} \probability{ s^{\alpha, \epsilon}_{n+1} - s^{\alpha, \epsilon}_n < \delta, s^{\alpha, \epsilon}_n < T } 
&\leq \lim_{\delta \to 0} \sup_{\alpha \in A} \sup_{n \geq 0} \probability{ s^{\alpha, \epsilon}_{n+1} - s^{\alpha, \epsilon}_n < 2 \delta, s^{\alpha, \epsilon}_n < T + 2\delta}  \\
&\leq \lim_{\delta \to 0} \sup_{\alpha \in A} \sup_{n \geq 0} \probability{ s^{\alpha, \epsilon}_{n+1} - s^{\alpha, \epsilon}_n < \delta, s^{\alpha, \epsilon}_n < T +1 } =0 \\
\end{align*}
and then applying Lemma \ref{SkorohodInfiniteModulusOfContinuityPullOutMin} and  Lemma \ref{SkorohodInfiniteModulusOfContinuityEquivalences}  we get (i).
\end{proof}



\chapter{Feller Processes}

We now specialize to the case of time homogeneous Markov processes and
develop an approach that allow one to bring powerful tools of
functional analysis to bear on the theory of Markov processes and
ultimately elucidates a deep connection between Markov processes and
partial differential equations.  Any treatment of this topic must make several pedagogical
decisions.  The functional analysis tools we will use are part of the theory of operator
semigroups and one could simply assume the reader has been exposed to them and quote the
results with appropriate references.  We deem such an approach an undue burden on the reader
as any treatment of semigroups is likely deeply embedded in a textbook in which the core results
and difficult to extract efficiently (much of semigroup theory is motivated by differential equations and
not probability theory).  Thus we have the choice of how to present the required functional
analysis.  One choice is to present the results in a separate chapter or appendix and the other is to 
present the results on an as needed basis.  While we have relegated the basic theory of Banach spaces to
an appendix, we have chosen the second path for the theory of operator semigroups hoping that the 
probabilistic development can provide motivation for the functional analysis and make it
easier to digest.  The disadvantage in doing things this way is that functional analysis results become spread
out thinly through the text and that a reader looking for a particular result cannot find it without knowing or guessing
the places that it might be used.  We accept this disadvantage in hopes that the spirit of the interaction between
the fields can be better appreciated.

TODO:
\begin{itemize}
\item Chapman Kolmogorov relation is equivalent to semigroup property
\item Feller process defined in terms of Feller semigroup properties
\item Feller semigroup generators
\item Feller semigroup strongly continuous (seen via Yosida approximation)
\item Kolmogorov forward/backward equations follow from strong continuity
\item Generators of strongly continuous semigroups and Feller semigroups characterized (Hille-Yosida)
\item Convergence/Approximation of semigroups in terms of generators
\item Convergence of Markov processes in terms of convergence of semigroups/generators
\item Every Feller semigroup has an associated cadlag Feller process (approximation by pure jump-type processes or by Kinney regularization)
\item Approximation of Feller semigroup by Markov chains
\item Continuous sample paths and elliptic generators
\end{itemize}

\section{Semigroups and Generators}

The first step is to change the point of view on transition kernels
slightly.  In the case of a time homogeneous Markov process, the
family of transition kernels is a single parameter family of kernels
$\mu_t$.  Note that in the discrete time case it is clear that the
entire family of kernels is generated by the single time unit kernel
$\mu = \mu_1$ via kernel multiplication $\mu_n = \mu^{n}$ (in
the case of discrete time Markov chains this is just matrix multiplication).  The
first question that we will pursue is whether there is an analogy in
the continuous time case.  The Chapman Kolmogorov relation gives us a
hint on how to proceed.  In the time homogeneous case the Chapman
Kolmogorov relation says that $\mu_s \mu_t = \mu_{s+t}$ which is the
\emph{semigroup property} and suggests that we may be able to write
$\mu_s$ as $exp(s A)$ for some appropriately defined $A$.  With
some additional assumptions this may be done, but first we want to
recast the transition kernels in a different light in which these
questions may be more naturally resolved.  Let $f$ be a measurable
function on $S$ that is either non-negative or bounded.  For any
probability kernel $\mu : S \to \mathcal{P}(S)$, by
Lemma \ref{KernelTensorProductMeasurability} we know that $\int f(t)
\, \mu(s,dt)$ is a itself a measurable function of $s$ that is non-negative
or bounded when $f$ is.  Thus if we are given the transition kernels
of a time homogeneous Markov process we may define an operator 
\begin{align*}
T_tf (s) &= \int f(u) \, \mu_t(s, du) 
\end{align*} 
on an appropriate space of measurable
functions to itself (say the space of bounded measurable functions).  The first thing to observe is that the
Chapman-Kolmogorov relations are equivalent to the semigroup property
for these operators.

\begin{prop}\label{SemigroupsAndChapmanKolmogorov}Let $\mu_t$ for $t \geq 0$ be a family of probability kernels on a measurable space $(S, \mathcal{S})$ and define 
$T_t f(x) = \int f(s) \, \mu_t(x, ds)$ for all bounded measurable function $f : S \to \reals$, then $\mu_t$ satisfies the Chapman-Kolmogorov relations if and only
if $T_t T_s = T_{t + s}$ for all $t,s \geq 0$.
\end{prop}
\begin{proof}
Let $A \in \mathcal{S}$ then we have $T_{t+s} \characteristic{B}(x) =  \mu_{t+s} (x, B)$ and
\begin{align*}
T_t T_s \characteristic{B} (x) &= \int T_s \characteristic{B} (y) \, \mu_t(x, dy) =\int \mu_s(y, B) \, \mu_t(x, dy) \\
&= \mu_t \mu_s (B)
\end{align*}
therefore the Chapman-Kolomogorov relations are equivalent to $T_t T_s \characteristic{B} = T_{t+s} \characteristic{B}$ for all $B \in \mathcal{S}$.  Therefore the semigroup property implies the Chapman-Kologorov relations and if the the Chapman-Kolmogorov relations hold then the semigroup property holds for simple functions by linearity.   If $f$ is positive, bounded and measurable then find simple functions $f_n \uparrow f$ then by Monotone Convergence 
\begin{align*}
T_t f(x) &= \int f(s) \, \mu_t(x, ds) = \lim_{n \to \infty} \int f_n(s) \, \mu_t(x, ds) = \lim_{n \to \infty} T_t f_n(x)
\end{align*}
Moreover $T_t f_n (x)$ is increasing by positivity of integral and therefore another application of Monotone Convergence shows
\begin{align*}
T_t T_s f(x) &= \lim_{n \to \infty} T_t T_s f_n (x) = \lim_{n \to \infty} T_{t + s} f_n (x) = T_{t+s} f(x)
\end{align*}
The semigroup property extends to arbitrary bounded measurable $f$ by writing $f = f_+ - f_-$ with $f_\pm \geq 0$ and using linearity of $T_t$.
\end{proof}

In the case that the kernels $\mu_t$ are the transition kernels of a Markov process we will call the semigroup $T_t$ the transition semigroup of the Markov process.  It is worth
recording for future use the expression for the transition semigroup in terms of the underlying Markov process.
\begin{prop}\label{TransitionSemigroupAsExpectation}Let $X$ be a homogeneous Markov process with time scale $\reals_+$ and state space $S$ then for every bounded measurable function $f : S \to \reals$ and $s \geq 0$,
\begin{align*} 
T_t f (X_s) &= \cexpectationlong{X_s}{f(X_{t+s})} = \cexpectationlong{\mathcal{F}_s}{f(X_{t+s})} 
\end{align*}
in particular, if $X_t$ is a Markov family (e.g. $X$ is canonical) then
\begin{align*} 
T_t f (x) &= \sexpectation{f(X_t)}{x}
\end{align*}
\end{prop}
\begin{proof}
By definition we know that $\cprobability{X_s}{X_{t+s} \in \cdot} (\omega) = \mu_t(X_s(\omega), \cdot)$ and therefore by distintigration (Theorem \ref{Disintegration}) we have 
\begin{align*}
\cexpectationlong{X_s}{f(X_{t+s})}(\omega) &= \int f(s) \mu_t(X_s(\omega), ds) = T_t f(X_s)
\end{align*}
If $X$ is a Markov family then for $x \in S$ under the measure $P_x$ we have by the Markov property (Theorem \ref{StrongMarkovPropertyMarkovProcessCountableValues}) and the
first part of this result
\begin{align*}
\sexpectation{f(X_t)}{x} &= \cexpectationlong{X_0}{f(X_{t})} = T_t f(X_0) = T_t f(x)
\end{align*}
\end{proof}

\begin{prop}Let $X$ be a pure jump-type Markov process with state space $S$ and a bounded rate kernel $\alpha$.  For every bounded measurable function $f : S \to \reals$ define
\begin{align*}
Af (x) &= \int (f(y) - f(x)) \, \alpha(x, dy)
\end{align*}
then $A$ is a bounded linear operator from $B(S) \to B(S)$ and $T_t = e^{tA}$.
\end{prop}
\begin{proof}
Linearity of $A$ is immediate from the linearity of the integral.  Write $\alpha(x,\cdot) = c(x) \mu(x, \cdot)$ so that $Af(x) = c(x) \int (f(y) - f(x)) \, \mu(x, dy) = c(x) (\int f(y) \, \mu(x, dy) - f(x))$ and the boundedness of $\alpha$ means that $\norm{c}_\infty < \infty$.  Now given $f \in B(S)$, the fact that $\alpha$ is a kernel (Lemma \ref{MeasurabilityRateKernel}) and Lemma \ref{KernelTensorProductMeasurability} imply that $Af$ is a measurable function on $S$.  In addition, we have $\abs{Af(x)} \leq \abs{c(x)} \int \abs{f(y) - f(x)} \mu(x, dy) \leq 2 \norm{c}_\infty \norm{f}_\infty$ which shows that $Af \in B(S)$ and moreover shows upon taking the supremum over $x \in S$ that $\norm{Af} \leq 2 \norm{c}_\infty \norm{f}_\infty$ so that $A$ is a bounded operator with $\norm{A} \leq 2 \norm{c}_\infty$.  

TODO:  Finish.  Kallenberg's proof is very elegant as it reduces to the case in which $c(x)$ is constant and therefore we have a pseudo-Poisson process.  I haven't proven that result as it is tied up a bit in my confusion about how to sort out Markov families.  I should try to come up with a proof that avoids the reduction and just uses the Strong Markov property.
\end{proof}

Given a general operator semigroup $T_t$ we can think a bit about how find an operator $A$ such that $T_t = e^{tA}$.  There are a couple of ways to proceed, but perhaps the simplest is to observe that looking at the equality pointwise in $B(S)$  and formally differentiating at $t=0$ we'd get $\lim_{t \to 0} \frac{T_tf - f}{t} = Af$.  The trick in making this formal is to not assume that the resulting $A$ will be defined for all $B(S)$.
\begin{defn}Let $T_t$ be an operator semigroup on a Banach space $X$ then the \emph{generator} is the operator $A$ defined by $A v = \lim_{t \to 0} \frac{T_tv -v}{t}$ for all $v \in X$ 
for which the limit on the right exists.  
\end{defn}
We reiterate that $A$ is not necessarily defined everywhere and that part of the definition of the generator is its domain of definition.  Unless it is explicity noted otherwise we will use $\domain{A}$ to denote the domain of a generator.

It is trivial to see the following.
\begin{prop}The generator is a linear operator on its domain of definition.
\end{prop}
\begin{proof}
This follows directly from linearity of the $T_t$ and linearity of limits.  If $v,w \in \domain{A}$ and $a \in \reals$ then
\begin{align*}
\lim_{t \to 0} \frac{T_t av - av}{t} &= a \lim_{t \to 0} \frac{T_t v - v}{t} = aAv \\
\lim_{t \to 0} \frac{T_t (v+w) - (v+w)}{t} &= \lim_{t \to 0} \frac{T_t v - v}{t} + \lim_{t \to 0} \frac{T_t w - w}{t}= Av + Aw \\
\end{align*}
\end{proof}

\section{Strongly Continuous and Feller Semigroups}

We begin the process of seeing how the theory of operator semigroups and Markov processes interacts from the semigroup point of view.  We shall enumerate some properties of
semigroups that make them well behaved in important ways and then define an associated class of Markov processes  as those whose semigroup posess these properties.  

\begin{defn}A \emph{semigroup of operators} $T_t$ on a Banach space $X$ is one parameter family of bounded continuous operators $T_t : X \to X$ such that
for all $0 \geq s,t < \infty$ we have $T_s \circ T_t = T_{s+t}$.
\end{defn}

\begin{defn}A semigroup of operators $T_t$ on a Banach space $X$ is said to be \emph{strongly continuous} if $\lim_{t \to 0} T_tv = v$ for all $v \in X$.
\end{defn}

\begin{examp}\label{ExponentialOfBoundedStronglyContinuousSemigroup}Let $A : X \to X$ be a bounded linear operator then $T_t = e^{tA}$ is a strongly continuous semigroup.  Strong continuity follows from the much stronger property that
$e^{tA}$ is a continuous function $\reals$ into $L(X)$.
\end{examp}

In fact paths of strongly continuous semigroups are continuous; to see this we first need the following result.
\begin{lem}\label{StronglyContinuousSemigroupNormBound}Let $T_t$ be a strongly continuous semigroup on a Banach space $X$ then there exists constants $M \geq 1$ and $c > 0$ such that 
$\norm{T_t} \leq M e^{ct}$.
\end{lem}
\begin{proof}
\begin{clm}There exists a $t_0 > 0$ and $M \geq 1$ such that $\norm{T_s} \leq M$ for all $0 \leq s \leq t_0$
\end{clm}
Suppose that the claim is not true then clearly we can find $0 \leq t_1 \leq 1$ such that $\norm{T_{t_1}} > 1$; since $\norm{T_0} = 1$ we actually know that $t_1 > 0$.  If we have $0 \leq t_n \leq \dotsb \leq t_1$ with $t_j \leq 1/j$ and $T_{t_j} \geq j$ for $j=1, \dotsc, n$ then there must also be $0 \leq t_{n+1} \leq t_{n} \wedge 1/(n+1)$ such that $\norm{T_{t_{n+1}}} > n+1$; as before since $\norm{T_0} = 1$ we know that $0 < t_{n+1}$.  In this way,  we find a sequence $t_n \downarrow 0$ for which $\norm{T_{t_n}} \geq n$.  Now by the Principle of Uniform Boundedness Theorem \ref{PrincipleOfUniformBoundedness} we conclude that $T_{t_n}$ are not pointwise bounded so there must exist a $v \in X$ for which $sup_n \norm{T_{t_n} v} = \infty$.  It then follows that $\lim_{n \to \infty} T_{t_n} v \neq v$ which is a contradiction.

Now let $c = t_0^{-1} \ln M$ and for an arbitrary $t \geq 0$ write $t = n t_0 + s$ for $0 \leq s < t_0$ then we have
\begin{align*}
\norm{T_t} &= \norm{T_s \circ T_{t_0}^k} \leq M M^{n} \leq M M^{t/t_0} = M e^{\frac{t}{t_0} \ln M} = M e^{ct}
\end{align*}
\end{proof}

\begin{prop}\label{StronglyContinuousSemigroupContinuousPaths}Let $T_t$ be a strongly continuous semigroup on a Banach space $X$ then for each $v \in X$ the function $T_t v : [0,\infty) \to X$ is a continuous function.
\end{prop}
\begin{proof}
Let $v \in X$ and $t \geq 0$ then 
\begin{align*}
\lim_{h \downarrow 0} \norm{T_{t+h} v - T_t v} &= \lim_{h \downarrow 0} \norm{T_{t}(T_hv - v)} \leq M e^{ct} \lim_{h \downarrow 0}\norm{T_h v - v} = 0
\end{align*}
for $t > 0$ we have
\begin{align*}
\lim_{h \downarrow 0} \norm{T_{t-h} v - T_t v} &= \lim_{h \downarrow 0} \norm{T_{t-h}(v - T_hv)} \leq M e^{c(t-h)} \lim_{h \downarrow 0}\norm{v - T_h v} = 0
\end{align*}
\end{proof}

Note that if $A$ is a bounded generator then paths $e^{tA}v$ are actually differentiable and are solutions of the abstract Cauchy problem $\frac{d}{dt} f(t) = A f(t)$ with $f(0) = v$.   As it turns out, for an arbitrary strongly continuous contraction semigroup, paths $T_t v$ are differentiable solutions of an abstract Cauchy problem so long as $v \in \domain{A}$.  In this context we may also refer to the differential equation $\frac{d}{dt} T_t v = A T_t v$  as the Kologorov forward equation.  
It also turns out that the Kolmogorov backward equation generalizes to arbitrary strongly continuous semigroups.
\begin{prop}\label{StronglyContinuousSemigroupKolomgorovBackwardEquation}Let $T_t$ be a strongly continuous semigroup on a Banach  space $X$ with generator $A$ then 
\begin{itemize}
\item[(i)] if $v \in X$ and $t \geq 0$ then $\int_0^t T_s v \, ds \in \domain{A}$ and moreover $T_t v - v = A \int_0^t T_sv \, ds$
\item[(ii)] if $v \in \domain{A}$ then $T_t v \in \domain{A}$ for all $t \geq 0$ and moreover $\frac{d}{dt} T_t v = A T_t v = T_t A v$
\item[(iii)] for each $v \in \domain{A}$ and $t \geq 0$ we have $T_t v - v = \int_0^t A T_sv \, ds = \int_0^t T_s A v \, ds$
\end{itemize}
\end{prop}
\begin{proof}
To see (i), by strong continuity of $T_t$, Proposition \ref{StronglyContinuousSemigroupContinuousPaths} implies  that $T_tv$ is continuous hence Riemann integrable on all finite intervals.  The same follows for $T_u T_t v = T_{u+t}v$ thus we may calculate using Proposition \ref{ClosedOperatorOfRiemannIntegral}, a change of integration variable and the Fundamental Theorem of Calculus \ref{FundamentalTheoremOfCalculusForBanachSpaceRiemannIntegrals}
\begin{align*}
A \int_0^t T_sv \, ds &= \lim_{h \to 0}  \frac{T_h \int_0^t T_sv \, ds - \int_0^t T_sv \, ds}{h} \\
&= \lim_{h \to 0}  \frac{\int_0^t T_{h+s} v \, ds - \int_0^t T_sv \, ds}{h} \\
&= \lim_{h \to 0}  \frac{\int_h^{t+h} T_{s} v \, ds - \int_0^t T_sv \, ds}{h} \\
&= \lim_{h \to 0}  \frac{\int_t^{t+h} T_{s} v \, ds - \int_0^h T_sv \, ds}{h} = T_t v - v\\
\end{align*}

To see (ii), note that a trivial consequence of the semigroup property is that for all $s,t \geq 0$ we have $T_s \circ T_t = T_{s+t} = T_t \circ T_s$.  Using this fact and
the continuity of $T_t$ we calculate for any $v \in \domain{A}$ 
\begin{align*}
A T_t v &= \lim_{h \to 0} \frac{T_h T_t v - T_t v}{h} =\lim_{h \to 0} \frac{T_t T_h v - T_t v}{h} \\
&= T_t \lim_{h \to 0} \frac{T_h v - v}{h} = T_t A v
\end{align*}
It remains to show that for $t > 0$ we also have $\lim_{h \to 0} \frac{T_{t-h} v - T_t v}{-h} = T_t A v$.  
We compute
\begin{align*}
\lim_{h \to 0} \frac{T_t v - T_{t-h} v}{h} &= \lim_{h \to 0} T_{t-h} \frac{T_h v - v}{h} 
= \lim_{h \to 0} T_{t-h} \left[\frac{T_h v - v}{h} -Av\right] + \lim_{h \to 0} T_{t-h} Av \\
&=\lim_{h \to 0} T_{t-h} \left[\frac{T_h v - v}{h} -Av\right] + T_t Av
\end{align*}
To see that the first limit on the last line is zero we can use Lemma \ref{StronglyContinuousSemigroupNormBound} and the fact that $v \in \domain{A}$ to see for all $0 < h \leq t$,
\begin{align*}
\lim_{h \to 0} \norm{T_{t-h} \left[\frac{T_h v - v}{h} -Av\right]} &\leq M e^{ct} \lim_{h \to 0} \norm{\frac{T_h v - v}{h} -Av} = 0
\end{align*}

To see (iii) we apply the Fundamental Theorem of Calculus \ref{FundamentalTheoremOfCalculusForBanachSpaceRiemannIntegrals} and (ii) to see that
\begin{align*}
\int_0^t A T_s v \, ds &= \int_0^t A T_s v \, ds = \int_0^t \frac{d}{ds} T_s v \, ds = T_t v - v
\end{align*}
\end{proof}
As a consequence of the fact $\frac{d}{dt} T_t v = A T_t v$ for $v \in \domain{A}$ by analogy it is sometime helpful to think of general paths $T_t v$ as being a type of generalized or weak solution to the forward equation.

\begin{cor}\label{StronglyContinuousSemigroupGeneratorClosedDomainDense}Let $A$ be the generator of a strongly continuous semigroup on the Banach space $X$ then $A$ is a closed operator and $\domain{A}$ is dense.
\end{cor}
\begin{proof}
If we let $v \in X$ then by Proposition \ref{StronglyContinuousSemigroupKolomgorovBackwardEquation} we know that $\int_0^t T_s v \, ds \in \domain{A}$ for all $t \geq 0$ so it follows that $t^{-1} \int_0^t T_s v \, ds \in \domain{A}$ for all $t > 0$.  Now observe that $\lim_{t \to 0} t^{-1} \int_0^t T_s v \, ds = v$ and we see that $\domain{A}$ is dense.

Now suppose that $v_n \in \domain{A}$ and suppose that $\lim_{n \to \infty} v_n = v$ and $\lim_{n \to \infty} A v_n  = w$.  Applying continuity of $T_t$ and Proposition \ref{StronglyContinuousSemigroupKolomgorovBackwardEquation}  we see that 
\begin{align*}
T_t v - v &= \lim_{n \to \infty} (T_t v_n - v_n) = \lim_{n \to \infty} \int_0^t T_s A v_n \, ds = \int_0^t T_s w \, ds
\end{align*}
TODO: Justify exchanging the limit and the integral...

From this is follows that 
\begin{align*}
\lim_{t \to 0} \frac{T_t v - v}{t} = \lim_{t \to 0} t^{-1} \int_0^t T_s w \, ds = w
\end{align*}
which shows that $v \in \domain{A}$ and $Av = w$.
\end{proof}

\subsection{The Hille-Yosida Theorem}

Our next goal is to prove a significant theorem that characterizes completely the unbounded operators $A$ that are generators of strongly continuous contraction semigroups.  In particular, we need to be able to take a given operator $A$ and construct from it the corresponding semigroup $T_t$.  Due to the Kolmogorov Forward Equation in Proposition \ref{StronglyContinuousSemigroupKolomgorovBackwardEquation} one useful way of think about the problem (which is in fact the historical motivation for the theory we present) is that we are trying to construct solutions to the differential equation $\frac{d}{dt} f(t) = A f(t)$ given an unbounded operator $A$.  If one chooses to stress the analogy with the case of a bounded operator $A$, another way of thinking about the task is that we are trying characterize those unbounded operators $A$ for which we can define $e^{tA}$ for $t \geq 0$.  In fact the problem that we are posing is a bit more restricted than we've indicated.  Since we are considering contraction semigroups that means that we are looking for \emph{bounded} solutions to the aforementioned problems.  

TODO : Motivate the study of the resolvent by showing what bad thing would happen if we have a singular value in $(0,\infty)$.  Note that a bounded operator can arbitrary spectrum since the exponential function is entire however if there is positive real spectrum then the contraction property is not obeyed.  We really don't need spectral theory in what follows; essentially we just need a proper definition of a value that \emph{isn't} in the spectrum of an unbounded operator.

\begin{defn}Let $A : X \to X$ be a closed linear operator and the \emph{resolvent set} $\resolventset{A}$ is the set of $\lambda \in \reals$ such that
\begin{itemize}
\item[(i)] $\lambda - A$ is injective on $\domain{A}$.
\item[(ii)] $\range{\lambda - A} = X$.
\item[(iii)] $(\lambda -A)^{-1}$ is a bounded linear operator
\end{itemize}
The operator $R_\lambda = (\lambda -A)^{-1}$ the \emph{resolvent} of $A$.
\end{defn}
Though we don't make use of the concept, the complement of the resolvent set (actually extended to the complex numbers case) is called the spectrum of $A$.

The resolvent operator of the generator of a strongly continuous contraction semigroup is also the Laplace transform of the semigroup.
\begin{prop}\label{SCCSResolventAsLaplaceTransform}Let $T_t$ be a strongly continuous contraction semigroup on $X$ with generator $A$ then $(0,\infty) \subset \resolventset{A}$ and 
for all $0 < \lambda < \infty$ and $v \in X$ we have
\begin{align*}
R_\lambda v &= (\lambda - A)^{-1}v = \int_0^\infty e^{-\lambda t} T_tv \, dt
\end{align*}
and $\norm{R_\lambda} \leq \lambda^{-1}$. 
\end{prop}
\begin{proof}
Let $0 < \lambda < \infty$.  We know that $e^{-\lambda t} T_t v$ is a continuous function of $t$ since $T_t$ is strongly continuous (Proposition \ref{StronglyContinuousSemigroupContinuousPaths}).  Moreover,
$\int_0^\infty \norm{e^{-\lambda t} T_t v} \, dt \leq \norm{v} \int_0^\infty e^{-\lambda t} \, dt \leq \norm{v} \lambda^{-1}$ and therefore $\int_0^\infty e^{-\lambda t} T_tv \, dt$ is well defined
and in fact defines a bounded linear operator $U_\lambda$ with $\norm{U_\lambda} \leq \lambda^{-1}$. 

\begin{clm}For every $0 < \lambda < \infty$ and $v \in X$ we have $\int_0^\infty e^{-\lambda t} T_t v \, dt \in \domain{A}$ and $(\lambda - A) \cdot \int_0^\infty e^{-\lambda t} T_t v \, dt = v$.
\end{clm}
Applying Proposition \ref{ClosedOperatorOfRiemannIntegral}, a change of integration variable, L'Hopital's Rule, the Fundamental Theorem of Calculus (Theorem \ref{FundamentalTheoremOfCalculusForBanachSpaceRiemannIntegrals}) and strong continuity of $T_t$
\begin{align*}
&\lim_{t \to 0} \frac{T_t \cdot \int_0^\infty e^{-\lambda s} T_s v \, ds - \int_0^\infty e^{-\lambda s} T_s v \, ds}{t} \\
&=\lim_{t \to 0} \frac{\int_0^\infty e^{-\lambda s} T_{t+s} v \, ds - \int_0^\infty e^{-\lambda s} T_s v \, ds}{t} \\
&=\lim_{t \to 0} \frac{\int_t^\infty e^{-\lambda (s-t)} T_{s} v \, ds - \int_0^\infty e^{-\lambda s} T_s v \, ds}{t} \\
&=\lim_{t \to 0} \frac{e^{\lambda t} \int_0^\infty e^{-\lambda s} T_{s} v \, ds - \int_0^\infty e^{-\lambda s} T_s v \, ds - e^{\lambda t} \int_0^t e^{-\lambda s} T_{s} v \, ds}{t} \\
&=\lim_{t \to 0} \frac{e^{\lambda t} - 1}{t} \int_0^\infty e^{-\lambda s} T_{s} v \, ds - \lim_{t \to 0} \frac{e^{\lambda t}}{t} \int_0^t e^{-\lambda s} T_{s} v \, ds\\
&= \lambda \int_0^\infty e^{-\lambda s} T_{s} v \, ds - \lim_{t \to 0} \lambda e^{\lambda t} \int_0^t e^{-\lambda s} T_{s} v \, ds - \lim_{t \to 0} e^{\lambda t}  e^{-\lambda t} T_{t} v\\
&= \lambda \int_0^\infty e^{-\lambda s} T_{s} v \, ds - v\\
\end{align*}
which shows the claim.

From the claim, it follows that $(\lambda -A)$ is surjective.  To see that $(\lambda -A)$ is injective on $\domain{A}$ we let $v \in \domain{A}$ and the apply Proposition \ref{StronglyContinuousSemigroupKolomgorovBackwardEquation} and Proposition \ref{ClosedOperatorOfRiemannIntegral} to see
\begin{align*}
\int_0^\infty e^{-\lambda s} T_s A v \, ds &= \int_0^\infty A T_s v \, ds = A \int_0^\infty T_s v \, ds 
\end{align*}
By this fact and the claim, if $(\lambda - A)v = 0$ then
\begin{align*}
0 &= \int_0^\infty e^{-\lambda s} T_s (\lambda - A) v \, ds = (\lambda - A) \int_0^\infty e^{-\lambda s} T_s v \, ds = v
\end{align*}
thus $(\lambda - A)$ is injective.

We know know that $(\lambda -A)^{-1}$ is a well defined linear operator and the claim shows that $(\lambda -A)^{-1} = U_\lambda$.  The rest of the result follows from the properties already proven of $U_\lambda$.
\end{proof}

We now want to derive some simple properties of resolvents.
\begin{prop}\label{SimplePropertiesOfResolvents}Let $A$ be a closed linear operator then
\begin{itemize}
\item[(i)]$(\mu - A)^{-1}(\lambda -A)^{-1} = (\lambda - A)^{-1}(\mu -A)^{-1} = (\lambda - \mu)^{-1}\left( (\mu - A)^{-1} - (\lambda -A)^{-1}\right)$ for all $\mu, \lambda \in \resolventset{A}$. 
\item[(ii)] $(\lambda -A)^{-1} A = A (\lambda -A)^{-1}$ on $\domain{A}$ for all $\lambda \in \resolventset{A}$.
\end{itemize}
\end{prop}
\begin{proof}
To see (i) we first write $(\lambda - \mu) = ((\lambda - A) - (\mu - A))$ and note that the right hand size has domain $\domain{A}$.  Since $\range{(\lambda - A)^{-1}} = \domain{A}$ for $\lambda \in \resolventset{A}$ we can compute
\begin{align*}
(\lambda - \mu) (\mu - A)^{-1}(\lambda -A)^{-1} &= (\mu - A)^{-1}((\lambda - A) - (\mu - A)) (\lambda -A)^{-1} \\
&= (\mu - A)^{-1} - (\lambda - A)^{-1}
\end{align*}

To see (ii) note that $\range{(\lambda -A)^{-1}} = \domain{A}$ so both sides of the equality are well defined operators with the left hand size having with domain $\domain{A}$ and right hand size domain of $X$.  Now
for $v \in \domain{A}$, 
\begin{align*}
(\lambda -A)^{-1} A v &= - (\lambda -A)^{-1} (\lambda - A) v + \lambda  (\lambda -A)^{-1} v \\
&= -v + \lambda  (\lambda -A)^{-1} v \\
&=- (\lambda - A) (\lambda -A)^{-1} v + \lambda  (\lambda -A)^{-1} v = A (\lambda -A)^{-1} v
\end{align*}
\end{proof}

\begin{prop}\label{SCCSResolventSetOpen}Let $T_t$ be a strongly continuous contraction semigroup on $X$ with generator $A$ then $\resolventset{A}$ is open.  For any $\lambda \in \resolventset{A}$ and 
$\abs{\mu - \lambda} < \norm{R_\lambda}^{-1}$ we have $\mu \in \resolventset{A}$ and 
\begin{align*}
R_\mu &= \sum_{n=0}^\infty (\lambda - \mu)^n R_\lambda^{n+1}
\end{align*}
\end{prop}
\begin{proof}
First note that for $\abs{\mu - \lambda} < \norm{R_\lambda}^{-1}$ we have
\begin{align*}
\sum_{n=0}^\infty \abs{\lambda - \mu}^n \norm{R_\lambda}^{n+1} 
&\leq \norm{R_\lambda} \sum_{n=0}^\infty \abs{\lambda - \mu}^n \norm{R_\lambda}^{n} 
= \frac{\norm{R_\lambda}}{1 - \abs{\lambda - \mu} \norm{R_\lambda}}
< \infty
\end{align*} 
and therefore $\sum_{n=0}^\infty (\lambda - \mu)^n R_\lambda^{n+1}$ defines a bounded linear operator.  Let $v \in X$ then for every $m \in \naturals$
$\sum_{n=0}^m  (\lambda - \mu)^n R_\lambda^{n+1} v \in \domain{A}$ therefore we can compute
\begin{align*}
(\lambda - A) \sum_{n=0}^m  (\lambda - \mu)^n R_\lambda^{n+1} v
&= \sum_{n=0}^m  (\lambda - \mu)^n R_\lambda^{n} v
\end{align*}
and 
\begin{align*}
\sum_{n=0}^\infty \abs{\lambda - \mu}^n \norm{R_\lambda}^{n} v &= \frac{\norm{v}}{1 - \abs{\lambda - \mu} \norm{R_\lambda}} < \infty
\end{align*}
Thus $(\lambda - A) \sum_{n=0}^m  (\lambda - \mu)^n R_\lambda^{n+1} v$ converges absolutely and since $\lambda -A$ is closed
(Corollary \ref{StronglyContinuousSemigroupGeneratorClosedDomainDense}) it follows that $\sum_{n=0}^\infty  (\lambda - \mu)^n R_\lambda^{n+1} v \in \domain{A}$ and
\begin{align*}
(\lambda -A) \sum_{n=0}^\infty  (\lambda - \mu)^n R_\lambda^{n+1}v  &= \sum_{n=0}^\infty  (\lambda - \mu)^n R_\lambda^{n} v
\end{align*}
Now we compute for any $v \in X$,
\begin{align*}
(\mu - A) \sum_{n=0}^\infty  (\lambda - \mu)^n R_\lambda^{n+1} v
&= (\mu - \lambda) \sum_{n=0}^\infty  (\lambda - \mu)^n R_\lambda^{n+1} v + \sum_{n=0}^\infty  (\lambda - \mu)^n R_\lambda^{n} v\\
&= -\sum_{n=1}^\infty  (\lambda - \mu)^n R_\lambda^{n} v + \sum_{n=0}^\infty  (\lambda - \mu)^n R_\lambda^{n} v  = v\\
\end{align*}
from which it follows that $\range{\mu - A} = X$.  To see that $\mu - A$ is injective on $\domain{A}$, assume that $v \in \domain{A}$ and  $(\mu - A) v = 0$.   We
apply $\sum_{n=0}^\infty  (\lambda - \mu)^n R_\lambda^{n+1}$ and use the fact that $A$ and $R_\lambda$ commute (Proposition \ref{SimplePropertiesOfResolvents})
\begin{align*}
0 &=\sum_{n=0}^\infty  (\lambda - \mu)^n R_\lambda^{n+1}  (\mu - A)  v =  (\mu - A)  \sum_{n=0}^\infty  (\lambda - \mu)^n R_\lambda^{n+1} v = v
\end{align*}
and this computation also shows that 
\begin{align*}
R_\mu &= (\mu-A)^{-1} = \sum_{n=0}^\infty  (\lambda - \mu)^n R_\lambda^{n+1}
\end{align*}
which we have already shown to be a bounded operator.  Thus $\mu \in \resolventset{A}$.
\end{proof}

\begin{defn}A linear operator $A : X \to Y$ is \emph{dissipative} if $\norm{\lambda v - A v} \geq \lambda \norm{v}$ for every $v \in \domain{A}$ and $\lambda>0$.
\end{defn}

The reason this definition is of interest is the following.
\begin{examp}\label{SCCSGeneratorDissipative}The generator of any strongly continuous contraction semigroup is dissipative. This follows from Proposition \ref{SCCSResolventAsLaplaceTransform} since by that result for any $v \in \domain{A}$ and $\lambda > 0$ 
\begin{align*}
\norm{v} &= \norm{(\lambda - A)^{-1} (\lambda -A) v} \leq \lambda^{-1} \norm{(\lambda -A) v} 
\end{align*}
\end{examp}

We also record the following example of a class of bounded dissipative operators.
\begin{examp}\label{BoundedTranslationDissipative}Let $A : X \to X$ be a bounded linear operator then for any $c \geq \norm{A}$ the operator $A - c$ is dissipative.  This follows from the triangle inequality
\begin{align*}
\norm{(\lambda - (A - c))v} &= \norm{(\lambda + c) v} - \norm{A v} \geq \norm{(\lambda + c) v} - \norm{A} \norm{v}  \geq \lambda \norm{v}
\end{align*}
\end{examp}

It is also useful to note that the set of dissipative operators is a cone.
\begin{examp}\label{DissipativeOperatorsCone}
If $A$ is dissipative and $c > 0$ then $c A$ is dissipative since 
\begin{align*}
\norm{(\lambda - c A) v} &= \norm{(\lambda/c - A) c v} \geq \lambda/c \norm{cv} = \lambda \norm{v}
\end{align*}
\end{examp}

Now we turn to less trivial matters.  It turns out that the property of being dissipative is a key part of the characterization of generators of strongly continuous semigroups.  As a step in the direction of proving such a result we first study the resolvent set properties of dissipative operators.

\begin{prop}\label{DissipativeAndClosed}Let $A$ be dissipative and $\lambda > 0$ then $A$ is closed if and only if $\range{\lambda - A}$ is closed.  Thus if $A$ is dissipative and $\range{\lambda - A}$ is closed for a single value of $\lambda > 0$ it is closed for all $\lambda > 0$.
\end{prop}
\begin{proof}
Let $A$ be closed.  Suppose $\lambda>0$ and $(\lambda -A) v_n$ converges, then it is Cauchy.  Let $\epsilon > 0$ be given.  We may find $N > 0$ such that $\norm{(\lambda -A)(v_n-v_m)} \leq \epsilon$ for all $n,m \geq N$.  Applying the fact that $A$ is dissipative we have $\norm{v_n - v_m} \leq \lambda^{-1} \epsilon$ which shows that $v_n$ is Cauchy in $X$ and therefore convergent; let $v$ be the limit of $v_n$.  Since $\lambda v_n$ converges to $\lambda v$ it follows that $A v_n = lambda v_n - (\lambda -A) v_n$ converges and since $A$
is assumed closed we know that $v \in \domain{A}$ and $\lim_{n \to \infty} A v_n = A v$.  It follows that $\lim_{n \to \infty} (\lambda -A)v_n  = (\lambda -A)v$ hence $\range{\lambda - A}$ is closed.

Now assume $\range{\lambda - A}$ is closed for all $\lambda > 0$.  Suppose $\lim_{n \to \infty} v_n = v$ and $\lim_{n \to \infty} A v_n = w$.  It follows that $\lim_{n \to \infty} (\lambda - A) v_n = \lambda v - w$.  Since $\range{\lambda - A}$ is closed there exists $u \in \domain{A}$ such that $(\lambda - A) u = \lambda v - w$.  Applying the fact that $A$ is dissipative,
\begin{align*}
\lim_{n \to \infty} \norm{v_n - u} &\leq \lambda^{-1}\lim_{n \to \infty} \norm{(\lambda - A)(v_n - u)} = \lambda^{-1}\lim_{n \to \infty} \norm{(\lambda - A)v_n - \lambda v + w} = 0
\end{align*}
Therefore $v = u$ hence $v \in \domain{A}$ and 
\begin{align*}
\lim_{n \to \infty} A v_n &= \lim_{n \to \infty} \lambda v_n - \lim_{n \to \infty} (\lambda -A) v_n = \lambda v - (\lambda v - A v) = Av
\end{align*}
hence $A$ is closed.

If $A$ is dissipative and $\range{\lambda - A}$ for a single $\lambda>0$ then $A$ is closed and dissipative and it follows that $\range{\lambda - A}$ is closed for all $\lambda > 0$.
\end{proof}


\begin{lem}\label{ClosedDissipativeResolventSet}Let $A$ be a closed dissipative operator on $X$, if there exists $\lambda > 0$ such that $\lambda \in \resolventset{A}$ then $\lambda \in \resolventset{A}$ for all $\lambda > 0$.
\end{lem}
\begin{proof}
We need a simple property of the topology of the real line.
\begin{clm}If $A \subset (0, \infty)$ is non-empty, closed and open then $A = (0,\infty)$.
\end{clm}
We know that all open sets of $(0,\infty)$ are countable unions of disjoint open intervals (Lemma \ref{OpenSetsOfReals}).  Suppose that there is a open interval $(a,b) \subset A$ with either $a\neq 0$ or $b \neq \infty$ then either $a$ or $b$ is a limit point of $A$ that is not contained in $A$ which contradicts the fact that $A$ is closed.

By the claim, the fact that $(0, \infty) \cap \resolventset{A}$ is assumed to be non-empty and the fact that $(0, \infty) \cap \resolventset{A}$ is open (Proposition \ref{SCCSResolventSetOpen}), it suffices to show that $(0, \infty) \cap \resolventset{A}$ is closed.  Thus suppose that we have a sequence 
$\lambda_n \in (0,\infty) \cap \resolventset{A}$ such that $\lim_{n\to \infty} \lambda_n = \lambda > 0$.  
Let $v \in X$ and define $v_n = (\lambda - A) R_{\lambda_n} v$ which is well defined since $\range{R_{\lambda_n}} = \domain{A}$.  Applying the norm bound $\norm{R_{\lambda_n}} \leq \lambda_n^{-1}$ (Proposition \ref{SCCSResolventAsLaplaceTransform}),
\begin{align*}
\lim_{n \to \infty} \norm{v_n - v} &= \lim_{n \to \infty} \norm{(\lambda -A) R_{\lambda_n}  v - (\lambda_n - A) R_{\lambda_n} v} \\
&=\lim_{n \to \infty} \norm{(\lambda -\lambda_n) R_{\lambda_n}  v}  \leq \lim_{n \to \infty} \frac{\abs{\lambda - \lambda_n}}{\lambda_n} \norm{v} = 0\\
\end{align*}
Therefore $\lim_{n \to \infty} v_n = v$; in particular $\range{\lambda - A}$ is dense in $X$.  However, since $A$ is dissipative and closed we can apply Proposition \ref{DissipativeAndClosed} to conclude that $\range{\lambda - A}$ is closed so in fact we have $\range{\lambda - A}=X$.  To see that $\lambda - A$ is injective, suppose
$(\lambda - A) v = 0$ then using the fact that $A$ is dissipative, $\norm{v} \leq \lambda^{-1} \norm{(\lambda - A) v} = 0$.  Similarly using the fact that $A$ is dissipative
we know that 
\begin{align*}
\norm{(\lambda - A)^{-1} v} &\leq \lambda^{-1} \norm{(\lambda-A)(\lambda - A)^{-1} v} = \lambda^{-1} \norm{v}
\end{align*}
which shows that $(\lambda - A)^{-1}$ is a bounded operator hence $\lambda \in \resolventset{A}$.
\end{proof}

\begin{defn}Let $A : X \to X$ be a closed linear operator and suppose $\lambda \in \resolventset{A}$ the we define the \emph{Yosida approximation} be be
\begin{align*}
A_\lambda &= \lambda A R_{\lambda}
\end{align*}
\end{defn}
To see why we refer to $A_\lambda$ as an approximation consider the case in which $A$ is bounded.  In that case for $\lambda > \norm{A}$ we can write
\begin{align*}
A_\lambda &= A (1 - \lambda^{-1} A)^{-1} = A + \lambda^{-1} A^2 + \lambda^{-2} A^3 + \cdots
\end{align*}
which shows that $\lim_{\lambda \to \infty} A_\lambda = A$.
The next result shows how this idea can be applied with unbounded $A$ : closed dissipative operators can be approximated by bounded operators using the Yosida approximation.
\begin{lem}[Yosida approximation]\label{ClosedDissipativeYosidaApproximation}Let $A$ be a closed dissipative operator with $\domain{A}$ dense and $(0,\infty) \subset \resolventset{A}$, then $A_\lambda$ satisfies
\begin{itemize}
\item[(i)] $A_\lambda$ is a a bounded linear operator and $e^{t A_\lambda}$ is a strongly continuous contraction semigroup.
\item[(ii)] $A_\lambda A_\mu = A_\mu A_\lambda$ for all $\lambda, \mu > 0$
\item[(iii)] $\lim_{\lambda \to \infty} A_\lambda v = A v$ for all $v \in \domain{A}$.
\end{itemize}
\end{lem}
\begin{proof}
We begin by showing (i).  Let $\lambda > 0$.  Then since $\lambda \in \resolventset{A}$ we have $R_\lambda$ is defined and $\range{R_\lambda} = \domain{A}$.   Because $A$ is dissipative we have for every $v \in X$,
\begin{align}\label{YosidaNormBound}
\norm{R_\lambda v} &\leq \lambda^{-1} \norm{(\lambda - A)R_\lambda v}  = \lambda^{-1} \norm{v}
\end{align}
hence $\norm{R_\lambda} \leq \lambda^{-1}$.

Using $(\lambda - A)R_\lambda = \IdentityMatrix$ on $X$ and $R_\lambda (\lambda -A) = \IdentityMatrix$ on $\domain{A}$ we get 
\begin{align}\label{YosidaAlternative}
A_\lambda &= \lambda A R_\lambda = \lambda (\lambda R_\lambda - \IdentityMatrix) = \lambda^2 R_\lambda - \lambda \text{ on $X$} 
\end{align}
and it follows that $A_\lambda$ is a bounded linear operator.  Therefore $e^{t A_\lambda}$ is a strongly continuous semigroup (Example \ref{ExponentialOfBoundedStronglyContinuousSemigroup}).  Furthermore by
\eqref{YosidaNormBound}
\begin{align*}
\norm{e^{t A_\lambda}} &= \norm{e^{t \lambda^2 R_\lambda} e^{-t \lambda}} \leq e^{t \lambda^2 \norm{R_\lambda}} e^{-t\lambda} \leq e^{t \lambda} e^{-t\lambda}  = 1
\end{align*}
so $e^{tA_\lambda}$ is contractive.  Thus (i) is shown.

Now (ii) follows from \eqref{YosidaAlternative} and the commutativity of resolvents (Proposition \ref{SimplePropertiesOfResolvents})
\begin{align*}
A_\lambda A_\mu &= (\lambda^2 R_\lambda - \lambda) (\mu^2 R_\mu - \mu) \\
&= \lambda^2 \mu^2 R_\lambda R_\mu - \lambda^2 \mu R_\lambda - \mu^2 \lambda R_\mu +\lambda \mu ) \\
&= \lambda^2 \mu^2 R_\mu R_\lambda - \lambda^2 R_\lambda - \mu^2 R_\mu + \mu \lambda)  \\
&= A_\mu A_\lambda
\end{align*}

To see (iii) first we have
\begin{clm}$\lim_{\lambda \to \infty} \lambda R_\lambda v = v$ for all $v \in X$.
\end{clm}
For $v \in \domain{A}$ we have using Proposition \ref{SimplePropertiesOfResolvents} and Proposition \ref{SCCSResolventAsLaplaceTransform}
\begin{align*}
\norm{\lambda R_\lambda v - v} &= \norm{\lambda R_\lambda v - (\lambda  -  A) R_\lambda v} \\
&= \norm{A R_\lambda v} = \norm{R_\lambda A v} \leq \lambda^{-1} \norm{Av}
\end{align*}
Now take the limit $\lambda \to \infty$ to see the claim for $v \in \domain{A}$.  By assumption $\domain{A}$ is dense in $X$ so for general $v \in X$ let $v_n \in \domain{A}$ such
that $\lim_{n \to \infty} v_n = v$.  Then for all $n \in \naturals$
\begin{align*}
\norm{\lambda R_\lambda v - v} &\leq \norm{(\lambda R_\lambda - \IdentityMatrix)(v - v_n)} + \norm{\lambda R_\lambda v_n - v_n} \\
&\leq 2 \norm{v-v_n} + \norm{\lambda R_\lambda v_n - v_n}
\end{align*}
so take the limit as $\lambda \to \infty$ and then as $n \to \infty$.  

To finish (iii) we use the claim and Proposition \ref{SimplePropertiesOfResolvents} to see that for all $v \in \domain{A}$,
\begin{align*}
\lim_{\lambda \to \infty} A_\lambda v &= \lim_{\lambda \to \infty} \lambda R_\lambda A v = A v
\end{align*}
\end{proof}

\begin{lem}\label{SemigroupBoundInTermsOfBoundedGenerator}Suppose $A$ and $B$ are communting bounded linear operators on a Banach space $X$ such that $\norm{e^{tA}} \leq 1$ and $\norm{e^{tB}} \leq 1$ for all $t \geq 0$. It follows that
\begin{align*}
\norm{e^{tA} - e^{tB}} &\leq t \norm{A-B}
\end{align*}
\end{lem}
\begin{proof}
Let $v \in X$ and $t \geq 0$ then by the Fundamental Theorem of Calculus (Theorem \ref{FundamentalTheoremOfCalculusForBanachSpaceRiemannIntegrals}) and the fact that $A$ and $B$ commute we get
\begin{align*}
e^{tA}v  - e^{tB} v &= \int_0^t \frac{d}{ds} e^{sA} e^{(t-s)B} v \, ds = \int_0^t e^{sA} e^{(t-s)B} (A-B) v \, ds
\end{align*}
Now if we take norms over both sides and use Proposition \ref{NormRiemannIntegralBanachSpace} and the hypotheses that $\norm{e^{sA}}, \norm{e^{(t-s)B}} \leq 1$ the result follows.
\end{proof}

\begin{thm}[Hille-Yosida Theorem]\label{HilleYosidaTheorem}Let $X$ be a Banach space, then a linear operator $A : X \to X$ is the generator of a strongly continuous contraction semigroup if and only if 
\begin{itemize}
\item[(i)] $ \domain{A}$ is dense in $X$
\item[(ii)] $A$ is dissipative
\item[(iii)] $\range{\lambda_0 -A} = X$ for some $\lambda_0 > 0$.
\end{itemize}
\end{thm}
\begin{proof}
If $A$ is the generator of a strongly continuous contraction semigroup then Corollary \ref{StronglyContinuousSemigroupGeneratorClosedDomainDense} implies (i), Example \ref{SCCSGeneratorDissipative} implies (ii) and Proposition \ref{SCCSResolventAsLaplaceTransform} implies that $\range{\lambda -A} = X$ for all $\lambda > 0$, so in particular (iii) holds.

Let us assume that (i), (ii) and (iii) hold.  

\begin{clm} $A$ is a closed operator and $\lambda_0 \in \resolventset{A}$.
\end{clm}
By (iii) we know $\range{\lambda_0 -A} = X$.  This implies $\range{\lambda_0 -A}$ is closed and since $A$ is dissipative Proposition \ref{DissipativeAndClosed} implies that $A$ is closed.  The fact that $A$ is dissipative implies that $\lambda_0 -A$ is injective (in fact for all $\mu > 0$, if $(\mu -A)v=0$ then $\norm{v} \leq \mu^{-1} \norm{(\mu -A)v} = 0$).  The fact that $A$ is dissipative implies that for all $v \in X$, $\norm{R_{\lambda_0} v} \leq \lambda_0^{-1} \norm{(\lambda_0 -A)R_{\lambda_0} v} = \lambda_0^{-1} v$ hence $R_{\lambda_0}$ is a bounded linear operator.  Therefore we have shown that $\lambda_0 \in \resolventset{A}$.

From the claim and Lemma \ref{ClosedDissipativeResolventSet} we know that $(0,\infty) \subset \resolventset{A}$.  This fact, the fact that $A$ is closed and dissipative and (i) means that we can apply the Yosida approximation Lemma \ref{ClosedDissipativeYosidaApproximation}.  For $\lambda>0$ define $A_\lambda =\lambda A R_\lambda$ and $T^\lambda_t = e^{tA_\lambda}$ so that $T^\lambda_t$ is a strongly continuous contraction semigroup.  By application of Lemma \ref {SemigroupBoundInTermsOfBoundedGenerator} we get the inequality
\begin{align}\label{HilleYosidaBoundSemigroupByGenerator}
\norm{T^\lambda_tv  - T^\mu_tv} &\leq t\norm{A_\lambda v - A_\mu v} \text{ for all $t\geq 0$ and $v \in X$}
\end{align}
Lemma \ref{ClosedDissipativeYosidaApproximation} also tells us that $\lim_{\lambda \to \infty} A_\lambda v = Av$ for all $v \in \domain{A}$.  Therefore the sequence $A_n v$ is Cauchy
hence \eqref{HilleYosidaBoundSemigroupByGenerator} tells us that $T^n_t v$ is uniformly Cauchy on every bounded interval of $[0, \infty)$ and therefore converges uniformly on bounded intervals to a continuous function $T_t v$ (recall that $T^n_t v$ is continuous by Proposition \ref{StronglyContinuousSemigroupContinuousPaths} and Lemma \ref{UniformConvergenceOnCompacts}).

\begin{clm} For every fixed $v \in X$, $T^n_t v$ converges in $C([0,\infty); X)$.
\end{clm}
By (i) we can find $v_m \in \domain{A}$ such that $\lim_{m \to \infty} v_m = v$.  
Now by the triangle inequality, \eqref{HilleYosidaBoundSemigroupByGenerator} and the fact that $T^\lambda_t$ is a contraction semigroup for all $\lambda > 0$, for all $m \in \naturals$
\begin{align*}
\norm{T^\lambda_t v - T^\mu_t v} &\leq  \norm{T^\lambda_t v - T^\lambda_t v_m}   + \norm{T^\lambda_t v_m - T^\mu_t v_m} + \norm{T^\mu_t v_m - T^\mu_t v} \\
&\leq 2 \norm{v -v_m} + t \norm{A_\lambda v_m - A_\mu v_m}
\end{align*}
The fact that $v_m \in \domain{A}$ and Lemma \ref{ClosedDissipativeYosidaApproximation} tell us that $\lim_{\lambda \to \infty} A_\lambda v_m = Av_m$  for all $m \in \naturals$. Thus for every $T \geq 0$ and $\epsilon > 0$ we can find $m \in \naturals$ such that $\norm{v -v_m} \leq \epsilon/4$ and a $N \geq 0$ such that $\norm{A_\lambda v_m - A_\mu v_m} \leq \epsilon/2T$ for $\lambda, \mu \geq N$ which shows
\begin{align*}
\sup_{0 \leq t \leq T} \norm{T^\lambda_t v - T^\mu_t v} \leq 2 \norm{v -v_m} + T \sup_{0 \leq t \leq T} \norm{A_\lambda v_m - A_\mu v_m} \leq 2 \norm{v -v_m} \leq \epsilon
\end{align*}
which shows that $T^\lambda_t v$ is Cauchy in $C([0,T];X)$ and therefore converges in $C([0,T];X)$.  It follows from Lemma \ref{UniformConvergenceOnCompacts} that $T^\lambda_t v$ converges in $C([0,\infty); X)$.

Now by the claim, for every $v \in X$ we have a $T_t v \in C([0,\infty); X)$ such that $T^\lambda_t v \to T_t v$ in $C([0,\infty); X)$.  
\begin{clm} $T_t$ is a strongly continuous contraction semigroup.
\end{clm}
To see the semigroup property we let $v \in X$ we use the fact that $T^\lambda_t v \to T_t v$ for every $0 \leq t < \infty$ and the fact that $\norm{T^\lambda_t} \leq 1$ to see
\begin{align*}
\norm{T_{s+t} v - T_s T_t v} &\leq \lim_{\lambda \to \infty} \left [\norm{T_{s+t} v - T^\lambda_{s+t} v} + \norm{T^\lambda_s T^\lambda_t v - T_s T_t v}  \right] \\
&\leq \lim_{\lambda \to \infty} \left[ \norm{T_{s+t} v - T^\lambda_{s+t} v } + \norm{T^\lambda_s (T^\lambda_t v - T_t v)} + \norm{T^\lambda_s T_t v - T_s T_t v} \right] \\
&\leq \lim_{\lambda \to \infty} \left[ \norm{T_{s+t} v - T^\lambda_{s+t} v } + \norm{T^\lambda_t v - T_t v} + \norm{T^\lambda_s T_t v - T_s T_t v} \right] \\
&=0
\end{align*}
The contraction property of $T_t$ follows easily from the contraction property of $T^\lambda_t$,
\begin{align*}
\norm{T_t v} &\leq \lim_{\lambda \to \infty} \left[\norm{T^\lambda_t v} + \norm{T_tv - T^\lambda_t v} \right] leq \norm{v} + \lim_{\lambda \to \infty} \norm{T_tv - T^\lambda_t v} = \norm{v}
\end{align*}
For strong continuity we need to use the full power of the fact that $T^\lambda_t v \to T_t v$ in $C([0,\infty); X)$:
\begin{align*}
\lim_{t \to 0} \norm{T_t v - v} &\leq \lim_{\lambda \to \infty} \lim_{t \to 0} \left[ \norm{T^\lambda_t v - v} + \norm{T_t v - T^\lambda_t v} \right] \\
&\leq \lim_{\lambda \to \infty} \lim_{t \to 0} \norm{T^\lambda_t v - v} + \lim_{\lambda \to \infty} \sup_{0 \leq t \leq 1} \norm{T_t v - T^\lambda_t v} \\
&= 0
\end{align*}

\begin{clm} $A$ is the generator of $T_t$.
\end{clm}
By the Kolomogorov backward equation (Proposition \ref{StronglyContinuousSemigroupKolomgorovBackwardEquation}) and the fact that $\domain{A_\lambda} = X$ we have for all $v \in X$,
\begin{align*}
T^\lambda_t v - v &= \int_0^t T^\lambda_s A_\lambda v \, ds
\end{align*}
If we assume that $v \in \domain{A}$ and $T > 0$ then using the fact that $\norm{T^\lambda_t} \leq 1$, the Yosida approximation and the definition of $T_t$,
\begin{align*}
\lim_{\lambda \to \infty} \sup_{0 \leq t \leq T} \norm{T^\lambda_t A_\lambda v - T_t A v} 
&\leq \lim_{\lambda \to \infty} \sup_{0 \leq t \leq T} \left[ \norm{T^\lambda_t A_\lambda v - T^\lambda_t A v} + \norm{T^\lambda_t A v - T_t A v} \right]\\
&\leq \lim_{\lambda \to \infty} \norm{A_\lambda v - A v} + \lim_{\lambda \to \infty} \sup_{0 \leq t \leq T} \norm{T^\lambda_t A v - T_t A v}  \\
&=0
\end{align*}
From the uniform convergence of $T^\lambda_t A_\lambda v $ on compacts we conclude 
\begin{align*}
\lim_{\lambda \to \infty} \int_0^t T^\lambda_s A_\lambda v \, ds &= \int_0^t T_s A v \, ds
\end{align*}
and therefore the Kolmogorov backward equation is established for $T_t$ and $A$:
\begin{align*}
T_t v - v &= \int_0^t T_s A v \, ds \text{ for $v \in \domain{A}$}
\end{align*}
The fact that $\lim_{t \to 0} t^{-1} (T_t v -v) =Av$ for all $v \in \domain{A}$ follows from the Fundamental Theorem of Calculus and the fact that $T_0 = \IdentityMatrix$.  If $B$ is the 
generator of $T_t$ we know that $A$ and $B$ agree on $\domain{A}$ thus it remains to show that $\domain{A}=\domain{B}$.  Suppose there exists $v \in \domain{B} \setminus \domain{A}$.  By (iii) we know that there is $\lambda > 0$ and $w \in \domain{A}$ such that $(\lambda - A) w = (\lambda - B) v$.  On the other hand $(\lambda - A) w =(\lambda - B) w$ and by (ii) we know that $\lambda - B$ is injective which is a contradiction.
\end{proof}

TODO:  Show this prior to proving Hille-Yosida and incorporate the next corollary into the statement of Hille-Yosida.  
The next important fact that we want to demonstrate is that the generator uniquely identifies a strongly continuous contraction semigroup.  That fact (as well as a few others) is a corollary of the following technical lemma.
\begin{lem}\label{DissipativeContractivePath}Let $A$ be a dissipative linear operator on $X$, let $u : [0,\infty) \to X$ be a continuous path in $X$ such that $u(t) \in \domain{A}$ for all $t > 0$, $A u : (0,\infty) \to X$ is continuous and 
\begin{align}\label{EpsilonBackwardEquation}
u(t) &= u(\epsilon) + \int_\epsilon^t A u(s) \, ds \text{ for all $0 < \epsilon < t$}
\end{align}
Then $\norm{u(t)} \leq \norm{u(0)}$.
\end{lem}
\begin{proof}
Since $Au(s)$ is uniformly continuous on $[\epsilon, t]$ given any $\delta > 0$ we may choose partition $0< \epsilon=t_0 < t_1 < \dotsb < t_n =t$ such that $\max_{1 \leq i \leq n} \sup_{t_{i-1} \leq s \leq t_i} \norm{A u(s) - A(t_i)} \leq \delta$.  Then writing a telescoping sum, using eqref \ref{EpsilonBackwardEquation}, the dissipative property (in the form $\norm{v} \leq \norm{v - \lambda^{-1}A v}$) and the triangle inequality 
\begin{align*}
\norm{u(t)} &= \norm{u(\epsilon)} + \sum_{i=1}^n \left [ \norm{u(t_i)} - \norm{u(t_{i-1})} \right] \\
&=\norm{u(\epsilon)} + \sum_{i=1}^n \left [ \norm{u(t_i)} - \norm{u(t_i) - \int_{t_{i-1}}^{t_i} A u(s) \, ds} \right] \\
&\leq \norm{u(\epsilon)} + \sum_{i=1}^n \left [ \norm{u(t_i) - (t_i - t_{i-1}) A u(t_i) } - \norm{u(t_i) - \int_{t_{i-1}}^{t_i} A u(s) \, ds} \right] \\
&\leq \norm{u(\epsilon)} + \sum_{i=1}^n  \norm{\int_{t_{i-1}}^{t_i} A u(s) \, ds - (t_i - t_{i-1}) A(t_i)} \\
&= \norm{u(\epsilon)} + \sum_{i=1}^n\norm{\int_{t_{i-1}}^{t_i} (A u(s) - A(t_i)) \, ds} \\
&\leq \norm{u(\epsilon)} + \sum_{i=1}^n\int_{t_{i-1}}^{t_i} \norm{A u(s) - A(t_i)} \, ds \\
&\leq \norm{u(\epsilon)} + (t - \epsilon) \max_{1 \leq i \leq n} \sup_{t_{i-1} \leq s \leq t_i} \norm{A u(s) - A(t_i)}  \\
&\leq \norm{u(\epsilon)} + (t - \epsilon) \delta \\
\end{align*}
As $\delta>0$ was arbitrary we may take the limit as $\delta \to 0$ conclude that $\norm{u(t)} \leq \norm{u(\epsilon)}$.  We may also take the limit as $\epsilon \to 0$ and use the continuity of $u$ at $0$ to conclude $\norm{u(t)} \leq \norm{u(0)}$.
\end{proof}

\begin{cor}\label{SCCSGeneratorDeterminesSemigroup}Let $T_t$ and $S_t$ be strongly continuous contraction semigroups both with generator $A$ then it follows that $T_t=S_t$ for all $t \geq 0$.
\end{cor}
\begin{proof}
By Example \ref{SCCSGeneratorDissipative} we know that $A$ is dissipative.  Let $v \in \domain{A}$.  Since $\domain{A}$ is dense (Corollary \ref{StronglyContinuousSemigroupGeneratorClosedDomainDense}) and $T_t$ and $S_t$ are bounded operators it suffices to show that $T_t v = S_t v$ for all $t \geq 0$.  By Proposition \ref{StronglyContinuousSemigroupContinuousPaths} we know that $T_tv$ and $S_tv$ are continuous.  By Proposition \ref{StronglyContinuousSemigroupKolomgorovBackwardEquation} we know that $A T_t v = T_t Av$ and $A S_t v = S_t Av$ hence $A T_t v$ and $A S_t v$ are continuous, that $T_tv, S_tv \in \domain{A}$ for all $t \geq 0$ and that $T_tv - S_tv = \int_0^t A (T_s v - S_s v) \, ds$ for all $t \geq 0$.  Thus applying Lemma \ref{DissipativeContractivePath} to the path
$T_t v - S_t v$ we see that $\norm{T_t v - S_t v} \leq \norm{T_0 v - S_o v}  = 0$ for all $t \geq 0$.  
\end{proof}

Now that we know that the generator determines the semigroup we know that a strongly continuous contraction semigroup is equal to the limit of the exponential of the Yosida approximations of its generators.  Moreover 
\begin{cor}\label{YosidaApproximationContinuity}Let $T_t$ be a strongly continuous contraction semigroup on $X$ with generator $A$  and let $A_\lambda$ be the Yosida approximation of $A$, then for all $v \in \domain{A}$ and $\lambda > 0$ 
\begin{align*}
\norm{e^{tA_\lambda} - T_t v} \leq t \norm{A_\lambda v - Av}
\end{align*}
and therefore for all $v \in X$ and $0 \leq t < \infty$, $\lim_{\lambda \to \infty} e^{t A_\lambda} = T_t v$ uniformly on bounded intervals.
\end{cor}
\begin{proof}
By Corollary \ref{SCCSGeneratorDeterminesSemigroup} and the proof of the Hille-Yosida theorem we know that $T_t v = \lim_{\mu \to \infty} e^{t A_\mu} v$ for all $v \in X$ and $0 \leq t < \infty$ as noted in the proof that convergence is in $C([0,\infty); S)$ for fixed $v \in X$.  From \eqref{HilleYosidaBoundSemigroupByGenerator} in the proof of Hille-Yosida theorem we know that for all $t\geq 0$ and $v \in X$
\begin{align*}
\norm{e^{\lambda t}v  - T_tv} &\leq \norm{e^{\lambda t}v  - e^{\mu t} v} + \norm{e^{\mu t}v  - T_tv} \\
&\leq t\norm{A_\lambda v - A_\mu v} + \norm{e^{\mu t}v  - T_tv} \\
&\leq t\norm{A_\lambda v - A v} + t\norm{A v - A_\mu v} + \norm{e^{\mu t}v  - T_tv} \\
\end{align*}
now let $\mu \to \infty$ and use the facts that $\lim_{\mu \to \infty} A_\mu v = Av$ and $\lim_{\mu \to \infty} e^{\mu t}v  = T_tv$.  
\end{proof}

The following corollary is a technical device that allows one to show that a semigroup preserves a subset by considering the behavior of its
generator.  In particular, we will be able to use this to show that a semigroup comprises positive operators on function spaces (i.e. the semigroup preserves
the positive cone of functions $f \geq 0$) by showing positivity of the resolvents.
\begin{cor}\label{SCCSInvarianceFromResolventInvariance}Let $T_t$ be a strongly continuous contraction semigroup on $X$ with generator $A$, let $Y \subset X$ and 
\begin{align*}
\Lambda_Y &= \lbrace \lambda > 0 \mid \lambda (\lambda - A)^{-1} (Y) \subset Y \rbrace
\end{align*}
If either
\begin{itemize}
\item[(i)] $Y$ is a closed convex subset of $X$ and $\Lambda_Y$ is unbounded
\item[(ii)] $Y$ is a closed subspace of $X$ and $\Lambda_Y$ is nonempty
\end{itemize}
then $T_t (Y) \subset Y$ for all $0 \leq t < \infty$.
\end{cor}
\begin{proof}
TODO:
\end{proof}

Sometimes it is inconvenient to deal with the full domain of a generator.  In particular, $\domain{A}$ might be too big in the sense that one lacks a clean characterization of the elements it contains.  In these situations it is often the case that there there is a subset of $\domain{A}$ which is easy to identify and which is big enough to capture all of the
information of $A$ in the sense the pair of $(A, \domain{A})$ is a limit of the restriction to the subset.  
\begin{defn}A linear operator is said to be \emph{closable} if it has a closed linear extension.  The smallest closed linear extension of $A$ is called the \emph{closure} of $A$.  Given a closable operator $A$ the closure of $A$ is denoted $\overline{A}$.
\end{defn}

\begin{prop}Let $A$ be a closable linear operator then the closure of the graph of $A$ defines a single valued closed linear operator $\overline{A}$.  Any closed extension of $A$ is an extension of $\overline{A}$.
\end{prop}
\begin{proof}
Suppose $B$ is a closed linear extension of $A$.  If we have a sequence $(v_n, A v_n)$ with $v_n \in \domain{A}$ that converges in $X \times X$ to $(v,w)$ then since $B$ is a closed extension of $A$ we know that $v \in \domain{B}$ and $w = B v$.  Thus if $(v,w)$ and $(v,u)$ are in the closure of the graph of $A$ we have $w = u = B v$; hence the closure of the graph of $A$ defines a function $\overline{A} : X \to X$.  The fact that $\overline{A}$ is linear follows from the linearity of limits and the linearity of $A$; pick $(v_n, A v_n) \to (v, Av)$ and $(w_n, A w_n) \to (w, A w)$ then 
\begin{align*}
(a v + b w, \overline{A}(a v + b w)) 
&=\lim_{n \to \infty} (a v_n + b w_n, A(a v_n + b w_n)) 
= \lim_{n \to \infty} (a v_n + b w_n, a A v_n + b A w_n) 
= (a v + b w, a \overline{A} v + b \overline{A} w) 
\end{align*}

If we let $C$ be any closed extension of $A$ and $v \in \domain{\overline{A}}$ then we may pick a sequence $v_n \in \domain{A}$ such that $\lim_{n \to \infty} (v_n, A v_n) = (v, \overline{A} v)$; but since $C$ is an extension of $A$, $(v_n, A v_n) = (v_n, C v_n)$ converges in $X \times X$ and therefore since $C$ is closed $(v, \overline{A} v) =  (v, C v)$ which shows that $C$ is an extension of $\overline{A}$.
\end{proof}

Our goal is to provide an alternative statement of the Hille-Yosida theorem in terms of closable operators rather than closed operators. We need to do a small bit of work to examine 
the interactions of some of our existing concepts with the new concept of the closure of a closable operator.

\begin{lem}\label{ClosableDissipativeClosure}Suppose $A$ is a closable linear operator then $A$ is dissipative if and only if $\overline{A}$ is dissipative.
\end{lem}
\begin{proof}
First suppose $\overline{A}$ is dissipative.  Then using the fact that $\overline{A}$ is an extension of $A$, for any $\lambda > 0$ and $v \in \domain{A}$
\begin{align*}
\norm{(\lambda - A)v} &= \norm{(\lambda - \overline{A})v} \geq \lambda \norm{v}
\end{align*}
On the other hand if $A$ is dissipative then for any $v \in \domain{\overline{A}}$ we may pick a sequence $v_n \in \domain{A}$ such that $\lim_{n \to \infty} (v_n,A v_n) = (v, \overline{A}v)$ and therefore for every $\lambda > 0$,
\begin{align*}
\norm{ (\lambda - \overline{A}) v} &= \lim_{n \to \infty} \norm{ (\lambda - A) v_n} \geq \lambda \lim_{n \to \infty} \norm{ v_n} = \lambda \norm{v} 
\end{align*}
\end{proof}

\begin{prop}\label{DissipativeDenseDomainImpliesClosable}Let $A$ be dissipative with $\domain{A}$ dense then $A$ is closable and $\overline{\range{\lambda -A}} = \range{\lambda - \overline{A}}$ for all $\lambda > 0$.
\end{prop}
\begin{proof}
\begin{clm} $A$ is closable
\end{clm}
Pick a sequence $v_n \in \domain{A}$ such that $\lim_{n \to \infty} v_n = 0$ and $\lim_{n \to \infty} A v_n = w$.  Since $\domain{A}$ is dense we may pick a sequence $w_n \in \domain{A}$ such that $\lim_{n \to \infty} w_n = w$.  Therefore for all $m \in \naturals$ and $\lambda > 0$ using the continuity of norms and the fact that $A$ is dissipative,
\begin{align*}
\norm{(\lambda - A) w_m - \lambda w} &= \lim_{n \to \infty} \norm{(\lambda - A) w_m - (\lambda - A) \lambda v_n} \\
&\geq \lambda \lim_{n \to \infty} \norm{w_m - \lambda v_n} = \lambda \norm{w_m}
\end{align*}
so after dividing by $\lambda$, and taking limits
\begin{align*}
\norm{w} &= \lim_{m \to \infty} \norm{w_m} \leq \lim_{m \to \infty} \lim_{\lambda \to \infty}\norm{ (\IdentityMatrix - \lambda^{-1} A) w_m - w} \\
&\leq \lim_{m \to \infty} \lim_{\lambda \to \infty} \left[ \norm{w_m - w} + \lambda^{-1} \norm{A w_m} \right] = 0
\end{align*}
and therefore $w = 0$.  

Now to see that $A$ is closable suppose that $(v_n, A v_n)$ and  $(u_n, A u_n)$ are convergent sequences in $X \times X$ with $v_n, u_n \in \domain{A}$ and $\lim_{n \to \infty} v_n = \lim_{n \to \infty} u_n$.  Then if follows that 
$\lim_{n \to \infty} (v_n - u_n) = 0$ and therefore by the preceeding argument $\lim_{n \to \infty} A(v_n - u_n) = \lim_{n \to \infty} Av_n - \lim_{n \to \infty}  A u_n = 0$; which is to say that $A$ is closable.

\begin{clm} $\range{\lambda - \overline{A}} \subset\overline{\range{\lambda -A}}$
\end{clm}
Now suppose that $w \in \range{\lambda - \overline{A}}$ and pick $v \in \domain{\overline{A}}$ with $(\lambda - \overline{A})v = w$.  By definition of $\overline{A}$ we may pick
a sequence $v_n \in \domain{A}$ such that $\lim_{n \to \infty} v_n$ and $\lim_{n \to \infty} A v_n = \overline{A} v = \lambda v - w$.  Therefore $w = \lim_{n \to \infty} (\lambda v_n - A v_n)$
which shows that $w \in \overline{\range{\lambda -A}}$.  

\begin{clm} $\overline{\range{\lambda -A}} \subset\range{\lambda - \overline{A}} $
\end{clm}
By Lemma \ref{ClosableDissipativeClosure} we know that $\overline{A}$ is dissipative and closed and 
by Proposition \ref{DissipativeAndClosed} we know this implies that $\range{\lambda - \overline{A}}$ is closed.  Since $\overline{A}$ is an
extension of $A$ we know that $\range{\lambda - A} \subset \range{\lambda - \overline{A}}$.  The claim follows by combining these two observations and taking closures.
\end{proof}

We can have enough to state a prove an alternative formulation of the Hille-Yosida theorem in terms of closable operators.
\begin{thm}[Hille-Yosida Theorem for Closable Operators]\label{HilleYosidaTheoremClosable}Let $X$ be a Banach space, then a linear operator $A : X \to X$ is closable with $\overline{A}$ the generator of a strongly continuous contraction semigroup if and only if 
\begin{itemize}
\item[(i)] $ \domain{A}$ is dense in $X$
\item[(ii)] $A$ is dissipative
\item[(iii)] $\range{\lambda_0 -A}$ is dense in $X$ for some $\lambda_0 > 0$.
\end{itemize}
\end{thm}
\begin{proof}
Suppose $A$ is closable and $\overline{A}$ is generator of a strongly continuous contraction semigroup.  Then by the Hille-Yosida Theorem \ref{HilleYosidaTheorem} we know that $\domain{\overline{A}}$ is dense in $X$, $\overline{A}$ is dissipative and $\range{\lambda_0 - \overline{A}} = X$ for some $\lambda_0 > 0$.  By Lemma \ref{ClosableDissipativeClosure} we know that $A$ is dissipative.   To see that $\domain{A}$ is dense,  just note that for any $\epsilon > 0$ and $v \in X$ we may find a $w \in \domain{\overline{A}}$ such that $\norm{v - w} \leq \epsilon/2$ but also that we may find a $u \in \domain{A}$ such that $\norm{w - u} \leq \epsilon/2$ (and $\norm{\overline{A}w - Au} \leq \epsilon/2$ although we don't need this fact).  Now that we know $\domain{A}$ is dense it follow from Proposition \ref{DissipativeDenseDomainImpliesClosable} that $\overline{\range{\lambda_0 - A}} = \range{\lambda_0 - \overline{A}}=X$.

On the other hand, suppose that $A$ satisfies (i), (ii) and (iii).  By Proposition \ref{DissipativeDenseDomainImpliesClosable} we know that $A$ is closable.  Then by Lemma \ref{ClosableDissipativeClosure} we know that $\overline{A}$ is dissipative.  It is immediate from (i) and the fact that $\domain{A} \subset \domain{\overline{A}}$ that $\domain{\overline{A}}$ is dense.  Lastly by (i), (ii) and (iii) we may apply Proposition \ref{DissipativeDenseDomainImpliesClosable} implies that $\range{\lambda_0 - \overline{A}} = \overline{\range{\lambda_0 - A}} =X$.  Thus we may apply the Hille-Yosida Theorem \ref{HilleYosidaTheorem} to conclude that $\overline{A}$ is the generator of a strongly continuous contraction semigroup.
\end{proof}

\subsection{Feller Semigroups}

TODO: Define Feller processes and use the Yosida approximation to show that they are strongly continuous.

Having presented the theory of semigroups in a generic Banach space setting we now specialize back to the case of interest for Markov processes.  In fact we specialize the state space and assume that $S$ is a locally compact separable metric space and consider the action on the Banach space of continuous functions vanishing at infinity (see Proposition 
\ref{BanachSpaceOfFunctionsVanishingAtInfinity}).  Recall that semigroups We want to apply the Hille-Yosida theorem  in the special case and as a new property makes an appearance.

\begin{defn}Let $S$ be a topological space then a semigroup $T_t : C_0(S) \to C_0(S)$ is said to be \emph{positive} if for every $f \geq 0$ we have $T_t f \geq 0$ for all $t \geq 0$. 
A positive strongly continuous contraction semigroup on $C_0(S)$ is called a \emph{Feller semigroup}.   An unbounded operator $A : C_0(S) \to C_0(S)$ is said to satisfy the \emph{positive maximum principle} if given $f \in \domain{A}$ and $x_0 \in S$ satisfying $\sup_{x \in S} f(x) = f(x_0) \geq 0$ we have $Af(x_0) \leq 0$.
\end{defn}


\begin{lem}\label{PositiveMaximumPrincipleDissipative}Let $S$ be locally compact Hausdorff and suppose that $A$ satisfies the positive maximum principle then $A$ is dissipative.
\end{lem}
\begin{proof}
Let $f \in \domain{A}$ and suppose $\lambda > 0$.  Using Corollary \ref{VanishingAtInfinityLocallyCompactAttainsNormInfSup} pick $x_0 \in S$  such that $\abs{f(x_0)} = \sup_{x \in S} \abs{f(x)}$.  First suppose that $f(x_0) \geq 0$.  It follows that $f(x_0) \geq f(x)$ for all $x \in S$ and from the positive maximum principle that $Af(x_0) \leq 0$, thus
\begin{align*}
\norm{\lambda f - A f} &\geq \abs{\lambda f(x_0) - Af(x_0)} = \lambda f(x_0) - Af(x_0) \geq \lambda f(x_0) = \lambda \norm{f}
\end{align*}
If $f(x_0) < 0$ then we can apply the same argument to $-f$.  It follows that $A$ is dissipative.
\end{proof}

\begin{thm}\label{HilleYosidaTheoremFellerSemigroup} Let $S$ be a locally compact separable metric space then $A: C_0(S) \to C_0(S)$ is closable and generates a Feller semigroup if and only if 
\begin{itemize}
\item[(i)] $ \domain{A}$ is dense in $X$
\item[(ii)] $A$ satisfies the positive maximum principle
\item[(iii)] $\range{\lambda_0 -A}$ is dense in $X$ for some $\lambda_0 > 0$.
\end{itemize}
\end{thm}
\begin{proof}
If $\overline{A}$ generates a Feller semigroup then we can apply the Hille-Yosida Theorem \ref{HilleYosidaTheoremClosable} to conclude (i) and (iii).  Moreover we know that $A$ is dissipative.  Suppose $f \in \domain{A}$  and $f(x_0) = \sup_{x \in S} f(x) \geq 0$.  Using positivity and the contraction property we get for all $t \geq 0$,
\begin{align*}
T_t f (x_0) &\leq T_t f_+(x_0) \leq \norm{f_+} = f(x_0)
\end{align*}
and therefore 
\begin{align*}
Af(x_0) &= \lim_{t \to 0} \frac{T_t f(x_0) - f(x_0)}{t} \leq 0
\end{align*}
and it follows that $A$ satisfies (ii).

On the other hand, suppose $A$ satisfies (i), (ii) and (iii) by Lemma \ref{PositiveMaximumPrincipleDissipative} we know that $A$ is dissipative and therefore we can apply 
Hille-Yosida Theorem \ref{HilleYosidaTheoremClosable} to conclude that $A$ is closable and $\overline{A}$ generates a strongly contractive semigroup $T_t$.  It remains
to prove that $T_t$ is positive.  

\begin{clm}\label{HilleYosidaFeller:ResolventPositivity}For all $f \in \domain{\overline{A}}$ and $\lambda > 0$ if $(\lambda - \overline{A}) f \geq 0$ then $f \geq 0$.
\end{clm}

We argue by contradiction.  Pick $f \in \domain{\overline{A}}$, $\lambda > 0$ and suppose $\inf_{x \in S} f(x) < 0$.  Pick $f_n \in \domain{A}$ such that 
$\lim_{n \to \infty} f_n = f$ and $\lim_{n \to \infty} A f_n = \overline{A} f$.  It follows that $\lim_{n \to \infty} (\lambda - A) f_n = (\lambda - \overline{A}) f$.  Using Corollary \ref{VanishingAtInfinityLocallyCompactAttainsNormInfSup}  for each $n \in \naturals$ we select
$x_n \in S$ such that $f_n(x_n) = \inf_{x \in S} f_n(x)$ and select $x_0 \in S$ such that $f(x_0) = \inf_{x \in S} f(x) < 0$.

We need to piece together some simple facts.
\begin{clm}$\lim_{n \to \infty} f_n(x_n) = f(x_0)$
\end{clm}
From the definition of $x_n$ and $\lim_{n \to \infty} f_n = f$ we see that $\lim_{n \to \infty} f_n(x_n) \leq \lim_{n \to \infty} f_n(x_0) = f(x_0)$.  On the other hand for every $\epsilon > 0$
there exists $N$ such that $\sup_{x \in S} \abs{f_n(x) - f(x)} < \epsilon$ for $n \geq N$ and therefore $f(x_0) - \epsilon \leq f(x_n) - \epsilon \leq f_n(x_n)$ for $n \geq N$.  This implies $f(x_0) - \epsilon \leq \lim_{n \to \infty} f_n(x_n)$ and since $\epsilon >0$ was arbitrary we get $f(x_0) \leq \lim_{n \to \infty} f_n(x_n)$.

Note that exactly the same argument shows that $\lim_{n \to \infty} \inf_{x \in S} (\lambda - A) f_n(x) =  \inf_{x \in S} (\lambda - \overline{A}) f(x)$.

\begin{clm}For $n$ sufficiently large $A f_n (x_n) \geq 0$.
\end{clm}
From the previous claim and the fact that $f(x_0) < 0$ we know that for sufficiently large $n$ we must have $f_n(x_n) \leq 0$ and thus $-f_n(x_n) = \sup_{x \in S} (-f_n(x)) 
\geq 0$ for sufficiently large $n$; by the positive maximum principle applied to $-f_n$ we see that $A f_n (x_n) \geq 0$.  

By the previous claim we get for every $n \in \naturals$,
\begin{align*}
\inf_{x \in S} (\lambda - A) f_n(x) &\leq (\lambda - A) f_n(x_n) \leq \lambda f_n(x_n)
\end{align*}
and therefore taking the limit as $n \to \infty$ we have
\begin{align*}
\inf_{x \in S} (\lambda - \overline{A}) f(x)  &= \lim_{n \to \infty} \inf_{x \in S} (\lambda - A) f_n(x) \leq \lim_{n \to \infty} \lambda f_n(x_n) = \lambda f(x_0) < 0
\end{align*}
and Claim \ref{HilleYosidaFeller:ResolventPositivity} is shown.  

To finish, note that the positive cone $\lbrace f \in C_0(S) \mid f \geq 0 \rbrace$ is closed and convex.  Claim \ref{HilleYosidaFeller:ResolventPositivity} shows that for all $\lambda > 0$ if $f \geq 0$ then $(\lambda - A)^{-1} f \geq 0$ so by Corollary \ref{SCCSInvarianceFromResolventInvariance} we conclude that $T_t$ is positive for all $t \geq 0$.
\end{proof}

As it turns out the strong continuity required in the definition of a Feller semigroup may be derived from a weaker pointwise continuity property.  
\begin{prop}Let $S$ be locally compact Hausdorff and suppose that $T_t : C_0(S) \to C_0(S)$ is a positive contraction semigroup such that
\begin{align*}
\lim_{t \to 0} T_tf (x) = f(x) \text{ for all $f \in C_0(S)$ and $x \in S$}
\end{align*}
then $T_t$ is a Feller semigroup.
\end{prop}
\begin{proof}
TODO: The proof follows by defining the resolvent for $\lambda > 0$ explicitly using the Laplace transform and then using the Yosida approximation.  
\end{proof}

\subsection{Cores}
\begin{defn}Let $A$ be a closed operator then $D \subset \domain{A}$ is a \emph{core} if $\overline{A \mid_D} = A$.  
\end{defn}

\begin{prop}\label{SCCSCoreViaDenseRange}Let $A$ be the generator of a strongly continuous contraction semigroup then $D$ is a core for $A$ if and only if $(\lambda_0 -A)(D)$ is dense for some $\lambda_0 > 0$.  In either case $(\lambda -A)(D)$ is dense for all $\lambda > 0$.
\end{prop}
\begin{proof}
Suppose $D$ is a core then let $v \in X$ then $R_\lambda v \in \domain{A}$ and therefore we can find $w_n \in D$ such that $\lim_{n \to \infty} w_n = R_\lambda v$ and $\lim_{n \to \infty} A w_n = A R_\lambda v$.  Therefore $\lim_{n \to \infty} (\lambda - A) w_n = (\lambda - A) R_\lambda v = v$.  

Suppose $(\lambda_0 -A)(D)$ is dense for some $\lambda_0 > 0$.  Let $v \in \domain{A}$ and choose $v_n \in D$ such that $\lim_{n \to \infty} (\lambda_0 - A) v_n = (\lambda_0 - A) v$.  From Proposition \ref{SCCSResolventAsLaplaceTransform} we know that $R_{\lambda_0}$ is bounded and therefore 
\begin{align*}
\lim_{n \to \infty} v_n &= R_{\lambda_0} \lim_{n \to \infty} (\lambda_0 - A) v_n = R_{\lambda_0}  (\lambda_0 - A) v = v
\end{align*}
from which we get 
\begin{align*}
\lim_{n \to \infty} A v_n &= \lim_{n \to \infty} \lambda_0 v_n - \lim_{n \to \infty} (\lambda_0 - A) v_n = \lambda_0 v - (\lambda_0 - A) v = Av
\end{align*}

TODO: In EK they add the assumption that $D$ is dense; do we need that or is it a consequence of $(\lambda_0 -A)(D)$ is dense?
\end{proof}

\begin{prop}\label{SCCSCoreViaInvariance}Let $A$ be the generator of a strongly continuous contraction semigroup and let $D_0 \subset D \subset \domain{A}$ be dense subspaces such that $T_t D_0 \subset D$ for all $t \geq 0$ then $D$ is a core.
\end{prop}
\begin{proof}
Let $v \in D_0$, $\lambda > 0$ and define
\begin{align*}
v_n &= \frac{1}{n} \sum_{k=0}^{n^2} e^{-\lambda k/n} T_{k/n} v \in D
\end{align*}
for all $n \in \naturals$.
\begin{align*}
\lim_{n \to \infty} (\lambda - A) v_n &= \lim_{n \to \infty} \frac{1}{n} \sum_{k=0}^{n^2} e^{-\lambda k/n} T_{k/n}(\lambda - A)  v \\
&=\int_0^\infty e^{-\lambda t} T_t (\lambda -A) v \, dt = R_\lambda (\lambda -A) v = v
\end{align*}
which shows us that $D_0 \in \overline{(\lambda - A)(D)}$.  Now by density of $D_0$ we see that $(\lambda - A)(D)$ is dense and by Proposition \ref{SCCSCoreViaDenseRange} we see that $D$ is a core.
\end{proof}

\section{Existence of Feller Processes}

Our next goal is to show that there is a cadlag Feller process associated with any Feller semigroup.  We begin with some motivational comments.  It isn't too hard to come up with a rough sketch for how to proceed:  
\begin{itemize}
\item since the semigroup is on the Banach space $C_0(S)$ for each $x \in S$ and $t \geq 0$ we get a positive linear functional $\lambda_{t,x} f = T_t f (x)$
\item by the Riesz Representation Theorem we therefore get transition measures $\mu_{t}(s, \cdot)$ each of which is a finite Radon measure
\item the semigroup property implies the Chapman Kolmogorov relations thus we can assert the existence of a Markov process with transition measures $\mu_t(x, \cdot)$.
\end{itemize}
A couple of things need to be sorted out.  We have not come up with an obvious plan for how to prove that there is a cadlag modification of the constructed Markov process.  Indeed
this requires some real ingenuity and work.  The second issue is much simpler; in the second step of our plan we mentioned that the Riesz Representation Theorem will only guarantee that we have a finite Radon measure but not necessarily a probability measure.  Indeed the contraction property of $T_t$ shows that the total mass of the measure is less than or equal to one (it is equal to $\norm{\lambda_{t,x}}$); however, in our definition of a Feller semigroup we have not introduced the condition that is necessary to guarantee that the constructed transition measures are probability measures.  From a Markov process point of view, we have not introduced the condition that prevents explosion.  We do that now.

\begin{defn}A Feller semigroup is \emph{conservative} if and only if $\sup_{f \leq 1} T_t f(x) = 1$ for all $x \in S$ and all $t \geq 0$.
\end{defn}
Morally we want the property of being conservative to simply say that $T_t 1 = 1$ for all $t \geq 0$ but we have chosen to define a Feller semigroup as being defined only on $C_0(S)$ and $1$ does not vanish at infinity unless $S$ is compact.  As we will see it is in fact possible to extend every Feller semigroup to the entire space $B_b(S)$ in which case the property $T_t 1 = 1$ for all $t \geq 0$ is a valid (and equivalent) definition of conservativeness.  We don't simply proceed by tacking on the hypothesis that $T_t$ is conservative, rather we show that every Feller semigroup can naturally be extended to a conservative Feller semigroup.  From a transition measure point of view we add a new state to $S$ and put the missing mass there so that every transition measure becomes a probability measure.  From a Markov process point of view we are adding an absorbing state to which explosions can transition.

We append a state $\Delta$ to the state space $S$.  The manner in which this is done depends on whether $S$ is compact or not.  If $S$ is compact then we
make $\Delta$ an isolated point of $S^{\Delta} = S \cup \lbrace \Delta \rbrace$ otherwise we let $\Delta$ be the point at infinity in the one point compactification of $S$ (Definition \ref{OnePointCompactificationDefinition}).  In the latter case, recall from Theorem \ref{OnePointCompactification} that $S^\Delta$ is compact and from Proposition \ref{IsometricEmbeddingVanishingAtInfinityIntoCompact} that we can regard every element of $f \in C_0(S)$ isometrically as an element of $C(S^\Delta)$ by defining $f(\Delta) = 0$.  In the former case we leave it to the reader to verify both properties (they follow immediately from the fact that $\lbrace \Delta \rbrace$ is an open set).  It what follows we will freely identify $C_0(S)$ with the subspace of $f \in C(S^\Delta)$ for which $f(\Delta) = 0$.

\begin{prop}\label{FellerSemigroupCompactification}Let $S$ be locally compact separable metric space and let $T_t$ be a Feller semigroup on $S$.  Define $T^\Delta_t$ on $C(S^\Delta)$ by
\begin{align*}
T^\Delta_t f &= f(\Delta) + T_t (f - f(\Delta))
\end{align*}
then $T^\Delta_t$ is a conservative Feller semigroup on $S^\Delta$.
\end{prop}
\begin{proof}
First note that for any $f \in C(S^\Delta)$, $T^\Delta_t f (\Delta) = f(\Delta) + T_t (f - f(\Delta)) (\Delta) = f(\Delta)$ since $T_t (f - f(\Delta)) \in C_0(S)$.
\begin{clm} $T^\Delta_t$ is a semigroup
\end{clm}
To see the semigroup property of $T^\Delta_t$ use the semigroup property of $T_t$ and the fact that $T^\Delta_s f(\Delta) = f(\Delta)$,
\begin{align*}
T^\Delta_{t + s} f &= f(\Delta) + T_{t + s} (f - f(\Delta)) = f(\Delta) + T_{t} T_{s} (f - f(\Delta)) \\
&=f(\Delta) + T_t (T^\Delta_{s} f - f(\Delta)) = T^\Delta_s f(\Delta) + T_{t} (T^\Delta_{s} f - T^\Delta_s f(\Delta))) = T^\Delta_t T^\Delta_s f
\end{align*}
and also
\begin{align*}
T^\Delta_0 f &= f(\Delta) + T_0 (f - f(\Delta)) = f(\Delta) + f - f(\Delta) = f
\end{align*}

Strong continuity of $T^\Delta_t$ follows easily from strong continuity of $T_t$,
\begin{align*}
\lim_{t \to 0} T^\Delta_t f &= f(\Delta) + \lim_{t \to 0} T_t (f - f(\Delta)) = f(\Delta) + f - f(\Delta) =f
\end{align*}

\begin{clm}$T^\Delta_t$ is positive
\end{clm}
Let $f \in C_0(S)$ and pick $\alpha \in \reals$ such that $\alpha + f \geq 0$.  Note that $\alpha + f \in C(S^\Delta)$ and by definition $T^\Delta_t (\alpha + f) = \alpha + T_t f$; also note that every $g \in C(S^\Delta)$ may be written in the form $\alpha + f$ for $f \in C_0(S)$.
Write $f = f_+ - f_-$ with $f_\pm \geq 0$ so that $T_t f_\pm \geq 0$.  Therefore $f_- + f = f_- + (f_+ - f_-) = f_+$ and positivity of $T_t$ implies $-T_t f \leq T_t f_-$.  Therefore
\begin{align*}
(T_t f)_- &= (-T_t f \vee 0) \leq (T_t f_- \vee 0) = T_t f_-
\end{align*}
Since $T_t$ is a contraction, $\norm{T_t f_-} \leq \norm{f_-} \leq \alpha$ and therefore $(T_tf)_- \leq \alpha$ which imples $\alpha + T_t f \geq 0$.

To see that $T^\Delta_t$ is a contraction, note that since $\norm{f} \pm f \geq 0$, by positivity $\norm{T^\Delta_t f} \leq T^\Delta_t \norm{f} = \norm{f}$ and therefore $\norm{T_t} \leq 1$.

To see that $T^\Delta_t$ is conservative from $T^\Delta_t 1 = 1$ and the positivity of $T^\Delta_t$,
\begin{align*}
\sup_{f \leq 1} T^\Delta_t f \leq T^\Delta_t 1 = 1
\end{align*}
\end{proof}

Now we can show that a unique transition kernel can be associated with
any Feller semigroup.  By the application Daniell-Kolmogorov Theorem an associated
Markov process exists as well.

\begin{prop}\label{FellerSemigroupToTransitionKernel}Let $S$ be locally compact separable metric space and let $T_t$ be a Feller semigroup on $S$, then there exists
a unique homogeneous transition kernel $\mu_t$ on $S^\Delta$ such that
\begin{align*}
T_t f(x) &= \int f(s) \mu_t(x, ds) \text{ for all $f \in C_0(S)$ and $x \in S$}
\end{align*}
and such that $\mu_t(\Delta, \lbrace \Delta \rbrace) =1$ for all $t \geq 0$.
\end{prop}
Moreover there exists a homogeneous Markov process $X$ with transition kernel $\mu_t$ and semigroup $T_t$.
\begin{proof}
Let $T^\Delta_t$ be the conservative Feller semigroup on $C(S^\Delta)$ constructed in Proposition \ref{FellerSemigroupCompactification}.  For fixed $x \in S^\Delta$ and $t \geq 0$, $T_t f(x)$ defines a positive linear functional on $C(S^\Delta)$.  By the Riesz-Markov Theorem \ref{RieszMarkov} we know that there exists a Radon measure $\mu_t(x, \cdot)$ such that $T_t f(x) = \int f(s) \, \mu_t(x, ds)$.  Since $T^\Delta_t$ is conservative we have $1 = T_t 1 = \mu_t(x, S)$ and therefore $\mu_t(x, \cdot)$ is a probability measure for all $x \in S^\Delta$ and $t \geq 0$.

To see that $\mu_t$ is a probability kernel first note that by Proposition \ref{StronglyContinuousSemigroupContinuousPaths} we know that $T_t f(x) = \int f(s) \, \mu_t(x, ds)$ is a continuous function of $x$ for every $f \in C(S^\Delta)$ and $t \geq 0$.  Given a open set $U \subset S^\Delta$ we pick a metric $d$ on $S$ and approximate $\characteristic{U}(s) = \lim_{n \to \infty} n d(s, U^c) \wedge 1$ use Dominated Convergence to see that $\mu_t(x, U) = \lim_{n \to \infty} \int (n d(s, U^c) \wedge 1) \, \mu(x, ds)$ and therefore $\mu_t(x, U)$ is measurable in $x$ (Lemma \ref{LimitsOfMeasurable}).  Finally let $t \geq 0$ be fixed and define $\mathcal{C} = \lbrace A \in \mathcal{B}(S) \mid \mu_t(x, A) \text{ is measurable}\rbrace$.  If $A, B \in \mathcal{C}$ and $A \subset B$ then $\mu_t(x, B \setminus A) = \mu_t(x, B) - \mu_t(x, A)$ is measurable hence $B \setminus A \in \mathcal{C}$ and if $A_1 \subset A_2 \subset \cdots$ with $A_n \in \mathcal{C}$ then $\mu_t(x, \cup_n A_n) = \lim_{n \to \infty} \mu_t(x, A_n)$ by Lemma \ref{ContinuityOfMeasure} hence $\cup_n A \in \mathcal{C}$.  It follows that $\mathcal{C}$ is a $\lambda$-system and it is clear that the set of open sets is a $\pi$-system therefore $\mathcal{C} = \mathcal{B}(S)$ by the $\pi$-$\lambda$ Thereom \ref{MonotoneClassTheorem}.

The Chapman-Kolmogorov relations follow from Proposition \ref{SemigroupsAndChapmanKolmogorov}.  To see $\mu_t(\Delta, \lbrace \Delta \rbrace) =1$ note that for every $f \in C_0(S)$ we have
\begin{align*}
\int f(s) \, \mu_t(\Delta, ds) &= T^\Delta_t f(\Delta) = f(\Delta) + T_t (f - f(\Delta))(\Delta) = 0
\end{align*}
so just approximate $\characteristic{S}$ by an sequence of nonnegative $f \in C_0(S)$ and use Fatou's Lemma (TODO: Explicity create such a sequence).

Since a compact metric space $S^\Delta$ is complete (Theorem \ref{CompactnessInMetricSpaces}) it is Polish and therefore Borel (Theorem \ref{PolishImpliesBorel}).  Applying Theorem \ref{ExistenceMarkovProcess} we conclude that there is a Markov process with homogeneous transition kernels $\mu_t$.
\end{proof}

As mentioned we have to be a bit more clever to show that we may find a cadlag version of Markov process $X$ constructed in the last result.  The basic idea is that we show that
many functions of $X$ are in fact supermartingales and therefore have cadlag versions; once enough functions $X$ are known to have
a cadlag version then we may approximate the identity and conclude that $X$ itself has a cadlag version.  Before stating the Theorem we need two preliminary lemmas.  The first constructs the supermartingales in question and the latter gives a probabilistic interpretation of the strong continuity of the Feller semigroup; the probabilistic continuity will
be used in the limiting process of showing $X$ is cadlag.

\begin{lem}\label{FellerResolventsAreSupermartingales}Let $S$ be locally compact separable metric space, let $T_t$ be a Feller semigroup on $S$, $X$ be the Markov process 
on $S^\Delta$ defined by $T_t$ and initial distribution $\nu$ and $\mathcal{F}$ be the filtration induced by $X$.  For every $f \in C_0(S)$ such that $f \geq 0$ define
\begin{align*}
Y_t &= e^{-t} R_1 f (X_t) = e^{-t} \int_0^\infty e^{-s} T_s f(X_t) \, ds
\end{align*}
then $Y_t$ is a $\mathcal{F}$-supermartinagle.
\end{lem}
\begin{proof}
Using the Markov property, the definitions of $T_t$ and $R_1$, a change of integration variables and the non-negativity of $f$,
\begin{align*}
\cexpectationlong{\mathcal{F}_{s}}{Y_t} 
&=\cexpectationlong{\mathcal{F}_{s}}{e^{-t} R_1 f(X_t)} \\
&=\sexpectation{e^{-t} R_1 f(X_{t-s})}{X_s} \\
&=T_{t-s} e^{-t} R_1 f(X_s) \\
&=T_{t-s} e^{-t} \int_0^\infty e^{-u} T_uf(X_s) \, du \\
&=e^{-t} \int_0^\infty e^{-u} T_{t+u-s} f(X_s) \, du \\
&=e^{-s} \int_0^\infty e^{-(t+u-s)} T_{t+u-s} f(X_s) \, du \\
&=e^{-s}\int_{t-s}^\infty e^{-u} T_{u} f(X_s) \, du \\
&\leq e^{-s} \int_{0}^\infty e^{-u} T_{u} f(X_s) \, du \\
&=e^{-s} R_1 f(X_s) = Y_s
\end{align*}
TODO:  Make sure everything here is sensible with respect to working on $S$ versus on $S^\Delta$.  The main point is that one can compute the resolvent of $f$ using either $T_t$ or the extension $T^\Delta_t$; however since $f(\Delta)=0$ we know that $T_t f = T^\Delta_t f$.

TODO: What about the use of $\mathcal{F}^X_+$ in EK?
\end{proof}

\begin{lem}\label{FellerSemigroupProbabilisticStrongContinuity}Let $(S, \rho)$ be compact separable metric space, let $X^x$ be a Markov process on $S$ starting at $x \in S$ and with a Feller transition semigroup $T_t$, then $\lim_{h \to 0^+} \sup_{x} \sexpectation{\rho(X^x_{t+h}, X^x_t) \wedge 1}{x} = 0$ for all $t \geq 0$.  In particular for every initial distribution $\mu$,  $X^\nu_{t+h} \toprob X^\nu_t$ as $h \downarrow 0$ for all $t \geq 0$.
\end{lem}
\begin{proof}
Since $S$ is compact and separable it follows that $C(S)$ is separable (Lemma \ref{SeparabilityOfBoundedUniformlyContinuous}).  Pick a countable dense set $f_1, f_2, \dotsc \in C(S)$.  

\begin{clm}Let $x_n$ be a sequence in $S$ then $\lim_{n \to \infty} x_n = x$ if and only if $\lim_{n \to \infty} f_m(x_n) = f_m(x)$ for all $m \in \naturals$.
\end{clm}
Clearly if $x_n \to x$ then $f_m(x_n) \to f_m(x)$ for all $f_m$ since $f_m$ is continuous.  On the other hand, suppose that $x_n$ does not converge.  Then by compactness there exists $x \in S$ and a subsequence $N \subset \naturals$ such that $x_n$ converges to $x$ along $N$.  Since $x_n$ does not have a limit there exists an open neighborhood of $x$ such that $x_n \notin U$ infinitely often; again by compactness we may pass to a convergent subsequence $N^\prime \subset \naturals$ and by construction $x_n$ converges to $y$ along $N^\prime$ and $x \neq y$.  The function $\rho(x, \cdot)$ is continuous and $\rho(x,x) \neq \rho(x,y)$.  Pick $f_m$ such that $\sup_{s \in S} \abs{\rho(x,s) - f_m(s)} < \rho(x,y)/2$ in particular by the triangle inequality $\abs{f_m(x) - f_m(y)} \geq \rho(x,y) - \abs{\rho(x,y) - f_m(y)} - \abs{f_m(x)} > 0$ and $f_m(x) \neq f_m(y)$.   By continuity of $f_m$, $f_m(x_n) \to f_m(x)$ along $N$ and $f_m(x_n) \to f_m(y)$ along $N^\prime$ and therefore $f_m(x_n)$ does not converge.

By the claim, it follows that $\rho$ is topologically equivalent to the metric 
\begin{align*}
\rho^{\prime}(x,y) &= \sum_{m=1}^\infty 2^{-m} (\abs{f_m(x) - f_m(y)} \wedge 1)
\end{align*}

Now suppose that $f \in C(S)$, $x \in S$  and $t,h \geq 0$ we compute
\begin{align*}
&\sexpectation{(f(X_t) - f(X_{t+h}))^2}{x}
=\sexpectation{f^2(X_t) - 2 f(X_t) f(X_{t+h}) + f ^2 (X_{t+h})} {x}\\
&=\sexpectation{f^2(X_t) - 2 f(X_t) \cexpectationlong{\mathcal{F}_t}{f(X_{t+h})} + \cexpectationlong{\mathcal{F}_t} {f ^2 (X_{t+h})}}{x} \\
&=\sexpectation{f^2(X_t) - 2 f(X_t) T_hf(X_{t}) + T_h f ^2 (X_{t})}{x} \\
&\leq \sup_{x \in S} \abs{f^2(x) - 2 f(x) T_hf(x) + T_h f ^2 (x)} \\
&\leq \sup_{x \in S} \abs{2f^2(x) - 2 f(x) T_hf(x)}  + \sup_{x \in S} \abs{T_h f ^2 (x) - f^2(x)} \\
&\leq 2 \sup_{x \in S}\abs{f(x)} \sup_{x \in S} \abs{f(x) - T_hf(x)}  + \sup_{x \in S} \abs{T_h f ^2 (x) - f^2(x)} \\
\end{align*}
From the strong continuity of $T_t$ we conclude $\lim_{h \to 0} \sup_{x \in S} \sexpectation{(f(X_t) - f(X_{t+h}))^2}{x} = 0$ for each fixed $f \in C(S)$.  
By Cauchy-Schwartz or Jensen's Inequality we know that $\sexpectation{\abs{f(X_t) - f(X_{t+h})}}{x}^2 \leq \sexpectation{(f(X_t) - f(X_{t+h}))^2}{x}$ and therefore we also 
have $\lim_{h \to 0} \sup_{x \in S} \sexpectation{\abs{f(X_t) - f(X_{t+h})}}{x} = 0$ for each fixed $f \in C(S)$.
In particular this is true for each
of the $f_m$ and therefore by Tonelli's Theorem (Corollary \ref{TonelliIntegralSum}) and Dominated Convergence we get
\begin{align*}
\lim_{h \to 0^+} \sup_{x} \sexpectation{\rho^\prime(X^x_{t+h}, X^x_t)}{x} 
&=\lim_{h \to 0^+} \sup_{x} \sum_{m=1}^\infty 2^{-m} \sexpectation{\abs{f_m(X^x_{t+h}) - f_m(X^x_t)} \wedge 1}{x} \\
&\leq \lim_{h \to 0^+} \sum_{m=1}^\infty 2^{-m} \sup_{x} \sexpectation{\abs{f_m(X^x_{t+h}) - f_m(X^x_t)} \wedge 1}{x} \\
&= \sum_{m=1}^\infty 2^{-m} \lim_{h \to 0^+} \sup_{x} \sexpectation{\abs{f_m(X^x_{t+h}) - f_m(X^x_t)} \wedge 1}{x} = 0\\
\end{align*}

TODO: Now argue that this implies the result for $\rho$ not just $\rho^\prime$.

To see the last statement by Lemma \ref{MarkovMixtures} and Dominated Convergence we get
\begin{align*}
\lim_{h \to 0^+} \sexpectation{\rho(X_{t+h}, X_t) \wedge 1}{\nu} &= \lim_{h \to 0^+} \int \sexpectation{\rho(X_{t+h}, X_t) \wedge 1}{x} \, \nu(dx) \\
&= \int \lim_{h \to 0^+} \sexpectation{\rho(X_{t+h}, X_t) \wedge 1}{x} \, \nu(dx) = 0
\end{align*}
so we use Lemma \ref{ConvergenceInProbabilityAsConvergenceInExpectation}.
\end{proof}

\begin{thm}\label{CadlagModificationFellerProcess}Let $S$ be locally compact separable metric space, let $T_t$ be a Feller semigroup on $S$ and let $X$ be the Markov process 
on $S^\Delta$ defined by $T_t$ and initial distribution $\nu$.  Then $X$ has a cadlag version $\tilde{X}$ such that $\Delta$ is an absorbing state for $\tilde{X}$.  If $T_t$ is conservative then $X$ has a cadlag version $\tilde{X}$ with values in $S$.
\end{thm}
\begin{proof}
By Lemma \ref{FellerResolventsAreSupermartingales} we know that $e^{-t} R_1 f(X_t)$ is a supermartingale for any $f \in C_0(S)$ with $f \geq 0$.  Thus applying Theorem \ref{CadlagModificationContinuousMartingale} we know that there is a $P_\nu$-null set in $N_f \in \mathcal{F}^X_\infty$ such that the restriction of $e^{-t} R_1 f(X_t)$ to $\rationals_+$ has left and right limits for all $t \geq 0$ outside of $N_f$; multiplying by the continuous function $e_t$, the same statement holds for $R_1 f(X_t)$.  Writing an arbitrary $f \in C_0(S)$ as $f = f_+ - f_-$ with $f_\pm \in C_0(S)$ and $f_\pm \geq 0$ and using the linearity of $R_1$ we see that $R_1 f(X_t)$ almost surely has all left and right limits along $\rationals_+$ for arbitrary $f \in C_0(S)$.  By definition of the resolvent and the Hille-Yosida theorem we know that $\range{R_1 f}$ is dense in $C_0(S)$ (or is it $C(S^\Delta)$?).
Thus given an arbitrary $f \in C_0(S)$ we may find $f_n \in C_0(S)$ such that $\lim_{n \to \infty} \sup_{x \in S} \norm{R_1 f_n(x) - f(x)}$.  For every $t \geq 0$, $q \in \rationals_+$ and $n \in \naturals$ we write
\begin{align*}
\abs{f(X_q) - f(X_t)} &\leq \abs{f(X_q) - R_1f_n(X_q)} + \abs{R_1f_n(X_q) - R_1f_n(X_t)} + \abs{R_1 f_n(X_t) - f(X_t)} \\
&\leq 2 \sup_{x \in S} \norm{R_1 f_n(x) - f(x)} + \abs{R_1f_n(X_q) - R_1f_n(X_t)}
\end{align*}
Each $R_1 f_n$ has left and right limits along $\rationals_+$ outside a null set $N_{n}$.  Letting either $q \uparrow t$ or $q \downarrow t$ and then taking $n \to \infty$ we see that $f(X_t)$ has left and right limits along $\rationals_+$ outside the null set $\cup_n N_n$.  Note also that $C_0(S)$ is separable (Corollary \ref{VanishingAtInfinityLocallyCompactSeparable}) so we may find a countable dense subset $f_1, f_2,\dotsc \in C_0(S)$ and letting $N = \cup_n N_{f_n}$ it follows by repeating the same limiting argument as above that for all $f \in C_0(S)$, $R_1 f$ has left and right limits along $\rationals$ outside of $N$ (where $N$ is now independent of $f$). 

\begin{clm} $X$ has left and right limits along $\rationals_+$ outside of the null set $N$.
\end{clm}
By a slight modification of the argument in Lemma \ref{FellerSemigroupProbabilisticStrongContinuity} one sees that the compactness of $S^\Delta$ implies that if we are given
a sequence $x_1, x_2, \dotsc \in S^\Delta$ such that $f(x_n)$ converges for every $f \in C_0(S)$ then it follows that $x_n$ converges in $S^\Delta$ (observe that if there are two convergent subsequences which converge to different elements of $S^\Delta$ then we can separate the two points with an element $f \in C_0(S)$).  From this fact, for every $\omega \notin N$ and we know that all increasing or decreasing sequences $q_n \in \rationals_+$ we have $\lim_{n \to \infty} f(X_{q_n}(\omega))$ exists and thus $\lim_{n \to \infty} X_{q_n}(\omega)$ exists.

Now we can define $\tilde{X}_t \equiv \Delta$ on $N$ and $\tilde{X}_t = \lim_{\substack{q \to t^+ \\ q \in \rationals}} X_q$ off of $N$ and by the proof of Theorem \ref{CadlagModificationContinuousMartingale} we know that $\tilde{X}_t$ is cadlag.  
\begin{clm} $\tilde{X}$ is a version of $X$.
\end{clm}
The claim means simply that for every $t \geq 0$ we have $\lim_{\substack{q \to t^+ \\ q \in \rationals}} X_q =X_t$ a.s. We know that the limit in question exists almost surely (i.e. off of $N$) so it suffices to show that $\lim_{\substack{q \to t^+ \\ q \in \rationals}} X_q =X_t$ a.s. along a subsequence.  Lemma \ref{FellerSemigroupProbabilisticStrongContinuity} we know that $X_q \toprob X_t$ and the convergence of the subsequence follows by Lemma \ref{ConvergenceInProbabilityAlmostSureSubsequence}.

Now let $f \in C_0(S)$ with $f > 0$ on $S$ (e.g. $f(x) = \rho(\Delta, x)$).  Note that $R_1 f > 0$; by strong continuity for each $x \in S$ we pick $\delta>0$ such that $\norm{T_t f - f} < f(x)/2$ for $0 \leq t \leq \delta$ and if follows that $\abs{T_t f(x)} \geq  f(x) - \abs{T_tf(x) - f(x)}  \geq f(x)/2 > 0$ for all $0  \leq t \leq \delta$ and therefore by positivity of $T_t$ we have $R_1 f (x) \geq \int_0^\delta e^{-u} T_u f(x) \, du \geq f(x)(1-e^{-\delta})/2 > 0$.  It is also clear that $f(\Delta) = 0$ implies $R_1 f(\Delta) = 0$.  Therefore $Y_t = e^{-t} R_1 f(\tilde{X}_t)$ is a non-negative cadlag supermartingale such that $Y_t = 0$ is equivalent to $\tilde{X}_t = \Delta$.  If we apply Lemma \ref{PositiveSupermartingaleAbsorption} to $Y_t$ and translate the conclusion in terms of $\tilde{X}_t$ we see that if we define $\tau = \inf \lbrace t \mid \tilde{X}_{t-} \wedge \tilde{X}_t =\Delta \rbrace$ then $\tilde{X}_t \equiv \Delta$ a.s. on $[\tau, \infty)$.  Setting $\tilde{X}$ to be identically $\Delta$ on the null set where the conclusion fails we that by a further modification of $\tilde{X}$ we can assume $\Delta$ is absorbing everywhere.

If $T_t$ is conservative on $C_0(S)$ and we assume that $\nu(\lbrace \Delta \rbrace) = 0$ then it follows that $\tilde{X}_t \in S$ a.s. for every $t \geq 0$ (TODO: Why?)  Therefore we must have $\tau > t$ a.s. for all $t \geq 0$ which implies that $\tau = \infty$ a.s.  Pick an arbitrary $x \in S$ and make an additional modification of $\tilde{X}$ to set $\tilde{X}_t \equiv x$ on $\tau < \infty$ and therefore $\tau = \infty$ everywhere.  This implies that $\tilde{X}_t$ and $\tilde{X}_{t-}$ take values in $S$.
\end{proof}

TODO:  Show that we get a homogeneous cadlag Markov family out of this construction.  At a minimum we can see that we get probability measures $\sprobabilityop{x}$ for each $x \in S^\Delta$ and this is a kernel from $S$ to $S^{[0,\infty)}$ (by Lemma \ref{MarkovMixtures}).

\begin{thm}\label{StrongMarkovFellerProcess} Let $X$ be a Feller family, $A \in (\mathcal{S}^\Delta)^{[0,\infty)}$  and let $\tau$ be an $\mathcal{F}^X_+$-optional time, then
\begin{align*}
\cprobability{\mathcal{F}^X_{\tau+}}{\theta_{\tau} X \in A} &= \sprobability{A}{X_\tau} \text{ a.s. on $\tau < \infty$}
\end{align*}
Let $X$ be the canonical Feller process, $\nu$ be an initial distribution, $\tau$ an $\mathcal{F}^X_+$-optional time, $\xi$ a non-negative random variable then
\begin{align*}
\csexpectationlong{\mathcal{F}^X_{\tau+}}{\xi \circ \theta_{\tau} }{\nu} &= \sexpectation{\xi}{X_\tau} \text{ $\sprobabilityop{\nu}$ a.s. on $\tau < \infty$}
\end{align*}
\end{thm}
\begin{proof}
By Proposition \ref{StrongMarkovFromStrongMarkovFiniteOptional} it suffices to assume that $\tau<\infty$ almost surely.  Let $A \in (\mathcal{S}^\Delta)^{[0,\infty)}$.  Define $\tau_n = 2^{-n}\floor{2^n \tau + 1}$ so that by Lemma \ref{DiscreteApproximationOptionalTimes} we know that $\tau_n$ are $\mathcal{F}^X$-optional times and $\tau_n \downarrow \tau$.  Since $\tau < \tau_n$ we also know that $\mathcal{F}^X_{\tau+} \subset \mathcal{F}^X_{\tau_n}$ for all $n \in \naturals$.  Since $\tau_n$ is countably valued $X$ is strong Markov at $\tau_n$ (Theorem \ref{StrongMarkovPropertyMarkovProcessCountableValues}) and
\begin{align*}
\cprobability{\mathcal{F}^X_{\tau_n}}{\theta_{\tau_n} X \in A} &=
\sprobability{A}{X_{\tau_n}} \text{ a.s. on $\tau_n < \infty$}
\end{align*}

It is useful to avoid appeal to Theorem \ref{StrongMarkovPropertyMarkovProcessCountableValues} so that the role of the Feller properties can be appreciated even in the case of countably valued $\tau$ (not sure I agree because the same continuity argument seems to appear when approximating by countably valued).  So let $\tau$ be $\mathcal{F}^X_+$-optional and let $T$ be the countable range of $\tau$, $f \in C(S^\Delta)$, $s \geq 0$ and $A \in \mathcal{F}^X_{t+}$.  Thus,
for every $t \geq 0$, $\epsilon > 0$ we have $A \cap \lbrace \tau=t \rbrace \in \mathcal{F}^X_{t+} \subset \mathcal{F}^X_{t+\epsilon}$.  Let $0 \leq \epsilon \leq s$ and use
the tower property of conditional expectation and Proposition \ref{TransitionSemigroupAsExpectation} to get
\begin{align*}
\expectation{f(X_{\tau + s}) ; A} &= \sum_{t \in T} \expectation{f(X_{t + s}) ; A \cap \lbrace \tau=t \rbrace } \\
&= \sum_{t \in T} \expectation{\cexpectationlong{\mathcal{F}^X_{t+\epsilon}}{f(X_{t + s})} ; A \cap \lbrace \tau=t \rbrace} \\
&= \sum_{t \in T} \expectation{T_{s-\epsilon} f (X_{t + \epsilon}) ; A \cap \lbrace \tau=t \rbrace} \\
&= \expectation{T_{s-\epsilon} f (X_{\tau + \epsilon}) ; A} \\
\end{align*}
By strong continuity of $T_tf$ and right continuity of $X_t$ we know that $\lim_{\epsilon \to 0} T_{s-\epsilon} f (X_{\tau + \epsilon}) = T_s f(X_\tau)$ (TODO: Show this in more detail; actually we don't need this since we only need countably valued $\mathcal{F}^X$-optional times) and therefore by Dominated Convergence
we get $\expectation{f(X_{\tau + s}) ; A} = \expectation{T_{s} f (X_{\tau }) ; A} $.   Since $T_s f (X_\tau)$ is $\mathcal{F}^X_{\tau+}$-measurable we conclude 
\begin{align*}
\cexpectationlong{\mathcal{F}^X_{\tau+}}{f(X_{\tau + s})} = T_s f (X_\tau)
\end{align*}
To extend this fact to arbitrary $\tau$ with $\tau < \infty$,


In what follows we are using a claim that $\sigma < \tau$ implies $\mathcal{F}_{\sigma+} \subset \mathcal{F}_\tau$.  TODO: Prove it or disprove it (if it turns out not to be true then we use the $\epsilon$ argument above taken from EK).

Let  $\tau_n = 2^{-n}\floor{2^n \tau + 1}$ so that $\tau_n$ are countably valued $\mathcal{F}^X$-optional times
such that  $\tau < \tau_n$  and $\tau_n \downarrow \tau$(Lemma \ref{DiscreteApproximationOptionalTimes}).  Now we use continuity of $T_sf$, Dominated Convergence for conditional expectations and the $\mathcal{F}^X_{\tau+}$-measurability of $T_s f (X_{\tau})$ (since $X$ is cadlag, it is progressive by Lemma \ref{ContinuityAndProgressiveMeasurability} thus $X_\tau$ is $\mathcal{F}^X_{\tau+}$-measurable by Lemma \ref{StoppedProgressivelyMeasurableProcess}) to see
\begin{align*}
&\cexpectationlong{\mathcal{F}^X_{\tau+}}{f(X_{\tau + s})} \\
&= \cexpectationlong{\mathcal{F}^X_{\tau+}}{\lim_{n \to \infty}  f(X_{\tau_n + s})} && \text{right continuity of $X$, continuity of $f$}\\
&= \lim_{n \to \infty}  \cexpectationlong{\mathcal{F}^X_{\tau+}}{f(X_{\tau_n + s})} && \text{Dominated Convergence}\\
&= \lim_{n \to \infty} \cexpectationlong{\mathcal{F}^X_{\tau+}}{\cexpectationlong{\mathcal{F}^X_{\tau_n}}{f(X_{\tau_n + s})}} && \text{chain rule of conditional expectations}\\
&=\lim_{n \to \infty} \cexpectationlong{\mathcal{F}^X_{\tau+}}{T_s f (X_{\tau_n})} &&  \text{Theorem \ref{StrongMarkovPropertyMarkovProcessCountableValues} and Proposition \ref{TransitionSemigroupAsExpectation}}\\
&=\cexpectationlong{\mathcal{F}^X_{\tau+}}{\lim_{n \to \infty} T_s f (X_{\tau_n})}  && \text{Dominated Convergence}\\
&=\cexpectationlong{\mathcal{F}^X_{\tau+}}{T_s f (X_{\tau})} && \text{right continuity of $X$, continuity of $T_sf$}\\
&=T_s f (X_{\tau}) && \text{$\mathcal{F}^X_{\tau+}$-measurability of $T_s f (X_{\tau})$}\\
&=\int f(u) \, \mu_s (X_{\tau}, du) \\
\end{align*}
Now we may apply Proposition \ref{StrongMarkovFromOneDimensionalDistribution} to complete the proof.

TODO: Show the canonical case...
\end{proof}

We first need to introduce the formalize notation and concepts surrounding bounded pointwise limits and closure.

\begin{defn}Let $S$ be a metric space and let $B_b(S)$ be the space of bounded measurable functions then given $f, f_1, f_2, \dotsc \in B_b(S)$ we say that $f = \bplim_{n \to \infty} f_n$ if and only if $sup_n \sup_{x \in S} \abs{f(x)} < \infty$ and $\lim_{n \to \infty} f_n(x) = f(x)$ for all $x \in S$.  We say that a set $M \subset S$ is \emph{bp-closed} if $f_1, f_2, \dotsc \in M$ and $f = \bplim_{n \to \infty} f_n$ implies $f \in M$.  
\end{defn}

TODO:  On $C_0(S)$ we know (or should prove) that the dual $C^*_0(S)$ is the space of finite signed Radon measures.  It turns out that a sequence $f = \bplim f_n$ if and only if $f_n$ converges weakly to $f$ (i.e. $\int f_n \, d \mu \to \int f  \, d\mu$ for all finite signed Radon measures $\mu$).  This is not true for arbitrary nets however.

\begin{defn}A multivalued operator $A \subset B_b(S) \times B_b(S)$ is \emph{conservative} if and only if $(1,0)$ is contained in the bp-closure of $A$.
\end{defn}

TODO: Understand the relationship between this definition of conservative and the more elementary statement that $T_t 1 = 1$ for all $t \geq 0$.  Understand the relationship between this definition and the Kallenberg definition of conservativeness of a the semigroup (NOT the generator) that says $\sup_{f \leq 1} T_t f (x) = 1$ for all $x \in S$.  Understand the relationship between convervativeness and the statement that a sub-Markov transition semigroup is Markov (i.e. the property that every measure has total mass 1).  Yet another defintion of conservative that is specific to Feller semigroups: if $f_n \uparrow 1$, $f_n \in C_0(S)$ then $T_t f_n \uparrow 1$ (this is essentially Kallenberg's definition).  I believe that the following is true: $(1,0) \in bp-closure(A)$ implies $T_t 1 = 1$ for all $t \geq 0$.  Also if $S$ is compact then $T_t 1 = 1$ for all $t \geq 0$ if and only if $1 \in \domain{A}$ and $A 1 = 0$.  Note also the fact that $1 \notin C_0(S)$ if $S$ is not compact so the definition in terms of $T_t 1 = 1$ not correct; it can be rescued by proving another theorem (which is a consequence of Riesz representation) that says any semigroup of continuous operators on $C_0(S)$ can be extended to a semigroup of continuous operators on $B_b(S)$.

\begin{lem}Let $A$ be the generator of a strongly continuous contraction semigroup on a subspace of $X \subset B_b(S)$ then if $A$ is conservative then
$T_t 1 = 1$ for all $t \geq 0$ (TODO: This statement requires the extension of $T_t$ to $C_b(S)$ or at least a space that contains $1$; this should follow from the construction of the transition semigroup constructed from the Feller semigroup).
\end{lem}
\begin{proof}
By the Kolmogorov backward equation Proposition \ref{StronglyContinuousSemigroupKolomgorovBackwardEquation} we know that 
\begin{align*}
\lbrace (f, Af) \mid f \in \domain{A} \rbrace &\subset \lbrace (f,g) \in B_b(S) \times B_b(S) \mid T_t f- f = \int_0^t T_s g \, ds \text{ for all $t\geq 0$} \rbrace
\end{align*}
\begin{clm}The right hand set is bp-closed
\end{clm}
Suppose $(f_n, g_n)$ is a sequence in the right hand set such that $\sup_n \sup_{x\in S} \abs{f_n(x)} < \infty$, $\sup_n \sup_{x\in S} \abs{g_n(x)} < \infty$, $\lim_{n \to \infty} (f_n(x), g_n(x)) = (f(x), g(x))$  for all $x \in S$.  In particular it is the case that $\lim_{n \to \infty} f_n = f$ and $\lim_{n \to \infty} g_n = g$ in $X$ (TODO: WRONG we only have pointwise limits not uniform limits; how do we know that $T_t f_n(x) \to T_t f(x)$ and $T_t g_n(x) \to T_t g(x)$?  I believe the answer is that for contraction semigroups norm continuity and weak continuity coincide) and therefore $\lim_{n \to \infty} T_t f_n = T_t f$
and $\lim_{n \to \infty} T_t g_n = T_t g$ for all $t \geq 0$.  Since $T_t$ is a contraction operator we also have $\sup_n \sup_{x \in S} \abs{T_t g_n(x)} < \infty$ and therefore by Dominated Convergence we also have $\lim_{n \to \infty} \int_0^t T_s g_n \, ds = \int_0^t T_s g \, ds$.  

Now since $A$ is conservative we conclude that $(1,0)$ is in the right hand set and lemma follows.
\end{proof}

\section{Approximation of Feller Processes}

In this section we begin considering the weak convergence theory of
Feller Processes.  Semigroup theory plays a central role and we begin by proving a convergence
theorem for strongly continuous contraction semigroups that will be applied in the Feller case.  We begin with 
two lemmas. The first extends Lemma \ref{SemigroupBoundInTermsOfBoundedGenerator}.

\begin{lem}\label{SemigroupBoundInTermsOfBoundedGenerator2}Let $X$ and $Y$ be Banach spaces, let $T_t$ be a strongly continuous contraction semigroup on $X$ with generator $A$,
let $S_t$ be a strongly continuous contraction semigroup on $Y$ with generator $B$ and let $\pi : X \to Y$ be a bounded linear operator.  Let
$v \in \domain{A}$ assume that $\pi T_t v \in \domain{B}$ for all $t \geq 0$ and that $B \pi T_t v : [0,\infty) \to Y$ is continuous then
\begin{align*}
S_t \pi v - \pi T_t v &= \int_0^t S_{t-s} (B \pi - \pi A) T_s v \, ds
\end{align*}
for all $t \geq 0$ and in particular, 
\begin{align*}
\norm{S_t \pi v - \pi T_t v} &= \int_0^t \norm{B \pi - \pi A) T_s v} \, ds
\end{align*}
\end{lem}
\begin{proof}
Let $v \in \domain{A}$ and consider the term $S_{t - s} \pi T_s v$ for $0 \leq s \leq t$.  Since $v \in \domain{A}$ we know that $T_s v$ is a differentiable function of $s$ and 
$\frac{d}{ds} T_s v = A T_s v$ (Proposition \ref{StronglyContinuousSemigroupKolomgorovBackwardEquation}).   Since $\pi$ is a bounded linear map we know from the chain rule (Proposition \ref{ChainRuleBanachSpaces}) that $\pi T_s v$ is also differentiable and $\frac{d}{ds} \pi T_s v = \pi A T_s v$.  Also for fixed $w \in \domain{B}$ and $0 \leq s \leq t$ we know from Proposition \ref{StronglyContinuousSemigroupKolomgorovBackwardEquation} that $S_{t -s} w$ is differentiable with respect to $s$ and $\frac{d}{ds} S_{t -s} w = - S_{t-s} B w$.  We claim that we have a product rule for differentiation that shows
\begin{align*}
\frac{d}{ds} S_{t - s} \pi T_s v = S_{t-s} \pi A T_s v - S_{t-s} B \pi T_s v = -S_{t -s} (B \pi - \pi A) T_s v
\end{align*}
TODO: Show this

Now we can just apply the Fundamental Theorem of Calculus \ref{FundamentalTheoremOfCalculusForBanachSpaceRiemannIntegrals} to see that
\begin{align*}
\int_0^t S_{t -s} (B \pi - \pi A) T_s v \, ds &= -S_{t - s} \pi T_s v \mid_{0}^t = S_t \pi v - \pi T_t v
\end{align*}
Also we can just apply Proposition \ref{NormRiemannIntegralBanachSpace} and the fact that $S_t$ is a contraction to see
\begin{align*}
\norm{S_t \pi v - \pi T_t v} &\leq \int_0^t \norm{S_{t -s} (B \pi - \pi A) T_s v} \, ds \leq \int_0^t \norm{(B \pi - \pi A) T_s v} \, ds
\end{align*}
\end{proof}

The second required lemma is a continuity property of the Yosida approximation.
\begin{lem}\label{SCCSYosidaContinuity}Let $X$ and $X_1, X_2, \dotsc$ be Banach spaces, $T_{n,t}$ be strongly continuous contraction semigroups on $X_n$ with generator $A_n$ and let $T_t$ be a strongly continuous contraction semigroup on $X$ with generator $A$.  Let $\pi_n : X \to X_n$ be bounded linear operators and assume that $\sup_{n} \norm{\pi_n} < \infty$.  Let $D$ be a
core for $A$ and suppose that for every $v \in D$ there exists $v_n \in \domain{A_n}$ such that $\lim_{n \to \infty} \norm{v_n - \pi_n v} = 0$ and $\lim_{n \to \infty} \norm{A_n v_n - \pi_n A v} = 0$.  Suppose that $A^\lambda_n$ and $A^\lambda$ denote the Yosida approximations of $A_n$ and $A$ then for all $v \in X$ and $\lambda > 0$ we have
\begin{align*}
\lim_{n \to \infty} \norm{A^\lambda_n \pi_n v - \pi_n A^\lambda v} &= 0
\end{align*}
\end{lem}
\begin{proof}
Let $\lambda > 0$ and $v \in D$.  Set $w = (\lambda - A) v$ and pick $v_n \in \domain{A_n}$ such that $\lim_{n \to \infty} {v_n - \pi_n v} = \lim_{n \to \infty} {A_nv_n - \pi_n Av} = 0$.  Note that by the triangle inequality $\lim_{n \to \infty} \norm{(\lambda - A_n) v_n - \pi_n w} = 0$.  Recall that since 
\begin{align*}
A^\lambda &= \lambda A R_\lambda = \lambda (\lambda - (\lambda - A)) R_\lambda = \lambda^2 R_\lambda - \lambda \IdentityMatrix
\end{align*}
and from $\norm{R_\lambda} \leq \lambda^{-1}$ shows that $\norm{A^\lambda} \leq 2 \lambda$.  The same holds for $A^\lambda_n$ where we use the notation $R_{n, \lambda}$ for the resolvent of $T_{n,t}$.

Using the above identity we compute
\begin{align*}
\norm{A^\lambda_n \pi_n w  - \pi_n A^\lambda w} 
&= \norm{(\lambda^2 R_{n,\lambda} - \lambda \IdentityMatrix) \pi_n w  - \pi_n (\lambda^2 R_{\lambda} - \lambda \IdentityMatrix)  w} \\
&=\lambda^2 \norm{(R_{n,\lambda} \pi_n - \pi_n R_{\lambda}) w}  \\
&\leq \lambda^2 \norm{R_{n,\lambda} \pi_n w - v_n}  + \lambda^2 \norm{v_n - \pi_n R_{\lambda} w}  \\
&= \lambda^2 \norm{R_{n,\lambda} [\pi_n w - (\lambda - A_n) v_n]}  + \lambda^2 \norm{v_n - \pi_n v}  \\
&= \lambda \norm{\pi_n w - (\lambda - A_n) v_n}  + \lambda^2 \norm{v_n - \pi_n v}  \\
\end{align*}
and therefore we see that $\lim_{n \to \infty} \norm{A^\lambda_n \pi_n w  - \pi_n A^\lambda w} = 0$ for $w \in (\lambda -A)(D)$.

On the other hand $(\lambda - A)(D)$ is dense in $X$ (Proposition \ref{SCCSCoreViaDenseRange}) and 
\begin{align*}
\norm{A^\lambda_n \pi_n - \pi_n A^\lambda} &\leq (\norm{A^\lambda_n} + \norm{A^\lambda}) \norm{\pi_n} \leq 4 \lambda \sup_{n} \norm{\pi_n} < \infty
\end{align*}
and therefore $\lim_{n \to \infty} \norm{A^\lambda_n \pi_n w  - \pi_n A^\lambda w} = 0$ for $w \in X$ (let $w_m \to w$ with $w_m \in  (\lambda -A)(D)$, write
$\norm{A^\lambda_n \pi_n w  - \pi_n A^\lambda w} \leq \norm{A^\lambda_n \pi_n w_m  - \pi_n A^\lambda w_m} + \sup_n \norm{A^\lambda_n \pi_n - \pi_n A^\lambda} \norm{w_m -w}$
and then let $n \to \infty$ followed by $m \to \infty$).
\end{proof}

Now we can present the semigroup convergence theorem itself.

\begin{thm}\label{KurtzSovaSemigroupConvergence}Let $X$ and $X_1, X_2, \dotsc$ be Banach spaces, $T_{n,t}$ be strongly continuous contraction semigroups on $X_n$ with generator $A_n$ and let $T_t$ be a strongly continuous contraction semigroup on $X$ with generator $A$.  Let $\pi_n : X \to X_n$ be bounded linear operators and assume that $\sup_{n} \norm{\pi_n} < \infty$.  Let $D$ be a
core for $A$ then the following are equivalent
\begin{itemize}
\item[(i)]For every $v \in X$ and every $t \geq 0$,  $\lim_{n \to \infty} \sup_{0 \leq s \leq t} \norm{T_{n,s} \pi_n v - \pi_n T_s v} = 0$.
\item[(ii)]For every $v \in X$ and every $t \geq 0$,  $\lim_{n \to \infty} \norm{T_{n,t} \pi_n v - \pi_n T_t v} = 0$.
\item[(iii)] For every $v \in D$ there exists $v_n \in \domain{A_n}$ such that $\lim_{n \to \infty} \norm{v_n - \pi_n v} = 0$ and $\lim_{n \to \infty} \norm{A_n v_n - \pi_n A v} = 0$.
\end{itemize}
\end{thm}
\begin{proof}
The implication (i) implies (ii) is immediate.

To see that (ii) implies (iii), let $\lambda > 0$, $v \in \domain{A}$ and set $w = (\lambda - A) v$.  By Proposition \ref{SCCSResolventAsLaplaceTransform} we have
\begin{align*}
v &= R_\lambda w = \int_0^\infty e^{- \lambda t} T_t w \, dt
\end{align*}
Now define
\begin{align*}
v_n &= \int_0^\infty e^{- \lambda t} T_{n,t} \pi_n w \, dt = R_{n,\lambda} \pi_n w
\end{align*}
and note that by (ii) we have $\lim_{n \to \infty} \norm{T_{n,t} \pi_n w - \pi_n T_t w} = 0$ and since $T_{n,t}$ and $T_t$ are contractions we have $\sup_n \norm{T_{n,t} \pi_n w - \pi_n T_t w} \leq 2 \sup_n \norm{\pi_n} \norm{w} < \infty$. Now apply Proposition \ref{ClosedOperatorOfRiemannIntegral}, Proposition \ref{NormRiemannIntegralBanachSpace} and Dominated Convergence
\begin{align*}
\lim_{n \to \infty} \norm{v_n - \pi_n v} &= \lim_{n \to \infty} \norm{\int_0^\infty e^{- \lambda t} T_{n,t} \pi_n w \, dt - \pi_n \int_0^\infty e^{- \lambda t} T_t w \, dt} \\
&= \lim_{n \to \infty} \norm{\int_0^\infty e^{- \lambda t} T_{n,t} \pi_n w \, dt - \int_0^\infty e^{- \lambda t} \pi_n  T_t w \, dt} \\
&\leq \lim_{n \to \infty} \int_0^\infty e^{- \lambda t} \norm{T_{n,t} \pi_n w \, dt - \pi_n  T_t w} \, dt \\
&= \int_0^\infty e^{- \lambda t} \lim_{n \to \infty} \norm{T_{n,t} \pi_n w \, dt - \pi_n  T_t w} \, dt = 0\\
\end{align*}
Also note that
\begin{align*}
\norm{A_n v_n - \pi_n A v} &= \norm{\lambda v_n - (\lambda - A_n) v_n - \lambda \pi_n v + \pi_n w} \\
&\leq \norm{\lambda v_n - \lambda \pi_n v + \pi_n (\lambda - A) v} + \norm{\pi_n w - (\lambda - A_n) v_n} \\
&= \norm{\lambda v_n - \lambda \pi_n v + \pi_n (\lambda - A) v} + \norm{\pi_n w - (\lambda - A_n) R_{n,\lambda} \pi_n w} \\
&= \norm{\lambda v_n - \lambda \pi_n v + \pi_n (\lambda - A) v} \\
\end{align*}
and therefore (iii) is proven.

To see that (iii) implies (i) let $A^\lambda_n$ and $A^\lambda$ be the Yosida approximations of $A_n$ and $A$ respectively; let $T^\lambda_{n,t} = e^{t A^\lambda_n}$ and $T^\lambda_t = e^{tA^\lambda}$ be the corresponding strongly continuous contraction semigroups.  Let $v \in D$ and choose $v_n \in \domain{A_n}$ such that $\lim_{n \to \infty} \norm{v_n - \pi_n v}=0$ and $\lim_{n \to \infty} \norm{A_n v_n - \pi_nA v}=0$.  Use the triangle inequality to write
\begin{align*}
\norm{T_{n,t} \pi_n v - \pi_n T_t v} &\leq \norm{T_{n,t}(\pi_n v - v_n)} + \norm{T_{n,t}v_n - T^\lambda_{n,t}v_n } + \norm{T^\lambda_{n,t} (v_n - \pi_n v)}  \\
&+\norm{T^\lambda_{n,t} \pi_n v - \pi_n T^\lambda_{t} v } + \norm{\pi_n ( T^\lambda_{t} v - T_t v)}
\end{align*}
Now we consider estimates of each of the five terms on the right hand side.  Let $t \geq 0$ be fixed.  For the first term note that since $T_{n,t}$ is a contraction
\begin{align*}
\limsup_{n \to \infty} \sup_{0 \leq s \leq t} \norm{T_{n,s}(\pi_n v - v_n)}  &\leq \lim_{n \to \infty} \norm{\pi_n v - v_n}  = 0
\end{align*}
and the same argument works for the third term as well.  For the second term, apply Corollary \ref{YosidaApproximationContinuity}, the triangle inequality, Lemma \ref{SCCSYosidaContinuity} and the fact that $\norm{A^\lambda_n} \leq 2 \lambda$ to see that
\begin{align*}
&\limsup_{n \to \infty} \sup_{0 \leq s \leq t} \norm{T_{n,s}v_n - T^\lambda_{n,s}v_n } 
\leq t \limsup_{n \to \infty} \norm{A_n v_n - A^\lambda_n v_n} \\
&\leq t \limsup_{n \to \infty} \left [ \norm{A_n v_n -\pi_n A v} + \norm{\pi_n A v - \pi_n A^\lambda v} + \norm{\pi_n A^\lambda v - A^\lambda_n \pi_n v} + \norm{A^\lambda_n \pi_n v - A^\lambda_n v_n} \right]\\
&\leq t \sup_n \norm{\pi_n}  \norm{A v - A^\lambda v}
\end{align*}

For the fourth term we apply Lemma \ref{SemigroupBoundInTermsOfBoundedGenerator2} to see that 
\begin{align*}
\sup_{0 \leq s \leq t} \norm{T^\lambda_{n,s} \pi_n v - \pi_n T^\lambda_{s} v } 
&\leq \int_0^t \norm{(A^\lambda_n \pi_n - \pi_n A^\lambda) T^\lambda_s v} \, ds
\end{align*}
By Lemma \ref{SCCSYosidaContinuity} we know that $\lim_{n \to \infty} \norm{(A^\lambda_n \pi_n - \pi_n A^\lambda) T^\lambda_s v} = 0$ and moreover $\norm{(A^\lambda_n \pi_n - \pi_n A^\lambda) T^\lambda_s v} \leq 4 \lambda \sup_n \norm{\pi_n} \norm{v}$ so that by Dominated Convergence we get  
\begin{align*}
\limsup_{n \to \infty} \sup_{0 \leq s \leq t} \norm{T^\lambda_{n,s} \pi_n v - \pi_n T^\lambda_{s} v } = 0
\end{align*}
For the fifth and final term we apply Corollary \ref{YosidaApproximationContinuity} to get
\begin{align*}
\limsup_{n \to \infty} \sup_{0 \leq s \leq t} \norm{\pi_n ( T^\lambda_{s} v - T_s v)} &\leq \sup_{n} \norm{\pi_n} \sup_{0 \leq s \leq t} \norm{T^\lambda_{s} v - T_s v} \\
&\leq t \sup_{n} \norm{\pi_n} \norm{A^\lambda v - A v}
\end{align*}

Putting all of the estimates together we see that
\begin{align*}
\limsup_{n \to \infty} \sup_{0 \leq s \leq t} \norm{T_{n,t} \pi_n v - \pi_n T_t v}  &\leq 2 t \sup_{n} \norm{\pi_n} \norm{A^\lambda v - A v}
\end{align*}
Letting $\lambda \to \infty$ and using Lemma \ref{ClosedDissipativeYosidaApproximation} we see that $\lim_{n \to \infty} \sup_{0 \leq s \leq t} \norm{T_{n,t} \pi_n v - \pi_n T_t v} =0$ for 
$v \in D$.  Since $D$ is dense by the Hille-Yosida Theorem \ref{HilleYosidaTheoremClosable} for $v \in X$ we may take $v_m \in D$ such that $\lim_{m \to \infty} v_m = v$ and then
\begin{align*}
&\limsup_{n \to \infty} \sup_{0 \leq s \leq t} \norm{T_{n,t} \pi_n v - \pi_n T_t v} \\
&\leq \limsup_{n \to \infty} \sup_{0 \leq s \leq t} \left[\norm{T_{n,t} \pi_n v- T_{n,t} \pi_n v_m} + \norm{T_{n,t} \pi_n v_m - \pi_n T_t v_m} + \norm{\pi_n T_t v - \pi_n T_t v_m}  \right] \\
&\leq 2 \sup_n \norm{\pi_n} \norm{v - v_m} + \lim_{n \to \infty} \sup_{0 \leq s \leq t} \norm{T_{n,t} \pi_n v_m - \pi_n T_t v_m} = 2 \sup_n \norm{\pi_n} \norm{v - v_m} 
\end{align*}
Now let $m \to \infty$ and (i) follows.
\end{proof}

TODO: Compare Kallenberg and EK accounts of the following theorem and understand the differences and similarities.  The main difference is that the Theorem in Kallenberg is really
a combination of two theorems in EK.  The first theorem is Theorem \ref{KurtzSovaSemigroupConvergence} and simply addresses equivalent conditions for convergence of sccs.  The second theorem is what we have as Theorem \ref{KurtzMackevicius} below and shows that Feller semigroup convergence implies weak convergence of Markov processes (using the standard FDD and tightness approach).  EK and Kallenberg use the same argument to show FDD convergence whereas Kallenberg uses the Aldous criterion directly but EK use a more sophisticated argument for tightness using the martingale property (which may ultimately boil down to the Aldous criterion).  Also EK are careful to prove the Theorem in such a way that the existence of cadlag $X$ is not assumed but rather it is derived whereas Kallenberg has already proved that a cadlag $X$ may be created from $T_t$ and thus may appeal to the strong Markov property in showing the Aldous criterion is satisfied.

\begin{thm}\label{KurtzMackevicius}Let $(S,r)$ be a locally compact and separable metric space and let $X^n$ for $n \in \naturals$ be
cadlag Feller processes in $S$ with semigroup $T_{n,t}$ on $C_0(S)$ respectively.  Suppose that $T_t$ is a Feller semigroup on $C_0(S)$ and
\begin{align*}
\lim_{n \to \infty} T_{n,t} f = T_t f \text{ for all $f \in C_0(S)$ and $t \geq 0$}
\end{align*}
If the distributions of $X^n(0)$ converges to a limit $\nu \in \mathcal{P}(S)$ then there exists a cadlag Feller process $X$ with semigroup
$T_t$ and initial distribution $\nu$ and $X^n \todist X$.
\end{thm} 
\begin{proof}
For $n \in \naturals$ let $A_n$ be the generator of $T_{n,t}$.

First we assume that $S$ is compact and that the $T_{n,t}$ and $T_t$ are conservative.  TODO: Remove this assumption.

\begin{clm} For all $m \in \integers_+$, $0=t_0 < t_1 < \dotsb < t_m$ and $f_i \in C(S)$ for $i=0, \dotsc, m$ we have
\begin{align*}
\lim_{n \to \infty} \expectation{f_0(X^n_{t_0}) \dotsb f_m(X^n_{t_m})} = \expectation{f_0(X_{t_0}) \dotsb f_{m}(X_{t_{m}})}
\end{align*}
\end{clm}
To see the claim, note that the case $m=0$ is simply the statement that $X^n_0 \todist X_0$ which is an assumption.  For $m > 0$, 
we apply Proposition \ref{TransitionSemigroupAsExpectation}
\begin{align*}
\lim_{n \to \infty} \expectation{f_0(X^n_{t_0}) \dotsb f_m(X^n_{t_m})} 
&=\lim_{n \to \infty} \expectation{f_0(X^n_{t_0}) \dotsb f_{m-1}(X^n_{t_{m-1}}) \cexpectationlong{X^n_{t_{m-1}}}{f_m(X^n_{t_m})}} \\
&=\lim_{n \to \infty} \expectation{f_0(X^n_{t_0}) \dotsb f_{m-1}(X^n_{t_{m-1}}) T_{n,t_m-t_{m-1}} f_m(X^n_{t_{m-1}})}\\
&=\lim_{n \to \infty} \expectation{f_0(X^n_{t_0}) \dotsb f_{m-1}(X^n_{t_{m-1}}) T_{t_m-t_{m-1}} f_m(X^n_{t_{m-1}})}\\
&=\expectation{f_0(X_{t_0}) \dotsb f_{m-1}(X_{t_{m-1}}) T_{t_m-t_{m-1}} f_m(X_{t_{m-1}})}\\
&=\expectation{f_0(X_{t_0}) \dotsb f_{m-1}(X_{t_{m-1}}) \cexpectationlong{X_{t_{m-1}}}{f_m(X_{t_{m}})}}\\
&=\expectation{f_0(X_{t_0}) \dotsb f_{m-1}(X_{t_{m-1}}) f_m(X_{t_{m}})}\\
\end{align*}
where in the third line we have the fact that $T_{n,t} f$ converges to $T_t f$  for every $f \in C(S)$ to see
\begin{align*}
&\lim_{n \to \infty} \abs{\expectation{f_0(X^n_{t_0}) \dotsb f_{m-1}(X^n_{t_{m-1}}) (T_{n,t_m - t_{m-1}} - T_{t_m-t_{m-1}}) f_m(X^n_{t_{m-1}})}}  \\
&\leq \norm{f_0} \dotsb \norm{f_{m-1}} \lim_{n \to \infty} \norm{(T_{n,t_m - t_{m-1}} - T_{t_m-t_{m-1}}) f_m} = 0
\end{align*}
and in the fourth line applied the induction hypothesis to the continuous functions $f_0, \dotsc, f_{m-2}, f_{m-1} T_{t_m-t_{m-1}} f_m$.

Because $S$ is separable we know that the finite dimensional distributions of $X^n$ converge to those of $X$.  (TODO: prove this; basically we use induction and the standard approximation of indicators by continuous functions to show that $\lim_{n \to \infty} \probability{(X^n_{t_0}, \dotsc, X^n_{t_m}) \in A_1 \times \dotsb \times A_m}$ and then use separability of $S$ to conclude that we have a generating $\pi$-system and monotone classes apply).

Now we need to show tightness of the family $X^n$.  We appeal to the
Aldous criterion so we must show that for every bounded sequence of
optional times $\tau_n$ and deterministic sequence $\delta_n$ such
that $\lim_{n \to \infty} \delta_n = 0$ we have $r(X^n_{\tau_n},
X^n_{\tau_n + \delta_n}) \toprob 0$.  It suffices to show that every subsequence $N^\prime \subset \naturals$ has a futher
subsequence $N^{\prime \prime} \subset N^{\prime}$ such that $r(X^n_{\tau_n},X^n_{\tau_n + \delta_n}) \toprob 0$ along $N^{\prime \prime}$ (for then
we can use Lemma \ref{ConvergenceInProbabilityAlmostSureSubsequence} to find yet another subsequence $N^{\prime \prime \prime} \subset N^{\prime \prime}$ such
that $r(X^n_{\tau_n},X^n_{\tau_n + \delta_n}) \toas 0$ along $N^{\prime \prime \prime}$ and using Lemma \ref{ConvergenceInProbabilityAlmostSureSubsequence} in the
opposite direction we conclude that $r(X^n_{\tau_n},X^n_{\tau_n + \delta_n}) \toprob 0$; alternatively just note that convergence in probability is convergence with respect to the
Ky Fan metric).

Let $\nu_n = \mathcal{L}(X^n_{\tau_n})$ for $n \in \naturals$.  Then
by compactness of $S$ the family $\nu_n$ is automatically tight and therefore relatively compact by Prohorov's Theorem \ref{Prohorov}.  Given any 
subsequence $N^\prime$ then there is a further subsequence $N^{\prime \prime}$ and a probability measure $\nu$ such that $\nu_n \todist \nu$.  In what follows we
implicit working along the subsequence $N^{\prime \prime}$ and mention it no further.

Let $f$ and $g$ be continuous functions on $S$ then since $\delta_n$ converges it is bounded and by Theorem \ref{KurtzSovaSemigroupConvergence} and strong continuity of $T$ we have
\begin{align*}
\lim_{n \to \infty} \norm{T_{n, \delta_n} g - g} &\leq \lim_{n \to \infty} \left[ \norm{T_{n, \delta_n} g - T_{\delta_n} g} + \norm{T_{\delta_n} g - g} \right ] \\ 
&\leq \lim_{n \to \infty} \sup_{0 \leq h \leq \sup_n \delta_n} \norm{T_{n,h} g - T_{h} g} + \lim_{n \to \infty}\norm{T_{\delta_n} g - g} = 0
\end{align*}
and by the Strong Markov Property Theorem \ref{StrongMarkovFellerProcess} and Proposition \ref{TransitionSemigroupAsExpectation}
\begin{align*}
\expectation{f(X^n_{\tau_n}) g(X^n_{\tau_n+\delta_n})} &= \expectation{\cexpectationlong{\mathcal{F}_{\tau_n}}{f(X^n_{\tau_n}) g(X^n_{\tau_n+\delta_n})}} \\
&= \expectation{\sexpectation{f(X^n_{0}) g(X^n_{\delta_n})}{X^n_{\tau_n}}} \\
&= \expectation{\sexpectation{f(X^n_{0}) \cexpectationlong{X^n_0}{g(X^n_{\delta_n})}}{X^n_{\tau_n}}} \\
&= \expectation{\sexpectation{f(X^n_{0}) T_{n,\delta_n}g(X^n_{0})}{X^n_{\tau_n}}} \\
&= \expectation{\cexpectationlong{\mathcal{F}_{\tau_n}}{f(X^n_{\tau_n}) T_{n,\delta_n}g(X^n_{\tau_n})}} \\
&= \expectation{f(X^n_{\tau_n}) T_{n,\delta_n}g(X^n_{\tau_n})} \\
\end{align*}
Putting these two facts together with the expectation rule Lemma \ref{ExpectationRule} and the prior claim we see that
\begin{align*}
\lim_{n \to \infty} \expectation{f(X^n_{\tau_n}) g(X^n_{\tau_n+\delta_n})} &= \lim_{n \to \infty} \expectation{f(X^n_{\tau_n}) T_{n,\delta_n}g(X^n_{\tau_n})}\\
&= \lim_{n \to \infty} \expectation{f(X^n_{\tau_n}) T_{n,\delta_n}g(X^n_{\tau_n}) - f(X^n_{\tau_n}) g(X^n_{\tau_n})} + \lim_{n \to \infty} \expectation{f(X^n_{\tau_n}) g(X^n_{\tau_n}) } \\
&= \lim_{n \to \infty} \int f(s) g(s) \, \nu_n(ds) \\
&= \int f(s) g(s) \, \nu(ds) \\
\end{align*}
Thus by the same argument we used in concluding that fdd's converged we have $\mathcal{L}(X^n_{\tau_n}, X^n_{\tau_n+\delta_n}) \toweak \nu$ where we are regarding $\nu$ as the measure on $S \times S$ concentrated on the diagonal.
By the Continuous Mapping Theorem \ref{ContinuousMappingTheorem} and the fact that $\nu$ is concentrated on the diagonal,
\begin{align*}
r(X^n_{\tau_n}, X^n_{\tau_n+\delta_n}) \todist 0
\end{align*}
and we conclude that $r(X^n_{\tau_n}, X^n_{\tau_n+\delta_n}) \toprob 0$ by Lemma \ref{ConvergeInDistributionToConstant} . Now apply Theorem \ref{SkorohodInfiniteAldousCriterion}.
\end{proof}

\subsection{Approximating Feller Processes by Markov Chains}

Now we turn to the consideration of approximating Feller processes by Markov chains.  Underlying the result is a theorem about the approximation of strongly continuous contraction semigroups by powers of contraction operators.   Before stating the theorem we call out an estimate that will be required in the proof.

\begin{lem}\label{KurtzDiscreteTimeSemigroupConvergenceEstimate}Let $B$ be a contraction operator on a Banach space $X$ then for all $v \in X$ and $n \in \integers_+$ we have
\begin{align*}
\norm{B^n v - e^{n(B - \IdentityMatrix)} v} &\leq \sqrt{n} \norm{Bv - v}
\end{align*}
\end{lem}
\begin{proof}
First let $n \geq m$ and note that
\begin{align*}
\norm{B^n v - B^m v} &= \norm{\sum_{k=0}^{n-m-1} B^{m+k} (B v - v)} \leq \sum_{k=0}^{n-m-1} \norm {B^{m+k}} \norm{B v - v} \leq (n-m) \norm{Bv-v}
\end{align*}
Expanding in power series and using this estimate and Cauchy Schwartz we get
\begin{align*}
\norm{B^n v - e^{n(B - \IdentityMatrix)} v} &= e^{-n} \norm{e^n B^n v - e^{nB} v} 
= e^{-n} \norm{\sum_{k=0}^\infty \frac{n^k}{k!} (B^n v - B^k v)} \\
&= e^{-n} \sum_{k=0}^\infty \frac{n^k}{k!} \abs{n-k} \norm{B v - v} \\
&\leq e^{-n} \norm{B v - v} \left( \frac{n^k}{k!} (n-k)^2\right)^{1/2} \left( \frac{n^k}{k!} \right)^{1/2} \\
&=\norm{B v - v} \left( e^{-n} \frac{n^k}{k!} (n-k)^2\right)^{1/2} = \sqrt{n} \norm{Bv-v}\\
\end{align*}
where in the last line we have used the formula for the variance of a Poisson random variable of rate $n$ (Proposition \ref{MomentsPoissonDistribution})
\end{proof}

\begin{thm}\label{KurtzDiscreteTimeSemigroupConvergence}Let $X$ and $X_1, X_2, \dotsc$ be Banach spaces, let $T_{n}$ be a contraction operator on $X_n$, $\epsilon_n \geq 0 $ such that $\lim_{n\to \infty} \epsilon_n = 0$ and let $T_t$ be a strongly continuous contraction semigroup on $X$ with generator $A$.  Let $\pi_n : X \to X_n$ be bounded linear operators and assume that $\sup_{n} \norm{\pi_n} < \infty$.  Let $D$ be a
core for $A$ and define $A_n = \epsilon_n^{-1}(T_n-\IdentityMatrix)$ then the following are equivalent
\begin{itemize}
\item[(i)]For every $v \in X$ and every $t \geq 0$,  $\lim_{n \to \infty} \sup_{0 \leq s \leq t} \norm{T^{\floor{t/\epsilon_n}}_{n} \pi_n v - \pi_n T_s v} = 0$.
\item[(ii)]For every $v \in X$ and every $t \geq 0$,  $\lim_{n \to \infty} \norm{T^{\floor{t/\epsilon_n}}_{n} \pi_n v - \pi_n T_t v} = 0$.
\item[(iii)] For every $v \in D$ there exists $v_n \in \domain{A_n}$ such that $\lim_{n \to \infty} \norm{v_n - \pi_n v} = 0$ and $\lim_{n \to \infty} \norm{A_n v_n - \pi_n A v} = 0$.
\end{itemize}
\end{thm}
\begin{proof}
Before we proceed first note that $e^{t A_n}$ is in fact a contraction operator for all $t \geq 0$,
\begin{align*}
\norm{e^{t A_n}} &= \norm{e^{t \epsilon_n^{-1}(T_n - \IdentityMatrix)}} = e^{-t \epsilon_n^{-1}} \norm{e^{t \epsilon_n^{-1} T_n}} \leq e^{-t \epsilon_n^{-1}} e^{t \epsilon_n^{-1} \norm{T_n}} \leq 1
\end{align*}
Clearly (i) implies (ii).  To see that (ii) implies (iii) let $\lambda > 0$ be given, $v \in \domain{A}$ and set $w = (\lambda - A) v$ (so that $v = \int_0^\infty e^{-\lambda t} T_t w \, dt$).  For $n \in \naturals$ define
\begin{align*}
v_n &= \epsilon_n \sum_{k=0}^\infty e^{-\lambda k \epsilon_n} T_n^k \pi_n w \in X_n
\end{align*}
TODO: We need to show that $\norm{v_n}$ is bounded (or maybe we use Tonneli to push the limit inside the sum and don't worry about it); there may be something to this.
From (ii) and Dominated Convergence we have
\begin{align*}
\lim_{n \to \infty} \norm{v_n - \pi_n v} &= \lim_{n \to \infty} \norm{v_n - \int_0^\infty e^{-\lambda t} \pi_n T_t w \, dt} \\
&\leq  \lim_{n \to \infty} \norm{v_n - \int_0^\infty e^{-\lambda t} T^{\floor{t/\epsilon_n}}_{n} \pi_n v \, dt} \\
&= \lim_{n \to \infty} \norm{v_n - \sum_{k=0}^\infty \int_{k\epsilon_n}^{(k+1)\epsilon_n} e^{-\lambda t} T^{k}_{n} \pi_n v \, dt} \\
&= \lim_{n \to \infty} \norm{v_n - () \sum_{k=0}^\infty e^{-\lambda k \epsilon_n} T^{k}_{n} \pi_n v }\\
&= \lim_{n \to \infty} (1 - \frac{1-e^{-\lambda \epsilon_n}}{\lambda})\norm{v_n}  = 0\\
\end{align*}
We also get
\begin{align*}
(\lambda - A_n) v_n &= (\lambda - \epsilon_n^{-1}(T_n - \IdentityMatrix)) \epsilon_n \sum_{k=0}^\infty e^{-\lambda k \epsilon_n} T_n^k \pi_n w \\
&=\lambda \epsilon_n \sum_{k=0}^\infty e^{-\lambda k \epsilon_n} T_n^k \pi_n w 
- \sum_{k=0}^\infty e^{-\lambda k \epsilon_n} T_n^{k+1} \pi_n w +  
\sum_{k=0}^\infty e^{-\lambda k \epsilon_n} T_n^{k} \pi_n w \\
&= \lambda \epsilon_n \pi_n w 
+ e^{-\lambda \epsilon_n} \lambda \epsilon_n \sum_{k=0}^\infty e^{-\lambda k \epsilon_n} T_n^{k+1} \pi_n w 
- \sum_{k=0}^\infty e^{-\lambda k \epsilon_n} T_n^{k+1} \pi_n w +  \\
&\pi_n w  +
e^{-\lambda \epsilon_n}  \sum_{k=0}^\infty e^{-\lambda k \epsilon_n} T_n^{k+1} \pi_n w \\
&= \lambda \epsilon_n \pi_n w 
+ \pi_n w  + 
(e^{-\lambda \epsilon_n} \lambda \epsilon_n -1 + e^{-\lambda \epsilon_n} ) \sum_{k=0}^\infty e^{-\lambda k \epsilon_n} T_n^{k+1} \pi_n w \\
\end{align*}
Now take a difference, a limit and apply Tonelli's Theorem 
\begin{align*}
\lim_{n \to \infty} \norm{(\lambda - A_n) v_n - \pi_n w} &\leq 
\lim_{n \to \infty} \norm{\epsilon_n \lambda \pi_n w} + \lim_{n \to \infty} \sum_{k=0}^\infty \abs{e^{-\lambda \epsilon_n} \lambda \epsilon_n -1 + e^{-\lambda \epsilon_n} }
e^{-\lambda k \epsilon_n} \norm{T_n^{k+1} \pi_n w} \\
&\leq \lambda \sup_n \norm{\pi_n} \norm{w} \lim_{n \to \infty} \epsilon_n + \sup_n \norm{\pi_n} \norm{w} \sum_{k=0}^\infty \lim_{n \to \infty} \abs{e^{-\lambda \epsilon_n} \lambda \epsilon_n -1 + e^{-\lambda \epsilon_n} }
e^{-\lambda k \epsilon_n}  \\
&= 0
\end{align*}
Therefore 
\begin{align*}
\lim_{n \to \infty} \norm{A_n v_n - \pi_n A v} &\leq \lim_{n \to \infty} \left( \norm{(\lambda - A_n) v_n - \pi_n (\lambda - A) v} + \lambda \norm{v_n - \pi_n v} \right) = 0
\end{align*}

To see that (iii) implies (i), let $v \in X$ and pick $v_n \in X_n$ such that $\lim_{n \to \infty} \norm{v_n - \pi_n v} = 0$ and $\lim_{n \to \infty} \norm{A_n v_n - \pi_n A v} = 0$.  For
$n \in \naturals$ adding and subtracting terms and using the triangle equality we get
\begin{align*}
\norm{T^{\floor{t/\epsilon_n}}_n \pi_n v - \pi_n T_t v} 
&\leq \norm{T^{\floor{t/\epsilon_n}}_n \pi_n v - T^{\floor{t/\epsilon_n}}_n v_n}  + 
\norm{T^{\floor{t/\epsilon_n}}_n v_n - e^{\epsilon_n \floor{t/\epsilon_n} A_n} v_n} + \\
&\norm{e^{\epsilon_n \floor{t/\epsilon_n} A_n} v_n - e^{\epsilon_n \floor{t/\epsilon_n} A_n} \pi_n v} + 
\norm{e^{\epsilon_n \floor{t/\epsilon_n} A_n} \pi_n v - e^{t A_n} \pi_n v} + \\
&\norm{e^{t A_n} \pi_n v - \pi_n T_t v}
\end{align*}
We consider each of the terms on the right hand side in sequence.  For the first term simply note that since $T_n$ is a contraction operator
\begin{align*}
\limsup_{n \to \infty} \sup_{0 \leq s \leq t} \norm{T^{\floor{s/\epsilon_n}}_n \pi_n v - T^{\floor{t/\epsilon_n}}_n v_n} &\leq \lim_{n \to \infty} \norm{\pi_n v - v} = 0
\end{align*}
For the second term we use Lemma \ref{KurtzDiscreteTimeSemigroupConvergenceEstimate} to see 
\begin{align*}
\limsup_{n \to \infty} \sup_{0 \leq s \leq t} \norm{T^{\floor{s/\epsilon_n}}_n v_n - e^{\epsilon_n \floor{s/\epsilon_n} A_n} v_n} 
&=\limsup_{n \to \infty} \sup_{0 \leq s \leq t} \norm{T^{\floor{s/\epsilon_n}}_n v_n - e^{\floor{s/\epsilon_n} (T_n- \IdentityMatrix)} v_n}  \\
&\leq \limsup_{n \to \infty} \sup_{0 \leq s \leq t} \sqrt{\floor{s/\epsilon_n}} \norm{(T_n- \IdentityMatrix) v_n} \\
&= \limsup_{n \to \infty} \sqrt{\floor{t/\epsilon_n}} \epsilon_n \norm{A_n v_n} \\
&\leq \limsup_{n \to \infty} \sqrt{\floor{t/\epsilon_n}} \epsilon_n (\sup_n \norm{\pi_n} \norm{A v}  + \norm{A_n v_n - \pi_n A v}) \\
&= 0
\end{align*}
For the third term using the fact that $e^{tA_n}$ is a contraction operator 
\begin{align*}
\limsup_{n \to \infty} \sup_{0 \leq s \leq t}\norm{e^{\epsilon_n \floor{s/\epsilon_n} A_n} v_n - e^{\epsilon_n \floor{s/\epsilon_n} A_n} \pi_n v}
&\leq \limsup_{n \to \infty} \norm{v_n -\pi_n v} =0
\end{align*}
For the fourth term we use the Fundamental Theorem of Calculus, the fact that $e^{tA_n}$ is a contraction operator and the fact that $\norm{\pi_n A v - A_n \pi_n v} \to 0$ to see
\begin{align*}
\limsup_{n \to \infty} \sup_{0 \leq s \leq t} \norm{e^{\epsilon_n \floor{s/\epsilon_n} A_n} \pi_n v - e^{s A_n} \pi_n v} 
&=\limsup_{n \to \infty} \sup_{0 \leq s \leq t} \norm{\int^s_{\epsilon_n \floor{s/\epsilon_n}} \frac{d}{du} e^{uA_n} \pi_n v \, du} \\
&=\limsup_{n \to \infty} \sup_{0 \leq s \leq t} \norm{\int^s_{\epsilon_n \floor{s/\epsilon_n}} e^{uA_n} A_n \pi_n v \, du} \\
&\leq \limsup_{n \to \infty} \epsilon_n \norm{e^{uA_n}} \norm{ A_n \pi_n v }\\
&\leq \limsup_{n \to \infty} \epsilon_n (\sup_n \norm{\pi_n} \norm{Av} +  \norm{\pi_n A v - A_n \pi_n v }) = 0\\
\end{align*}
Lastly for the fifth term we see that 
\begin{align*}
\limsup_{n \to \infty} \sup_{0 \leq s \leq t} \norm{e^{s A_n} \pi_n v - \pi_n T_s v} = 0
\end{align*}
from Theorem \ref{KurtzSovaSemigroupConvergence} applied to the semigroups $e^{t A_n}$ and $T_t$.
\end{proof}

TODO: Trotter product formula and Chernoff product formula.  The Trotter product formula is inspired by the Lie product formula for matrix groups.

Possible exercise: According to Bratelli (who attributes to Hille) the Trotter product formula implies the Weierstrass approximation theorem (I think Goldstein has this as well).  Let $X$ be Banach space of bounded uniformly continuous functions and let $T_tf(x) = f(x+t)$ be the translation semigroup with generator $A=\frac{d}{dx}$.  Then let $A_n = (T_{1/n} -1)/(1/n)$ and the Trotter product formula then shows 
\begin{align*}
f(t) &= \lim_{n \to \infty} \sum_{m=1}^\infty t^m (A^m_n f)(0)/m!
\end{align*} 
where the convergence is uniform on compacts.  Apparently this implies the result.  Also chase down and understand how Goldstein uses the Chernoff product formula to show the central limit theorem.

\chapter{Stochastic Approximation}

This chapter covers some of the basic results in the theory of stochastic approximation and in doing so provides some applications of discrete time martingale theory and weak convergence theory to optimization problems.  The statement of the stochastic approximation problem that one often encounters is so abstract and general that it can be difficult to understand how it could be relevant to any particular problem.  Indeed it is common to see stochastic approximation defined as the study of discrete time stochastic processes of the form 
\begin{align*}
\theta_{n+1} &= \theta_n + \epsilon_n Y_n
\end{align*} 
where $Y_n$ is a random vector.

To motivate the form of the problem statement, let us tie this into the problem of optimization specifically gradient descent.  Given a function $f$ we have a globally convergent algorithm for minimization given by $x_{n+1} = x_n - \alpha_n \nabla f(x_n)$ where $\alpha_n$ is a sequence of real numbers that satisfies Armijo conditions.  Now suppose that we don't have the ability to measure $-\nabla f(x_n)$ exactly but that we have some noise corrupted  version thereof.  If we call the observed approximate gradient $Y_n$, then the gradient descent algorithm has the form of a stochastic approximation problem and we can ask whether we still have convergence in a appropriate stochastic sense (e.g. almost sure).  In line with this specific case, we often think of the process $Y_n$ as being a sequence of observations and though it doesn't have any real mathematical meaning, we shall use the terminology in what follows.

As we've mentioned in our discussion of optimization, in practice constrained optimization is at least as important as unconstrained optimization and therefore we should look for how to incorporate constraints into stochastic approximation.  The way we shall do this at this point is to assume that the sequence $\theta_n$ is constrained to lie in some closed set $F$ and to maintain the constraint at each iteration by a brute force projection (say in $L^2$ norm) onto the set $F$.   Thus in the constrained case we are considering a stochastic process 
\begin{align*}
\theta_{n+1} &= \Pi_F \left[ \theta_n + \epsilon_n Y_n\right ]
\end{align*} 
where $Y_n$ is a random vector and $\Pi_F$ represents projection onto $F$.  It is common to define the projection correction term $Z_n =  \epsilon_n^{-1} \lbrace \Pi_F \left[ \theta_n + \epsilon_n Y_n\right ] -   \theta_n - \epsilon_n Y_n \rbrace$ so that we may write 
\begin{align*}
\theta_{n+1} &= \theta_n + \epsilon_n Y_n + \epsilon_n Z_n
\end{align*} 

In order to discuss the hypotheses that one might need to make on the stochastic process $Y_n$, it is convenient to assume a structural form for $Y_n$.  Let $\mathcal{F}_n = \sigma(\theta_0, Y_j ; j<n)$ be a filtration $\mathcal{F}$.  For our first results we shall assume that there exists functions $g_n$, an $\mathcal{F}$-martingale difference sequence $\delta M_n$ and a stochastic process $\beta_n$ such that $Y_n = g_n(\theta_n) + \delta M_n + \beta_n$.  The reader should think of these terms in the following way.  The term $g_n(\theta_n)$ represents the mean/true value of the process (e.g. the value of the gradient in the steepest descent case), the term $\delta M_n$ represents a noise term and $\beta_n$ represents a bias term in the observation.  The reason why the bias term $\beta_n$ is called out as being different from $g_n(\theta_n)$ is that we shall be assuming that it becomes asymptotically small.

One of the key techniques in proving theorems in stochastic approximation is the ODE method.  The idea is that one can view the process $\theta_n$ as a discretization of an ordinary differential equation that is described by the conditional means of $Y_n$.  

In many of the proofs we will considering the continuous time limits of discrete time processes.  To do this we be making interpolations of the discrete time processes and want to have recourse to compactness results that will give conditions under which limits exist.  A natural tool for this would be to use the Arzela-Ascoli Theorem for continuous functions or the Skorohod topology versions of that for cadlag functions.  As it turns out neither of these is the exact fit for what we do since we'll be considering cadlag processes that converge to continuous functions and we want uniform convergence on compact sets.  So what we want is a slight extension of Arzela-Ascoli Theorem.

The following is a version of the Arzela Ascoli Theorem \ref{ExtendedArzelaAscoliTheorem} that gives a
sufficent criteria for a sequence of possibly discontinuous functions
to have a continuous limit.  The referenced version of the Arzela Ascoli Theorem
doesn't apply in a non-trivial way since the criterion for
equicontinuity $\lim_{\delta \to 0} \sup_{f \in A} m(T, f, \delta) = 0$ implies that every $f$ is continuous.  However, note that if $f_n$ is a sequence of
functions in $C([0,\infty); \reals^d)$ then $\lbrace f_n \rbrace$ is equicontinuous if
and only if $\lim_{\delta \to 0} \limsup_{n \to \infty} m(T, f_n,
\delta) = 0$.  This criterion does not imply that each $f_n$ is
continuous and turns out to be a useful extension of equicontinuity
for sequences of non-continuous functions.
\begin{thm}[Extended Arzela-Ascoli
  Theorem]\label{ExtendedArzelaAscoliTheorem}Let $f_n : [0,\infty) \to
 \reals^d$ be measurable functions such that  
\begin{itemize}
\item[(i)]$\sup_{n} \abs{f_n(0)} < \infty$
\item[(ii)] $\lim_{\delta \to 0} \limsup_{n \to \infty} m(T, f_n,
  \delta) = 0$ for all $T > 0$.
\end{itemize}
then there exists $f \in C([0,\infty), \reals^d)$ such that $f_n$
converges to $f$ uniformly on compact sets.
\end{thm}
\begin{proof}
First note that if (i) and (ii) hold for the sequence $f_n$ then the conditions also hold for any subsequence of $f_n$.  
Suppose that $f_n$ satisfy (i) and (ii) and let $T > 0$.  By (ii) there exists a $\delta > 0$ such that 
\begin{align*}
\limsup_{n \to \infty} m(T, f_n,  \delta)  < 1
\end{align*}
hence there exists an $N \in \naturals$ such that $m(T, f_n,  \delta)  < 1$ for all $n \geq N$.  Now pick $m \in \naturals$ such 
that $m \delta < T \leq (m+1)\delta$ and just as in Theorem \ref{ArzelaAscoliTheorem}  by considering  the grid $0, \delta, 2\delta, \dotsc,
m\delta, T$ we can write the telescoping sum
\begin{align*}
f_n(T) - f_n(0) = f_n(T) - f_n(m\delta) + \sum_{k=1}^m f_n(k \delta) - f_n((k-1)\delta)
\end{align*}
and use the triangle inequality to conclude that $\abs{f_n(T)}
\leq \abs{f_n(0)} + m+1$ for every $n \geq N$.  Coupled with (i) this shows that
$\sup_{n \geq N} \abs{f_n(T)} < \infty$.  By local compactness of $\reals$ we see that  $f_n(T)$ converges 
along a subsequence of $\lbrace n, n+1, \dotsc \rbrace$.  Using the observation that $f_n$ along this subsequence still satisfies
(i) and (ii) we see that we can enumerate $T \in \rationals_+$ and use induction and a diagonal subsequence argument to get a 
single subsequence of $f_n$ that converges for all $T \in \rationals_+$.  Define $f(T)$ for $T \in \rationals_+$ as the limit of this subsequence of $f_n$.  

The proof that $f \in C([0,\infty); \reals^d)$ and that $f_n$ converges to $f$ uniformly on compact sets is almost exactly the same as the proof in Theorem
\ref{ArzelaAscoliTheorem}.  The only difference is that the condition (ii) applied the sequence $f_n$ only constrains terms $\abs{f_n(s) - f_n(t)}$ for $n$ sufficiently large.  Examining the proof of Theorem \ref{ArzelaAscoliTheorem} one will see that this is all that is required.
\end{proof}

We also need a partial converse, namely that a convergence sequence of functions is equicontinuous in the extended sense.
\begin{prop}\label{PartialConverseExtendedArzelaAscoli}Let $f_n : \reals \to \reals^d$ be a sequence of measurable functions that converges to a continuous function with convergence uniform on compact sets, then $f_n$ is equicontinuous in the extended sense.
\end{prop}
\begin{proof}
Let $f$ be the limit of $f_n$.  Pick an $N$ such that $\norm{f_n(0) - f(0)} < 1$ for all $n \geq N$ then it follows that $\norm{f_n(0)} \leq \norm{f_0(0)} \vee \dotsb \vee \norm{f_{N-1}(0)} \vee \norm{f_N(0)} + 1$ for all $n \in \naturals$.  Now let $T > 0$ and $\epsilon > 0$ be given and use the fact that $f$ is uniformly continuous on $[-T,T]$ to pick 
a $\delta > 0$ such that $m(T, f, \delta) < \epsilon/3$.  Now by uniform convergence of $f_n$ to $f$ on $[-T,T]$ we pick $N > 0$ such that $\sup_{-T \leq t \leq T} \norm{f_n(t) - f(t)} < \epsilon/3$ for $n \geq N$.  Thus for $n \geq N$,
\begin{align*}
m(T,f_n,\delta) &= \sup_{\substack{
\abs{s -t} < \delta \\
-T \leq s,t \leq T}} \norm{f_n(s) - f_n(t)} \\
&\leq \sup_{\substack{
\abs{s -t} < \delta \\
-T \leq s,t \leq T}} \norm{f_n(s) - f(s)} + \norm{f(s) - f(t)} + \norm{f_n(t) - f(t)} < \epsilon
\end{align*}
and it follows that $\limsup_{n \to \infty} m(T, f_n, \delta) < \epsilon$ and thus $\lim_{\delta \to 0} \limsup_{n \to \infty} m(T, f_n, \delta) = 0$.
\end{proof}

We shall now assume that we are in the situation of having a constraint set $F$ defined by continuously differentiable function $c_i(x)$ which satisfy the LICQ.  

TODO:  Use the KKT  conditions applied to $\min_{x \in F} \norm{x - (\theta_n + \epsilon_n Y_n)}^2$ to show that $Z_n$ is in the normal cone 


\begin{thm}Suppose we are given a process $Y_n$, a constraint set $F$, a random variable $\theta_0$ and a deterministic sequence $\epsilon_n$.  Define the process 
\begin{align*}
\theta_{n+1} &= \Pi_F \left[\theta_{n} + \epsilon_n Y_n \right]
\end{align*}
and suppose that there are measurable functions $g_n(\theta)$ such that if we write $\cexpectationlong{\theta_0, Y_i; 0 \leq i \leq n-1}{Y_n} = g_n(\theta_n) + \beta_n$ such that
\begin{itemize}
\item[(i)] $\sup_n \expectation{Y_n^2} < \infty$
\item[(ii)] $\epsilon_n$ for $n \in \integers$ is a sequence with $\epsilon_n = 0$ for $n < 0$, $\epsilon_n \geq 0$ for $n \geq 0$, $\lim_{n \to \infty} \epsilon_n = 0$,  $\sum_{n=0}^\infty \epsilon_n = \infty$ and $\sum_{n=0}^\infty \epsilon^2_n < \infty$.  
\item[(iii)] Suppose the $g_n(\theta)$ are uniformly continuous in $n$ and there is a continuous function $\overline{g}(\theta)$ such that for each $\theta \in F$ we have
\begin{align*}
\lim_{n \to \infty} \abs{\sum_{i=n}^{m(t_n+t)} \epsilon_i \lbrace g_i(\theta) - \overline{g}(\theta) \rbrace }&= 0
\end{align*}
\item[(iv)] $\beta_n \toas 0$
\end{itemize}
Then there is a set $A$ of probability zero such that for $\omega \notin A$ the set of functions $\lbrace \theta^n(\omega, \cdot), Z^n(\omega, \cdot); n < \infty \rbrace$ is equicontinuous.  If $(\theta(\omega, \cdot), Z(\omega, \cdot))$ is the limit of some convergent subsequence then the pair satisfies the projected ODE 
\begin{align*}
\dot{\theta} &= \overline{g}(\theta) + z \text{, $z \in \mathcal{N}(\theta)$}
\end{align*}
and $\theta_n(\omega)$ converges to a limit set of the projected ODE in $F$.  
\end{thm}
\begin{proof}
Let $\mathcal{F}_n = \sigma(\theta_0, Y_i; 0 \leq i \leq n)$ be the filtration defined by $Y_n$ and the initial condition $\theta_0$.
We write
\begin{align*}
\theta_{n+1} &= \Pi_F \left [ \theta_n + \epsilon_n Y_n \right ] \\
&=\theta_n + \epsilon_n Y_n + \epsilon_n Z_n \\
&=\theta_n + \epsilon_n \cexpectationlong{\mathcal{F}_{n-1}}{Y_n} + \epsilon_n \left( Y_n -  \cexpectationlong{\mathcal{F}_{n-1}}{Y_n} \right) + \epsilon_n Z_n \\
&=\theta_n + \epsilon_n g_n(\theta_n) + \epsilon_n \beta_n + \epsilon_n \delta M_n + \epsilon_n Z_n \\
\end{align*}
where we have defined $\delta M_n = Y_n -  \cexpectationlong{\mathcal{F}_{n-1}}{Y_n}$.  Note that $\epsilon_n \delta M_n$ is an
$\mathcal{F}$-martingale difference sequence and therefore by Proposition \ref{MartingaleDifferenceSequence} the process $M_n = \sum_{j=0}^n \epsilon_j \delta M_j$ is an 
$\mathcal{F}$-martingale.  Furthermore by Jensen's Inequality for conditional expectations (Theorem \ref{JensenConditionalExpectation}) and the fact that $Y_n$ is $L^2$-bounded
we also know that $\delta M_n$ is an $L^2$ martingale difference sequence hence by Proposition \ref{SquareIntegrableMartingaleDifferenceWhiteNoise} we know that $\expectation{\delta M_n  \delta M_m} = 0$.  For every fixed $m \in \integers_+$ we know that the process $(M_{n+m} - M_m)^2$ is a submartingale with respect to the shifted
filtration $\tilde{\mathcal{F}}_n = \mathcal{F}_{n+m}$ and if we apply Doob's Maximal Inequality (Lemma \ref{DoobMaximalInequalityDiscrete}) we get for every $\lambda > 0$ and $m < n$,
\begin{align*}
\probability{\sup_{m \leq j \leq n} \abs{M_j - M_m} \geq \lambda} &= \probability{\sup_{m \leq j \leq n} (M_j - M_m)^2 \geq \lambda^2} \\
&\leq \lambda^{-2} \expectation{(M_n - M_m)^2} \\
&=\lambda^{-2}\sum_{i=m+1}^n \sum_{j=m+1}^n  \epsilon_i \epsilon_j \expectation{\delta M_i \delta M_j} \\
&=\lambda^{-2}\sum_{j=m+1}^n \epsilon_j^2  \expectation{\delta M_j^2} \\
&\leq 2 \lambda^{-2} \sup_{n} \expectation{Y_n^2} \sum_{j=m+1}^\infty \epsilon_j^2  \\
\end{align*}
By continuity of measure, we can let $n \to \infty$ and then $m \to \infty$ and use the hypothesis that $\sum_{n=0}^\infty \epsilon^2_n < \infty$ to conclude that for every $\lambda >0$ we have 
\begin{align}\label{SASimpleMartingaleNoiseConvergence}
\lim_{m \to \infty} \probability{\sup_{m \leq j} \abs{M_j - M_m} \geq \lambda} &= 0
\end{align}
TODO: Could we have just appealed to an off the shelf Martingale Convergence theorem here; not sure this is a interesting kind of convergence because are looking at a subsequence that starts at a point that goes to infinity????  We have just proven that $\sup_{m \leq j} \abs{M_j - M_m} \toprob 0$.

Now we move to the interpolated process.  Recall that we define $t_0 = 0$ and $t_n = \sum_{i=0}^{n-1} \epsilon_i$ for $n \in \naturals$.  We define $m(t) = n$ for $t_n \leq t < t_{n-1}$.  Using $m(t)$ we define the interpolated processes for $t \geq 0$,
\begin{align*}
M^n(t) &= \sum_{i=n}^{m(t_n + t)-1} \epsilon_i \delta M_i &  
B^n(t) &=\sum_{i=n}^{m(t_n + t)-1} \epsilon_i \beta_i &
Z^n(t) &= \sum_{i=n}^{m(t_n + t)-1} \epsilon_i Z_i
\end{align*}
and for $t < 0$,
\begin{align*}
M^n(t) &= - \sum_{i=m(t_n + t)}^{n-1} \epsilon_i \delta M_i &  
B^n(t) &= - \sum_{i=m(t_n + t)}^{n-1} \epsilon_i \beta_i &  
Z^n(t) &= - \sum_{i=m(t_n + t)}^{n-1} \epsilon_i Z_i &  
\end{align*}
and note that $M^n(t) = M^0(t_n+t) - M^0(t_n)$ and similarly with $B^n$ and $Z^n$.  Moreover
$M^0(t) = M_{m(t) -1}$.
Furthermore we define
\begin{align*}
\overline{G}^n(t) &= \sum_{i=n}^{m(t_n + t)-1} \epsilon_i \overline{g}(\theta_n) &  
\tilde{G}^n(t) &=\sum_{i=n}^{m(t_n + t)-1} \epsilon_i \left( g_n(\theta_n) - \overline{g}(\theta) \right )&
\end{align*}
so that we have 
\begin{align*}
\theta^n(t) &= \theta_n + \overline{G}^n(t)  + \tilde{G}^n(t) + M^n(t) + Z^n(t)  + B^n(t) 
\end{align*}

\begin{clm}Almost surely for all $T > 0$, $\lim_{n \to \infty} \sup_{-T \leq t \leq T} M^n(t) = 0$.
\end{clm}

Let $T > 0$ be given.  By the definition of $M^n$ and the triangle inequality we get for every $n \in \naturals$ and $m < m(t_n -T)$ we have
\begin{align*}
\sup_{-T \leq t \leq T} \abs{M^n (t)}  
&=\sup_{-T \leq t \leq T} \abs{M^0 (t_n + t) - M^0(t_n)}  \\
&=2 \sup_{-T \leq t \leq T} \abs{M^0 (t_n + t) - M^0(t_n -T )} \\  
&\leq 2 \sup_{m(t_n-T) - 1 \leq j}  \abs{M_j - M_{m(t_n -T )-1}}\\
&\leq 4 \sup_{m \leq j}  \abs{M_j - M_{m}}\\
\end{align*}
If $\lim_{n \to \infty} \sup_{-T \leq t \leq T} M^n (t) != 0$ then there is a $\lambda>0$ and a subsequence $n_j$such that $\sup_{-T \leq t \leq T} \abs{M^{n_j} (t)} \geq \lambda$ for all $j \in \naturals$.  Since we know that $\lim_{j \to \infty} m(t_{n_j} -T) = \infty$, we know $\sup_{m \leq j}  \abs{M_j - M_{m}} \geq \lambda/4$ for all $m$ and therefore the claim follows from Equation \eqref{SASimpleMartingaleNoiseConvergence}.  


\begin{clm}Almost surely for all $T > 0$, $\lim_{n \to \infty} \sup_{-T \leq t \leq T} B^n(t) = 0$.
\end{clm}

We actually want this under a couple of different hypotheses: $\beta_n \toas 0$ and $\sum_{n=0}^\infty \epsilon_n \abs{\beta_n} < \infty$ a.s.  In the latter case 
\begin{align*}
\sup_{-T \leq t \leq T} \abs{B^n(t)} &\leq \sum_{i=m(t_n -T)}^{m(t_n+T) -1} \epsilon_i \abs{\beta_i} \leq \sum_{i=m(t_n -T)}^{\infty} \epsilon_i \abs{\beta_i} 
\end{align*}
so the result follows from the fact that $\lim_{n \to \infty} m(t_n -T) = \infty$.  TODO: What about the former case?

\begin{clm}$\theta_{n+1} - \theta_n \toas 0$.
\end{clm}
Using the Markov Inequality, $\sup_n \expectation{Y_n^2} < \infty$ and $\sum_{n=0}^\infty \epsilon_n^2 <\infty$ we get for any $\lambda > 0$,
\begin{align*}
\sum_{n=0}^\infty \probability{\epsilon_n \abs{Y_n} \geq \lambda} &\leq 
\sum_{n=0}^\infty \frac{\epsilon_n^2 \expectation{Y_n^2}}{\lambda^2} \\
&\leq \frac{\sup_n \expectation{Y_n^2}}{\lambda^2} \sum_{n=0}^\infty \epsilon_n^2  < \infty
\end{align*}
and therefore the Borel Cantelli Theorem \ref{BorelCantelli} implies that $\probability{\epsilon_n \abs{Y_n} \geq \lambda \text{ i.o.}} = 0$ and therefore by Lemma \ref{ConvergenceAlmostSureByInfinitelyOften} we conclude that $\epsilon_n \abs{Y_n} \toas 0$.  Thus the definition of $\Pi_F$ and the fact that $\theta_n \in F$, we see that (we are using the argument that $\Pi_F(\theta_n + \epsilon_n Y_n) = \argmin_{x \in F} \abs{\theta_n + \epsilon_n Y_n - x}$ hence since $\theta_n \in F$,
\begin{align*}
\abs{\theta_n + \epsilon_n Y_n - \Pi_F(\theta_n + \epsilon_n Y_n)} &\leq \abs{\theta_n + \epsilon_n Y_n - \theta_n} = \epsilon_n \abs{Y_n}
\end{align*}
is this always true or does it require some assumption like prox-regularity????)
\begin{align*}
\lim_{n \to \infty} \abs{\theta_{n+1} - \theta_n } &= \lim_{n \to \infty} \abs{ \Pi_F \left[ \theta_n + \epsilon_n Y_n \right] - \theta_n } \\
&\leq  \lim_{n \to \infty} \lbrace \abs{ \Pi_F \left[ \theta_n + \epsilon_n Y_n \right] - \theta_n - \epsilon_n Y_n} + \epsilon_n \abs{Y_n} \rbrace \\
&\leq 2 \lim_{n \to \infty} \epsilon_n \abs{ Y_n}  = 0
\end{align*}
almost surely.


By prior claims and Proposition \ref{PartialConverseExtendedArzelaAscoli} we know that almost surely, $M^n(t)$ and $B^n(t)$ are each equicontinuous in the extended sense. 

\begin{clm}$Z^n$ is almost surely equicontinuous in the extended sense.
\end{clm}
Here is the hyperrectangle case.  We work pathwise so lets assume that $\omega \in \Omega$ is fixed.  

\begin{clm}If $Z^n(\omega)$ is not equicontinuous in the extended sense then there is an $\epsilon > 0$, a sequence $n_k \in \naturals$ with $\lim_{k \to \infty} n_k = \infty$ and a sequence $\delta_k > 0$ with $\lim_{k \to \infty} \delta_k = 0$ such that $\abs{Z^{n_k}(\omega, \delta_k)} \geq \epsilon$.
\end{clm}
 If $Z^n$ is not equicontinuous in the extended sense then there is a $T > 0$ such that $\lim_{\delta \to 0} \limsup_{n \to \infty} m(T, Z^n, \delta) > 0$ i.e. an $\epsilon > 0$, a sequence $\delta_k$ with $\delta_k \to 0$, a sequence $n_k$ with $n_k \to \infty$ and $s_k, u_k$ with $-T \leq s_k < u_k \leq T$ and $u_k - s_k < \delta_k$ such that $\abs{Z^{n_k}(u_k) - Z^{n_k}(s_k)} \geq \epsilon$.  Recalling $Z^n(t) = Z^0(t_n+t) - Z^0(t_n)$  we see that $\abs{Z^{n_k}(u_k) - Z^{n_k}(s_k)} \geq \epsilon$ is equivalent to $\abs{Z^{0}(t_{n_k} + u_k) - Z^0(t_{n_k} + s_k)} \geq \epsilon$ and therefore if we define $m_k$ such that .  By redefining the sequence $n_k$ to be $\tilde{n}_k = m(t_{n_k} + s_k)$, $\tilde{s}_k = s_k  + t_{n_k} - t_{\tilde{n}_k}$  and $\tilde{u}_k = u_k + t_{n_k} - t_{\tilde{n}_k}$ we get $\abs{Z^{\tilde{n}_k}(\tilde{u}_k) - Z^{\tilde{n}_k}(\tilde{s}_k)} \geq \epsilon$ where by definition $t_{\tilde{n}_k} \leq s_k < t_{\tilde{n}_k + 1} = t_{\tilde{n}_k } + \epsilon_{\tilde{n}_k+1}$ and thus $0 \leq \tilde{s}_k < \epsilon_{\tilde{n}_k+1}$   but still 
\begin{align*}
\lim_{k \to \infty} \tilde{n}_k &= \lim_{k \to \infty} m(t_{\tilde{n}_k}) \geq \lim_{k \to \infty} m(t_{n_k} - T) = \infty
\end{align*} 
from which it also follows that $\lim_{k \to \infty} \tilde{s}_k = 0$ and therefore $\lim_{k \to \infty} \tilde{u}_k \leq  \lim_{k \to \infty} \tilde{s}_k + \delta_k = 0$.  Since $0 \leq \tilde{s}_k < \epsilon_{\tilde{n}_k+1}$ we also have $Z^{\tilde{n}_k}(\tilde{s}_k) = Z^{\tilde{n}_k}(0) = 0$ so the sequences $\tilde{n}_k$ and $\tilde{u}_k$ satisfy the claim.

TODO: Note that $\delta_k$ doesn't necessarily go to 0 faster than $\epsilon_{n_k} + \epsilon_{n_k + 1}$ so $Z^{n_k}(\delta_k)$ may be a sum of multiple jumps $\epsilon_{n_k} Z_{n_k} + \dotsb + \epsilon_{n_k + m_k} Z_{n_k + m_k}$.  This makes the geometry a tad confusing for me.  Can we reduce to a case in which $0 \leq \delta_k + \epsilon_{n_k} < \epsilon_{n_k+1}$?


TODO: I still don't see the geometry here.  Relevant facts: 
\begin{itemize}
\item $\theta_{n+1} \in \interior(F)$ implies $Z_n = 0$ (this follows from the next item and the fact that $N_F(\theta_{n+1}) = \lbrace 0 \rbrace$ if $\theta_{n+1} \in \interior(F)$
\item $Z_n \in N_F(\theta_{n+1})$
\item $\epsilon_n Z_n \toas 0$
\item $M^n(t) \toas 0$ uniformly on compacts
\item $B^n(t) \toas 0$ uniformly on compacts
\end{itemize}
In prose, a asymptotic jump in $Z^{n_k}$ cannot be into the interior because that would imply that $Z_{n_k} = 0$.  The claim is that in the limit, the jump therefore must be from the boundary of $F$ (I don't see this yet but I suspect it follows from $\theta_{n+1} - \theta_n \toas 0$) to another point on the boundary of $F$ (I get this follows from $Z_n \in N_F(\theta_{n+1})$) and that this contradicts the fact that $Z_n \in N_F(\theta_{n+1})$ (I don't see this yet).  The basic intuition is this: 
\begin{itemize}
\item asymptotically since $\epsilon_n Z_n \to 0$ a.s. we know that any jump $Z^n(0)$ in the limit must be from the boundary of $F$
\item jumps in $Z_n$ are always to the boundary and not to the interior thus in the limit this is true and therefore the asymptotic jump is from a boundary point to another boundary point
\item since $Z_n \in N_F(\theta_{n+1})$ and the normal cone points \emph{into} $F$ it is impossible for the jump to be between boundary points
\end{itemize}
So the relevant geometry here just precludes some kind of limit of $Z_n$ pointing out of the normal cone (this seems like it will be true with regularity assumptions for then we know that any limit of proximal normals is a limiting normal but with appropriate regularity assumptions we know that proximal normals are the same thing as limiting normals).

TODO: Finish
\end{proof}

\section{Dynamical Systems Approach}

This section follows Benaim's notes (which in turn summarize a bunch of the Benaim and Hirsch work).

\begin{defn}Let $(S,r)$ be a metric space a continuous map $\Phi : [0,\infty) \times S \to S$. We write $\Phi_t(x) = \Phi(t,x)$ for $0 \leq t < \infty$ and $\Phi^x(t) = \Phi(t,x)$ for $x \in S$.  $\Phi$ is said to be a \emph{semiflow} if 
\begin{itemize}
\item[(i)] $\Phi_0 = \IdentityMatrix$
\item[(ii)] $\Phi_s \circ \Phi_t = \Phi_{t+s}$ for all $0 \leq t,s < \infty$.
\end{itemize}
A continuous map $\Phi : (-\infty, \infty) \times S \to S$ with properties (i) and (ii) is called a \emph{flow}.  Given $x \in S$ we often refer to $\Phi^x$ as an \emph{orbit}.
\end{defn}

\begin{defn}Let $(S,r)$ be a metric space and let $\Phi$ be a semiflow on $S$, then $X \in C([0,\infty) ; S)$ is said to be an \emph{asymptotic pseudotrajectory} of $\Phi$ if for all $T > 0$ we have 
\begin{align*}
\lim_{t \to \infty} \sup_{0 \leq h \leq T} r(X(t+h), \Phi_h(X(t))) = 0
\end{align*}
If the image $X([0,\infty))$ has compact closure we often say that $X$ is \emph{precompact}.
\end{defn}

Our first goal is formulate some conditions that are equivalent to an $X$ being an asymptotic pseudotrajectory.  The idea is to investigate the orbits of $X$ under a canonical flow on the space $C((-\infty, \infty); S)$.  The first step is to define the \emph{translation flow} on $C((-\infty, \infty); S)$.  Recall that $C((-\infty, \infty); S)$ is a metric space with metric
\begin{align*}
d(f,g) &= \sum_{n=1}^\infty 2^{-n} (\sup_{-n \leq t \leq n} r(f(t), g(t)) \wedge 1)
\end{align*}

\begin{prop}Let $(S,r)$ be a metric space define $\Theta : (-\infty, \infty) \times C((-\infty, \infty); S) \to C (-\infty, \infty); S)$ by $\Theta(t,f)(s) =f(t+s)$, then $\Theta$ is a flow.
\end{prop}
\begin{proof}
The only non-trivial part is the proof that $\Theta$ is continuous.  Suppose that $(t, f), (t_n,f_n) \in  (-\infty, \infty) \times C((-\infty, \infty); S)$ and $\lim_{n \to \infty} (t_n, f_n) = (t,f)$.  Let $\epsilon > 0$ and $T > 0$ be given.  $f$ is uniformly continuous on $[-T-\abs{t}-1,T+\abs{t}+1]$ so we may pick a $\delta > 0$ such that 
\begin{align*}
\sup_{\substack{\abs{u}, \abs{v} \leq T + \abs{t}+1 \\ \abs{u-v} < \delta}} r(f(u), f(v)) < \epsilon/2
\end{align*}
Since $\lim_{n \to \infty} t_n = t$ and $\lim_{n \to \infty} f_n = f$ we may pick $N > 0$ such that $\abs{t_n - t} < \delta \wedge 1$ and $\sup_{\abs{s} \leq T + \abs{t} + 1} r(f_n(s), f(s)) < \epsilon/2$ for all $n \geq N$
Therefore for all $n \geq N$
\begin{align*}
\sup_{-T \leq s \leq T} r(\Theta(t_n, f_n)(s), \Theta(t,f)(s)) &= \sup_{-T \leq s \leq T} r(f_n(t_n +s), f(t+s)) \\
&\leq \sup_{-T \leq s \leq T} r(f_n(t_n +s), f(t_n+s)) + \sup_{-T \leq s \leq T} r(f(t_n +s), f(t+s)) \\
&\leq \sup_{\abs{s} \leq T+\abs{t}+1} r(f_n(s), f(s)) + \sup_{\abs{s} \leq T+\abs{t}+1} r(f(s), f(s)) < \epsilon \\
\end{align*}
\end{proof}

TODO: Is there a translation flow on $D((-\infty, \infty); S)$ in the uniform topology?  in the Skorohod topology?

If we are given a semiflow $\Phi : [0,\infty) \times S \to S$ it is often convenient to consider it as a flow $\Phi : (-\infty,\infty) \times S \to S$ by defining $\Phi(-t,x) = \Phi(0,x) = x$.  TODO: Exercise to show this is a flow (specifically continuity)

In order to compare the flow $\Phi$ with the flow $\Theta$ we use the following
\begin{prop}Let $(S,r)$ be a metric space and let $\Phi$ be a semiflow or flow on $S$.  If we define $H(x) = \Phi^x \in C((-\infty, \infty); S)$ and $S_\Phi = \range{H}$ then $H$ is a homeomorphism of $S$ and $S_\Phi$ and moreover
\begin{align*}
\Theta_t (H(x)) &= H(\Phi_t(x))
\end{align*}
where we assume that $t \geq 0$ if $\Phi$ is a semiflow and $-\infty < t < \infty$ is $\Phi$ is a flow.  Therefore $S_\Phi$ is a closed subset of $C((-\infty, \infty); S)$ and is invariant under $\Theta$.  The map $\hat{\Phi} : C((-\infty, \infty);S) \to S_\Phi$ defined by $\hat{\Phi}(X) = H(X(0)) = \Phi^{X(0)}$ is continuous retraction.
\end{prop}
\begin{proof}
TODO
\end{proof}

We can now reformulate the definition of asymptotic pseudotrajectory in terms of behavior under the translation flow.
\begin{lem}Let $(S,r)$ be a metric space, $X \in C([0,\infty); S)$ and $\Phi$ a semiflow on $S$ then $X$ is an asymptotic pseudotrajectory of $\Phi$ if and only if
\begin{align*}
\lim_{t \to \infty} d(\Theta_t(X), \hat{\Phi}(\Theta_t(X))) &= 0
\end{align*}
\end{lem}
\begin{proof}
By Lemma \ref{UniformConvergenceOnCompacts} we know that $\lim_{t \to \infty} d(\Theta_t(X), \hat{\Phi}(\Theta_t(X))) = 0$ if and only if $\lim_{t \to \infty} \sup_{0 \leq s \leq T} r(\Theta_t(X)(s), \hat{\Phi}(\Theta_t(X))(s)) $ for every $T>0$.  Substituting definitions of $\Theta$ and $\hat{\Phi}$ we see that this is equivalent to
$\lim_{t \to \infty} \sup_{0 \leq s \leq T} r(X(t+s), \Phi_s(X_t)) $ for every $T>0$.  
\end{proof}

\begin{thm}\label{AsymptoticPseudotrajectoryTranslationLimitPointCharacterization}Let $(S,r)$ be a metric space, $X \in C((-\infty, \infty); S)$ such that $\range{X}$ has compact closure in $S$ and $\Phi : [0,\infty) \times S \to S$ be a semiflow.  The $X$ is an asymptotic pseudotrajectory of $\Phi$ is and only if $X$ is uniformly continuous and every limit point of $\Theta_t(X)$ is in $S_\Phi$.  In either case $\lbrace \Theta_t(X) \rbrace$ is relatively compact in $C((-\infty, \infty); S)$.  
\end{thm}
\begin{proof}
To see (i) implies (ii) first assume that $X$ is an asymptotic pseudotrajectory; for every $T>0$ we have $\lim_{t \to \infty} \sup_{0 \leq s \leq T} r(X_{t+s}, \Phi_s(X_t)) =0$.  Let $K$ be the closure of $\range{X}$ so that $K$ is compact in $S$.  
\begin{clm}Let $\epsilon > 0$ be given and then there exists $\delta>0$ such that $r(\Phi_t(y), y)<\epsilon$ for all $\abs{t} < \delta$
\end{clm}
Let $x \in S$ be arbitrary.  Consider 
\begin{align*}
\Phi^{-1}( B(x, \epsilon/2)) &= \lbrace (t,y) \mid r(\Phi_t(y), x) < \epsilon/2 \rbrace
\end{align*}
Since $\Phi_0$ is identity, we know that $(0,x) \in \Phi^{-1}( B(x, \epsilon/2)) $ and therefore we can find a $0 < \delta_x < \epsilon/2$ such that $B((0,x), \delta_x) \subset \Phi^{-1}( B(x, \epsilon/2))$.  Moreover we assume are using the metric $\abs{\cdot} \vee r(\cdot, \cdot)$ on $[0,\infty) \times S$ so that  $B((0,x), \delta_x) = (-\delta_x, \delta_x) \times B(x,\delta_x)$ From these two facts it follows that for all $y \in B(x, \delta_x)$ then for all $\abs{t} < \delta_x$
\begin{align*}
r(\Phi_t(y), y) &\leq r(\Phi_t(y), x) + r(x,y) < \epsilon
\end{align*}
Now the set of $B(x, \delta_x)$ with $x \in K$ is an open cover of $K$ hence there is a finite subcover $B(x_1, \delta_{x_1}), \dotsc, B(x_n, \delta_{x_n})$.  Let $\delta = \delta_{x_1} \wedge \dotsb \wedge \delta_{x_n}$.  Every $y \in K$ belongs to some $B(x_j, \delta_{x_j})$ and therefore $r(\Phi_t(y), y) < \epsilon$ for $\abs{t} < \delta_{x_j}$ an a fortiori for $\abs{t} < \delta$.

\begin{clm}$X$ is uniformly continuous
\end{clm}
Let $\epsilon>0$ be given and pick $\delta>0$ as in the previous claim.  Because $X$ is an asymptotic pseudotrajectory we may pick a $t_0>0$ such that for all $t \geq t_0$ we have $\sup_{0 \leq h \leq \delta} r(X_{t+h}, \Phi_h(X_t)) < \epsilon$.  Thus
\begin{align*}
\sup_{0 \leq h \leq \delta} r(X_{t+h}, X_t) &\leq \sup_{0 \leq h \leq \delta} r(X_{t+h}, \Phi_h(X_t)) + \sup_{0 \leq h \leq \delta} r(\Phi_h(X_t), X_t) < 2\epsilon
\end{align*}
and uniformly continuity follows. 

Suppose that $Y$ is a limit point of $\Theta_t(X)$ and $Y \notin S_\Phi$.  Since $S_\Phi$ is closed it follows that $d(Y, S_\Phi) > 0$.  Let $n_k$ be a subsequence such that $\lim_{k \to \infty} \Theta_{n_k}(X) = Y$ and observe
\begin{align*}
\lim_{k \to \infty} d(\Theta_{t_k}(X), \hat{\Phi}(\Theta_{t_k}(X))) &\geq \lim_{k \to \infty} \lbrace d(Y, \hat{\Phi}(\Theta_{t_k}(X))) - d(\Theta_{t_k}(X),Y)\rbrace \geq d(Y, S_\Phi) > 0
\end{align*}
which is a contradiction.

Now to see that (ii) implies (i) suppose that $X$ is uniformly continuous and all the limit points of $\Theta_t(X)$ are in $S_\Phi$.  First we show that (ii) implies (iii).
\begin{clm}The family $\lbrace \Theta_t(X) \rbrace$ is relatively compact.
\end{clm}
First we establish equicontinuity of $\lbrace \Theta_t(X) \rbrace$.  Let $T > 0$ and $\epsilon > 0$ be given.  By uniform continuity there exists $\delta > 0$ such that $r(X(t), X(s)) < \epsilon$ for all $\abs{t-s}<\delta$. Therefore
\begin{align*}
\sup_{\substack{-T \leq u<v \leq T \\ v-u<\delta}} r(\Theta_t(X)(u), \Theta_t(X)(v)) &= \sup_{\substack{-T \leq u<v \leq T \\ v-u<\delta}} r(X(t+u), X(t+v)) \leq \sup_{\substack{-\infty <  u<v <\infty\\ v-u<\delta}} r(X(u), X(v)) < \epsilon
\end{align*}
Since $\range{X}$ is relatively compact it follows that $\lbrace \Theta_t(X)(s) \mid t \geq 0 \rbrace$ is relatively compact for every $s$ and therefore we may apply the Arzela-Ascoli Theorem \ref{ArzelaAscoliTheorem} to conclude that $\lbrace \Theta_t(X) \rbrace$ is relatively compact in $C((-\infty,\infty); S)$.  

Now suppose that $\lim_{t \to \infty} d(\Theta_t(X), \hat{\Phi}(\Theta_t(X))) \neq 0$ then there exists an $\epsilon > 0$ and a sequence $t_k$ such that $d(\Theta_{t_k}(X), \hat{\Phi}(\Theta_{t_k}(X))) \geq \epsilon$.  By relative compactness by passing to a subsequence and by using the assumption that every limit point of $\Theta_t(X)$ is in $S_\Phi$  we may assume that there exists $Y \in S_\Phi$ such that $\lim_{k \to \infty} \Theta_{t_k}(X) = Y$.  Since $\hat{\Phi}$ is a continuous retraction onto $S_\Phi$ it follows that 
\begin{align*}
\lim_{k \to \infty} \hat{\Phi}(\Theta_{t_k}(X)) &= \hat{\Phi}( Y) = Y = \lim_{k \to \infty} \Theta_{t_k}(X) 
\end{align*}
which is a contradiction.
\end{proof}

\begin{defn}Suppose we have a continuous function $F : \reals^d \to \reals^d$, a sequence of vectors $x_n \in \reals^d$ for $n \in \integers_+$, $U_n \in \reals^d$ for $n \in \naturals$ and $\gamma_n \in \reals_+$ for $n \in \naturals$ such that $\sum_{n=1}^\infty \gamma_n = \infty$ and $\lim_{n \to \infty} \gamma_n = 0$ satisfying
\begin{align*}
x_{n+1} - x_n &= \gamma_{n+1} ( F(x_n) + U_{n+1})
\end{align*}
we define a sequence 
\begin{align*}
\tau_0 = 0 \text{ and } \tau_n = \sum_{j=1}^n \gamma_j \text{ for $n \geq 1$}
\end{align*}
interpolations of $x_n$ $X, \overline{X} : \reals_+ \to \reals^d$
\begin{align*}
X(\tau_n + s) &= x_n + s\frac{x_{n+1} - x_n}{\tau_{n+1} - \tau_n} = x_n + s \gamma_{n+1}^{-1} (x_{n+1} - x_n) \text{, and } \overline{X}(\tau_n + s) = x_n \text{ for $n \in \naturals$ and $0 \leq s < \gamma_{n+1}$}
\end{align*}
the ``inverse'' of the sequence $\tau_n$, $m : \reals_+ \to \integers_+$
\begin{align*}
m(t) &= \sup \lbrace k \geq 0 \mid t \geq \tau_k \rbrace
\end{align*}
and continuous time interpolations $\overline{U} : \reals_+ \to \reals^d$ and $\overline{\gamma}: \reals_+ \to \reals_+$ defined by
\begin{align*}
\overline{U}(\tau_n + s) = U_{n+1} \text{, and } \overline{\gamma}(\tau_n +s) = \gamma_{n+1}  \text{ for $n \in \naturals$ and $0 \leq s < \gamma_{n+1}$}
\end{align*}
\end{defn} 

We note the following simple facts
\begin{prop}\label{StochasticApproximationInterpolations}For all $t \in \reals_+$, $\overline{X}(t) = X(\tau_{m(t)}) = x_{m(t)}$, $\overline{U}(t) = U_{m(t) + 1}$, $\overline{\gamma}(t) = \gamma_{m(t)+1}$, $\tau_{m(t)} \leq t < \tau_{m(t)+1}$
and 
\begin{align*}
X(t) &= X(0) + \int_0^t (F(\overline{X}(s)) + \overline{U}(s)) \, ds
\end{align*}
\end{prop}
\begin{proof}
By the assumption $\sum_{n=1}^\infty \gamma_n$ we know that $\lim_{n \to \infty} \tau_n = \infty$.  By non-negativity of $\gamma_n$ we know that the sequence $\tau_n$ is non-decreasing.  Therefore by definition we know that $m(t)$ is the unique non-negative integer such that $\tau_{m(t)} \leq t < \tau_{m(t)+1}$.  Note that it may be that for some $n < m(t)$ we also have $\tau_n = \tau_{m(t)}$ but then $\tau_n=\tau_{n+1}$ and it is not the case that $\tau_n \leq t < \tau_{n+1}$.  For such $n$ is also true that $\gamma_{n+1} = 0$ and thus the values at $n$ are not used in any of the interpolations.  In particular writing $t = \tau_{m(t)} +s$ with $0 \leq s < \gamma_{m(t)+1}$ by the definitions of $X$ and $\overline{X}$ we see that
\begin{align*}
\overline{X}(t) &= \overline{X}(\tau_{m(t)} + s) = x_{m(t)} = X(\tau_{m(t)})
\end{align*}
$\overline{U}(t) \overline{U}(\tau_{m(t)} + s) = U_{m(t)+1}$ and $\overline{\gamma}(t) \overline{\gamma}(\tau_{m(t)} + s) = \gamma_{m(t)+1}$.

To see the last equality we express the integral using the fact that the integrand is piecewise constant and that from the discussion above we may write $t = \tau_{m(t)} + s$ with $0 \leq s < \gamma_{m(t)+1}$ and use the recursion defining $x_n$
\begin{align*}
\int_0^t (F(\overline{X}(s)) + \overline{U}(s)) \, ds &= \sum_{n=0}^{m(t)-1} \gamma_{n+1} (F(x_n) + U_{n+1}) + s (F(x_{m(t)}) + U_{m(t)+1}) \\
&=\sum_{n=0}^{m(t)-1} (x_{n+1} - x_n)  + s (F(x_{m(t)}) + U_{m(t)+1}) \\
&= x_{m(t)} - x_0 + s \frac{x_{m(t) + 1} - x_{m(t)}}{\gamma_{m(t)+1}} \\
&= X(t) - X(0)
\end{align*}
\end{proof}

Thinking of the recursion $x_{n+1} = x_n + \gamma_{n+1} (F(x_n) + U_{n+1})$ as a perturbed version of the Euler method for solving an ordinary differential equation we pose the question of how well the interpolations $X(t)$ approximate solutions to the differential equation $\frac{dX}{dt} = F(X(t))$.  In particular we seek asymptotic decay conditions of the sequence of perturbations $\gamma_{n+1} U_{n+1}$ that allow us to prove that $X(t)$ is an asymptotic pseudotrajectory.  The following condition on the noise sequence is due to Kushner and Clark (TODO: Reference).
\begin{defn}We say that $\gamma_{n+1}$ and $U_{n+1}$ satisfy the \emph{Kushner-Clark criterion} if for every $T > 0$ we have
\begin{align*}
\lim_{n \to \infty} \sup \lbrace \norm{\sum_{i=n}^{k-1} \gamma_{i+1} U_{i + 1}} \mid k=n+1, \dotsc, m(\tau_n+T) \rbrace &= 0
\end{align*}
\end{defn}

\begin{prop}\label{KushnerClarkCriterionIntegralForm}Let $\gamma_{n+1}$ and $U_{n+1}$ be given and define $\tau_0=0$, $\tau_n = \sum_{i=1}^n \gamma_i$ for $n \in \naturals$ and
\begin{align*}
\Delta(t, T) = \sup_{0 \leq h \leq T} \norm {\int_t^{t+h} \overline{U}(s) \, ds}
\end{align*} 
then 
\begin{itemize}
\item[(i)] for every $\delta>0$ we have $\Delta(t,T) \leq 2 \Delta(t-\delta, T+\delta)$
\item[(ii)] for every $n \in \integers_+$ we have 
\begin{align*}
\sup \left \lbrace \norm{\sum_{i=n}^{k-1} \gamma_{i+1} U_{i + 1}} \mid k=n+1, \dotsc, m(\tau_n+T) \right \rbrace 
&= \Delta(\tau_n, m(\tau_n+T) - \tau_n) 
\leq \Delta(\tau_n, T)
\end{align*}
\end{itemize}
If $\lim_{n \to \infty} \gamma_n =0$ and $\sum_{n=1}^\infty \gamma_n = \infty$ then the Kushner-Clark condition is equivalent to $\lim_{t \to \infty} \Delta(t,T) = 0$ for all $T > 0$.
\end{prop}
\begin{proof}
To see (i) let $t \geq 0$, $T, \delta>0$ and $0 \leq h \leq T$ be given then
\begin{align*}
\norm{\int_t^{t+h}  \overline{U}(s) \, ds} &=  \norm{\int_{t-\delta}^{t+h}\overline{U}(s) \, ds - \int_{t-\delta}^t \overline{U}(s) \, ds} \\
&\leq \norm{\int_{t-\delta}^{(t-\delta)+h+\delta}\overline{U}(s) \, ds} + \norm{\int_{t-\delta}^t \overline{U}(s) \, ds} \\
&\leq 2 \sup_{0 \leq h \leq T+\delta} \norm{\int_{t-\delta}^{(t-\delta)+h}\overline{U}(s) \, ds} = 2 \Delta(t-\delta, T+\delta)
\end{align*}
To see (ii) we write
\begin{align*}
\sum_{i=n}^{k-1} \gamma_{i+1} U_{i + 1} &= \sum_{i=n}^{k-1} (\tau_{i+1} - \tau_i) \overline{U}(\tau_i) = \int_{\tau_n}^{\tau_k} \overline{U}(s) \, ds
\end{align*}
We note that $\int_{\tau_n}^{t} \overline{U}(s) \, ds$ is linear for $\tau_k \leq t < \tau_{k+1}$ so by the convexity of norms we see that 
\begin{align*}
\sup_{\tau_k \leq t < \tau_{k+1}} \norm{\int_{\tau_n}^{t} \overline{U}(s) \, ds } = \norm{\int_{\tau_n}^{\tau_k} \overline{U}(s) \, ds} \maxop \norm{\int_{\tau_n}^{\tau_{k+1}} \overline{U}(s) \, ds}
\end{align*}
Combining this fact with the fact that $m(\tau_n+T) \leq \tau_n+T$ we get
\begin{align*}
&\sup \left \lbrace \norm{\sum_{i=n}^{k-1} \gamma_{i+1} U_{i + 1}} \mid k=n+1, \dotsc, m(\tau_n+T) \right \rbrace  \\
&= \sup \left \lbrace \norm{\int_{\tau_n}^{\tau_k} \overline{U}(s) \, ds} \mid k=n+1, \dotsc, m(\tau_n+T) \right \rbrace \\
&= \sup \left \lbrace \norm{\int_{\tau_n}^{\tau_k} \overline{U}(s) \, ds} \mid k=n, \dotsc, m(\tau_n+T) \right \rbrace \\
&= \sup \left \lbrace \norm{\int_{\tau_n}^{t} \overline{U}(s) \, ds} \mid \tau_{n} \leq t \leq \tau_{m(\tau_n+T)} \right \rbrace \\
&= \Delta(\tau_n, m(\tau_n+T) - \tau_n) \\
&\leq \Delta(\tau_n, T)
\end{align*}

If $\tau_n \to \infty$ then the condition $\lim_{t \to \infty} \Delta(t,T) = 0$ for all $T > 0$ clearly implies $\lim_{n \to \infty} \Delta(\tau_n,T) = 0$ for all $T > 0$; by (ii) this implies the Kushner-Clark 
criterion.  On the other hand assume the Kushner-Clark criterion, let  $T>0$ and $\epsilon>0$ be given and pick $N_T>0$ such that  $\Delta(\tau_n, m(\tau_n+T+2) - \tau_n) < \epsilon/2$ for all $n \geq N_T$.  Because $\gamma_n \to 0$ we may pick $N_1$ large enough that $\gamma_{n+1} < 1$ for all $n \geq N_1$ from which it follows that for such $n$
\begin{align*}
T + 1 &< \tau_n + T + 2 - \gamma_{m(\tau_n+T+2)+1} - \tau_n \leq \tau_{m(\tau_n + T + 2)+1} - \gamma_{m(\tau_n+T+2)+1} - \tau_n = \tau_{m(\tau_n + T + 2)} - \tau_n
\end{align*}
For $n \geq N_1 \maxop N_T$ we have $\Delta(\tau_n, T+1) \leq \Delta(\tau_n, m(\tau_n+T+2) - \tau_n) < \epsilon/2$.  
Then for all $t \geq N_1 \maxop \tau{N_T}$ there is a unique $n \geq N_1 \maxop N_T$ such that $\tau_n \leq t < \tau_{n+1}$ which implies $t - \tau_{n} < 1$ so
\begin{align*}
\Delta(t,T) &= \Delta(t - \tau_n + \tau_n, T) \leq 2\Delta(\tau_n, T+ (t-\tau_n)) \leq 2 \Delta(\tau_n, T+1) < \epsilon
\end{align*}
which shows $\lim_{t \to \infty} \Delta(t,T) = 0$.
\end{proof}

\begin{prop}\label{KushnerClarkImpliesPseudotrajectory}Let $F : \reals^d \to \reals^d$ be a continuous vector field such that the differential equation $\dot{x} = F(x)$ has unique integral curves.  Assume that $\gamma_{n+1}$ and $U_{n+1}$ satisfy the Kushner-Clark criterion and either
\begin{itemize}
\item[(i)] $\sup_{n} \norm{x_n} < \infty$ 
\item[(ii)] $F$ is Lipschitz and bounded on a (convex?) neighborhod of $\lbrace x_n \rbrace$.
\end{itemize}
then $X$ is an asymptotic pseudotrajectory of the flow induced by $F$.  Furthermore if (ii) is true then
\begin{align*}
\sup_{0 \leq h \leq T} \norm{X(t+h) - \Phi_h(X(t))} &\leq C \left( \Delta(t-1, T+1) + \sup_{t \leq s \leq t+T} \norm{\overline{\gamma}(s)} \right)
\end{align*}
where the constant $C$ only depends on $T$ and $F$.
\end{prop}
\begin{proof}
\begin{clm}\label{KushnerClarkImpliesPseudotrajectory:UniformContinuity} $X$ is uniformly continuous.
\end{clm}
If (i) holds then by continuity of $F$ we know that $\sup_n \norm{F(x_n)} < \infty$.  If (ii) holds then  $\sup_n \norm{F(x_n)} < \infty$  because $F$ is assumed bounded on a neighborhood of $\lbrace x_n \rbrace$.  In either case there exists a constant $K$ such that $\sup_t \norm{F(X(t))} \leq K$.  From Proposition \ref{StochasticApproximationInterpolations} and Proposition \ref{KushnerClarkCriterionIntegralForm} we get
\begin{align*}
&\limsup_{t \to \infty} \sup_{0 \leq h \leq T} \norm{X(t+h) - X(t)} = \limsup_{t \to \infty}\sup_{0 \leq h \leq T} \norm{\int_t^{t+h} (F(\overline{X})(u) + \overline{U}(u)) \, du} \\
&\leq \limsup_{t \to \infty}\sup_{0 \leq h \leq T} \norm{\int_t^{t+h} F(\overline{X})(u) \, du} + \limsup_{t \to \infty} \sup_{0 \leq h \leq T}\norm{\int_t^{t+h} \overline{U}(u)) \, du} \leq KT
\end{align*}

From this it follows that $X$ is uniformly continuous (given $\epsilon > 0$ let $\delta_1 = \epsilon/2K  \minop 1$ and pick $s>0$ such that $\sup_{t \geq s} \sup_{0 \leq h \leq \delta_1} \norm{X(t+h) - X(t)} < K\delta_1 + \epsilon/2 = \epsilon$ for all $t\geq s$.  Since $[0,s+1]$ is compact we know that $X$ is uniformly continuous on $[0,s+1]$ (Theorem \ref{UniformContinuityOnCompactSets}) and  therefore there exists $\delta_2>0$ such that $\sup{0 \leq t \leq s} \sup_{0 \leq h \leq \delta_2} \norm{X(t+h) - X(t)} < \epsilon$.  Let $\delta= \delta_1 \minop \delta_2$ and it follows that $\sup_{0 \leq t < \infty} \sup_{0 \leq h \leq \delta} \norm{X(t+h) - X(t)} < \epsilon$).

Now lets break $\Theta_t (X)$ up into some terms that we'll examine individually.
\begin{align}\label{KushnerClarkImpliesPseudotrajectory:TranslationTerms}
\Theta_t(X)(s) &= X(t+s) = X(0) + \int_0^{t+s} F(\overline{X}(u)) \, du + \int_0^{t+s} \overline{U}(u) \, du \\
&=X(0) + \int_0^{t} (F(\overline{X}(u)) + \overline{U}(u) ) \, du + \int_t^{t+s} F(\overline{X}(u)) \, du + \int_t^{t+s} \overline{U}(u) \, du \text{ ????} \\
&=X(t) + \int_0^s F(X(t+u)) \, du + \int_t^{t+s} (F(\overline{X})(u) - F(X(u))) \, du + \int_t^{t+s} \overline{U}(u) \, du \\
&=\Theta_t(X)(0) + \int_0^s F(\Theta_t(X)(u)) \, du + \int_t^{t+s} (F(\overline{X})(u) - F(X(u))) \, du + \int_t^{t+s} \overline{U}(u) \, du \\
&=L_F(\Theta_t(X))(s) + A_t(s) + B_t(s)
\end{align}
where have defined
\begin{align*}
L_F(X)(s) &= X(0) + \int_0^s F(X(u)) \, du \text{ for $X \in C([0,\infty); \reals^d)$} \\ 
A_t(s) &= \int_t^{t+s} (F(\overline{X}(u)) - F(X(u))) \, du \\
B_t(s) &=  \int_t^{t+s} \overline{U}(u) \, du 
\end{align*}

Note that $L_F(X) = X$ if and only if $X$ is an integral curve of the differential equation $\dot{x} = F(x)$ (TODO: Presumably this is the fixed point operator used in Picard iteration).

The last term is the easiest to handle;  by Lemma \ref{UniformConvergenceOnCompacts} the Kushner-Clark assumption is equivalent to the statement that $\lim_{t \to \infty} B_t = 0$ in $C([0,\infty); \reals^d)$ (i.e. uniformly on compact sets).

We now turn to estimates on the second term which addresses errors that arise as a result of mixing the linear and constant interpolations of the $x_n$.
\begin{clm} $\lim_{t \to \infty} A_t = 0$ in $C([0,\infty); \reals^d)$
\end{clm}
Fix $t,T > 0$  and consider the interval $t \leq u \leq t+T$.  From Proposition \ref{StochasticApproximationInterpolations}  we get
\begin{align*}
\norm{X(u) - \overline{X}(u)} &= \norm{X(u) - X(\tau_{m(u)})} = \norm{\int_{\tau_{m(u)}}^u (F(\overline{X}(s)) + \overline{U}(s)) \, ds} \\
&\leq K (u - \tau_{m(u)}) + \norm{\int_{\tau_{m(u)}}^u \overline{U}(s) \, ds} \leq K \overline{\gamma}(u) + \norm{\int_{\tau_{m(u)}}^u \overline{U}(s) \, ds} 
\end{align*}

Since $\lim_{n \to \infty} \gamma_n = 0$ and $t \leq u$ we know that for $t$ sufficiently large we have 
\begin{align*}
\tau_{m(u)+1} - \tau_{m(u)} &= \gamma_{m(u)+1} = \overline{\gamma}(u) < 1
\end{align*}
and therefore $t \leq u < \tau_{m(u)+1} < 1 + \tau_{m(u)}$; in particular $t-1 < \tau_{m(u)} \leq u$.  Therefore we may write
\begin{align*}
\norm{\int_{\tau_{m(u)}}^u \overline{U}(s) \, ds} &= \norm{\int_{t-1}^u \overline{U}(s) \, ds - \int_{t-1}^{\tau_{m(u)}} \overline{U}(s) \, ds} \\
&\leq \norm{\int_{t-1}^u \overline{U}(s) \, ds} + \norm{\int_{t-1}^{\tau_{m(u)}} \overline{U}(s) \, ds} \leq 2\Delta(t-1, T+1)
\end{align*}
and so we get
\begin{align}\label{KushnerClarkImpliesPseudotrajectory:InterpolationError}
\sup_{t \leq u \leq t+T} \norm{X(u) - \overline{X}(u)} &\leq 2\Delta(t-1, T+1) + K \sup_{t \leq u \leq t+T}\overline{\gamma}(u) 
\end{align}
Under either assumption (i) or (ii), $F$ is uniformly continuous on a neighborhood of the $\lbrace x_n \rbrace$.  Fix $T > 0$ then for any
$\epsilon > 0$ there exists a $\delta$ such that $\norm{X(u) - \overline{X}(u)} < \delta$ implies 
$\norm{F(X(u)) - F(\overline{X}(u))} < \epsilon/T$.  By \eqref{KushnerClarkImpliesPseudotrajectory:InterpolationError}, the Kushner-Clark condition 
and the fact that $\gamma_n \to 0$ we know that $\sup_{t \leq u \leq t+T} \norm{X(u) - \overline{X}(u)} < \delta$ for sufficiently large $t$.  Hence
\begin{align*}
\sup_{0 \leq s \leq T} \norm{A_t(s)} &= \sup_{0 \leq s \leq T} \norm{\int_t^{t+s} (F(\overline{X}(u)) - F(X(u))) \, du} \leq T (\epsilon/T) = \epsilon
\end{align*}
hence $\lim_{t \to \infty} \sup_{0 \leq s \leq T} \norm{A_t(s)} =0$.  Since $T>0$ we arbitrary we see that $A_t \to 0$ in $C([0,\infty), \reals^d)$.

We note that the operator $L_F : C([0,\infty), \reals^d) \to C([0,\infty), \reals^d)$ is continuous (TODO: This should be put somewhere that we discuss
Picard iteration).  This follows from the fact that $F$ is continuous and therefore $X \to F \circ X$ is continuous, evaluation $X \to X(0)$ is continuous and
$X \to \int_0 X(s) \, ds$ is continuous.

Now suppose that $X^*$ is a limit point of $\lbrace \Theta_t(X) \rbrace$.  Thus there exists a sequence $t_n$ with $t_n \to \infty$ and $\lim_{n \to \infty} \Theta_{t_n}(X) = X^*$
in $C([0,\infty), \reals^d$.  It follows from the prior claims and the continuity of $L_F$ that
\begin{align*}
X^* &= \lim_{n \to \infty} \Theta_{t_n}(X) = \lim_{n \to \infty} L_F(\Theta_{t_n}(X)) + A_{t_n} + B_{t_n} = \lim_{n \to \infty} L_F(\Theta_{t_n}(X)) \\
&= L_F(\lim_{n \to \infty}\Theta_{t_n}(X)) = L_F(X^*)
\end{align*}
and therefore $X^*$ is a solution of $\dot{x} = F(x)$.  Since we have assumed that $F$ has unique integral curves it follows that in fact $X^* = \Phi^{X^*(0)} \in S_\Phi$.
From Claim \ref{KushnerClarkImpliesPseudotrajectory:UniformContinuity} and  Theorem \ref{AsymptoticPseudotrajectoryTranslationLimitPointCharacterization} it follows that $X$ is an asymptotic pseudotrajectory.  (TODO: Don't we need relative compactness of $X([0,\infty))$ which only holds under (i)?)

Now suppose that (ii) holds and that $\norm{F(x) - F(y)} \leq L \norm{x - y}$ on a neighborhood of the $\lbrace x_n \rbrace$.  In this case from \eqref{KushnerClarkImpliesPseudotrajectory:InterpolationError} we get for $t$ sufficiently large and $T>0$
\begin{align*}
\norm{A_t(s)} &= \norm{\int_t^{t+s} (F(\overline{X}(u)) - F(X(u))) \, du} \leq T L (2\Delta(t-1, T+1) + K\sup_{t \leq u \leq t+T}\overline{\gamma}(u) ) \text{ for $0 \leq s \leq T$} \\
\end{align*}
and we also have
\begin{align*}
\norm{B_t(s)} &= \norm{\int_t^{t+s} \overline{U}(u) \, du} \leq \Delta(t,T) \leq 2 \Delta(t-1, T+1)
\end{align*}
By \eqref{KushnerClarkImpliesPseudotrajectory:TranslationTerms} and the fact that as an integral curve we have $\Phi_s(X(t)) = \Phi^{X(t)}(s)= L_F(\Phi^{X(t)})$ and Gronwall's Inequality (Proposition \ref{GronwallsInequality}) for sufficiently large $t$, all $T>0$ and $0 \leq s \leq T$
\begin{align*}
\norm{X(t+s) - \Phi_s(X(t))} &= \norm{\Theta_t(X)(s) - L_F(\Theta_t(X))(x) +  L_F(\Theta_t(X))(s) + L_F(\Phi^{X(t)})(s)} \\
&=\norm{A_t(s) + B_t(s) +  \int_0^s (F(\Theta_t(X)(u)) - F(\Phi_u(X(t)))) \, du} \\
&\leq \norm{A_t(s)} + \norm{B_t(s)} + L \int_0^s \norm{X(t+u) - \Phi_u(X(t))} \, du \\
&\leq 2(TL + 1) \Delta(t-1, T+1) + TLK \sup_{t \leq u \leq t+T}\overline{\gamma}(u) + L \int_0^s \norm{X(t+u) - \Phi_u(X(t))} \, du \\
&\leq (2(TL + 1) \Delta(t-1, T+1) + TLK \sup_{t \leq u \leq t+T}\overline{\gamma}(u)) (1 + \int_0^s e^{L(s-u)} \, du) \\
&\leq (2(TL + 1) \Delta(t-1, T+1) + TLK \sup_{t \leq u \leq t+T}\overline{\gamma}(u)) (1 + L^{-1} e^{TL}) \\
\end{align*}
Now take the supremeum over all $0 \leq s \leq T$.
\end{proof}

We now return to the realm of probability and we assume that the sequences $x_n$ and $U_n$ are now random processes.  The first question is whether we can find probabilistic hypotheses 
that guarantee the Kushner-Clark conditions hold almost surely.  

We need a small consequence of H\"{o}lder's inequality.
\begin{lem}\label{BMPHolderInequality}Let $a_i \geq 0$, $b_i \in \reals$, $p > 1$ and $0 < \delta < 1$ then 
\begin{align*}
\left(\sum_{i=n}^m \abs{ a_i b_i} \right)^p &\leq \left( \sum_{i=n}^m a_i^{\delta p/(p-1)} \right)^{p-1} \sum_{i=n}^m a_i^{1-\delta)p} \abs{b_i}^p
\end{align*}
\end{lem}
\begin{proof}
Noting that $p$ and $\frac{p}{p-1}$ are conjugate exponents we simply apply H\"{o}lder's inequality
\begin{align*}
\sum_{i=n}^m \abs{a_i b_i} &= \sum_{i=n}^m (a_i^\delta) (a_i^{1-\delta} \abs{b_i}) \\
&\leq \left(\sum_{i=n}^m a_i^{\delta p/p-1} \right )^{p-1/p} \left( \sum_{i=n}^m a_i^{1-\delta)p} \abs{b_i}^p \right){1/p}
\end{align*}
and take the $p^{th}$ power.  TODO: Why do we need $0 < \delta < 1$?
\end{proof}

TODO: Do we need a vector valued version of Burkholder or is the current proof provided valid in $\reals^d$???

\begin{prop}\label{AlmostSureKushnerClarkBoundedMoments}Let $(\Omega, \mathcal{A}, P)$ be a probability space with a filtration $\mathcal{F}_n$.  Suppose that $x_n$ and $U_n$ are adapted processes, $U_n$ is a martingale difference sequence (i.e. $\cexpectationlong{\mathcal{F}_n}{U_{n+1}}=0$ for all $n \in \integers_+$), $\gamma_n$ is a deterministic sequence such that $\lim_{n \to \infty} \gamma_n = 0$ and $\sum_{n=1}^\infty \gamma_n = \infty$ and 
\begin{align*}
x_{n+1} &= x_n + \gamma_{n+1} (F(x_n) + U_{n+1})
\end{align*}
If for some $q \geq 2$ we have $\sup_n \expectation{\norm{U_{n+1}}^q} < \infty$ and $\sum_{n=1}^\infty \gamma_n^{1+q/2} < \infty$ then the Kushner-Clark condition holds almost surely.
\end{prop}
\begin{proof}
\begin{clm}For all $T > 0$ and $t \geq 0$ there exists a constant $C_q$ such that
\begin{align*}
\expectation{\Delta(t, T)^q} &\leq C_q  T^{q/2 -1} \sup_m \norm{U_m}^q \int_{t}^{t+T} \overline{\gamma}^{q/2}(s) \, ds
\end{align*}
\end{clm}
The proof of the claim requires an inequality.  Let $\psi_1, \psi_2, \dotsc$ be a sequence of non-negative numbers and define $\sigma_0=0$ and $\sigma_n = \sum_{i=1}^n \psi_i$ for $n \geq 1$.   Since $\psi_{n+1} U_{n+1}$ is an $\mathcal{F}$ martingale difference sequence we know that $\sum_{i=1}^n \psi_{n+1} U_{n+1}$ is an $\mathcal{F}$-martingale and therefore by 
the right hand side of Burkholder's Inequality (Theorem \ref{BurkholderInequalities}) we conclude that for every $t \geq 0$ and $n \in \integers_+$
\begin{align*}
\expectation{\sup_{n < k \leq m(\sigma_n + T)} \norm{\sum_{i=n}^{k-1} \psi_{i+1} U_{i+1}}^q} &\leq C_q \expectation{\left(\sum_{i=n}^{m(\sigma_n + T)-1} \psi^2_{i+1}\norm{ U_{i+1}}^2\right)^{q/2}}
\end{align*}
If we suppose $q > 2$ then we can apply Lemma \ref{BMPHolderInequality} with $p=q/2$, $\delta=(q-2)/2q$, $a_i=\psi^2_{i+1}$ and $b_i = \norm{U_{i+1}}^2$ hence ($p/(p-1) = (q/2)/((q/2)-1) = q/(q-2)$ and $1-\delta = (q+2)/2q$)  to conclude
\begin{align*}
&\expectation{\sup_{n < k \leq m(\sigma_n + T)} \norm{\sum_{i=n}^{k-1} \psi_{i+1} U_{i+1}}^q} \\
&\leq C_q \expectation{\left(\sum_{i=n}^{m(\sigma_n + T)-1} \psi^{2\left( \frac{q-2}{2q} \right) \left ( \frac{q}{q-2} \right)}_{i+1}\right)^{q/2-1} \sum_{i=n}^{m(\sigma_n + T)-1} \psi_{i+1}^{2\left( \frac{q+2}{2q} \right) \left( \frac{q}{2} \right)} \norm{U_{i+1}}^{2 \left(\frac{q}{2} \right)}} \\
&= C_q \expectation{\left(\sum_{i=n}^{m(\sigma_n + T)-1} \psi_{i+1}\right)^{q/2-1} \sum_{i=n}^{m(\sigma_n + T)-1} \psi_{i+1}^{1+q/2} \norm{U_{i+1}}^{q}} \\
&= C_q \expectation{\left(\sigma_{m(\sigma_n+T)} - \sigma_n \right)^{q/2-1} \sum_{i=n}^{m(\sigma_n + T)-1} \psi_{i+1}^{1+q/2} \norm{U_{i+1}}^{q}} \\
&\leq C_q (\sigma_n + T - \sigma_n)^{q/2 -1} \expectation{\sum_{i=n}^{m(\sigma_n + T)-1} \psi_{i+1}^{1+q/2} \norm{U_{i+1}}^{q}} \\
&\leq C_q  T^{q/2 -1} \sup_m \norm{U_m}^q \sum_{i=n}^{m(\sigma_n + T)-1} \psi_{i+1}^{1+q/2} \\
\end{align*}
Now if we fix $t \geq 0$ and we consider $\Delta(t,T) = \sup_{t \leq u \leq t+T} \norm{\int_t^{u} \overline{U}(s) \, ds}$.  As in the proof of Proposition \ref{KushnerClarkCriterionIntegralForm} the piecewise linearity of the integral as a function of $u$ and the convexity of the norm implies that the supremum is attained at some $u \in \lbrace t, m(t)+1, \dotsc, m(t+T), t+T \rbrace$.
Define the sequence $\psi_{m(t)} = t$, $\psi_{m(t)+1} = \tau_{m(t)+1} - t = \gamma_{m(t)+1} - (t - \tau_{m(t)})$, $\psi_i = \gamma_i$ for $i=m(t)+1, \dotsc, m(t+T)$ and $\psi_{m(t+T)+1} = t+T - m(t+T)$.  Applying the above inequality (noting $\sigma_n = t$) we get (TODO: there is some ambiguity to clear up about $m$ defined by the $\gamma_i$ and $m$ defined by the $\psi$; the salient point is these two functions are equal on the interval $[t,t+T]$).
\begin{align*}
\expectation{\Delta(t,T)} &= \expectation{\norm{\sup_{m(t) < k \leq m(t+T)} \psi_{i+1} U_{i+1}}^q} \\
&\leq C_q  T^{q/2 -1} \sup_m \norm{U_m}^q \sum_{i=m(t)}^{m(t + T)-1} \psi_{i+1}^{1+q/2}  \\
&\leq C_q  T^{q/2 -1} \sup_m \norm{U_m}^q \sum_{i=m(t)}^{m(t + T)-1} \psi_{i+1} \gamma_{i+1}^{q/2}  \\
&=C_q  T^{q/2 -1} \sup_m \norm{U_m}^q \int_t^{t+T} \overline{\gamma}^{q/2}(s) \, ds
\end{align*}

TODO: Handle the case $q=2$ which is more direct.

For $q \geq 2$ and every $\epsilon > 0$
\begin{align*}
\sum_{k=0}^\infty \probability{\Delta(kT,T) > \epsilon} &\leq \epsilon^{-q} \sum_{k=0}^\infty \expectation{\Delta(kT, T)^q} \\
&\leq \epsilon^{-q} C_q T^{q/2 -1} \sup_m \norm{U_m}^q \sum_{k=0}^\infty \int_{kT}^{(k+1)T} \overline{\gamma}^{q/2}(s) \, ds \\
&= \epsilon^{-q} C_q T^{q/2 -1} \sup_m \norm{U_m}^q \int_0^\infty \overline{\gamma}^{q/2}(s) \, ds \\
&= \epsilon^{-q} \sum_{n=1}^\infty \gamma_{n+1}^{1 + q/2} < \infty
\end{align*}
and therefore by the Borel Cantelli Theorem \ref{BorelCantelli} we get $\probability{\Delta(kT,T) > \epsilon i.o.} =0$ and by Lemma \ref{ConvergenceAlmostSureByInfinitelyOften}
$\lim_{k \to \infty} \Delta(kT,T) = 0$ almost surely.

For an arbitrary $0 \leq t < \infty$ there exists a unique $k \in \integers_+$ such that $kT \leq t < (k+1)T$ and for such a $k$ we have for $0 \leq h \leq T$  
$\int_t^{(t+h) \minop (k+1)T} = \int_{kT}^{(t+h) \minop (k+1)T} - \int_{kT}^t$ and therefore $\norm{\int_t^{(t+h) \minop (k+1)T}} \leq \norm{\int_{kT}^{(t+h) \minop (k+1)T}} + \norm{\int_{kT}^t} \leq 2 \sup_{0 \leq h \leq T} \int_{kT}^{kT+h}$ also
$\norm{\int_{(t+h) \minop (k+1)T}^{t+h}} \leq sup_{0 \leq h \leq T}\norm{\int_{(k+1)T}^{(k+1)T+h}}$ and so $\Delta(t,T) \leq 2 \Delta(kT, T) + \Delta((k+1)T,T)$.  This shows that 
$\lim_{t \to \infty} \Delta(t,T) = 0$ almost surely.  Now we apply Proposition \ref{KushnerClarkCriterionIntegralForm}.
\end{proof}

\section{Exercises}
\begin{xca}Let $f : \reals \to \reals$ be a right continuous function,
  show that $f$ is Borel measurable.
\end{xca}
\begin{proof}
It suffices to show that $f^{-1}(t, \infty)$ is Borel measurable for
all $t \in \reals$.  Let $x \in f^{-1}(t, \infty)$ then by right
continuity there exists some $y_x$ with $x < y_x$ such that $[x,y_x) \subset
f^{-1}(t, \infty)$.  Clearly we may write $f^{-1}(t, \infty) = \cup_{x
  \in f^{-1}(t, \infty)} [x, y_x)$.  We now show that we can make this
a countable union of intervals.  For a fixed $q \in \rationals$
consider the set $A_q = \cup_{\substack{x \in f^{-1}(t, \infty) \\ q
    \in [x, y_x)}} [x, y_x)$.  It is easy to see that $A_q$ is either
empty or an
interval (either open or half open) by taking least upper bounds and
greatest lower bounds of the intervals in the union.  Thus each $A_q$
is measurable.  More over each $[x,y_x)$ contains a rational number so
it follows that $[x,y_x) \subset A_q$ for some $q \in rationals$.
From this it follows that $f^{-1}(t, \infty) = \cup_{q \in
  \rationals}A_q$ which is a countable union of measurable sets and
therefore measurable.
\end{proof}

\begin{xca}Let $f(x)$ be a Lebesgue integrable function on $\reals$.
  Show that there exists a measurable $a(x)$ with $\lim_{x \to \infty}
  a(x) = \infty$ such that $a(x)f(x)$ remains integrable.
\end{xca}
\begin{proof}
It suffices to assume that $f(x) \geq 0$ and $\int f(x) \, dx = 1$.
We know from Fundamental Theorem of Calculus that $g(y) = \int_{-\infty}^y
f(x) \, dx$ is almost everywhere differentiable (and montone) and
$g^\prime(y) = f(y)$.   By
definition $\lim_{y \to \infty} g(y) = 1$.  Now define $h(z) = 1 -
\sqrt{1 -z}$ and note that by the Chain Rule (TODO: Show that the
Chain Rule is still valid for functions that are merely absolutely continuous)
\begin{align*}
\frac{d}{dy} h(g(y)) &= \frac{f(y)}{2 \sqrt{1 - g(y)}}
\end{align*}
Now by the Fundamental Theorem of Calculus again, if we define $a(x) =
\frac{1}{2 \sqrt{1 - g(x)}}$ 
then 
\begin{align*}
\int a(x) f(x) \, dx &= \lim_{y \to \infty} h(g(y)) = h(1) = 1
\end{align*}
but 
\begin{align*}
\lim_{x \to \infty} a(x) &= \lim_{x \to \infty} \frac{1}{2 \sqrt{1 -
    g(x)}} = \infty
\end{align*}
\end{proof}
\begin{xca}Let $\xi$ be a random variable, show that for all $\lambda > 0$,
\begin{align*}
\min_k \expectation{\xi^q} \lambda^{-k} \leq \inf_{s>0} \expectation{e^{s(\xi-\lambda)}}
\end{align*}
Note that this shows that the best moment bound for a tail
probability is always better than the best Chernoff bound.
\end{xca}
\begin{proof}
Let $q = \argmin_k \expectation{\xi^k} \lambda^{-k}$.  Now expand as a
series
\begin{align*}
\expectation{e^{s(\xi-\lambda)}} &= e^{-s\lambda} \sum_{k=0}^\infty
\frac{s^k \expectation{\xi^k}}{k!} \\
& \geq e^{-s\lambda} \expectation{\xi^q} \lambda^{-q} \sum_{k=0}^\infty
\frac{s^k \lambda^k}{k!} = \expectation{\xi^q} \lambda^{-q}
\end{align*}
\end{proof}
\begin{xca}Let $\xi$ be a nonnegative integer valued random variable.
  Show $\probability{\xi \neq 0} \leq \expectation{\xi}$ and 
\begin{align*}
\probability{\xi = 0} \leq \frac{\variance{\xi}}{\variance{\xi} + \left(\expectation{\xi}\right)^2}
\end{align*}
\end{xca}
\begin{proof}
For the first inequality,
\begin{align*}
\probability{\xi \neq 0} = \sum_{k=1}^\infty \probability{\xi = k} \leq
\sum_{k=1}^\infty k\probability{\xi = k} = \expectation{\xi}
\end{align*}
For the second inequality, use Cauchy-Schwartz
\begin{align*}
\left(\expectation{\xi}\right)^2 &\leq
\left(\expectation{\characteristic{\xi > 0}\xi}\right)^2 \\
&\leq \expectation{\xi^2} \probability{\xi > 0}
\end{align*}
Now use $\probability{\xi > 0} = 1 - \probability{\xi=0}$ and
$\variance{\xi} = \expectation{\xi^2} -
\left(\expectation{\xi}\right)^2$ and rearrangement of terms to get
the result.
\end{proof}

\begin{xca}Let $f : S \to T$ be function.  If $\mathcal{T}$ is a
  $\sigma$-algebra on $T$ then $\mathcal{T} \subset f_*
  f^{-1}(\mathcal{T})$.  If $\mathcal{S}$ is a $\sigma$-algebra on
  $S$, then $f^{-1}f_*(\mathcal{S}) \subset \mathcal{S}$.  Find examples where the inclusions are strict.
\end{xca}
\begin{proof}
To see the inclusions just unwind the definitions.  For the first inclusion
\begin{align*}
f_* f^{-1}(\mathcal{T}) &= \lbrace A \subset T \mid f^{-1}(A) \in
f^{-1}(\mathcal{T}) \rbrace \\
&= \lbrace A \subset T \mid f^{-1}(A) =
f^{-1}(B) \text { for some } B \in \mathcal{T} \rbrace \\
&\supset \mathcal{T}
\end{align*}
and for the second
\begin{align*}
f^{-1} f_* (\mathcal{S}) &= \lbrace f^{-1}(A) \mid A \in  f_*
(\mathcal{S}) \rbrace \\
&= \lbrace f^{-1}(A) \mid A \subset T \text{ and } f^{-1}(A) \in
\mathcal{S} \rbrace \\
&\subset \mathcal{S}
\end{align*}

TODO: Find the examples of strict inclusion.
\end{proof}
\begin{xca}Let $f : S \to T$ be a set function and let $\mathcal{C}
  \subset 2^T$ then $f^{-1}(\sigma(\mathcal{C})) = \sigma(f^{-1}(\mathcal{C}))$.
\end{xca}
\begin{proof}
We know that $f^{-1}(\sigma(\mathcal{C}))$ is a $\sigma$-algebra and
clearly $f^{-1}(\mathcal{C}) \subset f^{-1}(\sigma(\mathcal{C}))$
therefore showing $\sigma(f^{-1}(\mathcal{C})) \subset
f^{-1}(\sigma(\mathcal{C}))$.  

To see the reverse inclusion we know that 
\begin{align*}
f_* (\sigma(f^{-1}(\mathcal{C}))) &= \lbrace A \subset T \mid
f^{-1}(A) \in \sigma(f^{-1}(\mathcal{C})) \rbrace
\end{align*}
is a $\sigma$-algebra and clearly $\mathcal{C} \subset f_*
(\sigma(f^{-1}(\mathcal{C})))$.  This implies $\sigma(\mathcal{C}) \subset f_*
(\sigma(f^{-1}(\mathcal{C})))$ and thus by the result of the previous
exercise
\begin{align*}
f^{-1}(\sigma(\mathcal{C})) &\subset 
f^{-1} (f_*(\sigma(f^{-1}(\mathcal{C})))) \subset \sigma(f^{-1}(\mathcal{C}))
\end{align*}
\end{proof}
\begin{xca}Let $f(x) = \abs{x}$.  Show that $f_*(\mathcal{B}(\reals))$
  is a strict $\sigma$-subalgebra of $\mathcal{B}(\reals)$.
\end{xca}
\begin{xca}Let $f : S \to T$ be a function, $\mathcal{C} \in
  2^S$ and define $f_*(\mathcal{C}) =  \{A \subset T \mid
    f^{-1}(A) \in \mathcal{C} \}$.  Show by counterexample that
    $\sigma(f_*(\mathcal{C})) \neq f_*(\sigma(\mathcal{C}))$.
\end{xca}

\begin{xca}Let $A_n$ be a sequence of events. Show that 
\begin{align*}
\probability{A_n \text{ i.o.}} \geq \limsup_{n \to \infty} \probability{A_n}
\end{align*}
\end{xca}
\begin{proof}
Note that we know that for every $k\geq n$, $A_k \subset
\cup_{k=n}^\infty A_k$ and therefore monotonicity of measure implies $\probability{A_k} \leq
\probability{\cup_{k=n}^\infty A_k}$ for $k\geq n$.  Therefore we know
$\sup_{k\geq n} \probability{A_k} \leq
\probability{\cup_{k=n}^\infty A_k}$.

By definition and continuity of measure and applying the above,
\begin{align*}
\probability{A_n \text{ i.o.}} &= \probability{\cap_{n=1}^\infty
  \cup_{k=n}^\infty A_k} \\
&= \lim_{n \to \infty} \probability{\cup_{k=n}^\infty A_k} \\
&\geq \lim_{n \to \infty} \sup_{k \geq n} \probability{ A_k} =
\limsup_{n \to \infty} \probability{ A_n} 
\end{align*}
\end{proof}

\begin{xca}Suppose we toss a coin repeatedly and the probability of
  heads is $0 < p < 1$ (i.e. the coin may be unfair but not
  pathological).  Without using the Strong Law of Large Numbers show
  that the probability of flipping only a finite number heads is $0$.
\end{xca}
\begin{proof}
Let $A_n = \lbrace \text{heads is flipped on the }n^{th}\text{
  toss}\rbrace$.  We know that $\probability{A_n} = p >0$, therefore
$\sum_{n=1}^\infty \probability{A_n} = \infty$.  We also know that
$A_n$ are independent events, therefore the converse of the
Borel-Cantelli Theorem (Theorem \ref{BorelCantelli}) tells us that $\probability{ A_n \text{ i.o.}}
= 1$.  The probability of tossing only a finite number of
heads is $1 - \probability{ A_n \text{ i.o.}} = 0$.
\end{proof}

\begin{xca}A sequence of random variables $\xi_1, \xi_2, \dots$ is said
  to be \emph{completely convergent} to $\xi$ if for every $\epsilon > 0$,
\begin{align*}
\sum_{n=1}^\infty \probability{\abs{\xi_n - \xi} > \epsilon} < \infty
\end{align*}
Show that if $\xi_n$ are independent then complete convergence is
equivalent to almost sure convergence.
\end{xca}
\begin{proof}
First assume that $\xi=0$.  

We first assume complete convergence.  If for a given $\epsilon > 0$, we know $\sum_{n=1}^\infty
\probability{\abs{\xi_n} > \epsilon} < \infty$ then we can apply
Borel Cantelli to conclude that $\probability{\xi_n > \epsilon \text{ i.o.}} =
0$.  Thus there exists a set $A_\epsilon$ of measure zero such that
for all $\omega \notin A_\epsilon$, we can find $N>0$ such that
$\xi_n(\omega) \leq \epsilon$.  Define $A = \cup_{m=1}^\infty A_\frac{1}{m}$,
note that $\probability{A} = 0$ and that for every $\omega \notin A$,
and every $\epsilon >0$ we can pick $\frac{1}{m} < \epsilon$  and then
we know  $N>0$ such that $\xi_n(\omega) \leq \frac{1}{m} \leq \epsilon$

Then if $\xi_n \toas 0$, then there
exists an event $A$ with $\probability{A} = 1$ and such that for any
$\omega \in A$, $\epsilon>0$ we can find $N > 0$ such that
$\abs{\xi_n} < \epsilon$, thus $\probability{\abs{\xi_n} > \epsilon
  \text{ i.o.}} \leq 1 - \probability{A} = 0$.  By independence of
$\xi_n$ and Borel Cantelli we conclude that $\sum_{n=1}^\infty
\probability{\abs{\xi_n} > \epsilon} < \infty$.

Now in the case in which $\xi \neq 0$ we can reduce to the case in
which $\xi = 0$.  Note that by Corollary
\ref{ConstantLimitOfIndependent} to the Kolmogorov 0-1 Theorem, we
know that $\xi$ is almost surely a constant $c$.  Then we can define $\xi_n
- c$ and note that $\xi_n - c$ are independent by Lemma \ref{IndependenceComposition}.
\end{proof}

\begin{xca}Suppose $\eta, \xi_1, \xi_2, \dots$ are random variables with
  $\abs{\xi_n} \leq \eta$ a.s. for all $n > 0$.  Show that $\sup_n
  \abs{\xi_n} \leq \eta$ a.s.
\end{xca}
\begin{proof}
Let $A_n = \lbrace \xi_n \leq \eta \rbrace$ and $A = \cup_n A_n$.
By assumption,
$\probability{A_n} =0$ and therefore by countable subadditivity of measure,
$\probability{A} = 0$.  For all $\omega \notin A$, we know for all
$n>0$, $\xi_n(\omega) \leq \eta(\omega)$ and therefore
$\sup_n\xi_n(\omega) \leq \eta(\omega)$.
\end{proof}

\begin{xca}\label{ExConvProb1}Let $\xi, \xi_n$ be random elements in a metric space $S$
  such that $\xi_n \toprob \xi$, let $A_n$ be events such that
  $\probability{A_n} = 1$ and let $\eta_n$ be random elements in
  $S$ such that $\eta_n = \xi_n$ on $A_n$, show that $\eta_n \toprob \xi$.
\end{xca}
\begin{proof}
Fix $\epsilon > 0$ and note that
\begin{align*}
\lim_{n \to \infty} \probability{d(\eta_n, \xi) > \epsilon} &=
\lim_{n \to \infty} \probability{d(\eta_n, \xi) > \epsilon ; A_n } +
\lim_{n \to \infty} \probability{d(\eta_n, \xi) > \epsilon;A_n^c}
\leq \lim_{n \to \infty} \probability{d(\xi_n, \xi) > \epsilon } +
\lim_{n \to \infty} \probability{A^c_n} = 0
\end{align*}
\end{proof}

\begin{xca}Suppose $\xi, \xi_1, \xi_2, \dots$ are random variables with
  $\xi_n \toas \xi$ and $\xi < \infty$ a.s.  Let $\eta = \sup_n
  \abs{\xi_n}$ and show that $\eta < \infty$ a.s.
\end{xca}
\begin{proof}
TODO
\end{proof}

\begin{xca}[Kallenberg Ex 3.6]Let $\mathcal{F}_{t,n}$ with $t \in T$ and $n \in \naturals$
  be $\sigma$-algebras such that for a fixed $t$ they are
  nondecreasing in $n$ and for a fixed $n$ they are independent in
  $t$.  Show that the $\sigma$-algebras $\bigvee_n \mathcal{F}_{t,n}$
  are independent.
\end{xca}
\begin{proof}
Because for fixed $t \in T$, we have $\mathcal{F}_{t,0} \subset
\mathcal{F}_{t,1} \subset \cdots$ we can see that $\bigcup_n
\mathcal{F}_{t,n}$ is a $\pi$-system.  Since by definition $\bigcup_n
\mathcal{F}_{t,n}$ generates $\bigvee_n
\mathcal{F}_{t,n}$ by Lemma \ref{IndependencePiSystem} it suffices to
show that $\bigcup_n \mathcal{F}_{t,n}$ are independent.

Pick $A_{t_1} \in \mathcal{F}_{t_1,n_1}, \dotsc, A_{t_m} \in
\mathcal{F}_{t_m,n_m}$. Let $n = n_1 \orop \dotsb \orop n_m$ and use
the nondecreasing property of $\mathcal{F}_{t,n}$ to observe that $A_{t_1} \in \mathcal{F}_{t_1,n}, \dotsc, A_{t_m} \in
\mathcal{F}_{t_m,n}$.  By the assumption that each of $\mathcal{F}_{t_j,n}$ is independent
therefore $\probability {A_1 \cup \dotsb \cup A_m} = \probability{A_1}
\dotsm \probability{A_m}$ and we are done.
\end{proof}

\begin{xca}[Kallenberg Ex 3.7]Let $T$ be an arbitrary index set and let
  $(S_t, \mathcal{B}(S_t))$ be metric spaces with Borel
  $\sigma$-algebras.  For each $t\in T$ suppose have random elements
  random elements $\xi^t, \xi^t_n \in S_t$  for $n \in \naturals$ such
  that $\xi^t_n \toas \xi^t$.  If for each fixed $n \in \naturals$ the
  $\xi^t_n$ are independent show that $\xi^t$ are independent.
\end{xca}
\begin{proof}
Pick a finite subset $\lbrace t_1, \dotsc, t_m \rbrace \subset T$ and
assume we are given bounded continuous functions $f_j : S_{t_j} \to
\reals$ for $j=1, \dotsc, m$.
By Lemma \ref{IndependenceExpectations} and the independence of the
$\xi^{t_j}_n$ we have $\expectation{f_1(\xi^{t_1}_n) \dotsm
  f(\xi^{t_m}_n)} = \expectation{f_1(\xi^{t_1}_n)}\dotsm
 \expectation{ f(\xi^{t_m}_n)}$ for each $n \in \naturals$.  But now
 we can use the boundedness and continuity of the $f_j$ 
\begin{align*}
&\expectation{f_1(\xi^{t_1}) \dotsm f_m(\xi^{t_m})} \\
&=
\expectation{\lim_{n \to \infty} f_1(\xi_n^{t_1}) \dotsm
  f_m(\xi_n^{t_m})} & & \text{by continuity} \\
&= \lim_{n \to \infty} \expectation{f_1(\xi_n^{t_1}) \dotsm
  f_m(\xi_n^{t_m})} & & \text{boundedness of $f_j$ and Dominated
  Convergence}  \\
&= \lim_{n \to \infty} \expectation{f_1(\xi_n^{t_1})} \dotsm
 \expectation{ f_m(\xi_n^{t_m})} & & \text{independence}  \\
&= \expectation{f_1(\xi^{t_1})} \dotsm
 \expectation{ f_m(\xi^{t_m})} & &
 \text{continuity and Dominated Convergence} \\
\end{align*}

We now prove a slight extension of Lemma
\ref{IndependenceExpectations} that shows this is sufficient to see
that $\xi^t$ are independent.  Let $(S,d)$ be a metric space and let
$U \subset S$ be open.  We show how to approximate the indicator
function $\characteristic{U}$ be bounded continuous functions.  Let
$d(x, U^c) = \inf \lbrace d(x,y) \mid y \not U \rbrace$.  Note that
$d(x, U^c)$ is continuous (see proof Lemma
\ref{DistanceToSetLipschitz}).  Let $f_n(x) = 1 \wedge n d(x, U^c)$
and observe that $f_n \uparrow \characteristic{U}$.  Now suppose
$U_{j} \subset S_{t_j}$ are open sets for $j=1, \dotsc,m$ and use the
construction just presented to create bounded continuous functions
$f^j_n \uparrow \characteristic{U_j}$.  Then it is also true that
$f^1_n \dotsm f^m_n \uparrow \characteristic{U_1} \dotsm
\characteristic{U_m}$ and so we can apply Montone convergence to see 
\begin{align*}
\probability{\xi^{t_1} \in U_1 \cap \dotsb \cap \xi^{t_m} \in U_m} &=
\lim_{n \to \infty} \expectation{f^1_n(\xi^{t_1}) \dotsm
f^m_n(\xi^{t_m})} \\
&= \lim_{n \to \infty} \expectation{f^1_n(\xi^{t_1}) } \dotsm
\expectation{f^m_n(\xi^{t_m})} \\
&= \probability{\xi^{t_1} \in U_1} \dotsm\probability{\xi^{t_m} \in
  U_m} 
\end{align*}
Now it suffices to note that the open sets  in a metric space are a
$\pi$-system that generates all of the Borel sets so by Lemma
\ref{IndependencePiSystem} it suffices to
show independence on open sets.
\end{proof}

A simpler subcase of the above
\begin{xca}Let $\xi, \xi_n$ be random elements in a metric space $S$
  such that $\xi_n \toprob \xi$ and each $\xi_n$ is
  $\mathcal{F}_n$-measurable.  Furthermore suppose $\mathcal{G}$ is a
  $\sigma$-algebra such that $\mathcal{F}_n \Independent \mathcal{G}$
  for all $n \in \naturals$, then show $\xi$ is independent of
  $\mathcal{G}$.
TODO: In the proof we mention that $\mathcal{F}_1 \subset
\mathcal{F}_2 \subset \cdots$.  Is that really required?  If not
provide a counter example.
\end{xca}
\begin{proof}
Since $\xi_n \toprob \xi$ we know there is a subsequence  that
converges almost surely.  Note that all of the hypotheses restrict
cleanly to the subsequence so we might as well assume that $\xi_n
\toas \xi$.  By the $\mathcal{F}_n$ measurability of $\xi_n$ we see
that each $\xi_n$ is $\bigvee_n \mathcal{F}_n$-measurable and
therefore $\xi$ is almost surely equal to a $\bigvee_n
\mathcal{F}_n$-measurable function.  It therefore suffices to show
that $\bigvee_n
\mathcal{F}_n \Independent \mathcal{G}$ (TODO: show this simple fact; if $\xi
= \eta$ a.s. and $\xi \Independent \mathcal{G}$ then $\eta \Independent
\mathcal{G}$).  This follows from the fact
that the nestedness of the $\mathcal{F}_n$ implies $\bigcup_n
\mathcal{F}_n$ is a $\pi$-system.  Since by definition it generates $\bigvee_n
\mathcal{F}_n$ we get the result from Lemma \ref{IndependencePiSystem}.
\end{proof}

\begin{xca}Let $\xi_1, \xi_2, \dots$ be independent random variables with values in
  $[0,1]$.  Show that $\expectation{\prod_{n =1}^\infty \xi_n} =
  \prod_{n=1}^\infty \expectation{\xi_n}$.  In particular, for
  independent events $A_n$ we have $\probability{\cup_{n=1}^\infty
    A_n} = \prod_{n=1}^\infty \probability{A_n}$.
\end{xca}
\begin{proof}
Note that because $\xi_n$ have values in $[0,1]$, the partial products
$\prod_{k=1}^n \xi_k \leq 1$ and therefore by Dominated Convergence
and Lemma \ref{IndependenceExpectations}, we have
\begin{align*}
\expectation{\prod_{k =1}^\infty \xi_k} &= \lim_{n \to \infty}
\expectation{\prod_{k=1}^n \xi_k} = \lim_{n \to \infty}
\prod_{k=1}^n\expectation{ \xi_k} = \prod_{k=1}^\infty \expectation{ \xi_k} 
\end{align*}
\end{proof}

\begin{xca}Provide an example of uncorrelated but non-independent
  random variables.
\end{xca}
\begin{proof}See Example \ref{UncorrelatedNotIndependent}.
\end{proof}

\begin{xca}Let $\xi_1, \xi_2, \dots$ be random variables.  Show that
  there exist constants $c_1 > 0, c_2 >0, \cdots$ such that
  $\sum_{n=1}^\infty c_n \xi_n$ converges almost surely.
\end{xca}
\begin{proof}First note that we can make a few assumptions about $\xi_n$ without
loss of generality.  First, we can assume that $\xi_n \geq 0$ for all
$n$; knowing that that will show absolute convergence for all series.  Next,
note that by a comparison test argument, we may further assume that $\xi_n >
0$ for all $n$ (e.g. for a random variable $\xi$ that takes $0$ as a value we can
always create the modification $\xi + \characteristic{\xi^{-1}(0)}$
which is nonzero and dominates $\xi$).  

The idea here is to leverage freshman calculus and use the ratio
test.  We first verify the following almost sure version of the ratio
test: Let $\xi_n$ be positive random variables such that there exists
a $0 < C < 1$ such that
$\sum_{n=1}^\infty \probability{\frac{\abs{\xi_{n+1}}}{\abs{\xi_n}} >
  C} < \infty$, then $\sum_{n=1}^\infty \xi_n$ converges almost
surely.

To verify the claim, we apply Borel Cantelli to conclude that $\probability{\frac{\abs{\xi_{n+1}}}{\abs{\xi_n}} >
  C \text{ i.o.}} = 0$.  Unwinding the definitions in this statement, we see that for
almost every $\omega \in \Omega$, there exists an $N>0$ such that
$\frac{\abs{\xi_{n+1}(\omega)}}{\abs{\xi_n(\omega)}} \leq C$
for all $n > N$.  The ratio test tells us $\sum_{n=1}^\infty
\xi_n(\omega)$  converges and the almost sure convergence is verified.

Now we apply the claim in our case by choosing $C=\frac{1}{2}$ and inductively defining $c_n$ so
that we guarantee $\probability{\frac{c_{n+1} \xi_{n+1}}{c_n \xi_n} >
  \frac{1}{2}} < \frac{1}{n^2}$.  To see that this is possible,
suppose we've defined $c_n$ and note that because $\xi_n > 0$, we know
that $0 < \frac{\xi_{n+1}}{c_n \xi_n} < \infty$.  This tells us that
$\lim_{N \to \infty} \probability{\frac{\xi_{n+1}}{c_n \xi_n} > N} =0$
and therefore we can find $M>0$ such that
$\probability{\frac{\xi_{n+1}}{c_n \xi_n} > N} < \frac{1}{n^2}$ for
all $N\geq M$.  Pick $c_{n+1} = \frac{1}{2M}$ and we are done.

Here is some things that I tried that proved to be a dead end.  Is there
a learning opportunity in looking at this?  Note that almost sure
convergence of $\sum_{n=1}^\infty c_n \xi_n$ is equivalent to
$\probability{\abs{\sum_{n=1}^\infty c_n \xi_n} \geq N \text{
    i.o.}}$.  The idea was to try to find $c_n$ so that we could
provide bounds on $\probability{c_n\abs{\xi_n} \geq N}$ and leverage
those to show bounds on the series.  The problem I had with this
approach is that to go from a bound on $c_n\abs{\xi_n}$ to convergence
of the series meant that $c_n\abs{\xi_n}$ had to decay fast enough to
get convergence.  If we assume a finite moment then Markov could
provide a rate of decay but in the absence of that one has to deal
with the fact that tails of $\xi_n$ can decay increasingly slowly.
I tried a truncation argument but fact that $\xi_n$ are not related
meant that I couldn't figure out how to control the residuals of the
truncations.  Maybe this line of reasoning could be made to work but I
got stuck.

Guolong asks a good follow on question: either prove this or (more
likely) provide a
counterexample on general (non-finite) measure spaces (e.g. Lebesgue measure on $\reals$).
\end{proof}

\begin{xca}Let $\xi_1, \xi_2, \dots$ be positive independent random
  variables, then $\sum_{n=1}^\infty \xi_n$ converges almost surely if
  and only if $\sum_{n=1}^\infty \expectation{\xi_n \wedge 1} <
  \infty$.
TODO: Provide hints
\end{xca}
\begin{proof}One direction is easy and doesn't require the assumption
  of independence; namely assume that $\sum_{n=1}^\infty
  \expectation{\xi_n \wedge 1} < \infty$.
Apply Tonelli's Theorem (Corollary \ref{TonelliIntegralSum}) to
conclude $\expectation{\sum_{n=1}^\infty \xi_n \wedge 1} < \infty $
which implies that $\sum_{n=1}^\infty \xi_n \wedge 1 < \infty$ almost
surely.  For any $\omega \in \Omega$ such that $\sum_{n=1}^\infty
\xi_n(\omega) \wedge 1 < \infty$ this implies $lim_{n \to \infty}
\xi_n(\omega) \wedge 1 = 0$ so there exists an $N_\omega > 0$ such
that $\xi_n(\omega) \wedge 1 = \xi_n(\omega)$  for all $n>N_\omega$ 
and therefore $\sum_{n=1}^\infty\xi_n(\omega) < \infty$ as
well.

Now lets assume $\sum_{n=1}^\infty \xi_n < \infty$.  Since $\xi_n
\wedge 1 \leq \xi_n$ we know that $\sum_{n=1}^\infty \xi_n < \infty$, 
so without loss of generality we
can assume $0 \leq \xi_n \leq 1$.

\begin{align*}
0 &< \expectation{e^{-\sum_{n=1}^\infty \xi_n}} 
&=\expectation{\prod_{n=1}^\infty e^{-\xi_n}} 
&= \prod_{n=1}^\infty \expectation{ e^{-\xi_n}} \\
&\leq \prod_{n=1}^\infty \left ( 1 - a\expectation{\xi_n} \right ) &
&\text{ where $a=1-e^{-1}$ by Lemma
  \ref{BasicExponentialInequalities}} \\
&\leq \prod_{n=1}^\infty e^{- a\expectation{\xi_n} } & & \text{since
  $1+x\leq e^x$ by Lemma \ref{BasicExponentialInequalities}} \\
&= e^{-a \sum_{n=1}^\infty \expectation{\xi_n} }
\end{align*}
which shows that $\sum_{n=1}^\infty \expectation{\xi_n} < \infty$.
\end{proof}

\begin{xca}\label{MeasurabilityKernelExtraParameter}Let $\mu : S \times \mathcal{T} to [0,1]$ be a probability kernel and let $f : U \times T \to \reals$ be measurable then $\int f(u, t) \, \mu(s, dt)$ is $\mathcal{U} \times \mathcal{S}$ measurable.
\end{xca}
\begin{proof}
Assume first that $f$ is the characteristic function of a set $A \times B \in \mathcal{U} \otimes \mathcal{T}$.  Then 
\begin{align*}
\int \characteristic{ A\times B} (u, t) \, \mu(s, dt) &= \characteristic{A}(u) \mu(s, B)
\end{align*}
which is clearly measurable since $\mu$ is a kernel.  We know that sets $A \times B$ are a $\pi$-system generating $\mathcal{U} \otimes \mathcal{T}$ so can argue with monotone classes to extend to general characteristic functions.  To be specific, let 
\begin{align*}
\mathcal{C} &= \lbrace C \in \mathcal{U} \otimes \mathcal{T}  \mid \int \characteristic{C}(u,t) \, \mu(s,dt) \text{ is measurable} \rbrace
\end{align*}
If $A \in \mathcal{C}$ and $B \in \mathcal{C}$ with $A \subset B$ then 
$\int \characteristic{ B \setminus A} (u, t) \, \mu(s, dt) = \int \characteristic{ B } (u, t) \, \mu(s, dt) - \int \characteristic{A} (u, t) \, \mu(s, dt)$ is measurable so that $B \setminus A \in \mathcal{C}$.  Similarly if $A_1 \subset A_2 \subset \dotsb$ with $A_n \in \mathcal{C}$ then defining $A = \cup_{n=1}^\infty A_n$ we have by Monotone Convergence 
$\int \characteristic{A}(u,t) \, \mu(s, dt) = \lim_{n \to \infty} \int \characteristic{A_n}(u,t) \, \mu(s, dt)$ is a limit of measurable function hence is measurable.  This shows that $\mathcal{C}$ is a $\lambda$-system and therefore by the $\pi$-$\lambda$ Theorem \ref{MonotoneClassTheorem} we know that $\mathcal{C} \subset \mathcal{U}\otimes \mathcal{T}$.  By linearity it follows that $\int f(u,t) \, \mu(s, dt)$ is measurable for all simple functions.

Now given an arbitrary non-negative measurable $f : U \times T \to [0,\infty)$ we find an increasing sequence of simple functions $f_n \uparrow f$, note that for each fixed $u \in U$ it remains true that the sections $f_n(u, \cdot) \uparrow f(u, \cdot)$ and thus we can use Monotone Convergence to see that $\int f(u,t) \, \mu(s,dt) = \lim_{n \to \infty} \int f_n(u,t) \, \mu(s,dt)$ for every $(u,s) \in U \times S$ so that $\int f(u,t) \, \mu(s,dt)$ is measurable.  For an arbitrary measurable function $f$ just write $f = f_+ - f_-$ and use the result for non-negative measurable functions.
\end{proof}

\begin{xca}Suppose $\xi$ is a random variable, let $\mathcal{F}$ be
  a $\sigma$-algebra and let $A$ be a measurable set.  Show that
  $\cexpectationlong{\mathcal{F},A}{\xi} =
  \frac{\cexpectationlong{\mathcal{F}}{\xi ;
      A}}{\cprobability{\mathcal{F}}{A}}$ on $A$.
\end{xca}
\begin{proof}
Note by Localization we know that $\characteristic{A}
\cexpectationlong{\mathcal{F},A}{\xi}  =
\cexpectationlong{\mathcal{F},A}{\xi;A}$, therefore we may assume that
$\xi = \characteristic{A} \xi$ and show
$\cexpectationlong{\mathcal{F},A}{\xi} = \characteristic{A}\frac{\cexpectationlong{\mathcal{F}}{\xi}}{\cprobability{\mathcal{F}}{A}}$ almost surely.

Pick $F \in \mathcal{F}$ and calculate
\begin{align*}
\expectation{\characteristic{A} \frac{\cexpectationlong{\mathcal{F}}
{\xi}}
{\cprobability{\mathcal{F}}{A}}
; A \cap F} 
&= \expectation{\cexpectationlong{\mathcal{F}}{\frac{\xi ; F}
{\cprobability{\mathcal{F}}{A}}} 
; A } & & \text{by pushout}\\
&= \expectation{\cexpectationlong{\mathcal{F}}{\frac{\xi ; F}
{\cprobability{\mathcal{F}}{A}}} 
\cprobability{\mathcal{F}}{A} } \\
&= \expectation{\cexpectationlong{\mathcal{F}}{\xi ; F}} & & \text{by
  pushout} \\
&= \expectation{\xi ; F} = 
\expectation{\xi ; A \cap F} & & \text{by tower property}
\end{align*}
and trivially
\begin{align*}
\expectation{\characteristic{A} \frac{\cexpectationlong{\mathcal{F}}
{\xi}}
{\cprobability{\mathcal{F}}{A}}
; A^c \cap F}  &= 0 = \expectation{\xi ; A^c \cap F}
\end{align*}
Since sets of the form $A \cap F$, $A^c \cap F$ and $F$ for $F \in
\mathcal{F}$ form a $\pi$-system that generate $\sigma(A,
\mathcal{F})$ we have shown the result.
\end{proof}

\begin{xca}Let $A_1, A_2, \cdots$ be a disjoint partition of $\Omega$
  and let $\mathcal{F} = \sigma(A_1, A_2, \dots)$.  Show that for
  every integrable random variable $\xi$ we have
$\cexpectationlong{\mathcal{F}}{\xi} = \sum_{\probability{A_n} \neq 0}
\frac{\expectation{\xi ; A_n}}{\probability{A_n}} \characteristic{A_n}$ almost surely.
\end{xca}
\begin{proof}First note that it is trivial that $\sum_{\probability{A_n} \neq 0}
\frac{\expectation{\xi ; A_n}}{\probability{A_n}}\characteristic{A_n}$ is
$\mathcal{F}$-measurable.  Because the $A_n$ are a disjoint partition,
  they are a $\pi$-system and the it will suffice to show the
  averaging property for the sets $A_n$.
Pick an $A_m$ such that $\probability{A_m} \neq 0$, they by
disjointness of the $A_n$ we get
\begin{align*}
\expectation{\sum_{\probability{A_n} \neq 0}
\frac{\expectation{\xi ; A_n}}{\probability{A_n}}
\characteristic{A_n}; A_m} &= \expectation{\frac{\expectation{\xi ;
    A_m}}{\probability{A_m}} \characteristic{A_m}} = \expectation{\xi ; A_m}
\end{align*}
For any $A_m$ with $\probability{A_m} = 0$ and again applying the
disjointness of the $A_n$ we get
disjointness of the $A_n$ that 
\begin{align*}
0 &= \expectation{\sum_{\probability{A_n} \neq 0}
\frac{\expectation{\xi ; A_n}}{\probability{A_n}}
\characteristic{A_n}; A_m}  = \expectation{\xi ; A_m}
\end{align*}
\end{proof}

\begin{xca}Suppose $\xi$ is a random element in $S$ such that
  $\cprobability{\mathcal{F}}{\xi \in \cdot}$ has a regular version
  $\nu$.  Let $f : S \to T$ be measurable.  Show that
  $\cprobability{\mathcal{F}}{f(\xi) \in \cdot}$ has a regular version
  given by $\pushforward{f} {\nu} (\omega, A) = \nu(\omega, f^{-1}(A))$.
\end{xca}
\begin{proof}
Our hypothesis is that for every $A$, $\cprobability{\mathcal{F}}{\xi
  \in A}(\omega) = \mu(\omega, A)$.  We calculate 
\begin{align*}
\cprobability{\mathcal{F}}{f(\xi) \in A}(\omega) &=
\cexpectationlong{\mathcal{F}}{\characteristic{f^{-1}(A)}(\xi)} \\
&=\int \characteristic{f^{-1}(A)}(s) \, d\mu(\omega, s) & & \text{by
  Theorem \ref{Disintegration}} \\
&=\mu(\omega, f^{-1}(A))
\end{align*}
and we are done.
\end{proof}

\begin{xca}Let $\xi$ be a random element in $S$.  Show that $\xi$ is
  $\mathcal{F}$-measurable if and only if $\delta_\xi$ is a regular
  version of 
  $\cprobability{\mathcal{F}}{\xi \in \cdot}$.

TODO: Refine this statement to include almost sureness...
\end{xca}
\begin{proof}
$\mathcal{F}$-measurability of $\xi$ is equivalent to
$\mathcal{F}$-measurability of $\characteristic{A}(\xi)$ for all $A$
which is equivalent to $\cprobability{\mathcal{F}}{\xi \in A} =
\characteristic{A}(\xi)$ almost surely for all $A$.  Evaluating the
last equality at $\omega$ we see that 
\begin{align*}
\cprobability{\mathcal{F}}{\xi \in A}(\omega) &= \begin{cases}
1 & \text{if $\xi(\omega) \in A$} \\
0 & \text{if $\xi(\omega) \notin A$} 
\end{cases}\\
&= \delta_{\xi(\omega)}(A)
\end{align*}

The fact that $\delta_\xi$ is a probability kernel is simple.  It is
trivial that for fixed $\omega$, $\delta_\xi(\omega)$ is a probability
measure.  If we fix $A$ then $\delta_\xi(\omega)(A)$ is clearly seen
to be measurable since it is just the characteristic function of the
measurable set $A$.
\end{proof}

\begin{xca}Let $\xi$ be an integrable random variable for which
  $\cexpectationlong{\mathcal{F}}{\xi} \eqdist \xi$.  Show that in
  fact $\cexpectationlong{\mathcal{F}}{\xi} = \xi$ a.s.
\end{xca}
\begin{proof}
Here is a simple and conceptual proof in the case that
$\cexpectationlong{\mathcal{F}}{\xi}$ (and therefore $\xi$) take
finitely many values/are simple functions.  Let $y_1 < \cdots < y_n$ be the values of
$\xi$ such that $\probability{\xi = y_i} \neq 0$.   Consider $A_1 = \lbrace \cexpectationlong{\mathcal{F}}{\xi} =
y_1 \rbrace$.  By definition of conditional expectation
$\expectation{\xi ; A_1} =
\expectation{\cexpectationlong{\mathcal{F}}{\xi} ; A_1} = y_1
\probability{A_1}$.  Because $y_1$ is the minimum value of $\xi$ it
follows that we must have $\xi = y_1$ identically on $A_1$.  Since $\xi \eqdist
\cexpectationlong{\mathcal{F}}{\xi} $, we know that $\probability{\xi
  = y_1} = \probability{A_1}$ and therefore $\xi \geq y_2$ almost
surely off of $A_1$.  Now induct.

If we want to apply standard machinery to go from the
simple function case.  Then we could approximate $\xi$ by an
increasing family of simple functions of the form $f_n(\xi)$ but then
we know that $f_n(\xi) \eqdist
f_n(\cexpectationlong{\mathcal{F}}{\xi})$ but not necessarily that $f_n(\xi) \eqdist
\cexpectationlong{\mathcal{F}}{f_n(\xi)}$ which is what we would need
in order to use the simple function case.  All roads seem to lead to a
need to show that $\cexpectationlong{\mathcal{F}}{f(\xi)}$ and
$f(\cexpectationlong{\mathcal{F}}{\xi})$ are equal in some sense
(either a.s. or in distribution).

The idea is to use Jensen's inequality.  First note that
  we can find a strictly convex function $f$ such that $0 \leq f(x)
  \leq \abs{x}$.  Therefore we know that $\expectation{f(\xi)} <
  \infty$.  

Moreover, by Theorem \ref{ExistenceConditionalDistribution} be have a
regular version $\nu$ for $\cprobability{\mathcal{F}}{\xi \in A}$.  By
Theorem \ref{Disintegration} we know that
$\cexpectationlong{\mathcal{F}}{f(\xi)} = \int s \, d\mu(s)$.

Because $\xi \eqdist \cexpectationlong{\mathcal{F}}{\xi}$ we also know
that $f(\xi) \eqdist f(\cexpectationlong{\mathcal{F}}{\xi})$ which
shows us that ... 

TODO: I am aiming to show that $\pushforward{f}{\mu}$ is a regular version
for
$\cprobability{\mathcal{F}}{f(\cexpectationlong{\mathcal{F}}{\xi}) \in
\cdot}$.
If we could get that then we could calculate
\begin{align*}
f(\cexpectationlong{\mathcal{F}}{\xi}) &=
\cexpectationlong{\mathcal{F}}
{f(\cexpectationlong{\mathcal{F}}{\xi})} \\
&= \int s d \pushforward{f}{\mu} (s) & & \text{by Theorem \ref{Disintegration}}\\
&= \int f(s) \, d\mu(s)  & & \text{by Expectation Rule} \\
&= \cexpectationlong{\mathcal{F}}{f(\xi)} & & \text{by Theorem \ref{Disintegration}}
\end{align*}
Now apply the strictly convex case of Jensen's Inequality to conclude
the result.

If we assume finite second moments then there should be a proof of
this by showing that the conditional variance is $0$.  TODO: Define
conditional variance and show the result.
\end{proof}

\begin{xca}Prove or disprove the following statement.  Suppose $\xi \eqdist \eta$, show that for every $A$, 
  $\cprobability{\mathcal{F}}{\xi \in A} =
  \cprobability{\mathcal{F}}{\eta \in A}$ a.s.
\end{xca}
\begin{proof}
This is false.  Let $\Omega = \lbrace 0,1 \rbrace$ with uniform distribution and
power set $\sigma$-algebra.  Let
$\xi(x) = x$ and let $\eta(x) = 1 - x$.  Note that $\xi \eqdist
\eta$ (both have a uniform distribution on $\lbrace 0,1 \rbrace$). Now take $\mathcal{F} = \mathcal{A}$ so that
$\cprobability{\mathcal{F}}{\xi \in A} = \characteristic{\xi \in A}$
and $\cprobability{\mathcal{F}}{\eta \in A} = \characteristic{\eta \in
  A}$ and take $A = \lbrace 0 \rbrace$ or $A = \lbrace 1 \rbrace$.
\end{proof}

\begin{xca}Find $\xi, \eta, \mathcal{F}$ such that $\xi \eqdist \eta$
  but $\cexpectationlong{\mathcal{F}}{\xi} \neq
  \cexpectationlong{\mathcal{F}}{\eta}$ a.s.
\end{xca}
\begin{proof}
Pick sets $A,B,C$ such that $\probability{A} = \probability{B}$ but
$\probability{A \cap C} \neq \probability{B \cap C}$.  Even more
trivially, take $\mathcal{F} = \mathcal{A}$ so that
$\cexpectationlong{\mathcal{F}}{\xi} = \xi$ and similarly with
$\eta$.  Now the statement is equivalent to show two random elements
that not almost surely equal but have the same distribution.
\end{proof}

\begin{xca}Suppose $\xi, \tilde{\xi}$ are integrable random variables
  and $\eta, \tilde{\eta}$ are random elements in $(T, \mathcal{T})$
  such that $(\xi,\eta) \eqdist (\tilde{\xi}, \tilde{\eta})$.  Show
  that $\cexpectationlong{\eta}{\xi} \eqdist \cexpectationlong{\tilde{\eta}}{\tilde{\xi}}$.
\end{xca}
\begin{proof}
First, note the intuition behind the statement.  As a result of
$(\xi,\eta) \eqdist (\tilde{\xi}, \tilde{\eta})$ we can also conclude
that $\xi \eqdist \tilde{\xi}$ and $\eta \eqdist \tilde{\eta}$.
However, we also expect that the conditional distributions on $T$ are
equal (thinking heuristically of a formula like $\cprobability{B}{A} =
\probability{A \cap B} / \probability{B}$).  The first order of
business is to formulate this intuition precisely and prove it.

By Theorem \ref{ExistenceConditionalDistribution} there are
probability kernels $\mu$ and $\tilde{\mu}$ such that
$\cprobability{\eta}{\xi \in A} = \mu(\eta, A)$ and
$\cprobability{\tilde{\eta}}{\tilde{\xi} \in A} =
\tilde{\mu}(\tilde{\eta}, A)$ for all Borel sets $A$.  Our first claim
is that $\mu = \tilde{\mu}$ almost surely with respect to
$\mathcal{L}{\eta}$.

Pick a Borel set $A$ and let $B = \lbrace t \in T \mid \mu(t, A) >
\tilde{\mu}(t, A) \rbrace$. 
\begin{align*}
0 &= \probability{\xi \in A; \eta \in B} - \probability{\tilde{\xi}
  \in A; \tilde{\eta} \in B} & \text{by hypothesis}\\
&=\expectation{\int \characteristic{A \times B}(s, \eta) \,
  d\mu(\eta, s) - \int \characteristic{A \times B}(s, \tilde{\eta}) \,
  d\tilde{\mu}(\eta, s)} & \text{by Theorem \ref{Disintegration}}\\
&=\expectation{\characteristic{B}(\eta) \mu(\eta, A) -
  \characteristic{B}(\tilde{\eta}) \tilde{\mu}(\tilde{\eta}, A)} \\
&=\int \characteristic{B}(t) \mu(t, A) -
  \characteristic{B}(t) \tilde{\mu}(t, A) \, d\mathcal{L}(\eta)(t) &
  \text{by Lemma \ref{ChangeOfVariables} and $\mathcal{L}(\eta) = \mathcal{L}(\tilde{\eta})$.}
\end{align*}
which by choice of $B$ shows that $\mu(t, A) =\tilde{\mu}(t, A)$
almost surely $\mathcal{L}(\eta)$.  We can show this almost surely for all $A =
(-\infty, r]$ with $r \in \rationals$ by taking the union of a
countable number of null sets.  This shows that $\mu = \tilde{\mu}$
a.s.

Having shown equality of the conditional distributions it follows from
Theorem \ref{Disintegration} that if we define $f(t) = \int s \, d\mu(t,
s)$ then we have $\cexpectationlong{\eta}{\xi} = f(\eta)$ and
$\cexpectationlong{\tilde{\eta}}{\tilde{\xi}} = f(\tilde{\eta})$.
Since $\eta \eqdist \tilde{\eta}$ it follows that $f(\eta) \eqdist
f(\tilde{\eta})$ and the result is proven.
\end{proof}

\begin{xca}Suppose $\xi$ is a random element in a Borel space $(S, \mathcal{S})$, let
  $\mathcal{F}$ be a $\sigma$-algebra and let $\eta =
  \cprobability{\mathcal{F}}{\xi \in \cdot}$, show $\cindependent{\xi}{\mathcal{F}}{\eta}$.
\end{xca}
\begin{proof}
First it is worth clarifying the question.  Since we have assume $S$
is Borel then by Theorem \ref{ExistenceConditionalDistribution} be may
assume that $\eta$ is an $\mathcal{F}$-measurable random measure on
$S$.  We are asked to show conditional independence of $\xi$ and
$\mathcal{F}$ relative to this random measure.  Conceptually, the
conditional distribution captures all of the dependence between a
random element and a $\sigma$-algebra (think of the case $\mathcal{F}
= \sigma(\zeta)$ for a random element $\zeta$ to make this even more concrete).  

By Lemma \ref{ConditionalIndependenceDoob} it will suffice to show for
every $A \in \mathcal{S}$, 
\begin{align*}
\cexpectationlong{\eta}{\xi \in  A} 
&= \cexpectationlong{\eta,\mathcal{F}}{\xi \in  A} 
= \cexpectationlong{\mathcal{F}}{\xi \in  A} 
\end{align*}
where the last equality follows from the $\mathcal{F}$-measurability
of $\eta$.  However this is easily verified since the $\sigma$-algebra
on the space of probability measures $\mathcal{P}(S)$ is the smallest $\sigma$-algebra that makes
evaluation maps $ev_B(\mu) = \mu(B)$ measurable (here $B \in \mathcal{S}$).  Thus we have by
definition of $\eta$, $\cexpectationlong{\mathcal{F}}{\xi \in  A} =
ev_A(\eta)$ which shows that $\cexpectationlong{\mathcal{F}}{\xi \in
  A}$ is in fact $\eta$-measurable.
\end{proof}

\begin{xca}Suppose $\cindependent{\xi}{\zeta}{\eta}$ and
  $\cindependent{\gamma}{(\xi,\eta, \zeta)}{}$, show that
  $\cindependent{\xi}{\zeta}{\eta,\gamma}$ and $\cindependent{\xi}{(\zeta,\gamma)}{\eta}$.
\end{xca}
\begin{proof}
By Lemma \ref{ConditionalIndependenceChainRule},
$\cindependent{\xi}{(\zeta,\gamma)}{\eta}$ is equivalent to
$\cindependent{\xi}{\zeta}{\eta}$ and
$\cindependent{\xi}{\gamma}{\eta, \zeta}$.  The fact that
$\cindependent{\xi}{\zeta}{\eta}$ is a hypothesis whereas
$\cindependent{\xi}{\gamma}{\eta, \zeta}$ follows from another
application of Lemma \ref{ConditionalIndependenceChainRule} to show
that $\cindependent{\gamma}{(\xi,\eta, \zeta)}{}$ is equivalent to
$\cindependent{\gamma}{\zeta}{}$ and
$\cindependent{\gamma}{\eta}{\zeta}$ and
$\cindependent{\gamma}{\xi}{\zeta, \eta}$

Now by Lemma \ref{ConditionalIndependenceChainRule} 
we know $\cindependent{\xi}{(\gamma, \zeta)}{\eta}$ is equivalent to
$\cindependent{\xi}{\gamma}{\eta}$ and
$\cindependent{\xi}{\zeta}{\eta, \gamma}$
hence implies $\cindependent{\xi}{\zeta}{\eta, \gamma}$.
\end{proof}

\begin{xca}Suppose we have $\sigma$-algebras $\mathcal{F}$,
  $\mathcal{G}_1$, $\mathcal{G}_2$, $\mathcal{H}$ with $\mathcal{G}_1
  \subset \mathcal{G}_2$.  If
  $\cindependent{\mathcal{F}}{\mathcal{H}}{\mathcal{G}_1}$
is it true that $\cindependent{\mathcal{F}}{\mathcal{H}}{\mathcal{G}_2}$?
Prove or give a counterexample.
\end{xca}
\begin{proof}
Here is a counterexample in which $\mathcal{G}_1$ is the trivial
$\sigma$-algebra.  Perform two independent Bernoulli trials with rate
$1/2$.  Thus we have sample space $\Omega = \lbrace HH, HT, TT, TH
\rbrace$ with the uniform distribution.  Let $A = \lbrace HH, HT
\rbrace$ (and let $\mathcal{F} = \lbrace \emptyset, \Omega, A, A^c
\rbrace$) and let $B = \lbrace HT, TT \rbrace$ (and let $\mathcal{H} =
\lbrace \emptyset , \Omega, B, B^c \rbrace$).  Note that $A$ and $B$
are independent.  Now let $C = \lbrace HH, TT \rbrace$ (and let
$\mathcal{G}_2 = \lbrace \emptyset, \Omega, C, C^c \rbrace$ and note
that $A$ and $B$ are not conditionally independent given $C$ because
$\cprobability{C}{A \cap B} = 0$ whereas $\cprobability{C}{A} = 1/2$
and $\cprobability{C}{B} = 1/2$.

Note
that primary conceptual point here is that given two independent
events (here ``first toss is heads'' and ``second toss is tails'') one
can condition that there is a relationship between them (here ``first toss
equals the second toss'') and destroy independence.  
\end{proof}

\begin{xca}Suppose $\mathcal{F}$ is independent of $\mathcal{G}$ and
  $\mathcal{H}$, is it true that $\mathcal{F}$ is independent of
  $\sigma(\mathcal{G}, \mathcal{H})$?  Prove or give a counterexample.
\end{xca}
\begin{proof}
Note that $\mathcal{F}$ is independent of
  $\sigma(\mathcal{G}, \mathcal{H})$ if and only if
  $\cindependent{\mathcal{F}}{\mathcal{G}}{}$
and $\cindependent{\mathcal{F}}{\mathcal{H}}{\mathcal{G}}$.  Because
of this equivalence the previous exercise is a counterexample here as
well.  Using the
notation of the previous exercise, let $\mathcal{F} = \sigma(A)$ and
let $\mathcal{G} = \sigma(C)$ and note that $A$ and $C$ are
independent
by direct calculation (this is also intuitively clear).  We also saw
in the previous exercise that $A$ and $B$ are independent and that $A$
is not conditionally independent of $B$ given $C$; hence $A$ is not
indepndent of $\sigma(B,C)$.

Note that we can also show this directly without using the Lemma. A
little work shows that $\sigma(B,C) = 2^\Omega$; it suffices to
note that $B \cap C = \lbrace TT \rbrace$,  $B^c \cap C^c = \lbrace TH
\rbrace$, $B \cap C^c = \lbrace HT \rbrace$ and $B^c \cap C = \lbrace HH
\rbrace$.  Given this fact
it is easy to see that $A$ is not independent of $\sigma(B,C)$ by noting that, because
$P(A) = 1/2$, it is not independent of itself.

Note also that the the key to the failure here is the fact that $A$, $B$
and $C$ are not jointly independent (they are pairwise independent),
otherwise we could appeal to Lemma \ref{IndependenceGrouping}.
To see the lack of joint independence consider $\probability {A \cap B
\cap C} = 0$.
\end{proof}

\begin{xca}Suppose we are given random elements such that $(\xi, \eta,
  \zeta) \eqdist (\tilde{\xi}, \tilde{\eta}, \tilde{\zeta})$, then
  $\cindependent{\xi}{\zeta}{\eta}$ if and only if $\cindependent{\tilde{\xi}}{\tilde{\zeta}}{\tilde{\eta}}$.
\end{xca}
\begin{proof}
First we 
\begin{align*}
\
\end{align*}
\end{proof}

\begin{xca}Suppose $\tau$ and $\sigma$ are discrete optional
  times with respect the filtration $\mathcal{F}_0 \subset
  \mathcal{F}_1 \subset \cdots$. Then $\sigma \wedge \tau$ and
  $\sigma$ and $\sigma \vee \tau$ are optional times.  In addition, 
\begin{align*}
\mathcal{F}_{\tau \wedge
    \sigma} &\subset \mathcal{F}_\sigma \subset \mathcal{F}_{\tau \vee
    \sigma}
\end{align*}
\end{xca}
\begin{proof}
First we show that $\tau \wedge \sigma$ and $\tau \vee \sigma$ are
actually optional times.  This is simple by noting
\begin{align*}
\lbrace \tau \wedge \sigma \leq n \rbrace = \lbrace \tau \leq n
\rbrace \cup \lbrace \sigma \leq n \rbrace \in \mathcal{F}_n
\end{align*}
and
\begin{align*}
\lbrace \tau \vee \sigma \leq n \rbrace = \lbrace \tau \leq n
\rbrace \cap \lbrace \sigma \leq n \rbrace \in \mathcal{F}_n
\end{align*}
If we are given $A \in \mathcal{F}_\sigma$ the by definition for all
$n$, $A \cap \lbrace \sigma \leq n \rbrace \in \mathcal{F}_n$.
Therefore since by definition of optional time we also have $\lbrace
\tau \leq n \rbrace \in \mathcal{F}_n$ we have
\begin{align*}
A \cap \lbrace \tau \vee \sigma \leq n \rbrace &= (A \cap \lbrace \sigma \leq n
\rbrace) \cap \lbrace \tau \leq n \rbrace \in \mathcal{F}_n
\end{align*}
which shows $A \in \mathcal{F}_{\sigma \vee \tau}$.

Now if we assume that $A \in \mathcal{F}_{\sigma \wedge \tau}$, then
for all $n$ we have
\begin{align*}
A \cap \lbrace \tau \wedge \sigma \leq n \rbrace &= A \cap \lbrace \tau \leq n
\rbrace \cup A \cap \lbrace \sigma \leq n \rbrace \in \mathcal{F}_n
\end{align*}
Since we have $\lbrace \sigma \leq n \rbrace, \lbrace \tau \leq n
\rbrace \in \mathcal{F}_n$, then we know that $\lbrace \tau \leq n
\rbrace \setminus \lbrace \sigma \leq n\rbrace \in \mathcal{F}_n$ and
so 
\begin{align*}
\left( A \cap \lbrace \tau \leq n
\rbrace \right ) \cup \left (A \cap \lbrace \sigma \leq n \rbrace
\right ) \cup \left ( \lbrace \tau \leq n
\rbrace \setminus \lbrace \sigma \leq n\rbrace\right)^c &= A \cap \lbrace \sigma \leq n
\rbrace \in \mathcal{F}_n
\end{align*}
which shows $A \in \mathcal{F}_\sigma$.
\end{proof}

\begin{xca}Suppose $\tau$ is a discrete optional
  time with respect the filtration $\mathcal{F}_0 \subset
  \mathcal{F}_1 \subset \cdots$, then $\tau$ is $\mathcal{F}_\tau$-measurable.
\end{xca}
\begin{proof}
For any $n, m$, we have 
\begin{align*}
\lbrace \tau = m \rbrace  \cap \lbrace \tau \leq n \rbrace &=
\begin{cases}
\emptyset & \text{if $m > n$} \\
\lbrace \tau = m \rbrace & \text{if $m\leq n$}
\end{cases}
\end{align*}
hence in all cases is in $\mathcal{F}_n$.
\end{proof}

\begin{xca}Suppose $\tau$ and $\sigma$ are discrete optional
  times with respect the filtration $\mathcal{F}_0 \subset
  \mathcal{F}_1 \subset \cdots$. Then each of $\lbrace \sigma < \tau
  \rbrace$, $\lbrace \sigma \leq \tau
  \rbrace$ and $\lbrace \sigma = \tau
  \rbrace$ is in $\mathcal{F}_{\sigma} \cap \mathcal{F}_{\tau}$.
\end{xca}
\begin{proof}
It suffice to prove two of the three since the third set can be
constructed using finite unions or intersections of the other two.
First we show that $\lbrace \sigma < \tau \rbrace \in
\mathcal{F}_\tau$.
Pick an $n$ and we calculate
\begin{align*}
\lbrace \sigma < \tau \rbrace \cap \lbrace \tau \leq n \rbrace &=
\cup_{m\leq n}\lbrace \sigma < \tau \rbrace \cap \lbrace \tau = m
\rbrace \\
&= \cup_{m\leq n}\lbrace \sigma < m \rbrace \cap \lbrace \tau = m
\rbrace \\
\end{align*}
Now each $\lbrace \sigma < m \rbrace \in \mathcal{F}_m \subset
\mathcal{F}_n$ and each $\lbrace \tau = m
\rbrace \in \mathcal{F}_m \subset
\mathcal{F}_n$ by definition of optional time so the union is and we
have shown $\lbrace \sigma < \tau \rbrace \in \mathcal{F}_\tau$.  The
same argument clearly shows that the other sets are in
$\mathcal{F}_\tau$ as well.  To see that all sets are in
$\mathcal{F}_\sigma$, it suffices to note for example that 
\begin{align*}
\lbrace \sigma < \tau \rbrace^c &= \lbrace \tau \leq \sigma \rbrace
\in \mathcal{F}_\sigma
\end{align*}
by what we have already proven. Apply the closure of $\sigma$-algebras
under complement to get the result.
\end{proof}

\begin{xca}Let $\sigma$ and $\tau$ be optional times with respect to
  the filtration $\mathcal{F}_0 \subset \mathcal{F}_1 \subset
  \cdots$.  Show that 
\begin{align*}
\cexpectationlong
{\mathcal{F}_\tau}{\cexpectationlong{\mathcal{F}_\sigma}{\xi}} &=
\cexpectationlong
{\mathcal{F}_\sigma}{\cexpectationlong{\mathcal{F}_\tau}{\xi}} = \cexpectationlong
{\mathcal{F}_{\sigma \wedge \tau}}{\xi}
\end{align*}
\end{xca}
\begin{proof}
The first thing to do is show how to calculate conditional
expectations with respect to $\sigma$-algebras of the form
$\mathcal{F}_\sigma$ for an arbitrary optional time $\sigma$.  Given
an integrable random variable $\xi$ we let $M^\xi_n =
\cexpectationlong{\mathcal{F}_n}{\xi}$ be the martingale generated by
$\xi$.  We claim
\begin{align*}
\cexpectationlong{\mathcal{F}_\sigma}{\xi} &= M^\xi_\sigma
\end{align*}
TODO:  Dude, this is just optional stopping (at least for the
uniformly integrable case); is that supposed to be available?
To see this, pick an $A \in \mathcal{F}_\sigma$ and then note that for
every $n$, use the fact that $A \cap \lbrace \sigma = n \rbrace \in
\mathcal{F}_n$ and the telescoping rule for conditional expectation to see
\begin{align*}
\expectation{\characteristic{A} \characteristic{\lbrace \sigma = n
    \rbrace } \xi} &= 
\expectation{\characteristic{A} \characteristic{\lbrace \sigma = n
    \rbrace } \cexpectationlong{\mathcal{F}_n}{\xi} } = 
\expectation{\characteristic{A} \cexpectationlong{\mathcal{F}_n}{\characteristic{\lbrace \sigma = n
    \rbrace } \xi} } 
\end{align*}
which is easy to extend by linearity 
\begin{align*}
\expectation{\characteristic{A} \xi} &= \sum_{n=0}^\infty \expectation{\characteristic{A} \characteristic{\lbrace \sigma = n
    \rbrace } \xi} = \sum_{n=0}^\infty \expectation{\characteristic{A} \cexpectationlong{\mathcal{F}_n}{\characteristic{\lbrace \sigma = n
    \rbrace } \xi}} = \expectation{\characteristic{A}
  \sum_{n=0}^\infty \cexpectationlong{\mathcal{F}_n}{\characteristic{\lbrace \sigma = n
    \rbrace } \xi}} \\
&= \expectation{\characteristic{A} M^\xi_\sigma}
\end{align*}
Using this formula twice we have
\begin{align*}
\cexpectationlong{\mathcal{F}_\sigma}{\cexpectationlong{\mathcal{F}_\tau}{\xi}}
&=
\cexpectationlong{\mathcal{F}_\sigma}{M^\xi_\tau}
\\
&= \sum_{n=0}^\infty \characteristic{\lbrace \sigma = n \rbrace}
\cexpectationlong{\mathcal{F}_n}{M^\xi_\tau} \\
&= \sum_{n=0}^\infty \sum_{m=0}^\infty \characteristic{\lbrace \sigma = n \rbrace}
\cexpectationlong{\mathcal{F}_n}{\cexpectationlong{\mathcal{F}_m}{\characteristic{
    \lbrace \tau = m \rbrace } \xi}} \\
\end{align*}
Now consider each term $\characteristic{\lbrace \sigma = n \rbrace}
\cexpectationlong{\mathcal{F}_n}{\cexpectationlong{\mathcal{F}_m}{\characteristic{
    \lbrace \tau = m \rbrace } \xi}}$; there are two cases. If $m \leq n$
then since $\mathcal{F}_m \subset \mathcal{F}_n$ we can write
\begin{align*}
\characteristic{\lbrace \sigma = n \rbrace}
\cexpectationlong{\mathcal{F}_n}{\cexpectationlong{\mathcal{F}_m}{\characteristic{
    \lbrace \tau = m \rbrace } \xi}} &= \characteristic{\lbrace \sigma = n \rbrace}\cexpectationlong{\mathcal{F}_m}{\characteristic{
    \lbrace \tau = m \rbrace } \xi} = \characteristic{\lbrace \sigma = n \rbrace}\characteristic{
    \lbrace \tau = m \rbrace }\cexpectationlong{\mathcal{F}_m}{ \xi} 
\end{align*}
If $n \leq m$ then because $\mathcal{F}_n \subset \mathcal{F}_m$ and
the telescoping rule,
\begin{align*}
\characteristic{\lbrace \sigma = n \rbrace}
\cexpectationlong{\mathcal{F}_n}{\cexpectationlong{\mathcal{F}_m}{\characteristic{
    \lbrace \tau = m \rbrace } \xi}} &= 
\cexpectationlong{\mathcal{F}_n}{\cexpectationlong{\mathcal{F}_m}{\characteristic{\lbrace
      \sigma = n \rbrace} \characteristic{
    \lbrace \tau = m \rbrace } \xi}}  = \cexpectationlong{\mathcal{F}_n}{\characteristic{\lbrace \sigma = n \rbrace}\characteristic{
    \lbrace \tau = m \rbrace } \xi} 
\end{align*}
These two forms are a bit different and are not equivalent because we
cannot ascertain the $\mathcal{F}_{m \wedge n}$-measurability of $\characteristic{\lbrace \sigma = m \rbrace}\characteristic{
    \lbrace \tau = m \rbrace }$.  However, we do know that $\lbrace
  \sigma > m \rbrace = \lbrace \sigma \leq m \rbrace^c$ is
  $\mathcal{F}_m$-measurable and $\lbrace
  \tau > n \rbrace = \lbrace \tau \leq n \rbrace^c$ is
  $\mathcal{F}_n$-measurable.  So if we sum using the case $n \leq m$,
  we get,
\begin{align*}
\sum_{m>n} \characteristic{\lbrace \sigma = n \rbrace}
\cexpectationlong{\mathcal{F}_n}{\cexpectationlong{\mathcal{F}_m}{\characteristic{
    \lbrace \tau = m \rbrace } \xi}} 
&= \sum_{m>n} \cexpectationlong{\mathcal{F}_n}{\characteristic{\lbrace \sigma = n \rbrace}\characteristic{
    \lbrace \tau = m \rbrace } \xi} \\
&= \cexpectationlong{\mathcal{F}_n}{\characteristic{\lbrace \sigma = n \rbrace}\characteristic{
    \lbrace \tau > n \rbrace } \xi} \\
&=\characteristic{\lbrace \sigma = n \rbrace}\characteristic{
    \lbrace \tau > n \rbrace }  \cexpectationlong{\mathcal{F}_n}{\xi} \\
&= \sum_{m>n} \characteristic{\lbrace \sigma = n \rbrace}\characteristic{
    \lbrace \tau=m \rbrace }  \cexpectationlong{\mathcal{F}_n}{\xi} 
\end{align*}
So this shows us how to get everything into a common form if we break
up the sum properly, 
\begin{align*}
\cexpectationlong{\mathcal{F}_\sigma}{\cexpectationlong{\mathcal{F}_\tau}{\xi}} 
&= \sum_{n=0}^\infty \sum_{m=n+1}^\infty \characteristic{\lbrace \sigma = n \rbrace}
\cexpectationlong{\mathcal{F}_n}{\cexpectationlong{\mathcal{F}_m}{\characteristic{
    \lbrace \tau = m \rbrace } \xi}} + \\
&\sum_{m=0}^\infty \sum_{n=m}^\infty \characteristic{\lbrace \sigma = n \rbrace}
\cexpectationlong{\mathcal{F}_n}{\cexpectationlong{\mathcal{F}_m}{\characteristic{
    \lbrace \tau = m \rbrace } \xi}}  \\
&=\sum_{n=0}^\infty \sum_{m=n+1}^\infty \characteristic{\lbrace \sigma = n \rbrace}\characteristic{
    \lbrace \tau=m \rbrace }  \cexpectationlong{\mathcal{F}_n}{\xi} + \\
&\sum_{m=0}^\infty \sum_{n=m}^\infty \characteristic{\lbrace \sigma = n \rbrace}\characteristic{
    \lbrace \tau=m \rbrace }  \cexpectationlong{\mathcal{F}_m}{\xi}
  \\
&= \sum_{m=0}^\infty \sum_{n=0}^\infty \characteristic{\lbrace \sigma = n \rbrace}\characteristic{
    \lbrace \tau=m \rbrace }  \cexpectationlong{\mathcal{F}_{m\wedge
      n}}{\xi} \\
&= M^\xi_{\sigma \wedge \tau} = \cexpectationlong{\mathcal{F}_{\sigma
    \wedge \tau}}{\xi}
\end{align*}
\end{proof}

\begin{xca}\label{SumOfOptionalTimes}Let $\sigma$ and $\tau$ be $\mathcal{F}$-optional times on
  either $\integers_+$ or $\reals_+$.
  Show that $\sigma + \tau$ is $\mathcal{F}$-optional.
\end{xca}
\begin{proof}
For the case of $\integers_+$ valued optional times we pick $n \geq 0$
and note that
\begin{align*}
\lbrace \sigma + \tau = n \rbrace = \cup_{m=0}^n \lbrace \sigma = m
\rbrace \cap \lbrace \tau = n-m \rbrace
\end{align*}
which is in $\mathcal{F}_n$ since for $0 \leq m \leq n$ we have
$\lbrace \sigma = m \rbrace \in \mathcal{F}_m \subset \mathcal{F}_n$
and $\lbrace \tau = n - m \rbrace \in \mathcal{F}_{n-m} \subset
\mathcal{F}_n$.

Pick $t \in $ and note that it suffices to show
\begin{align*}
\lbrace \sigma + \tau > t \rbrace &= \lbrace \sigma > t \rbrace \cup \cup_{\substack{q < t \\ q \in
    \rationals}} \lbrace \sigma > q \rbrace \cap \lbrace \tau > t - q \rbrace
\end{align*}
by reasoning similar to the discrete case.  To see this equality for
one inclusion note
that for all $q \in \rationals$ we have $\lbrace \sigma > q \rbrace
\cap \lbrace \tau > t - q \rbrace \subset \lbrace \sigma + \tau > t
\rbrace$.  By positivity of $\tau$ we know that $ \lbrace \sigma > t
\rbrace \subset \lbrace \sigma + \tau > t \rbrace $.

For the other inclusion suppose $\sigma(\omega) + \tau(\omega) > t$.
If $\sigma(\omega) \leq t$ then by density of rationals we can pick $q \in
\rationals$ such that $t - \tau(\omega) < q < \sigma(\omega) \leq t$ and
we have $\omega \in \lbrace \sigma > q \rbrace \cap \lbrace \tau > t -
q\rbrace$.  If $\sigma(\omega) > t$ then it follows that $\omega \in
\lbrace \sigma > t \rbrace$ so we are done.
\end{proof}

\begin{xca}Show that a random variable $\xi$ has subexponential tails
  if and only if there exists $C > 0$ such that
  $\expectation{\abs{\xi}^k} \leq C k^C$ for all integers $k > 0$.
\end{xca}
\begin{proof}
TODO: Mimic the proof of Lemma \ref{SubgaussianEquivalence}.
\end{proof}

\begin{xca}Let $X$ be a right continuous
  submartingale then almost surely $X$ is cadlag.
\end{xca}
\begin{proof}
By Theorem \ref{CadlagModificationContinuousMartingale} we know that
there is a null set $A$ such that
the process $Z_t = \characteristic{A^c} \lim_{q \to t^+}X_q$ is a
cadlag process (in fact a cadlag $\overline{\mathcal{F}}^+$-submartingale).
As $X$ is almost surely right continuous, it follows that almost surely $Z = X$ and we
conclude that almost surely $X$ has cadlag sample paths.
\end{proof}

\begin{xca}Suppose we are given $\sigma$-algebras $\mathcal{G},
  \mathcal{H}, \mathcal{F}_1, \mathcal{F}_2, \dotsc$ and define
  $\mathcal{F}_\infty = \bigvee_n \mathcal{F}_n$.  If
  $\cindependent{\mathcal{G}}{\mathcal{H}}{\mathcal{F}_n}$ for all $n
  \in \naturals$ then $\cindependent{\mathcal{G}}{\mathcal{H}}{\mathcal{F}_\infty}$.
\end{xca}
\begin{proof}
By definition of conditional independence we see that for every $G \in
\mathcal{G}$ and $H \in \mathcal{H}$ we have
$\cprobability{\mathcal{F}_n}{G \cap H} =
\cprobability{\mathcal{F}_n}{G} \cprobability{\mathcal{F}_n}{H}$.  By
Levy-Jessen Theorem \ref{JessenConditioningLimits} we conclude
\begin{align*}
\cprobability{\mathcal{F}_\infty}{G \cap H} &=  \lim_{n \to \infty} \cprobability{\mathcal{F}_n}{G \cap H} =
 \lim_{n \to \infty}\cprobability{\mathcal{F}_n}{G}
 \cprobability{\mathcal{F}_n}{H} = \cprobability{\mathcal{F}_\infty}{G}
 \cprobability{\mathcal{F}_\infty}{H}
\end{align*}
which shows the result.
\end{proof}

\begin{xca}Let $B_t$ be a standard Brownian motion and define $\tau =
  \inf \lbrace t > 0 \mid B_t = 1 \rbrace$.  Show that
  $B_\tau = 1$ almost surely and $\expectation{\tau^c} < \infty$ for
  all $0 \leq c < 1/2$.
\end{xca}
\begin{proof}
Note that $\tau < \infty$ almost surely since $\limsup_{t \to \infty}
B_t = \infty$ almost surely and $B_t$ is continuous.  For any $\lambda
\geq 0$ note that $ \lbrace \tau \geq t \rbrace = \lbrace
\sup_{0 \leq s \leq t} B_s \leq 1$.  Since the law of $\sup_{0 \leq s
  \leq t} B_s$ is the same as the law of $\abs{B_t}$ by Lemma
\ref{TailsAndExpectations} we get
\begin{align*}
\expectation{\tau^c} &= c^{-1} \int_0^\infty t^{c-1} \probability{\tau
  \geq t} \, dt = \frac{2}{c \sqrt{2\pi}} \int_0^\infty
\int_0^{1/\sqrt{t}}  t^{c-1} e^{-x^2/2} \, dx \, dt \\
&= \frac{2}{c \sqrt{2\pi}} \int_0^\infty
\int_0^{1/x^2}  t^{c-1} e^{-x^2/2} \, dt \, dx = \frac{2}{\sqrt{2\pi}}
\int_0^\infty x^{-2c} e^{-x^2/2} \, dx 
\end{align*}
For $0 \leq c < 1/2$ an integration by parts shows that this integral
is finite.

Note also that integration by parts also shows that
$\expectation{\tau^c} = \infty$ for $c \geq 1/2$ (as we know must be
true because of the BDG/Optional Stopping argument above).
\end{proof}

\begin{xca}Let $B_t$ be a standard Brownian motion show that for every
  $c \in \reals$ the process $M_t = e^{c B_t - \frac{c^2t}{2}}$ is a martingale.
\end{xca}
\begin{proof}
Adaptedness follows from the fact that $B_t$ is
$\mathcal{F}_t$-measurable and $e^x$ is continuous hence Borel
measurable.  First to see that $M_t$ is integrable we compute by Lemma
\ref{ExpectationRule} and completing the
square
\begin{align*}
\expectation{e^{cB_t}} &= \frac{1}{\sqrt{2\pi t}}
\int_{-\infty}^\infty e^{c x} e^{-x^2/2t} \, dx = \frac{ e^{c^2t/2}}{\sqrt{2\pi t}}
\int_{-\infty}^\infty e^{-(x-ct)^2/2t} \, dx = e^{c^2t/2} < \infty
\end{align*}
If we take $0 \leq s < t < \infty$ then using the pullout rule of
conditional expectation, the fact that $B_t -
B_s$ is independent of $\mathcal{F}_s$ and the above computation of
the expectation to see that
\begin{align*}
\cexpectationlong{\mathcal{F}_s}{e^{c B_t - \frac{c^2t}{2}}}
&=e^{- \frac{c^2t}{2}}\cexpectationlong{\mathcal{F}_s}{e^{c (B_t -B_s)} e^{cB_s} } 
= e^{- \frac{c^2t}{2}}\expectation{e^{c (B_t -B_s)}} e^{cB_s} \\
&= e^{- \frac{c^2t}{2} }e^{\frac{c^2(t-s)}{2}}  e^{cB_s} 
= e^{c B_s - \frac{c^2s}{2}}
\end{align*}
\end{proof}

\begin{xca}Let $B_t$ be a standard Brownian motion, show that $\inf
  \lbrace t > 0 \mid B_t > 0 \rbrace = 0$ a.s.  (Hint: Use
  Blumenthal's 0-1 Law).
\end{xca}
\begin{proof}
Let $\tau = \inf \lbrace t > 0 \mid B_t > 0 \rbrace$.  Clearly the
event 
\begin{align*}
\lbrace \tau = 0 \rbrace &= \cap_{n=1}^\infty \cup_{\substack{0 < q <
    1/n\\ q \in \rationals}} \lbrace B_q > 0 \rbrace
\end{align*}
is  $\mathcal{F}^+_0$-measurable so by Lemma
  \ref{Blumenthal01LawBrownianMotion} we know that it has probability
  0 or 1.  It therefore suffices to show that $\probability { \tau =
    0} \neq 0$.  To see this note that for each $s > 0$ we have
$\lbrace \tau \leq s \rbrace \supset \lbrace B_s > 0 \rbrace$ hence
$\probability{\tau \leq s} \geq 1/2$ and by continuity of measure we
know that $\probability {\tau = 0} = \lim_{s \downarrow 0} \probability{\tau \leq s} \geq 1/2$.
\end{proof}

\begin{xca}[Law of Large Numbers for Brownian Motion]Let $B_t$ be a standard Brownian motion show that $M_t =
  t^{-1}B_t$ is a backward martingale.  From this conclude that
  $t^{-1}B_t \toas 0$ and $t^{-1}B_t \tolp{p} 0$ for all $p > 0$.
\end{xca}
\begin{proof}
Adaptedness and integrability are immediate.  For the backward
martingale property, let $s < t$ and we first find the density function for the
conditional distribution $\cprobability{B_t}{B_s \in \cdot}$.  To find
the joint density $(B_s, B_t)$ we note that $(B_s,B_t) = (x,y)$ if and
only if $(B_s,B_t - B_s) = (x, y - x)$ so by the independence of $B_s$
and $B_t - B_s$ and completing the square we get
\begin{align*}
\probability{B_s = x ; B_t = y} &= \frac{1}{\sqrt{2\pi s} } e^{-x^2/2s}
  \frac{1}{\sqrt{2\pi(t- s)}} e^{-(y-x)^2/2(t-s)} \\
&= \frac{1}{\sqrt{2\pi s}} \frac{1}{\sqrt{2\pi(t- s)}}
    e^{-\frac{t}{2s(t-s)}(x - \frac{s}{t}y)^2} e^{-y^2/2t}
\end{align*}
So we see that the conditional density $B_s$ given $B_t$ is Gaussian
with mean $\frac{s}{t} B_t$.  Thus $\cexpectationlong{B_t}{s^{-1} B_s}
= t^{-1} B_t$.  By the extended Markov property (Lemma
\ref{ExtendedMarkovProperty}) we know that $\cindependent{B_s}{\bigvee_{u \geq t}\sigma(
    B_u)}{B_t}$ and therefore $\cexpectationlong{B_t}{s^{-1} B_s} = \cexpectationlong{\bigvee_{u \geq t}\sigma(
    B_u)}{s^{-1} B_s}$ (Lemma \ref{ConditionalIndependenceDoob}) which
  shows the backward martingale property.

Now we need to show that $\cap_{t > 0} \bigvee_{u \geq t} \sigma(B_u)$
is a trivial $\sigma$-field (Lemma \ref{Blumenthal01LawBrownianMotion});
from that it follows that for all $t > 1$, 
\begin{align*}
t^{-1} B_t &= \cexpectationlong{\bigvee_{u \geq t} \sigma(B_u)}{B_1} 
\end{align*}
and by Jessen-Levy and triviality we have 
\begin{align*}
\cexpectationlong{\bigvee_{u \geq t}
  \sigma(B_u)}{B_1} \toas \cexpectationlong{\cap_{t>0} \bigvee_{u \geq t}
  \sigma(B_u)}{B_1} = \expectation{B_1} = 0
\end{align*}

TODO: Finish the $L^p$ argument; presumably we need $L^p$ boundedness.
\end{proof}

\begin{xca}[Kallenberg Exercise 13.19]\label{BrownianBridgeMartingale}Let $X$ be a Brownian bridge,
  show that $(1-t)^{-1}X_t$ is a martingale with respect to the
  induced filtration and is not $L^1$ bounded on $[0,1)$.
\end{xca}
\begin{proof}
Since $X_t$ is Gaussian with variance $t -t^2$ we have 
\begin{align*}
\frac{1}{1-t} \expectation{\abs{X_t}} &= \frac{1}{1-t}\frac{2}{\sqrt{2\pi(t - t^2)}}
                          \int_0^\infty x e^{-x^2/2(t - t^2)} \, dx = \sqrt{\frac{2t}{\pi(1-t)}}
\end{align*}
which shows that $(1 -t)^{-1}X_t$ is integrable on $[0,1)$ but not
$L^1$ bounded.  

Let $\mathcal{F}_t = \sigma((1-s)^{-1} X_s; 0 \leq s \leq t)$.  If we let $B$ be a Brownian motion then for $0 \leq s < t < 1$ we have
\begin{align*}
\frac{1}{(1-t)(1-s)} \expectation{X_t X_s} 
&=\frac{s(1-t)}{(1-t)(1-s)} = \frac{s}{1-s}
\end{align*}
which is does not depend on $t$.  Since $(1-t)^{-1}X_t$ is Gaussian this implies that for every $0 \leq r_1 < \dotsb < r_m \leq s < t < 1$ we have
$((1-r_1)^{-1}X_{r_1}, \dotsc, (1-r_m)^{-1}X_{r_m}, (1-s)^{-1}X_{s}, (1-t)^{-1}X_{t})$ is a Gaussian random vector and the same is true of
$((1-r_1)^{-1}X_{r_1}, \dotsc, (1-r_m)^{-1}X_{r_m}, (1-t)^{-1}X_{t} - (1-s)^{-1}X_{s})$ (Example \ref{LinearTransformationGaussian}).
By Proposition \ref{GaussianIndependence} we know that $(1-t)^{-1}X_{t} - (1-s)^{-1}X_{s} \Independent ((1-r_1)^{-1}X_{r_1}, \dotsc, (1-r_m)^{-1}X_{r_m})$ and
thus for all $0 \leq s < t < 1$ by Lemma \ref{IndependencePiSystem} we have $(1-t)^{-1}X_{t} - (1-s)^{-1}X_{s} \Independent \mathcal{F}_s$. 
The martingale property follows in the standard way,
\begin{align*}
\cexpectationlong{\mathcal{F}_s}{(1-t)^{-1}X_t}
&=\cexpectationlong{\mathcal{F}_s}{(1-t)^{-1}X_t - (1-s)^{-1}X_s} + 
\cexpectationlong{\mathcal{F}_s}{(1-s)^{-1}X_s} \\
&=\expectation{(1-t)^{-1}X_t - (1-s)^{-1}X_s} + (1-s)^{-1}X_s =   (1-s)^{-1}X_s\\
\end{align*}
\end{proof}

\begin{xca}\label{ProductOfIndependentMarkov}Let $X$ be a Markov process in $(S, \mathcal{S})$  on time scale $T$ with
  transition kernel $\mu_{s,t}$ and
  let $Y$ be a Markov process in $(U, \mathcal{U})$ on time scale $T$ with 
  transition kernel $\nu_{s,t}$.  Show that if $X$ and $Y$ are
  independent
  that $(X,Y)$ is a Markov process in $(S \times U, \mathcal{S}
  \otimes \mathcal{U})$ on time scale $T$ with transition kernel $\mu_{s,t}
  \otimes \nu_{s,t}$ (note that kernel $\mu_{s,t}
  \otimes \nu_{s,t}$ is just the pointwise product measure).
\end{xca}
\begin{proof}
TODO: Finish

Pick $t < u \in T$.  Let $C \in \mathcal{S}$ and $D \in \mathcal{U}$ and compute using the
independence of $X$ and $Y$, let
\begin{align*}
&\probability{(X_u,Y_u) \in A \times B ; (X_t, Y_t) \in C
  \times D} \\
&=\probability{X_u \in A ; X_t \in C} \probability{Y_u \in B ; Y_t \in  D} \\
&=\expectation{\cprobability{X_u \in A}{X_t} ; X_t \in C}
\expectation{\cprobability{Y_u \in B}{Y_t} ;  Y_t \in  D} \\
&=\expectation{\cprobability{X_u \in A}{X_t} ; X_t \in C;
\cprobability{Y_u \in B}{Y_t} ;  Y_t \in  D} \\
\end{align*}
which since sets of the form $ (X_t, Y_t) \in C
  \times D$ are a generating
$\pi$-system of $\sigma(X_t,Y_t)$ we the claim is shown by Lemma
\ref{ConditionalExpectationExtension}.

Finally we conclude that $\cprobability{(X_u, Y_u) \in
  \cdot}{(X_t,Y_t)} = \mu_{t,u} \otimes \nu_{t,u}$ 
by the uniqueness of product measure.
\end{proof}

\begin{xca}[Kallenberg Exercise 17.1]Show that if $M$ is a local
  martingale and $\xi$ is a $\mathcal{F}_0$-measurable random variable 
then $N_t = \xi M_t$ is also a local martingale.
\end{xca}
\begin{proof}
Let $\tau_n$ be a localizing sequence for $(M-M_0)$ so that
$(M-M_0)^{\tau_n}$ is a martingale for all $n \in \naturals$ and
$\tau_n \uparrow \infty$ almost surely.  Let $\sigma_n = \tau_n
\characteristic{\abs{\xi} \leq n}$.  Since $\xi$ is almost surely
finite then it follows that $\sigma_n \uparrow \infty$ almost surely
(specifically on the intersection of the event that $\tau_n \uparrow
\infty$ and $\abs{\xi} < \infty$).  Moreover since $\lbrace \sigma_n
\leq t \rbrace = \lbrace\tau_n \leq t \rbrace \cup \lbrace \abs{\xi}
\leq n \rbrace$ and $\xi$ is $\mathcal{F}_0$-measurable it follows
that $\sigma_n$ are $\mathcal{F}$-optional times.  Now note that
$\xi M$ is $\mathcal{F}$-adapted since $\xi$ is
$\mathcal{F}_0$-measurable
and for each $n \in \naturals$  and $0 \leq t < \infty$ we have
\begin{align*}
\expectation{\abs{\xi M_{t \wedge \sigma_n} - \xi M_0}} 
&= \expectation{\abs{\xi} \abs{M_{t \wedge \tau_n} - M_0} ; \abs{\xi}
  \leq n} 
\leq  n \expectation{( M -
  M_0)^{\tau_n}} < \infty
\end{align*} 
and moreover by the pullout rule of
conditional expectation and the fact that $\tau_n$ localizes $M$,
\begin{align*}
\cexpectationlong{\mathcal{F}_s}{(\xi M_{\sigma_n \wedge t} - \xi M_0)}
&=\cexpectationlong{\mathcal{F}_s}{\xi  (M_{\sigma_n \wedge t} - M_0)
  ; \abs{\xi} \leq n} \\
&= \xi \characteristic{\abs{\xi} \leq n}  \cexpectationlong{\mathcal{F}_s}{M_{\tau_n \wedge t} - M_0} \\
&= \xi \characteristic{\abs{\xi} \leq n} (M_{\tau_n \wedge s} - M_0) \\
&= \xi M_{\sigma_n \wedge s} - \xi M_0
\end{align*}
showing that $(M-M_0)^{\sigma_n}$ is a martingale.
\end{proof}

\begin{xca}[Kallenberg Exercise 17.2]Show that a
  local martingale $M$ with $M \geq 0$ for all $0 \leq t < \infty$
  almost surely and $\expectation{M_0} < \infty$ is a
  supermartingale.  Give an example to show that $M$ is not necessarily a martingale.
\end{xca}
\begin{proof}
Let $\tau_n$ be a localizing sequence for $M$.  We know that $(M -
M_0)^{\tau_n}$ is a martingale but by the integrability and 
$\mathcal{F}_0$-measurability of $M_0$ we
see that in fact $M^{\tau_n}$ is a martingale.  Now 
since $M \geq 0$, $M_{\tau_n \wedge t} \toas M_t$ for all $0 \leq t < \infty$ and 
Fatou's Lemma for conditional expectations we have for all $0 \leq s < t < \infty$
\begin{align*}
\cexpectationlong{\mathcal{F}_s}{M_t} &\leq 
\liminf_{n \to \infty}\cexpectationlong{\mathcal{F}_s}{M_{\tau_n \wedge t}}
= \liminf_{n \to \infty} M_{\tau_n \wedge t} = M_{s}
\end{align*}
almost surely.  Choosing $s=0$ and taking expectations shows that $\expectation{M_t} \leq \expectation{M_0} < \infty$ 
and therefore $M$ is a supermartingale.

TODO: Do the last part.
\end{proof}
\begin{xca}[Kallenberg Exercise 17.4]Let $M_n$ be a sequence of continuous local
  martingales starting at zero and let $\tau_n$ be a sequence of
  optional times, then $(M^*_n)_{\tau_n} \toprob 0$ if and only if
  $[M_n]_{\tau_n} \toprob 0$.  State and prove a corresponding result
  for stochastic integrals.
\end{xca}
\begin{proof}
Define $\tilde{M}_n = M_n^{\tau_n}$ and apply Lemma
\ref{QuadraticCovariationAndContinuity} to conclude that
$\tilde{M}_n^* \toprob 0$ if and only if $[\tilde{M}_n]_\infty \toprob
0$.  Now note that 
\begin{align*}
\tilde{M}^*_n 
&= \sup_{0 \leq t < \infty} \abs{(\tilde{M}_n)_t} 
= \sup_{0 \leq t < \infty} \abs{(M_n)_{\tau_n \wedge t}} 
= (M_n)^*_{\tau_n}
\end{align*}
and by Theorem \ref{OptionalQuadraticCovariation} we get
\begin{align*}
[\tilde{M}_n]_\infty
&= [M^{\tau_n}_n]_\infty
=[M_n]^{\tau_n}_\infty
=[M_n]_{\tau_n}
\end{align*}

The corresponding result for stochastic integrals says that
given continuous local martingales $M_n$, processes $V_n \in
L(M_n)$ and optional times $\tau_n$ we have $\left(\int V_n \,
dM_n\right)^*_{\tau_n} \toprob 0$ if and only if $\int_0^{\tau_n} V_n^2(s) \, d[M_n](s)
\toprob 0$.  This follows from what we have just proven and 
Lemma \ref{BasicPropertiesStochasticIntegralContinuousMartingale} to see
\begin{align*}
[\int V_n \, dM_n]_{\tau_n} &= \int_0^{\tau_n} V_n(s) \, d[M_n](s)
\end{align*}
\end{proof}

\begin{xca}[Kallenberg Exercise 17.6]Let $B_t$ be a Brownian motion starting at zero and $\tau$ be an optional
  time.  Show that $\expectation{\tau^{1/2}} < \infty$ implies
  $\expectation{B_\tau} = 0$ and $\expectation{\tau} < \infty$ implies
  $\expectation{B^2_\tau} = \expectation{\tau}$.
\end{xca}
\begin{proof}For $\tau$ bounded by a constant $T$ these are both consequences of
  Optional Stopping.  For then since $B_t$ is a martingale we have
\begin{align*}
\expectation{B_\tau} &=
\expectation{\cexpectationlong{\mathcal{F}_\tau}{B_T}} =
\expectation{B_T} = 0
\end{align*}
and since $B^2_t - t$ is a martingale we have
\begin{align*}
\expectation{B^2_\tau} - \expectation{\tau}&=
\expectation{\cexpectationlong{\mathcal{F}_\tau}{B^2_T - T}} =
\expectation{B^2_T - T} = 0
\end{align*}
Now consider the sequence of bounded optional times $\tau \wedge n$.
If we have $\expectation{\tau^{1/2}} < \infty$ the we can apply the
BDG inequality (Lemma \ref{BDGInequalities}) to the stopped process
$B^\tau$ (which is a priori only a continuous local martingale) to see that there is a
constant $c_1 > 0$ such that $\expectation{\sup_{0 \leq t \leq \tau}
  \abs{B_t}} \leq c \expectation{\tau^{1/2}} < \infty$ and therefore
since $\abs{B_{\tau \wedge n}} \leq \sup_{0 \leq t \leq \tau}
\abs{B_t}$ we can use Dominated Convergence to see that
$\expectation{B_\tau} = \lim_{n\to \infty} \expectation{B_{\tau \wedge
    n}} = 0$.  Similarly when $\expectation{\tau} < \infty$, we get a constant $c_2 > 0$ such that $\expectation{\sup_{0 \leq t \leq \tau}
  \abs{B^2_t}} \leq c_2 \expectation{\tau} < \infty$ and therefore
Dominated Convergence gives us 
\begin{align*}
\expectation{B^2_\tau} &= \lim_{n \to \infty} \expectation{B^2_{\tau
    \wedge n}} = \lim_{n \to \infty} \expectation{\tau \wedge n} = \expectation{\tau}
\end{align*}

While we're at it, we can provide a different proof that
$\expectation{B_\tau} = 0$ under the weaker assumption
$\expectation{\tau} < \infty$ that doesn't rely on the BDG
inequalities.  As above, it suffices to show that $\abs{B_{\tau \wedge
    t}}$ is dominated by an integrable random variable.  The proof here is taken from Peres and Morters.  For
each integer $k \geq 0$ consider 
\begin{align*}
M_k = \sup_{0 \leq t \leq 1} \abs{B_{t+k} - B_k}
\end{align*}
TODO: Finish
\end{proof}

\begin{xca}[Kallenberg Exercise 17.8] Let $X$ and $Y$ be continuous semimartingales show that
  $[X+Y]_t^{1/2} \leq [X]_t^{1/2} + [Y]_t^{1/2}$ for all $0 \leq t <
  \infty$ almost surely.
\end{xca}
\begin{proof}
From bilinearity and the Cauchy Schwartz inequality Lemma
\ref{CourregeCauchySchwartz} we have
\begin{align*}
[X+Y]_t &= [X]_t + 2[X,Y]_t + [Y]_t \leq  [X]_t + 2\abs{[X,Y]_t} +
          [Y]_t \leq 
[X]_t + 2[X]_t^{1/2}[Y]_t^{1/2} + [Y]_t = ([X]_t^{1/2} + [Y]_t^{1/2})^2
\end{align*}
for all $0 \leq t < t$ almost surely.
The result follows by noting the non-negativity of quadratic
variation and taking square rooots.
\end{proof}

\begin{xca}[Kallenberg Exercise 17.11]Let $X$ be a continuous
  semimartingale and let $U,V \in L(X)$ be such that $U = V$ a.s. on a
  set $A \in \mathcal{F}_0$.  Use Lemma
  \ref{StoppingIntegralsContinuousSemimartingale} to show that $\int U \, dX = \int V \, dX$
  a.s on $A$.
\end{xca}
\begin{proof}
Define 
\begin{align*}
\tau(\omega) &= \begin{cases}
\infty & \text{where $\omega \in A$} \\
0 & \text{when $\omega \notin A$}
\end{cases}
\end{align*}
and note that $\tau$ is an optional time since $A \in
\mathcal{F}_0$. Also note that 
\begin{align*}
\characteristic{[0,\tau(\omega)]}( t)
\cdot U_t(\omega)  &= \begin{cases}
U_t(\omega) & \text{when $\omega \in A$ or $\omega \notin A$ and
  $t=0$} \\
0 & \text{when $\omega \notin A$ and $t > 0$}
\end{cases}
\end{align*}
and similarly with $V$ and therefore we conclude
$\characteristic{[0,\tau]} \cdot U = \characteristic{[0,\tau]} \cdot
V$ almost surely and therefore $\int \characteristic{[0,\tau]}  U \,
dM = \int \characteristic{[0,\tau]}  V \, dM$ almost surely by Lemma
\ref{BasicPropertiesStochasticIntegralContinuousMartingale}.   From Lemma
\ref{StoppingIntegralsContinuousSemimartingale} and the fact that
$\int U \, dX$ starts at zero we get almost surely
\begin{align*}
\characteristic{A} \int_0^t U \, dM &= \characteristic{A} \int_0^t U
\, dM + \characteristic{A^c} \int_0^0 U \, dM 
=\int_0^{t \wedge \tau} U \, dM 
= \int_0^{t} \characteristic{[0,\tau]} U \, dM \\
\end{align*}
and similarly with $V$ and the result follows.
\end{proof}

\begin{xca}[Kallenberg Exercise 17.13]Let $X$ be Brownian bridge then
  $X_{t \wedge 1}$ is a semimartingale.
\end{xca}
\begin{proof}
We know from Exercise \ref{BrownianBridgeMartingale} that
$Y_t = (1-t)^{-1}X_t$ is a martingale on $[0,1)$.  Now apply the local Ito
Lemma to the semimartingale $(1 \wedge t ,(1-t)^{-1}X_t)$ using the
function $f(t,x) = (1-t) x$ on the domain $(-\infty,1) \times \reals$ to see that
\begin{align*}
X_{t \wedge 1} &= f(t,Y_t) - f(0,Y_0) = 
\int_0^t (1-s) \, dY_s -\int_0^t Y_s \, ds
\end{align*}
where the first stochastic integral is continuous local martingale and the second random Stieltjes integral is of finite variation.

TODO: Get all the details around the local Ito stuff spelled out.
\end{proof}

\begin{xca}\label{QuadraticCovariationMatrixVectorStochasticIntegral}Let
  $M$ be a continuous local martingale in $\reals^r$ and let $V \in L(X)$ be an $d
  \times r$ matrix valued process in $L(M)$, then 
\begin{align*}
\left[ \left(\int V_s \, dM_s \right)^{(i)},  \left(\int V_s \, dM_s
  \right)^{(i)} \right]_t &= \sum_{k=1}^r \sum_{l=1}^r \int_0^t
                            V^{ik}_s V^{jl}_s \, d[M^k,M^l]_s \text{ for all
                            $1 \leq i,j \leq d$}
\end{align*}
which we write stylistically as $[\int V \, dM] = \int V d[M] V^T$.
\end{xca}
\begin{proof}
This just follows from bilinearity of quadratic covariation and Lemma \ref{BasicPropertiesStochasticIntegralContinuousMartingale}
\begin{align*}
\left[ \left(\int V_s \, dM_s \right)^{(i)},  \left(\int V_s \, dM_s
  \right)^{(i)} \right]_t 
&= \left[ 
\sum_{k=1}^r \int V^{ik}_s \,dM^{k}_s, 
\sum_{l=1}^r \int V^{jl}_s \,dM^{l}_s 
\right]_t \\
&=\sum_{k=1}^r \sum_{l=1}^r \left[ \int V^{ij}_s \,dM^{k}_s, \int V^{jl}_s \,dM^l_s \right]_t \\
&=\sum_{k=1}^r \sum_{l=1}^r \int V^{ik}_s V^{jl}_s \, d[ M^{k}, M^l ]_s \\
\end{align*}
\end{proof}

\begin{xca}Let $(S,d)$ be a totally bounded metric space.  Prove that $S$ is separable and that every uniformly continuous function $f : S \to \reals$ is bounded.
\end{xca}
\begin{proof}For each $n \in \integers$ we may find a finite set $x^n_1, \dots, x^n_{m_n}$ such that $B(x^n_j; 1/n)$ covers $S$.  The union of $x^n_{j}$ for $n \in \integers$ and $j=1, \dotsc, m_n$ is a countable dense subset of $S$.  Assume that $f$ is uniformly continuous, pick an $\epsilon > 0$ such that $d(x,y) < \epsilon$ implies $\abs{f(x) - f(y)} < 1$.  Now pick $x_1, \dotsc, x_n$ such that $B(x_1; \epsilon), \dotsc, B(x_n; \epsilon)$ covers $S$.  Let $K = \abs{f(x_1)} \vee \dotsb \vee \abs{f(x_n)}$ and note that for every $x$ we may pick $x_j$ such that $x \in B(x_j; \epsilon)$ and therefore $\abs{f(x)} \leq K + 1$.
\end{proof}

\begin{xca}\label{SkorohodInfiniteJ1PushForwardContinuous}Let $(S,r)$ and $(T, r^\prime)$ be metric spaces and $g : S  \to T$ be a continuous.  Define $g_* : D([0,\infty); S) \to D([0,\infty); T)$ by $g_*(f)(t) = g(f(t))$ then $g_*$ is
continuous in the $J_1$ topology.
\end{xca}
\begin{proof}
Since the $J_1$ topology is metrizable it suffices to show that $g_*$ takes convergence sequences convergent sequences.  Let $f, f_1, f_2, \dotsc \in  D([0,\infty); S)$ be such that
$f_n \to f$.  Then by Proposition \ref{SkorohodInfiniteJ1EquivalenceC} we know that for every $T >0$ and every $t, t_1, t_2, \dotsc \in [0,T]$ with $\lim_{n \to \infty} t_n = t$ we have 
\begin{itemize}
\item[(i)] $\lim_{n \to \infty} r(f_n(t_n), f(t)) \wedge r(f_n(t_n), f(t-)) = 0$. 
\item[(ii)] If $\lim_{n \to \infty} r(f_n(t_n), f(t)) = 0$ then for every sequence
$s_n$ such that $t_n \leq s_n \leq T$ and $\lim_{n \to \infty} s_n = t$ we have $\lim_{n \to \infty} r(f_n(s_n), f(t)) = 0$
\item[(iii)] If $\lim_{n \to \infty} r(f_n(t_n), f(t-)) = 0$ then for every sequence
$s_n$ such that $0 \leq s_n \leq t_n$ and $\lim_{n \to \infty} s_n = t$ we have $\lim_{n \to \infty} r(f_n(s_n), f(t-)) = 0$
\end{itemize}
Using continuity of $g$ at $f(t)$ and $f(t-)$ for every $\epsilon>0$ there exists $\delta > 0$ such that $r(x, f(t)) < \delta$ implies $r^\prime(g(x), g(f(t))) < \epsilon$ and 
$r(x, f(t-)) < \delta$ implies $r^\prime(g(x), g(f(t-))) < \epsilon$.  By (i) we may find $N$ such that for all $n \geq N$ we have $r(f_n(t_n), f(t)) \wedge r(f_n(t_n), f(t-)) < \delta$; thus
$r^\prime(g(f_n(t_n)), g(f(t))) \wedge r^\prime(g(f_n(t_n)), g(f(t-))) < \epsilon$ which shows $\lim_{n \to \infty} r^\prime(g(f_n(t_n)), g(f(t))) \wedge r^\prime(g(f_n(t_n)), g(f(t-))) = 0$.

Also that $\lim_{s \to t^-} g(f(s)) = g(f(t-))$ since  $g$ is continuous at $f(t-)$: given $\epsilon > 0$ we take $\delta > 0$ such that $r(x, f(t-)) < \delta$ implies $r^\prime(g(x), g(f(t-))) < \epsilon$; now take $\rho > 0$ such that $t - \rho < s < t$ implies $r(f(s), f(t-)) < \delta$.  For the moment we fix $T>0$ and $0 \leq t \leq T$.  We consider two cases separately.

Case 1: $g(f(t)) = g(f(t-))$.  In this conditions (i), (ii) and (iii) reduce to the assertion that for all $t_n \to t$ we have $g(f_n(t_n)) \to g(f(t))$. From the continuity of $g$ given $\epsilon > 0$ then $g^{-1}(B(g(f(t)), \epsilon))$ is open in $S$ and contains both $f(t)$ and $f(t-)$ (it doesn't matter whether $f(t) = f(t-)$ or not).  We may find a $\delta >0$ such that $B(f(t), \delta) \subset g^{-1}(B(g(f(t)), \epsilon))$ and $B(f(t-), \delta) \subset g^{-1}(B(g(f(t)), \epsilon))$.  By the property (i) of $f$ and the $f_n$ we may find $N>0$ such that $f_n(t_n) \in B(f(t), \delta) \cup B(f(t-), \delta)$ for all $n \geq N$ and we are done.  I think this even easier because for this case (i) is equivalent to (i), (ii) and (iii) for $g_*(f)$ and $g_*(f_n)$ and we have already shown that (i) implies (i).
 
Case 2:  $g(f(t)) = g(f(t-))$ and $f(t) \neq f(t-)$.   We already know that (i) holds for $g_*(f)$ and $g_*(f_n)$.  Suppose $t_n \to t$ and $g(f_n(t_n)) \to g(f(t))$.  Then because $f_n \to f$ we know that $r(f_n(t_n), f(t)) \wedge r(f_n(t_n), f(t-)) \to 0$. If it is not true that $r(f_n(t_n), f(t)) \to 0$ then we can find a subsequence $n_k$ such that $f_{n_k}(t_{n_k}) \to f(t-)$ but by continuity of $g$ we conclude $g(f_{n_k}(t_{n_k})) \to g(f(t-))$ which is a contradiction (recall that $g(f(t-)) = \lim_{s \to t^-} g \circ f(s)$ by continuity of $g$).  Therefore we get $f_n(t_n) \to f(t)$ and for any $s_n \geq t_n$ with $s_n \to t$ we conclude $f_n(s_n) \to f(t)$ hence $g(f_n(t_n)) \to g(f(t))$ by continuity of $g$.  By a similar argument, if we assume that $g(f_n(t_n)) \to g(f(t-))$ then we conclude that $f_n(t_n) \to f(t-)$ and therefore for all $s_n \leq t_n$ with $s_n \to t$ we have $g(f_n(s_n)) \to g(f(t-))$ and thus (iii) holds.

Now we apply Proposition \ref{SkorohodInfiniteJ1EquivalenceC} in the opposite direction to conclude that $g \circ f_n \to g \circ f$ in the $J_1$ topology.
\end{proof}

\begin{xca}Define $\psi : D([0,\infty); \reals) \to D([0,\infty); \reals)$ by $\psi (f) (t) = \sup_{0 \leq s \leq t} f(s)$.  Show that $\psi$ is continuous in the $J_1$ topology.
\end{xca}
\begin{proof}
$\psi(f)$ is non-decreasing and finite therefore it cadlag.  To see continuity suppose that $f_n \to f$ in the $J_1$ topology.  Pick $\lambda_n$ such that $\gamma(\lambda_n) \to 0$
and $\lim_{n \to \infty} \sup_{0 \leq t \leq T} \abs{f_n(t) - f(\lambda_n(t))} =0$ for all $T > 0$.  Fix $T >0$ and note that
for any $\delta > 0$, $n \in \naturals$ and $0 \leq t \leq T$ we may pick $0 \leq w_n, u_n \leq t \leq T$ such that $\sup_{0 \leq s \leq t} f_n(s) \leq f_n(u_n) + \delta$ and 
$\sup_{0 \leq s \leq t} f(\lambda_n(s)) \leq f(\lambda_n(w_n)) + \delta$.  From these two inequalities and the fact that $\psi(f)(\lambda_n(t)) = \sup_{0 \leq s \leq \lambda_n(t)} f(s) = \sup_{0 \leq s \leq t} f(\lambda_n(s))$, we get
\begin{align*}
\psi(f_n)(t) &\leq f_n(u_n) + \delta \leq f(\lambda_n(u_n)) + \delta + \sup_{0 \leq t \leq T} \abs{f_n(t) - f(\lambda_n(t))}  \\
&\leq \sup_{0 \leq s \leq t} f(\lambda_n(s)) + \delta + \sup_{0 \leq t \leq T} \abs{f_n(t) - f(\lambda_n(t))}  \\
&= \psi(f)(\lambda_n(t)) + \delta + \sup_{0 \leq t \leq T} \abs{f_n(t) - f(\lambda_n(t))}  \\
\intertext{and}
\psi(f)(\lambda_n(t)) &\leq f(u) + \delta \leq f_n(w_n) + \delta + \sup_{0 \leq t \leq T} \abs{f_n(t) - f(\lambda_n(t))}  \\
&\leq \sup_{0 \leq s \leq t} f_n(s) + \delta + \sup_{0 \leq t \leq T} \abs{f_n(t) - f(\lambda_n(t))}  \\
&=\psi(f_n)(t) + \delta + \sup_{0 \leq t \leq T} \abs{f_n(t) - f(\lambda_n(t))} 
\end{align*}
thus $\sup_{0 \leq t \leq T} \abs{\psi(f_n)(t) - \psi(f)(\lambda_n(t))} \leq \delta + \sup_{0 \leq t \leq T} \abs{f_n(t) - f(\lambda_n(t))}$ and the result follows by taking the limit as $n \to \infty$ and then letting $\delta \to 0$.
\end{proof}


\appendix
\chapter{Techniques}

This section is a place to collect some of the recurring proof
techniques that one should be familiar with.

\section{Standard Machinery}
The standard measure theory arguments that proceed by showing a result
for indicator functions, simple random variables and the positive
random variables.  TODO:  There are a ton of examples of this such as
Lemma \ref{ChangeOfVariables} and Lemma \ref{ChainRuleDensity}.

\subsection{Monotone Class Arguments}
Part of the standard machinery that has independent utility is the
monotone class argument.  This allows one to demonstrate that a
property holds for an entire $\sigma$-algebra of sets by showing that
property holds for a simpler subclass of sets.  Good examples are
Lemma \ref{UniquenessOfMeasure} and Lemma \ref{IndependencePiSystem}.

\section{Almost Sure Convergence}
When one needs to show almost sure convergence of a sequence of random
variables the Borel Cantelli Theorem is a workhorse.  Good examples of
this are Lemma \ref{SLLNL4} and Lemma
\ref{ConvergenceInProbabilityAlmostSureSubsequence}.

Another technique to use that is related is to show that the sum of
the random variables is integrable.  Then you can conclude that the
sum of random variables is almost surely finite and therefore the
terms of the sequence converge to zero a.s.
Good examples of
this are Lemma \ref{SLLNL2} and Lemma
\ref{ConvergenceInProbabilityAlmostSureSubsequence}.

\section{Bounding Expectations}

A common task that one faces is to provide bounds for an expected
value (or more generally a moment).  For example, one may need to know
that a random variable has a finite expectation for use with the
Dominated Convergence Theorem.

\subsection{Using Tail Bound}
A problem I have encountered is trying to use a tail bound to prove
that an expectation is finite.  The problem that I sometime have is
that I write:
\begin{align*}
\expectation{f(\xi)} = \expectation{ \characteristic{\xi \leq \lambda}
  \cdot f(\xi)} + \expectation{\characteristic{\xi > \lambda} \cdot f(\xi)}
\end{align*}
Often knowing $\xi \leq \lambda$ we can show that the first
expectation is bounded (this is often easy).  The problem is usually
that one might be given a tail bound that controls $\probability{\xi >
  \lambda}$ but there is no control over the behavior of $f(\xi)$ that
allows one to provide a bound for the second expectation.  Are there
general approaches for dealing with this?  Possible answer here is
that one might need to take a different approach and use Lemma
\ref{TailsAndExpectations}.  A good example of how to do this is with 
Lemma \ref{SubgaussianEquivalence}.

TODO: Passing from $L^p$ convergence to almost sure convergence.  Note
that we easily get almost sure convergence along a subsequence.

\section{Proving Inequalities}
\subsection{Using Calculus}
If one wants to show that $f(x) \geq 0$ on an interval $[a,b]$ one of
the easiest ways to show the inequality is to find the minimum of
$f(x)$ on $[a,b]$ and to show this value is bigger than zero.  Finding
the minimum is a lot easier if $f(x)$ is differentiable.  A common
special case one can easily show $f(a) \geq 0$ and $f(b) \geq
0$ and show that $f(x)$ is increasing or decreasing on $[a,b]$ by
showing $f^{\prime}(x)$ is positive or negative.  The problem with
this technique is that it is really a proof technique and requires
that one knows the answer beforehand (e.g. one usually wants to show
$g(x) \leq h(x)$ and the bound $h(x)$ is what you are trying to figure
out).  Sometimes the technique can be
used to guess the answer by taking a simpler known inequality and
antidifferentiating (see Lemma \ref{GaussianTailsElementary} for a
non-trivial example).

\subsection{Using Taylor's Theorem}
Taylor's Theorem is also a good way of both guessing and proving
inequalities; if one can show that the remainder term (in either
integral or Lagrange form usually) is of a particular sign over an
interval an inequality follows.  

\chapter{Integrals}
\begin{align*}
\int_0^\infty e^{-x^2} dx &= \frac{\sqrt{\pi}}{2} \\
\int_0^\infty x^{2n} e^{-x^2} dx &= \frac{\sqrt{\pi} \left(2n-1\right)!!}{2^{n+1}} \\
\int_0^\infty x^{2n+1} e^{-x^2} dx &= \frac{n!}{2}\\
\end{align*}
\begin{align*}
\Gamma(z) &= \int_0^\infty x^{z-1} e^{-x} dx
\end{align*}
\chapter{Inequalities}
From time to time in these notes we'll have a need for some simple
inequalities for elementary functions.  The following Lemma collects
them in one place since they are all proven by use of basic calculus.
\begin{lem}\label{BasicExponentialInequalities}The following
  inequalities hold:
\begin{itemize}
\item[(i)] $1+x \leq e^x$ for all $x \in \reals$.
\item[(ii)] $e^x \leq 1 + 2x$ for all $x \in [0,1]$.
\item[(iii)] $e^x \leq 1 + x + x^2$ for all $x \leq 1$.
\item[(iv)] $\frac{1}{2}(e^x + e^{-x}) \leq e^{x^2/2}$ for all $x \in \reals$.
\item[(v)] $\abs{\sin(x)} < \abs{x}$ for all $x  \neq 0$.
\item[(vi)] $1 - \frac{x^2}{2} \leq \cos(x)$ for all $x \in \reals$.
\item[(vii)] $x + \log(1-x) \leq 0$ for all $x \in [0,1)$.
\item[(viii)] $e^{-x} \leq 1 - (1 - e^{-1}) x$ for all $x \in [0,1]$.
\end{itemize}
\end{lem}
\begin{proof}
Note that for ${x\geq0}$ we can consider $f(x) = e^x - x -1$ and note
that $f(0)=0$ and moreover we can see that $f(x)$ has a global minimum
at $x=0$ since $f'(x) = e^x - 1$ vanishes precisely at $x=0$ and
$f^{\prime \prime}(x)=e^x$ is strictly positive.  Alternative this can
be seen by Taylor's Theorem.  One writes using the Lagrange form of
the remainder $e^x = 1 + x + \frac{x^2}{2} e^c$ for some $c$.
Since the remainder is positive the result follows.

In a similar vein to show (ii), define $f(x) = 1+2x-e^x$ and notice that $f(x)$ has
a global maximum at $x=\ln(2)$ and no other local maximum.  Thus, it
suffices to validate the inequality at the endpoints $x=0$ and $x=1$
which is obvious.

To show (iv) we just manipulate series expansions.
\begin{align*}
\frac{1}{2}\left(e^x + e^{-x}\right) & = \frac{1}{2}\left(\sum_{n=0}^\infty
\frac{x^n}{n!} + \sum_{n=0}^\infty \frac{(-x)^n}{n!}\right) \\
& = \sum_{n=0}^\infty \frac{x^{2n}}{(2n)!} \\
& \leq \sum_{n=0}^\infty \frac{x^{2n}}{2^n n!} = e^{\frac{x^2}{2}}\\
\end{align*}

To see (v),  because the function $x - \sin(x)$ is odd, it suffices to
show that it is strictly positive for $x > 0$.
Clearly $x - \sin(x) > 0$ for $x > 1$.   For $0 < x < 1$ we just use
Taylor's Theorem with Lagrange remainder to see that $\sin(x) = x -
\frac{x^2}{2} \cos(c)$ for some $0 < c < x < 1$.  The remainder is
negative so the result follows.

To show (vi), define $f(x) = \frac{x^2}{2} -1 +  \cos(x)$.  Calculate
the first derivative $f^\prime(x) = x - \sin(x)$.  The function
$f^\prime(x) = 0$ if and only if $x=0$ by (v) and moreover $f^\prime(x)$ changes sign at $x=0$
which shows that $f(0) = 0$ is a strict global minimum.

To show (vii), define $f(x) = x + \log(1 -x)$ and differentiate to see
that $f^\prime(x) = 1 - \frac{1}{1-x} = \frac{-x}{1-x} < 0$ for $x \in
(0,1)$.  Therefore $f(x) \leq f(0)=0$ for $x \in [0,1)$.

To show (viii), let $a = 1 - e^{-1}$ and $f(x) = 1 - ax - e^{-x}$.
Take first derivative $f^\prime(x) = -a + e^{-x}$ which has a zero at
$x = -\ln a \approx 0.5$.  Furthermore $f^{\prime \prime}(x) = -e^{-x}
< 0$ so we have a global maximum at $x = -\ln a$, therefore to show
$f(x) \geq 0$  for $x \in [0,1]$ it suffices to show it at the
endpoints: $f(0) = f(1) = 0$.
\end{proof}

When dealing with characteristic functions, it is useful to have
estimates for the function $e^{ix}$.  We collect a few useful ones
here.
\begin{thm}\label{BasicComplexExponentialInequalities}The following
  inequalities hold:
\begin{itemize}
\item[(i)] $\abs{e^{ix} -1 - ix} \leq \frac{x^2}{2}$ for all $x \in \reals$.
\item[(ii)] $\abs{e^{ix} -1 - ix + \frac{x^2}{2}} \leq x^2 R(x)$ for
  all $x \in \reals$ where $\abs{R(x)} \leq 1$ and $\lim_{x \to 0}
  R(x) = 0$.
\end{itemize}
\end{thm}
\begin{proof}
To see (i) we use Taylor's Theorem with the Lagrange form of the
remainder to write $e^{ix} - 1 - ix = -\frac{x^2}{2} e^{ic}$ for some
$c \in \reals$.  Now take absolute values and use the fact that
$\abs{e^{ic}}=1$.

To see (i) we use Taylor's Theorem with the integral form of the
remainder to write $e^{ix} - 1 - ix = - \int_0^x (x - t) e^{it} \,
dt$.  Now we write 
\begin{align*}
 \int_0^x (x - t) e^{it} \, dt &=  \int_0^x (x - t) (e^{it} - 1) \, dt
+ \int_0^x (x-t) \, dt 
= \int_0^x (x - t) (e^{it} - 1) \, dt + \frac{x^2}{2}
\end{align*}
so that $x^2 R(x) = -\int_0^x (x - t) (e^{it} - 1) \, dt$.  Observe that on the one hand
\begin{align*}
\abs{R(x)}
&=\frac{1}{x^2} \abs{\int_0^x (x - t) (e^{it} - 1) \, dt} 
\leq \frac{\sup_{0 \leq t \leq x} \abs{e^{it} - 1}}{x^2} \int_0^x (x - t) \, dt \\
&=\sup_{0 \leq t \leq x} \abs{e^{it} - 1}
\end{align*}
and thus continuity of $e^{ix}$ implies $\lim_{x \to 0} R(x) = 0$.
On the other hand
\begin{align*}
\abs{\int_0^x (x - t) (e^{it} - 1) \, dt} 
&\leq \int_0^x (x - t) \abs{e^{it} - 1} \, dt
&\leq 2 \int_0^x (x - t) \, dt = x^2
\end{align*}
which shows $\abs{R(x)} \leq 1$.  
\end{proof}

\begin{thm}[Arithmetic Mean Geometric Mean Inequality]\label{AMGM}Let $x_1, \dotsc, x_n$ be non-negative real
  numbers and let $p_1, \dotsc, p_n$ be non-negative real numbers such
  that $\sum_{j=1}^n p_j = 1$ then
\begin{align*}
x_1^{p_1} \dotsm x_n^{p_n} &\leq p_1 x_1 + \dotsm + p_n x_n
\end{align*}
\end{thm}
\begin{proof}
TODO:
\end{proof}

\begin{prop}\label{SimplePowerMean}Let $x_1, \dotsc, x_n$ be non-negative real numbers and $m
  \in \naturals$ then 
\begin{align*}
\left( \frac{x_1 + \dotsm + x_n}{n} \right)^m &\leq \frac{x^m_1 + \dotsm + x^m_n}{n}
\end{align*}
\end{prop}
\begin{proof}
We first validate the result for $m=2$.  To see this case observe that
by Theorem \ref{AMGM} we have for arbitrary non-negative integers
$\alpha$ and $\beta$ and non-negative reals $x,y$
\begin{align*}
x^\alpha y^\beta &= 
(x^{\alpha+\beta})^{\frac{\alpha}{\alpha + \beta}}
+ (y^{\alpha+\beta})^{\frac{\beta}{\alpha + \beta}}
\leq \frac{\alpha}{\alpha + \beta} x^{\alpha+\beta} + 
\frac{\beta}{\alpha + \beta} y^{\alpha+\beta}
\end{align*}
Therefore by the Binomial Theorem and the fact that $\binom{m}{k} = \binom{m}{m-k}$
\begin{align*}
(x_1 + x_2)^m &= \sum_{k=0}^m \binom{m}{k} x_1^k x_2^{m-k}
\leq \sum_{k=0}^m \binom{m}{k} \left(\frac{k}{m} x_1^m + \frac{m-k}{m} x_2^m \right) \\
&= \begin{cases}
\left(x_1^m + x_2^m \right) \sum_{k=0}^{\floor{\frac{m}{2}}} \binom{m}{k} \text{ if $m$ is odd}\\
\left(x_1^m + x_2^m \right) \lbrace\sum_{k=0}^{\frac{m}{2} -1} \binom{m}{k} + \frac{1}{2}\binom{m}{m/2} \rbrace \text{ if $m$ is even}\\
\end{cases} \\
&= 2^{m-1} \left(x_1^m + x_2^m \right)
\end{align*}

Now an easy induction shows that the result holds for any $n=2^k$:
\begin{align*}
\left( \frac{x_1 + \dotsm + x_{2^k}}{2^k} \right)
&\leq
\frac{1}{2} \left \lbrace \left(\frac{x_1 + \dotsm + x_{2^{k-1}}}{2^{k-1}} \right) +
\left(\frac{x_{2^{k-1} + 1} + \dotsm + x_{2^{k}}}{2^{k-1}} \right) \right \rbrace \\
&\leq \frac{x_1^m + \dotsm + x_{2^k}^m}{2^k}
\end{align*}

It remains to extend the result to arbitrary $n$.  Suppose that $2^{k-1} \leq n < 2^k$ and 
define 
\begin{align*}
A &= \left ( \frac{x_1^m + \dotsm + x_n^m}{n} \right)^{1/m}
\end{align*}
they by the result for $2^k$ we get
\begin{align*}
\left( \frac{x_1^m + \dotsm + x_n^m + (2^k - n)A}{2^k} \right)^m 
&\leq
\frac{x_1^m + \dotsm + x_n^m +  (2^k - n)A^m}{2^k} = A^m
\end{align*}
If we take $m^{th}$ roots and collect terms involving $A$ we get $\frac{x_1 + \dotsm + x_k}{k} \leq A$.
Now take $m^{th}$ power and use the definition of $A$ to get the
result.

TODO: I actually think we need this result for non-integral $m$.  Get
the full blown power mean inequality from Steele and then fix up the
BDG vector inequality.
\end{proof}

\begin{thm}\label{StirlingsFormula}
\end{thm}
\end{document}