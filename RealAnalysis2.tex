\section{More Real Analysis}
Holding area for more advanced topics in real analysis that are
eventually required (and in some cases there may be some topics that I
am just interested in).
\subsection{Topological Spaces}
\begin{defn}
\begin{itemize}
\item[(i)]A topological space is said to be \emph{separable} if and
  only if it has a countable dense subset.
\item[(ii)]A topological space is said to be \emph{first countable} if and
  only if every point has a countable local base.
\item[(ii)]A topological space is said to be \emph{second countable} if and
  only if every the topology has a countable base.
\end{itemize}
\end{defn}
\begin{lem}A metric space is separable if and only if it is second countable.
\end{lem}
\begin{proof}
TODO:
outline of proof is to pick a countable dense subset $\lbrace x_n
\rbrace$ and then pick the open balls $B(x_n; \frac{1}{m})$ for $m \in
\naturals$.  Show this is a base of the topology.
\end{proof}
\begin{defn}Given a topological space $(X, \mathcal{T})$ the Baire
  $\sigma$-algebra is smallest $\sigma$-algebra for which all bounded
  continuous functions are measurable.  Equivalently 
\begin{align*}
Ba(X,\mathcal{T}) &= \sigma(\lbrace f^{-1}(U) \mid U \subset \reals
\text{ is open; } f \in C_b(X,\reals)\rbrace)
\end{align*}
\end{defn}
\begin{lem}For every topological space $(X, \mathcal{T})$, $Ba(X)
  \subset \mathcal{B}(X)$.  For a metric space $(S,d)$, $Ba(S) = \mathcal{B}(S)$.
\end{lem}
\begin{lem}
To see the inclusion $Ba(X)
  \subset \mathcal{B}(X)$, note that by continuity of $f \in
  C_b(X;\reals)$, every set $f^{-1}(U)$ is open.

Now suppose $(S,d)$ is a metric space.  To show $\mathcal{B}(S)
\subset Ba(S)$, it suffices if we show every closed set $F \subset S$
can be written as $f^{-1}(G)$ where $G \subset \reals$ is closed and
$f \in C_b(S; \reals)$.  By the triangle inequality (see e.g. Lemma
\ref{DistanceToSetLipschitz}) we know
that $g(x) = d(x, F)$ is continuous (in fact Lipschitz) and by Lemma
\ref{MaxMinOfLipschitz} we know that $f(x) = d(x, F) \wedge 1$ is also
Lipschitz and therefore $f(x) \in C_b(S; \reals)$.  Because $F$ is
closed we also know that $F = f^{-1}(\lbrace 0 \rbrace)$ and we are done.
\end{lem}

TODO: How much this stuff on regularity can be extended to outer
measures????  I want to understand the overlap with the results in
Evans and Gariepy.

\begin{lem}\label{InnerRegularSetsSigmaAlgebra}Let $X$ be a Hausdorff topological space, $\mathcal{A}$
  a $\sigma$-algebra on $X$ and $\mu$ a finite tight measure.  Then
\begin{align*}
\mathcal{R} &= \lbrace A \in \mathcal{A} \mid A \text { and } A^c
\text{ are $\mu$-inner regular} \rbrace
\end{align*}
is a $\sigma$-algebra.  The same is true if the condition is replaced
by sets that are $\mu$-closed inner regular (without the requirement
that $\mu$ is tight).
\end{lem}
\begin{proof}
By definition, $\mathcal{R}$ is closed under complement.  By
assumption that $\mu$ is tight we have $X \in \mathcal{R}$ so all that
needs to be shown is closure under countable union.

Assume $A_1, A_2, \dots \in \mathcal{R}$ and let $\epsilon>0$ be
given.  By finiteness of $\mu$, $\mu(\cup_{n=1}^\infty A_n) < \infty$ and
continuity of measure (Lemma \ref{ContinuityOfMeasure}) there exists $M>0$ such that $\mu(\cup_{n=1}^M
A_n) > \mu(\cup_{n=1}^\infty A_n) - \epsilon$.
 By assumption that $A_n \in \mathcal{R}$ and finiteness of $\mu$, for each
$A_n$ there exists a compact $K_n$ such that $\mu(A_n \setminus K_n) <
\frac{\epsilon}{2^n}$ and there exists compact $L_n$ such that $\mu(A_n^c \setminus L_n) <
\frac{\epsilon}{2^n}$. Let
\begin{align*}
K &= \cup_{n=1}^M K_n \\
L &= \cap_{n=1}^\infty L_n
\end{align*}
and note that both $K$ and $L$ are compact (in the latter case,
because X is Hausdorff we know that each $L$ is closed hence the
intersection is a closed subset of a compact set hence compact).
Furthermore we can compute
\begin{align*}
\mu(\cup_{n=1}^\infty A_n \setminus K) &= \mu(\cup_{n=1}^\infty A_n
\setminus \cup_{n=1}^M K_n)  \\
&= \mu(\cup_{n=1}^M A_n
\setminus \cup_{n=1}^M K_n)  + \mu(\cup_{n=1}^\infty A_n \setminus \cup_{n=1}^M A_n
\setminus \cup_{n=1}^M K_n)\\
&\leq \mu(\cup_{n=1}^M A_n \setminus K_n)  + \mu(\cup_{n=1}^\infty A_n
\setminus \cup_{n=1}^M A_n)\\
&\leq \sum_{n=1}^M(A_n \setminus K_n)  + \epsilon \\
&\leq 3 \epsilon
\end{align*}
and
\begin{align*}
\mu((\cup_{n=1}^\infty A_n)^c \setminus L) &=\mu(\cap_{n=1}^\infty
A_n^c \setminus \cap_{n=1}^\infty L_n) \\
 &=\mu(\cap_{n=1}^\infty
A_n^c \cap \cup_{n=1}^\infty L_n^c) \\
 &=\mu(\cup_{n=1}^\infty \cap_{m=1}^\infty
A_m^c \cap L_n^c) \\
 &\leq \mu(\cup_{n=1}^\infty 
A_n^c \cap L_n^c) \\
 &\leq \sum_{n=1}^\infty (
A_n^c \setminus L_n) \\
&\leq 2 \epsilon
\end{align*}

TODO: The closed inner regular case...
\end{proof}

TODO:  In metric space, tightness is equivalent to inner regularity.
Then Ulam's Theorem that finite measures on separable metric spaces
are automatically inner regular.  Also finite measures on arbitrary
metric spaces are closed inner regular as well as outer regular.

\begin{lem}\label{FiniteMeasuresOnMetricSpacesAreClosedInnerRegular}
Let $(S,d)$ be a metric space and $\mu$ be a Borel measure on $(S,
\mathcal{B}(S))$, the $\mu$ is closed inner regular.  If in addition
$\mu$ is a finite measure then it is outer regular.
\end{lem}
\begin{proof}
Let $U$ be an open set in $S$.  Then $U^c$ is closed and the
function $f(x) = d(x, U^c)$ is continuous.  If we define 
\begin{align*}
F_n &= f^{-1}([1/n, \infty))
\end{align*}
then each $F_n$ is closed, $F_1 \subset F_2 \subset \cdots$ and
$\cup_{n=1}^\infty F_n = U$.  By continuity of measure (Lemma
\ref{ContinuityOfMeasure}) we know that $\lim_{n \to \infty} \mu(F_n)
= \mu(U)$.  So this shows that every open set is inner closed
regular.  Furthermore it is trivial to note that $U^c$ is inner closed
regular because it is closed.  

By Lemma \ref{InnerRegularSetsSigmaAlgebra} we know know that 
\begin{align*}
\mathcal{B}(S) &\subset \mathcal{R} = \lbrace A \subset S \mid A
\text{ and } A^c
\text{ are inner closed regular}  \rbrace
\end{align*}

Outer regularity follows from taking complements and using the
finiteness of $\mu$.
\end{proof}

If we add the criterion that the metric space is separable, then we
can upgrade the regularity to inner regularity.
\begin{lem}\label{SeparableInnerRegularTight}Let $(S,d)$ be a separable metric space and $\mu$ be a Borel measure on $(S,
\mathcal{B}(S))$, then $\mu$ is inner regular if and only if it is tight.
\end{lem}
\begin{proof}
Clearly inner regularity implies tightness (which is just inner
regularity of the set $S$), so it suffices to show
that tightness implies inner regularity.

Suppose that $\mu$ is a tight measure.  By Lemma
\ref{InnerRegularSetsSigmaAlgebra} it suffices to show that both open
and closed sets are inner regular.

Pick $\epsilon >0$ and select $K \subset S$ a compact set such that $\mu(S \setminus K) < \frac{\epsilon}{2}$.
By Lemma \ref{FiniteMeasuresOnMetricSpacesAreClosedInnerRegular} we
know that for any Borel set $B$ there exists a closed set $F \subset
B$ such that $\mu(B \setminus F) < \frac{\epsilon}{2}$.  Note that $F
\cap K$ is compact.   We have
\begin{align*}
\mu(B \setminus (F \cap K)) &\leq \mu(B \cap F^c) + \mu(B \cap K^c) \leq \mu(B \cap F^c) + \mu(S \cap K^c) < \epsilon
\end{align*}
\end{proof}
\begin{thm}[Ulam's Theorem]\label{UlamsTheorem}Let $(S,d)$ be a separable metric space and $\mu$ be a Borel measure on $(S,
\mathcal{B}(S))$, then $\mu$ is inner regular.
\end{thm}
\begin{proof}
By Lemma \ref{SeparableInnerRegularTight} it suffices to show that $\mu$ is tight.  Pick
$\epsilon > 0$ and we construct a compact set $K \subset S$ such that
$\mu(S \setminus K) < \epsilon$.  Let
$\overline{B}(x,r)$ denote the closed ball of radius $r$ around $x \in
S$.  Pick
a countable dense subset $x_1, x_2, \dotsc \in S$.  For each $m \in
\naturals$, by density of $\lbrace x_n \rbrace$, we know $\cap_{n=1}^\infty \left ( S
\setminus \cup_{j=1}^n \overline{B}(x_j, \frac{1}{m}) \right ) =
\emptyset$, thus by
continuity of measure (Lemma \ref{ContinuityOfMeasure}) there exists
$N_m > 0$ such that $\mu(S
\setminus \cup_{j=1}^n \overline{B}(x_j, \frac{1}{m}) < \frac{\epsilon}{2^m}$ for
all $n \geq N_m$.
If we define
\begin{align*}
K &= \cap_{m=1}^\infty \cup_{j=1}^{N_m} \overline{B}(x_j, \frac{1}{m})
\end{align*}
we claim that $K$ is compact.  Note that $K$ is easily seen to be
closed as it is an intersection of a finite union of closed balls.
Since $S$ is complete this implies that $K$ is also complete.  Also it
is easy to see that $K$ is totally bounded since by construction we
have demonstrated a cover by a finite number of balls of radius
$\frac{1}{m}$ for each $m \in \naturals$.  So by Theorem
\ref{CompactnessInMetricSpaces} we know $K$ is compact.

To finish the result we claim $\mu(S \setminus K) < \epsilon$:
\begin{align*}
\mu(S \setminus K) 
&= \mu(S \cap \left(\cap_{m=1}^\infty
  \cup_{j=1}^{N_m} \overline{B}(x_j, \frac{1}{m})\right)^c) \\
&= \mu(S \cap \cup_{m=1}^\infty \left(
  \cup_{j=1}^{N_m} \overline{B}(x_j, \frac{1}{m})\right)^c) \\
&= \mu(\cup_{m=1}^\infty S \setminus 
  \cup_{j=1}^{N_m} \overline{B}(x_j, \frac{1}{m})) \\
&\leq \sum_{m=1}^\infty \mu( S \setminus 
  \cup_{j=1}^{N_m} \overline{B}(x_j, \frac{1}{m})) \\
&< \epsilon
\end{align*}
\end{proof}
\subsection{Riesz Representation}

\begin{defn}Let $\mu$ be a measure on the Borel $\sigma$-algebra of a
  topological space $S$.  
\begin{itemize}
\item[(i)] A Borel set $B$ is \emph{inner regular} if for
 $\mu(B) = \sup_{K \subset B} \mu(K)$ where $K$
  is compact. $\mu$ is inner regular if every Borel set is inner regular.
\item[(ii)]A Borel set $B$ is \emph{outer regular} if $\mu(B) = \inf_{U \supset B} \mu(U)$ where $U$
  is open.  A measure $\mu$ is outer regular if every Borel set
  $B$ is outer regular.
\item[(iii)] $\mu$ is \emph{locally finite} if every $x \in S$ has an
  open neighborhood $x \in U$ such that $\mu(U) < \infty$.
\item[(iv)] $\mu$ is a \emph{Radon measure} it is inner regular and
  locally finite.
\item[(v)] $\mu$ is a \emph{Borel measure} when?????  In some cases
  I've seen it required that $\mu(B) < \infty$ for all Borel sets $B$
  (reference?) and in other cases just that the Borel sets are measurable.
\item[(vi)]A Borel set  $B$ is \emph{closed regular} if $\mu(B) = \inf_{F \subset B} \mu(F)$ where $F$
  is closed (e.g. Dudley pg. 224).  A measure $\mu$ is closed regular
  if every Borel set $B$ is closed regular.
\item[(vii)] If $\mu$ is finite, then we say \emph{tight} if and only if
  X is inner regular (e.g. Dudley pg. 224).
\end{itemize}
\end{defn}

\begin{defn}Let $\mu$ be a Borel measure on a Hausdorff topological space. A set measurable set $A$ is called \emph{regular} if 
\begin{itemize}
\item[(i)]$\mu(A) = \inf_{U \supset A} \mu(A)$ where $U$ are open
\item[(ii)]$\mu(A) = \sup_{F \subset A} \mu(A)$ where $F$ are closed 
\end{itemize}
TODO: Alternative def assumes that $F$ are compact (see inner
regularity above).  If every measurable set is regular then $\mu$ is
said to be regular.  Note that if we assume the definition of
regularity uses compact inner approximations then regular measures are
inner and outer regular (although inner and outer regularity refer to
only Borel sets; is that a meaningful distinction?)
\end{defn}


TODO: Regularity of outer measures and the relationship to regularity
of measures as defined above (see Evans and Gariepy).  Note that
regularity of outer measure implies that if we take an outer measure $\mu$
and the measure on the $\mu$-measurable sets and then take the induced
outer measure we get $\mu$ back if and only $\mu$ is a regular outer
measure.  Evans and Gariepy show that Radon outer measures on
$\reals^n$ are inner
regular as measures on the $\mu$-measurable sets.  Note that inner
regular is part of the most common definition of Radon measure so
their result can be taken as showing a weaker definition of Radon
measure holds on $\reals^n$ (but also they phrase everything in terms
of outer measures...).

\begin{thm}Let $\mu$ be a finite Borel measure on a metric space $S$,
  then $\mu$ is closed regular.  If $\mu$ is tight then $\mu$ is regular.
\end{thm}
TODO: Specialize the definition of Radon measure in the presence of
more assumptions on $X$ (in particular local compactness,
$\sigma$-compactness, second countability).

TODO: Are Radon measures automatically outer regular?

Tao proves Riesz representation under assumption of local compactness
and $\sigma$-compactness.

Kallenberg proves Riesz representation under assumption of local
compactness and second countability (this is more general than the Tao
result as $\sigma$-compactness implies second countability (I think)).

Evans and Gareipy prove Riesz representation only on $\reals^n$.

Dudley proves Riesz representation of compact Hausdorff spaces (in
which cases the dual measures are Baire measures instead of Radon
measures).  Dudley does not really discuss Radon measures.  

\subsection{Covering Theorems in $\reals^n$}

Since our purposes have been to understand probability theory we have
hitherto avoided making assumptions that we are dealing with
$\reals^n$.  While this decision has benefits, it has drawbacks as
well.  Among them we lose sight of some history but also some of the
beautiful and deep understanding of the measure theory of the reals.
TODO: Vitali and Besicovich.

\subsection{Hausdorff Measure}

\subsubsection{Introduction}

In this section we discuss the construction of a family of outer
measures on $\reals^n$ called \emph{Hausdorff measures}.  Note the
construction can be generalized to metric spaces.  The following is
motivation why a tool like Hausdorff measure may be useful.  Suppose
very specifically that we are
in $\reals^3$, then the Lebesgue product measure essentially
corresponds to a notion of volume.  What about the surface area of a
$2$-dimensional object or the length of a $1$-dimensional object?  As
you may have learned in advanced calculus these ideas can indeed be
describe in great generality by the notion of differential forms.
However, the formalism of forms usually has some notion of smoothness
associated with it (hence the adjective differential); a natural question to ask is whether one can fine
a purely measure theoretic approach to the problem.  Hausdorff measures
provide one answer to this question.   The broad form of the theory
is perhaps a bit more general than one might expect; for any space
there is a Hausdorff outer measure for every real number $s$.  The
case of integers
$s=1$ corresponds to arclength, $s=2$ surface area, $s=3$ volume and so
on.  Measures with $s$ non-integral are
\emph{fractal}.  On $\reals^n$, the Hausdorff measure with $s=n$ is equal to
Lebesgue measure and any Hausdorff measure with $s > n$ is trivial
(gives $0$ measure to all sets).  We'll prove all of this and more in
what follows.

\subsubsection{Construction of Hausdorff Measure}

The following technical Lemma is useful (we'll use it when
discussing Hausdorff outer measures).  If the reader is in a hurry,
no harm will come from skipping over this result and returning to it
when the need arises.  Note that if the user is only interested in
probability theory this result may never come up.
\begin{lem}[Caratheodory Criterion]\label{CaratheodoryCriterion}Let $(S,d)$ be a metric space with an outer measure $\mu^*$.
  Then $\mu^*$ is a Borel outer measure (i.e. all Borel sets are
  $\mu^*$-measurable) if and only if $\mu^*(A \cup B) = \mu^*(A) +
  \mu^*(B)$ for all $A,B$ such that $d(A, B) > 0$.
\end{lem}
\begin{proof}
We begin with the only if direction.  Let $A$ be a closed set in $S$
and let $B \subset S$.  To show $A$ is $\mu^*$-measurable it suffices
to show $\mu^*(B) \geq \mu^*(A \cap B) + \mu^*(A^c \cap B)$.  Since
the inequality is trivially satisfied when $\mu^*(B) = \infty$ we
assume that $\mu^*(B) < \infty$.  For
every $n \in \naturals$, let $A_n =
\lbrace x \in S \mid d(x, A) \leq \frac{1}{n} \rbrace$.  By definition
of $A_n$, we have $d(A,
A_n^c) > \frac{1}{n} > 0$ and therefore $d(A \cap B, A_n^c \cap B)
> \frac{1}{n} > 0$.  Now by our assumption, we can conclude $\mu^*((A
\cap B) \cup (A_n^c \cap B)) = \mu^*(A \cap B) +
\mu^*(A_n^c \cap B)$.

We claim that $\lim_{n \to \infty} \mu^*(A_n^c \cap B) = \mu^*(A^c
\cap B)$.  Note that if we prove the claim the Lemma is proven because then we have
\begin{align*}
\mu^*(B) &\geq \mu^*((A
\cap B) \cup (A_n^c \cap B)) & & \text{by monotonicity}\\
&= \mu^*(A \cap B) +
\mu^*(A_n^c \cap B)
\end{align*}
and taking limits we have 
\begin{align*}
\mu^*(B) \geq \lim_{n\to \infty} \mu^*(A \cap B) +
\mu^*(A_n^c \cap B) &= \mu^*(A \cap B) +
\mu^*(A^c \cap B)
\end{align*}
To prove the claim we observe that monotonicity of outer measure
implies that $\lim_{n \to \infty} \mu^*(A_n^c \cap B) \leq \mu^*(A^c
\cap B)$ so we just need to
work on the opposite inequality.  To see it first define the rings
around $A$
\begin{align*}
R_n &= \lbrace x \mid \frac{1}{n+1} < d(x, A) \leq \frac{1}{n} \rbrace
\end{align*}
and note that because $A$ is closed, for each $n$,
\begin{align*}
A^c &= \lbrace x \in S \mid d(x, A) > 0 \rbrace \\
&=\lbrace x \in S \mid d(x, A) > n \rbrace \cup \bigcup_{m=n}^\infty \lbrace
x \in S \mid \frac{1}{m+1} < d(x, A) \leq \frac{1}{m} \rbrace \\
&=A_n^c \cup \bigcup_{m=n}^\infty R_m
\end{align*}
It follows that
$A^c \cap B = A_n^c \cap B \cup \cup_{m=n}^\infty
R_m \cap B$ and therefore by subadditivity of outer measure 
\begin{align*}
\mu^*(A^c \cap B) \leq \mu^*(A_n^c \cap B) + \sum_{m=n}^\infty
\mu^*(R_m \cap B)
\end{align*}
The claim will follow if we can show $\lim_{n \to \infty} \sum_{m=n}^\infty
\mu^*(R_m \cap B)=0$ which in turn will follow if we can show that $\sum_{m=1}^\infty
\mu^*(R_m \cap B)$ converges.  By construction, $d(R_{2m}, R_{2n})
> 0$ and therefore $d(R_{2m} \cap B, R_{2n} \cap B)
> 0$ for any $m \neq n$.  So if we consider only the even terms of the
series we can use our hypothesis to show that for any $n$
\begin{align*}
\sum_{m=1}^n \mu^*(R_{2m} \cap B) &= \mu^*(\cup_{m=1}^n
R_{2m} \cap B) \leq \mu^*(B) < \infty
\end{align*}
and by taking limits $\sum_{m=1}^\infty \mu^*(R_{2m} \cap B) \leq \mu^*(B)$
The same argument applies to the odd indexed terms and we get
\begin{align*}
\sum_{m=1}^\infty \mu^*(R_{m} \cap B) &\leq 2\mu^*(B) < \infty
\end{align*}
The claim and the Lemma follow.
\end{proof}


TODO:  Here I am taking the path of Evans and Gariepy and normalizing
Hausdorff measure so that $\mathcal{H}^n = \lambda_n$.  I am not sure
if this winds up being inconvenient when one considers Hausdorff
measure in arbitrary metric spaces (nor do I know whether we'll bother
considering Hausdorff measures in metric spaces).

\begin{lem}Let $\lambda_n$ be Lebesgue measure on $\reals^n$, then
  $\lambda_n(B(0, 1)) = \frac{\pi^{n/2}}{\Gamma(\frac{n}{2} + 1)}$.
\end{lem}
\begin{proof}
TODO
\end{proof}

\begin{defn}Let $(S,d)$ be a metric space and $A \subset S$, the
  \emph{diameter} of $A$ is 
\begin{align*}
\diam(A) &= \sup \lbrace d(x,y) \mid x,y \in A \rbrace
\end{align*}
\end{defn}

\begin{defn}Let $(S,d)$ be a metric space, $0 \leq s < \infty$ and $0
  < \delta$.  Then for $A \subset S$,
\begin{align*}
\mathcal{H}^s_\delta(A) &= \inf \lbrace \sum_{n=1}^\infty \alpha(s)
\left ( \frac{\diam(C_n)}{2}\right )^s \mid A \subset
\cup_{n=1}^\infty C_n \text{ where } \diam(C_n) \leq \delta \text{ for
  all } n\rbrace
\end{align*}
where 
\begin{align*}
\alpha(s) &= \frac{\pi^{n/2}}{\Gamma(\frac{n}{2} + 1)}
\end{align*}
For $A$ and $s$ as above define
\begin{align*}
\mathcal{H}^s(A) &= \lim_{\delta \to 0} \mathcal{H}_\delta^s(A) = \sup_{\delta>0} \mathcal{H}_\delta^s(A)
\end{align*}
\end{defn}