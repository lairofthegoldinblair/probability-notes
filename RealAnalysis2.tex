\section{More Real Analysis}
Holding area for more advanced topics in real analysis that are
eventually required (and in some cases there may be some topics that I
am just interested in).
\subsection{Topological Spaces}
\begin{lem}\label{OpenAlternative}A set $U \subset X$ is open if and only if for every $x \in
  U$ there is an open set $V \subset U$ such that $x \in V$.
\end{lem}
\begin{proof}
Suppose $U$ is open and $x \in U$, then let $V = U$.

Suppose for every $x \in U$ there exist an open set $V_x$ such that $x
\in V_x \subset U$.  Note that $\cup_x V_x \subset U$ because each
$V_x \subset U$ and on the other hand $\cup_x V_x \supset U$ since
every $x \in U$ satisfies $x \in V_x$.  Thus $U = \cup_x V_x$ which
shows that $U$ is open.
\end{proof}
\begin{defn}A mapping $f : X \to Y$ between topological spaces is said
  to be \emph{continuous} if and only if $f^{-1}(V)$ is open in $X$
  for every $V$ open in $Y$.
\end{defn}
\begin{defn}A mapping $f : X \to Y$ between topological spaces is said
  to be \emph{continuous at x} if and only if for every $V$ open in
  $Y$ such that $f(x) \in V$, there exists an open set $U$ in $X$ with $x \in U$ and $f(U)
  \subset V$.
\end{defn}
\begin{lem}A mapping $f : X \to Y$ between topological spaces is
  continuous if and only if it is continuous at $x$ for every $x \in X$.
\end{lem}
\begin{proof}
Suppose $f$ is continuous and let $x \in X$ and $V$ be open in $Y$
with $f(x) \in V$.  By continuity of $f$, we know that $f^{-1}(V)$ is
open in $X$ and $x \in f^{-1}(V)$.  By Lemma \ref{OpenAlternative} we
can pick an open set $U$ such that $x \in U$ and $U \subset
f^{-1}(V)$.  It follows that $f(U) \subset V$.

Now suppose $f$ is continuous at every $x \in X$ and let $V$ be open
in $Y$.  If $x \in f^{-1}(V)$ then $f$ is continuous at $x$ hence
there exists and open $U$ such that $x \in U$ and $f(U) \subset V$.
It follows that $U \subset f^{-1}(V)$ and by Lemma
\ref{OpenAlternative}  we have shown that $f^{-1}(V)$ is open.
\end{proof}

\begin{defn}A \emph{base} of a topology $\mathcal{T}$ at a point $x
  \in X$ is a collection
  of sets $\mathcal{B}$ such that for every open set $U \in
  \mathcal{T}$ such that $x \in U$ there exists a $B \in \mathcal{B}$ such
  that $x \in B \subset U$.  A base of a topology is a collection of
  sets that is a base at all points $x \in X$.
\end{defn}

\begin{lem}A set $\mathcal{B}$ of sets $B \subset X$ is a base of a
  topology if and only if for every $x \in X$ there exists $B \in
  \mathcal{B}$ such that $x \in B$ and for every $A, B \in
  \mathcal{B}$ and $x \in A \cap B$ there exists $C \in \mathcal{B}$
  such that $x \in C \subset A \cap B$.
\end{lem}
\begin{proof}
Suppose $\mathcal{B}$ satisfies the hypothesized conditions and let
\begin{align*}
\tau &= \lbrace U \subset X \mid \text { for every } x \in U \text{
  there exists } B \in \mathcal{B} \text{ such that } x \in B \subset
U \rbrace
\end{align*}
It is certainly the case that $\mathcal{B}\subset \tau$ and we claim that $\tau$ is a topology.  Certainly $\emptyset \in \tau$.
Let $U_\alpha$ for $\alpha \in \Lambda$ are sets in $\tau$.
Then if $x \in \cup_{\alpha \in \Lambda} U_\alpha$ there exists an
$\alpha \in \Lambda$ such that $x \in U_\alpha$ and by hypothesis we
pick $B$ such that $x \in B \subset U_\alpha \subset  \cup_{\alpha \in
  \Lambda} U_\alpha$.  If $U_1, \dotsc, U_n \in \tau$ and $x \in U_1
\cap \dotsc \cap U_n$ then there exists $B_1, \dotsc, B_n$ such that
$x \in B_j \subset U_j$ for $j = 1, \dotsc, n$ and therefore $x \in
B_1 \cap \dotsc \cap B_n \subset U_1 \cap \dotsc \cap U_n$.  A simple
induction on the hypothesis shows that $B_1 \cap \dotsc \cap B_n \in \mathcal{B}$.
Because $\mathcal{B}$ is cover of $X$ we have $X = \cup_{B \in
  \mathcal{B}} B \in \tau$ and therefore $\tau$ is a topology.  By the
definition of $\tau$ it is immediate that $\mathcal{B}$ is a base of
the topology.
\end{proof}


\begin{defn}
\begin{itemize}
\item[(i)]A topological space is said to be \emph{separable} if and
  only if it has a countable dense subset.
\item[(ii)]A topological space is said to be \emph{first countable} if and
  only if every point has a countable local base.
\item[(ii)]A topological space is said to be \emph{second countable} if and
  only if every the topology has a countable base.
\end{itemize}
\end{defn}
\begin{lem}A metric space is separable if and only if it is second countable.
\end{lem}
\begin{proof}
TODO:
outline of proof is to pick a countable dense subset $\lbrace x_n
\rbrace$ and then pick the open balls $B(x_n; \frac{1}{m})$ for $m \in
\naturals$.  Show this is a base of the topology.
\end{proof}

TODO: The goal of the next set of results is to show that separable
complete metric spaces are Borel.


The following appears in Royden as Theorem 8.11 (with proof delgated
to exercises)
\begin{lem}Let $X$ be a Hausdorff topological space, $Y$ be a
  complete metric space and $Z \subset X$ be a dense subset.  If $f :
  Z \to X$ is a homeomorphism then $Z$ is a countable intersection of
  open sets.
\end{lem}
\begin{proof}
For each $n$ let 
\begin{align*}
O_n &= \lbrace x \in X \mid \text{there exists $U$
  open with $x \in U$ and $\diam(f(U \cap Z)) < \frac{1}{n}$} \rbrace
\end{align*}
Note that $O_n$ is open because for any $x \in O_n$ by definition we
have the open set $U$ that provides the evidence that $x \in O_n$;
$U$ also provides the evidence that proves that every $y \in U$
belongs to $O_n$.  Also
note that $Z \subset O_n$ since for any $n$, by continuity of $f$ at $x \in Z$ and Lemma
\ref{OpenAlternative}  we
can find an open $U \subset X$ such that $x \in U \cap Z$ and $f(U \cap Z) \subset B(f(x),
\frac{1}{2n})$ (sets of the form $U \cap Z$ being precisely the open
sets in $Z$).

Now define $E = \cap_n O_n$.  As noted we know $Z \subset E$ so we
will be done if we can show $E
\subset Z$ as well.  Let $x \in E$; we will construct $z \in Z$
such that $x = z$.  For each $n$ pick $U_n$ such $x \in U_n$ and $\diam(f(U_n \cap Z)) <
\frac{1}{n}$ and let $x_n$ be an arbitrary point in $\cap_{j=1}^n
U_j \cap Z$ (the intersection is non-empty because $Z$ is dense in
$X$).  
For every $n$ and $m \geq n$ we have by construction that $x_n
\in U_n$ and $x_m \in U_n$ hence $d(f(x_n), f(x_m)) < \frac{1}{n}$.
Therefore $f(x_n)$ is Cauchy in
$Y$ and by completeness of $Y$ we know that $f(x_n)$ converges to a
value $y \in Y$ with $d(y, f(x_n)) \leq \frac{1}{n}$.  
Because $f$ is a homeomorphism we know that 
there is a unique $z \in Z$ such that $f(z) = y$; we claim that $x =
z$.  Suppose that $x
\neq z$, then by the Hausdorff property on $X$ we can pick open sets $U$ and
$V$ such that $U \cap V = \emptyset$, $x \in U$ and $z \in V$.  Since
$f$ is a homeomorphism, we know $f(Z \cap V)$ is open and contains
$f(z)$ hence for sufficiently large $n$, $f^{-1}(B(f(z), \frac{1}{n}))
\subset Z \cap V \subset V$.  On
the other hand, by the definition of $x$ we have $U_{2n}$ open such that
$x \in U_{2n}$ and $\diam(f(Z \cap U_{2n})) < \frac{1}{2n}$.  By openness of
$U \cap U_{2n}$ and density of $Z$ we know there is a $w \in U \cap
U_{2n} \cap Z$.  Putting these observations together we have
\begin{align*}
d(f(w), f(z)) &\leq  d(f(w), f(x_{2n})) + d(f(x_{2n}), f(z)) 
< \frac{1}{2n} + \frac{1}{2n} = \frac{1}{n}
\end{align*}
which implies $w \in V$ providing a contradiction of $U \cap V =
\emptyset$ hence we conclude $x = z$.
\end{proof}

\begin{defn}Given a topological space $(X, \mathcal{T})$ the Baire
  $\sigma$-algebra is smallest $\sigma$-algebra for which all bounded
  continuous functions are measurable.  Equivalently 
\begin{align*}
Ba(X,\mathcal{T}) &= \sigma(\lbrace f^{-1}(U) \mid U \subset \reals
\text{ is open; } f \in C_b(X,\reals)\rbrace)
\end{align*}
\end{defn}
\begin{lem}For every topological space $(X, \mathcal{T})$, $Ba(X)
  \subset \mathcal{B}(X)$.  For a metric space $(S,d)$, $Ba(S) = \mathcal{B}(S)$.
\end{lem}
\begin{proof}
To see the inclusion $Ba(X)
  \subset \mathcal{B}(X)$, note that by continuity of $f \in
  C_b(X;\reals)$, every set $f^{-1}(U)$ is open.

Now suppose $(S,d)$ is a metric space.  To show $\mathcal{B}(S)
\subset Ba(S)$, it suffices if we show every closed set $F \subset S$
can be written as $f^{-1}(G)$ where $G \subset \reals$ is closed and
$f \in C_b(S; \reals)$.  By the triangle inequality (see e.g. Lemma
\ref{DistanceToSetLipschitz}) we know
that $g(x) = d(x, F)$ is continuous (in fact Lipschitz) and by Lemma
\ref{MaxMinOfLipschitz} we know that $f(x) = d(x, F) \wedge 1$ is also
Lipschitz and therefore $f(x) \in C_b(S; \reals)$.  Because $F$ is
closed we also know that $F = f^{-1}(\lbrace 0 \rbrace)$ and we are done.
\end{proof}

TODO: How much this stuff on regularity can be extended to outer
measures????  I want to understand the overlap with the results in
Evans and Gariepy.

\begin{lem}\label{InnerRegularSetsSigmaAlgebra}Let $X$ be a Hausdorff topological space, $\mathcal{A}$
  a $\sigma$-algebra on $X$ and $\mu$ a finite tight measure.  Then
\begin{align*}
\mathcal{R} &= \lbrace A \in \mathcal{A} \mid A \text { and } A^c
\text{ are $\mu$-inner regular} \rbrace
\end{align*}
is a $\sigma$-algebra.  The same is true if the condition is replaced
by sets that are $\mu$-closed inner regular (without the requirement
that $\mu$ is tight).
\end{lem}
\begin{proof}
By definition, $\mathcal{R}$ is closed under complement.  By
assumption that $\mu$ is tight we have $X \in \mathcal{R}$ so all that
needs to be shown is closure under countable union.

Assume $A_1, A_2, \dots \in \mathcal{R}$ and let $\epsilon>0$ be
given.  By finiteness of $\mu$, $\mu(\cup_{n=1}^\infty A_n) < \infty$ and
continuity of measure (Lemma \ref{ContinuityOfMeasure}) there exists $M>0$ such that $\mu(\cup_{n=1}^M
A_n) > \mu(\cup_{n=1}^\infty A_n) - \epsilon$.
 By assumption that $A_n \in \mathcal{R}$ and finiteness of $\mu$, for each
$A_n$ there exists a compact $K_n$ such that $\mu(A_n \setminus K_n) <
\frac{\epsilon}{2^n}$ and there exists compact $L_n$ such that $\mu(A_n^c \setminus L_n) <
\frac{\epsilon}{2^n}$. Let
\begin{align*}
K &= \cup_{n=1}^M K_n \\
L &= \cap_{n=1}^\infty L_n
\end{align*}
and note that both $K$ and $L$ are compact (in the latter case,
because X is Hausdorff we know that each $L$ is closed hence the
intersection is a closed subset of a compact set hence compact).
Furthermore we can compute
\begin{align*}
\mu(\cup_{n=1}^\infty A_n \setminus K) &= \mu(\cup_{n=1}^\infty A_n
\setminus \cup_{n=1}^M K_n)  \\
&= \mu(\cup_{n=1}^M A_n
\setminus \cup_{n=1}^M K_n)  + \mu(\cup_{n=1}^\infty A_n \setminus \cup_{n=1}^M A_n
\setminus \cup_{n=1}^M K_n)\\
&\leq \mu(\cup_{n=1}^M A_n \setminus K_n)  + \mu(\cup_{n=1}^\infty A_n
\setminus \cup_{n=1}^M A_n)\\
&\leq \sum_{n=1}^M(A_n \setminus K_n)  + \epsilon \\
&\leq 3 \epsilon
\end{align*}
and
\begin{align*}
\mu((\cup_{n=1}^\infty A_n)^c \setminus L) &=\mu(\cap_{n=1}^\infty
A_n^c \setminus \cap_{n=1}^\infty L_n) \\
 &=\mu(\cap_{n=1}^\infty
A_n^c \cap \cup_{n=1}^\infty L_n^c) \\
 &=\mu(\cup_{n=1}^\infty \cap_{m=1}^\infty
A_m^c \cap L_n^c) \\
 &\leq \mu(\cup_{n=1}^\infty 
A_n^c \cap L_n^c) \\
 &\leq \sum_{n=1}^\infty (
A_n^c \setminus L_n) \\
&\leq 2 \epsilon
\end{align*}

TODO: The closed inner regular case...
\end{proof}

TODO:  In metric space, tightness is equivalent to inner regularity.
Then Ulam's Theorem that finite measures on separable metric spaces
are automatically inner regular.  Also finite measures on arbitrary
metric spaces are closed inner regular as well as outer regular.

\begin{lem}\label{FiniteMeasuresOnMetricSpacesAreClosedInnerRegular}
Let $(S,d)$ be a metric space and $\mu$ be a Borel measure on $(S,
\mathcal{B}(S))$, the $\mu$ is closed inner regular.  If in addition
$\mu$ is a finite measure then it is outer regular.
\end{lem}
\begin{proof}
Let $U$ be an open set in $S$.  Then $U^c$ is closed and the
function $f(x) = d(x, U^c)$ is continuous.  If we define 
\begin{align*}
F_n &= f^{-1}([1/n, \infty))
\end{align*}
then each $F_n$ is closed, $F_1 \subset F_2 \subset \cdots$ and
$\cup_{n=1}^\infty F_n = U$.  By continuity of measure (Lemma
\ref{ContinuityOfMeasure}) we know that $\lim_{n \to \infty} \mu(F_n)
= \mu(U)$.  So this shows that every open set is inner closed
regular.  Furthermore it is trivial to note that $U^c$ is inner closed
regular because it is closed.  

By Lemma \ref{InnerRegularSetsSigmaAlgebra} we know know that 
\begin{align*}
\mathcal{B}(S) &\subset \mathcal{R} = \lbrace A \subset S \mid A
\text{ and } A^c
\text{ are inner closed regular}  \rbrace
\end{align*}

Outer regularity follows from taking complements and using the
finiteness of $\mu$.
\end{proof}

If we add the criterion that the metric space is separable, then we
can upgrade the regularity to inner regularity.
\begin{lem}\label{SeparableInnerRegularTight}Let $(S,d)$ be a separable metric space and $\mu$ be a Borel measure on $(S,
\mathcal{B}(S))$, then $\mu$ is inner regular if and only if it is tight.
\end{lem}
\begin{proof}
Clearly inner regularity implies tightness (which is just inner
regularity of the set $S$), so it suffices to show
that tightness implies inner regularity.

Suppose that $\mu$ is a tight measure.  By Lemma
\ref{InnerRegularSetsSigmaAlgebra} it suffices to show that both open
and closed sets are inner regular.

Pick $\epsilon >0$ and select $K \subset S$ a compact set such that $\mu(S \setminus K) < \frac{\epsilon}{2}$.
By Lemma \ref{FiniteMeasuresOnMetricSpacesAreClosedInnerRegular} we
know that for any Borel set $B$ there exists a closed set $F \subset
B$ such that $\mu(B \setminus F) < \frac{\epsilon}{2}$.  Note that $F
\cap K$ is compact.   We have
\begin{align*}
\mu(B \setminus (F \cap K)) &\leq \mu(B \cap F^c) + \mu(B \cap K^c) \leq \mu(B \cap F^c) + \mu(S \cap K^c) < \epsilon
\end{align*}
\end{proof}
\begin{thm}[Ulam's Theorem]\label{UlamsTheorem}Let $(S,d)$ be a separable metric space and $\mu$ be a Borel measure on $(S,
\mathcal{B}(S))$, then $\mu$ is inner regular.
\end{thm}
\begin{proof}
By Lemma \ref{SeparableInnerRegularTight} it suffices to show that $\mu$ is tight.  Pick
$\epsilon > 0$ and we construct a compact set $K \subset S$ such that
$\mu(S \setminus K) < \epsilon$.  Let
$\overline{B}(x,r)$ denote the closed ball of radius $r$ around $x \in
S$.  Pick
a countable dense subset $x_1, x_2, \dotsc \in S$.  For each $m \in
\naturals$, by density of $\lbrace x_n \rbrace$, we know $\cap_{n=1}^\infty \left ( S
\setminus \cup_{j=1}^n \overline{B}(x_j, \frac{1}{m}) \right ) =
\emptyset$, thus by
continuity of measure (Lemma \ref{ContinuityOfMeasure}) there exists
$N_m > 0$ such that $\mu(S
\setminus \cup_{j=1}^n \overline{B}(x_j, \frac{1}{m}) < \frac{\epsilon}{2^m}$ for
all $n \geq N_m$.
If we define
\begin{align*}
K &= \cap_{m=1}^\infty \cup_{j=1}^{N_m} \overline{B}(x_j, \frac{1}{m})
\end{align*}
we claim that $K$ is compact.  Note that $K$ is easily seen to be
closed as it is an intersection of a finite union of closed balls.
Since $S$ is complete this implies that $K$ is also complete.  Also it
is easy to see that $K$ is totally bounded since by construction we
have demonstrated a cover by a finite number of balls of radius
$\frac{1}{m}$ for each $m \in \naturals$.  So by Theorem
\ref{CompactnessInMetricSpaces} we know $K$ is compact.

To finish the result we claim $\mu(S \setminus K) < \epsilon$:
\begin{align*}
\mu(S \setminus K) 
&= \mu(S \cap \left(\cap_{m=1}^\infty
  \cup_{j=1}^{N_m} \overline{B}(x_j, \frac{1}{m})\right)^c) \\
&= \mu(S \cap \cup_{m=1}^\infty \left(
  \cup_{j=1}^{N_m} \overline{B}(x_j, \frac{1}{m})\right)^c) \\
&= \mu(\cup_{m=1}^\infty S \setminus 
  \cup_{j=1}^{N_m} \overline{B}(x_j, \frac{1}{m})) \\
&\leq \sum_{m=1}^\infty \mu( S \setminus 
  \cup_{j=1}^{N_m} \overline{B}(x_j, \frac{1}{m})) \\
&< \epsilon
\end{align*}
\end{proof}

\begin{lem}Let $(S,d)$ be a separable metric space, then $X$ is
  homeomorphic to a subset of $[0,1]^{\integers_+}$.
\end{lem}
\begin{proof}
Pick a countable dense subset $x_1, x_2, \dotsc$ of $S$ and define 
$f : S \in [0,1]^{\integers_+}$ by 
\begin{align*}
f(x) &= \left ( \frac{d(x_1, x)}{1 + d(x_1, x)}, \frac{d(x_2, x)}{1 +
    d(x_2, x)}, \dotsc \right )
\end{align*}
For any $z \neq y$ we find $\epsilon > 0$ such that $B(z ; \epsilon)
\cap B( y ; \epsilon) = \emptyset$ and then using density of $x_1,
x_2, \dotsc$ to pick an $x_n$ such that $d(z,x_n) < \epsilon$ and
$d(y, x_n) \geq \epsilon$ showing $f(z) \neq f(y)$.  

TODO: Finish
\end{proof}

\begin{lem}For $f,g \in C([0,\infty) ; \reals)$ define
\begin{align*}
\rho(f,g) &= \sum_{n=1}^\infty \frac{1}{2^n} \sup_{0 \leq t \leq n}
(\abs{f(t) - g(t)} \wedge 1)
\end{align*}
then $\rho$ is a metric on $C([0,\infty) ; \reals)$ and $C([0,\infty);
\reals)$ is complete and separable with respect to this metric.
\end{lem}
\begin{proof}
It is clear that $\rho(f,f) = 0$ and furthermore if $\rho(f,g) = 0$
then $f = g$ on every interval $[0,n]$ and therefore $f = g$.
Symmetry and the triangle inequality of $\rho$ is immediate from the
corresponding properties of the absolute value (TODO: OK the triangle
inequality may need a bit more of an argument).

We claim that the set of polynomials with rational coefficients is
dense in $C([0,\infty); \reals)$.  Pick $f \in C([0,\infty); \reals)$
and let $\epsilon > 0$ be given.  Now take $m > 0$ sufficiently large
so that $1/2^m < \epsilon / 2$ and by the Stone Weierstrass Theorem \label{StoneWeierstrassApproximation} we
pick a polynomial with rational coefficients $p$ such that $\sup_{0
  \leq t \leq m} \abs{f(t) - p(t)} < \epsilon/2$ then we have
\begin{align*}
\rho(f,p) &\leq \sum_{n=1}^m\frac{1}{2^n} \sup_{0 \leq t \leq n}
\abs{f(t) - p(t)} + \sum_{n=m+1}^\infty \frac{1}{2^n} \\
&\leq \sup_{0 \leq t \leq m}
\abs{f(t) - p(t)} \sum_{n=1}^m\frac{1}{2^n} + \epsilon/2 <\epsilon
\end{align*}

Completeness follows from arguing over intervals $[0,n]$.  Suppose
$f_n$ is a Cauchy sequence in $C([0,\infty); \reals)$.  Given
$\epsilon > 0$ and $n > 0$ we can find $N > 0$ such that $\rho(f_m,
f_N) < \epsilon/2^n$ for all $m \geq N$.  Thus $\sup_{0 \leq t \leq n}
\abs{f_m(t) - f_N(t)} < \epsilon$ for all $m \geq N$ so we see that
$f_n$ is uniformly  Cauchy on every interval $[0,n]$.  By completeness
of $C([0,n];\reals)$ we know that the pointwise limit of $f_n$ exists
on every $[0,n]$ and is a continuous function.  Therefore we have a
limit $f$ defined on $[0,\infty)$ and since continuity is a local
property $f \in C([0,\infty); \reals)$.  It remains to show that $f_n$
converges to $f$ in the metric $\rho$.  This follows arguing as we
have above.  Let $\epsilon > 0$ be given and choose $n > 0$ such that
$\frac{1}{2^n} < \epsilon/2$ and choose $N > 0$ such that $\sup_{0 \leq t \leq n}
\abs{f_m(t) - f_N(t)} < \epsilon/2$ and then observe
\begin{align*}
\rho(f_m, f_N) &\leq \sum_{k=1}^n \frac{1}{2^k} \sup_{0 \leq t \leq k}
\abs{f_m(t) - f_N(t)} + \sum_{k=n+1}^\infty \frac{1}{2^k}  < \epsilon
\end{align*}
\end{proof}

\begin{defn}Given a function $f : [0,T] \to \reals$ the \emph{modulus
    of continuity} is the function
\begin{align*}
m(T, f, \delta) &= \sup_{\substack{\abs{s - t} < \delta \\
0 \leq s,t \leq T}} \abs{f(s) - f(t)}
\end{align*}
\end{defn}

\begin{lem}For fixed $T > 0$ and $\delta > 0$, $m(T, f, \delta)$ is a
  continuous function on $C([0,\infty); \reals)$.  For fixed $T > 0$
  and function $f : \reals \to \reals$, $m(T,f,\delta)$ is
  nonincreasing in $\delta$ and 
\begin{align*}
\lim_{\delta \to 0} m(T, f, \delta) = 0
\end{align*}
provided $f \in C([0,\infty); \reals)$.
\end{lem}
\begin{proof}
To see continuity on $C([0,\infty); \reals)$ let $f \in C([0,\infty);
\reals)$, $T > 0$, $\delta > 0$ and $\epsilon > 0$ be given and pick $g$ that $\rho(f,g) <
\epsilon/2^{\ceil{T}+1}$.
From the definition of the metric $\rho$ for any $n > 0$, $\sup_{0 \leq t \leq n} \abs{f(t) - g(t)} \wedge
1 \leq 2^n \epsilon$, so for any $T > 0$, 
\begin{align*}
\sup_{0 \leq t \leq T} \abs{f(t) - g(t)} \wedge
1 &\leq \sup_{0 \leq t \leq \ceil{T}} \abs{f(t) - g(t)} \wedge
1 \leq \epsilon/2
\end{align*}  
Therefore by the triangle inequality,
\begin{align*}
\sup_{\substack{
\abs{s -t} < \delta \\
0 \leq s,t \leq T}} \abs{g(s) - g(t)} \wedge 1 
&\leq 
\sup_{\substack{
\abs{s -t} < \delta \\
0 \leq s,t \leq T}} \left ( \abs{g(s) - f(s)} + \abs{f(s) - f(t)} + \abs{f(t)
- g(t)} \right ) \wedge 1 \\
&\leq \epsilon/2 + 
\sup_{\substack{
\abs{s -t} < \delta \\
0 \leq s,t \leq T}} \abs{f(s) - f(t)} \wedge 1 + \epsilon/2
\end{align*}
and therefore arguing with the roles of $f$ and $g$ reversed shows 
$\abs{m(T, f, \delta) - m(T, g, \delta) } \leq \epsilon$.

The fact that $m(T, f, \delta)$ is descreasing in $\delta$ is clear
because the definition shows that for $\delta_1 \leq \delta_2$ we
have 
\begin{align*}
\lbrace \abs{f(t) - f(s) } \mid 0 \leq s,t \leq T \text{ and }
  \abs{s-t} < \delta_1 \rbrace 
&\subset 
\lbrace \abs{f(t) - f(s) } \mid 0 \leq s,t \leq T \text{ and }
  \abs{s-t} < \delta_2 \rbrace
\end{align*} and therefore $m(T, f, \delta_2) \leq
  m(T, f, \delta_1)$.

Lastly if we suppose $f \in C([0,\infty); \reals)$ then $f$ is
uniformly continuous on $[0,T]$ for every $T > 0$ (Theorem
\ref{UniformContinuityOnCompactSets}).  Thus given an $\epsilon > 0$
there exists $\delta>0$ such that 
\begin{align*}
\sup_{\substack{
\abs{s -t} < \delta \\
0 \leq s,t \leq T}} \abs{f(s) - f(t)} < \epsilon
\end{align*}
which shows $\lim_{\delta \to 0} m(T, f, \delta) = 0$.
\end{proof}
The following Theorem is a version of the Arzela-Ascoli Theorem of
real analysis.
\begin{thm}[Arzela-Ascoli Theorem]\label{ArzelaAscoliTheorem}A set $A
  \subset C([0,\infty); \reals)$ is relatively compact if and only if 
\begin{itemize}
\item[(i)]$\sup_{f \in A} \abs{f(0)} < \infty$
\item[(ii)]$\lim_{\delta \to 0} \sup_{f \in A} m(T, f, \delta) = 0$
  for all $T > 0$.
\end{itemize}
\end{thm}
\begin{proof}
To see the necessity of condition (i), observe that $\overline{A}$ is
compact and by completeness of $C([0,\infty); \reals)$ we know that
$\overline{A}$ comprises continuous functions.  Therefore we know that
$A \subset \overline{A} \subset \cup_{n=1}^\infty \lbrace f \in
C([0,\infty) \mid
\abs{f(0)} < n\rbrace$.  Since each $\lbrace f \in
C([0,\infty) \mid
\abs{f(0)} < n\rbrace$ is easily seen to be an open set, by
compactness of $\overline{A}$ we have a finite subcover which implies
there exists an $N$ such that $A \subset \overline{A} \subset \lbrace f \in
C([0,\infty) \mid
\abs{f(0)} < N\rbrace$.

To see the necessity of condition (ii), fix $\epsilon > 0$, $T > 0$
and define for each $\delta > 0$ the set 
\begin{align*}
F_\delta &= \lbrace f \in \overline{A} \mid m(T, f, \delta) \geq
\epsilon \rbrace
\end{align*}
By continuity of $m(T, f, \delta)$ we know that $F_\delta$ is closed.
Since $F_\delta \subset \overline{A}$ with $\overline{A}$ compact we
conclude that $F_\delta$ is compact.  Furthermore since for fixed $f
\in \overline{A}$ continuity (more specifically uniform continuity on compact
sets) implies $\lim_{\delta \to 0} m(T,f,\delta) = 0$, we know that
$\cap_{\delta > 0} F_\delta = \emptyset$.  By nestedness and
compactness of the
$F_\delta$ we know that there is some specific $\delta>0$ for which $F_\delta =
\emptyset$ (Lemma \ref{IntersectionOfNestedCompactSets}) and (ii) is established.

To see the sufficiency of conditions (i) and (ii).
\end{proof}

\subsection{Riesz Representation}

\begin{defn}Let $\mu$ be a measure on the Borel $\sigma$-algebra of a
  topological space $S$.  
\begin{itemize}
\item[(i)] A Borel set $B$ is \emph{inner regular} if for
 $\mu(B) = \sup_{K \subset B} \mu(K)$ where $K$
  is compact. $\mu$ is inner regular if every Borel set is inner regular.
\item[(ii)]A Borel set $B$ is \emph{outer regular} if $\mu(B) = \inf_{U \supset B} \mu(U)$ where $U$
  is open.  A measure $\mu$ is outer regular if every Borel set
  $B$ is outer regular.
\item[(iii)] $\mu$ is \emph{locally finite} if every $x \in S$ has an
  open neighborhood $x \in U$ such that $\mu(U) < \infty$.
\item[(iv)] $\mu$ is a \emph{Radon measure} it is inner regular and
  locally finite.
\item[(v)] $\mu$ is a \emph{Borel measure} when?????  In some cases
  I've seen it required that $\mu(B) < \infty$ for all Borel sets $B$
  (reference?) and in other cases just that the Borel sets are measurable.
\item[(vi)]A Borel set  $B$ is \emph{closed regular} if $\mu(B) = \inf_{F \subset B} \mu(F)$ where $F$
  is closed (e.g. Dudley pg. 224).  A measure $\mu$ is closed regular
  if every Borel set $B$ is closed regular.
\item[(vii)] If $\mu$ is finite, then we say \emph{tight} if and only if
  X is inner regular (e.g. Dudley pg. 224).
\end{itemize}
\end{defn}

\begin{defn}Let $\mu$ be a Borel measure on a Hausdorff topological space. A set measurable set $A$ is called \emph{regular} if 
\begin{itemize}
\item[(i)]$\mu(A) = \inf_{U \supset A} \mu(A)$ where $U$ are open
\item[(ii)]$\mu(A) = \sup_{F \subset A} \mu(A)$ where $F$ are closed 
\end{itemize}
TODO: Alternative def assumes that $F$ are compact (see inner
regularity above).  If every measurable set is regular then $\mu$ is
said to be regular.  Note that if we assume the definition of
regularity uses compact inner approximations then regular measures are
inner and outer regular (although inner and outer regularity refer to
only Borel sets; is that a meaningful distinction?)
\end{defn}


TODO: Regularity of outer measures and the relationship to regularity
of measures as defined above (see Evans and Gariepy).  Note that
regularity of outer measure implies that if we take an outer measure $\mu$
and the measure on the $\mu$-measurable sets and then take the induced
outer measure we get $\mu$ back if and only $\mu$ is a regular outer
measure.  Evans and Gariepy show that Radon outer measures on
$\reals^n$ are inner
regular as measures on the $\mu$-measurable sets.  Note that inner
regular is part of the most common definition of Radon measure so
their result can be taken as showing a weaker definition of Radon
measure holds on $\reals^n$ (but also they phrase everything in terms
of outer measures...).

\begin{thm}Let $\mu$ be a finite Borel measure on a metric space $S$,
  then $\mu$ is closed regular.  If $\mu$ is tight then $\mu$ is regular.
\end{thm}
TODO: Specialize the definition of Radon measure in the presence of
more assumptions on $X$ (in particular local compactness,
$\sigma$-compactness, second countability).

TODO: Are Radon measures automatically outer regular?

Tao proves Riesz representation under assumption of local compactness
and $\sigma$-compactness.

Kallenberg proves Riesz representation under assumption of local
compactness and second countability (this is more general than the Tao
result as $\sigma$-compactness implies second countability (I think)).

Evans and Gareipy prove Riesz representation only on $\reals^n$.

Dudley proves Riesz representation of compact Hausdorff spaces (in
which cases the dual measures are Baire measures instead of Radon
measures).  Dudley does not really discuss Radon measures.  

\subsection{Covering Theorems in $\reals^n$}

Since our purposes have been to understand probability theory we have
hitherto avoided making assumptions that we are dealing with
$\reals^n$.  While this decision has benefits, it has drawbacks as
well.  Among them we lose sight of some history but also some of the
beautiful and deep understanding of the measure theory of the reals.
TODO: Vitali and Besicovich.

\subsection{Hausdorff Measure}

\subsubsection{Introduction}

In this section we discuss the construction of a family of outer
measures on $\reals^n$ called \emph{Hausdorff measures}.  Note the
construction can be generalized to metric spaces.  The following is
motivation why a tool like Hausdorff measure may be useful.  Suppose
very specifically that we are
in $\reals^3$, then the Lebesgue product measure essentially
corresponds to a notion of volume.  What about the surface area of a
$2$-dimensional object or the length of a $1$-dimensional object?  As
you may have learned in advanced calculus these ideas can indeed be
describe in great generality by the notion of differential forms.
However, the formalism of forms usually has some notion of smoothness
associated with it (hence the adjective differential); a natural question to ask is whether one can fine
a purely measure theoretic approach to the problem.  Hausdorff measures
provide one answer to this question.   The broad form of the theory
is perhaps a bit more general than one might expect; for any space
there is a Hausdorff outer measure for every real number $s$.  The
case of integers
$s=1$ corresponds to arclength, $s=2$ surface area, $s=3$ volume and so
on.  Measures with $s$ non-integral are
\emph{fractal}.  On $\reals^n$, the Hausdorff measure with $s=n$ is equal to
Lebesgue measure and any Hausdorff measure with $s > n$ is trivial
(gives $0$ measure to all sets).  We'll prove all of this and more in
what follows.

\subsubsection{Construction of Hausdorff Measure}

The following technical Lemma is useful (we'll use it when
discussing Hausdorff outer measures).  If the reader is in a hurry,
no harm will come from skipping over this result and returning to it
when the need arises.  Note that if the user is only interested in
probability theory this result may never come up.
\begin{lem}[Caratheodory Criterion]\label{CaratheodoryCriterion}Let $(S,d)$ be a metric space with an outer measure $\mu^*$.
  Then $\mu^*$ is a Borel outer measure (i.e. all Borel sets are
  $\mu^*$-measurable) if and only if $\mu^*(A \cup B) = \mu^*(A) +
  \mu^*(B)$ for all $A,B$ such that $d(A, B) > 0$.
\end{lem}
\begin{proof}
We begin with the only if direction.  Let $A$ be a closed set in $S$
and let $B \subset S$.  To show $A$ is $\mu^*$-measurable it suffices
to show $\mu^*(B) \geq \mu^*(A \cap B) + \mu^*(A^c \cap B)$.  Since
the inequality is trivially satisfied when $\mu^*(B) = \infty$ we
assume that $\mu^*(B) < \infty$.  For
every $n \in \naturals$, let $A_n =
\lbrace x \in S \mid d(x, A) \leq \frac{1}{n} \rbrace$.  By definition
of $A_n$, we have $d(A,
A_n^c) > \frac{1}{n} > 0$ and therefore $d(A \cap B, A_n^c \cap B)
> \frac{1}{n} > 0$.  Now by our assumption, we can conclude $\mu^*((A
\cap B) \cup (A_n^c \cap B)) = \mu^*(A \cap B) +
\mu^*(A_n^c \cap B)$.

We claim that $\lim_{n \to \infty} \mu^*(A_n^c \cap B) = \mu^*(A^c
\cap B)$.  Note that if we prove the claim the Lemma is proven because then we have
\begin{align*}
\mu^*(B) &\geq \mu^*((A
\cap B) \cup (A_n^c \cap B)) & & \text{by monotonicity}\\
&= \mu^*(A \cap B) +
\mu^*(A_n^c \cap B)
\end{align*}
and taking limits we have 
\begin{align*}
\mu^*(B) \geq \lim_{n\to \infty} \mu^*(A \cap B) +
\mu^*(A_n^c \cap B) &= \mu^*(A \cap B) +
\mu^*(A^c \cap B)
\end{align*}
To prove the claim we observe that monotonicity of outer measure
implies that $\lim_{n \to \infty} \mu^*(A_n^c \cap B) \leq \mu^*(A^c
\cap B)$ so we just need to
work on the opposite inequality.  To see it first define the rings
around $A$
\begin{align*}
R_n &= \lbrace x \mid \frac{1}{n+1} < d(x, A) \leq \frac{1}{n} \rbrace
\end{align*}
and note that because $A$ is closed, for each $n$,
\begin{align*}
A^c &= \lbrace x \in S \mid d(x, A) > 0 \rbrace \\
&=\lbrace x \in S \mid d(x, A) > n \rbrace \cup \bigcup_{m=n}^\infty \lbrace
x \in S \mid \frac{1}{m+1} < d(x, A) \leq \frac{1}{m} \rbrace \\
&=A_n^c \cup \bigcup_{m=n}^\infty R_m
\end{align*}
It follows that
$A^c \cap B = A_n^c \cap B \cup \cup_{m=n}^\infty
R_m \cap B$ and therefore by subadditivity of outer measure 
\begin{align*}
\mu^*(A^c \cap B) \leq \mu^*(A_n^c \cap B) + \sum_{m=n}^\infty
\mu^*(R_m \cap B)
\end{align*}
The claim will follow if we can show $\lim_{n \to \infty} \sum_{m=n}^\infty
\mu^*(R_m \cap B)=0$ which in turn will follow if we can show that $\sum_{m=1}^\infty
\mu^*(R_m \cap B)$ converges.  By construction, $d(R_{2m}, R_{2n})
> 0$ and therefore $d(R_{2m} \cap B, R_{2n} \cap B)
> 0$ for any $m \neq n$.  So if we consider only the even terms of the
series we can use our hypothesis to show that for any $n$
\begin{align*}
\sum_{m=1}^n \mu^*(R_{2m} \cap B) &= \mu^*(\cup_{m=1}^n
R_{2m} \cap B) \leq \mu^*(B) < \infty
\end{align*}
and by taking limits $\sum_{m=1}^\infty \mu^*(R_{2m} \cap B) \leq \mu^*(B)$
The same argument applies to the odd indexed terms and we get
\begin{align*}
\sum_{m=1}^\infty \mu^*(R_{m} \cap B) &\leq 2\mu^*(B) < \infty
\end{align*}
The claim and the Lemma follow.
\end{proof}


TODO:  Here I am taking the path of Evans and Gariepy and normalizing
Hausdorff measure so that $\mathcal{H}^n = \lambda_n$.  I am not sure
if this winds up being inconvenient when one considers Hausdorff
measure in arbitrary metric spaces (nor do I know whether we'll bother
considering Hausdorff measures in metric spaces).

\begin{lem}Let $\lambda_n$ be Lebesgue measure on $\reals^n$, then
  $\lambda_n(B(0, 1)) = \frac{\pi^{n/2}}{\Gamma(\frac{n}{2} + 1)}$.
\end{lem}
\begin{proof}
TODO
\end{proof}

\begin{defn}Let $(S,d)$ be a metric space and $A \subset S$, the
  \emph{diameter} of $A$ is 
\begin{align*}
\diam(A) &= \sup \lbrace d(x,y) \mid x,y \in A \rbrace
\end{align*}
\end{defn}

\begin{defn}Let $(S,d)$ be a metric space, $0 \leq s < \infty$ and $0
  < \delta$.  Then for $A \subset S$,
\begin{align*}
\mathcal{H}^s_\delta(A) &= \inf \lbrace \sum_{n=1}^\infty \alpha(s)
\left ( \frac{\diam(C_n)}{2}\right )^s \mid A \subset
\cup_{n=1}^\infty C_n \text{ where } \diam(C_n) \leq \delta \text{ for
  all } n\rbrace
\end{align*}
where 
\begin{align*}
\alpha(s) &= \frac{\pi^{n/2}}{\Gamma(\frac{n}{2} + 1)}
\end{align*}
For $A$ and $s$ as above define
\begin{align*}
\mathcal{H}^s(A) &= \lim_{\delta \to 0} \mathcal{H}_\delta^s(A) = \sup_{\delta>0} \mathcal{H}_\delta^s(A)
\end{align*}
\end{defn}