\chapter{Probability}
Here we begin to focus on the special case of probability spaces.  The
development of measure theoretic probabilty begins with the
assumptions that we are given a
\begin{defn}A \emph{probability space} is a measure space $(\Omega,
  \mathcal{A}, P)$ such that $\probability{\Omega} = 1$.
\end{defn}
Given a measurable function $\xi : \Omega \to (S,
\mathcal{S})$ we will refer to $\xi$ as a \emph{random
  element} of $S$.  The special case of a measurable function $\xi : \Omega \to (\reals, \mathcal{B}(\reals))$
is called a \emph{random variable}.  For a random element $\xi$, by
Lemma \ref{PushforwardMeasure} we can
push forward the probability measure to get a measure $(\pushforward{\xi}{P})$ called the \emph{distribution} or \emph{law} of $\xi$. One
sometimes writes $\mathcal{L}(\xi)$ to denote the distribution of
$\xi$ and one writes $\xi \eqdist \eta$ to denote that $\xi$ and $\eta$ have
the same distribution. 

In probability theory the existence of a probability space is critical
to the formal development of the theory however it is almost always
the case that one is only concerned with results that don't depend on
the exact choice of probability space.  To make this statement more
precise we introduce 
\begin{defn}A probability space $(\Omega^\prime, \mathcal{A}^\prime,
  P^\prime)$ is an \emph{extension} of $(\Omega, \mathcal{A},
  P)$ if there is a surjective measurable map $\pi : \Omega^\prime \to
  \Omega$ such that $P = \pushforward{\pi}{P^\prime}$.
\end{defn}

A result is considered properly \emph{probabilistic} if it is
preserved under extension of sample space.  Note that this is a
cultural statement and not a mathematical theorem.  As an example of a
probabilistic concept, we have the ability to talk about an
\emph{event} $A$ and its probability $\probability{A}$ since given any
$\pi$  we can unambiguously refer to $\pi^{-1}(A)$ as the same event
in $\Omega^\prime$ and we know that probability is preserved.  As an
example of a non-probabilistic concept we have the cardinality of an
event.

In keeping with the philosophy that probabilistic results are
invariant under extension of the underlying probability space, we will
follow common practice and try to avoid explicit mention of the underlying
probability space in many definitions and results.  

\begin{defn}Given a random vector $\xi = (\xi_1, \dots, \xi_n)$ in
  $\reals^n$ we define the \emph{distribution function} to be 
\begin{align*}
F(x_1, \dots, x_n) = \probability {\cap_{i=1}^n \left ( \xi_i \leq x_i
    \right ) }
\end{align*}
\end{defn}
\begin{lem}\label{DistributionFunctionCharacterizeProbability}Let $\xi$ and $\eta$ be random vectors in $\reals^n$ with
  distribution functions $F$ and $G$, then
  $\xi \eqdist \eta$ if and only if $F = G$.
\end{lem}
\begin{proof}
For any $(x_1, \dotsc, x_n), (y_1, \dotsc, y_n) \in \reals^n$ we have
\begin{align*}
(-\infty, x_1] \times \cdots \times (-\infty,  x_n] \cap (-\infty,
  y_1] \times \cdots \times (-\infty, y_n]
&= (-\infty, x_1 \minop y_1] \times \cdots \times (-\infty,  x_n  \minop y_n] 
\end{align*}
Thus  $\reals^n$ plus all sets of the form $(-\infty, x_1] \times \cdots \times (-\infty,
  x_n]$ for $(x_1, \dotsc, x_n) \in \reals^n$ form a $\pi$-system that contains $\reals^n$.  By Corollary \ref{MultidimensionalIntervalsGenerateBorel}
we know that this $\pi$-system generates $\mathcal{B}(\reals^n)$ so the result follows from 
the Uniqueness of Finite Measure Lemma \ref{UniquenessOfMeasure}.
\end{proof}

The construction of Lebesgue-Stieltjes measure shows that every
Borel probability measure on $\reals$ is determined uniquely by its distribution function.
\begin{lem}\label{CumulativeDistributionFunctions}Probability measures of $(\reals, \mathcal{B}(\reals))$ are
  in one to one correspondence with $F : \reals \to \reals$ that are
  right continuous, nondecreasing such that $\lim_{x \to -\infty} F(x)
  = 0$ and $\lim_{x \to \infty} F(x) = 1$ via the mapping $F(x) =
  \probability{(-\infty, x]}$.
\end{lem}
\begin{proof}Clearly any probability measure is locally finite so we
  apply Lemma \ref{LebesgueStieltjesMeasure} to create a 1-1
  correspondence with $\hat{F}$, 
  right continuous and nondecreasing such that $\probability{(a,b]} =
  \hat{F}(b) - \hat{F}(a)$.  Now define $F(x) = \hat{F}(x) +
  \probability{(-\infty, 0]}$.
\end{proof}

\begin{defn}The \emph{expectation} of a random variable $\xi$ on a
  probability space $(\Omega, \mathcal{A}, P)$ is
  defined to be 
\begin{align*}
\expectation{\xi} = \int \xi \, dP
\end{align*}
\end{defn}
A very useful corollary to the abstract change of variables Lemma
\ref{ChangeOfVariables} is the following
\begin{lem}[Expectation Rule]\label{ExpectationRule}Let $\xi$ be a random variable and $f: \reals \to \reals$
  be a Borel measurable function.  Then 
\begin{align*}
\expectation{f(\xi)} = \int f \, d (\pushforward{\xi}{P})
\end{align*}
In particular, 
\begin{align*}
\expectation{\xi} = \int x \, d (\pushforward{\xi}{P})
\end{align*}
\end{lem}
\begin{proof}This is just a restatement of Lemma
  \ref{ChangeOfVariables} for the special case of random variables and
  measurable functions on $\reals$.
\end{proof}

The following lemma is useful for relating tail bounds and expectations.
\begin{lem}\label{TailsAndExpectations}Let $\xi$ be a positive random variable with finite
  expectation.  Then $\expectation{\xi} = \int_0^\infty \probability{\xi \geq
    \lambda} d\lambda$.
\end{lem}
\begin{proof}
This is just an application of Tonelli's Theorem,
\begin{align*}
\int_0^\infty \probability{\xi \geq \lambda}d\lambda &= \int_0^\infty \left[\int
\characteristic{\xi \geq \lambda} dP\right] d\lambda \\
&= \int\left[\int_0^\infty \characteristic{\xi \geq \lambda} d\lambda\right] dP \\
&= \int\left[\int_0^{\xi} d\lambda\right] dP \\
&= \int \xi \, dP \\
&= \expectation{\xi} \\
\end{align*}
\end{proof}

\begin{lem}[Cauchy Schwartz Inequality]\label{CauchySchwartz}Let $\xi$
  and $\eta$ satisfy $\expectation{\xi^2}, \expectation{\eta^2} < \infty$ then $\xi \eta$ is integrable and
  $\expectation{\xi \eta}^2 \leq \expectation{\xi^2}
  \expectation{\eta^2}$.
\end{lem}
\begin{proof}
Since we have both $0 \leq (\xi + \eta)^2$ and $0 \leq (\xi - \eta)^2$
we have $\abs{\xi \eta} \leq \frac{1}{2}(\xi^2 + \eta^2)$ which shows
that $\xi \eta$ is integrable.

There are a host of different proofs of Cauchy Schwartz inequality.  Here is perhaps
the simplest one.  Note that for all $t \in \reals$, $0 \leq \expectation{(t \xi +
\eta)^2} = \expectation{\xi^2} t^2 + 2 \expectation{\xi \eta} t +
\expectation{\eta^2}$.  The quadratic formula implies that
$\sqrt{4 \expectation{\xi^2} \expectation{\eta^2} - (2
  \expectation{\xi \eta})^2 } \geq 0$ which in turn implies the
result.

The proof we just provided is probably the slickest one available but
has the disadvantage of being very specific to the quadratic case.  
There is a different proof of Cauchy Schwartz that we provide that
involves two steps that have a broader application.  The idea is to
derive Cauchy Schwartz from the trival fact that for all real numbers
$x,y$ we have $xy \leq \frac{x^2}{2} + \frac{y^2}{2}$ (which we used when showing
integrability of $\xi\eta$).  Applying this fact to $\xi$ and $\eta$ we see
that
\begin{align*}
\expectation{\xi \eta} &\leq \frac{\expectation{\xi^2}}{2} + \frac{\expectation{\eta^2}}{2}
\end{align*}
To finish the proof, we apply a \emph{normalization trick} by defining
$\hat{\xi} = \frac{\xi}{\sqrt{\expectation{\xi^2}}}$ and $\hat{\eta} =
\frac{\eta}{\sqrt{\expectation{\eta^2}}}$ so that
$\expectation{\hat{\xi^2}} = \expectation{\hat{\eta^2}} =1$.  Now we apply the above bound and linearity of expectation to see that
\begin{align*}
\frac{1}{\sqrt{\expectation{\xi^2}}\sqrt{\expectation{\eta^2}}}
\expectation{\xi\eta} &= \expectation{\hat{\xi}\hat{\eta}} \leq 1
\end{align*}
which yields the result.
\end{proof}

Applications of Cauchy Schwatz are ubiquitous in analysis.  Only
slightly less common are applications of the following
generalization.  First a definition
\begin{defn}Given any $p > 0$ and random variable $\xi$, the \emph{$L^p$
  norm} of $\xi$ is 
\begin{align*}
\norm{\xi}_p = \left (\expectation{\abs{\xi}^p} \right )^ {\frac{1}{p}}
\end{align*}
\end{defn}
\begin{lem}[H\"{o}lder Inequality]\label{Holder}Given $p,q,r > 0$ such
  that $\frac{1}{r} = \frac{1}{p} + \frac{1}{q}$ and random variables
  $\xi$ and $\eta$, we have
\begin{align*}
\norm{\xi \eta}_r \leq \norm{\xi }_p \norm{\eta}_q
\end{align*}
\end{lem}
\begin{proof}
We start by assuming that $r=1$.  The proof here is a direct generalization of the second proof we
provided for Cauchy Schwartz.  To get started we need to find a
generalization of the simple fact that $xy \leq \frac{x^2}{2} +
\frac{y^2}{2}$.  
 
The inequality we need is called Young's Inequality and is derived
from the following fact.  Let $f$ be an
continuous increasing function $f : [0,c] \to \reals$ such that $f(0)
= 0$.  Then the area interpretation of integral tells us that for $0
\leq a \leq c$ and $0 \leq b \leq f(c)$ we have
\begin{align*}
ab \leq \int_0^a f(x) \, dx + \int_0^b f^{-1}(x) \, dx
\end{align*}
with equality if and only if $b=f(a)$.  

For our case, we first assume that $r = 1$.  Define $f(x) =
x^{p-1}$ then observe that $f^{-1}(x) = x^{q-1}$ since
$1 = \frac{1}{p} + \frac{1}{q}$ is equivalent to $(p-1)(q-1) = 1$.
Therefore we have Young's Inequality, $ab \leq \frac{a^p}{p} +
\frac{b^q}{q}$.

Now applying the normalization trick by defining $\hat{\xi} =
\frac{\abs{\xi}}{\norm{\xi}_p}$ and $\hat{\eta} =
\frac{\abs{\eta}}{\norm{\eta}_q}$ so that ${\norm{\hat{\xi}}_p} =
{\norm{\hat{\eta}}_q} = 1$.  We now apply Young's Inequality to
$\hat{\xi}$ and $\hat{\eta}$ to see
\begin{align*}
\frac{1}{\norm{\hat{\xi}}_p \norm{\hat{\eta}}_q} \expectation{\abs{\xi \eta}} &=
\expectation{\hat{\xi}\hat{\eta}} \leq \frac{1}{p} + \frac{1}{q} = 1
\end{align*}

Lastly we generalize to general $r>0$.  Given $\frac{1}{r} =
\frac{1}{p} + \frac{1}{q}$ we define $\hat{p} = \frac{p}{r}$ and 
$\hat{q} = \frac{q}{r}$ so that $1 =
\frac{1}{\hat{p}} + \frac{1}{\hat{q}}$ and 
\begin{align*}
\expectation{\abs{\xi \eta}^r} &\leq \norm{\xi^r}_{\hat{p}}
\norm{\eta^r}_{\hat{q}} = \norm{\xi}_p^r \norm{\eta}_q^r 
\end{align*}
Taking $r^{th}$ roots we are done.
\end{proof}
\begin{cor}\label{IncreasingMoments}For $p > r > 0$ and any random variable $\xi$, we
  have $\norm{\xi}_r \leq \norm{\xi}_p$.
\end{cor}
\begin{proof}Define $q = \frac{p -r}{pr} > 0$ and apply H\"older's
  Inequality to see that $\norm{\xi}_r \leq \norm{\xi}_p \norm{1}_q = \norm{\xi}_p $.
\end{proof}
It worth noting that the corollary above is generally true on finite
measure spaces but fails for non-finite measure spaces (e.g. consider
$f(x) = \frac{1}{x}$ which has finite $L^p$ norm on $[1,\infty)$ for
$p > 1$ but infinite $L^1$ norm on $[1,\infty)$).

\section{Convexity and Jensen's Inequality}
\begin{defn}A function $\varphi : \reals^n \to \reals$ is said to be
  \emph{convex} if for all $x, y \in \reals^n$ and $t \in [0,1]$, we
  have
\begin{align*}
\varphi( tx + (1-t)y) \leq t\varphi(x) + (1-t)\varphi(y)
\end{align*}
$\varphi$ is said to be \emph{strictly convex} if it is convex and for
all $t \in (0,1)$, 
\begin{align*}
\varphi( tx + (1-t)y) < t\varphi(x) + (1-t)\varphi(y)
\end{align*}
\end{defn}

TODO: Convex functions are continuous; in fact they are locally Lipschitz (hence differentiable a.e. by Rademacher).

Convex functions are almost surely differentiable.
\begin{lem}\label{ThreeChordLemma}Let $\varphi : [a,b] \to \reals$ be convex.  Then for every
  $a < x < b$, we have
\begin{align*}
\frac{\varphi(x) - \varphi(a)}{x - a} &\leq \frac{\varphi(b) -
  \varphi(a)}{b - a}  \leq \frac{\varphi(b) - \varphi(x)}{b - x} 
\end{align*}
If $\varphi$ is strictly convex then the inequalities may be replaced
by strict inequalities.
\end{lem}
\begin{proof}
Note that we can write $x = t a + (1-t) b$
with $t = \frac{b-x}{b-a} \in [0,1]$.  So applying the definition of
convexity we know that $\varphi(x) \leq t \varphi(a) +
(1-t)\varphi(b)$ and using the fact that $1-t = \frac{x-a}{b-a}$ we get
\begin{align*}
\frac{\varphi(x) - \varphi(a)}{x - a} &\leq \frac{t \varphi(a) +
(1-t)\varphi(b) - \varphi(a)}{x - a} = \frac{1-t}{x-a} (\varphi(b) -
\varphi(a) ) = \frac{\varphi(b) -
  \varphi(a)}{b - a} 
\end{align*}
and in a similar way,
\begin{align*}
\frac{\varphi(b) - \varphi(x)}{b - x} &\geq \frac{ \varphi(b)  - t \varphi(a) -
(1-t)\varphi(b) }{b -x} = \frac{t}{b-x} (\varphi(b) -
\varphi(a) ) = \frac{\varphi(b) -
  \varphi(a)}{b - a} 
\end{align*}
It is clear from the definition of strict convexity that the
inequalities above may be replaced by strict inequalities if $\varphi$ is strictly convex.
\end{proof}
\begin{lem}\label{ConvexHasDini}Let $\varphi : [a,b] \to \reals$ be a convex function, then
  for every $x \in (a,b)$, $D^-\varphi(x)$ and $D^+\varphi(x)$ exist
  and furthermore for $a < x < y < b$ we have
\begin{align*}
D^-\varphi(x) &\leq D^+\varphi(x) \leq \frac{\varphi(y) -
  \varphi(x)}{y - x} \leq D^-\varphi(y) \leq D^+\varphi(y)
\end{align*}
If $\varphi$ is strictly convex then we have
\begin{align*}
D^+\varphi(x) &< \frac{\varphi(y) -  \varphi(x)}{y - x} < D^-\varphi(y)
\end{align*}
Thus each of $D^\pm \varphi$ is monotone and has at most a countable number of discontinuities.
\end{lem}
\begin{proof}
Lemma \ref{ThreeChordLemma} shows that for $a < x < b$ and $h > 0$,
$\frac{\varphi(x + h) - \varphi(x)}{h}$ is an increasing function of
$h$ bounded below by $\frac{\varphi(x) - \varphi(a)}{x - a}$.  Thus $D^+\varphi(x) = \lim_{h \downarrow 0} \frac{\varphi(x + h) -
  \varphi(x)}{h}$ is a decreasing limit hence exists.
Similarly $\frac{\varphi(x - h) -
  \varphi(x)}{-h}=\frac{\varphi(x)-\varphi(x - h)}{h}$ is an decreasing
function of $h$ bounded above by $\frac{\varphi(b) - \varphi(x)}{b - x}$.  Thus $D^-\varphi(x) = \lim_{h \downarrow 0} \frac{\varphi(x - h) -
  \varphi(x)}{-h}$ is a bounded increasing limit hence
exists.  

The inequalities follow directly from Lemma \ref{ThreeChordLemma}.
For example, since $D^+\varphi(x) = \lim_{h \downarrow 0} \frac{\varphi(x + h) -
  \varphi(x)}{h}$ and for all $x < x+h < y$, we have $\frac{\varphi(x + h) -
  \varphi(x)}{h} \leq \frac{\varphi(y) -
  \varphi(x)}{y-x}$ we get $D^+\varphi(x) \leq \frac{\varphi(y) -
  \varphi(x)}{y-x}$.  In the strictly convex case, we know that for
any $w$ with
$x < w < y$ we have  by what we have just shown and
another application of Lemma \ref{ThreeChordLemma}
\begin{align*}
D^+\varphi(x) &\leq \frac{\varphi(w) -
  \varphi(x)}{w-x} < \frac{\varphi(y) -
  \varphi(x)}{y-x}
\end{align*}  The case of $D^-\varphi(y)$ follows analogously.

The fact that $D^\pm \varphi$ has at most a countable number of
discontinuities follows from the fact that the functions are nondecreasing.
On every finite interval $[c,d] \subset (a,b)$ and $n \in \naturals$ know that right
hand and left hand limits exist hence all discontinuities are jump discontinuities.  For 
every $n \in \naturals$ the number of discontinuities of size bigger than $1/n$ is finite
(bounded by $(c-d)*n$).  Now take the union over all $n$ and a countable set of intervals $[c,d]$ covering
$(a,b)$.
\end{proof}
\begin{cor}\label{ConvexHasSubderivative}Let $\varphi : [a,b] \to
  \reals$ be convex then for $x \in (a,b)$ there exists constants $A,B
  \in \reals$ such that $Ay + B \leq \varphi(y)$ for all $y \in [a,b]$
  and $Ax + B = \varphi(x)$.  If $\varphi$ is strictly convex then we
  may assume that $Ay + B < \varphi(y)$ for $y \neq x$.
\end{cor}
\begin{proof}
By Lemma \ref{ConvexHasDini} we can pick $D^-\varphi (x) \leq A \leq D^+\varphi (x)$.
Also by that result we know that for all $h > 0$, in fact we
have
\begin{align*}
\frac{\varphi(x) - \varphi(x-h)}{h} &\leq A \leq \frac{\varphi(x+h) - \varphi(x)}{h}
\end{align*}
which gives the result upon clearing denominators and defining $B =
\varphi(x)$.
Once again, the strictly convex case follows easily.
\end{proof}
TODO: Extend this to $\reals^n$ (presumably this can be done by taking
partial Dini Derivatives.

\begin{thm}[Jensen's Inequality]\label{Jensen}Let $\xi$ be a random vector in $\reals^n$
  and $\varphi : \reals^n \to \reals$ be a convex function such that
  $\xi$ and $\varphi(\xi)$ are integrable.  Then 
\begin{align*}
\varphi(\expectation{\xi}) \leq \expectation{\varphi(\xi)}
\end{align*}
If $\varphi$ is strictly convex then we have $\varphi(\expectation{\xi}) = \expectation{\varphi(\xi)}$ if and only if $\xi =
\expectation{\xi}$ a.s.
\end{thm}
\begin{proof}We use the fact that for every $x \in \reals^n$ we have a
  subdifferential $\langle a, y \rangle + b$ that satisfies
\begin{align*}
\langle a, y \rangle + b &\leq \varphi(y) \\
\langle a,x \rangle + b &= \varphi(x)
\end{align*}
In particular, choose such an $a,b \in \reals^n$ for the choice $x =
\expectation{\xi}$.  Then by monotonicity and linearity of integral
\begin{align*}
\expectation{\varphi(\xi)} &\geq \expectation{\langle a, \xi\rangle +
  b} \\
&= \langle a, \expectation{\xi} \rangle + b = \varphi(\xi)
\end{align*}
which gives the result.

If $\varphi$ is strictly convex then when $\xi \neq \expectation{\xi}$,
we have 
\begin{align*}
0 &< \varphi(\xi) - \varphi(\expectation{\xi}) - \langle a , \xi -
\expectation{\xi} \rangle
\end{align*}  Thus if $\varphi(\expectation{\xi}) =
\expectation{\varphi(\xi)}$ using linearity of expectation
\begin{align*}
\expectation{ (\varphi(\xi) - \varphi(\expectation{\xi}) - \langle a , \xi -
\expectation{\xi}\rangle); \xi \neq \expectation{\xi}} &= \expectation{\varphi(\xi) - \varphi(\expectation{\xi}) - \langle a , \xi -
\expectation{\xi}\rangle} \\
&=0
\end{align*}
from which we conclude $\characteristic{\xi \neq \expectation{\xi}} = 0$ a.s.
\end{proof}

We will use the following fact later on in the text.  There is no harm
is skipping over it now.
\begin{lem}\label{AbsoluteContinuityOfConvexFunctions}Let $\varphi : [a,b] \to \reals$ be convex with $f(a) = f(a+)$ and $f(b) = f(b-)$, then $\varphi$ is absolutely continuous and $\varphi^\prime$ is increasing.  In particular $D^- \varphi$ and $D^+ \varphi$ are integrable on $[a,b]$ and
for every $x \in [a,b]$, 
\begin{align*}
\varphi(x) &= \varphi(a) + \int_a^x D^+\varphi(s) \, ds = \varphi(a) + \int_a^x D^-\varphi(s) \, ds 
\end{align*}
Conversely if $\varphi$ is absolutely continuous and $\varphi$ is increasing almost everywhere then $\varphi$ is convex.
\end{lem}
\begin{proof}
Suppose that $\varphi$ is convex.  Since the result is trivial for $x=a$ we may assume that $a < x \leq b$.  Pick $\alpha$ and $\beta$ with $a < \alpha < \beta < x$.  Both $D^+\varphi$ and $D^- \varphi$ are bounded on $[\alpha, \beta]$ (e.g. a lower bound of $\frac{\varphi(\alpha) - \varphi(a)}{\alpha - a}$ and an upper bound of $\frac{\varphi(b) - \varphi(\beta)}{b - \beta}$).  Let 
$n \in \naturals$ be given and consider the partition $x_i = \alpha + i \frac{\beta - \alpha}{n}$ for $i=0, \dotsc, n$.  For any $i=2, \dotsc, n-1$ we have
\begin{align*}
\int_{x_{i-2}}^{x_{i-1}} D^+\varphi(s) \, ds  &\leq D^+\varphi (x_{i-1}) \frac{\beta - \alpha}{n} \leq \varphi(x_i) - \varphi(x_{i-1}) \leq D^-\varphi (x_i) \leq \int_{x_{i}}^{x_{i+1}} D^-\varphi(s) \, ds
\end{align*}
Summing up for $i=2, \dotsc, n-1$ we get
\begin{align*}
\int_\alpha^{x_{n-2}} D^+\varphi(s) \, ds &\leq \varphi(x_{n-1}) - \varphi(x_{1}) \leq \int_{x_2}^{\beta} D^-\varphi(s) \, ds
\end{align*}
Now let $n \to \infty$ using Dominated Convergence on the integrals and continuity of $\varphi$ (TODO: Where do we show this?)  to conclude that 
\begin{align*}
\int_\alpha^{\beta} D^+\varphi(s) \, ds &\leq \varphi(\beta) - \varphi(\alpha) \leq \int_{\alpha}^{\beta} D^-\varphi(s) \, ds
\end{align*}
On the other hand from Lemma \ref{ConvexHasDini} we know that $D^- \varphi(s) \leq D^+ \varphi(s)$ for $a <  s < \beta$ and therefore we conclude
\begin{align*}
\int_\alpha^{\beta} D^+\varphi(s) \, ds &= \int_{\alpha}^{\beta} D^-\varphi(s) \, ds = \varphi(\beta) - \varphi(\alpha) 
\end{align*}

We know that $D^\pm \varphi(s)$ are bounded above by $D^+ \varphi (\beta)$ (hence also bounded above by $D^+ \varphi (b)$ right; does this help with anything?) therefore we may apply Monotone Convergence
to $\characteristic{[\alpha,\beta]}(s) (D^+ \varphi (\beta) - D^\pm \varphi(s))$ and use the assumption $f(a) = f(a+)$ to conclude
\begin{align*}
\int_a^\beta D^\pm \varphi(s) \, ds = -(\beta - a) D^+ \varphi (\beta) -\lim_{\alpha \to a^+} \int_\alpha^\beta(D^+ \varphi (\beta) - D^\pm \varphi(s)) \, ds \\
&=-(\beta - a) D^+ \varphi (\beta) - \lim_{\alpha \to a^+} ((\beta-\alpha) D^+ \varphi (\beta) - (\varphi(\beta) - \varphi(\alpha))) = \varphi(\beta) - \varphi(a) \\
\end{align*}
In particular, 
\begin{align*}
\int_a^\beta (-D^\pm \varphi(s) \maxop 0) \, ds &= -\int_a^\beta D^\pm \varphi(s) \, ds + (\beta -a) D^+ \varphi (\beta) = \varphi(a) - \varphi(\beta) + (\beta -a) D^+ \varphi (\beta) < \infty
\end{align*}
which shows that the negative part of $D^\pm \varphi$ is integrable on every interval $[a, \beta]$ with $\beta < b$.

TODO: Finish and understand the subtly of how integrability is shown...
\end{proof}