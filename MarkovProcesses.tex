\section{Markov Processes}

TODO: Thinking about Markov processes as dynamical/deterministic systems with noise.

\subsection{Markov Processes}
\begin{defn}Let $X$ be a process in $(S, \mathcal{S})$ with time scale
  $T$ which is adapted to a filtration $\mathcal{F}_t$.  We say that
  $X$ has the \emph{Markov property} if 
$\cindependent{\mathcal{F}_s}{X_t}{X_s}$ for all $s \leq t \in T$.
\end{defn}
Given any process that satisfies the Markov property it is not hard to
show using properties of conditional independence that it
automatically satisfies a seemingly stronger condition
\begin{lem}[Extended Markov Property]\label{ExtendedMarkovProperty}
Let $X$ be a process that satsifies the Markov property
  then $\cindependent{\mathcal{F}_t}{\sigma(\bigvee_{u \geq t}
    X_u)}{X_t}$ for all $t \in T$.
\end{lem}
\begin{proof}
Let $t_0 \leq t_1 \leq \cdots $ with $t_i \in T$.  By the
Markov property we know for each $0 \leq n$ that
$\cindependent{\mathcal{F}_{t_n}}{X_{t_{n+1}}}{X_{t_n}}$.  Because $X$ is
adapted to $\mathcal{F}$, we know that $X_{t_m}$ is
$\mathcal{F}_{t_n}$-measurable for $m \leq n$ and therefore
$\cindependent{\sigma(X_{t_0}, \dots
  ,X_{t_{n-1}},\mathcal{F}_{t_n})}{X_{t_{n+1}}}{X_{t_n}}$.
By Lemma \ref{ConditionalIndependenceChainRule} we conclude that $\cindependent{\mathcal{F}_{t_n}}{X_{t_{n+1}}}{X_{t_0}, \dots
  ,X_{t_n}}$ for all $n \geq 0$; because $\mathcal{F}_{t_0} \subset
\mathcal{F}_{t_n}$ we get $\cindependent{\mathcal{F}_{t_0}}{X_{t_{n+1}}}{X_{t_0}, \dots
  ,X_{t_n}}$ for all $n \geq 0$.  Another application of Lemma
\ref{ConditionalIndependenceChainRule} shows that
$\cindependent{\mathcal{F}_{t_0}}{\sigma(X_{t_1}, X_{t_2},
  \dots)}{X_{t_0}}$.

Since the union of the $\sigma$-algebras $\sigma(X_{t_1}, X_{t_2},
  \dots )$ for all $t_0 \leq t_1 \leq \cdots $ is
  clearly a $\pi$-system that generates $\sigma(\bigvee_{u \geq t_0}X_u)$, the result follows by montone classes (specifically Lemma \ref{ConditionalIndependencePiSystem}).
\end{proof}

TODO: Introduce the example of Markov Chains here as it is quite a bit
simpler and helps the
understanding of the abstract case quite a bit.

We know make a regularity assumption that for each pair $s,t \in T$
with $s \leq t$, we
have a probability kernel $\mu_{s,t} : S \times \mathcal{S} \to
\reals$ such that for every $A \in \mathcal{S}$
\begin{align*}
\mu_{s,t}(X_s, A) &= \cprobability{X_s}{X_t \in A} =
\cprobability{\mathcal{F}_s}{X_t \in A} \text{ a.s.}
\end{align*}
(e.g. if $S$ is a Borel space then this is true by Theorem
\ref{ExistenceConditionalDistribution}).
We let $\nu_t$ denote the distribution of $X_t$.
These conditional distributions characterize the distribution of the
process $X$ itself.  In particular we have the following nice formula
for finite dimensional distributions of the process.
\begin{lem}\label{MarkovDistributions}Let $X$ be a stochastic process
  on a time scale $T \subset \reals_+$ that has the Markov property,
  one dimensional distributions $\nu_t$ and transition kernels
  $\mu_{s,t}$.  Then for all $t_0 \leq \cdots \leq t_n$ and $A \in
  \mathcal{S}^{\otimes n}$ we have
\begin{align*}
\probability{(X_{t_1}, \dots, X_{t_n}) \in A} 
&= \nu_{t_1} \otimes
\mu_{t_1, t_2} \otimes \cdots \otimes \mu_{t_{n-1},t_n}(A) \\
\cprobability{\mathcal{F}_{t_0}}{(X_{t_1}, \dots, X_{t_n}) \in
  A}(\omega) 
&= \mu_{t_0, t_1} \otimes \cdots \otimes \mu_{t_{n-1},t_n}(X_{t_0}(\omega),A) \\
\end{align*}
\end{lem}
\begin{proof}
We begin by proving the first equality via induction.  The case $n=0$
is true by definition.  The induction step is 
really just a specific case of Theorem 
\ref{Disintegration} applied to the Markov transition kernels.  Let $A
\in \otimes_{i=0}^n \mathcal{S}$ then 
\begin{align*}
&\probability{(X_{t_0}, \cdots, X_{t_n}) \in A} \\
&=\expectation{\characteristic{A}(X_{t_0}, \cdots, X_{t_n})} \\
&=\expectation{\int \characteristic{A}(X_{t_0}, \cdots, X_{t_{n-1}},s)
  \, \mu_{t_{n-1}, t_n} (X_{n-1}, ds)} \\
&=\int \left [ \int \characteristic{A}(u_0, \cdots,u_{n-1}, s)   \,
  \mu_{t_{n-1}, t_n} (X_{n-1}, ds) \right ] \nu_{t_0} \otimes \cdots \otimes
  \mu_{t_{n-2}, t_{n-1}}(du_0, \dotsc, du_{n-1})\\
&=\nu_{t_0} \otimes \cdots \otimes  \mu_{t_{n-1}, t_{n}}(A)
\end{align*}

The second equality is derived from the first.  Suppose we have $A \in
\mathcal{S}$ and $B \in \mathcal{S}^{\otimes n}$.  Then we can compute
\begin{align*}
&\expectation{\characteristic{A}(X_{t_0}) \characteristic{B}(X_{t_1},
  \dotsc, X_{t_n})}  \\
&= \nu_{t_0} \otimes \mu_{t_0, t_1} \otimes \cdots \otimes  \mu_{t_{n-1},
  t_{n}}(A \times B) \\
&= \int \left [ \int \characteristic{B}(u_1, \dotsc, u_n) \mu_{t_0,
    t_1} \otimes \cdots \otimes \mu_{t_{n-1}, t_n} (u_0, du_1, \dotsc, du_n) \right ]
\characteristic{A}(u_0) \nu_{t_0}(u_0) \\
&=\expectation{\mu_{t_0, t_1} \otimes \cdots \otimes \mu_{t_{n-1},
    t_n}(X_0, B) \characteristic{A}(X_0)}
\end{align*}
Now the $\sigma(X_{t_0})$-measurability of $\mu_{t_0, t_1} \otimes \cdots \otimes \mu_{t_{n-1},
    t_n}(X_0, B)$ tells us that 
\begin{align*}
\cprobability{X_{t_0}}{(X_{t_1},\dotsc, X_{t_n}) \in B} 
&= \mu_{t_0, t_1} \otimes \cdots \otimes \mu_{t_{n-1},    t_n}(X_0, B)
\end{align*}

The last thing is to show that
$\cprobability{X_{t_0}}{(X_{t_1},\dotsc, X_{t_n}) \in B} =
\cprobability{\mathcal{F}_{t_0}}{(X_{t_1},\dotsc, X_{t_n}) \in B}$
a.s.  This follows from Lemma \ref{ExtendedMarkovProperty} since by
the tower property of conditional expectations and that result 
for any $A \in \mathcal{S}^{\otimes n}$ and $B \in \mathcal{F}_{t_0}$
\begin{align*}
\probability{(X_{t_1}, \dotsc, X_{t_n}) \in A ; B} 
&=\expectation{\cprobability{X_{t_0}}{(X_{t_1}, \dotsc, X_{t_n}) \in A ;  B}} \\
&=\expectation{\cprobability{X_{t_0}}{(X_{t_1}, \dotsc, X_{t_n}) \in A}
\cprobability{X_{t_0}}{B}} \\
&=\expectation{\cprobability{X_{t_0}}{(X_{t_1}, \dotsc, X_{t_n}) \in A}
\characteristic{B}} \\
\end{align*}
so the $\mathcal{F}_{t_0}$-measurability of
$\cprobability{X_{t_0}}{(X_{t_1}, \dotsc, X_{t_n}) \in A}$ gives the
result by the defining property of conditional expectations.
\end{proof}
A special case of the relations above should be called out as it
motivates a property that will assume as part of the definition of a
Markov process.  But first we need a definition.
\begin{defn}Let $\mu$ and $\nu$ be probability kernels from $S$ to
  $S$.  Then we define the probability kernel $\mu \nu$ from $S$ to
  $S$ by
\begin{align*}
\mu \nu (s, A) &= (\mu \otimes \nu)(s, A \times S)
\end{align*}
for all $s \in S$ and $A \in \mathcal{S}$.
\end{defn}
\begin{examp}
Let $S$ be a finite set and view $\mu$ and $\nu$ as $S \times S$
matrices.  Then $\mu \nu$ is just matrix multiplication:
\begin{align*}
\mu \nu (s, \lbrace t \rbrace) 
&= \iint \characteristic{\lbrace t  \rbrace \times S}(u,v) \nu(u, dv) \mu(s, du) 
=\int \nu(u, \lbrace t \rbrace) \mu(s,du) 
=\sum_{u \in S} \nu_{u, t} \mu_{s, u}
\end{align*}
\end{examp}
\begin{cor}[Chapman-Kolmogorov
  Relations]\label{ChapmanKolmogorovWeak}Let $X$ be a stochastic
  process on a time scale $T \subset \reals$ with values in Borel
  space $(S, \mathcal{S})$ and suppose that $X$ has the Markov
  property.  Then for every $s, t, u \in T$ with $s \leq t \leq u$ we
  have
\begin{align*}
\mu_{s,t} \mu_{t,u} &= \mu_{s,u} \text{ a.s. $\nu_s$}
\end{align*}
\end{cor}
\begin{proof}
Since we have assume $S$ is a Borel space we know from Theorem \ref{ExistenceConditionalDistribution} that regular
versions $\mu_{s,t}$ exist.  By definition of $\mu_{s,t} \mu_{t,u}$, Lemma
\ref{MarkovDistributions} and the uniqueness clause of Theorem \ref{ExistenceConditionalDistribution}
\begin{align*}
\mu_{s,t} \mu_{t,u}(x, A) &= 
(\mu_{s,t} \otimes \mu_{t,u})(x, S \times A) \\
&= \cprobability{\mathcal{F}_s}{(X_t, X_u) \in S \times A} \\
&=\cprobability{\mathcal{F}_s}{ X_u \in A} \\
&=\cprobability{X_s}{ X_u \in A} \\
&= \mu_{s,u}(x, A) \text{  a.s. $\nu_s$}
\end{align*}
\end{proof}

The ability to derive the almost sure version of the
Chapman-Kolmogorov relations is really just motivational for our
purposes.  In fact we will want to assume they hold identically in
what follows.  Absent a workable set of conditions from which we can
derive this fact, we build into our definitions.  Collecting all of
the conditional independence and regularity properties we've
identified we finally make the formal definition of a Markov process.
\begin{defn}
A \emph{Markov Process} is a stochastic process $X_t$ on a time scale $T
\subset \reals_+$ and a state space $(S, \mathcal{S})$ such that 
\begin{itemize}
\item[(i)]$\cindependent{\mathcal{F}_s}{X_t}{X_s}$ for all $s \leq t$
\item[(ii)]there exists a regular version $\mu_{s,t} :
  S \times \mathcal{S} \to [0,1]$ of $\cprobability{\mathcal{F}_s}{X_t
    \in \cdot}$ for each $s \leq t$.
\item[(iii)]$\mu_{s,t} \mu_{t,u} = \mu_{s,u}$ everywhere on $S$ for
  each $s \leq t \leq u$.
\end{itemize}
\end{defn}

In lieu of general techinique for proving that a process is Markov
from general principle, we give a result that shows that we can
construct them from a set of transition kernels that obey the Chapman-Kolmogorov relations.
\begin{thm}\label{ExistenceMarkovProcess}Suppose we are given
\begin{itemize}
\item[(i)] a  time scale starting at $0$, $T \subset \reals_+$ 
\item[(ii)]a Borel space $(S, \mathcal{S})$ 
\item[(iii)]a probability distribution $\nu$ on $(S, \mathcal{S})$
\item[(iv)]probability kernels $\mu_{s,t} : S \times
  \mathcal{S} \to [0,1]$ for each $s \leq t \in T$ such that 
\begin{align*}
\mu_{s,t} \mu_{t,u} &= \mu_{s,u} \text{ for all $s\leq
  t\leq u \in T$}
\end{align*}
then there exists a Markov process $X_t$ with initial distribution
$\nu$ and transition kernels $\mu_{s,t}$.
\end{itemize} 
\end{thm}
\begin{proof}
TODO: Define FDDs, show consistency and use Kolmogorov extension theorem
\end{proof}

\begin{defn}Suppose that a family of transition kernels $\mu_{s,t}$ is
  given.  For a distribution $\nu$ on $(S, \mathcal{S})$, let
  $\sprobabilityop{\nu}$ denote the distribution on $S^T$ of the
  Markov process with initial distribution $\nu$.  If $\nu=\delta_x$
  for some $x \in S$ then it is customary to write
  $\sprobabilityop{x}$ instead of $\sprobabilityop{\delta_x}$.
\end{defn}
\begin{lem}\label{MarkovMixtures}The family $\sprobabilityop{x}$ is a kernel from $S$ to
  $S$.  Futhermore, given an initial distribution $\nu$
\begin{align*}
\sprobability{A}{\nu} = \int \sprobability{A}{x} \, d\nu(x)
\end{align*}
\end{lem}
\begin{proof}
First assume that $A = (\pi_{t_1}, \dots, \pi_{t_n})^{-1}(B)$ for some
$B \in \mathcal{S}^{\otimes n}$.  We can use Lemma \ref{MarkovDistributions} to compute for
any $\nu$,
\begin{align*}
\sprobability{A}{\nu} &= \sprobability{(\pi_0, \pi_{t_1}, \dots,
  \pi_{t_n})^{-1}(S \times B)}{x} \\
&=\nu \otimes \mu_{0, t_1} \otimes \cdots \mu_{t_{n-1}, t_n}(S \times
B) \\
&= \int \mu_{0, t_1} \otimes \cdots \mu_{t_{n-1}, t_n}(x,B) \, d\nu(x)
\end{align*}
In particular, for $\nu = \delta_x$ we get
\begin{align*}
\sprobability{A}{x} &=\mu_{0, t_1} \otimes \cdots \mu_{t_{n-1},
  t_n}(x,B)
\end{align*}
which shows both that $\sprobability{A}{x}$ is measurable and that
$\sprobability{A}{\nu} = \int \sprobability{A}{x} \, d\nu(x)$.

To extend to general measurable sets, we note that the set of $A$ of
the form given above is a $\pi$-system therefore we can apply Lemma
\ref{KernelMeasurability} to conclude $\sprobabilityop{x}$ is a
kernel.  Similarly we may conclude that $\sprobability{A}{\nu} = \int
\sprobability{A}{x} \, d\nu(x)$ for arbitrary measurable $A$ by the fact that probability measures
are uniquely determined by their values on a generating $\pi$-system
(Lemma \ref{UniquenessOfMeasure}).
\end{proof}

\subsection{Homogeneous Markov Processes}

We have described a relatively general version of Markov processes
compared to what it needed in many applications and the goal of this
section is to define the assumptions that lead to useful
simplifications and to understand how to look at these simplifying
assumptions from a couple of points of view.

\begin{defn}Suppose $(S, \mathcal{S})$ is a measurable Abelian group
  and $\mu : S \times \mathcal{S} \to [0,1]$ is a kernel.  We say
  $\mu$ is \emph{homogeneous} if for every $s \in S$ and
  $A\in \mathcal{S}$ we have $\mu(0, A) = \mu(s, A+s)$.
\end{defn}

A useful observation for computing conditional expectations is that
integrals are invariant under certain changes of variables.
\begin{lem}\label{HomogeneousKernelExpectationRule}Let $(S, \mathcal{S})$ be a measurable Abelian group with a
  homogeneous kernel $\mu : S \times \mathcal{S} \to [0,1]$, then for
  each $y,z \in S$ and integrable $f : S \to \reals$,
\begin{align*}
\int f(x + y) \, \mu(z, dx) &= \int f(x) \, \mu(y+z, dx)
\end{align*}
\end{lem}
\begin{proof}
For $y \in S$, let $t_y : S \to S$ be translation by $y$: $t_y(x) = x
+ y$.  Thinking of the kernel as a measurable measure
valued map (which we denote $\mu(z)$) we compute the pushforward of $\mu(z)$ under $t_y$
using homogeneity
\begin{align*}
\pushforward {t_y}{\mu(z)}(A) &= \mu(z, t_y^{-1}(A)) = \mu(z, A
- y) = \mu(z+y, A)
\end{align*}
thus showing $\pushforward {t_y}{\mu(z)} = \mu(y+z)$.
Now we can apply the Expectation Rule (Lemma \ref{ExpectationRule}) to
see that 
\begin{align*}
\int f(x + y) \, \mu(z, dx) &= \int f(x) \, d\left [ \pushforward {t_y}{\mu(z)} \right] = \int f(x) \, \mu(y+z, dx)
\end{align*}
\end{proof}

A Markov process with homogeneous kernels is said to be
\emph{space-homogeneous}; intuitively the probability of starting out
in a set $A$ at time $s$ and winding up in set $B$ at time $t$ only
depends on the relative positions of $A$ and $B$ (under translations).
\begin{defn}Suppose $(S, \mathcal{S})$ is a measurable Abelian group
  and let $X_t$ be a Markov process with transition kernels
  $\mu_{s,t}$.  Then $X_t$ is \emph{space-homogeneous} if and only if
  $\mu_{s,t}$ is homogeneous for every $s \leq t$.
\end{defn}

\begin{lem}\label{SpaceHomogeneousMarkovDistributions}Let $\mu_{s,t}$
  be a family of space homogeneous transition kernels on a
  measurable Abelian group, then for every $A \in \mathcal{S}^T$ and $x \in S$,
  $\sprobability{A}{x} = \sprobability{A-x}{0}$.
\end{lem}
\begin{proof}
TODO: This proof only seems to require space homogeneity of the
kernels $\mu_{0,t}$; is this a mistake (or does Chapman Kolmogorov
imply the rest of the kernels are space homogeneous as well...)

We begin by establishing the result for sets of the form $\lbrace (X_{t_1},
\dotsc, X_{t_n}) \in A \rbrace$ for $A \in \mathcal{S}^{\otimes n}$ and $t_1 \leq
\cdots \leq t_n$.  The key point is that we know from the proof of
Lemma \ref{MarkovMixtures} that $\sprobability{(X_{t_1},
\dotsc, X_{t_n}) \in A }{x} = \mu_{0, t_1} \otimes \dotsc \otimes
\mu_{t_{n-1}, t_n}(x, A)$, so in particular the case $n=1$ follows
directly from the assumption that each $\mu_{0, t}$ is homogeneous.
To see the result for $n > 1$ we calculate using Lemma \ref{HomogeneousKernelExpectationRule}
\begin{align*}
&\sprobability{(X_{t_1},\dotsc, X_{t_n}) \in A }{x} \\
&= \mu_{0, t_1} \otimes \dotsc \otimes \mu_{t_{n-1}, t_n}(x, A) \\
&= \int \int \characteristic{A}(x_1, x_2, \dotsc, x_n) \mu_{t_1, t_2} \otimes \cdots \otimes \mu_{t_{n-1},
  t_n}(x_1, dx_2, \dotsc, dx_n) \, \mu_{0,t_1}(x, dy) \\
&= \int \int \characteristic{A}(x_1+x, x_2, \dotsc, x_n) \mu_{t_1, t_2} \otimes \cdots \otimes \mu_{t_{n-1},
  t_n}(x_1, dx_2, \dotsc, dx_n) \, \mu_{0,t_1}(0, dy) \\
&= \int \int \characteristic{A-x}(x_1, x_2, \dotsc, x_n) \mu_{t_1, t_2} \otimes \cdots \otimes \mu_{t_{n-1},
  t_n}(x_1, dx_2, \dotsc, dx_n) \, \mu_{0,t_1}(0, dy) \\
&= \mu_{0, t_1} \otimes \dotsc \otimes \mu_{t_{n-1}, t_n}(0, A-x) \\
&=\sprobability{(X_{t_1},\dotsc, X_{t_n}) \in A - x}{0} \\
\end{align*}

Now we complete the result by a monotone class argument.  We know that
sets of the form $\lbrace (X_{t_1}, \dotsc, X_{t_n}) \in A \rbrace$
are a generating $\pi$-system so by the $\pi$-$\lambda$ Theorem
(Theorem \ref{MonotoneClassTheorem}) it suffices to show that $\mathcal{C}
= \lbrace A \mid \sprobability{A}{x} = \sprobability{A-x}{0} \rbrace$ is a
$\lambda$-system. If $A,B \in \mathcal{C}$ with $A \subset B$ then 
\begin{align*}
\sprobability{B\setminus A}{x} &= \sprobability{B}{x} -
\sprobability{A}{x} = \sprobability{B-x}{0} - \sprobability{A-x}{0} =
\sprobability{B\setminus A-x}{0} 
\end{align*}
where we have used the elementary fact that $B\setminus A - x =
(B-x)\setminus(A-x)$ (let $y \in B$ and $y \notin A$ then clearly $y-x
\in B-x$ and $y-x \notin A-x$).  Similarly if $A_n \in \mathcal{C}$
for $n \in \naturals$ with $A_1 \subset A_2 \subset \cdots$ then it is
also true that $A_1 -x  \subset A_2-x \subset \cdots$ and continuity
of measure (Lemma \ref{ContinuityOfMeasure}) shows
\begin{align*}
\sprobability{\cup_n A_n}{x} &=
\lim_{n \to \infty}\sprobability{A_n}{x} = 
\lim_{n \to \infty}\sprobability{A_n-x}{0} = 
\sprobability{\cup_n A_n-x}{0} 
\end{align*}
\end{proof}

There is another way of thinking about the space-homogeneous Markov
processes.  We know that for any $s \leq t$, given the value of $X_s$ the probability
distribution of $X_t$ is independent of the history of $X$ up to
$s$.  Space homogeneity tells us that moreover that the probability
distribution $X_t$ only depends on the \emph{increment} $X_t - X_s$.
Putting these two observations together we should expect that $X_t -
X_s$ is independent (not just conditionally independent) of the
history of $X$ up to $s$.  In fact this provides an equivalent
characterisation of space homogeneous Markov processes as we prove in
the following result.

\begin{defn}Let $(S, \mathcal{S})$ be a measurable Abelian group with a
  time scale $T \subset \reals_+$, a filtration $\mathcal{F}_t$ and an
 $S$-valued $\mathcal{F}$-adapted process $X_t$. We say that $X_t$ has
 $\mathcal{F}$-independent increments if and only if $X_t - X_s$ is
 independent of $\mathcal{F}_s$ for all $s \leq t$.
\end{defn}

\begin{lem}\label{IndependentIncrements}Let $(S, \mathcal{S})$ be a measurable Abelian group with a
  time scale $T \subset \reals_+$, a filtration $\mathcal{F}_t$ and an
 $S$-valued $\mathcal{F}$-adapted process $X_t$.  The $X_t$ has
$\mathcal{F}$-independent increments if and only if $X_t$ is a space-homogeneous
 Markov process.  In this case the transition kernels of $X_t$ are
 given by
\begin{align*}
\mu_{s, t}(x, A) &= \probability{X_t - X_s \in A - x} \text{ for $x
  \in S$, $A \in \mathcal{S}$ and $s \leq t \in T$}
\end{align*}
TODO: The proof actually requires regular versions of
$\cprobability{\mathcal{F}_s}{X_t}$; do we need to assume that
$G$ is Borel or something?  Also we've defined a Markov process as
satisfying the Chapman Kolmogorov relations identically; can that be derived?
\end{lem}
\begin{proof}
Suppose that $X_t$ is a space homogeneous Markov Process with
transition kernels $\mu_{s,t}$.  Then for every $s\leq t$ and $A \in \mathcal{S}$,
\begin{align*}
\cprobability{\mathcal{F}_s}{X_t - X_s \in A} &= 
\int \characteristic{A}(x - X_s) \, \mu_{s,t}(X_s, dx) & & \text{by
  Theorem \ref{Disintegration}} \\
&=\int \characteristic{A}(x) \, \mu_{s,t}(0, dx) & & \text{by Lemma
  \ref{HomogeneousKernelExpectationRule}} \\
&= \mu_{s,t}(0,A)
\end{align*}
which shows that $\cprobability{\mathcal{F}_s}{X_t - X_s \in A}$ is
almost surely constant hence $\cindependent{X_t -
  X_s}{\mathcal{F}_s}{}$.  Moreover by the tower rule we also know
that $\probability{X_t - X_s \in A}=\cprobability{\mathcal{F}_s}{X_t -
  X_s \in A}=\mu_{s,t}(0,A)$ and therefore by another application of
space homogeneity, $\mu_{s, t}(x, A) = \mu_{s,t}(0, A-x) = \probability{X_t - X_s \in A}$.

Suppose that $X_t$ has independent increments.  The key point is that
this property determines the conditional distributions 
\begin{align*}
\mu_{s,t}(x, A) &= \probability{X_t - X_s \in A - x}
\end{align*}
and moreover this form is a regular version.  First note that $\probability{X_t - X_s \in A - x}$ is a probability
kernel since for fixed $A$ it is measurable in $x$ by Lemma
\ref{MeasurableSections} and for fixed $x$ it is just the distribution
of the measurable random element $X_t - X_s -x$.  

Showing that $\probability{X_t - X_s \in A - x}$ is a version of
$\cprobability{\mathcal{F}_s}{X_t \in A}$ is not hard but requires a
bit of care because the random element $X_s$ plays two different roles
in the calculation and it is worth making this fact explicit.
We start by defining 
$\tilde{\mu}_{s,t}(x, A) = \probability{X_t - X_s \in A}$ and observing
that because $\cindependent{X_t - X_s}{\mathcal{F}_s}{}$,
$\tilde{\mu}_{s,t}$ is a kernel for $\cprobability{\mathcal{F}_s}{X_t
  - X_s \in \cdot}$.  With this fact and the $\mathcal{F}$-adaptedness
of $X$, we can apply Theorem
\ref{Disintegration} (using the function $f(x,y) =
\characteristic{A-y}(x)$ evaluated at $(X_t - X_s, X_s)$) to conclude
\begin{align*}
\cprobability{\mathcal{F}_s}{X_t \in A} &= 
\cprobability{\mathcal{F}_s}{X_t - X_s \in A - X_s} \\
&=\int \characteristic{A - X_s}(x) \, \tilde{\mu}_{s,t}(dx) \\
&=\tilde{\mu}_{s,t}(A - X_s) \\
&=\mu_{s,t}(X_s, A)
\end{align*}
Now note that $\mu_{s,t}(X_s, A)$ is $X_s$-measurable hence we have
$\cprobability{\mathcal{F}_s}{X_t \in A} = \cprobability{X_s}{X_t
  \in A} $ for all $A \in \mathcal{S}$ thus the Markov property
holds by Lemma \ref{ConditionalIndependenceDoob}.  
Using the explicit form of the kernel we calculate
\begin{align*}
\mu_{s,t}(x, A) &=  \probability{X_t - X_s \in A-x} = \mu_{s,t}(0, A-x) 
\end{align*}
demonstrating space homogeneity.
\end{proof}

Here is what the proof that space homogeneous Markov implies
independent increments looks like in elementary probability theory
(discrete time countable state space).
\begin{proof}
Space homogeneity means that $\cprobability{X_{n-1} = y}{X_n = x} =
\cprobability{X_{n-1} = 0}{X_n = x-y}$.  This implies that for any
$y\in S$ we have
$\probability{X_n - X_{n-1} = z} = \cprobability{X_{n-1} = y}{X_n =
  z+y}$:
\begin{align*}
\probability{X_n - X_{n-1} = z} &= \sum_x \probability{X_n - X_{n-1} =
  z ; X_{n-1}=x} \\
& =\sum_x \cprobability {X_{n-1}=x}{X_n - X_{n-1} =  z} \probability
{X_{n-1}=x} \\
& =\sum_x \cprobability {X_{n-1}=x}{X_n =  z+x} \probability
{X_{n-1}=x} \\
& =\cprobability {X_{n-1}=y}{X_n =  z+y} \sum_x \probability
{X_{n-1}=x} \\
&= \cprobability {X_{n-1}=y}{X_n =  z+y} 
\end{align*}
Now we use this fact along with the Markov property to see
\begin{align*}
&\probability{X_n - X_{n-1} = z; X_1=x_1 ; \cdots ; X_{n-1}=x_{n-1}} \\
&=\probability{X_n = z+x_{n-1}; X_1=x_1 ; \cdots ; X_{n-1}=x_{n-1}} \\
&=\cprobability{X_1=x_1 ; \cdots ; X_{n-1}=x_{n-1}}{X_n = z+x_{n-1}} 
\probability{X_1=x_1 ; \cdots ; X_{n-1}=x_{n-1}}\\
&=\cprobability{X_{n-1}=x_{n-1}}{X_n = z+x_{n-1}} 
\probability{X_1=x_1 ; \cdots ; X_{n-1}=x_{n-1}}\\
&=\probability{X_n = z} 
\probability{X_1=x_1 ; \cdots ; X_{n-1}=x_{n-1}}\\
\end{align*}
\end{proof}

TODO: Motivate time homogeneity by thinking about discrete time and
the fact that you can generate everything from the unit time
transitions.  Time homogeneity is the property that all of these
transition kernels are the same and therefore the Markov process is
determined by a single kernel (and the initial distribution).

\begin{defn}A time homogenous Markov process ... TODO
\end{defn}

TODO: Show that a time homogenous Markov process is a dynamical system
with noise.

\subsection{Strong Markov Property}

\begin{lem}Let $X$ be a time homogeneous Markov process and let $\tau$ be an optional
  time with at most countably many values.  Then for every measurable
  $A \subset S^T$,
\begin{align*}
\cprobability{\mathcal{F}_\tau}{\theta_\tau X \in A}(\omega) &=
\sprobability{A}{X_\tau(\omega)} \text{ for all $\omega$ such that $\tau(\omega) < \infty$}
\end{align*}
\end{lem}
\begin{proof}
We first prove the result for deterministic times and extend to
countably valued optional times.  Note that the content of result is
vacuous for an infinite deterministic time, so pick a finite
deterministic time $t$ and let $t_1
\leq \cdots \leq t_n$ and $A \in \mathcal{S}^{\otimes n}$ and
calculate using Lemma \ref{MarkovDistributions}, time homogeneity and Lemma \ref{MarkovMixtures}
\begin{align*}
\cprobability{\mathcal{F}_t}{((\theta_t X)_{t_1}, \dotsc,
  (\theta_t X)_{t_n}) \in A} &= \cprobability{\mathcal{F}_t}{(X_{t
    + t_1}, \dotsc, X_{t+ t_n}) \in A} \\
&= \mu_{t, t + t_1} \otimes \cdots \otimes \mu_{t+t_{n-1}, t+t_n}(X_t,
A) \\
&=\mu_{0, t_1} \otimes \cdots \otimes \mu_{t_{n-1}, t_n}(X_t, A) \\
&= \sprobability {A}{X_t}
\end{align*}
Now we know that sets of the form 
$\lbrace ((\theta_t X)_{t_1}, \dotsc,  (\theta_t X)_{t_n}) \in A \rbrace$ are a
generating $\pi$-system for the $\sigma$-algebra $\mathcal{S}^T$ and
the full result for deterministic times $t$ follows from a monotone
class argument.

To see this we simply show that the set of $A$ such that
$\cprobability{\mathcal{F}_t}{\theta_t X \in A} =
\sprobability{A}{X_t}$ a.s. is a $\lambda$-system.  The case for $B
\setminus A$ follows from linearity of conditional expectation and
finite additivity of measure and the case $A_1 \subset A_2 \subset
\cdots$ follows from monotone convergence for conditional expectations
and continuity of measure.

Now we extend to the case of countably valued optional times.  Let $A
\in \mathcal{S}^T$ and $B \in \mathcal{F}_\tau$ and calculate
\begin{align*}
\expectation{\characteristic{A}(\theta_\tau X) ; B} &= \sum_t
\expectation{\characteristic{A}(\theta_t X) ; \lbrace \tau = t \rbrace
  \cap B} \\
&= \sum_t \expectation{\sprobability{A}{X_t} ; \lbrace \tau = t \rbrace
  \cap B} \\
&= \expectation{\sprobability{A}{X_\tau} ; B} 
\end{align*}
so the result follows by the definition of conditional expectation. 

TODO: What about the $\mathcal{F}_\tau$-measurability of
$\sprobability{A}{X_\tau}$?  Note that this is a consequence of result
since we haven't assumed $X$ is progressive(see Lemma
\ref{StrongIndependentIncrements} below where we make this implication explicit).  Double check that we
don't assume it in the proof above.

TODO: Some comments about the well-definedness of the expressions on
$\lbrace \tau < \infty\rbrace$; this boils down to the locality of
conditional expectation.
\end{proof}

In the case of a space homogeneous Markov process the strong Markov
property can be expressed more concisely as an extension of the
independent increments characterization of Lemma
\ref{IndependentIncrements} to optional times.  In many scenarios it
is more convenient to use these properties.  Note that the Lemma does
not require the countable range assumption.
\begin{lem}\label{StrongIndependentIncrements}Let $S$ be a measurable
  Abelian group with a filtration $\mathcal{F}$, $X$ be a time
  homogeneous and space homogeneous $S$-
  valued Markov process and $\tau$ be an almost surely finite optional time.  
Then 
\begin{align*}
\cprobability{\mathcal{F}_\tau}{\theta_\tau X \in A} &= \sprobability{A}{X_\tau}
\end{align*}
if and only if $X_\tau$ is $\mathcal{F}_\tau$-measurable, $\cindependent{\theta_\tau X -
  X_\tau}{\mathcal{F}_\tau}{}$ and $X - X_0 \eqdist \theta_\tau X - X_\tau$
\end{lem}
\begin{proof}
Assume that $X$ satisfies $\cprobability{\mathcal{F}_\tau}{\theta_\tau
  X \in A} = \sprobability{A}{X_\tau}$ for all $A \in \mathcal{S}^T$.  To see that $X_\tau$ is
$\mathcal{F}_\tau$-measurable observe that if we let $\pi_0 : S^T \to
S$ be
evaluation at time $0$, then for any $B \in \mathcal{S}$ and $x \in S$,
\begin{align*}
\sprobability{\pi_0^{-1}B}{x} &= \begin{cases}
1 & \text{if $x \in B$} \\
0 & \text{if $x \notin B$}
\end{cases}
\end{align*}
therefore we have
\begin{align*}
\characteristic{X_\tau \in B} &=
\sprobability{\pi_0^{-1}B}{X_\tau} =
\cprobability{\mathcal{F}_\tau}{\theta_\tau X \in \pi_0^{-1} B}
\end{align*}
which shows that $\lbrace X_\tau \in B \rbrace \in \mathcal{F}_\tau$.

Having established $\mathcal{F}_\tau$-measurability of $X_\tau$ we
know that $P_{X_\tau}$ is a not just a \emph{regular} version for
$\cprobability{\mathcal{F}_\tau}{\theta_\tau X \in \cdot}$ and we can
apply Theorem \ref{Disintegration} and space homogeneity of
$\probabilityop_x$ (Lemma \ref{SpaceHomogeneousMarkovDistributions})
to calculate for $A \in \mathcal{S}^T$ (using $f: S^T \times S \to
\reals_+$ given by $f(x,y) = \characteristic{A+y}(x)$ in the disintegration) 
\begin{align*}
\cprobability{\mathcal{F}_\tau}{\theta_\tau X - X_\tau \in A} &= 
\int \characteristic{A + X_\tau} (x) \,
\probabilityop_{X_\tau}(dx) 
=\sprobability{A + X_\tau}{X_\tau}
=\sprobability{A}{0} \text{ a.s.}
\end{align*}
which is almost surely constant and therefore independence is proven.
This also shows that the distribution of $\theta_\tau X - X_\tau$ is
equal to $\probabilityop_0$ and letting $\tau = 0$ shows $\theta_\tau
X - X_\tau \eqdist X - X_0$.

To prove the converse, note that $X - X_0$ is has
initial distribution $\delta_0$ hence using our independence and
equidistribution assumptions and the definition of the measure
$\probabilityop_0$ we get for any $A \in \mathcal{S}^T$,
\begin{align*}
\cprobability{\mathcal{F}_\tau}{\theta_\tau X - X_\tau \in A} &=
\probability{\theta_\tau X - X_\tau \in A} 
=\probability{X - X_0 \in A} 
=\sprobability{A}{0}
\end{align*}
which provides us with a regular version for
$\cprobability{\mathcal{F}_\tau}{\theta_\tau X - X_\tau \in \cdot}$.
Now by the $\mathcal{F}_\tau$-measurability of $X_\tau$ we can apply
Theorem \ref{Disintegration} and Lemma
\ref{SpaceHomogeneousMarkovDistributions} to get
\begin{align*}
\cprobability{\mathcal{F}_\tau}{\theta_\tau X \in A}
&=\cprobability{\mathcal{F}_\tau}{\theta_\tau X - X_\tau \in A-X_\tau}\\
&= \int \characteristic{A-X_\tau}(x) \, \probabilityop_0(dx) \\
&=\sprobability{0}{A - X_\tau} \\
&=\sprobability{X_\tau}{A }
\end{align*}
and we are done.
\end{proof}