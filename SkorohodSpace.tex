\chapter{Skorohod Space}
TODO: Currently going through this.  

TODO:  The development below is using the absolute value notation even though
paths take values in an arbitrary metric space.  Clean this up (which is actually pretty barfy because
there are a bunch of different metrics floating around).

TODO: Show that for $S$ complete, the sup norm makes $D([0,T];S)$ into a complete non-separable metric space.
We actually use the completeness in showing that the metric $d$ is complete.

Question 1:  In the definition of the $J_1$ topology on $D([0,\infty);
S)$ given a time shift $\lambda(t)$ we define $d(f,g,\lambda, u) = \sup_{t \geq 0} q(f(t \wedge u),
g(\lambda(t) \wedge u))$ and take the distance given the time shift as
$\int_0^\infty e^{-u} d(f,g,\lambda,u) \, du$.  Why is $d$ defined
this way and not as  $d(f,g,\lambda, u) = \sup_{0 \leq t \leq u} q(f(t),
g(\lambda(t)))$?  Would the latter fail to define a metric or would it
fail to be complete?

Question 2: Given a cadlag function $f : [0,1] \to S$, we know that
$f$ has only countably many jump discontinuities; is there some notion
of uniform continuity that can be preserved?  E.g. can we say that
given $\epsilon > 0$ for
all points of continuity $x$ of $f$  there exists a uniform $\delta > 0$
such that $\abs{x-y} < \delta$ implies $q(f(x), f(y)) < \epsilon$?

TODO: It would be convenient to treat the case of $[0,\infty]$ below

\begin{defn}Let $S$ be a topological space, then for every $0 < T < \infty$ we let $D([0,T]; S)$ denote the set of functions
$f : [0,T] \to S$ such that for every $0 \leq t < T$ we have $f(t) = \lim_{s \to t^+} f(s)$ and for every $0 < t \leq T$ the limit 
$\lim_{s \to t^{-}} f(s)$ exists and is finite.  The space $D([0,\infty); S)$ is the set of functions $f : [0,\infty)$ such that for all 
$t \geq 0$ we have $f(t) = \lim_{s \to t^+} f(s)$ and for all $t >0$ we have $\lim_{s \to t^{-}} f(s)$ exists and is finite.
\end{defn}

In what follows we will often use the notation $f(t-)$ to denote the limit $\lim_{s \to t^{-}} f(s)$.

\begin{lem}\label{CadlagCountableDiscontinuitySet}If $x \in D([0,T];
  S)$ or $x \in D([0,\infty); S)$ then $x$ is continuous at all but a
  countable number of points.
\end{lem}
\begin{proof}
We begin by considering the case of $x \in D([0,T]; S)$.  Pick an
$\epsilon > 0$ and define
\begin{align*}
A_\epsilon &= \lbrace 0 \leq t \leq T \mid r(x(t-), x(t)) \geq
\epsilon \rbrace
\end{align*}

\begin{clm} $A_\epsilon$ is finite.
\end{clm}

Suppose otherwise, then by compactness of $[0,T]$ there is an
accumulation point $t$ of $A_\epsilon$.  By passing to a further
subsequence we can assume that we have a sequence $t_n$ such that $t_n
\in A_\epsilon$ and
either $t_n \downarrow t$ or $t_n \uparrow t$.  First consider the
case $t_n \downarrow t$.  For every $n$ by the existence of the left
limit $x(t_n-)$ we can find $t_n^\prime$ such that $t_{n+1} >
t_n^\prime > t_n$ and $r(x(t_n), x(t_n^\prime)) > \epsilon/2$.  Now by
construction we
know that $t_n^\prime \downarrow t$ and by right continuity we get
$\lim_{n \to \infty} x(t_n) = \lim_{n \to \infty} x(t_n) = x(t)$.  
However this is a contradiction since we can find $N > 0$ such that
$r(x(t), x(t_N)) < \epsilon/4$ and $r(x(t), x(t_N^\prime) <
\epsilon/4$ which yields $r(x(t_N), x(t_N^\prime)) < \epsilon/2$.  If
$t_n \uparrow t$ we argue similarly construction a sequence
$t_n^\prime$ such that $t_{n-1} < t_n^\prime < t_n$ and $r(x(t_n),
x(t_n^\prime)) > \epsilon/2$.  By existence of left limits, we know that $\lim_{n \to \infty}
x(t_n^\prime) = \lim_{n \to \infty} x(t_n) = x(t-)$ and this gives a
contradiction as before.

Now simply note that the set of
discontinuities of $x$ is $\cup_{n=1}^\infty A_{1/n}$ and is therefore
countable.  In a similar way we see that the set of discontinuities
for $x \in D([0,\infty); S)$ is countable since it is equal to the
union of the discontinuities of $x$ restricted to $[0,n]$ for $n \in \naturals$.
\end{proof}

\begin{defn}Let $(S,r)$ be a metric space, define $\Lambda$ denote the set of all $\lambda : [0,T] \to
  [0,T]$ such that $\lambda$ is continuous, strictly increasing and
  bijective.  Then for each $\lambda \in \Lambda$ we define 
\begin{align*}
\rho(x,y,\lambda) &= \sup_{t \in [0,T]}\abs{\lambda(t) - t} \vee \sup_{t \in [0,T]} r(x(t), y(\lambda(t))
\end{align*}
and define $\rho : D([0,T]; S) \times D([0,T]; S) \to  \reals$ by 
\begin{align*}
\rho(x,y) &= \inf_{\lambda \in \Lambda} \rho(x,y,\lambda)
= \inf_{\lambda \in \Lambda} 
\sup_{t \in [0,T]}\abs{\lambda(t) - t} \vee \sup_{t \in [0,T]} r(x(t), y(\lambda(t))
\end{align*}
\end{defn}

\begin{lem}\label{SkorohodJ1RhoMetric}$\rho$ is a metric on $D([0,T];S)$.
\end{lem}
\begin{proof}
It is clear that $\rho(x,y) \geq 0$, now suppose that $\rho(x,y) =
0$.  By definition we can find a sequence $\lambda_n \in \Lambda$ such
that $\sup_{t \in [0,T]} \abs{\lambda_n(t) - t} < 1/n$ and $\sup_{t
  \in [0,T]} r(x(t), y(\lambda_n(t)) < 1/n$.  From the former
inequality we see that $\lim_{n \to \infty} \lambda_n(t) = t$ and the second
inequality we see that $\lim_{n \to \infty} y(\lambda_n(t)) = x(t)$.
The sequence $\lambda_n(t)$ has either a decreasing subsequence or a increasing subsequence
therefore passing to that subsequence and using the fact that $y$ is cadlag we see that either $x(t) = y(t)$ or $x(t) =
y(t-)$.  In particular, $x(t) = y(t)$ at all continuity points of
$y(t)$ and since the set of discontinuity points of $y$ is countable
it follows that the set of continuity points is dense in $[0,T]$.
Therefore for every $0 \leq t < T$ we can find a sequence $t_n$ of continuity points of $y$
such that $t_n \downarrow t$ and therefore by right continuity
of $y$ we conclude $x(t) = y(t)$.  The fact that $x(T) = y(T)$ follows from
the fact that $\lambda_n(T)=T$ for all $n \in \naturals$ thus 
\begin{align*}
y(T) &= \lim_{n \to \infty} y(\lambda_n(T)) = x(T)
\end{align*}

To see symmetry of $\rho$ we first note that $\lambda \in \Lambda$
implies $\lambda^{-1} \in \Lambda$.   To see this, it is first off
clear that $\lambda^{-1}$ exists because $\lambda$ is a bijection.
The fact that $\lambda^{-1}$ is strictly increasing follows because
if $0 \leq t < s \leq T$ and $0 \leq \lambda^{-1}(s) \leq \lambda^{-1}(t) \leq T$ then strictly
increasing and bijective nature of $\lambda$ tells $s \leq t$ which is
contradiction.  To see that $\lambda^{-1}$ is continuous, pick $0 < t
< T$ and let $\epsilon > 0$ be given such that $0 < \lambda^{-1}(t) -
\epsilon < \lambda^{-1}(t)  < \lambda^{-1}(t) + \epsilon < T$.  By strict increasingness
and bijectivity 
of $\lambda$ we know that $0 < \lambda(\lambda^{-1}(t) -
\epsilon) < t < \lambda(\lambda^{-1}(t) + \epsilon) < T$.  Let 
\begin{align*}
\delta &< ( t -  \lambda(\lambda^{-1}(t) -\epsilon)) \wedge (
\lambda(\lambda^{-1}(t) +\epsilon) -t)
\end{align*}
and note by the strict increasingness of $\lambda^{-1}$ we have
\begin{align*}
0 &< \lambda^{-1}(t) -\epsilon < \lambda^{-1}(t - \delta) <
\lambda^{-1}(t) < \lambda^{-1}(t + \delta) < \lambda^{-1}(t) + \epsilon
< T
\end{align*}
Now by the bijectivity of $\lambda$ we know that by a change of variables
\begin{align*}
\sup_{0 \leq t \leq T} \abs{\lambda(t) - t} &= \sup_{0 \leq s \leq T} \abs{s - \lambda^{-1}(s)} \\
\sup_{0 \leq t \leq T} r(x(t), y(\lambda(t))) &= \sup_{0 \leq s \leq
  T} r(x(\lambda^{-1}(s)), y(s)) = 
\sup_{0 \leq s \leq  T} r(y(s), x(\lambda^{-1}(s)))\\
\end{align*}
and therefore $\rho(x,y,\lambda) = \rho(y,x,\lambda^{-1}$.  Because
inversion is a bijection on $\Lambda$ we then get
\begin{align*}
\rho(x,y) &= \inf_{\lambda \in \Lambda} \rho(x,y,\lambda) =
\inf_{\lambda \in \Lambda} \rho(y,x,\lambda^{-1}) = \inf_{\lambda^{-1}
  \in \Lambda} \rho(y,x,\lambda^{-1}) = \rho(y,x)
\end{align*}
\end{proof}

The metric $\rho$ defines the Skorohod $J_1$ topology on the space
$D([0,T];S)$.  We emphasize here that we are actually interested in
the underlying topology as much as the metric space structure itself
since $\rho$ is not a complete metric.

\begin{examp}\label{NoncompletenessSkorohod}Let $f_n =
  \characteristic{[1/2, 1/2 + 1/(n+2))}$ for $n >0$ be a sequence in
  $D([0,1];\reals)$.  We show that $f_n$ is a Cauchy sequence with
  respect to $\rho$ but $f_n$ does not converge in the $J_1$ topology.
To see that $f_n$ is Cauchy, let $n > 0$ be given and suppose $m \geq
n$.  Define 
\begin{align*}
\lambda_{n,m}(t) &= \begin{cases}
t & \text{if $0 \leq t \leq 1/2$} \\
\frac{n+m+2}{n+2}(t - 1/2) + 1/2 & \text{if $1/2 \leq t < 1/2 +
  1/(n+m+2)$} \\
\frac{\frac{1}{2} - \frac{1}{n}}{\frac{1}{2} - \frac{1}{n+m+2}}(t - \frac{1}{2} - \frac{1}{n+m+2}) + \frac{1}{2} + \frac{1}{n} & \text{if $1/2 +
  1/(n+m+2) \leq t \leq 1$} \\
\end{cases}
\end{align*}
so that $f_{n+m} (t) = f_{n}(\lambda_{m+n}(t))$ for all $t \in [0,1]$
and $\sup_{0 \leq t \leq 1} \abs{\lambda_{m+n}(t) - t} = \frac{1}{n} -
\frac{1}{n+m+2} < \frac{1}{n}$ which shows $\rho(f_n, f_{n+m}) < \frac{1}{n}$.

\begin{clm}If $f_n$ converges in then it must converge to $0$.
\end{clm}

Suppose that $f_n$ converges to some $f \in D([0,1];\reals)$.  Then
there exist $\lambda_n \in \Lambda$ such that $\lim_{n \to \infty} \sup_{0 \leq t \leq 1}
\abs{\lambda_n(t) - t} = 0$ and 
$\lim_{n \to \infty} \sup_{0 \leq t \leq 1}\abs{f_n(t) -
  f(\lambda_n(t))} = 0$.  Therefore for each $0 \leq t \leq 1$ that is
a point of continuity of $f$ we have 
$\lim_{n \to \infty} f_n(t) = \lim_{n \to \infty} f(\lambda_n(t)) =
f(t)$.  By definition of $f_n(t)$ and Lemma
\ref{CadlagCountableDiscontinuitySet} we see that $f(t) = 0$ for all
but a countable number of $0 \leq t \leq 1$.  Therefore by right
continuity and the existence of left limits we conclude $f(t) = 0$ for
all $0 \leq t \leq 1$.  Since $f(\lambda(t))$ is identically zero for
all $\lambda \in \Lambda$ we conclude that $\rho(f_n, 0) = 1$ hence
$f_n$ does not converge.
\end{examp}

\begin{defn}Given $f \in D([0,T];S)$ the function
\begin{align*}
w(f,\delta) &= \inf_{\substack{0=t_0 < t_1 < \dotsb < t_n = T \\
  \min_{1 \leq i \leq n} (t_i - t_{i-1}) > \delta \\ n \in \naturals}}
\max_{1 \leq i \leq n} \sup_{t_{i-1} \leq s < t < t_i} r(f(s), f(t))
\end{align*}
is called the modulus of continuity.
\end{defn}

\begin{lem}\label{SkorohodJ1ModulusOfContinuity}If $f \in D([0,T];S)$ then $\lim_{\delta \to 0} w(f,\delta)
  = 0$.
\end{lem}
\begin{proof}
First note that for fixed $f$ the function $w(f, \delta)$ is a
non-decreasing function of $\delta$.  This is simply because any
candidate partition $0 = t_0 < t_1 < \dotsb < t_n=T$ with $\min_{1
  \leq i \leq n} (t_i - t_{i-1}) > \delta$ is also a candidate for any
smaller value of $\delta$.  Thus the set of candidate partitions gets
larger as $\delta$ shrinks and the infimum over the set of candiates
shrinks.

Let $\epsilon > 0$ be given.  Define $t_0 = 0$ then so long as
$t_{i-1} < T$ we inductively define
$t_i = \inf \lbrace t > t_{i-1} \mid r(f(t), f(t_{i-1})) > \epsilon \rbrace \wedge T$.  We claim that
there exists $n$ such $t_n = T$.  First, note that the sequence $t_i$
is strictly increasing while $t_i < T$ by the right continuity of
$f$.  If there are an infinite number of $t_i < T$ then by compactness
of $[0,T]$ there is a limit point $0 \leq t \leq T$.  However the
existence of the left limit $f(t-)$ says
exists $\delta > 0$ such that for all $0 < t - s < \delta$ we have
$r(f(s), f(t-)) < \epsilon/3$.
This is a contradiction since we can find an $n > 0$ such that for all
$i \geq n$ we have $t - t_i < \delta$.  By definition of the
$t_i$ for any $i \geq n+1$ we can pick
$t_i \leq s < t$ such that $r(f(s), f(t_{i-1})) >
\epsilon$ which provides us with $0 < t -s < \delta$ and 
\begin{align*}
r(f(s),f(t-)) &> r(f(s), f(t_{i-1})) - r(f(t_{i-1}), f(t-)) > \epsilon -
\epsilon/2 = \epsilon/2
\end{align*}

Thus we have constructed a sequence $0 =t_0 < t_1 < \dotsb < t_n = T$
such that $\max_{1 \leq i \leq n} \sup_{t_{i-1} < s < t < t_i}
r(f(s),f(t)) < 2 \epsilon$ so it we define $\delta = \frac{1}{2} \min_{1 \leq i
  \leq n} (t_i - t_{i-1})$ we have shown $w(f, \delta) \leq 2
\epsilon$.  Since $\epsilon$ was arbitrary and $w(f,\delta)$ is a
non-decreasing function of $\delta$ we are done.
\end{proof}

Even though the metric $\rho$ is not complete, the underlying topology
is Polish because we can define a topologically equivalent metric that is
complete.  To repair the incompleteness of $\rho$ we have to be a bit
more strict about the types of time changes that are allowed; more
specifically we have to prevent time changes are asymptotically flat
(or by considering taking the inverse of a time change prevent time
changes that are asymptotically vertical).  The following is a way of
quantifying such a requirement.
\begin{defn}For every $\lambda \in \Lambda$ define 
\begin{align*}
\gamma(\lambda) =
  \sup_{0 \leq s < t \leq T} \abs{\log \frac{\lambda(t) -
      \lambda(s)}{t-s}}
\end{align*}
For every $x,y \in D([0,T]; E)$ define 
\begin{align*}
d(x,y) = \inf_{\substack{\lambda \in \Lambda \\ \gamma(\lambda) <
    \infty}} \gamma(\lambda) \vee \sup_{0 \leq t \leq T} r(x(t) ,y(\lambda(t)))
\end{align*}
\end{defn}

The main goal is to prove that $d$ is a metric that is topologically equivalent to $\rho$.
Before proving that we need a few facts about $\gamma$.
\begin{lem}\label{SkorohodJ1GammaFacts}$\gamma(\lambda) = \gamma(\lambda^{-1})$ and
  $\gamma(\lambda_1 \circ \lambda_2) \leq \gamma(\lambda_1) +
  \gamma(\lambda_2)$.
\end{lem} 
\begin{proof}
These both follow from reparameterizations using the fact that
$\lambda^{-1}$ is a strictly increasing bijection.  For the first
\begin{align*}
\gamma(\lambda) &= 
\sup_{0 \leq s < t \leq T} \abs{\log  \frac{\lambda(t) -
    \lambda(s)}{t-s}} \\
&=\sup_{0 \leq \lambda^{-1}(s) < \lambda^{-1}(t) \leq T} \abs{\log
  \frac{\lambda(\lambda^{-1}(t)) -
    \lambda(\lambda^{-1}(s))}{\lambda^{-1}(t)-\lambda^{-1}(s)}} \\
&= \sup_{0 \leq s < t \leq T} \abs{\log  \frac{\lambda^{-1}(t) - \lambda^{-1}(s)}{t-s}} 
\end{align*}
and for the second
\begin{align*}
\gamma(\lambda) &= 
\sup_{0 \leq s < t \leq T} \abs{\log  \frac{\lambda_2(\lambda_1(t)) -
    \lambda_2(\lambda_1(s))}{t-s}} \\
&\leq \sup_{0 \leq s < t \leq T} \abs{\log  \frac{\lambda_2(\lambda_1(t)) -
    \lambda_2(\lambda_1(s))}{\lambda_1(t)-\lambda_1(s)}} +
\sup_{0 \leq s < t \leq T} \abs{\log  \frac{\lambda_1(t) -
    \lambda_1(s)}{t-s}} \\
&\leq \sup_{0 \leq s < t \leq T} \abs{\log  \frac{\lambda_2(t) -
    \lambda_2(s)}{t-s}} +
\sup_{0 \leq s < t \leq T} \abs{\log  \frac{\lambda_1(t) -
    \lambda_1(s)}{t-s}} \\
&=\gamma(\lambda_2) + \gamma(\lambda_1)
\end{align*}
\end{proof}

\begin{lem}\label{SkorohodEquivalenceA}For all $\lambda \in \Lambda$ such that $\gamma(\lambda) < 1/2$ we have
  $\sup_{0 \leq t \leq T} \abs{\lambda(t) - t} \leq
  2T\gamma(\lambda)$.  For all $f,g \in D([0,T]; S)$ such that $d(f,g) < 1/2$ we
  have $\rho(f,g) \leq 2Td(f,g)$.
\end{lem}
\begin{proof}
From the inequality $1+x \leq e^x$ we have
$\log(1+2x) \leq 2x$ for all $x > -1/2$ and therefore for $0 < x <
1/2$ we have $\log(1-2x) \leq -2x < -x < 0$.  
Similarly we have
$\log(1-2x) \leq -2x$ for all $x < 1/2$ and therefore for $0 < x <
1/2$ we have $\log(1-2x) \leq -2x < -x < 0$ for $0 < x < 1/2$.  On the
other hand, we see that $\frac{d}{dx} \left ( \log(1+2x) - x \right) =
\frac{2}{1+2x} - 1$ is positive for $0 < x < 1/2$ and therefore we
conclude 
\begin{align*}
\log(1-2x) &< -x < 0 < x < \log(1+2x) \text{ for } 0 < x < 1/2
\end{align*}

Suppose $\gamma(\lambda) < 1/2$ and let $0 < t \leq T$.  By definition and the
fact that $\lambda(0) = 0$ we have
\begin{align*}
\abs{\log \frac{\lambda(t)}{t}} &\leq \sup_{0 \leq s < t \leq
  T}\abs{\log \frac{\lambda(t) - \lambda(s)}{t-s}} = \gamma(\lambda) 
\end{align*}
and therefore we get
\begin{align*}
\log(1-2\gamma(\lambda)) &< -\gamma(\lambda) < \log \frac{\lambda(t)}{t} < \gamma(\lambda) < \log(1+2\gamma(\lambda))
\end{align*}
and exponentiating
\begin{align*}
1 - 2\gamma(\lambda) < \frac{\lambda(t)}{t} < 1 + 2 \gamma(\lambda)
\end{align*}
and therefore $\abs{\lambda(t) - t} < 2 T \gamma(\lambda)$ for $0 < t \leq T$.  Since $\lambda(0) - 0 = 0$ 
it follows that $\sup_{0 \leq t \leq T} \abs{\lambda(t) - t} \leq 2 T \gamma(\lambda)$.

Now suppose we have $d(f,g) < 1/2$.  Let $0 < \epsilon < 1/2 - d(f,g)$ be given and
select $\lambda \in \Lambda$ such that $\gamma(\lambda) <
d(f,g) + \epsilon$ and $\sup_{0 \leq t \leq T} r(f(t),
g(\lambda(t))) < d(f,g) + \epsilon$.  By what we have just shown, we
get that $\sup_{0 \leq t \leq T} \abs{\lambda(t) - t} < 2 T
\gamma(\lambda) < 2T(d(f,g) + \epsilon)$ and therefore $\rho(f,g) <
2T(d(f,g) + \epsilon)$.  Now let $\epsilon \to 0$.
\end{proof}

\begin{lem}\label{SkorohodEquivalenceB}For any $\delta>1/4$ if $\rho(f,g) < \delta^2$ then $d(f,g) \leq 4\delta + w(f,\delta)$.
\end{lem}
\begin{proof}
Let $\delta > 0$ be given, by definition of $w(f,\delta)$ choose a partition $0=t_0 < t_1 < \dotsb < t_n=T$ such that 
$t_i - t_{i-1} > \delta$ and $\sup_{t_{i-1} \leq s \leq t < t_i} \abs{f(t) - f(s)} < w(f,\delta) + \delta$ 
for all $i=1, \dotsc, n$.  Using the fact that $\rho(f,g) < \delta^2$ to pick a $\mu \in \Lambda$ such that 
$\sup_{0 \leq t \leq T} r(f(t),g(\mu(t))) < \delta^2$  and $\sup_{0 \leq t \leq T} \abs{\mu(t) - t} < \delta^2$.  Note that 
by the properties of $\mu$ we also have $\sup_{0 \leq t \leq T} r(f(\mu^{-1}(t)), g(t)) < \delta^2$.

Now we construct an appropriate $\lambda$ with which to bound $d(f,g)$.  
Define $\lambda(t_i) = \mu(t_i)$ for each $i=0,\dotsc, n$ and extend by linear
interpolation
\begin{align*}
\lambda(t) &= \frac{t-t_{i-1}}{t_i-t_{i-1}} \mu(t_i) + \frac{t-t_{i}}{t_{i-1}-t_{i}} \mu(t_{i-1}) \text{ for $t_{i-1} \leq t \leq t_i$}
\end{align*}
From the fact that $\mu(t_{i-1}) < \mu(t_i)$ it follows that $\lambda(t)$ is strictly increasing, $\lambda(0) = \mu(0) = 0$, 
$\lambda(T) = \mu(T) = T$ and $\lambda$ is piecewise linear hence continuous; thus $\lambda \in \Lambda$.  Moreover by
the increasingness of $\lambda$ and $\mu$ (and $\mu^{-1}$) we have $t_{i-1} \leq t \leq t_i$ is equivalent to $\lambda(t_{i-1}) \leq \lambda(t) \leq \lambda(t_i)$
which is in turn equivalent to $\mu^{-1}(\lambda((t_{i-1})) = t_{i-1} \leq \mu^{-1}(\lambda(t)) \leq t_i = \mu^{-1}(\lambda( t_i))$.
Thus
\begin{align*}
r(f(t), g(\lambda(t))) &\leq r(f(t), f(\mu^{-1}(\lambda(t))))  + r(f(\mu^{-1}(\lambda(t)))- g(\lambda(t))) \\
&\leq w(f,\delta) + \sup_{0 \leq t \leq T}  f(\mu^{-1}(\lambda(t))) \leq w(f,\delta) + \delta^2
\end{align*}
As for bounding $\gamma(\lambda)$ we have
\begin{align*}
\abs{\lambda(t_i) - \lambda(t_{i-1}) - (t_i - t_{i-1})} &= \abs{\mu(t_i) - \mu(t_{i-1}) - (t_i - t_{i-1})} \\
&\leq \abs{\mu(t_i) -  t_{i}}  + \abs{\mu(t_{i-1}) - t_{i-1}} \leq 2 \delta^2 < 2 \delta (t_i - t_{i-1})
\end{align*}
Recalling that $\lambda$ is linear on each interval $[t_{i-1},t_i]$ we note that this inequality simply says that the
slope of $\lambda$ on the linear segment $[t_{i-1},t_i]$ is in the interval $(1 - 2\delta, 1-2\delta)$.  Thus the inequality trivially extends 
to any $t_{i-1} \leq s \leq t \leq t_i$.  For other $s<t$  pick $i<j$ with $t_{i-1} \leq s \leq t_i$ and $t_{j-1} \leq t \leq t_j$.  Then we have
\begin{align*}
\abs{\lambda(t) - \lambda(s) - (t-s)} 
&\leq \abs{\lambda(t) - \lambda(t_{j-1}) - (t-t_{j-1})} + \sum_{k=i}^{j-2} \abs{\lambda(t_{k+1}) - \lambda(t_{k}) - (t_{k+1}-t_{k})}  + \\
&\abs{\lambda(t_i) - \lambda(s) - (t_i-s)} \\
&\leq 2 \delta (t-t_{j-1}) + 2 \delta \sum_{k=i}^{j-2} (t_{k+1}-t_{k})  + 2\delta (t_i-s) \\
&= 2\delta (t-s) \\
\end{align*}
and therefore 
\begin{align*}
\log(1 - 2\delta) \leq \log \left( \frac{\lambda(t) - \lambda(s)}{t-s} \right) \leq \log(1+2\delta)
\end{align*}
For arbitrary $\delta$ we have $\log(1+2\delta) \leq 2\delta < 4 \delta$ (Theorem \ref{BasicExponentialInequalities}) and by 
Taylor's Theorem (Lemma \ref{LagrangeFormRemainder}) we have for any $0 < \delta < 1/4$ there is a $0 < c < \delta < 1/4$ such that
\begin{align*}
\log(1 - 2\delta) &= \frac{-2\delta}{1 - 2c} > -4\delta
\end{align*}
and therefore $\gamma(\lambda) < 4\delta$.
\end{proof}
Now we are ready to show that $d$ is a metric and generates the same topology as $\rho$.
\begin{thm}\label{SkorohodJ1Metric}$d$ is a metric on $D([0,T];S)$ that is topologically equivalent to
  $\rho$.
\end{thm}
\begin{proof}
The fact that $d(f,g) \geq 0$ is immediate. Suppose $d(f,g)$ and pick $\lambda_n$ such that
$\lim_{n \to \infty} \gamma(\lambda_n) = 0$ and $\lim_{n \to \infty} \sup_{0 \leq t \leq
  T} r(f(t), g(\lambda_n(t))) = 0$.  By Lemma
\ref{SkorohodEquivalenceA}  we know that $\lim_{n \to \infty} \sup_{0
  \leq t \leq T} \abs{\lambda_n(t) - t} = 0$ as well and therefore we
can repeat the argument of Lemma \ref{SkorohodJ1RhoMetric} to conclude $f=g$.

To see symmetry just note that by reparametrizing and Lemma \ref{SkorohodJ1GammaFacts}
\begin{align*}
d(f,g) &= \inf_{\substack{\lambda \in \Lambda \\ \gamma(\lambda) <
    \infty}} \gamma(\lambda) \vee \sup_{0 \leq t \leq T} r(f(t),
g(\lambda(t))) \\
&= \inf_{\substack{\lambda \in \Lambda \\ \gamma(\lambda) <
    \infty}} \gamma(\lambda^{-1}) \vee \sup_{0 \leq \lambda^{-1}(t) \leq T}
r(g(t), f(\lambda^{-1}(t))) = d(g,f)
\end{align*}
and similarly with the triangle inequality.

TODO: Write out the triangle inequality part.

To see that $\rho$ and $d$ define the same topology we first show that every ball
in the $\rho$ metric contains a ball in the $d$ metric and vice versa.  For general notation
let $B_\rho(f,\epsilon)$ and $B_d(f,\epsilon)$ denote balls of radius $\epsilon$ centered at $f$ in the $\rho$ and $d$ metric
respectively.  Let $f$ and $r > 0$ be given.  Define $\delta < \frac{\epsilon}{2T} \wedge \frac{1}{4}$ and then apply Lemma \ref{SkorohodEquivalenceA}
to conclude that $d(f,g) \leq \delta$ implies $\rho(f,g) \leq 2T d(f,g) < \epsilon$; thus $B_d(f,\delta) \subset B_\rho(f,\epsilon)$.  On the other hand for a given $\epsilon>0$
because $\lim_{\delta \to 0} 4 \delta + w(f,\delta) = 0$ we can find $0 < \delta < 1/4$ such that $4 \delta + w(f,\delta) < \epsilon$ and therefore
by Lemma \ref{SkorohodEquivalenceB} we have $B_\rho(f, \delta^2) \subset B_d(f, \epsilon)$.

Now let $U$ be an open set in the topology defined by $d$.  For every $f \in U$, by openness of $U$ we find $\epsilon_f>0$ such that $B_d(f,\epsilon_f) \subset U$.
By the above argument we may find $\delta_f>0$ such that $B_\rho(f, \delta_f) \subset B_d(f,\epsilon_f) \subset U$.  Therefore we can write $U = \cup_{f \in U} B_\rho(f, \delta_f)$
which shows that $U$ is an open set in the topology defined by $\rho$ as well.  It is clear that the argument is symmetric in the role of $\rho$ and $d$ and therefore we
have shown that $d$ and $\rho$ are topologically equivalent metrics.
\end{proof}

The goal in introducing $d$ was to provide a complete metric; a useful
thing to check first is that $d$ fixes the example which showed $\rho$
was not a complete metric.
\begin{examp}Here we continue the Example
  \ref{NoncompletenessSkorohod} by showing directly that $f_n$ is not
  Cauchy in the metric $d$.  Because $f_n$ are indicator functions it
  follows that $\sup_{0 \leq t \leq 1} r(f_{n+m}(t), f_{n}(\lambda(t)))$ is
  either $0$ or $1$.  Therefore if $f_n$ is Cauchy then we can find
  $\lambda_{nm}(t)$ such that $\sup_{0 \leq t \leq 1} r(f_{n+m}(t),
    f_{n}(\lambda_{nm}(t))) = 0$.  By definition this tells us that
  $\lambda_{nm}([1/2,1/2 + 1/n+m+2]) =  [1/2,1/2+1/n]$ 
(of course $\lambda_{nm}([0,1/2]) = [0,1/2]$ and
$\lambda_{nm}([1/2 + 1/n+m+2, 1]) = [1/2 + 1/n,1]$ as well).  
From this fact we see that $\gamma(\lambda_{mn}) \geq
\frac{n+m+2}{n+2} > 1$ which shows that $d(f_n, f_{n+m}) \geq 1$ so
$f_n$ is not Cauchy with respect to $d$.
\end{examp}

Now we show that the metric $d$ is indeed complete.  In addition we show
that the $J_1$ topology is separable which shows us that it defines a Polish space.
This will allow us to apply our theory of weak convergence.
\begin{thm}\label{SkorohodJ1MetricPolish}Let $S$ be a complete metric space, then the metric $d$ on $D([0,T]; S)$
is complete.  Morever if $S$ is separable then $D([0,T];S)$ is separable in the $J_1$ topology.
\end{thm}
\begin{proof}
Let $f_n$ be a Cauchy sequence in $D([0,T]; S)$ with the metric $d$.  As a general principle of metric spaces it
suffices to show that $f_n$ has a convergent subsequence $f_{n_j}$.  Suppose that such a subsequence exists and converges
to $f$.  Then we may find $n_j$ such that $d(f,f_{n_j}) < \epsilon/2$ and $d(f_m, f_{n_j}) < \epsilon/2$ for all $m \geq n_j$; it follows
that in fact $d(f,f_m)<\epsilon$ for all $m \geq n_j$ and thus $f_n \to f$.

Using the Cauchy property we can find a subsequence $n_j$ such that $d(f{n_j}, f_{n_{j+1}}) < 2^{-j}$.  Therefore we have $\lambda_j$ 
such that $\gamma{\lambda_j} < 2^{-j}$ and $\sup_{0 \leq t \leq T} r(f_{n_j}(t), f_{n_{j+1}}(\lambda_j(t))) < 2^{-j}$.  Moreover from 
Lemma \ref{SkorohodEquivalenceA} we have $\sup_{0 \leq t \leq T} \abs{\lambda_j(t) -t } < T 2^{-j+1}$ for all $j \in \naturals$.  
Therefore for every $m,n \in \naturals$,
\begin{align*}
&\sup_{0 \leq t \leq T} \abs{\lambda_{m+n+1} \circ \lambda_{m+n} \circ \dotsb \circ \lambda_n(t) -  \lambda_{m+n} \circ \dotsb \circ \lambda_n(t)} \\
&= \sup_{0 \leq t \leq T} \abs{\lambda_{m+n+1}(t) - t} \leq T 2^{-(m+n)}
\end{align*}
which shows us that for fixed $n \in \naturals$ the sequence $\lambda_{m+n} \circ \dotsc \circ \lambda_n$ is Cauchy in $C([0,T], \reals)$ with
the sup norm.  Since this latter space is complete we know that there is  a limit $\nu_n$.  

\begin{clm} $\nu_n \in \Lambda$ and $\gamma(\nu_n) \leq 2^{-n+1}$.
\end{clm}

It is clear that $\nu_n$ is continuous as the uniform limit of continuous functions.  Moreover $\nu_n(0) = \lim_{m \to \infty} \lambda_{m+n} \circ \dotsb \lambda_{n}(0) = 0$
and similarly $\nu_n(T) = T$.  As a uniform limit of strictly increasing functions we also know that $\nu_n$ is non-decreasing.  To estimate $\gamma(\nu_n)$ we compute
\begin{align*}
&\abs{\log \frac{\lambda_{m+n} \circ \dotsb \circ \lambda_n(t) - \lambda_{m+n} \circ \dotsb \circ \lambda_n(s)}{t-s}} \\
&\leq \gamma(\lambda_{m+n} \circ \dotsb \circ \lambda_n) \\
&\leq \gamma(\lambda_{m+n}) + \dotsm +\gamma(\lambda_n) < 2^{-(m+n)} + \dotsm + 2^{-n} < 2^{-n+1}
\end{align*}
and therefore taking the limit as $m \to \infty$ we conclude 
\begin{align*}
\abs{\log \frac{\nu_n(t) - \nu_n(s)}{t-s}} &\leq 2^{-n+1}
\end{align*}
which show both that $\nu_n$ is strictly increasing (hence $\nu_n \in \Lambda$) and moreover that $\gamma(\nu_n) < 2^{-n+1}$ and the claim is shown.

Now note that $\nu_n = \nu_{n+1} \circ \lambda_n$.  From this it follows that
\begin{align*}
\sup_{0 \leq t \leq T} r(f_{n_j}(\nu_j^{-1}(t)), f_{n_{j+1}}(\nu_{j+1}^{-1}(t))) &= \sup_{0 \leq t \leq T} r(f_{n_j}(t),  f_{n_{j+1}}(\lambda_j(t))) < 2^{-j}
\end{align*}
and therefore for $j,m \in \naturals$ we have
\begin{align*}
\sup_{0 \leq t \leq T} r(f_{n_j}(\nu_j^{-1}(t)), f_{n_{j+m}}(\nu_{j+m}^{-1}(t))) &\leq 2^{-j} + \dotsm + 2^{-j-m+1} < 2^{-j+1}
\end{align*}
which shows that $f_{n_j} \circ \nu_j^{-1}$ is a Cauchy sequence in $D([0,T]; S)$ with respect to the sup norm.  Recalling that the sup norm
makes $D([0,T];S)$ into a complete (but not separable) space we can find a limit $f \in D([0,T]; S)$.  Thus $\sup_{0 \leq t \leq T} r(f_{n_j}(\nu_j^{-1}(t)), f(t)) \to 0$
and moreover $\gamma(\nu_j) \to 0$ which shows us that $d(f_{n_j}, f) \to 0$.

TODO: Show separability
\end{proof}

\section{Compactness and Tightness in Skorohod Space}

The entire point behind constructing the $J_1$ topology on $D([0,T]; S)$ was to
make it a Polish space so that Prohorov's Theorem can be applied to understand
weak convergence of cadlag stochastic processes.  The other pieces of the puzzle
in applying Prohorov's Theorem are being able to prove tightness and being able to
characterize limits (e.g. by understanding the finite dimensional distributions).  The latter
problem has little to do with topological aspects of path space (and to be honest in many
cases is an impossibly difficult nut to crack).  The former problem is deeply tied into the
nature of compactness in path space and we now turn to the consideration of such matters.

The first thing to do is to prove an analogue of the Arzela-Ascoli theorem that characterizes
relatively compact sets in $D([0,T]; S)$.  The proof of such a theorem ultimately rests on
approximation by step functions so we first prove that certain collections of step functions
are compact.

\begin{lem}\label{SkorohodJ1CompactSetsOfStepFunctions}Let $(S,r)$ be a metric space and let $K \subset S$.  Let $\delta > 0$ and define
$A(K, \delta) \subset D([0,T];S)$ be the set of functions $f$ for which there is a partition
$0=t_0 < t_1 < \dotsb < t_n=T$ with $t_i-t_{i-1} >  \delta$ for $i=1, \dotsc, n$, points $x_1, \dotsc, x_n \in K$ with $x_j \neq x_{j-1}$ for $j=2, \dotsc, n$ such that $f(t) = x_i$ for $t_{i-1} \leq t < t_i$.  If $K$ is compact in $S$ then $A(K, \delta)$ is relatively compact in the $J_1$ topology.
\end{lem}
\begin{proof}
It suffices to show that every sequence in $A(K,\delta)$ has a convergent subsequence. 
For each $f_n$ we by the definition of $A(K,\delta)$, let $0=t_{0,n} < \dotsm < t_{m_n, n}=T$   be a partition with $t_{j,n}-t_{j-1,n}>\delta$ for $j=1, \dotsc, m_n$ and $x_{1, n}, \dotsc, x_{m_n, n}$ be elements of $K$ such that $f_n(t) = x_{j,n}$ for $t_{j-1,n} \leq t < t_{j,n}$.  Let $m = \liminf_{n \to \infty} m_n$; that is to say $m$ is the smallest $m_n$ for which there are an infinite number of $f_n$ whose partitions have length $m$.
Note that $m \leq T/\delta < \infty$.  Since we are only looking for a convergence subsequence we can pass to the subsequence of $f_n$ for which $m_n = m$ and therefore we assume that all partitions have length $m$.

Consider the sequence $t_{1,n}$ and the sequence $x_{1,n}$.  Let $t_1 = \limsup_{n \to \infty} t_{1,n}$.  We know that $\delta < t_{1,n} \leq T$ thus $t_1 \in [\delta,T]$; pick a subsequence $N^1$ such that $t_{1,n} \to t_1$ along $N^1$. 
We also have $x_{1,n} \in K$ along $N^1$ so by compactness of  $K$ there 
is $x_1 \in K$ and a subsequence $N^2 \subset N^1$ such that $t_{1,n} \to t_1$ and $x_{1,n} \to x_1$ along $N^2$.  
In fact we can ask for another property of the sequence of $t_{1,n}$.  The function $g(t) = \abs{\frac{t_1}{t}}$ is continuous at $t_1$ 
and equals $g(t_1) = 0$ thus $g(t_{1,n}) \to 0$ along $N^2$ and we may pass to a further subsequence $N^3 \subset N^2$ (which we now
denote by $n_j$) to arrange that
\begin{itemize}
\item[(i)] $\lim_{j \to \infty} t_{1,n_j} = t_1$ with $\delta \leq t_1 \leq T$
\item[(ii)] $\abs{\frac{t_1}{t_{1, n_j}}} < \frac{1}{j}$ for $j \in \naturals$
\item[(iii)] $\lim_{j \to \infty} x_{1,n_j} = x_1$ with $x_1 \in K$
\end{itemize}

TODO: Get rid of the $k=1$ step since the induction step is clear once we define the trivial $k=0$ step.

If $m = 1$ we stop here (and note that we must have $t_1 = T$), otherwise we iterate to find further subsequences. 
To see the induction step we suppose that we have run the procedure $k<m$ times 
so that we have a subsequence $N^k$ and for each
$1 \leq i \leq k$ we have $i\delta \leq t_i < T$, $x_i \in K$ such that
\begin{itemize}
\item[(i)] $\lim_{j \to \infty} t_{i,n_j} = t_1$ 
\item[(ii)] $\abs{\frac{t_i - t_{i-1}}{t_{i, n_j} - t_{i-1, n_j}}} < \frac{1}{j}$ for $j \in \naturals$
\item[(iii)] $\lim_{j \to \infty} x_{i,n_j} = x_1$ 
\end{itemize}
We now let $t_{k+1} = \limsup_{j \to \infty} t_{k+1, n_j}$.  We know 
that $(k+1) \delta \leq t_{k+1} \leq T - (m - k)\delta$ and in fact $t_{k+1} = T$ if and only if $k+1 = m$.  
Now we replay the argument we used for $k=1$.  We extract a subsequence $N^{1,k} \subset N^k$ such that $t_{k+1, n_j} \to t_k$ along $N^{1,k}$.  Note also
that $m_{n_j} \geq k+1$ along this subsequence.  We use
compactness of $K$ to the get a further subsequence $N^{2,k}$ such that there is $x_{k+1} \in K$ with $x_{k+1, n_j} \to x_{k+1}$ along $N^{2,k}$.  The we use
continuity of $\abs{\frac{t_{k+1} - t_{k}}{t - t_{k, n_j}}}$ at $t_k$ to arrange for a final subsequence $N^{3,k}$ so that if we redefine $n_j$ to be the subsequence
$N^{3,k}$ we have $\abs{\frac{t_{k+1}- t_{k}}{t_{k+1, n_j} - t_{k, n_j}}} < \frac{1}{j}$ for all $j \in \naturals$.

Define $f(t)$ to be equal to $x_k$ on $t_{k-1} \leq t < t_k$ for $k=1, \dotsc, m$ (clearly $f \in D([0,T]; S)$).  We claim $f_n \to f$ in the $J_1$ topology along the subsequence $N^m$.
To see this, let $\lambda_j(t_{k, n_j}) = t_k$ for $k=1, \dotsc, m$ and extend by linear interpolation.  This is well defined because of the fact that $t_m = T$.  
Moreover from the property (ii) we have $\gamma(\lambda_j) \leq \frac{1}{j}$ so that
$\lim_{j \to \infty} \gamma(\lambda_j) = 0$.  We also have $r(f_{n_j}(t), f(\lambda_j(t))) = r(x_{n_j, k}, x_k)$ for $t_{k-1, n_j} \leq t < t_{k, n_j}$.  Thus because there are only finitely many
$x_k$ we can conclude that $\lim_{j \to \infty} \sup_{0 \leq t \leq T} r(f_{n_j}(t), f(\lambda_j(t))) = 0$ and therefore $\lim_{j \to \infty} d(f_{n_j}, f) = 0$.

TODO: Can the proof be simplified if we use the metric $\rho$ to demostrate convergence?
\end{proof}

\begin{thm}\label{ArzelaAscoliTheoremJ1Topology}Let $(S,r)$ be a complete metric space.  A set $A \subset D([0,T]; S)$ is relatively compact in the $J_1$ topology if and only if 
\begin{itemize}
\item[(i)]for each rational number
$t \in [0,T] \cap \rationals$ there exists a compact set $K_t \subset S$ such that $\cup_{f \in A} f(t) \subset K_t$
\item[(ii)]$\lim_{\delta \to 0} \sup_{f \in A} w(f, \delta) = 0$
\end{itemize}
In fact when $A \subset D([0,T]; S)$ is relatively compact in the $J_1$ topology then there is a compact set $K \subset S$ such that $\cup_{f \in A} \cup_{0 \leq t \leq T} f(t) \subset K$.
\end{thm}
\begin{proof}
We consider $D([0,T];S)$ as a metric space with the complete metric $d$.  We first suppose that $A$ is a set satisfying (i) and (ii).
Since a set $A$ is totally bounded if and only if 
its closure is totally bounded and a closed set of  a complete metric space is complete, it suffices to
show that $A$ is totally bounded with respect to $d$ (Theorem \ref{CompactnessInMetricSpaces}).  By (ii) every $k \in \naturals$ we
pick $0 < \delta_k < 1$ such that $\sup_{f \in A} w(f, \delta_k) < \frac{1}{k}$.  Pick $m_k \in \naturals$ with $\frac{1}{m_k} < \delta_k$ and 
define 
\begin{align*}
K^{(k)}  &= \cup_{i=1}^{T m_k} K_{i/m_k}
\end{align*}
so that $K^{(k)}$ is compact and $A_k = A(K^{(k)}, \delta_k)$ is relatively compact by Lemma \ref{SkorohodJ1CompactSetsOfStepFunctions}.
Pick $f \in A$ and since $w(f,\delta_k) < \frac{1}{k}$ we may choose a partition $0=t_0 < t_1 < \dotsb < t_n=T$ such that  for $i=1, \dotsc, n$ we have 
$t_i-t_{i-1} > \delta_k$ and $r(f(s), f(t)) < \frac{1}{k}$ for all $s,t \in [t_{i-1}, t_i)$.  Since we have chose $\frac{1}{m_k} < \delta_k < t_i - t_{i-1}$ note that every
interval $[\frac{j-1}{m_k}, \frac{j}{m_k}]$ has either $0$ or $1$ element of $t_i$ in it.  Define $g \in D([0,T]; S)$ by
\begin{align*}
g(t) &= f(\frac{\floor{m_k t_{i-1}} + 1}{m_k}) \text{ for $t_{i-1} \leq t < t_i$ and $i=1, \dotsc, n$}
\end{align*}
(this says that we define $g(t)$ on $[t_{i-1}, t_i)$ to be the value $f(j/m_k)$ where $j$ is the smallest integer such that $t_{i-1} < j/m_k$).
It is clear that $g \in A_k$ since $g$ is a step function on the partition $\lbrace t_j \rbrace$ and takes values in $K_{j/m_k}$ for appropriate $0 \leq j \leq T m_k$.
Furthermore we have for $t_{i-1} \leq t < t_i$ we have
\begin{align*}
r(f(t), g(t)) &\leq r(f(t), f(\frac{\floor{m_k t_{i-1}} + 1}{m_k})) < \frac{1}{k}
\end{align*}
which shows that $d(f,g) < \frac{1}{k}$.  

Now let $\epsilon > 0$ be given and pick $\frac{1}{k} < \epsilon /2$.  Since $A_k$ is relatively compact it
is totally bounded and thus there exist $g_1, \dotsc, g_n$ such that the balls $B(g_i, \epsilon/2)$ cover
$A_k$.  By the above argument for any $f \in A$ we can find $g \in A_k$ such that $d(f,g) < \frac{1}{k} < \epsilon/2$ 
and therefore by the triangle inequality the balls $B(g_i, \epsilon)$ cover $A$ and $A$ is totally bounded.
 
Now assume that $A$ is compact in $D([0,T];S)$.  To show (i) we actually show the stronger criteria that there exists a compact set $K$ such that $\cup_{f \in A} \cup_{0 \leq t \leq T} f(t) \subset K$.  It suffices to show that $\cup_{f \in A} \cup_{0 \leq t \leq T} f(t)$ is totally bounded.  Let $\delta > 0$ given and use the fact that $A$ is totally bounded to get a set $f_1, \dotsc, f_n \in A$ such that $B(f_j, \delta/2)$ cover $A$.  For each $f_j$ we pick a partition $0=t^j_0 < t^j_1 < \dotsb < t^j_{n_j} = T$ with $\max_{1 \leq i \leq n_j} \sup_{t^j_{i-1} \leq s < t < t^j_i} r(f_j(s), f_j(t)) < \delta/2$.  We claim that the ball $B(f_j(t^j_i); \delta)$ for $0 \leq i \leq n_j$ cover $\cup_{f \in A} \cup_{0 \leq t \leq T} f(t)$.  Now let $f \in A$ and $0 \leq t \leq T$ be given and pick $1 \leq j \leq n$ such that $f \in B(f_j, \delta/2)$ and then pick $\lambda \in \Lambda$ such that $\sup_{0 \leq s \leq T} r(f(s), f_j(\lambda(s))) < \delta/2$; in particular $r(f(t), f_j(\lambda(t))) < \delta/2$.  Pick $1 \leq i \leq n_j$ such that $t^j_{i-1} \leq \lambda(t) < t^j_i$ (TODO: What if $t = T$?)  and then it follows that 
\begin{align*}
r(f(t), f_j(t^j_i)) &\leq r(f(t), f_j(\lambda(t))) + r( f_j(\lambda(t)), f_j(t^j_{i-1})) < \delta
\end{align*}

To see (ii) we argue by contradiction.  Suppose that there exists $\epsilon$ such that $\sup_{f \in A} w(f, \delta) \geq \epsilon$ for all $\delta > 0$.  In particular we can find a sequence $f_n \in A$ such that $w(f_n, 1/n) \geq \epsilon$.  By relative compactness of $A$ we can pass to a convergent subsequence $f_{n_k} \to f$ with $f \in D([0,T];S)$.  But then
we have from Lemma \ref{SkorohodJ1ModulusOfContinuity} (actually we have only proven the relevant part  in Lemma \ref{SkorohodInfiniteJ1ModulusOfContinuity})
\begin{align*}
\epsilon &\leq \limsup_{k \to \infty} w(f_{n_k}, \delta) \leq w(f, \delta)
\end{align*}
for all $\delta > 0$ which is a contradicts $\lim_{\delta \to 0}  w(f, \delta) = 0$.
 
Here is another attempt at an argument to show (i) that I didn't finish.  It suffices to show that $\cup_{f \in A} \cup_{0 \leq t \leq T} f(t) \subset K$ is relatively compact so let $f_n(t_n)$ be a sequence with $f_n \in A$ and $0 \leq t_n \leq T$.  We first make some simple reductions.  By compactness of $[0,T]$ we know there is a subsequence such that $t_n$ converges to a value $0 \leq t \leq T$ and by passing to a further subsequence we may assume that $t_n < t$ or $t_n \geq t$.  By passing to a third subsequence using the compactness of $A$ we may assume that $f_n$ converges to some $f \in D([0,T];S)$ in the $J_1$ topology.  For notational cleanliness we assume that $f_n(t_n)$ represents this final subsequence.  By the fact that $f_n \to f$ in the $J_1$ topology we may assume that there exist $\lambda_n \in \Lambda$ such that $\lim_{n \to \infty} \sup_{0 \leq s \leq T} \abs{\lambda(s) - s} = 0$.  In particular, $\lim_{n \to \infty} \lambda_n(t) = t$ and by passing to a subsequence we may assume that $\lambda_n(t) < t$ or $\lambda_n(t) \geq t$ for all $n$ along the subsequence.   TODO: How to finish this off; it is clear how this shows that $\cup_{f \in A} f(t)$ is relatively compact for a fixed $t$????
\end{proof}

\begin{lem}\label{SkorohodBorelGeneratedByProjections}For every $t \in [0,T]$ let $\pi_t : D([0,T];S) \to
  S$ be the evaluation map $\pi_t(f) = f(t)$.  The Borel $\sigma$-algebra on $D([0,T]; S)$ is
  equal to $\sigma(\lbrace \pi_t \mid t \in [0,T] \rbrace)$ and
  therefore $\mathcal{B}(D([0,T]; S)) = D([0,T];
  S) \cap \mathcal{B}(\reals)^{\otimes [0,\infty)}$.
\end{lem}
\begin{proof}
TODO:
\end{proof}


\section{The space $D([0,\infty); S)$}

For applications it is equally or perhaps more important to deal with the space $D([0,\infty); S)$ of cadlag functions on the half infinite interval.  
Essentially all of the results we have proven for $D([0,T]; S)$ hold but there is some subtlety in getting the definitions right so that the topology
on $D([0,\infty); S)$ gives the right behavior to the restriction maps $D([0,\infty); S) \to D([0,T]; S)$.  The problem that has to be dealt with is best 
illustrated with an example.  Consider the function 
\begin{align*}
f(x) &= \begin{cases}
0 & \text{if $0 \leq x < 1$} \\
1 & \text{if $1 \leq x$}
\end{cases}
\end{align*}
and the approximating sequence
\begin{align*}
f_n(x) &= \begin{cases}
0 & \text{if $0 \leq x < 1+1/n$} \\
1 & \text{if $1+1/n \leq x$}
\end{cases}
\end{align*}
Our intuition is the $J_1$ topology on $D([0,\infty); \reals)$ should make functions close if there is a small time change that makes them
uniformly close; thus we should expect that $f_n$ converges to $f$.  However consider the restriction of these functions to $D([0,1]; \reals)$.  The restriction of
$f$ has a jump discontinuity of size $1$ at the endpoint $x=1$ while the restrictions of $f_n$ are all identically zero.  Because the endpoint of the domain
$[0,1]$ cannot be moved by a time change it is easy to see that $f_n$ does not converge to $f$ in the $J_1$ topology on $D([0,1]; \reals)$.  Another way of 
looking at the example is to observe that is shows the restriction map $D([0,T]; S)$ to $D([0,1]; S)$ for $T > 1$ is not continuous in the $J_1$ topology.  As a
side effect of this one cannot simply glue the spaces $D([0,n];S)$ for $n \in \naturals$ together (formally to create a projective limit) to get a topology on $D([0,\infty); S)$ in the 
same way that one can do so with $C([0,T];S)$ and $C([0,\infty);S)$.  

The good news is that the example we have given illustrates the only problem that has to be
dealt with: namely restricting to a point of discontinuity of an element $f \in D([0,\infty), S)$.  For a given $f$ we have already seen that there are only a countable
number of discontinuities of $f$ so in particular the set of discontinuities has Lebesgue measure zero.  The trick is that if we integrate the restrictions to $D([0,T];S)$ 
over $0 \leq T < \infty$ then these discontinuities won't matter and we can create a metric on $D([0,\infty);S)$.  While one can proceed in this fashion by using the
existing metric on $D([0,T];S)$ to create a metric on $D([0,\infty);S)$ it turns out to be about as much work to just start from scratch.  The advantage in doing so
is that the development can be formally independent of the case of a finite interval (though we do refer to some proofs for details that are left out).  

\begin{defn}Let $(S,r)$ be a metric space, define $\Lambda_\infty$ denote the set of all $\lambda : [0,\infty) \to
  [0,\infty)$ such that $\lambda$ is continuous, strictly increasing and
  bijective.  For $\lambda \in \Lambda_\infty$ define
\begin{align*}
\gamma(\lambda) =
  \sup_{0 \leq s < t \leq T} \abs{\log \frac{\lambda(t) -
      \lambda(s)}{t-s}}
\end{align*}
Then for each $\lambda \in \Lambda_\infty$ we define 
\begin{align*}
\psi(x,y,\lambda, T) &= \sup_{0 \leq t < \infty} r(x(t \wedge T), y(\lambda(t) \wedge T)) \wedge 1
\end{align*}
and define $\rho : D([0,\infty); S) \times D([0,\infty); S) \to  \reals$ by 
\begin{align*}
\rho(x,y) &= \inf_{\lambda \in \Lambda_\infty} 
\int_0^\infty \gamma(\lambda) \vee e^{-T} \left[ \psi(x,y,\lambda,T) \vee (\sup_{0 \leq u \leq T}\abs{\lambda(t) - t} \wedge 1) \right] \, dT 
\end{align*}
$d : D([0,\infty); S) \times D([0,infty); S) \to  \reals$ by 
\begin{align*}
d(x,y) &= \inf_{\substack{\lambda \in \Lambda_\infty \\
\gamma(\lambda) < \infty}} 
\left[ \gamma(\lambda) \vee \int_0^\infty e^{-T} \psi(x,y,\lambda,T) \, dT \right]
\end{align*}
\end{defn}
So note with this definition for each $T$ we are restricting each pair of cadlag functions $x,y$ to the interval $[0,T]$ but also
implicitly thinking of $D([0,T])$ as being embedded in $D([0,\infty))$ by extending as a constant function.

TODO: Note the measurability of $\psi(x,y,\lambda,T)$ as a function of $T$.

Note that $\psi(x,y,\lambda,T)$ is not a continuous function of $T$ for fixed $x,y,\lambda$ (however I believe it is cadlag?)
\begin{examp}
Let $f = \characteristic{[1, \infty)}$ and $g = \characteristic{[2, \infty)}$ and $\lambda(1) = 2$, then $\psi(f,g,\lambda,t) = \characteristic{[0,1) \cup [2,\infty)}$.
\end{examp}

In general when thinking of convergence relative to $d$ (which we have not yet shown is a metric of course) we usually use 
the following rendering of the definition.  We warn the reader that this result is used so frequently that we will rarely make explicit note
of it.

\begin{lem}\label{SkorohodInfinityConvergenceToZero}Let $f_n$ and $g_n$ be sequences in $D([0,\infty); S)$ then $\lim_{n \to \infty} d(f_n, g_n) = 0$ if and only if there exist $\lambda_n \in \Lambda_\infty$ such
that $\gamma(\lambda_n) < \infty$ for all $n \in \naturals$, $\lim_{n \to \infty} \gamma(\lambda_n) = 0$ and 
\begin{align*}
\lim_{n \to \infty} m \lbrace t \in [0, T] \mid \sup_{0 \leq s < \infty} r(f_n(s \wedge t), g_n(\lambda_n(s) \wedge t)) > \epsilon \rbrace = 0
\end{align*} 
for all $0 < T < \infty$ and $0 < \epsilon < 1$.  Moreover if $\lim_{n \to \infty} \gamma(\lambda_n) = 0$ then we also have $\lim_{n \to \infty}   \sup_{0 \leq t \leq T} \abs{\lambda_n(t) - t} = 0$ for every $T > 0$.
\end{lem}
\begin{proof}
Suppose that such $\lambda_n$ exist then for every $0 < T < \infty$ and $0 < \epsilon < 1$, then it follows from the definition of $\psi$ that $\sup_{0 \leq s < \infty} r(f_n(s \wedge t), g_n(\lambda_n(s) \wedge t)) > \epsilon$ if and only if $\psi(f_n,g_n, \lambda_n, t) > \epsilon$
\begin{align*}
\lim_{n \to \infty} \int_0^\infty e^{-t} \psi(f_n,g_n, \lambda_n, t) \, dt
&\leq \lim_{n \to \infty} \int_0^T e^{-t} \psi(f_n,g_n, \lambda_n, t) \, dt + \int_T^\infty e^{-t} dt \\
&\leq \epsilon \int_0^T e^{-t} \, dt + \lim_{n \to \infty} m \lbrace t \in [0, T] \mid \psi(f_n,g_n,\lambda_n,t) > \epsilon \rbrace + e^{-T} \\
&\leq  \epsilon + e^{-T}
\end{align*}
now let $T \to \infty$ and $\epsilon \to 0$ to see that $\lim_{n \to \infty} \int_0^\infty e^{-t} \psi(f_n,g_n, \lambda_n, t) \, dt = 0$.

On the other hand, suppose that $\lim_{n \to \infty} d(f_n, g_n) = 0$ and pick $\lambda_n$ such that $\lim_{n \to \infty} \gamma(\lambda_n) = 0$ and 
$\lim_{n \to \infty} \int_0^\infty e^{-t} \psi(f_n,g_n, \lambda_n, t) \, dt = 0$.  By a Markov bound we know that $\psi(f_n,g_n, \lambda_n, t)  \toprob 0$ with respect to
the exponential distribution with rate $1$ on $[0,\infty)$.  Thus we have for every $T > 0$ and $\epsilon > 0$,
\begin{align*}
m \lbrace t \in [0, T] \mid \psi(f_n,g_n,\lambda_n,t) > \epsilon \rbrace 
&= \int_0^T \characteristic{\psi(f_n,g_n,\lambda_n,t) > \epsilon} \, dt \\
&\leq e^{T} \int_0^\infty e^{-t} \characteristic{\psi(f_n,g_n,\lambda_n,t) > \epsilon} \, dt = e^T \probability{\psi(f_n,g_n,\lambda_n,t) > \epsilon} \\
\end{align*}
and let $n \to \infty$.

By the proof of Lemma \ref{SkorohodEquivalenceA} we know that for every $T > 0$ and for $n \in \naturals$ such that $\gamma(\lambda_n) < 1/2$ we have $\sup_{0 \leq t \leq T} \abs{\lambda_n(t) - t} \leq 2T \gamma(\lambda_n)$ and therefore $\lim_{n \to \infty} \sup_{0 \leq t \leq T} \abs{\lambda_n(t) - t} = 0$ for every $T > 0$.
\end{proof}

\begin{prop}$d$ is a metric on $D([0,\infty); S)$.
\end{prop}
\begin{proof}
It is immediate from the definition that $d(f,g) \geq 0$.  Suppose $d(f,g) = 0$.  Pick $\lambda_n$ as per Lemma \ref{SkorohodInfinityConvergenceToZero} applied to constant sequences $f_n \equiv f$ and $g_n \equiv g$.  Thus,
$\lim_{n \to \infty}   \sup_{0 \leq t \leq T} \abs{\lambda_n(t) - t} = 0$ for all $0 < T < \infty$ and $\lim_{n \to \infty} m \lbrace 0 \leq t \leq T \mid \sup_{0 \leq s < \infty} r(f(s \wedge t), g(\lambda_n(s) \wedge t)) > \epsilon \rbrace = 0$ for all $0 < T < \infty$ and $0 < \epsilon < 1$.  So for any fixed $0 \leq u < \infty$ we let $0 < \epsilon < 1$ be arbitrary and
pick $T > u+\epsilon$. 
Therefore $m \lbrace 0 \leq t \leq T \mid \sup_{0 \leq s < \infty} r(f(s \wedge t), g(\lambda_n(s) \wedge t)) > \epsilon \rbrace < T -u-\epsilon$ for sufficiently large $n$ thus 
$(u+\epsilon,T) \cap \lbrace 0 \leq t \leq T \mid \sup_{0 \leq s < \infty} r(f(s \wedge t), g(\lambda_n(s) \wedge t)) \leq \epsilon \rbrace \neq \emptyset$ for sufficiently large $n$  
and therefore for sufficiently large $n$ we may pick $u+\epsilon < w_n < T$ such that
\begin{align*}
r(f(u), g(\lambda_n(u) \wedge w_n)) \leq \sup_{0 \leq s < \infty} r(f(s \wedge w_n), g(\lambda_n(s) \wedge w_n)) \leq \epsilon
\end{align*}
We also know that $\lim_{n \to \infty} \lambda_n(u) = u$ and by passing to a subsequence $N$, we may assume that $\lambda_n(u) \geq u$ or $\lambda_n(u) \leq u$ along $N$. 
In particular, $\lambda_n(u) < w_n$ for large $n$ and thus $\lim_{n \to \infty} r(f(u), g(\lambda_n(u) \wedge w_n)) = r(f(u), g(u))$ or $\lim_{n \to \infty} r(f(u), g(\lambda_n(u) \wedge w_n)) = r(f(u), g(u-))$ where the limits are taken along the subsequence $N$.  Now we argue as in Lemma \ref{SkorohodJ1RhoMetric},  $f(u) = g(u)$ at all continuity points of $g$ but the discontinuity points are a countable set thus $f(u) = g(u)$ everywhere by right continuity.

Symmetry of $d$ follows just as for the $D([0,T])$ case Theorem \ref{SkorohodJ1Metric} by using the facts that $\gamma(\lambda) = \gamma(\lambda^{-1})$ and $\psi(f,g,\lambda,t) = \psi(g,f,\lambda^{-1},t)$ for all $f,g \in D([0,\infty),S)$ and $0 \leq t < \infty$.

TODO: Finish the triangle inequality part...
\end{proof}

 Because the form of the Skorohod metric is a bit opaque, some effort will go into developing different criteria for convergence
in the topology.  A first such result follows; note that this result is very close to saying that implies that $\lim_{n \to \infty} d(f_n,f) = 0$ precisely when the restrictions
to $[0,T]$ converge in $D([0,T];S)$ whenever $T$ is a continuity point of $f$.  TODO: What extra work has to be done to massage the $\lambda_n$ so that the equivalence is proven???

\begin{prop}\label{SkorohodInfiniteJ1EquivalenceA}Let $f,f_n \in D([0,\infty); S)$ then $\lim_{n \to \infty} d(f_n,f) = 0$ if and only if there exists $\lambda_n \in \Lambda_\infty$ such that $\lim_{n \to \infty} \gamma(\lambda_n) = 0$ and 
\begin{align*}
\lim_{n \to \infty} \psi(f_n,f,\lambda_n,t) = 0 \text{ for all continuity points of $f$}
\end{align*}
If $\lim_{n \to \infty} d(f_n,f) = 0$ then $\lim_{n \to \infty} f_n(t) = f(t)$ when $t$ is a continuity point of $f$.
\end{prop}
\begin{proof}
Suppose that we have $\lambda_n$ as in the hypotheses.  Then as the the number of discontinuity points of $f$ is countable, $\lim_{n \to \infty} \psi(f_n,f,\lambda_n,t) = 0$ for almost every $0 \leq t < \infty$.   Since
$0 \leq e^{-t} \psi(f_n,f,\lambda_n,t) \leq 1$ we may apply Dominated Convergence to conclude that $\lim_{n \to \infty} \int_0^\infty e^{-t} \psi(f_n,f,\lambda_n,t)  \, dt = 0$.

On the other hand suppose $\lim_{n \to \infty} d(f_n,f) =0$ and let $0 \leq T < \infty$ be fixed.  
\begin{clm}There exists $\lambda_n$ and $T < T_n < \infty$ such that $\lim_{n \to \infty} T_n = \infty$, $\lim_{n \to \infty} \gamma(\lambda_n) = 0$ and 
\begin{align*}
\lim_{n \to \infty} \sup_{0 \leq t < \infty} r(f_n(t \wedge T_n), f(\lambda_n(t) \wedge T_n)) = 0
\end{align*}
\end{clm}
Pick $\lambda_n$ as per Lemma \ref{SkorohodInfinityConvergenceToZero} applied to sequences $f_n$ and $g_n \equiv f$.   So $\lim_{n \to \infty} \gamma(\lambda_n) = 0$ and for each
$k \in \naturals$ we may find a $N_k$ such that 
\begin{align*}
m \lbrace 0 \leq t \leq T+k \mid \sup_{0 \leq s < \infty} r(f_n(s \wedge t), f(\lambda_n(s) \wedge t)) > 1/k \rbrace < 1/k \text{ for all $n \geq N_k$}
\end{align*}
It follows that $(T+k-1, T+k) \not \subset \lbrace 0 \leq t \leq T+k \mid \sup_{0 \leq s < \infty} r(f_n(s \wedge t), f(\lambda_n(s) \wedge t)) > 1/k \rbrace$ for every $n \geq N_k$.
Now working inductively on $k$, for $0 \leq n < N_1$ pick $T_n$ arbitrarily and for $k \in \naturals$ and for each $N_k \leq n < N_{k+1}$, pick a $T_n$ with $T \leq T+k-1 < T_n < T+k$ such that 
\begin{align*}
\sup_{0 \leq s < \infty} r(f_{n}(s \wedge T_n), f(\lambda_{n}(s) \wedge T_n)) \leq 1/k
\end{align*}
By construction, $\lim_{n\to \infty} T_n = \infty$ and $\lim_{n \to \infty} \sup_{0 \leq s < \infty} r(f_{n}(s \wedge T_n), f(\lambda_{n}(s) \wedge T_n)) =0$.

Let $\lambda_n$ and $T_n$ be chosen as in the claim, and by the triangle inequality
\begin{align*}
\sup_{0 \leq t < \infty} r(f_n(t \wedge T), f(\lambda_n(t) \wedge T)) 
&\leq \sup_{0 \leq t < \infty} r(f_n(t \wedge T), f(\lambda_n(t \wedge T) \wedge T_n)) +\\
&\sup_{0 \leq t < \infty} r(f(\lambda_n(t \wedge T) \wedge T_n), f(\lambda_n(t) \wedge T)) 
\end{align*}
We look at each of the terms on the right hand side in detail.  For the first term, since $T < T_n$,
\begin{align*}
\sup_{0 \leq t < \infty} r(f_n(t \wedge T), f(\lambda_n(t \wedge T) \wedge T_n))
&=\sup_{0 \leq t \leq T} r(f_n(t), f(\lambda_n(t) \wedge T_n)) \\
&=\sup_{0 \leq t \leq T} r(f_n(t \wedge T_n), f(\lambda_n(t) \wedge T_n))  \\
&\leq \sup_{0 \leq t} r(f_n(t \wedge T_n), f(\lambda_n(t) \wedge T_n))
\end{align*}
and for the second term, 
\begin{align*}
&\sup_{0 \leq t < \infty} r(f(\lambda_n(t \wedge T) \wedge T_n), f(\lambda_n(t) \wedge T))  \\
&=\sup_{0 \leq t \leq T} r(f(\lambda_n(t) \wedge T_n), f(\lambda_n(t) \wedge T))  \vee
\sup_{T \leq t < \infty} r(f(\lambda_n(T) \wedge T_n), f(\lambda_n(t) \wedge T))  \\
&=\sup_{T \leq t \leq (\lambda_n(T) \wedge T_n) \vee T} r(f(t), f(T))  \vee
\sup_{\lambda_n(T) \wedge T \leq t \leq T} r(f(\lambda_n(T) \wedge T_n), f(t))  \\
&\leq \sup_{T \leq t \leq \lambda_n(T) \vee T} r(f(t), f(T))  \vee
\sup_{\lambda_n(T) \wedge T \leq t \leq T} r(f(T), f(t))  + 
r(f(\lambda_n(T) \wedge T_n), f(T)))\\
&\leq \sup_{\lambda_n(T) \wedge T\leq t \leq \lambda_n(T) \vee T} r(f(t), f(T)) + 
r(f(\lambda_n(T) \wedge T_n), f(T)))\\
\end{align*}
Thus we have 
\begin{align*}
&\lim_{n \to \infty} \sup_{0 \leq t < \infty} r(f_n(t \wedge T), f(\lambda_n(t) \wedge T)) 
\leq \lim_{n \to \infty} \sup_{0 \leq t < \infty} r(f_n(t \wedge T_n), f(\lambda_n(t) \wedge T_n)) + \\
&\lim_{n \to \infty} \sup_{\lambda_n(T) \wedge T\leq t \leq \lambda_n(T) \vee T} r(f(t), f(T)) + 
\lim_{n \to \infty} r(f(\lambda_n(T) \wedge T_n), f(T))) \\
&=0
\end{align*}
where the first limit on the right hand side is zero by the claim and the second two limits are zero
because $\lim_{n \to \infty} \lambda_n(T) = T$, $T < T_n$ and $f$ is continuous at $T$.
\end{proof}

TODO: The main part of the following proposition is (iii) implies (ii).  I think there is a lot of redundancy between the argument provided and the one showing the modulus of continuity converges to zero.  We should probably introduce the modulus of continuity first and then pick a partition to assist with the proof.  Can we phrase the argument in terms
of open balls as done in the $D([0,T])$ case that I culled from Bass?

\begin{prop}\label{SkorohodInfiniteJ1EquivalenceB}Let $f,f_n \in D([0,\infty); S)$ then the following are equivalent
\begin{itemize}
\item[(i)] $\lim_{n \to \infty} d(f_n,f) = 0$ 
\item[(ii)] there exists $\lambda_n \in \Lambda_\infty$ with $\lim_{n \to \infty} \gamma(\lambda_n) = 0$ and 
\begin{align*}
\lim_{n \to \infty} \sup_{0 \leq t \leq T} r(f_n(t), f(\lambda_n(t)) = 0 \text{ for all $T > 0$}
\end{align*}
\item[(iii)] for every $T > 0$ there exists $\lambda_n \in \Lambda_\infty$ with $\lim_{n \to \infty} \sup_{0 \leq t \leq T} \abs{\lambda_n(t) -t} = 0$ and 
\begin{align*}
\lim_{n \to \infty} \sup_{0 \leq t \leq T} r(f_n(t), f(\lambda_n(t)) = 0
\end{align*}
\end{itemize}
\end{prop}
\begin{proof}
TODO: Finish

(ii) $\implies$ (iii) is immediate since any single sequence $\lambda_n$ that satisfies (ii) also satisfies (iii) for all $T > 0$ by Lemma \ref{SkorohodInfinityConvergenceToZero}.

(iii) $\implies$ (ii)  For each $N \in \naturals$, pick $\lambda^N_n \in \Lambda_\infty$ such that $\lim_{n \to \infty} \sup_{0 \leq t \leq N} \abs{\lambda^N_n(t) - t} = 0$ and $\lim_{n \to \infty} \sup_{0 \leq t \leq N} r(f_n(t), f(\lambda^N_n(t))) = 0$.  Since the relevant conditions are independent of the values of $\lambda^N_n(t)$ for $t > N$ we may also assume that
$\lambda^N_n(t) = \lambda^N_n(N) + t - N$ for $t > N$.  Now define $\tau^N_0 = 0$ and inductively define
\begin{align*}
\tau^N_k = \inf \lbrace t > \tau^N_{k-1} \mid r(f(t), f(\tau^N_{k-1})) > 1/N \rbrace
\end{align*}
if $\tau^N_{k-1} < \infty$ and define $\tau^N_k = \infty$ otherwise.

\begin{clm}If $\tau^N_k < \infty$ then $\tau^N_k < \tau^N_{k+1}$.
\end{clm}
By right continuity there exists $\delta > 0$ such that $r(f(\tau^N_k), f(t)) < 1/N$ for all $\tau^N_k < t < \tau^N_{k} + \delta$ and therefore $\tau^N_{k+1} \geq \tau^N_k + \delta$.

\begin{clm}If $\lim_{k \to \infty} \tau^N_k = \infty$.
\end{clm}
If $\lim_{k \to \infty} \tau^N_k < \infty$ then the previous claim and by the existence of left limits we know $f(\lim_{k \to \infty} \tau^N_k) = \lim_{k \to \infty} f(\tau^N_k) < \infty$.  On the other hand we know that $r(f(\tau^N_k), f(\tau^N_{k+1})) \geq 1/N$ which shows that such a limit cannot exist.

For each $n \in \naturals$ and $k \in \integers_+$ define
\begin{align*}
u^N_{k,n} = (\lambda^N_n)^{-1}(\tau^N_k)
\end{align*}
when $\tau^N_k < \infty$ and $u^N_{k,n} = \infty$ if $\tau^N_k = \infty$.  Define
\begin{align*}
\mu^N_n(t) = \begin{cases}
\tau^N_k + \left( \frac{\tau^N_{k+1} - \tau^N_k}{u^N_{k+1,n} - u^N_{k,n}} \right) (t - u^N_{k,n}) & \text{for $t \in [u^N_{k,n}, u^N_{k+1,n}) \cap [0,N]$} \\
\mu^N_n(N) + t - N & \text{for $t > N$}
\end{cases}
\end{align*}
where we use the convention that $\frac{\infty}{\infty} = 1$.  Note that $\mu^N_n(u^N_{k,n}) = \tau^N_k= \lambda^N_n(u^N_{k,n})$.  Note also that for $u^N_{k,n} \leq t < u^N_{k+1,n}$ we have $\tau^N_k \leq \lambda^N_t(t) < \tau^N_{k+1}$ which implies $r(f(\tau^N_k), f(\lambda^N_n(t))) < 1/N$ and similarly $r(f(\tau^N_k), f(\mu^N_n(t))) < 1/N$.  Thus 
$\sup_{0 \leq t \leq T} r(f(\lambda^N_n(t)),  f(\mu^N_n(t))) < 2/N$.

Since we have defined $\mu^N_n$ to be piecewise linear it follows that
\begin{align*}
\gamma(\mu^N_n) &= \max_{0 \leq k \leq ?} \abs{\log \frac{\tau^N_{k+1} - \tau^N_k}{u^N_{k+1,n} - u^N_{k,n}} } < \infty
\end{align*}
(TODO: Note that $\mu^N_n$ only has a finite number of different slopes; but that exact number doesn't seem to have a simple formula)
and moreover
\begin{align*}
\sup_{0 \leq t \leq T} r(f_n(t), f(\mu^N_n(t))) &\leq \sup_{0 \leq t \leq T} r(f_n(t), f(\lambda^N_n(t))) + \sup_{0 \leq t \leq T} r(f_n(\lambda^N_n(t)), f(\mu^N_n(t))) \\
&\leq \sup_{0 \leq t \leq T} r(f_n(t), f(\lambda^N_n(t))) + 2/N
\end{align*}

\begin{clm}$\lim_{n \to \infty} \gamma(\mu^N_n) = 0$.
\end{clm}
Since $\lim_{n \to \infty} \sup_{0 \leq t \leq T} \abs{\lambda_n(t) - t} = 0$ it follows that $\lim_{n \to \infty} u^N_{k,n} = \lim_{n \to \infty}  (\lambda^N_n)^{-1}(\tau^N_k) = \tau^N_k$ hence
\begin{align*}
\lim_{n \to \infty} \gamma(\mu^N_n) &= \max_{0 \leq k \leq ?} \lim_{n \to \infty} \abs{\log \frac{\tau^N_{k+1} - \tau^N_k}{u^N_{k+1,n} - u^N_{k,n}} } = 0
\end{align*}

Let $n_0 =0$ and for each $n \in \naturals$ pick $n_N > n_{N-1}$ such that 
\begin{itemize}
\item[(i)] $\gamma(\mu^N_n) < 1/N$ for all $n \geq n_N$
\item[(ii)] $\sup_{0 \leq t \leq N} r(f_n(t), f(\lambda^N_n(t))) < 1/N$ for all $n \geq n_N$
\end{itemize}

For $1 \leq n < n_1$ define $\hat{\lambda}_n \in \Lambda_\infty$ satisfy $\gamma(\hat{\lambda}_n) < \infty$ but otherwise be arbitrary and for $n_N \leq n < n_{N+1}$ let $\hat{\lambda}_n = \mu^N_n$.  Then for any $T>0$, $N > \ceil{T}$ and $n_N \leq n < n_{N+1}$ we have 
\begin{align*}
\sup_{0\leq t \leq T} r(f_n(t), f(\hat{\lambda}_n(t))) &\leq \sup_{0\leq t \leq N} r(f_n(t), f(\hat{\lambda}_n(t))) \\
&=\sup_{0\leq t \leq N} r(f_n(t), f(\mu^N_n(t)))  \leq \sup_{0\leq t \leq N} r(f_n(t), f(\lambda^N_n(t))) + 2/N < 3/N
\end{align*}
which shows that $\lim_{n \to \infty} \sup_{0\leq t \leq T} r(f_n(t), f(\hat{\lambda}_n(t)))$.
\end{proof}

In order to understand compactness in $D([0,\infty);S)$ we need to have a notion of equicontinuity.  The basic idea is that we
express equicontinuity relative to bounded intervals $[0,T]$ and we must account for the existence jumps on such an interval
and particularly at $T$ itself.
\begin{defn}Let $(S,r)$ be a metric space and let $f \in D([0,\infty); S)$ be given then for each $\delta > 0$ and $T > 0$ we define the \emph{modulus of continuity}
\begin{align*}
w^\prime(f,\delta, T) &= \inf_{\substack{0=t_0 < t_1 < \dotsb < t_{n-1} <  T \leq t_n \\
  \min_{1 \leq i \leq n} (t_i - t_{i-1}) > \delta \\ n \in \naturals}}
\max_{1 \leq i \leq n} \sup_{t_{i-1} \leq s < t < t_i} r(f(s), f(t))
\end{align*}
\end{defn}

TODO: Unify the proofs here with the case of $D([0,T];S)$.

Note that we allow the right hand endpoint of the partition $\lbrace t_i \rbrace$ to extend beyond $T$ in the definition above.  TODO: Presumably this makes dealing with jumps
at $T$ easier; on the other hand we don't do this in the case of $D([0,T];S)$ ; is it necessary to define it this way?
\begin{lem}\label{SkorohodInfiniteJ1ModulusOfContinuity}
Let $f, g\in D([0,\infty); S)$ then 
\begin{itemize}
\item[(i)] $w^\prime(f,\delta, T)$ is a non-decreasing function of $\delta$ and $T$
\item[(ii)] For every $\delta > 0$ and $T > 0$
\begin{align*}
w^\prime(f,\delta, T) &\leq w^\prime(g,\delta, T) + 2 \sup_{0 \leq t \leq T+ \delta} r(f(t), g(t))
\end{align*}
\item[(iii)]For each fixed $T > 0$, $w^\prime(f,\delta, T)$ is a right continuous function of $\delta$ and
\begin{align*}
\lim_{\delta \to 0^+} w^\prime(f,\delta, T) &= 0
\end{align*}
\item[(iv)]If $f_n \in D([0,\infty); S)$ and $\lim_{n \to \infty} d(f_n,g) = 0$ then for every $\delta > 0$, $T > 0$ and $\epsilon > 0$ 
\begin{align*}
\limsup_{n \to \infty}  w^\prime(f_n,\delta, T)  &\leq  w^\prime(f,\delta, T+\epsilon) 
\end{align*}
\item[(v)]For fixed $\delta > 0$ and $T>0$, $ w^\prime(f,\delta, T)$ is a Borel measurable function of $f$.
\end{itemize}
\end{lem}
\begin{proof}
(i) is immediate from the fact that the set of partitions over which the infimum is calculated is non-decreasing with respect to set inclusion as either $\delta$ decreases or $T$ increases.  Suppose $0 < \delta < \delta^\prime$, then if we have a partition $t_0 < \dotsb < t_{n-1} < T \leq t_n$ and $t_i - t_{i-1} > \delta^\prime$ for all $i=1, \dotsc, n$ then it follows that $t_i - t_{i-1} > \delta$ for all $i=1, \dotsc, n$ as well.  Thus 
\begin{align*}
&\lbrace (t_0, \dotsc, t_n) \mid t_0 < \dotsb < t_{n-1} < T \leq t_n \text{ and } t_i - t_{i-1} > \delta^\prime \rbrace \subset  \\
&\lbrace (t_0, \dotsc, t_n) \mid t_0 < \dotsb < t_{n-1} < T \leq t_n \text{ and } t_i - t_{i-1} > \delta \rbrace 
\end{align*}
and taking unions over $n \in \naturals$ and infimums it follows that $w(f,\delta, T) \leq w(f,\delta^\prime, T)$.  Suppose that $0 < T < T^\prime$, given a partition $t_0 < \dotsb < t_{n-1} < T \leq t_n$ with $t_i - t_{i-1} > \delta$ for all $i=1, \dotsc, n$ we may extend it to a partition $t^\prime_0 < \dotsb < t^\prime_{n^\prime-1} < T^\prime \leq t^\prime_{n^\prime}$ with $t^\prime_i - t^\prime_{i-1} > \delta$ for all $i=1, \dotsc, n^\prime$ and $t^\prime_i = t_i$ for all $0 \leq i \leq n$; moreover we have $\max_{1 \leq i \leq n} \sup_{t_{i-1} \leq s < t < t_i} r(f(s),f(t)) \leq \max_{1 \leq i \leq n^\prime} \sup_{t^\prime_{i-1} \leq s < t < t^\prime_i} r(f(s),f(t))$ and therefore
\begin{align*}
w^\prime(f,\delta, T) &= \inf_{\substack{0=t_0 < t_1 < \dotsb < t_{n-1} <  T \leq t_n \\
  \min_{1 \leq i \leq n} (t_i - t_{i-1}) > \delta \\ n \in \naturals}}
\max_{1 \leq i \leq n} \sup_{t_{i-1} \leq s < t < t_i} r(f(s), f(t)) \\
&\leq \inf_{\substack{0=t_0 < t_1 < \dotsb < t_{n-1} <  T \leq t_n \\
  \min_{1 \leq i \leq n} (t_i - t_{i-1}) > \delta \\ n \in \naturals}}
\max_{1 \leq i \leq n^\prime} \sup_{t^\prime_{i-1} \leq s < t < t^\prime_i} r(f(s), f(t)) \\
&\leq \inf_{\substack{0=t_0 < t_1 < \dotsb < t_{n-1} <  T^\prime \leq t_n \\
  \min_{1 \leq i \leq n} (t_i - t_{i-1}) > \delta \\ n \in \naturals}}
\max_{1 \leq i \leq n} \sup_{t_{i-1} \leq s < t < t_i} r(f(s), f(t)) \\
&= w(f, \delta, T^\prime)\\
\end{align*}

(ii) follows from the fact that for fixed $\delta > 0$, $T>0$ and any valid partition $t_1 < \dotsb < t_{n-1} < T \leq t_n$ we have $0 \leq t_i \leq T+\delta$ for all $1 \leq i \leq n$ and therefore by the triangle inequality
\begin{align*}
\sup_{t_{i-1} \leq s < t < t_i} r(f(s), f(t)) &\leq \sup_{t_{i-1} \leq s < t < t_i} r(f(s), g(s)) +\sup_{t_{i-1} \leq s < t < t_i} r(g(s), g(t)) + \sup_{t_{i-1} \leq s < t < t_i} r(g(t), f(t)) \\
&\leq \sup_{t_{i-1} \leq s < t < t_i} r(g(s), g(t))  + 2 \sup_{0 \leq t \leq T+\delta} r(g(t), f(t))
\end{align*}
Now take the maximum over $1 \leq i \leq n$ and the infimum over all partitions.

To see the right continuity in (iii), let $T>0$, $\delta>0$ and $\epsilon > 0$ be given and pick a partition $t_0 < \dotsb < t_{n-1} < T \leq t_n$ such that $\min_{1 \leq i \leq n} (t_i - t_{i-1}) > \delta$ and $\max_{1 \leq i \leq n} \sup_{t_{i-1} \leq s < t < t_i} r(f(s), f(t)) \leq w^\prime(f,\delta, T) + \epsilon$.  Using the fact that $w^\prime(f,\delta, T)$ is a nondecreasing function of $\delta$, the existence of this partition shows that $w^\prime(f,\delta^\prime, T) < w^\prime(f,\delta, T) +\epsilon$ for all $\delta^\prime -\delta < \frac{1}{2}\min_{1 \leq i \leq n} (t_i - t_{i-1})$.  The fact that $\lim_{\delta \to 0^+} w^\prime(f,\delta, T) = 0$ follows by the same argument as in Lemma \ref{SkorohodJ1ModulusOfContinuity}.

To see (iv) we know that there exist $\lambda_n$ such that $\lim_{n \to \infty} \sup_{0 \leq t \leq T+\delta}\abs{\lambda_n(t) - t} = 0$ and $\lim_{n \to \infty} \sup_{0 \leq t \leq T+\delta} r(f_n(t), f(\lambda_n(t))) = 0$.  Define $\delta_n = \sup_{0 \leq t \leq T}\abs{\lambda_n(t+\delta) - \lambda_n(t)}$ and note that $\delta_n \to 0$.  If we let $t_0 < \dotsb t_{m-1} < \lambda(T) \leq t_m$ be a partition with $t_i - t_{i-1} > \delta_n$ for all $1 \leq i \leq m$ then it follows that $\lambda_n^{-1}(t_0) < \dotsb \lambda_n^{-1}(t_{m-1}) < T \leq \lambda_n^{-1}(t_m)$ and
\begin{align*}
\lambda^{-1}_n(t_i) - \lambda^{-1}_n(t_{i-1}) > \delta
\end{align*}
for all $1 \leq i \leq m$.  Since we also have $\sup_{\lambda^{-1}_n(t_{i-1})  \leq s < t < \lambda^{-1}_n(t_i) } r(f(s), f(t)) = \sup_{t_{i-1}  \leq s < t < t_i} r(f(\lambda_n(s)), f(\lambda_n(t)))$ it follows that $w^\prime(f \circ \lambda_n,\delta, T) \leq w^\prime(f,\delta_n, \lambda_n(T))$.  Using this fact, (i), (ii)  and $\lim_{n \to \infty} \lambda_n(T) = T$ we get for any $\epsilon > 0$,
\begin{align*}
\limsup_{n \to \infty}  w^\prime(f_n,\delta, T) 
&\leq \limsup_{n \to \infty} \left[ w^\prime(f \circ \lambda_n,\delta, T) + 2 \sup_{0 \leq t \leq T+\delta} r(f_n(t), f(\lambda_n(t)) \right] \\
&=\limsup_{n \to \infty}  w^\prime(f \circ \lambda_n,\delta, T) \\
&\leq \limsup_{n \to \infty}  w^\prime(f , \delta_n, \lambda_n(T)) \\
&\leq \limsup_{n \to \infty}  w^\prime(f , \delta \vee \delta_n, T+\epsilon) \\
&=w^\prime(f,\delta, T+\epsilon) 
\end{align*}

To see (v) define $w^\prime(f,\delta, T+) = \lim_{\epsilon \to 0^+} w^\prime(f,\delta, T+\epsilon)$, then by the fact that $w^\prime(f_n,\delta, T)$ is non-decreasing in $T$ and (iv) we for every $\epsilon > 0$,
\begin{align*}
\limsup_{n \to \infty}  w^\prime(f_n,\delta, T+) &\leq \limsup_{n \to \infty}  w^\prime(f_n,\delta, T+\epsilon) \leq w^\prime(f,\delta, T+2\epsilon) 
\end{align*}
Now let $\epsilon \to 0^+$ to conclude that  $w^\prime(f,\delta, T+)$ is an upper semicontinuous function of $f$, hence Borel measurable.  Now we claim that 
$w^\prime(f,\delta, T) = \lim_{n \to \infty} w^\prime(f,\delta, (T-1/n)+)$ which shows that $w^\prime(f,\delta, T)$ is Borel measurable.


I believe that this last statement is supported by the following (which we should add to the statement of the Lemma):
\begin{clm}$w^\prime(f,\delta, T)$ is a left continuous function of $T$
\end{clm}
Let $\epsilon > 0$ be given and using the existence of the left limit $\lim_{t \to T^-} f(t)$, pick $\rho > 0$ such that $\sup_{T - \rho \leq s < t < T} r(f(s), f(t)) < \epsilon/2$.  Since 
$w^\prime(f,\delta, T)$ is non-decreasing in $T$, we know that $\lim_{t \to T^-} w^\prime(f,\delta, t)$ exists and $w^\prime(f,\delta, s) \leq \lim_{t \to T^-} w^\prime(f,\delta, t)$ for all $0 \leq s < T$.  We can pick a partition $t_0 < \dotsb < t_{m -1} < T - \rho \leq t_{m}$ with $\min_{1 \leq i \leq m} (t_i - t_{i-1}) > \delta$ and 
\begin{align*}
\max_{1 \leq i \leq m} \sup_{t_{i-1} \leq s < t < t_i} r(f(s),f(t)) &< w^\prime(f,\delta, T-\rho) + \epsilon/2 \leq \lim_{t \to T^-} w^\prime(f,\delta, t) + \epsilon/2
\end{align*}
If we have $t_{m} \geq T$ then we can conclude 
\begin{align*}
w^\prime(f,\delta,T) &\leq \max_{1 \leq i \leq m} \sup_{t_{i-1} \leq s < t < t_i} r(f(s),f(t)) \leq \lim_{t \to T^-} w^\prime(f,\delta, t) + \epsilon/2
\end{align*}  
It that is not the case then
we modify these chosen partition to make one with which we can bound $w^\prime(f,\delta,T)$.  Specifically supposing $t_{m} < T$, we define $\tilde{t}_j = t_j$ for $0 \leq j < m$ and define $\tilde{t}_{m} = T$.  It is clear from the properties of the partition $\lbrace t_i \rbrace$ and the fact we have only moved the rightmost endpoint of the partition further to the right that $\tilde{t}_0 < \dotsb < \tilde{t}_{m -1} < T = \tilde{t}_{m}$, that
$\min_{1 \leq i \leq m} (\tilde{t}_i - \tilde{t}_{i-1}) \geq \min_{1 \leq i \leq m} (t_i - t_{i-1}) > \delta$ and for $1 \leq i < m$
\begin{align*}
\sup_{\tilde{t}_{i-1} \leq s < t < \tilde{t}_i} r(f(s),f(t))  &= \sup_{t_{i-1} \leq s < t < t_i} r(f(s),f(t))  \leq \sup_{t_{i-1} \leq s < t < t_i} r(f(s),f(t)) + \epsilon/2
\end{align*}
Moreover we may pick a point $T^*$ such that  
\begin{align*}
\tilde{t}_{m-1} &= t_{m-1} < T - \rho < T^*  < t_{m} < T = \tilde{t}_m
\end{align*} 
and use this to break down the supremum into cases:
\begin{align*}
&\sup_{\tilde{t}_{m-1} \leq s < t < \tilde{t}_{m}} r(f(s),f(t)) \\
&=\sup_{\tilde{t}_{m-1} \leq s < t \leq T^*} r(f(s),f(t)) \vee \sup_{T^* \leq s < t < \tilde{t}_{m}} r(f(s),f(t)) \vee \sup_{\tilde{t}_{m-1} \leq s < T^* < t < \tilde{t}_{m}} r(f(s),f(t)) \\
&\leq \sup_{t_{m-1} \leq s < t <  t_{m}} r(f(s),f(t)) \vee \sup_{T - \rho < s < t < T} r(f(s),f(t)) \vee \\
&\sup_{t_{m-1} \leq s < T^* < t < T} \left[ r(f(s),f(T^*)) + r(f(T^*), f(t)) \right] \\
&\leq \sup_{t_{m-1} \leq s < t < t_{m}} r(f(s),f(t)) \vee \sup_{T - \rho < s < t < T} r(f(s),f(t)) \vee \\
&\left(\sup_{t_{m-1} \leq s < t < t_{m}} r(f(s),f(t)) + \sup_{T - \rho < s < t \leq T} r(f(s),f(t)) \right) \\
&\leq \sup_{t_{m-1} \leq s < t < t_{m}} r(f(s),f(t)) + \epsilon/2
\end{align*}
Thus we have 
\begin{align*}
w^\prime(f,\delta, T) &\leq \max_{1 \leq i \leq m} \sup_{\tilde{t}_{i-1} \leq s < t < \tilde{t}_i} r(f(s),f(t))  \\
&\leq \max_{1 \leq i \leq m} \sup_{t_{i-1} \leq s < t < t_i} r(f(s),f(t)) + \epsilon/2 \\
&\leq \lim_{t \to T^-} w^\prime(f,\delta, t) + \epsilon
\end{align*}
Now since $\epsilon > 0$ was arbitrary we let $\epsilon \to 0$ and conclude $w^\prime(f,\delta, T)  \leq \lim_{t \to T^-} w^\prime(f,\delta, t) $.
On the other hand it is clear from the fact that $w^\prime(f,\delta,T)$ is non-decreasing in $T$ that $\lim_{t \to T^-} w^\prime(f,\delta, t)  \leq w^\prime(f,\delta, T)$ and therefore $\lim_{t \to T^-} w^\prime(f,\delta, t)  = w^\prime(f,\delta, T)$.
\end{proof}


\begin{lem}\label{SkorohodInfiniteBorelGeneratedByProjections}For every $t \in [0,\infty)$ let $\pi_t : D([0,\infty); S) \to
  S$ be the evaluation map $\pi_t(f) = f(t)$.  
\begin{itemize}
\item[(i)] For arbitrary $S$ every $\pi_t$ is Borel measurable.  
\item[(ii)] $\sigma(\pi_t ; 0 \leq t < \infty) = \sigma(\pi_t ; t \in D)$ for any dense subset $D \subset [0,\infty)$.
\item[(iii)]If $S$ is separable then the Borel $\sigma$-algebra on $D([0,\infty); S)$ is
  equal to $\sigma(\lbrace \pi_t \mid t \in [0,\infty) \rbrace)$ and
  therefore $\mathcal{B}(D([0,\infty); S)) = D([0,\infty);
  S) \cap \mathcal{B}(S)^{\otimes [0,\infty)}$.
\end{itemize}
\end{lem}
\begin{proof}
Let $\psi : S \to \reals$ be a bounded continuous function and suppose $\epsilon > 0$ and $0 \leq t < \infty$ are fixed.  Define
\begin{align*}
\psi^\epsilon_t(f) &= \frac{1}{\epsilon} \int_t^{t + \epsilon} \psi(\pi_s(f)) \, ds = \frac{1}{\epsilon} \int_t^{t + \epsilon} \psi(f(s)) \, ds
\end{align*}

\begin{clm}$\psi^\epsilon_t : D([0,\infty); S) \to \reals$ is a continuous function 
\end{clm}
Suppose that $f_n \to f$ in the $J_1$ topology.  By Proposition \ref{SkorohodInfiniteJ1EquivalenceA} we know that $f_n(t) \to f(t)$ at every continuity point of $f$.  Since the set of discontinuity points of $f$ is countable it has measure zero and it follows that $f_n \to f$ almost everywhere.  Thus by continuity of $\psi$ we have $\psi \circ f_n \to \psi \circ f$ almost everywhere and thus by Dominated Convergence we conclude that $\psi^\epsilon_t (f_n) \to \psi^\epsilon_t (f)$.  

By the Fundamental Theorem of Calculus we have $\lim_{\epsilon \to 0^+} \psi^\epsilon_t(f) = \psi(\pi_t(f))$ which shows that $\psi \circ \pi_t$ is Borel measurable for every bounded
continuous $\psi : S \to \reals$.  Since every bounded Borel measurable function is a limit of bounded continuous functions (TODO: ) if follows that $\psi \circ \pi_t$ is 
Borel measurable for every bounded measurable $\psi : S \to \reals$.  In particular for $A \in \mathcal{B}(S)$ then $\pi_t^{-1}(A) = (\characteristic{A} \circ \pi_t)^{-1}(\lbrace 1 \rbrace)$ is Borel measurable.

To see (ii) let $0 \leq t < \infty$ be given and pick a sequence $t_1, t_2, \dotsc$ with $t_n \in D \cap [t,\infty)$ such that $\lim_{n \to \infty} t_n = t$.  By right continuity of the elements
of $D([0,\infty); S)$ we see that $\pi_t = \lim_{n \to \infty} \pi_{t_n}$.

To see (iii) we first set about showing that open balls are in $\sigma(\pi_t ; 0 \leq t < \infty)$.  To show that is equivalent to showing that for fixed $g \in D([0,\infty); S)$ the function
$d(\cdot, g) : D([0,\infty);S) \to reals$ is $\sigma(\pi_t ; 0 \leq t < \infty)$-measurable, so we set to work on that.  To create approximations of $d(\cdot, g)$, we assume that
a partition $0=t_0 < t_1 < \dotsb < t_n < t_{n+1}=\infty$ is given and define $\eta : S^{n+1} \to D([0,\infty);S)$ by
\begin{align*}
\eta(x_0, \dotsc, x_n)(t) = \sum_{j=0}^n x_j \characteristic{[t_j,t_{j+1})}(t)
\end{align*}
By considering the $\lambda(t) = t$ we see that $d(\eta(x_0, \dotsc, x_n), \eta(y_0, \dotsc, y_n)) \leq \max_{0 \leq j \leq n} r(x_j,y_j)$ which shows that $\eta$ is continuous.  Since
$S$ is separable we know that $\mathcal{B}(S^{n+1}) = \mathcal{B}(S)^{\otimes n+1}$ and therefore $\eta$ is Borel measurable as well.  Now let $g \in D([0,\infty); S)$ be fixed and consider $d(\eta \circ (\pi_{t_0}, \dotsc, \pi_{t_n}), g) : D([0, \infty) ; S) \to \reals$ which is therefore $\sigma(\pi_t ; 0 \leq t < \infty)$-measurable.  Now apply this construction to
the sequence of partitions $t^m_j = j/m$ for $j=0, \dotsc, m^2$ and $m \in \naturals$, letting $\eta_m$ be the $m^{th}$ constructed embedding.
\begin{clm}$\lim_{m \to \infty} d(\eta_m \circ (\pi_{t^m_0}, \dotsc, \pi_{t^m_{m^2}}), g) = d(\cdot, g)$.
\end{clm}
Suppose $\epsilon > 0$ and $T > 0$ be given.  By Lemma \ref{SkorohodInfiniteJ1ModulusOfContinuity} we may find a partition $0=t_0 < t_1 < \dotsb < t_{m-1} < T \leq t_m$ such that
$\max_{1 \leq i \leq m} \sup_{t_{i-1} \leq s < t < t_i} r(f(s), f(t)) < \epsilon$.  Pick $N \in \naturals$ such that $1/N < \min_{1 \leq i \leq m} (t_i - t_{i-1}) \wedge 1/t_m$ and for every $n \geq N$ and $1 \leq i \leq m$ we let $j(i,n)$ be the integer such that $\frac{j(i,n) -1}{n} < t_i \leq \frac{j(i,n)}{n}$.  Define $\lambda_n(j(i,n)/n) = t_i$ for $1 \leq i \leq m$, linearly interpolate in between and $\lambda_n(t) = t_m + (t - j(m,n)/n)$ for $t >  j(m,n)/n$.  For any $0 \leq t \leq T$, we pick $k \in \naturals$ such that $k-1/n \leq t < k/n$ and we have by definition $f_n(t) = f((k-1)/n)$.  Furthermore, since $0 \leq t \leq T$,  we may pick $1 \leq i \leq m$ such that $j(i-1,n) \leq k-1 < j(i,n)$ and it follows that $j(i-1,n)/n \leq t < j(i,n)/n$ hence $t_{i-1} \leq \lambda_n(t)) < t_i$ and also 
\begin{align*}
t_{i-1} \leq \frac{j(i-1,n)}{n} \leq \frac{k-1}{n} \leq  \frac{j(i,n)-1}{n} < t_i
\end{align*} 
Therefore we conclude $r(f_n(t), f(\lambda_n(t))) \leq \sup_{t_{i-1} \leq s < t < t_i} r(f(s), f(t)) < \epsilon$ for all $n \geq N$ and we have shown that $\lim_{n \to \infty} \sup_{0 \leq t \leq T} r(f_n(t), f(\lambda_n(t)) = 0$.
Since $\lambda_n$ is defined to be piecewise linear it follows that $\sup_{0 \leq t \leq T} \abs{\lambda_n(t) - t} = \max_{0 \leq m} \abs{\lambda_n(j(i,n)/n) - j(i,n)/n}= \max_{0 \leq m} \abs{t_i - j(i,n)/n}$.  It is clear by the definition of $j(i,n)$ that $\lim_{n\to \infty} j(i,n)/n = t_i$ and thus it follows that $\sup_{0 \leq t \leq T} \abs{\lambda_n(t) - t} \to 0$.  Proposition \ref{SkorohodInfiniteJ1EquivalenceB} shows this is sufficient to show convergence (recall that we may pick $\lambda_n$ depending on $T$ according to that Lemma).

By the claim, we see that for every $g \in D([0,\infty); S)$ the function $d(\cdot, g)$ is $\sigma(\pi_t ; 0 \leq t < \infty)$-measurable from which it follows that every open
ball $B(g, r) \in \sigma(\pi_t ; 0 \leq t < \infty)$.  Since $S$ is separable, every open set is a countable union of open balls and therefore every open set belongs to $\sigma(\pi_t ; 0 \leq t < \infty)$ and we are done.
\end{proof}

\begin{thm}\label{SkorohodInfiniteArzelaAscoliTheoremJ1Topology}Let $(S,r)$ be a complete metric space.  A set $A \subset D([0,\infty); S)$ is relatively compact in the $J_1$ topology if and only if 
\begin{itemize}
\item[(i)]for each rational number
$t \in [0,\infty) \cap \rationals$ there exists a compact set $K_t \subset S$ such that $\cup_{f \in A} f(t) \subset K_t$
\item[(ii)] For all $T>0$, $\lim_{\delta \to 0} \sup_{f \in A} w^\prime(f, \delta, T) = 0$
\end{itemize}
\end{thm}
\begin{proof}
TODO: Not significantly different than the case of $D([0,T];S)$.
\end{proof}

\begin{lem}\label{SkorohodInfiniteTightnessOfJ1Topology}A set of Borel probability measures $\mu_\alpha$ on $D([0,\infty); S)$ is tight if and only if 
\begin{itemize}
\item[(i)]For every $\epsilon >0$ and $t \in [0,\infty) \cap \rationals$ there exists a compact set $K_{\epsilon, t} \subset S$ such that $\sup_\alpha \mu_\alpha(f(t) \in K_{\epsilon, t}) > 1 - \epsilon$.
\item[(ii)] For every $\lambda > 0$ and $T > 0$ 
\begin{align*}
\lim_{\delta \to 0} \sup_\alpha \mu_\alpha(w^\prime(f, \delta, T) \geq \lambda) &= 0
\end{align*}
\end{itemize}
\end{lem}
\begin{proof}
Let $\mu_\alpha$ be tight.  Let $\epsilon > 0$ be given and pick
$K \subset D^\infty([0,\infty); S)$ compact with $\mu_\alpha(K) >
1-\epsilon/2$ for all $\alpha$.  Then by Theorem \ref{SkorohodInfiniteArzelaAscoliTheoremJ1Topology} we know that 
for every $t \in [0,\infty) \cap \rationals$ there exists a compact set $K_{\epsilon,t} \subset S$ such that
$f(t) \in K_{\epsilon,t}$ for every $f \in K$ and therefore by a union bound
\begin{align*}
\sup_\alpha \mu_\alpha(f(t) \in K_{\epsilon,t}) \geq \sup_\alpha \mu_\alpha(K) \geq 1 - \epsilon/2 > 1 - \epsilon
\end{align*}

Similarly applying Theorem \ref{SkorohodInfiniteArzelaAscoliTheoremJ1Topology} we know that for
every $T > 0$ and $\lambda>0$
there exists $\delta>0$ such that $\sup_{f \in K} w^\prime(f, \delta, T) <
\lambda$.  Therefore $\lbrace f \mid w^\prime(f, \delta, T) \geq \lambda \rbrace
\subset K^c$ and by a union bound applied for every $\alpha$ we have $\sup_\alpha \mu_\alpha(w^\prime(f, \delta, T) \geq  \lambda) 
\leq \mu_\alpha(K^c) < \epsilon$.  Since $w^\prime(f, \delta, T)$ is a non-decreasing function of $\delta$ this it follows that for all $0 < \rho \leq \delta$,
\begin{align*}
\sup_\alpha \mu_\alpha(w^\prime(f, \rho, T) \geq  \lambda)  \leq \sup_\alpha \mu_\alpha(w^\prime(f, \delta, T) \geq  \lambda)  < \epsilon
\end{align*}
and we have shown (ii).

Now assume that (i) and (ii) hold and suppose that $\epsilon > 0$ is
given.  Let $q_1, q_2, \dotsc$ be an enumeration of $t \in [0,\infty) \cap \rationals$.  By (i) for every $q_M$ there exists compact $K_{\epsilon, M} \subset S$ such that $\sup_{\alpha}
\mu_{\alpha} (f(q_M) \notin K_{\epsilon, M}) < \epsilon/2^{M+1}$.  By (ii)
for every $N,k \in \naturals$, there exists a $\delta_{N,k}$
such that $\sup_{\alpha} \mu_\alpha(w^\prime(f, \delta_{N,k}, N)\geq 1/k)< \epsilon/2^{N+k+1}$.  If we define 
\begin{align*}
A_N &= \lbrace f \mid 
w^\prime(f,\delta_{N,k}, N) < 1/k \text{ for all } k \geq 1\rbrace
\end{align*}
so that $A^c_N \subset \cup_{k=1}^\infty \lbrace f \mid w^\prime(f, \delta_{N,k}, N) \geq 1/k \rbrace$ 
then by a union bound
\begin{align*}
\sup_{\alpha} \mu_\alpha(A_N) &= \sup_{\alpha} (1 - \mu_\alpha(A^c_N))\\
&\geq \sup_{\alpha} \left(1 - \sum_{k=1}^\infty \mu_\alpha(w^\prime(f, \delta_{N,k}, N) \geq  1/k) \right ) \\
&\geq 1 - \epsilon/2^{T+1}
\end{align*}
If we define $K = \cap_{M=1}^\infty \lbrace f(q_M) \in K_{\epsilon, M} \rbrace \cap
\cap_{N=1}^\infty A_N$ then another union bound shows 
\begin{align*}
\sup_{\alpha} \mu_\alpha(K^c) &=\sup_{\alpha} \mu_\alpha(\cup_{M=1}^\infty \lbrace f(q_M) \notin K_{\epsilon, M} \rbrace 
\cup
\cup_{N=1}^\infty A^c_N) \\
&\leq \sup_{\alpha} \left[ \sum_{M=1}^\infty \mu_\alpha( f(q_M) \notin K_{\epsilon, M})
+
\sum_{N=1}^\infty \mu_\alpha(A^c_N) \right] \\
&\leq \sum_{M=1}^\infty \epsilon/2^{M+1} + \sum_{N=1}^\infty \epsilon/2^{T+1} = \epsilon
\end{align*} 
and by construction the set $K$ satisfies the conditions of Theorem \ref{SkorohodInfiniteArzelaAscoliTheoremJ1Topology} so is proven compact.  
\end{proof}
 
TODO: Understand the Ethier and Kurtz construction and its relationship with Lindvall's.  One proposal is that the only difference is the manner in which we are embedding $D([0,\infty))$ in $D_0^\infty$ (accounting for the lack of linearity).  Here we define for each $n \in \naturals$ the map $c_n : D([0,\infty)) \to D([0,\infty])$ by $c_n(f)(t) = f(t \wedge n)$ and then map $f$ to $\Psi(f) = (c_1(f), c_2(f), \dotsc)$. The key things that we need to have be true are that
\begin{itemize}
\item[(i)] $\Psi$ is an injection
\item[(ii)] $\Psi(D([0,\infty)))$ is closed in $D_0[0,\infty]^\infty$.
\end{itemize}

It is obvious that $\Psi$ is linear and furthermore if $f \neq 0$ the we pick $0 \leq t < \infty$ such that $f(t) \neq 0$ and then $n > t$ and it follows that $c_n(f)(t) = f(t) \neq 0$ thus $\Psi(f) \neq 0$.  

\begin{lem}For each $m \in \naturals$, $c_m$ is continuous on the set 
\begin{align*}
D_0 = \lbrace f \in D([0,\infty);S) \mid \lim_{t \to \infty} f(t) \text{ exists and is finite} \rbrace
\end{align*}  
Furthermore, $\Psi(D)$ is closed.
\end{lem}
\begin{proof}
Suppose $f_n \to f$ in $D_0$.  Then there exist $\lambda_n$ such that $\lim_{n \to \infty} \sup_{0 \leq t < \infty} r(f_n(\lambda_n(t)) , f(t)) = 0$ and $\lim_{n \to \infty} \sup_{0 \leq t < \infty} \abs{\lambda_n(t) - t} = 0$.  We apply the sequence $c_m(f_n)(t) = f_n(t \wedge m)$.  
\begin{align*}
\sup_{0 \leq t < \infty} r(c_m(f_n)(\lambda_n(t)) , c_m(f)(t)) 
&=\sup_{0 \leq t < \infty} r(f_n(\lambda_n(t) \wedge m) , f(t \wedge m)) \\
&=\sup_{0 \leq t < \infty} r(f_n(\lambda_n(t) \wedge m) , f_n(\lambda_n(t \wedge m)))+ r( f_n(\lambda_n(t \wedge m)) , f(t \wedge m)) \\
\end{align*}

Note that 
\begin{align*}
\sup_{0 \leq t < \infty} r( f_n(\lambda_n(t \wedge m)) , f(t \wedge m)) = \sup_{0 \leq t \leq m} r( f_n(\lambda_n(t)) , f(t)) 
\leq \sup_{0 \leq t < \infty} r( f_n(\lambda_n(t)) , f(t)) \to 0
\end{align*}

Oops.  It is not true that $\sup_{0 \leq t < \infty} r(f_n(\lambda_n(t) \wedge m) , f_n(\lambda_n(t \wedge m))) \to 0$ for a counterexample
let $0 < x < \infty$ and define 
\begin{align*}
\lambda_n(t) = \begin{cases}
(1-1/nx) t & 0 \leq t \leq x \\
2t - x - 1/n & x < t < x+1/n \\
t & t \geq x+1/n
\end{cases}
\end{align*}
so that $f_n(\lambda_n(x) \wedge x) = f_n((x - 1/n) \wedge x) = f_n(x-1/n)$ and $f_n(\lambda_n(x \wedge x) = f_n(x)$.  If $f_n$ all have a jump of the same size $\epsilon>0$ at 
$t=x$ it follows that $\sup_{0 \leq t < \infty} r(f_n(\lambda_n(t) \wedge m) , f_n(\lambda_n(t \wedge m))) \geq \epsilon > 0$.

Note that this doesn't imply that $f \to f(\cdot \wedge n)$ isn't a continuous map necessarily we might just need to be more creative in finding the sequence $\lambda_n$.
In fact the example $f = \characteristic{[x, \infty)}$ and $f_n = \characteristic{[x+1/n)}$ provide an example showing that $c_x$ is not continuous.  Can it still be true that $\Psi(D)$ is closed?
\end{proof}

TODO: Lindvall's approach only works with $S$ a linear space since we use multiplication to tamp things down.  Maybe there are some worthwhile exercises to be culled from that.
For each $n \in \naturals$ we define 
\begin{align*}
g_n(t) &= \begin{cases}
1 & 0 \leq t \leq n \\
1 - t - n & n < t < n + 1 \\
0 & t \geq n+1
\end{cases}
\end{align*}
and for every $f \in D([0,\infty)$.

\section{Aldous Criterion}

TODO: How to tie this in with what Ethier and Kurtz claim to be the Aldous criterion.
\begin{thm}Let $X^n$ be a sequence of processes in $D([0,T];S)$ with $(S,r)$ a metric space and suppose for every $\epsilon > 0$ and $t \in [0,T] \cap \rationals$ there exists a compact set $K_{\epsilon,t}$ such that $\probability{X^n_t \in K_{\epsilon,t}} > 1 - \epsilon$, then for every $\lambda > 0$ $X^n$ satisfies $\lim_{\delta \to 0} \sup_n \probability{w(X^n, \delta) \geq \lambda} =0$ if and only if for any sequence of $\mathcal{F}^{X^n}$-optional times $\tau_n$ and any sequence $\delta_n > 0$ with $\lim_{n \to \infty} \delta_n =0$ we have
\begin{align*}
r(X^n_{\tau_n}, X^n_{\tau_n + \delta_n}) \toprob 0
\end{align*}
\end{thm}
\begin{proof}
TODO:
\end{proof}

In practice it turns out that it is difficult to apply the criterion $\lim_{\delta \to 0} \sup_{\alpha} \mu_\alpha( w^\prime(f, \delta, T) \geq \lambda) = 0$ to show tightness of a family of
measures $\mu_\alpha$ on Skorohod space.  In some sense this is not surprising as, being an infimum over a set of partitions, the modulus of continuity is a rather complicated object.  What we need to develop are tools for estimating $w^\prime(f, \delta, T)$ that are strong enough to imply the tightness condition.  One technique for finding upper bounds of $w^\prime(f, \delta, T)$ is clear; one simply needs to find a particular partition $0=t_0 < \dotsb < t_{n-1} < T \leq t_n$ for which we can calculate (or upper bound) each term
$\sup_{t_{i-1} \leq s < t < t_i} r(f(s), f(t))$.  

Our first step is to work pointwise in $D([0,\infty); S)$ and show how create a useful partition for an arbitrary cadlag function $f$.  For the construction we assume that $\epsilon > 0$ and $f \in D([0,\infty); S)$ are both given.  First define inductively $\tau_0=0$ and $n \in \naturals$
\begin{align*}
\tau_{n} &= \begin{cases}
\inf \lbrace t > \tau_{n-1} \mid r(f(t), f(\tau_{n-1})) > \epsilon/2 \rbrace & \text{if $\tau_{n-1} < \infty$} \\
\infty & \text{if $\tau_{n-1} = \infty$}
\end{cases}
\end{align*}
and then define for $n \in \integers_+$
\begin{align*}
\sigma_{n} &= \begin{cases}
\sup \lbrace t \leq \tau_{n} \mid r(f(t), f(\tau_{n})) \vee r(f(t-), f(\tau_{n}))   \geq \epsilon/2 \rbrace & \text{if $\tau_n < \infty$} \\
\infty & \text{if $\tau_n = \infty$}
\end{cases}
\end{align*}

Note that by right continuity of $f$ we have $r(f(\tau_n), f(\tau_{n-1})) \geq \epsilon/2$ whenever $\tau_n < \infty$ (in particular $\tau_{n-1} < \tau_n$ whenever $\tau_{n-1} < \infty$).

\begin{clm}Let $\delta > 0, T>0$ be given. If $w^{\prime}(f,\delta, T) < \epsilon/2$  then $\min \lbrace \tau_{n+1} - \sigma_n \mid \tau_n < T \rbrace > \delta$.  
\end{clm}
The claim is verified by contradiction, so suppose that we have $\tau_n < T$ and $\tau_{n+1} - \sigma_n \leq \delta$ for some $n \in \integers_+$.  If we are given a partition
$0=t_0 < \dotsb t_{m-1} < T \leq t_m$ with $\min_{1 \leq i \leq m} (t_i-t_{i-1}) > \delta$ it follows that there is some $1 \leq i \leq m$ such that $t_{i-1} \leq \tau_n < \tau_i$.  If $\sigma_n \leq t_{i-1} < t_i \leq \tau_{n+1}$ then $\tau_{n+1} - \sigma_n \geq t_i - t_{i-1} > \delta$ which is a contradiction therefore either $t_{i-1} < \sigma_n \leq \tau_n < t_i$ or 
$t_{i-1} \leq \tau_n < \tau_{n+1} < t_i$ or both.  In the first case by definition of $\sigma_n$ we can find a $t_{i-1} < u \leq \sigma_n$ such that $r(f(u), f(\tau_n)) \geq \epsilon/2$ and in the second case we have already observed $r(f(\tau_n), f(\tau_{n-1})) \geq \epsilon/2$; thus $\max_{1 \leq i \leq m} \sup_{t_{i-1} \leq s < t < t_{i}} r(f(s), f(t)) \geq \sup_{t_{i-1} \leq s < t < t_{i}} r(f(s), f(t)) \geq \epsilon/2$.  Now take the infimum over all partitions.

Next we define the partition that will generate our upper bound on the modulus of continuity.
\begin{align*}
s_n &= \frac{\sigma_n + \tau_n}{2}
\end{align*}
and note that
\begin{align*}
\sigma_n &\leq s_n \leq \tau_n \leq \sigma_{n+1} \leq s_{n+1} \leq \tau_{n+1}
\intertext{and}
s_{n+1} - s_n &= 
\frac{\sigma_{n+1} + \tau_{n+1}}{2} - \frac{\sigma_n + \tau_n}{2} \geq 
\frac{\tau_{n} + \tau_{n+1}}{2} - \frac{\sigma_n + \tau_n}{2} = 
\frac{\tau_{n+1} - \sigma_n}{2} 
\end{align*}

\begin{clm}For every $\delta >0, T>0$ if $\min \lbrace \tau_{n+1} - \sigma_n \mid \tau_n < T + \delta/2 \rbrace > \delta$ then 
$\min \lbrace s_{n+1} - s_n \mid s_n < T \rbrace > \delta/2$
\end{clm}
We argue by contradiction, so suppose that $s_n < T$ and $s_{n+1} - s_n \leq \delta/2$.  Then $\tau_n \leq s_{n+1} \leq s_n + \delta/2 < T + \delta/2$ and 
\begin{align*}
\tau_{n+1} - \sigma_n &\leq 2(s_{n+1} - s_n) \leq \delta
\end{align*}

\begin{clm}For every $\delta >0, T>0$ if $\min \lbrace s_{n+1} - s_n  \mid s_n < T \rbrace > \delta/2$ then 
$w^{\prime}(f, \delta/2, T)  \leq \epsilon$
\end{clm}
The claim follows if we can show $\sup_{s_n \leq s < t < s_{n+1}} r(f(s), f(t)) \leq \epsilon$ for then $0=s_0 < \dotsb < s_{n} < T \leq s_{n+1}$ is a partition which shows that
$w^{\prime}(f, \delta/2, T)  \leq \epsilon$ (recall that $s_n \to \infty$ so there are only finitely many $s_n < T$).
The property $\sup_{s_n \leq s < t < s_{n+1}} r(f(s), f(t)) \leq \epsilon$ follows from the triangle inequality if we can show that for any $s_n \leq s < s_{n+1}$ we have 
$r(f(s), f(\tau_n)) < \epsilon/2$.  To see this last fact we consider two cases.
First assume $s_n \leq \tau_n \leq s < s_{n+1} \leq \tau_{n+1}$ then by the definition of
$\tau_{n+1}$ we know that $r(f(s), f(\tau_n)) < \epsilon/2$.
If on the other hand, $s_n \leq s < \tau_n$ then this implies that $\sigma_n < \tau_n$ hence $\sigma_n < s_n$ and therefore either $\sigma_n < s < \tau_n$
By the definition of $\sigma_n$ we know that $r(f(s), f(\tau_n)) < \epsilon/2$.  

TODO: What are the steps and examples that lead one to considering the definitions of $\sigma_n$ and $s_n$ (the case for the definition of $\tau_n$ is clear).

\begin{clm}If $S$ is separable the $\tau_n$, $\sigma_n$ and $s_n$ are Borel measurable functions on $D([0,\infty); S)$ to $[0,\infty]$.
\end{clm}
TODO:

\begin{lem}\label{SkorohodInfiniteModulusOfContinuityEquivalences}Let $(S,r)$ be a separable metric space, let $A$ be an arbitrary index set and let $X^\alpha$ for $\alpha \in A$ be a family of stochastic processes with values in $D([0,\infty); S)$.  Define $\tau^{\alpha, \epsilon}_n, \sigma^{\alpha, \epsilon}_n$ and $s^{\alpha, \epsilon}_n$ as above then the following are equivalent
\begin{itemize}
\item[(i)]$\lim_{\delta \to 0} \inf_{\alpha \in A} \probability{w^{\prime}(X^\alpha, \delta, T) < \epsilon} = 1$ for all $\epsilon > 0$ and $T > 0$
\item[(ii)]$\lim_{\delta \to 0} \inf_{\alpha \in A} \probability{\min \lbrace \tau^{\alpha, \epsilon}_{n+1} - \sigma^{\alpha, \epsilon}_n \mid \tau^{\alpha, \epsilon}_n < T \rbrace \geq \delta} = 1$ for all 
$\epsilon > 0$ and $T > 0$
\item[(iii)]$\lim_{\delta \to 0} \inf_{\alpha \in A} \probability{\min \lbrace s^{\alpha, \epsilon}_{n+1} - s^{\alpha, \epsilon}_n \mid s^{\alpha, \epsilon}_n < T \rbrace \geq \delta} = 1$ for all 
$\epsilon > 0$ and $T > 0$
\end{itemize}
\end{lem}
\begin{proof}
Let $\epsilon > 0$ and $T > 0$ be given then by (i) we know that $\lim_{\delta \to \infty} \inf_{\alpha \in A} \probability{w^{\prime}(X^\alpha, \delta, T) < \epsilon/2} = 1$.  On the other hand,
for all $\alpha \in A$ we know that 
\begin{align*}
\probability{w^{\prime}(X^\alpha, \delta, T) < \epsilon/2} &\leq 
\probability{\min \lbrace \tau^{\alpha, \epsilon}_{n+1} - \sigma^{\alpha, \epsilon}_n > \delta \mid \tau^{\alpha, \epsilon}_n < T \rbrace > \delta} \\
&\leq
\probability{\min \lbrace \tau^{\alpha, \epsilon}_{n+1} - \sigma^{\alpha, \epsilon}_n > \delta \mid \tau^{\alpha, \epsilon}_n < T \rbrace \geq \delta} \\
\end{align*}
Now take the infimum over $\alpha \in A$ and let $\delta \to 0$ to conclude (ii).

Assume (ii) holds.  Let $\epsilon > 0$ and $T > 0$ be given.  Now pick $T^\prime > T$, let $\eta > 0$ then by (ii) there exists $0 < \delta < 2(T^\prime - T)$ such that 
\begin{align*}
1 - \eta &< \inf_{\alpha \in A} \probability{\min \lbrace \tau^{\alpha, \epsilon}_{n+1} - \sigma^{\alpha, \epsilon}_n \mid \tau^{\alpha, \epsilon}_n < T^\prime \rbrace \geq \delta} \\
&\leq \inf_{\alpha \in A} \probability{\min \lbrace \tau^{\alpha, \epsilon}_{n+1} - \sigma^{\alpha, \epsilon}_n \mid \tau^{\alpha, \epsilon}_n < T^\prime \rbrace > \rho}  \text{ for all $0 < \rho < \delta$}\\
&\leq \inf_{\alpha \in A} \probability{\min \lbrace s^{\alpha, \epsilon}_{n+1} - s^{\alpha, \epsilon}_n \mid s^{\alpha, \epsilon}_n < T^\prime - \rho/2 \rbrace > \rho/2}  \text{ for all $0 < \rho < \delta$} \\
&\leq \inf_{\alpha \in A} \probability{\min \lbrace s^{\alpha, \epsilon}_{n+1} - s^{\alpha, \epsilon}_n \mid s^{\alpha, \epsilon}_n < T \rbrace \geq \rho/2}  \text{ for all $0 < \rho < \delta$} \\
\end{align*}
which shows (iii).

Assume (iii) holds, let $T > 0$ and $\epsilon > 0$ be given.  Let $\eta>0$ be arbitrary and by (iii) pick a $\delta > 0$ such that
\begin{align*}
1 -\eta &< \inf_{\alpha \in A} \probability{\min \lbrace \tau^{\alpha, \epsilon}_{n+1} - \sigma^{\alpha, \epsilon}_n \mid \tau^{\alpha, \epsilon}_n < T \rbrace \geq \delta} \\
&\leq \inf_{\alpha \in A} \probability{\min \lbrace \tau^{\alpha, \epsilon}_{n+1} - \sigma^{\alpha, \epsilon}_n \mid \tau^{\alpha, \epsilon}_n < T \rbrace > \rho} \text{ for all $0 < \rho < \delta$} \\
&\leq \inf_{\alpha \in A} \probability{w^{\prime}(X^\alpha, \rho, T)  \leq \epsilon} \text{ for all $0 < \rho < \delta$} \\
\end{align*}
which shows $\lim_{\delta \to 0} \inf_{\alpha \in A} \probability{w^{\prime}(X^\prime, \rho, T)  \leq \epsilon} = 1$ for all $T > 0$ and $\epsilon > 0$.  This is equivalent to (i) since for any $\alpha \in A$, $T > 0$ and $0 < \epsilon$ we have 
\begin{align*}
\probability{w^{\prime}(X^\alpha, \rho, T)  \leq \epsilon/2} \leq \probability{w^{\prime}(X^\alpha, \rho, T)  < \epsilon} \leq \probability{w^{\prime}(X^\alpha, \rho, T)  \leq 2\epsilon}
\end{align*}
\end{proof}

\begin{lem}\label{SkorohodInfiniteModulusOfContinuityPullOutMin}Let $A$ be an arbitrary index set and for every $\alpha \in A$ let $0=s^\alpha_0 < s^\alpha_1 < \dotsb$ be a sequence of random variables such that $\lim_{n \to \infty} s^\alpha_n = \infty$.  Let $T > 0$ be arbitrary and define $K(\alpha,T) = \max \lbrace n \in \naturals \mid s^\alpha_n < T \rbrace$ and $F : [0,\infty) \to [0,1]$ by $F(t) = \sup_{\alpha \in A} \sup_{n \geq 0} \probability{(s^\alpha_{n+1} - s^\alpha_n) < t , s^\alpha_n < T}$.  Then for all $\delta > 0$ and $L \in \integers_+$ 
\begin{align*}
F(\delta) &\leq \sup_{\alpha \in A} \probability{\min_{0 \leq n \leq K(\alpha, T)} (s^\alpha_{n+1} - s^\alpha_n) < \delta} &\leq LF(\delta) + e^T \int_0^\infty L e^{-Lt} F(t) \, dt
\end{align*}
Therefore
\begin{align*}
\lim_{\delta \to 0} \sup_{\alpha \in A} \probability{\min_{0 \leq n \leq K(\alpha, T)} (s^\alpha_{n+1} - s^\alpha_n) } &= 0
\end{align*}
in and only if $F(0+) = 0$.
\end{lem}
\begin{proof}
For each fixed $\alpha \in A$ and every $n \in \naturals$, 
\begin{align*}
\lbrace (s^\alpha_{n+1} - s^\alpha_n) < \delta , s^\alpha_n < T \rbrace &\subset \lbrace \min_{0 \leq n \leq K(\alpha, T)} (s^\alpha_{n+1} - s^\alpha_n) < \delta \rbrace
\end{align*}
and therefore a union bound, taking the supremum over $n \in \naturals$ and $\alpha \in A$ yields the first inequality.

To see the second inequality, 
\begin{align*}
&\probability{\min_{0 \leq n \leq K(\alpha, T)} (s^\alpha_{n+1} - s^\alpha_n) < \delta} \\
&=\probability{\min_{0 \leq n \leq K(\alpha,T)} (s^\alpha_{n+1} - s^\alpha_n) < \delta, K(\alpha, T) \leq L-1} + \probability{\min_{0 \leq n \leq K(\alpha,T)} (s^\alpha_{n+1} - s^\alpha_n) < \delta, K(\alpha, T) \geq L} \\
&\leq \sum_{n=0}^{L-1} \probability{(s^\alpha_{n+1} - s^\alpha_n) < \delta, s^\alpha_n < T}  + \probability{K(\alpha, T) \geq L} \\
&\leq L F(\delta) + \expectation{e^{T - \sum_{n=0}^{L-1} (s^\alpha_{n+1} - s^\alpha_n)} ; K(\alpha, T) \geq L} \\
&\leq L F(\delta) + e^T \Pi_{n=0}^{L-1} \expectation{e^{-L(s^\alpha_{n+1} - s^\alpha_n)}; K(\alpha, T) \geq L}^{1/L} \\
&\leq L F(\delta) + e^T \Pi_{n=0}^{L-1} \expectation{e^{-L(s^\alpha_{n+1} - s^\alpha_n)}; s^\alpha_k < T}^{1/L} \\
\end{align*}
Here there is a multivariate generalization of Cauchy Schwartz used I think ($\expectation{\Pi_{k=1}^L f_k}^L \leq \Pi_{k=1}^L \expectation{f^L_k}$ for $f_k \geq 0$; TODO: prove this, instead of Young's inequality use the AMGM inequality).
\end{proof}

The last two lemmas can be combined to yield the following additional equivalent criterion for equicontinuity of a family of stochastic processes.
\begin{prop}Let $(S,r)$ be a separable metric space, let $A$ be an arbitrary index set and let $X^\alpha$ for $\alpha \in A$ be a family of stochastic processes with values in $D([0,\infty); S)$.  Define $\tau^{\alpha, \epsilon}_n, \sigma^{\alpha, \epsilon}_n$ and $s^{\alpha, \epsilon}_n$ as above then the following are equivalent
\begin{itemize}
\item[(i)]$\lim_{\delta \to 0} \inf_{\alpha \in A} \probability{w^{\prime}(X^\alpha, \delta, T) < \epsilon} = 1$ for all $\epsilon > 0$ and $T > 0$
\item[(ii)]$\lim_{\delta \to 0} \sup_{\alpha \in A} \sup_{n \geq 0} \probability{\tau^{\alpha, \epsilon}_{n+1} - \sigma^{\alpha, \epsilon}_n < \delta, \tau^{\alpha, \epsilon}_n < T} = 0$ \text{ for all $\epsilon > 0$ and $T>0$}
\end{itemize}
\end{prop}
\begin{proof}
\begin{clm} For all $\epsilon > 0$, $T > 0$, $\delta > 0$ and $\alpha \in A$,
\begin{align*}
\probability{ s^{\alpha, \epsilon}_{n+1} - s^{\alpha, \epsilon}_n < \delta/2, s^{\alpha, \epsilon}_n < T } 
&\leq \probability{ \tau^{\alpha, \epsilon}_{n+1} - \sigma^{\alpha, \epsilon}_n < \delta, \tau^{\alpha, \epsilon}_n < T + \delta} 
\leq \probability{s^{\alpha, \epsilon}_{n+1} - s^{\alpha, \epsilon}_n < \delta, s^{\alpha, \epsilon}_n < T + \delta} 
\end{align*}
\end{clm}

Recalling the defintion $s^{\alpha, \epsilon}_n = (\tau^{\alpha, \epsilon}_{n} + \sigma^{\alpha, \epsilon}_n)/2$ we see that if $s^{\alpha, \epsilon}_{n+1} - s^{\alpha, \epsilon}_n < \delta/2$ and $s^{\alpha, \epsilon}_n < T$ then it follows from $\tau_n \leq \sigma_{n+1}$ that 
\begin{align*}
\tau^{\alpha, \epsilon}_{n+1} - \sigma^{\alpha, \epsilon}_{n} &\leq  \tau^{\alpha, \epsilon}_{n+1} - \sigma^{\alpha, \epsilon}_{n} + \sigma^{\alpha, \epsilon}_{n+1} - \tau^{\alpha, \epsilon}_{n}  \\
&= 2 (s^{\alpha, \epsilon}_n  - \sigma^{\alpha, \epsilon}_n)  < \delta
\end{align*}
and from $\tau^{\alpha, \epsilon}_n \leq  s^{\alpha, \epsilon}_{n+1}$,
\begin{align*}
\tau^{\alpha, \epsilon}_n &< s^{\alpha, \epsilon}_{n} + \delta/2 < T + \delta
\end{align*}
Further if $\tau^{\alpha, \epsilon}_{n+1} - \sigma^{\alpha, \epsilon}_{n} < \delta$ and $\tau^{\alpha, \epsilon}_n < T + \delta$ then we have 
\begin{align*}
s^{\alpha, \epsilon}_{n+1} - s^{\alpha, \epsilon}_n &\leq \tau^{\alpha, \epsilon}_{n+1} - \sigma^{\alpha, \epsilon}_{n} < \delta
\end{align*}
and $ s^{\alpha, \epsilon}_n \leq  \tau^{\alpha, \epsilon}_n < T + \delta$.
The claim follows from the union bound implied by these set inclusions.

Now if we assume (i) then $\epsilon$ and $T > 0$ be given.  By Lemma \ref{SkorohodInfiniteModulusOfContinuityEquivalences} we know that 
\begin{align*}
1 &= \lim_{\delta \to 0} \inf_{\alpha \in A} \probability{\min \lbrace s^{\alpha, \epsilon}_{n+1} - s^{\alpha, \epsilon}_n \mid s^{\alpha, \epsilon}_n < T \rbrace \geq \delta} \\
&=1 - \lim_{\delta \to 0} \sup_{\alpha \in A} \probability{\min \lbrace s^{\alpha, \epsilon}_{n+1} - s^{\alpha, \epsilon}_n \mid s^{\alpha, \epsilon}_n < T \rbrace < \delta} 
\end{align*}
thus by the claim and Lemma \ref{SkorohodInfiniteModulusOfContinuityPullOutMin} we know that 
\begin{align*}
\lim_{\delta \to 0} \sup_{\alpha \in A} \sup_{n \geq 0} \probability{(\tau^{\alpha, \epsilon}_{n+1} - \sigma^{\alpha, \epsilon}_n) < \delta , \tau^{\alpha, \epsilon}_n < T} 
&\leq \lim_{\delta \to 0} \sup_{\alpha \in A} \sup_{n \geq 0} \probability{(s^{\alpha, \epsilon}_{n+1} - s^{\alpha, \epsilon}_n) < \delta , s^{\alpha, \epsilon}_n < T} = 0\\
\end{align*}

If we have (ii) then by the claim,
\begin{align*}
\lim_{\delta \to 0} \sup_{\alpha \in A} \sup_{n \geq 0} \probability{ s^{\alpha, \epsilon}_{n+1} - s^{\alpha, \epsilon}_n < \delta, s^{\alpha, \epsilon}_n < T } 
&\leq \lim_{\delta \to 0} \sup_{\alpha \in A} \sup_{n \geq 0} \probability{ s^{\alpha, \epsilon}_{n+1} - s^{\alpha, \epsilon}_n < 2 \delta, s^{\alpha, \epsilon}_n < T + 2\delta}  \\
&\leq \lim_{\delta \to 0} \sup_{\alpha \in A} \sup_{n \geq 0} \probability{ s^{\alpha, \epsilon}_{n+1} - s^{\alpha, \epsilon}_n < \delta, s^{\alpha, \epsilon}_n < T +1 } =0 \\
\end{align*}
and then applying Lemma \ref{SkorohodInfiniteModulusOfContinuityPullOutMin} and  Lemma \ref{SkorohodInfiniteModulusOfContinuityEquivalences}  we get (i).
\end{proof}

