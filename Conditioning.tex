\section{Conditioning}

\subsection{$L^p$ Spaces}
Prior to discussing the general formulation of the notion of
conditional probabilities we shall need to lay down some techniques of
functional analysis pertaining to spaces of measurable (and
integrable) random variables.

\begin{defn}Given a measure space $(\Omega, \mathcal{A}, \mu)$ and $p
  \geq 1$ we let $L^p(\Omega, \mathcal{A}, \mu)$ be the space of equivalence
  classes of measurable functions such that $\int \abs{f}^p \, d\mu <
  \infty$ under the equivalence relation of
  almost everywhere equality.  For any element $f \in   L^p(\Omega,
  \mathcal{A}, \mu)$ we define 
\begin{align*}
\norm{f}_p &= \left ( \int \abs{f}^p \, d\mu \right )^{\frac{1}{p}}
\end{align*}
\end{defn}

It is clear that the spaces $L^p(\Omega, \mathcal{A}, \mu)$ but our first goal is to establish that each is a complete
normed vector space (a.k.a. Banach space).  As our first step in that
direction we need to prove the triangle inequality
\begin{lem}[Minkowski Inequality]\label{MinkowskiInequality}Given $f,g
  \in L^p(\Omega,  \mathcal{A}, \mu)$ then $f+g \in L^p(\Omega,
  \mathcal{A}, \mu)$ and $\norm{f+g}_p \leq \norm{f}_p + \norm{g}_p$.
\end{lem}
\begin{proof}
Note that it suffices to assume that $f \geq 0$ and $g \geq 0$ since
if we have the inequality for positive elements then it follows for
all elements by applying the ordinary triangle inequality on $\reals$
and using the fact that $x^p$ is increasing to see
\begin{align*}
\norm{f+g}_p &\leq \norm{\abs{f}+\abs{g}}_p \leq
\norm{\abs{f}}_p+\norm{\abs{g}}_p = \norm{f}_p+\norm{g}_p 
\end{align*}
The case $p=1$ follows immediately from linearity of integral (in fact
we have equality).  

For $1 < p < \infty$, first use the following
crude pointwise bound to see that $f+g \in L^p(\Omega,  \mathcal{A}, \mu)$:
\begin{align*}
(f+g)^p &\leq (f \vee g + f \vee g)^p = 2^p f^p \vee g^p \leq 2^p (f^p
+ g^p)
\end{align*}
and therefore $\norm{f+g}_p^p \leq 2^p (\norm{f}_p^p + \norm{g}_p^p) <
\infty$.  To see the triangle inequality, note that we can assume that
$\norm{f+g}_p > 0$ for otherwise the triangle inequality follows by
positivity of the norm.  Write
\begin{align*}
\norm{f+g}_p^p &= \int (f+g)^p \, d\mu = \int f (f+g)^{p-1} \, d\mu +
\int g (f+g)^{p-1} \, d\mu \\
\end{align*}
Now we can apply the H\"{o}lder Inequality (Lemma
\ref{Holder}) to each of the terms on the right hand side and use the
fact that $\frac{1}{p} + \frac{1}{q}$ is equivalent to $q = (p-1)q$ to see
\begin{align*}
\int f (f+g)^{p-1} \, d\mu &\leq \left(\int f^p \,
  d\mu\right)^\frac{1}{p}\left(\int (f+g)^{(p-1)q} \,
  d\mu\right)^\frac{1}{q} =\norm{f}_p \norm{f+g}_q^{p/q}
\end{align*}
Applying this argument to the term $\int g (f+g)^{p-1} \, d\mu$ as
well we get
\begin{align*}
\norm{f+g}_p^p &\leq  (\norm{f}_p + \norm{g}_p) \cdot
\norm{f+g}_q^{p/q}
\end{align*}
and dividing through by $\norm{f+g}_p^{p/q}$ and using $p - \frac{p}{q}=1$ we get $\norm{f+g}_p \leq  \norm{f}_p + \norm{g}_p$.
\end{proof}

\begin{lem}\label{CompletenessOfLp}For $p \geq 1$ the normed vector
  space $L^p(\Omega,  \mathcal{A}, \mu)$ is complete.
\end{lem}
\begin{proof}
Let $f_n$ be a Cauchy sequence in $L^p(\Omega,  \mathcal{A}, \mu)$.
The first step of the proof is to show that there is a subsequence of
$f_n$ that converges almost everywhere to an element $f \in
L^p(\Omega,  \mathcal{A}, \mu)$.

By the Cauchy property, for each $j \in \naturals$ we can find an $n_j > 0$ such that
$\norm{f_m - f_{n_j}}_p \leq \frac{1}{2^j}$ for all $m > n_j$.  In
this way we get a subsequence $f_{n_j}$ such that $\norm{f_{n_{j+1}} -
    f_{n_j}}_p \leq \frac{1}{2^j}$ for all $j \in \naturals$.  Now by
  applying Monotone Convergence and the triangle inequality we have
\begin{align*}
\norm{\sum_{j=1}^\infty \abs{f_{n_{j+1}} - f_{n_j}} }_p &= \lim_{N \to
  \infty} \norm{\sum_{j=1}^N \abs{f_{n_{j+1}} - f_{n_j}} }_p \\
&\leq \lim_{N \to
  \infty} \sum_{j=1}^N \norm{f_{n_{j+1}} - f_{n_j}}_p \\
&\leq \lim_{N \to
  \infty} \sum_{j=1}^N \frac{1}{2^j} < \infty
\end{align*}
and therefore we know that $\sum_{j=1}^\infty \abs{f_{n_{j+1}} -
  f_{n_j}}$ is almost surely finite.  Anywhere this sum is finite it
follows that $f_{n_j}$ is a Cauchy sequence in $\reals$.  To see this,
suppose we are given
$\epsilon > 0$ we pick $N > 0$ such that $\sum_{j=N}^\infty \abs{f_{n_{j+1}} -
  f_{n_j}} < \epsilon$, then for any $k \geq j \geq N$ we have 
\begin{align*}\abs{f_{n_k} -
  f_{n_j}} = \abs{\sum_{m=j}^k (f_{n_{m+1}} -  f_{n_m})} \leq
\sum_{m=j}^k \abs{ f_{n_{m+1}} -  f_{n_m}} < \epsilon
\end{align*}

We know that the set where $f_{n_j}$ converges is measurable (TODO:
Where is this?) so we can
define $f$ to be the limit of the Cauchy sequence $f_{n_j}$ where
valid and define it to be zero elsewhere (a set of measure zero).

To see that $f \in L^p(\Omega,  \mathcal{A}, \mu)$ and to show that
$f_n$ converges to $f$, suppose $\epsilon > 0$ is given and pick $N
\in \naturals$ such that for all $m,n \geq N$ we have
$\norm{f_m-f_n}_p < \epsilon$.  Now we can use Fatou's
Lemma (Theorem \ref{Fatou})  to see for any $n \geq N$, 
\begin{align*}
\int \abs{f-f_n}^p \, d\mu &\leq \liminf_{j \to \infty} \int
\abs{f_{n_j} - f_n}^p \, d\mu \leq \sup_{m\geq n} \int
\abs{f_{m} - f_n}^p \, d\mu < \epsilon^p
\end{align*}
Therefore by the Minkowski Inequality, we see that $f = f_n + (f -
f_n)$ is in  $L^p(\Omega,  \mathcal{A}, \mu)$ and $f_n \tolp{p} f$.
\end{proof}

We know that measurable functions can be approximated by simple
functions (Lemma \ref{PointwiseApproximationBySimple}) with pointwise
convergence.  It is useful to extend this approximation to $L^p$ spaces.
\begin{lem}\label{LpApproximationBySimple}Simple functions are dense in $L^p(\Omega, \mathcal{A}, \mu)$.
\end{lem}
\begin{proof}
Let's assume for the moment that $\mu$ is a finite measure.  In that
case, every simple function is in $L^p$.  Pick a positive function $f
\in L^p(\Omega, \mathcal{A}, \mu)$ and sequence of simple functions
such that $f_n \uparrow f$.  Then it is also true that $f_n^p \uparrow
f^p$ and Monotone Convergence tells us that $\lim_{n \to \infty}
\norm{f_n}_p  = \norm{f}_p$.  By Lemma \ref{ConvergenceInMeanConvergenceOfMeans} we conclude that $f_n
\tolp{p} f$.

To finish the proof, take an arbitrary $f$ and write it as $f = f_+ -
f_-$ and use linearity.

TODO: What about non-finite measures?  This argument clearly extends
to $\sigma$-finite by restricting each $f_n$ to a finite subset.  In
the general case we need to handle the fact that all simple functions
are not in $L^p$.
\end{proof}

Note that for any $\sigma$-algebra $\mathcal{F} \subset \mathcal{A}$
we can also consider the space $ L^p(\Omega,  \mathcal{F}, \mu)$.  
As we shall soon see, it will become important to understand a bit
about these spaces as $\mathcal{F}$ vary.  The first thing to note is
that for $\mathcal{G} \subset \mathcal{F}$, $L^p(\Omega,  \mathcal{G},
\mu)$ is a closed linear subspace of $L^p(\Omega,  \mathcal{F},
\mu)$.  The inclusion is trivial since any $\mathcal{G}$-measurable
function is also $\mathcal{F}$-measurable; closure follows from the
completeness of the space $L^p(\Omega, \mathcal{G}, \mu)$ (Lemma
\ref{CompletenessOfLp}).

The following approximation result will be used only occasionaly.
\begin{lem}\label{LpDensityUnionSubsigmaAlgebras}$\cup_n L^p(\Omega, \mathcal{F}_n, \mu)$ is dense in
  $L^p(\Omega, \bigvee_n \mathcal{F}_n, \mu)$
\end{lem}
\begin{proof}The first thing to show the result for indicator
  functions.  A general fact, suppose $V$ is a closed linear subspace of
  $L^p$ and let $\mathcal{C} = \lbrace A \mid \characteristic{A} \in
   V\rbrace$.  We claim that $\mathcal{C}$ is a $\lambda$-system.
    Given $A, B \in \mathcal{C}$ with $A \subset B$, we have $B
    \setminus A \in \mathcal{C}$ since $\characteristic{B \setminus A}
    = \characteristic{B} - \characteristic{A}$ and $V$ is a linear
    space.  Now assume that $A_1 \subset A_2 \subset \cdots \in
    \mathcal{C}$.  We have that $\characteristic{A_n} \uparrow
    \characteristic{A}$ and continuity of measure (Lemma
    \ref{ContinuityOfMeasure}) tells us that
    $\lim_{n \to \infty} \norm{\characteristic{A_n}}_p =
    \norm{\characteristic{A}}_p$ so Lemma
    \ref{ConvergenceInMeanConvergenceOfMeans} implies $\characteristic{A_n}
    \tolp{p} \characteristic{A}$.  Since $V$ is closed we know
    $\characteristic{A} \in V$.
\end{proof}

TODO: Develop inner product and projection for $L^2$ spaces.

\subsection{Conditional Expectation}
 Before getting into the technical details we want to get set the
intuition for the problem and the form that solutions will take.
Given a random element $\xi$ in $S$ and a random variable $\eta$, we want to
formulate the notion of the expected value of $\eta$ given a value of
$\xi$.  The immediate way to think of representing such an object is as a map from
$S$ to $\reals$.  In practice the representation is expressed in a
different but equivalent way.  Recall from Lemma
\ref{FunctionalRepresentation} that any random variable
$\gamma$ that is $\xi$-measurable can be factored as $f \circ
\xi$ for some measurable $f : S \to \reals$.  In this way the
conditional expectation may equally be considered as $\xi$-measurable
random variable.  It is this latter representation that is most
convenient for working with (and constructing) conditional
expectations.  To remove matters a little further from the initial
intuition, one often makes use of the fact that the conditional
expection winds up only depending on the $\sigma$-field induced by
$\xi$ and discusses conditioning with respect to arbitrary sub
$\sigma$-fields.

TODO: Elaborate on the three faces of conditional expectation:
projection, density/Radon-Nikodym derivative and disintegration.

Existence via Radon-Nikodym.  The Radon-Nikodym theorem (Theorem
\ref{RadonNikodym}) can
be given a martingale proof (hence derived in some sense from the
existence of conditional expectations).  However, the standard proof
for Radon-Nikodym using
Hahn Decomposition does not depend on the existence of conditional
expection and in fact, the Radon-Nikodym theorem can easily be used to
prove the existence of conditional expectations.
Given $\xi \geq 0$ and $\mathcal{F} \subset \mathcal{A}$, then define
the probability measure $\nu(A) =
\expectation{\xi \characteristic{A}}$.  Note that $\nu$ is absolutely
continuous with respect to $\mu$ on $\mathcal{F}$.  Therefore, the Radon-Nikodym
derivative with respect to $(\Omega, \mathcal{F})$ exists and
satisfies 
\begin{align*}
\nu(A) = \expectation{\xi \characteristic{A}} =
\expectation{\frac{d\nu}{d\mu} \characteristic{A}}
\end{align*}
for all $A \in \mathcal{F}$.  This equality shows that
$\frac{d\nu}{d\mu}$ is a conditional expectation of $\xi$.  For
general $\xi$, write $\xi=\xi_+ - \xi_-$ and proceed as above.
 
TODO: Make sure we have covered the following:  Definition of $L^p$
spaces, completeness of $L^p$ spaces, definition of Hilbert space,
orthogonal projections in Hilbert spaces.  Density of $L^2$ in $L^1$.
Unique extension of a bounded linear operator from a dense subspace
of a complete normed linear space.

On the other hand, there is very appealing construction of conditional
expectation using function spaces that we provide here.  Recall that
for a measurable space $(\Omega, \mathcal{A}, \mu)$ we have associated
Banach spaces of $p$-integrable functions $L^p(\Omega, \mathcal{A}, \mu)$ with norm $\norm{f}_p =
\left ( \int \abs{f}^p \, d \mu \right ) ^ \frac{1}{p}$.  In the
special case $p=2$ we actually have a Hilbert space $L^2(\Omega,
\mathcal{A}, \mu)$ with inner product $<f, g> = \int f g \, d \mu$.
Suppose we have a sub $\sigma$-algebra $\mathcal{F} \subset
\mathcal{A}$ and we have a canonical inclusion $L^p(\Omega, \mathcal{F},
\mu) \subset L^p(\Omega, \mathcal{A},
\mu)$ as a subspace.  In fact by the completeness of $L^p(\Omega,
\mathcal{F},\mu)$, we know that this is a \emph{closed} subspace.
Therefore if we specialize to the case of $L^2(\mathcal{F}) \subset
L^2(\mathcal{A})$ then we have the orthogonal projection onto
$L^2(\mathcal{F})$.  For square integrable random variables, this
orthogonal projection defines the conditional expectation.  In the
following, we extend this defintion to all integrable random variables
and prove the basic properties.

TODO: Elaborate on the ``a.s. uniqueness'' in the definition.

\begin{thm}[Conditional Expectation]\label{ConditionalExpectation}For
  any $\mathcal{F} \subset \mathcal{A}$ there exists a unique linear
  operator $\cexpectationop{\mathcal{F}} : L^1 \to L^1(\mathcal{F})$
  such that 
\begin{itemize} 
\item[(i)]$\expectation{\cexpectation{\mathcal{F}}{\xi} ; A} = \expectation{\xi ;
    A}$ for all $\xi \in L^1$, $A \in \mathcal{F}$
\end{itemize}
The following properties also hold for $\xi, \eta \in L^1$,
\begin{itemize}
\item[(ii)]$\expectation{\abs{\cexpectation{\mathcal{F}}{\xi}}} \leq \expectation{\abs{\xi}}$ a.s.
\item[(iii)]$\xi \geq 0$ implies $\cexpectation{\mathcal{F}}{\xi} \geq
  0$ a.s.
\item[(iv)]$0 \leq \xi_n \uparrow \xi$ implies
  $\cexpectation{\mathcal{F}}{\xi_n}  \uparrow
  \cexpectation{\mathcal{F}}{\xi}$ a.s.
\item[(v)]$\cexpectation{\mathcal{F}}{\xi \eta} = \xi
  \cexpectation{\mathcal{F}}{\eta}$ if $\xi$ is
  $\mathcal{F}$-measurable and $\xi\eta,
  \xi\cexpectation{\mathcal{F}}{\eta} \in L^1$
\item[(vi)]$\expectation{\cexpectation{\mathcal{F}}{\xi}
    \cdot \cexpectation{\mathcal{F}}{\eta}} = \expectation{\xi \cdot
    \cexpectation{\mathcal{F}}{\eta}} = \expectation{
    \cexpectation{\mathcal{F}}{\xi} \cdot \eta} $
\item[(vii)]$\cexpectation{\mathcal{F}}{\cexpectation{\mathcal{G}}{\xi}}
= \cexpectation{\mathcal{F}}{\xi}$ a.s. for all $\mathcal{F} \subset \mathcal{G}$.
\end{itemize}
\end{thm}
\begin{proof}
Begin by defining $\cexpectationop{\mathcal{F}} : L^2 \to
L^2(\mathcal{F})$ as orthogonal projection.  If we pick $A \in
\mathcal{F}$, then $\characteristic{A} \in L^2(\mathcal{F})$ and
therefore, $\xi - \cexpectation{\mathcal{F}}{\xi} \perp
\characteristic{A}$ which shows
\begin{align*}
\expectation{\xi ; A} &= <\xi, \characteristic{A}> =
<\cexpectation{\mathcal{F}}{\xi}, \characteristic{A}> =
\expectation{\cexpectation{\mathcal{F}}{\xi} ; A}
\end{align*}
If we define $A = \lbrace \cexpectation{\mathcal{F}}{\xi} \geq 0
\rbrace$ the above implies
\begin{align*}
\expectation{\abs{\cexpectation{\mathcal{F}}{\xi}}} &=
\expectation{\cexpectation{\mathcal{F}}{\xi} ; A} -
\expectation{\cexpectation{\mathcal{F}}{\xi} ; A^c} & & \text{by
  linearity of expectation}\\
&= \expectation{\xi ; A} - \expectation{\xi ; A^c} & &\text{by (i)} \\
&\leq \expectation{\abs{\xi}; A} + \expectation{\abs{\xi}; A^c} & &
\text{since $\xi \leq \abs{\xi}$ and 
  $-\xi \leq \abs{\xi}$} \\
&= \expectation{\abs{\xi}} & & \text{by linearity of expecation}
\end{align*}
This inequality shows us that the linear operator
$\cexpectationop{\mathcal{F}}$ is bounded in the $L^1$ norm as well as
in the $L^2$ norm.  On the other hand, we know that $L^2$ is dense in
$L^1$ and $L^1$ is complete so there is a unique extension of $\cexpectationop{\mathcal{F}}$
to a bounded linear operator $L^1 \to L^1{\mathcal{F}}$.  Concretely,
for any $\xi \in L^1$, we pick a sequence $\xi_n \in L^2$ such that
$\lim_{n \to \infty} \xi_n \to \xi$ in the $L^1$ norm and define
$\cexpectation{\mathcal{F}}{\xi} = \lim_{n \to \infty}
\cexpectation{\mathcal{F}}{\xi_n}$ where the limit is in the $L^1$
norm.  Since the $L^1$ closure of $L^2(\mathcal{F})$ is
$L^1(\mathcal{F})$, we see that the definition is plausible.  

TODO: Show independence, linearity and boundedness of the extension.
Perhaps factor this out into a separate Lemma; it is a generic
construction.

To see that the condition (i) uniquely defines
$\cexpectation{\mathcal{F}}{\xi} $ a.s., suppose we had two
$\mathcal{F}$-measurable random variables $\eta$ and $\rho$ for which
$\expectation{\eta ; A} = \expectation{\rho ; A}$ for all $A \in
\mathcal{F}$.  Let $A = \lbrace \eta > \rho \rbrace$ which is
$\mathcal{F}$-measurable and so we have assumed
$\expectation{\eta - \rho ; A} = 0$.  If we apply Lemma \ref{ZeroIntegralImpliesZeroFunction} we
know that $(\eta - \rho)\characteristic{A} = 0$ a.s. which shows that
$\probability{A} = 0$.   The same argument shows that
$\rho > \eta$ with probability $0$, hence $\eta = \rho$ a.s.

To see (iii), let $A = \lbrace \cexpectation{\mathcal{F}}{\xi} < 0 \rbrace$ and
observe that 
\begin{align*}
0 &\leq \expectation{-\cexpectation{\mathcal{F}}{\xi} ; A} =
\expectation{-\xi ; A} \leq 0
\end{align*}
and therefore $\expectation{-\cexpectation{\mathcal{F}}{\xi} ; A} = 0$
which applying Lemma \ref{ZeroIntegralImpliesZeroFunction} implies
$\probability{A}=0$.

 To see (iv), suppose $0 \leq \xi_n \uparrow \xi$ a.s.  Then by Monotone
 Convergence, $\lim_{n \to \infty} \expectation{\abs{\xi - \xi_n}} =
 0$.  Now by (ii) and linearity of conditional expection, 
\begin{align*}
0 \leq \lim_{n \to \infty} \expectation{\abs{\cexpectation{\mathcal{F}}{\xi}
  - \cexpectation{\mathcal{F}}{\xi_n}} } \leq \lim_{n \to \infty} \expectation{\abs{\xi - \xi_n}} =
 0
\end{align*}
which shows that $\cexpectation{\mathcal{F}}{\xi_n}$ converges to
$\cexpectation{\mathcal{F}}{\xi}$ in $L^1$.  Now by Lemma
\ref{ConvergenceInMeanImpliesInProbability} this
implies that the converges is in probability and by Lemma \ref{ConvergenceInProbabilityAlmostSureSubsequence} there is a
subsequence that converges a.s.  By (iii) we know that $\cexpectation{\mathcal{F}}{\xi_n}$
 is non-decreasing so we know by Lemma \ref{IncreasingSequenceWithConvergentSubsequence} that that almost sure convergence of the
 subsequence extends to the almost sure convergence of the entire sequence.


To see (v), note that if $\xi$ is $\mathcal{F}$-measurable then for
every $\eta \in L^1$, we know $\xi\cexpectation{\mathcal{F}}{\eta}$ is
$\mathcal{F}$-measurable and by simple calculation
\begin{align*}
\expectation{\xi\cexpectation{\mathcal{F}}{\eta}; A} &= \expectation{\xi\eta; A}
\end{align*}
by the apply the extension of the property (i) to the
$\mathcal{F}$-measurable function
$\xi\characteristic{A}$.  Now by (v) follows by applying (i) again.

For the property (vi), by symmetry we only have to prove $\expectation{\cexpectation{\mathcal{F}}{\xi}
    \cdot \cexpectation{\mathcal{F}}{\eta}} = \expectation{\xi \cdot
    \cexpectation{\mathcal{F}}{\eta}}$.  To prove this first assume
  that $\xi, \eta \in L^2$.  In that case, we know that
  $\cexpectation{\mathcal{F}}{\eta}
  \in L^2(\mathcal{F})$ and $\xi - \cexpectation{\mathcal{F}}{\xi}
  \perp L^2(\mathcal{F})$, so 
\begin{align*}
\expectation{\cexpectation{\mathcal{F}}{\xi}
    \cdot \cexpectation{\mathcal{F}}{\eta}} &= <\cexpectation{\mathcal{F}}{\xi}
    ,\cexpectation{\mathcal{F}}{\eta}> \\
&= <\cexpectation{\mathcal{F}}{\xi} - \xi, \cexpectation{\mathcal{F}}{\eta}> + <\xi, \cexpectation{\mathcal{F}}{\eta}>\\
&= <\xi, \cexpectation{\mathcal{F}}{\eta}> = \expectation{\xi \cdot \cexpectation{\mathcal{F}}{\eta}}\\
\end{align*}
Now by the density of $L^2 \subset L^1$, for general $\xi, \eta \in
L^1$ we pick $\xi_n \tolp{1} \xi$ and $\eta_n \tolp{1} \eta$ with
$\xi_n, \eta_n \in L^2$.  By the above 
Lastly, we prove (vii).  Suppose we are given $\sigma$-algebras
$\mathcal{F} \subset \mathcal{G}$.  Then for $A \in \mathcal{F}
\subset \mathcal{G}$,
\begin{align*}
\expectation{\cexpectation{\mathcal{G}}{\xi} ; A} &= \expectation{\xi
  ; A} & & \text{by (i) applied to
  $\cexpectation{\mathcal{G}}{\xi}$}\\
&= \expectation{\cexpectation{\mathcal{F}}{\xi}
  ; A} & & \text{by (i) applied to
  $\cexpectation{\mathcal{F}}{\xi}$}\\
\end{align*}
where are the equalities are a.s.   By definition $\cexpectation{\mathcal{F}}{\xi}$ is
$\mathcal{F}$-measurable which shows by (i) that
$\cexpectation{\mathcal{F}}{\cexpectation{\mathcal{G}}{\xi}}
= \cexpectation{\mathcal{F}}{\xi}$ a.s.
\end{proof}

When verifying the defining property of conditional expectation it is
often useful to observe that it suffices to check indicator functions
for sets in a generating $\pi$-system.
\begin{lem}\label{ConditionalExpectationExtension}Suppose $\xi, \eta$ are integrable or non-negative random
  variables and $\mathcal{F}$ is a $\pi$-system such that $\Omega \in
  \mathcal{F}$ and for all $A
  \in \mathcal{F}$, we have $\expectation{\xi ; A} = \expectation{
    \eta; A}$.  Then we have $\expectation{\xi ; A} = \expectation{
    \eta; A}$ for all $A \in \sigma(\mathcal{F})$.
\end{lem}
\begin{proof}We first let $\mathcal{G}$ be the set of all $A$ such that  $\expectation{\xi ; A} = \expectation{
    \eta; A}$ and show that it is a $\lambda$-system.  If $A, B \in
  \mathcal{G}$ and $B \supset A$ then
\begin{align*}
\expectation{ \xi ; B\setminus A} &= \expectation{\xi;B} -
\expectation{\xi; A} = \expectation{\eta;B} -
\expectation{\eta; A} = \expectation{ \eta ; B\setminus A}
\end{align*}

Now suppose that we have $A_1 \subset A_2 \subset \cdots \in
\mathcal{G}$.  We claim that $\lim_{n \to \infty} \expectation{ \xi ;
  A_n} = \expectation{ \xi ; \cup_n A_n}$ and similarly with $\eta$.
In the case that we assume $\xi$ is integrable then we have $\abs{\xi
  \characteristic{A_n}} \leq \abs{\xi}$, so we may use Dominated
Convergence whereas in the case that $\xi$ is non-negative we may use
Monotone Convergence.  In either case,
\begin{align*}
\expectation{ \xi ; \cup_n A_n} &= \lim_{n \to \infty} \expectation{ \xi ;
  A_n} = \lim_{n \to \infty} \expectation{ \eta ;
  A_n} = \expectation{ \eta ; \cup_n A_n}
\end{align*}
We have assumed that $\Omega \in \mathcal{G}$ therefore we have shown $\mathcal{G}$ is a $\lambda$-system and our
assumption is that $\mathcal{F} \subset \mathcal{G}$ so we apply the
$\pi$-$\lambda$ Theorem (Theorem \ref{MonotoneClassTheorem}) to get
the result.
\end{proof}

Occasionally it can be useful to extend the defining property of
conditional expectation beyond indicator functions.
\begin{lem}Let $\xi \in L^1$ then for a $\sigma$-algebra $\mathcal{F}$
  and all bounded $\eta \in L^1(\mathcal{F})$,
  $\expectation{\cexpectation{\mathcal{F}}{\xi}\cdot\eta} = \expectation{\xi\cdot\eta} $.
\end{lem}
\begin{proof}
This is a simple application of the standard machinery.
Property (i) is exactly this statement for $\mathcal{F}$-measurable indicator functions.
Linearity of expectation shows that the statement then holds for
$\mathcal{F}$-measurable bounded simple functions.  For
$\mathcal{F}$-measurable bounded $\eta \geq 0$, we pick an increasing
approximation by bounded simple functions $\eta_n \uparrow \eta$.  First we
assume that $\xi \geq 0$ as well and note by (iii) that
$\cexpectation{\mathcal{F}}{\xi} \geq 0$ a.s. as well.  Now we can
apply monotone convergence to the sequences
$\cexpectation{\mathcal{F}}{\xi} \cdot \eta_n$ and $\xi \cdot \eta_n$,
\begin{align*}
\expectation{\xi \cdot \eta} &= \lim_{n \to \infty} \expectation{\xi
  \cdot \eta_n} & & \text{by Monotone Convergence} \\
&=\lim_{n \to \infty} \expectation{\cexpectation{\mathcal{F}}{\xi} 
  \cdot \eta_n} \\
&=\expectation{\cexpectation{\mathcal{F}}{\xi} 
  \cdot \eta} & & \text{by Monotone Convergence} \\
\end{align*}
For general integrable $\xi$, we write $\xi = \xi_+ - \xi_-$ and use
linearity of expectation and conditional expectation.  Lastly we
extend to arbitrary bounded $\mathcal{F}$-measurable $\eta$ in a similar
way.  


TODO: Is there a shorter argument (that probably amounts to the
same thing) using the density of $L^2$ in $L^1$?  TODO:  This
statement is FALSE!  The product of integrable random variables is not
necessarily integrable!  The above is true for BOUNDED random
variables, but that isn't enough for our needs below.  Indeed it is
better to prove these in $L^2$ and then take limits using (iv).
\end{proof}

TODO: Provide an example of conditional expectation and a dyadic
$\sigma$-algebra.

A last observation is that conditional expectations depend only 
``local'' information in both the random variable and the
$\sigma$-algebra.  This has an intuitive appeal as one can think of
the $\sigma$-algebra against which the conditional expectation is
taken as a specifying a coarser resolution of the random variable and
this coarsening is obtained by averaging/integration.  So long as the
domains over which we integrate are contained entirely inside of a
set we are interested in, the conditional expectation should only
depend on the $\sigma$-algebra restricted to that set and the values
of the random variable on that set.  We proceed to make this idea more
formal and give a proper proof.

\begin{defn}Given $\sigma$-algebras $\mathcal{F}$, $\mathcal{G}$ and
  $\mathcal{A}$ with $\mathcal{F} \subset \mathcal{A}$ and
  $\mathcal{G} \subset \mathcal{A}$ and a set $A \in \mathcal{F} \cap
  \mathcal{G}$, we way that $\mathcal{F}$ and $\mathcal{G}$
  \emph{agree on $A$} if for every $B \subset A$, $B \in \mathcal{F}$
  if and only if $B \in \mathcal{G}$.
\end{defn}
\begin{lem}Given $\sigma$-algebras $\mathcal{F}$, $\mathcal{G}$ and
  $\mathcal{A}$ with $\mathcal{F} \subset \mathcal{A}$ and
  $\mathcal{G} \subset \mathcal{A}$ and a set $A \in \mathcal{F} \cap
  \mathcal{G}$ such that $\mathcal{F}$ and $\mathcal{G}$ agree on $A$
  and random variables $\xi$ and $\eta$ such that $\xi$ and $\eta$
  agree almost surely on $A$ then
\begin{align*}
\cexpectationlong{\mathcal{F}}{\xi} &=
\cexpectationlong{\mathcal{G}}{\eta} \text{ a.s. on $A$}
\end{align*}
\end{lem}
\begin{proof}
First note that because $A \in \mathcal{F} \cap \mathcal{G}$ and
because $\characteristic{A}\xi = \characteristic{A}\eta$ a.s., we have
\begin{align*}
\cexpectationlong{\mathcal{F}}{\xi}  &=
\cexpectationlong{\mathcal{F}}{\characteristic{A}\xi} =
\cexpectationlong{\mathcal{F}}{\characteristic{A}\eta} = \cexpectationlong{\mathcal{F}}{\eta} 
\end{align*}
so it suffices to show that $\cexpectationlong{\mathcal{F}}{\eta} =
\cexpectationlong{\mathcal{G}}{\eta}$ a.s.
\end{proof}

The definition of conditional expectation as given is rather abstract
but in the case of random variables with densities, we can make the
concept more concrete.

TODO: Where to put this?
\begin{lem}Let $(\xi, \eta)$ be a random vector in $\reals^2$.
  Suppose that $(\xi, \eta)$ has a density $f$, then 
\begin{itemize}
\item[(i)]Both  $\xi$ and $\eta$ have a densities given by the
  formulas
\begin{align*}
f_{\xi}(y) = \int_{-\infty}^\infty f(y,z) \, dz & & f_{\eta}(z) = \int_{-\infty}^\infty f(y,z) \, dy
\end{align*}
\item[(ii)]$\xi$ and $\eta$ are independent if and only if $f(y,z) = f_{\xi}(y) f_{\eta}(z) $.
\item[(iii)]For any $y \in \reals$ such that $f_{\xi}(y) \neq 0$, we
  have the density
\begin{align*}
f_{\xi=y}(z) &= \frac{f(y,z)}{f_{\xi}(y)}
\end{align*}
\item[(iv)]If we define $h_{\eta}(y) = \int_{-\infty}^\infty z f_{\xi=y}(z)
  \, dz$ then for every measurable $g : \reals \to \reals$ such that
$g(\xi)$ is integrable, we have
\begin{align*}
\expectation{g(\xi) \cdot h_{\eta}(\xi)} &= \expectation{\xi \cdot \eta}
\end{align*}
\end{itemize}
\end{lem}

If we consider $\eta$ a random element in some $(T, \mathcal{T})$, $\xi$ an integrable random
variable then we usually write $\cexpectationlong{\sigma(\eta)}{\xi} =
\cexpectationlong{\eta}{\xi}$ and speak of the \emph{conditional
  expectation of $\xi$ with respect to $\eta$}.  
\begin{lem}There exists a measurable function $f : T \to \reals$ such
  that $\cexpectationlong{\eta}{\xi} = f(\eta)$, furthermore such an
  $f$ is unique almost surely $\pushforward{\eta}{P}$.  If we are
  given another pair $\tilde{\xi}$ and $\tilde{\eta}$ such that $(\xi,
  \eta) \eqdist (\tilde{\xi}, \tilde{\eta})$ then $\cexpectationlong{\tilde{\eta}}{\tilde{\xi}} = f(\tilde{\eta})$.
\end{lem}
\begin{proof}This is a simple corollary of Lemma
  \ref{FunctionalRepresentation} and the almost sure uniqueness of
  conditional expectations.
\end{proof}

Having defined $\cexpectationlong{\eta}{\xi}$ in terms of conditional
expectation of $\xi$ with respect the $\sigma$-algebra $\sigma(\eta)$
is natural to think of the latter as being the more general case.
However note that if we are given $\mathcal{F}$ and define $\eta :
(\Omega, \mathcal{A}) \to (\Omega, \mathcal{F})$ to be identity
function then in fact we see the two notions are equivalent.  In some
cases, authors (Kallenberg in particular) will refer to conditional
expectation with respect to a $\sigma$-algebra as the special case.
We'll try to avoid making statements about the relative level of
generality of the two ideas but will try to avoid using the notation
$\cexpectationlong{\eta}{\xi}$ when we know that $\eta$ is an
identity map.

\subsection{Conditional Independence}

\begin{defn}Given $\sigma$-algebras $\mathcal{F}$, $\mathcal{G}$ and
  $\mathcal{H}$ we say that $\mathcal{F}$ and $\mathcal{H}$ are
  \emph{conditionally independent given} $\mathcal{G}$ if for all $F
  \in \mathcal{F}$ and all $H \in \mathcal{H}$ we have 
\begin{align*}
\cprobability{\mathcal{G}}{F \cap H} &= \cprobability{\mathcal{G}}{F} \cprobability{\mathcal{G}}{H} 
\end{align*}
We often write $\cindependent{\mathcal{F}}{\mathcal{H}}{\mathcal{G}}$.
\end{defn}

A technical result that can be helpful when trying to prove
conditional independence is the following analogue of Lemma
\ref{IndependencePiSystem}
\begin{lem}\label{ConditionalIndependencePiSystem}Suppose we are given
  a $\sigma$-algebra $\mathcal{G}$ and two
  $\pi$-systems $\mathcal{S}$ and $\mathcal{T}$ in a probability space
  $(\Omega, \mathcal{A}, P)$ such that
  $\cprobability{\mathcal{G}}{A \cap B} = \cprobability{\mathcal{G}}{A} \cprobability{\mathcal{G}}{B}$ for all
  $A \in \mathcal{S}$ and $B \in \mathcal{T}$.  Then
  $\sigma(\mathcal{S})$ and $\sigma(\mathcal{T})$ are conditionally independent
  given $\mathcal{G}$.
\end{lem}
\begin{proof}
TODO: A straightforward extension of the proof of Lemma
\ref{IndependencePiSystem}.
\end{proof}

\begin{lem}\label{ConditionalIndependenceDoob}Given $\sigma$-algebras $\mathcal{F}$, $\mathcal{G}$ and
  $\mathcal{H}$, then
  $\cindependent{\mathcal{F}}{\mathcal{H}}{\mathcal{G}}$ if and only
  if for all $H \in \mathcal{H}$, we have
  $\cprobability{\mathcal{G}}{H} =
  \cprobability{\mathcal{F},\mathcal{G}}{H}$.
In particular, $\cindependent{\mathcal{F}}{\mathcal{H}}{\mathcal{G}}$
if and only if $\cindependent{\left (\mathcal{F}, \mathcal{G} \right )}{\mathcal{H}}{\mathcal{G}}$
\end{lem}
\begin{proof}
We first assume that
$\cindependent{\mathcal{F}}{\mathcal{H}}{\mathcal{G}}$.  Let $F \in
\mathcal{F}$ and $G \in \mathcal{G}$ and calculate
\begin{align*}
\expectation{\characteristic{F}\characteristic{G}\characteristic{H}}
&=
\expectation{\cexpectationlong{\mathcal{G}}{\characteristic{F}\characteristic{G}\characteristic{H}}}
\\
&=
\expectation{\characteristic{G}\cexpectationlong{\mathcal{G}}{\characteristic{F}\characteristic{H}}} \\
&=
\expectation{\characteristic{G}\cexpectationlong{\mathcal{G}}{\characteristic{F}}\cexpectationlong{\mathcal{G}}{\characteristic{H}}} \\
&=
\expectation{\cexpectationlong{\mathcal{G}}{\characteristic{F}\characteristic{G}}\cexpectationlong{\mathcal{G}}{\characteristic{H}}} \\
&=
\expectation{\characteristic{F}\characteristic{G}\cexpectationlong{\mathcal{G}}{\characteristic{H}}} \\
\end{align*}
Now note that set of all intersections $F \cap G$ is a $\pi$-system
that contains $\Omega$ and therefore by Lemma
\ref{ConditionalExpectationExtension} and the defining property of
conditional expectation we have
$\cexpectationlong{\mathcal{G}}{\characteristic{H}} =
\cexpectationlong{\mathcal{F},\mathcal{G}}{\characteristic{H}}$.

To show the converse, we take $F \in \mathcal{F}$ and $H \in
\mathcal{H}$ and
\begin{align*}
\cexpectationlong{\mathcal{G}}{\characteristic{F}\characteristic{H}} &=
\cexpectationlong{\mathcal{G}}{\cexpectationlong{\mathcal{F},
    \mathcal{G}}{\characteristic{F}\characteristic{H}}} \\
&= \cexpectationlong{\mathcal{G}}{\characteristic{F}\cexpectationlong{\mathcal{F},
    \mathcal{G}}{\characteristic{H}}} \\
&= \cexpectationlong{\mathcal{G}}{\characteristic{F}} \cexpectationlong{\mathcal{F},
    \mathcal{G}}{\characteristic{H}}\\
&= \cexpectationlong{\mathcal{G}}{\characteristic{F}} \cexpectationlong{\mathcal{G}}{\characteristic{H}}\\
\end{align*}
Now the last claim follows simply we have shown both
statements are equivalent to the fact that
$\cprobability{\mathcal{G}}{H} =
\cprobability{\mathcal{F},\mathcal{G}}{H}$ for all $H \in \mathcal{H}$.
\end{proof}

\begin{lem}\label{ConditionalIndependenceChainRule}Given $\sigma$-algebras $\mathcal{G}$, $\mathcal{H}$ and
  $\mathcal{F}_1, \mathcal{F}_2, \dots$, then
  $\cindependent{\mathcal{H}}{\left( \mathcal{F}_1, \mathcal{F}_2,
      \dots \right)}{\mathcal{G}}$ if and only if $\cindependent{\mathcal{H}}{ \mathcal{F}_{n+1}}{\left(  \mathcal{G},\mathcal{F}_1, \mathcal{F}_2,
      \dots , \mathcal{F}_n\right)}$ for all $n \geq 0$.
\end{lem}
\begin{proof}
If we assume the second property then we can conclude from Lemma
\ref{ConditionalIndependenceDoob} and an induction on $n \geq 0$ that
for every $H \in \mathcal{H}$,
\begin{align*}
\cprobability{\mathcal{G}}{H} &= 
\cprobability{\mathcal{G}, \mathcal{F}_1}{H} = 
\cprobability{\mathcal{G}, \mathcal{F}_1, \mathcal{F}_2}{H} = \cdots
\end{align*}
and therefore by another application of Lemma
\ref{ConditionalIndependenceDoob}, we know that
$\cindependent{\mathcal{H}}{\left (\mathcal{F}_1, \dots,
    \mathcal{F}_n\right)}{\mathcal{G}}$ for every $n \geq 1$.  Now
$\cup_n \sigma(\mathcal{F}_1, \dots, \mathcal{F}_n)$ is a $\pi$-system that generates
$\sigma(\mathcal{F}_1, \mathcal{F}_2, \dots)$ and therefore
application of Lemma \ref{ConditionalIndependencePiSystem} shows us
that  $\cindependent{\mathcal{H}}{\left( \mathcal{F}_1, \mathcal{F}_2,
      \dots \right)}{\mathcal{G}}$.

On the other hand, if we assume $\cindependent{\mathcal{H}}{\left( \mathcal{F}_1, \mathcal{F}_2,
      \dots \right)}{\mathcal{G}}$ then for any $n \geq 1$, and $H \in
  \mathcal{H}$, we apply the telescoping rule, Lemma
  \ref{ConditionalIndependenceDoob} and the pull out rule to get
\begin{align*}
\cprobability{\mathcal{G}, \mathcal{F}_1, \dots, \mathcal{F}_n}{H} &=
\cexpectationlong{\mathcal{G}, \mathcal{F}_1, \dots,
  \mathcal{F}_n}{\cprobability{\mathcal{G}, \mathcal{F}_1,
    \mathcal{F}_2, \dots }{H}} \\
&=\cexpectationlong{\mathcal{G}, \mathcal{F}_1, \dots,
  \mathcal{F}_n}{\cprobability{\mathcal{G}}{H}} \\
&=\cprobability{\mathcal{G}}{H}
\end{align*}
so in particular, for all $n \geq 0$,
\begin{align*}
\cprobability{\mathcal{G}, \mathcal{F}_1, \dots, \mathcal{F}_n}{H} &= \cprobability{\mathcal{G}, \mathcal{F}_1, \dots, \mathcal{F}_{n+1}}{H}
\end{align*}
Another application of Lemma \ref{ConditionalIndependenceDoob} shows
that $\cindependent{\mathcal{H}}{ \mathcal{F}_{n+1}}{\left(  \mathcal{G},\mathcal{F}_1, \mathcal{F}_2,
      \dots , \mathcal{F}_n\right)}$ for all $n \geq 0$.
\end{proof}

\begin{lem}Suppose $\cindependent{\mathcal{F}}{\mathcal{H}}{\mathcal{G}}$ and $\mathcal{A} \subset \mathcal{F}$,
  then $\cindependent{\mathcal{F}}{\mathcal{H}}{\mathcal{A},\mathcal{G}}$.
\end{lem}
\begin{proof}
By Lemma \ref{ConditionalIndependenceDoob}, we know for all $H \in
\mathcal{H}$, 
$\cprobability{\mathcal{G}}{H} =
\cprobability{\mathcal{F},\mathcal{G}}{H}$. On the other hand, since
$\mathcal{A} \subset \mathcal{F}$ we also have $\mathcal{G} \subset
\sigma(\mathcal{A}, \mathcal{G}) \subset \sigma(\mathcal{F},
\mathcal{G})$ and therefore we can conclude
$\cprobability{\mathcal{F},\mathcal{G}}{H} = \cprobability{\mathcal{A},\mathcal{G}}{H}$.
Since $\mathcal{A} \subset \mathcal{F}$ we know that $\sigma(\mathcal{A},
\mathcal{F}, \mathcal{G}) = \sigma(\mathcal{F}, \mathcal{G})$ and we
get $\cprobability{\mathcal{F},\mathcal{A},\mathcal{G}}{H} =
\cprobability{\mathcal{A},\mathcal{G}}{H}$.
Another application of Lemma \ref{ConditionalIndependenceDoob} tells
us that $\cindependent{\mathcal{F}}{\mathcal{H}}{\mathcal{A},\mathcal{G}}$.
\end{proof}

\subsection{Conditional Distributions and Disintegration}
Now for a more subtle concept in conditioning.  Consider a random
element $\xi$ in a measurable space $(S,\mathcal{S})$ and a random
element $\eta$ in a measurable space $(T,\mathcal{T})$.  We'd like to
make sense of the conditional distribution of $\xi$ given a value of
$\eta$.  Two things should occur to us.  First, such an object sounds
like it should a mapping from $T$ to a space of measures on $S$.  Second, we
expect that we'll actually define this object in terms of the
conditional expectation and that it will likely wind up as an
$\eta$-measurable random measure on $\Omega$.  A third thing might also
occur to us: namely these two representations are equivalent.  As it
turns out, due to the fact that conditional expectations are only
defined up to almost sure equivalence, this last supposition is not true and we often must make
additional assumptions to arrange for the existence of the mapping of
$T$ to the space of measures on $S$. 

\subsubsection{Probability Kernels}
Before jumping into the development of conditional distributions
proper we need to step back a bit and make sure we've laid a proper
foundation for the discussion.  We wrote heuristically above about a
mapping to a space of measures.  This is a concept that will come up in a variety of contexts from this point
on and we glossed over the fact that we want such a mapping to have
measurability properties.  There are a couple of equivalent ways of
formulating the notion of a measurable family of measures;  we
explore these now.
To formalize, we have the following definition
\begin{defn}
Let $(S, \mathcal{S})$ and $(T, \mathcal{T})$ be measurable spaces.  A
\emph{probability kernel} from $S$ to $T$ is a function $\mu : S
\times \mathcal{T} \to [0,1]$ 
such that for every fixed $s \in S$, $\mu(s, \cdot) : \mathcal{T} \to
[0,1]$ is a probability measure and for every fixed $A \in
\mathcal{T}$, $\mu(\cdot, A) : S \to [0,1]$ is Borel measurable.
\end{defn}

It is useful to have some alternative characterizations of the
measurability properites of kernels but before we can state them we
need another definition.
\begin{defn}Given a measurable space $(S, \mathcal{S})$, then
  $\mathcal{P}(S)$ is the space of probability measures on $S$ with
  the $\sigma$-algebra generated by all sets of the form $\lbrace \mu
  \mid \mu(A) \in B \rbrace$ for $A \in \mathcal{S}$ and $B \in
  \mathcal{B}([0,1])$.  Alternatively, for each $A \in \mathcal{S}$,
  define the evaluation map $\pi_A : \mathcal{P}(S) \to [0,1]$ by
  $\pi_A(\mu) = \mu(A)$ and then take the $\sigma$-algebra generated
  by all of the evaluation maps.
\end{defn}

As promised, we have the following lemma that gives a couple of
alternative characterizations of the measurability condition of a
kernel; including the obligatory monotone class argument.
\begin{lem}\label{KernelMeasurability}Let $(S, \mathcal{S})$ and $(T, \mathcal{T})$ be measurable
  spaces and $\mu_s$ be a family of probability measures on $T$.  Then
  the following are equivalent
\begin{itemize}
\item[(i)]$\mu : S \times \mathcal{T} \to [0,1]$ is a probability kernel
\item[(ii)]$\mu : S \to \mathcal{P}(T)$ is measurable
\item[(iii)]$\mu(s, A) : S \to [0,1]$ is Borel measurable for every $A$
  belonging to a $\pi$-system that generates $\mathcal{S}$.
\end{itemize}
\end{lem}
\begin{proof}
First suppose that $\mu$ is a kernel, $A \in \mathcal{T}$
and $B$ is a Borel
measurable subset of $[0,1]$.  Then 
\begin{align*}
\mu^{-1}(\lbrace \nu \mid \nu(A)
\in B \rbrace) &= \lbrace s \in S \mid \mu(s,A) \in B \rbrace =
\mu(\cdot, A)^{-1}(B)
\end{align*}
which is measurable by the kernel property.  Since sets of the form $\lbrace \nu \mid \nu(A)
\in B \rbrace$ generate the $\sigma$-algebra on $\mathcal{P}(T)$ we
see that $\mu$ is measurable by Lemma \ref{MeasurableByGeneratingSet}.

To see that (ii) implies (i), observe that for a fixed $A \in
\mathcal{T}$ and let $\pi_A(\nu) = \nu(A)$ be the evaluation map.  By
construction the $\pi_A$ are measurable.  For such a fixed $A$, we see
that $\mu(s, A) = \pi_A(\mu)$ therefore as a composition of measurable
maps we see that $\mu(s,A)$ is $\mathcal{S}$-measurable (Lemma
\ref{CompositionOfMeasurable}).

The implication (i) implies (iii) is immediate.  If we assume (iii)
then we derive (i) by a monotone class argument.  By Theorem
\ref{MonotoneClassTheorem} it suffices to show that $\mathcal{C} =
\lbrace A \mid \mu(s, A) : S \to [0,1] \text { is measurable}\rbrace$
is a $\lambda$-system.  If $A \subset B$ with $A,B \in \mathcal{C}$
then $\mu(s, B \setminus A) = \mu(s, B) - \mu(s,A)$ is measurable.  If
$A_1 \subset A_2 \subset \cdots$ with $A_n \in \mathcal{C}$ then by
continuity of measure (Lemma \ref{ContinuityOfMeasure}) applied
pointwise in $s$, we see $\mu(s, \cup_n A_n) = \lim_n \mu(s, A_n)$
which shows measurability by Lemma \ref{LimitsOfMeasurable}.
\end{proof}

There is a useful generalization of the product measure construction
involving kernels.  It is a type of ``twisted'' product construction.
\begin{defn}Let $\mu : S \times \mathcal{T} \to [0,1]$ be a
  probability kernel from $S$ to $T$ and $\nu : S \times T \times
  \mathcal{U} \to [0,1]$ be a probability kernel from $S \times T$ to
  $U$, we then define $\mu \otimes \nu : S \times \mathcal{T} \otimes
  \mathcal{U} \to [0,1]$ by
\begin{align*}
\mu \otimes \nu(s, A) &= \int \int \characteristic{A}(t,u) \,
d\nu(s,t,du) \, d\mu(s, dt)
\end{align*}
\end{defn}

The fact that this construction defines a probability kernel is the
content of the next Lemma.
\begin{lem}Suppose $\mu : S \times \mathcal{T} \to [0,1]$ is a
  probability kernel from $S$ to $T$ and $\nu : S \times T \times
  \mathcal{U} \to [0,1]$ be a probability kernel from $S \times T$ to
  $U$.  Let $f : S \times T \to \reals_+$ and $g
  : S \times T \times \to U$  be measurable then 
\begin{itemize}
\item[(i)] $\int f (s, t) \, d\mu(s,dt)$ is a measurable function of $s \in S$.
\item[(ii)] $\mu_s \circ (g(s, \cdot))^{-1}$ is a kernel from $S$ to $U$.
\item[(iii)] $\mu \otimes \nu$ is a kernel from $S$ to $T \times U$.
\end{itemize}
\end{lem}
\begin{proof}
To see (i), we apply the standard machinery.  First consider $f(s,t) = \characteristic{A\times B}(s,t)$
for $A \in \mathcal{S}$ and $B \in \mathcal{T}$.  In this case, 
\begin{align*}
\int \characteristic{A \times B} (s, t) \, d\mu(s,dt) &=
\characteristic{A} (s)\int \characteristic{B} (t) \, d\mu(s,dt)
=\characteristic{A} (s) \mu(s,B)
\end{align*}
which is $\mathcal{S}$-measurable by measurability of $A$ and the fact
that $\mu$ is a kernel.  We extend to the case of general characteristic
functions by observing that products $A \times B$ are a generating
$\pi$-system for the $\sigma$-algebra $\mathcal{S} \otimes
\mathcal{T}$.  Additionally we must show that $\mathcal{C} = \lbrace C
\in \mathcal{S} \otimes \mathcal{T} \mid \int \characteristic{A \times
  B} (s, t) \, d\mu(s,dt) \text { is measurable} \rbrace$ is a
$\lambda$-system.  To see this first assume that $A \subset B$ with
$A,B \in \mathcal{C}$.  Then by linearity of integral, $\int
\characteristic{B \setminus A} (s, t) \, d\mu(s,dt) = \int
\characteristic{B} (s, t) \, d\mu(s,dt) - \int
\characteristic{A} (s, t) \, d\mu(s,dt)$ which shows $B \setminus A \in
\mathcal{C}$.  Secondly if $A_1 \subset A_2 \subset \cdots$ is a chain
in $\mathcal{C}$ then by Monotone Convergence applied pointwise in
$s$, we have $\int \characteristic{\cup_n A_n} (s, t) \, d\mu(s,dt) =
\lim_{n\to \infty} \int \characteristic{A_n} (s, t) \, d\mu(s,dt)$
which shows $\cup_n A_n \in \mathcal{C}$ because limits of measurable functions are measurable
(Lemma \ref{LimitsOfMeasurable}).  Now an application of Theorem
\ref{MonotoneClassTheorem} shows the result.

By $\mathcal{S}$-measurability for characteristic functions and
linearity of integral, we see that $\int f (s, t) \, d\mu(s,dt)$ is
$\mathcal{S}$-measurable for simple functions and by definition of
integral we see that for any positive measurable $f$ with an
approximation by simple functions $f_n \uparrow f$ we note that for
each fixed $s$, $f_n$ are simple functions of $t$ alone so $\int f (s,
t) \, d\mu(s,dt) = \lim_{n} \int f_n (s,
t) \, d\mu(s,dt)$ showing $\mathcal{S}$-measurability by another
application of Lemma \ref{LimitsOfMeasurable}.  Lastly extending to
general integrable $f$, write $f = f_+ - f_-$ and use linearity of
integral.

Having proven (i) we derive (ii) and (iii) from it.  To see (ii)
assume that $A \in \mathcal{U}$ and note that for fixed $s$, if we
denote the section of $g$ at $s$ by $g_s : T \to U$ then it is
elementary that $\characteristic{g_s^{-1}(A)}(t) =
\characteristic{g^{-1}(A)}(s,t)$ and thus
\begin{align*}
\mu_s \circ (g(s, \cdot))^{-1}(A) &= \mu(s, g^{-1}(s, A)) = \mu(s, g^{-1}(A)) 
\end{align*}
which we have shown is $\mathcal{S}$-measurable in (i).

To see (iii), pick $A \in \mathcal{T} \otimes \mathcal{U}$ and recall
that by definition
\begin{align*}
\mu \otimes \nu (A) (s) &= \int \int \characteristic{A}(t,u) \,
d\nu(s,t,du) \, d\mu(s, dt)
\end{align*}
We know that $\characteristic{A}(t,u)$ is $\mathcal{T} \otimes
\mathcal{U}$-measurable hence also $\mathcal{S} \otimes \mathcal{T} \otimes
\mathcal{U}$-measurable.  Therefore we can apply (i) to conclude that $\int \characteristic{A}(t,u) \,
d\nu(s,t,du)$ is $\mathcal{S} \otimes \mathcal{T}$-measurable.  Now
apply (i) again to conclude that $\mu \otimes \nu (A) (s)$ is $\mathcal{S}$-measurable.
\end{proof}

\begin{thm}\label{ExistenceConditionalDistribution}Let $(S, \mathcal{S})$ be a Borel space and
  $(T, \mathcal{T})$ be an arbitrary measuable space.  Let $\xi$ be a
  random element in $S$ and $\eta$ be a random element in $T$.  There
  is exists a probability kernel $\mu : T \times \mathcal{S} \to
  \reals$ such that $\cprobability{\eta }{\xi \in A}(\omega) =
  \mu(\eta(\omega), A)$ for all $A \in \mathcal{S}$ and $\omega \in
  \Omega$.  Furthermore, if $\tilde{\mu}$ is another probability
  kernel satisfying this property then $\mu = \tilde{\mu}$ almost
  surely with respect to $\mathcal{L}(\eta)$.
\end{thm}
\begin{proof}
TODO:  Reduce to the case of $S = \reals$ and use density of rationals
and properties of distribution functions to create a regular version.
\end{proof}

\begin{thm}\label{Disintegration}Let $(S, \mathcal{S})$ and $(T, \mathcal{T})$ be measurable
  spaces and let $\xi$ be a random element in $S$ and $\eta$ be a
  random element in $T$.  Suppose $\cprobability{\mathcal{F}}{\xi \in
    \cdot}$ has a regular version $\nu : \Omega \times \mathcal{S} \to
  \reals$ and suppose $\eta$ is
  $\mathcal{F}$-measurable.  Suppose $f : S \times T \to \reals$ is
  measurable with either $f \geq 0$ or $\expectation{\abs{f(\xi, \eta)}}< \infty$.  Then 
\begin{align*}
\expectation{f(\xi, \eta)} &= \expectation{\int f(s, \eta) \, d\nu(s)}
\end{align*}
and moreover
\begin{align*}
\cexpectationlong{\mathcal{F}}{f(\xi, \eta)} &= \int f(s, \eta) \, d\nu(s)
\text{ a.s.}
\end{align*}
\end{thm}
\begin{proof}
The proof is an application of the standard machinery.
To start with we assume that $f = \characteristic{A \times B}$ for $A
\in \mathcal{S}$ and $B \in \mathcal{T}$.  
Then 
\begin{align*}
\expectation{f(\xi, \eta)}  &= 
\expectation{\characteristic{A}(\xi) \characteristic{B}(\eta)} \\
&= \expectation{\cexpectationlong{\mathcal{F}}{\characteristic{A}(\xi)} \characteristic{B}(\eta) } \\
&= \expectation{\nu(A) \characteristic{B}(\eta)} \\
&= \expectation{\int \characteristic{A}(s) \characteristic{B}(\eta)
  d\nu(s) } \\
&= \expectation{\int f(s, \eta)  d\nu(s) } \\
\end{align*}

Now we extend to the set of all $C \in \mathcal{S}\otimes
\mathcal{T}$ by using a Monotone Class Argument (Theorem
\ref{MonotoneClassTheorem}).  Let $\mathcal{C} = \lbrace C \in \mathcal{S}\otimes
\mathcal{T} \mid \expectation{\characteristic{C}(\xi, \eta)} = \expectation{\int
  \characteristic{C}(s, \eta)  d\nu(s) } \rbrace$
Since the set of all $A \times B$ is a $\pi$-system
containing $S \times T$ it suffices to show that $\mathcal{C}$ is a $\lambda$-system.
Suppose $C, D \in \mathcal{D}$ and $C \subset D$; then we see $D
\setminus C \in \mathcal{C}$ by
noting $\characteristic{D \setminus C} = \characteristic{D} -
\characteristic{C}$ and applying linearity of expectation and
integral.  If we assume $C_1 \subset C_2 \subset \cdots$ with $C_n \in
\mathcal{C}$, then
$\characteristic{\cup_n C_n} = \lim_{n \to \infty}
\characteristic{C_n}$ and the Monotone Convergence Theorem implies
$\expectation{\characteristic{\cup_n C_n}(\xi, \eta)} = \lim_{n \to
  \infty}\expectation{\characteristic{C_n}(\xi, \eta)}$.  Similarly
for fixed $\omega \in \Omega$, $\int \characteristic{\cup_n C_n}(s,
\eta) \, d\nu(s) = \lim_{n \to \infty} \int \characteristic{C_n}(s,
\eta) \, d\nu(s)$, moreover monotonicity of integral implies that $\int \characteristic{C_n}(s,
\eta) \, d\nu(s)$
is increasing in $n$.  Therefore we may apply Monotone Convergence a
second time to conclude that
\begin{align*}
\expectation{\int \characteristic{\cup_n C_n}(s,\eta) \, d\nu(s) } &=
\lim_{n\to \infty}\expectation{\int \characteristic{C_n}(s,\eta) \, d\nu(s) } 
\end{align*}
Therefore we see that $\cup_n C_n \in
\mathcal{C}$.  

Extending the result to simple functions is trivial since both sides
are linear in $f$.

Now we suppose that $f : S \times T \in \reals$ is positive
measurable.  We pick an approximation of $f$ by an increasing sequence
of positive simple functions $0 \leq f_n \uparrow f$.  Now $f_n(\xi,
\eta)$ is an increasing sequence of positive simple functions with
$\lim_{n \to \infty} f_n(\xi, \eta) = f(\xi, \eta)$ and therefore by
definition of expectation, $\expectation{f(\xi, \eta)} = \lim_{n \to
  \infty} \expectation{f_n(\xi, \eta)}$.  Similarly for fixed $\omega
\in \Omega$ we have $f_n(s, \eta)$ are positive simple functions increasing to
$f(s, \eta)$ and therefore $\int f(s, \eta) \, d\nu(s) = \lim_{n \to
  \infty} \int f_n(s, \eta) \, d\nu(s)$.  Monotonicity of integral shows
that  the sequence $\int f_n(s, \eta) \, d\nu(s)$ is positive and
increasing and therefore we may apply Monotone Convergence and the
fact that result holds for the $f_n$ to show
that 
\begin{align*}
\expectation{\int f(s, \eta) \, d\nu(s)} &= \lim_{n \to
  \infty} \expectation{\int f_n(s, \eta) \, d\nu(s)} = \lim_{n \to \infty} \expectation{f_n(\xi, \eta)} = \expectation{f(\xi, \eta)}
\end{align*}  
Therefore the
result for positive measurable $f$.

Lastly for general integrable $f$, we know by the result for positive
$f$ that
\begin{align*}
\expectation{\int \abs{f(s, \eta)} \, d\nu(s)} &=
\expectation{\abs{f(\xi, \eta)}} < \infty
\end{align*}
Which shows us that $\int \abs{f(s, \eta)} \, d\nu(s) < \infty$ almost
surely.  Then we can write $f = f_+ - f_-$ and use the the result for
postive $f$ and linearity.

The last thing to do is to extend the result to the case of
conditional expectations.  Let $f : S \times T \to \reals_+$ be
positive and let $A \in \mathcal{F}$.  Consider $(\eta,
\characteristic{A})$ as a random element of $T \times \lbrace 0,1\rbrace$.  Note that this
random element is $\mathcal{F}$-measurable since $\eta$ is and $A \in
\mathcal{F}$.  Therefore we can apply the case just proven to the
function $\tilde{f} : S \times T \times \lbrace 0,1 \rbrace \to
\reals_+$ given by $\tilde{f}(s,t,u) = u f(s,t)$ and the elements $\xi$
and $(\eta, \characteristic{A})$ to get
\begin{align*}
\expectation{f(\xi, \eta) ; A} &= \expectation {\int f(s, \eta)
  \characteristic{A} \, d\nu(s)} = \expectation {\int f(s, \eta)
  \, d\nu(s) ; A}
\end{align*}
which shows that $\cexpectationlong{\mathcal{F}}{f(\xi,\eta)} = \int f(s, \eta)
  \, d\nu(s)$ a.s. for $f \geq 0$.  The case of integrable $f$ follows as
  usual by taking differences.
\end{proof}

Special case of random vectors with densities.  Suppose we are given
$\xi : \Omega \to \reals^m$ and $\eta: \Omega \to \reals^n$ such that
$(\xi,\eta)$ has density $f$ on $\reals^{m+n}$.  Then $\xi$ and $\eta$ have
densities $f_{\xi}$ and $f_\eta$ called the marginal densities and we
get a conditional densities $f(x,y)/f_\xi(x)$ and $f(x,y)/f_\eta(y)$.
TODO: Tie this back to conditional distributions as defined in the
general case (this is an exercise in Kallenberg for example).

We've seen that given a specified distribution we can always find a
random variable with that specfied distribution.  Moreover, we know
that if we allow ourselves to to extend the probability space then we
can construct such a random variable to be independent of any existing
random elements (or $\sigma$-algebras).  We now turn our attention to
the analogous problem space for conditional distributions.  The
simplest such result shows that given a random element and a
prescribed probability kernel we can always find a second random
element whose conditional distribution is the kernel.
\begin{lem}Let $(S, \mathcal{S})$ and $(T, \mathcal{T})$ be measurable
  spaces, $\mu : T \times \mathcal{S} \to \reals$ be a
  probability kernel and $\eta$ be a random element in $T$.  There
  exists an extension $\hat{\Omega}$ and a random element $\xi$ in
  $\hat{\Omega} \to S$
  such that $\cprobability{\eta}{\xi \in \cdot} = \mu(\eta, \cdot)$
  a.s.   and $\cindependent{\xi}{\zeta}{\eta}$ for every random element
  $\zeta$ defined on $\Omega$.
\end{lem}
\begin{proof}
The appropriate construction is thrust upon us by Theorem
\ref{Disintegration}.  Note that if we succeed in constructing $\xi$
then that result tells how to compute expectations on $\hat{\Omega}$.
Following that lead, define
$(\hat{\Omega}, \hat{\mathcal{A}}) = (S \times \Omega,
\mathcal{S} \otimes \mathcal{A})$.  Define the probability measure 
\begin{align*}
\hat{P}(A) &= \expectation{\int \characteristic{A}(s,\omega) \,
  d\mu(\eta, s)}
\end{align*}
Note that $\hat{P}$ is an extension since for $A \in \mathcal{A}$, 
\begin{align*}
\hat{P}(S \times A) &= \expectation{\int \characteristic{S}(s)\characteristic{A}(\omega) \,
  d\mu(\eta, s)} = \expectation{\characteristic{A}(\omega)} = P(A)
\end{align*}

Now define $\xi(s, \omega) = s$ and note that for $A \in \mathcal{S}$
and $B \in \mathcal{A}$, 
\begin{align*}
\hat{P}(\xi \in A ; B) &=  \expectation{\int \characteristic{A}(s)\characteristic{B}(\omega) \,
  d\mu(\eta, s)} = \expectation{\mu(\eta, A) ; B}
\end{align*}
which shows $\cprobability{\mathcal{A}}{\xi \in A}= \mu(\eta, A)$ a.s. 
by the defining property of conditional expectation (note that since
$\mu(\eta, A)$ and $\characteristic{B}$ are both
$\mathcal{A}$-measurable, their expectation with respect to $P$ is the
same as their expectation with respect to $\hat{P}$).  In particular,
since we know that $\mu(\eta, A)$ is $\eta$-measurable we also know
that $\cprobability{\mathcal{A}}{\xi \in A}=\cprobability{\eta}{\xi \in A} = \mu(\eta, A)$.

This last observation also shows $\cindependent{\xi}{\mathcal{A}}{\eta}$ by an application of Lemma \ref{ConditionalIndependenceDoob}.
\end{proof}

The next result is closely related but uses a different construction
that shows how one may use a single uniform randomization variable.

\begin{lem}Let $(S, \mathcal{S})$ be a Borel space and $(T, \mathcal{T})$ be 
  a general measurable space.  Let $\xi$ be a random element in $S$
  and let $\eta, \tilde{\eta}$ be random elements in $T$.  There
  exists a measurable function $f : T \times [0,1] \to S$ such that if
  $\vartheta$ is a uniform random variable with
  $\cindependent{\vartheta}{\tilde{\eta}}{}$ and we define
  $\tilde{\xi} = f(\tilde{\eta}, \vartheta)$ then $(\xi, \eta) \eqdist
  (\tilde{\xi}, \tilde{\eta})$.
\end{lem}
\begin{proof}
TODO: I believe in some applications of this Lemma it can be
convenient to assume that $\xi, \eta$ and $\tilde{\xi}, \tilde{\eta}$
live on different probability spaces.  Validate this fact and restate
to make it clear that this is true.

First assume $S = \reals$.  By Theorem
\ref{ExistenceConditionalDistribution} we have a probability kernel
$\mu : T \times \mathcal{S} \to \reals$ such that
$\cprobability{\eta}{\xi \in \cdot} = \mu(\eta, \cdot)$.  

Furthermore, we know by Lemma TODO:???  we can find measurable $f : T
\times [0,1] \to S$ such that the distribution of $f(t, \vartheta)$ is
$\mu(t)$.  Now define $\tilde{\xi} = f(\tilde{\eta}, \vartheta)$,
assume we have a measurable $g : S \times T \to \reals_+$ and
calculate
\begin{align*}
\expectation{g(\tilde{\xi}, \tilde{\eta})} &=
\expectation{g(f(\tilde{\eta}, \vartheta), \tilde{\eta})} \\
&= \expectation{\int_0^1 g(f(\tilde{\eta},x), \tilde{\eta}) \, dx} & &
\text{by independence of $\tilde{\eta}$ and $\vartheta$}\\
&= \expectation{\int_0^1 g(f(\eta,x), \eta) \, dx} & &
\text{by $\eta \eqdist \tilde{\eta}$}\\
&= \expectation{\int g(s, \eta) \, d\mu(\eta,s)} & & \text{by
  Lemma \ref{ExpectationRule}} \\
&= \expectation{g(\xi, \eta)} & & \text{by Theorem \ref{Disintegration}}
\end{align*}
\end{proof}

\begin{thm}[Jensen's Inequality]\label{JensenConditionalExpectation}Let $\xi$ be a random vector and $\mathcal{F}$ be a
  $\sigma$-algebra.  If $\varphi$ is a convex function then
  $\varphi(\cexpectationlong{\mathcal{F}}{\xi}) \leq
    \cexpectationlong{\mathcal{F}}{\varphi(\xi)}$ a.s.
If $\varphi$ is strictly convex then $\varphi(\cexpectationlong{\mathcal{F}}{\xi}) =
    \cexpectationlong{\mathcal{F}}{\varphi(\xi)}$ if and only if $\xi =
      \cexpectationlong{\mathcal{F}}{\xi}$ a.s.
\end{thm}
\begin{proof}
TODO:
\end{proof}
