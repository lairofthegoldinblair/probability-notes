\chapter{Characteristic Functions And Central Limit Theorem}
In this section we study the weak convergence of random vectors more carefully.
Our first goal is to develop just enough of the theory of
characteristic functions in order to prove the classical Central Limit Theorem.
After that we delve more deeply into theory of characteristic
functions.  

The motivation for the theory we are about to develop is the intuition
that most of the behavior of a probability distribution on $\reals$ is
captured by its moments.  If one could put the information about all
of the distribution's moments into a single package simulataneously
then the resulting package might characterize the probability
distribution in a useful way.  A initial naive approach might be to
use a \emph{generating function} methodology.  For example, one might
try to define a function $f(t) = \sum_{n=0}^\infty M_n t^n$ where
$M_n$ denotes the $n^{th}$ moment.  Alas, such a approach fails rather
miserably as it is a very rare thing for moments to decrease quickly
enough for the formal power series for $f(t)$ to ever converge and make a useful
function object.  A better approach is to scale the moments to give
the series a chance to converge.  For example, being a bit sloppy we
could write
\begin{align*}
f(t) &= \int e^{tx} dP = \sum_{n=0}^\infty \frac{M_n}{n!} t^n
\end{align*}
This idea has a lot more merit and can be used effectively but it has
the distinct disadvantage that it only works for distributions that
have moments of all orders.

The wonderful idea that we will be exploring in this chapter is that
by passing into the domain of complex numbers we get a
characterization of the distribution that is always defined and (at
least conceptually) captures all moments in a generating function.
Specifically, we define 
\begin{align*}
f(t) &= \int e^{itx} dP 
\end{align*}
which is the \emph{Fourier Transform} of the probability distribution
and we get an object that uniquely determines the distribution and can
often be much easier to work with.  In particular we we will see that
convergence in distribution is described as pointwise convergence of
characteristic functions and through that connection we will get
another proof of the Central Limit Theorem.

In this section we start to make use of integrals of complex valued
measurable functions.  Let's establish the basic definitions and facts
that we require.
\begin{defn}A function $f : (\Omega, \mathcal{A}, \mu) \to \complexes$
  is measurable if and only $f = h + i g$ where $h,g : (\Omega,
  \mathcal{A}, \mu) \to \reals$ are measurable.  Equivalently,
  $\complexes$ is given the Borel $\sigma$-algebra.
\begin{itemize}
\item[(i)] If $\mu(A) < \infty$, then $\abs{\int f \, d\mu} \leq \int
  \abs{f} \, d\mu$.
\end{itemize}
\end{defn}
\begin{proof}
By the triangle inequality for the complex norm, we know that given
any two $z,w \in \complexes$ and $t \in [0,1]$, $\abs{(1-t) z + t w}
\leq (1-t) \abs{z} + t \abs{w}$ and therefore the complex norm is
convex.  Then by Jensen's Inequality (Theorem \ref{Jensen}, $\abs{\int f \, d\mu} \leq \int
\abs{f} \, d\mu$.
\end{proof}

\begin{defn}Let $\mu$ be a probability measure on $\reals^n$. Its
  \emph{Fourier Transform} is denoted $\hat{\mu}$ and is the complex function on
  $\reals^n$ defined by 
\begin{align*}
\hat{\mu}(u) &= \int e^{i\langle u,x \rangle} d \mu(x) = \int \cos(\langle u,x \rangle)
\, d \mu(x) + i \int \sin(\langle u,x \rangle) \, d\mu(x)
\end{align*}
\end{defn}
The first order of business is to establish the basic properties of
the Fourier Transform of a probability measure including the fact that
the definition makes sense.
\begin{thm} \label{CharacteristicFunctionBoundedAndContinuous}Let $\mu$ be a probability measure, then $\hat{\mu}$
  exists and is a bounded uniformly continuous function with
  $\hat{\mu}(0) = 1$.
\end{thm}
\begin{proof}
To see that $\hat{\mu}$ exists, use the representation
\begin{align*}
\hat{\mu}(u) &= \int \cos(\langle  u,x \rangle) \, d\mu(x)+ i \int \sin(\langle
  u,x \rangle) \, d\mu(x)
\end{align*} and use the facts that $\abs{\cos \theta } \leq 1$
and $\abs {\sin \theta} \leq 1$ to conclude that both integrals are
bounded.

To see that $\hat{\mu}(0) = 1$, simply calculate
\begin{align*}
\hat{\mu}(0) &= \int \cos(\langle  0,x \rangle) \, d\mu(x)+ i \int \sin(\langle
  0,x \rangle) \, d\mu(x) = \int d\mu(x) = 1
\end{align*} 
In a similar way, boundedness is a simple calculation
\begin{align*}
\abs{\hat{\mu}(u)} &\leq \int \abs{e^{i \langle u,x \rangle}} \, d\mu(x) =
\int \, d\mu(x) = 1
\end{align*}
Lastly, to prove uniform continuity, first note that for any $u,v \in
\reals^n$, we have
\begin{align*}
\abs{e^{i \langle  u,x \rangle} - e^{i \langle  v,x \rangle} }^2 &=
\abs{e^{i \langle  u-v,x \rangle} - 1}^2 \\
&= (\cos(\langle  u-v,x
\rangle) - 1)^2 + \sin^2(\langle  u-v,x \rangle) \\
&= 2(1 - \cos(\langle  u-v,x
\rangle) \\
&\leq \langle  u-v,x\rangle^2 & & \text{by Lemma
  \ref{BasicExponentialInequalities}} \\
&\leq \norm{u-v}^2_2 \norm{x}^2_2 & & \text{by Cauchy Schwartz}
\end{align*}
On the other hand, it is clear from the triangle inequality that
\begin{align*}
\abs{e^{i \langle  u,x \rangle} - e^{i \langle  u,x \rangle} } \leq
\abs{e^{i \langle  u,x \rangle}} + \abs{e^{i \langle  u,x \rangle}  }
\leq 2
\end{align*}
and therefore we have the bound $\abs{e^{i \langle  u,x \rangle} -
  e^{i \langle  u,x \rangle} } \leq \norm{u-v}_2 \norm{x}_2 \wedge
2$.  
Note that pointwise in $x \in \reals^n$, $\lim_{n \to \infty}
\frac{1}{n} \norm{x}_2 \wedge 2 = 0$ and trivially $\frac{1}{n}
\norm{x}_2 \wedge 2 \leq 2$ so Dominated Convergence shows
that $\lim_{n \to \infty} \int \frac{1}{n} \norm{x}_2 \wedge
  2 \, d\mu(x) = 0$.  Given an $\epsilon >0$, pick $N > 0$ such that
$\int \frac{1}{N} \norm{x}_2 \wedge
  2 \, d\mu(x) < \epsilon$ then for  $\norm{u-v}_2
\leq \frac{1}{N}$,
\begin{align*}
\abs{\hat{\mu}(u) - \hat{\mu}(v)} &\leq \int \abs{e^{i \langle  u,x \rangle} -
  e^{i \langle  u,x \rangle}} \, d\mu(x) \\
&\leq \int \norm{u-v}_2 \norm{x}_2 \wedge
2  \, d\mu(x) \\
&\leq \int \frac{1}{N} \norm{x}_2 \wedge 
2  \, d\mu(x) < \epsilon
\end{align*}
proving uniform continuity.
\end{proof}
\begin{defn}Let $\xi$ be an $\reals^n$-valued random variable. Its
  characteristic function is denoted $\varphi_\xi$ and is the complex
  valued function on
  $\reals^n$ defined by 
\begin{align*}
\varphi_\xi(u) &= \expectation{e^{i\langle u,\xi \rangle}} \\
&= \int e^{i\langle u,x \rangle} \distribution{\xi}(dx) =
\hat{\distribution{\xi}}(u) 
\end{align*}
\end{defn}

We motivated the definition of the characteristic function by
considering how we might encode information about the moments of a
probability measure.  To make sure that we've succeeded we need to
show how to extract moments from the characteristic function.  To see
what we should expect, let's specialize to $\reals$ and suppose that
we can write out a power series:
\begin{align*}
\hat{\mu}(t) &= \int e^{itx} \, d\mu = \sum_{n=0}^\infty \frac{i^n
  M_n}{n!} t^n
\end{align*}
Still working formally, we see that we can differentiate the series with respect
to $t$ to isolate each individual moment $M_n$
\begin{align*}
\frac{d^n}{dt^n} \hat{\mu}(0) &= i^n M_n
\end{align*}
The above computation was rather formal and we won't try to make the
entire thing rigorous (specifically we won't consider the series
expansions).  What we make rigorous in the next Theorem is the connection between
moments of $\mu$ and derivatives of the characteristic function.
\begin{thm}\label{MomentsAndDerivatives} Let $\mu$ be a probability measure on $\reals^n$ such that
  $f(x) = \left| x \right| ^ m$ is integrable with respect to $\mu$.
  Then $\hat{\mu}$ is has continuous partial derivatives up to order
  $m$ and 
\begin{equation*}
 \frac{\partial^m \hat{\mu}} {\partial x_{j_1} \ldots \partial x_{j_m}}(u) =
 i^m \int x_{j_1} \ldots x_{j_m}e^{i\langle u,x \rangle} \mu(dx)
\end{equation*}
\end{thm}
\begin{proof}First we proceed with $m=1$.  Pick $1 \leq j \leq n$ and let $v \in \reals^n$ be the
  vector with $v_j = 1$ and $v_i = 0$ for $i \neq j$.  Then for $u \in
  \reals^n$ and $t > 0$,
\begin{align*}
\frac{\hat{\mu}(u + t v_j) - \hat{\mu}(u)}{t} &= \frac{1}{t} \int e^{i
  \langle u + t v_j, x \rangle} - e^{i\langle u,x \rangle} d\mu(x) \\
&= \frac{1}{t} \int e^{i\langle u,x \rangle} \left ( e^{i  t x_j} - 1 \right ) d \mu(x)
\end{align*}
But note that 
\begin{align*}
\abs{\frac{1}{t} e^{i\langle u,x \rangle} \left ( e^{i t x_j} - 1
  \right )}^2 &= \abs{\frac{ e^{i  t x_j} - 1 }{t} }^2 \\
&= \frac{\cos^2(t
  x_j) - 2 \cos(t x_j) + 1 + \sin^2(t x_j)}{t^2} \\
&= 2\left ( \frac{1 - \cos(t x_j)}{t^2} \right ) \\
&\leq x_j^2 & & \text{by Lemma \ref{BasicExponentialInequalities}}
\end{align*}
But $\abs{x_j}$ is assumed to be integrable hence we can apply the
Dominated Convergence Theorem to see 
\begin{align*}
\frac{\partial}{\partial x_j}  \int e^{i\langle u, x \rangle} d \mu(x)
&= \lim_{t \to 0} \frac{1}{t} \int e^{i\langle u + t v_j, x \rangle} -
e^{i\langle u, x \rangle} d \mu(x) \\
&=  \int \lim_{t \to 0}  \frac{e^{i\langle u + t v_j, x \rangle} -
e^{i\langle u, x \rangle}}{t} d \mu(x) \\
&= i \int x_j e^{i\langle u, x \rangle} d \mu (x)
\end{align*}

Continuity of the derivative follows from the formula we just proved.
Suppose that $u_n \to u \in \reals^n$.  Then we have shown that
\begin{align*}
\frac{\partial}{\partial x_j} \hat{\mu} (u_n)= i \int x_j e^{i\langle u_n, x \rangle} d \mu (x)
\end{align*}
and we have the bound on the integrands $\abs{x_j e^{i\langle u_n, x
    \rangle}} < \abs{x_j}$ with $\abs{x_j}$ integrable by assumption.
We apply Dominated Convergence to see that 
\begin{align*}
\lim_{n \to \infty} \frac{\partial}{\partial x_j} \hat{\mu} (u_n) &=  i \int \lim_{n\to \infty} x_j
e^{i\langle u_n, x \rangle} d \mu (x) \\
&= i \int x_j e^{i\langle u, x
\rangle} d \mu (x) \\
&= \frac{\partial}{\partial x_j} \hat{\mu}(u)
\end{align*}

TODO: Fill in the details of the induction step (it is pretty obvious
that argument above IS the induction step).
\end{proof}

The key in unlocking the relationship between weak convergence and
characteristic functions is a basic property of Fourier Transforms
that is often called the Plancherel Theorem.  In our particular case
the Plancherel Theorem shows that one may evaluate integrals of
continuous functions against probability measures equally well using
Fourier Transforms; in this way we'll see that the characteristic
function of a probability measure is a faithful representation of the
measure when viewed as a functional (the point of view implicit in
the definition of weak convergence).
\begin{thm}\label{PlancherelTheorem}Let 
\begin{align*}
\rho_\epsilon(x) &= \frac{1}{\epsilon\sqrt{2 \pi}}
  e^{-\frac{x^2}{2\epsilon^2}}
\end{align*}  be the Gaussian density with variance $\epsilon^2$.  Given a Borel probability measure
  $(\reals, \mathcal{B}(\reals), \mu)$ and an integrable $f : \reals
  \to \reals$, then for any $\epsilon > 0$,
\begin{align*}
\int_{-\infty}^\infty f * \rho_\epsilon (x) \, d\mu(x) &=
\frac{1}{2\pi} \int_{-\infty}^\infty e^{-\frac{\epsilon^2 u^2}{2}} \hat{f}(u) \overline{\hat{\mu}(u)}\, du
\end{align*}
If in addition, $f \in C_b(\reals)$ and $\hat{f}(u)$ is integrable
then
\begin{align*}
\int_{-\infty}^\infty f \, d\mu &= \frac{1}{2 \pi}
\int_{-\infty}^\infty \hat{f}(u) \overline{\hat{\mu}(u)}\, du
\end{align*}
\end{thm}
\begin{proof}
This is a calculation using Fubini's Theorem (Theorem \ref{Fubini}) to
the triple integral
\begin{align*}
\int \int \int e^{-\frac{\epsilon^2 u^2}{2}} f(x) e^{iux} e^{-iuy}
\, d\mu(y) \, dx \, du
\end{align*}
Note that by Tonelli's Theorem, 
\begin{align*}
\int \int \int \abs{ e^{-\frac{\epsilon^2 u^2}{2}} f(x) e^{iux} e^{-iuy}}
\, d\mu(y) \, dx \, du &= \int \int \int e^{-\frac{\epsilon^2 u^2}{2}} \abs{ f(x)} 
\, d\mu(y) \, dx \, du \\
&=\int \abs{ f(x)}  \, dx \int e^{-\frac{\epsilon^2 u^2}{2}} \, du < \infty
\end{align*}
and therefore we are justified in using using Fubini's Theorem to
calculate via iterated integrals
\begin{align*}
 \frac{1}{2 \pi} \int_{-\infty}^\infty e^{-\frac{\epsilon^2 u^2}{2}} \hat{f}(u)
 \overline{\hat{\mu}(u)}\, du &=  \frac{1}{2 \pi} \int_{-\infty}^\infty
 e^{-\frac{\epsilon^2 u^2}{2}} \left( \int_{-\infty}^\infty f(x)e^{iux} \, dx \right )
 \left (\int_{-\infty}^\infty e^{-iuy} \, d\mu(y) \right )\, du \\
&=  \frac{1}{2 \pi} \int_{-\infty}^\infty f(x)
\left( \int_{-\infty}^\infty 
 \left (\int_{-\infty}^\infty e^{iu(x-y)}  e^{-\frac{\epsilon^2 u^2}{2}}
   \, du \right ) \, d\mu(y) \right ) \, dx\\
\end{align*}
Now the inner integral is just the Fourier Transform of a Gaussian
with mean $0$ and variance $\frac{1}{\epsilon^2}$ which we have calculated in Exercise \ref {FourierTransformGaussian},
so we have by that calculation, another application of Fubini's
Theorem and the definition of convolution,
\begin{align*}
&= \frac{1}{2 \pi} \int_{-\infty}^\infty f(x)
\left( \int_{-\infty}^\infty \frac{\sqrt{2\pi}}{\epsilon}
  e^{-(x-y)^2/2\epsilon^2} \, d\mu(y) \right ) \, dx\\
&= \int_{-\infty}^\infty f(x)
\left( \int_{-\infty}^\infty \rho_\epsilon(x -y) \, d\mu(y) \right ) \, dx\\
&= \int_{-\infty}^\infty 
\left( \int_{-\infty}^\infty f(x) \rho_\epsilon(x -y) \, dx \right )
\, d\mu(y) \\
&= \int_{-\infty}^\infty f * \rho_\epsilon(y) \, d\mu(y) \\
\end{align*}

The second part of the theorem is just an application of Lemma
\ref{UniformApproximationByGaussians} and the first part of the
Theorem.  By the Lemma, we know that for any $f \in C_c(\reals;
\reals)$, we have $\lim_{\epsilon \to 0} \sup_x
\abs{f*\rho_\epsilon (x) - f(x)} = 0$.  So we have, 
\begin{align*}
\lim_{\epsilon \to 0} \abs{\int_{-\infty}^\infty f - f*\rho_\epsilon
  \, d\mu} &\leq \lim_{\epsilon \to 0} \int_{-\infty}^\infty \abs{ f -
  f*\rho_\epsilon} \, d\mu &\leq \lim_{\epsilon \to 0} \sup_x \abs{ f -
  f*\rho_\epsilon} = 0 \\
\end{align*}
and by integrability of $\hat{f}(u)$, the fact that
$\abs{\hat{\mu}} \leq 1$ (Lemma \ref{CharacteristicFunctionBoundedAndContinuous}) we may use Dominated Convergence
to see that 
\begin{align*}
\lim_{\epsilon \to 0} \int_{-\infty}^\infty f*\rho_\epsilon  \, d\mu
&= \frac{1}{2\pi} \lim_{\epsilon \to 0} \int_{-\infty}^\infty
e^{-\frac{1}{2}\epsilon^2 u^2} \hat{f}(u) \overline{\hat{\mu}(u)} \,
du \\
&= \frac{1}{2\pi}\int_{-\infty}^\infty  \lim_{\epsilon \to 0} 
e^{-\frac{1}{2}\epsilon^2 u^2} \hat{f}(u) \overline{\hat{\mu}(u)} \,
du \\
&=\frac{1}{2\pi}\int_{-\infty}^\infty  \hat{f}(u)
\overline{\hat{\mu}(u)} \, du 
\end{align*}
and therefore we have the result.
\end{proof}
As it turns out, we'll get a lot more mileage our the first statement
of the Theorem above.  We won't really ever be in a position in which
we have the required integrability of the Fourier Transform
$\hat{f}(t)$ to use the second part.  However, the technique used in the proof
of the second part of the Theorem will be replayed several times.
First we show that the characteristic function completely
characterizes probability measures.
\begin{thm}\label{EqualCharacteristicFunctionEqualMeasures} Let $\mu$ and $\nu$ be a probability measures on
  $\reals^n$ such that $\hat{\mu} = \hat{\nu}$, then $\mu=\nu$.
\end{thm}
\begin{proof}
Let $f \in C_c(\reals)$, then we know by Lemma
\ref{UniformApproximationByGaussians} that $\lim_{\epsilon \to 0}
\norm{\rho_\epsilon * f - f}_\infty = 0$.  Then for each $\epsilon >
0$, and using the Plancherel Theorem
\begin{align*}
\abs{\int f \, d\mu - \int f \, d\nu} &\leq \abs{\int \rho_\epsilon *
  f \, d\mu - \int \rho_\epsilon * f \, d\nu} + \int \abs{\rho_\epsilon *
  f  - f}\, d\mu + \int \abs{\rho_\epsilon *
  f  - f}\, d\nu \\
&\leq \abs{\frac{1}{2\pi} \int_{-\infty}^\infty e^{-\frac{\epsilon^2
      u^2}{2}} \hat{f}(u) (\overline{\hat{\mu}(u)} - \overline{\hat{\nu}(u)})\, du}+ 2\norm{\rho_\epsilon *
  f  - f}_\infty \\
&=  2\norm{\rho_\epsilon *
  f  - f}_\infty 
\end{align*}
Taking the limit as $\epsilon$ goes to $0$, we see that $\int f \,
d\mu = \int f \, d\nu$ for all $f \in C_c(\reals)$.

Now, take a finite interval $[a,b]$ and approximate
$\characteristic{[a,b]}$ by the compactly supported continuous
functions
\begin{align*}
f_n(x) &= \begin{cases}
1 & \text{for $a \leq x \leq b$} \\
0 & \text{for $x < a- \frac{1}{n}$ or $x > b + \frac{1}{n}$} \\
n(x-a) + 1 & \text{for $a-\frac{1}{n} \leq x < a$} \\
1 - n(x-b)& \text{for $b < x \leq b + \frac{1}{n}$} \\
\end{cases}
\end{align*}
It is clear that $f_n(x)$ is decreasing in $n$ and $\lim_{n \to
  \infty}f_n(x) = \characteristic{[a,b]}$ so by Monotone Convergence
\begin{align*}
\mu([a,b]) &= \lim_{n \to \infty} \int f_n \,
d\mu = \lim_{n \to \infty} \int f_n \,
d\nu = \nu([a,b])
\end{align*}
Since the Borel $\sigma$-algebra is generated by the closed intervals,
we see that $\mu = \nu$.
\end{proof}
\begin{thm}\label{IndependenceProductCharacteristicFunctions} Let $\xi = \left( \xi_1, \ldots , \xi_n \right)$ be an
    $\reals^n$-valued random variable.  Then the $\reals$-valued
    random variables $\xi_i$ are independent if and only if 
\begin{equation*}
\varphi_\xi(u_1, \ldots , u_n) = \prod_{j=1}^n \varphi_{\xi_j}(u_j)
\end{equation*}
\end{thm}
\begin{proof}
TODO: This is a simple corollary that follows by calculating the
characteristic function of the product and then using the fact that
the characteristic function uniquely defines the distribution.
First suppose that the $\xi_i$ and independent.  Then we calculate
\begin{align*}
\varphi_\xi(u) &= \expectation{e^{i\langle u, \xi
    \rangle}} = \expectation{\prod_{k=1}^ne^{i u_k \xi_k}}
=\prod_{k=1}^n \expectation{e^{i u_k \xi_k}} = \prod_{k=1}^n \varphi_{\xi_k}(u_k) 
\end{align*}
Note that here we have used Lemma \ref{IndependenceExpectations} on
a bounded complex valued function.
TODO: Do the simple validation that the Lemma extends to this situation.

On the other hand, if we assume that $\varphi_\xi(u_1, \ldots , u_n) =
\prod_{j=1}^n \varphi_{\xi_j}(u_j)$, then we know that if we pick
independent random variables $\eta_j$ where each $\eta_j$ has the same
distribution as $\xi_j$ then by the above calculation $\varphi_\xi(u)
= \varphi_\eta(u)$.
By Theorem \ref{EqualCharacteristicFunctionEqualMeasures} we know
that $\xi$ and $\eta$ have the same distribution.  Thus the $\xi_j$
are also independent by Lemma \ref{IndependenceProductMeasures} and
the equality of the distributions of each $\xi_j$ and $\eta_j$.
\end{proof}
\begin{lem}Let $\xi$ and $\eta$ be independent random vectors in
  $\reals^n$.  Then $\varphi_{\xi + \eta}(u) = \varphi_\xi(u) \varphi_\eta(u)$.
\end{lem}
\begin{proof}
This follows from the calculation 
\begin{align*}
\varphi_{\xi + \eta}(u) &= \expectation{e^{i\langle u, \xi + \eta
    \rangle}} = \expectation{e^{i\langle u, \xi
    \rangle}e^{i\langle u, \eta \rangle}} \\
&=\expectation{e^{i\langle u, \xi
    \rangle}} \expectation{e^{i\langle u, \eta \rangle}} =
\varphi_{\xi }(u)\varphi_{\eta}(u)& & \text{by
  Lemma \ref{IndependenceExpectations}}
\end{align*}
\end{proof}

\begin{examp}\label{FourierTransformGaussian}Let $\xi$ be an $N(0,1)$ random variable.  Then
  $\varphi_\xi(u) = e^{\frac{-u^2}{2}}$.  The least technical way of
  seeing this requires a bit of a trick.  First note that because $\sin
  ux$ is an odd function we have
\begin{align*}
\varphi_\xi(u) &= \frac{1}{\sqrt{2\pi}}\int_{-\infty}^\infty e^{iux}
e^{\frac{-x^2}{2}} \, dx \\
&= \frac{1}{\sqrt{2\pi}}\int_{-\infty}^\infty e^{\frac{-x^2}{2}}\cos ux
\, dx + \frac{i}{\sqrt{2\pi}}\int_{-\infty}^\infty e^{\frac{-x^2}{2}}\sin ux
\, dx \\
&=  \frac{1}{\sqrt{2\pi}}\int_{-\infty}^\infty e^{\frac{-x^2}{2}}\cos
ux \, dx
\end{align*}
On the other hand by Lemma \ref{MomentsAndDerivatives} and the fact
that $x \cos ux$ is an odd function we have 
\begin{align*}
\frac{d\varphi_\xi(u)}{du} &=
\frac{i}{\sqrt{2\pi}}\int_{-\infty}^\infty x e^{iux}
e^{\frac{-x^2}{2}} \, dx \\
&= \frac{i}{\sqrt{2\pi}}\int_{-\infty}^\infty x e^{\frac{-x^2}{2}}\cos ux
\, dx - \frac{1}{\sqrt{2\pi}}\int_{-\infty}^\infty x e^{\frac{-x^2}{2}}\sin ux
\, dx \\
&=  \frac{-1}{\sqrt{2\pi}}\int_{-\infty}^\infty x e^{\frac{-x^2}{2}}\sin ux
\, dx
\end{align*}
This last integral can be integrated by parts (let $df = x
e^{\frac{-x^2}{2}} dx$ and $g = \sin ux$, hence $f =
-e^{\frac{-x^2}{2}}$ and $dg = u\cos ux$) to yield
\begin{align*}
\frac{d\varphi_\xi(u)}{du} &=\frac{-u}{\sqrt{2\pi}}\int_{-\infty}^\infty e^{\frac{-x^2}{2}}\cos
ux \, dx
\end{align*}
and therefore we have shown that characteristic function satisfies the
simple first order differential equation $\frac{d\varphi_\xi(u)}{du} =
-u \varphi_\xi(u)$ which has the general solution $\varphi_\xi(u) = C
e^\frac{-u^2}{2}$ for some constant $C$.  To determine the constant,
we use Lemma
\ref{CharacteristicFunctionBoundedAndContinuous} to see that
$\varphi_\xi(0) = C = 1$ and we are done.
\end{examp}

To extend the previous example to arbitrary normal distributions, we
prove the following result that has independent interest.
\begin{lem}\label{CharacteristicFunctionOfAffineTransform}Let $\xi$ be a random vector in $\reals^d$ then for $a \in
  \reals^r$ and $A$ an $r \times d$ matrix, we have
\begin{align*}
\varphi_{a + A\xi}(u) &= e^{i\langle a, u \rangle} \varphi_\xi(A^* u)
\end{align*}
where $A^*$ denotes the transpose of $A$.
\end{lem}
\begin{proof}
This is a simple calculation
\begin{align*}
\varphi_{a + A\xi}(u) &= \expectation{e^{i \langle u, a + A \xi
    \rangle}} = \expectation{e^{i \langle u, a \rangle} e^{i \langle u,  A \xi
    \rangle}}
= e^{i \langle u, a \rangle}\expectation{e^{i \langle A^* u,  \xi
    \rangle}} = e^{i\langle a, u \rangle} \varphi_\xi(A^* u)
\end{align*}
where we have used the elementary fact from linear algebra that
\begin{align*}
\langle u,  Av \rangle &= u^*Av = (u^*Av)^* = v^* A^*u = \langle A^*u,  v \rangle
\end{align*}
\end{proof}

\begin{examp}\label{FourierTransformGeneralGaussian}Let $\xi$ be an $N(\mu,\sigma^2)$ random variable.  Then
  $\varphi_\xi(u) = e^{iu\mu - \frac{1}{2} u^2\sigma^2}$.  We know
  that if $\eta$ is an $N(0,1)$ random variable then $\mu +
  \sigma\eta$ is $N(\mu, \sigma^2)$, so by the previous Lemma
  \ref{CharacteristicFunctionOfAffineTransform} and Example \ref{FourierTransformGaussian}
\begin{align*}
\varphi_\xi(u) &= e^{iu\mu} \varphi_\eta(\sigma u) = e^{iu\mu - \frac{1}{2} u^2\sigma^2}
\end{align*}
\end{examp}


The last piece of the puzzle that we need to put into place before
proving the Central Limit Theorem is a result that shows we can test
convergence in distribution by looking at pointwise convergence of
associated characteristic functions.
\begin{thm}[Glivenko-Levy Continuity
  Theorem]\label{GlivenkoLevyContinuity}If $\mu, \mu_1, \mu_2, \dots$
  are probability measures on $(\reals^n, \mathcal{B}(\reals^n))$,
  then $\mu_n$ converge weakly to $\mu$ if and only if $\hat{\mu}_n(u)$
  converge to $\hat{\mu}(u)$ pointwise.
\end{thm}
\begin{proof}
By Theorem \ref{WeakConvergenceWithSmoothTestFunctions} it suffices to
show that for every $f \in C^\infty_c(\reals^n, \reals)$, we have $\lim_{n
  \to \infty} \int f
\, d\mu_n = \int f \, d\mu$.  By \ref{UniformApproximationByGaussians}
we know that $\lim_{\epsilon \to 0}
\norm{\rho_\epsilon * f - f}_\infty = 0$.  Pick $\delta > 0$ and find
$\epsilon > 0$ such that $\norm{\rho_\epsilon * f - f}_\infty <
\delta$.
Now, 
\begin{align*}
\abs{\int f \, d\mu_n - \int f \, d\mu} &\leq \abs{\int (f - \rho_\epsilon * f) \, d\mu_n} + \abs{\int \rho_\epsilon * f \, d\mu_n -
  \int \rho_\epsilon * f \, d\mu} + \abs{\int (\rho_\epsilon * f - f)
  \, d\mu} \\
&\leq \delta + \frac{1}{2\pi} \abs{\int \hat{f}(t)
  e^{-\frac{1}{2}\epsilon^2t^2} (\hat{\mu}_n(t) - \hat{\mu}(t)) \, dt}
+ \delta 
\end{align*}
where we have used the Plancherel Theorem (Theorem
\ref{PlancherelTheorem}) and the uniform
approximation of $f$ by $\rho_\epsilon * f$ in going from the first to
the second line.

Because $f$ is compactly supported, we know that $\hat{f}(t) \leq
\norm{f}_\infty$ and together with Lemma
\ref{CharacteristicFunctionBoundedAndContinuous} we see that 
\begin{align*}
\abs{\hat{f}(t)
  e^{-\frac{1}{2}\epsilon^2t^2} (\hat{\mu}_n(t) - \hat{\mu}(t))} \leq
2\norm{f}_\infty e^{-\frac{1}{2}\epsilon^2t^2}
\end{align*} where the upper bound
is an integrable function of $t$.  Therefore by Dominated Convergence
we see that $\limsup_{n \to \infty} \abs{\int f \, d\mu_n - \int f \,
  d\mu}  \leq 2\delta$.  Since $\delta>0$ was arbitrary, we have $\int f \, d\mu_n = \int f \,  d\mu$.
\end{proof}

Note that part of the hypothesis in the above theorem is the fact that
the pointwise limit of the characteristic functions is assumed to be
the characteristic function of a probability measure.  There is a stronger form of the above theorem that
characterizes when a pointwise limit of charactersistic functions is
in fact the characteristic function of a probability measure.  That
stronger result is not needed to prove the Central Limit Theorem so we
postpone its statement and proof until later.

\begin{thm}[Central Limit Theorem]\label{CentralLimitTheorem}Let
  $\xi, \xi_1, \xi_2, \dots$ be i.i.d. random variables with $\mu =
  \expectation{\xi}$ and 
  $\sigma = \variance{\xi_n} < \infty$, then 
\begin{align*}
\sqrt{n} (\frac{1}{n}\sum_{i=1}^n \xi_i -  \mu) \todist N(0,\sigma^2)
\end{align*}
\end{thm}
\begin{proof}
The first thing to note is that by using the Theorem on $\frac{\xi_i -
  \mu}{\sigma}$, it suffices to assume that $\mu = 0$
and $\sigma = 1$.  Thus we only have to show that $\frac{1}{\sqrt{n}}
\sum_{k=1}^n \xi_k \todist N(0,1)$.  

Define $S_n = \sum_{k=1}^n \xi_k$.  By Theorem
\ref{GlivenkoLevyContinuity} it suffices to show that 
\begin{align*}
\lim_{n \to \infty} \expectation{e^{itS_n/\sqrt{n}}} &= e^{t^2/2}
\end{align*}
To calculate the limit, first note that by independence and i.i.d. we
have 
\begin{align*}
\expectation{e^{itS_n/\sqrt{n}}} &= \prod_{k=1}^n
\expectation{e^{it\xi_k/\sqrt{n}}}  = \left[\expectation{e^{it\xi/\sqrt{n}}}\right]^n
\end{align*}

In order to evaluate the limit, we take the Taylor expansion of the exponential $e^{ix} = 1 + i x -
\frac{1}{2}x^2 + R(x)$ where by Lagrange form of the remainder and the
fact that $\abs{\frac{d}{dx} e^{ix}} \leq 1$, we
see that $\abs{R(x)} \leq \frac{1}{6}\abs{x}^3$.  Note that this
estimate isn't very good for large $\abs{x}$ but it is easy to do
better for $\abs{x} > 1$ just using the triangle inequality
\begin{align*}
\abs{e^{ix} - 1 - ix + \frac{1}{2}x^2} &\leq 2 + \abs{x} +
\frac{1}{2}x^2 \leq \frac{7}{2}x^2
\end{align*}
Therefore we have the bound $\abs{R(x)} \leq \frac{7}{2}(\abs{x}^3
\wedge x^2)$.  Applying the Taylor expansion and using the zero mean
and unit variance assumption, we get
\begin{align*}
\expectation{e^{itS_n/\sqrt{n}}} &= \left(1 - \frac{t^2}{2n} + \expectation{R(\frac{t\xi}{\sqrt{n}})}\right)^n
\end{align*}
By our estimate on the remainder term, we can see that
\begin{align*}
n \abs{\expectation{R(\frac{t\xi}{\sqrt{n}})}} &\leq
\frac{7}{2}\expectation{\frac{t^3\abs{\xi}^3}{\sqrt{n}} \wedge t^2
  \xi^2} \\
&\leq \frac{7}{2} \expectation{ t^2
  \xi^2} = \frac{7t^2}{2} 
\end{align*}
By the above inequalities and Dominated Convergence we can
conclude that 
\begin{align*}
\lim_{n \to \infty} n
\abs{\expectation{R(\frac{t\xi}{\sqrt{n}})}} = 0
\end{align*}
so if we define $\epsilon_n = \frac{2n}{t^2}
\abs{\expectation{R(\frac{t\xi}{\sqrt{n}})}}$ then we have $\lim_{n
  \to \infty} \epsilon_n = 0$ and 
\begin{align*}
\lim_{n \to \infty} \expectation{e^{itS_n/\sqrt{n}}} &= \lim_{n \to 
  \infty} \left(1 - \frac{t^2}{2n} (1 + \epsilon_n) \right)^n =\lim_{n \to 
  \infty}  e^{n \log(1 - \frac{t^2}{2n} (1 + \epsilon_n))} =
e^{-t^2/2}
\end{align*}
\end{proof}

It is also useful to call out a useful corollary of the continuity
theorem that allows one to characterize convergence in distribution of
random vectors by considering one dimensional projections.
\begin{cor}[Cramer Wold Device]\label{CramerWoldDevice}Let $\xi, \xi_1, \xi_2, \dotsc$ be 
  random vectors in $\reals^d$.  Then $\langle c, \xi_n \rangle
  \todist \langle c, \xi \rangle$ for all $c \in \reals^d$ if and only
  if $\xi_n \todist \xi$.
\end{cor}
\begin{proof}
Simply note that for all random vectors $\xi$,
$c \in \reals^d$ and $x \in \reals$
\begin{align*}
\varphi_{\langle c , \xi\rangle}(x) &= \expectation{e^{i x \langle
    c , \xi \rangle} }= \varphi_\xi(x c)
\end{align*}
 
Therefore if $\langle c, \xi_n \rangle
  \todist \langle c, \xi \rangle$ for all $c \in \reals^d$ then by the
  Glivenko-Levy Continuity Theorem \ref{GlivenkoLevyContinuity}
we know that 
\begin{align*}
\lim_{n \to \infty} \varphi_{\xi_n}(c) &= \lim_{n \to \infty}
\varphi_{\langle c, \xi_n \rangle}(1) = \varphi_{\langle c, \xi
  \rangle}(1) = \varphi_\xi(c)
\end{align*}
and applying the Theorem again we conclude that $\xi_n \todist \xi$.  In
a completely analogous way, if we assume that $\xi_n \todist \xi$ then
for all $c \in \reals^d$ and $x\in \reals$, then
\begin{align*}
\lim_{n \to \infty} \varphi_{\langle c, \xi_n \rangle}(x) &= \lim_{n \to \infty}
\varphi_{\xi_n}(x c) = \varphi_{\xi}(x c) = \varphi_{\langle c, \xi \rangle}(x)
\end{align*}
from which we conclude that $\langle c, \xi_n \rangle \todist \langle
c, \xi \rangle$.
\end{proof}

\begin{thm}[Prokhorov's Theorem, special case]Let $\mu_n$ be a tight sequence of
  measures on $\reals^n$.  Then there is a subsequence of that
  converges in distribution.
\end{thm}
\begin{proof}
TODO
\end{proof}
 
TODO: Do the full Levy Continuity Theorem (and Prokhorov's Theorem) that shows a characteristic
function that is continuous at $0$ is the characterisitic function of
a probability measure (the basic point is that the pointwise limit of
characteristic functions of probability measures is almost the
characteristic function of a probability measure; the associated
distribution function may not have the correct limits at $\pm
\infty$ due to mass escaping to infinity.  If we assume continuity at $0$, then we can prove tightness
which keeps the mass from escaping and shows that the limits are $0,1$ as required of a distribution
function.  Note that the pointwise limit of a sequence of
characteristic functions is the characteristic function of a measure
(though not necessarily a probability measure); this fact is often
know as the Helly Selection Theorem.  It can be restated in terms of a
topology on the space of locally finite measures called the vague
topology and the Helly Selection Theorem can be restated as saying
that the space of probability measures is relatively sequentially
compact in the vague topology on the locally finite measures on $\reals^n$.

\section{Gaussian Random Vectors and the Multidimensional Central
  Limit Theorem}
There is a version of the Central Limit Theorem for random vectors in
$\reals^d$ in which Gaussian distributions also occur.  The nature of
Gaussians in this context is a bit more subtle than in the one
dimensional case.  We lead with a definition
\begin{defn}A random vector $\xi$ in $\reals^d$ is said to be a \emph{Gaussian random
  vector} is for every $a \in \reals^d$, the random variable $\langle
  a,\xi \rangle$ is a univariate normal or is almost surely 0 (which
  we take as the degenerate univariate normal $N(0,0)$).
\end{defn}
The first theorem that we prove gives an alternative characterization
of the property in terms of characteristic functions.  This result is
sometimes used as the definition of a Gaussian random vector; the only
real benefit to the definition we've given is that it is more
elementary.
\begin{thm}\label{GaussianVectorCharacteristicFunction}A random vector
  $\xi$ in $\reals^d$ is Gaussian if and only if there is a $\mu \in
  \reals^d$ and a symmetric positive semi-definite matrix $Q \in
  \reals^{d\times d}$ such that 
\begin{align*}
\varphi_\xi(u) &= e^{i\langle u, \mu\rangle - \frac{1}{2}\langle u, Qu
\rangle}
\end{align*}
For $\xi$ with characteristic function of this form, $\mu =
\expectation{\xi}$ and $Q = \covariance{\xi}$; we say that $\xi$ is
$N(\mu, Q)$.
\end{thm}
\begin{proof}
First we assume that we have a characteristic function of the above
form.  Let $a \in \reals^d$ and consider the random variable $\langle
a, \xi \rangle$.  Notice that $\langle a, \xi \rangle = a^* \xi$ is a
special case of an affine transformation so we
can apply Lemma
\ref{CharacteristicFunctionOfAffineTransform} to calculate
\begin{align*}
\varphi_{\langle a, \xi \rangle}(u) &= \varphi_\xi(a u) = e^{i u \langle
  a, \mu\rangle - \frac{1}{2}\langle a, Q a\rangle u^2} 
\end{align*}
Now, by Example \ref{FourierTransformGeneralGaussian}
we see that $\langle a, \xi \rangle$ is $N(\langle
  a, \mu\rangle, \langle a, Q a\rangle)$.  Since $a$ was arbitrary,
  this shows that $\xi$ is Gaussian.

Now we assume that $\xi$ is Gaussian.  Let $\mu = (\mu_1, \dots,
\mu_d) = \expectation{\xi}$ and let $Q = \covariance{\xi}$.  Pick $a
\in \reals^d$ and note that 
\begin{align*}
\expectation{\langle a,\xi \rangle} &= \langle a, \mu \rangle \\
\variance{\langle a,\xi \rangle} &= \expectation{(\langle a,\xi
  \rangle - \expectation{\langle a,\xi \rangle})^2} \\
&=  \expectation{(\langle a,\xi - \mu  \rangle)^2} \\
&=  \expectation{a^* (\xi - \mu) (\xi - \mu)^* a} \\
&= a^*\expectation{ (\xi - \mu) (\xi - \mu)^*}a = \langle a, Q a
\rangle \\
\end{align*}
Now we know by our assumption and the expectation and variance
calculation above  that $\langle a, \xi \rangle$ is
$N(\langle a, \mu \rangle, \langle a, Q a \rangle)$ and by Example
\ref{FourierTransformGeneralGaussian}, we have
\begin{align*}
\varphi_{\langle a, \xi \rangle}(u) &= e^{iu \langle a, \mu \rangle -
  \frac{1}{2}\langle a, Q a \rangle u^2}
\end{align*}
As above we can apply Lemma
\ref{CharacteristicFunctionOfAffineTransform}
to see
\begin{align*}
\varphi_\xi(a) &= \varphi_{\langle a, \xi \rangle}(1) =  e^{i \langle a, \mu \rangle -  \frac{1}{2}\langle a, Q a \rangle}
\end{align*}
Together with the fact two measures with the same characteristic
function must be equal (Theorem
\ref{EqualCharacteristicFunctionEqualMeasures} ), this also proves the last part of the Theorem since we have shown by
construction that $\mu = \expectation{\xi}$ and $Q = \covariance{\xi}$.
\end{proof}

\begin{examp}Let $\xi_1, \dots, \xi_d$ be independent random variables
  with $\xi_i$ being normal $N(\mu_i, \sigma_i^2)$.  Then $\xi =
  (\xi_1, \dots, \xi_d)$ is a Gaussian random vector.  In fact, if we
  let $\mu = (\mu_1, \dots, \mu_d)$ and 
\begin{align*}
Q = \diag(\sigma_1^2, \dots, \sigma_d^2)
\end{align*}
then $\xi = N(\mu, Q)$.
\end{examp}

\begin{examp}\label{LinearTransformationGaussian}Let $(\xi_1, \dots, \xi_d)$ be $\xi = N(\mu, Q)$.  Let
  $A$ be a $r \times d$ matrix then $A \xi$ is $N( A \mu, A Q
  A^T)$.

To see this note that if $a \in \reals^d$ then $\langle a , A \xi
\rangle = \langle a^T A, \xi \rangle$ so we see that $A \xi$ is in
fact Gaussian.  The fact that $\expectation{A \xi} =
A\expectation{\xi}$ is immediate from linearity of expectation.  To
calculate the covariance we 
\begin{align*}
\covariance{A\xi} &= \expectation{A(\xi - \mu) (\xi - \mu)^T A^T} =
A\expectation{(\xi - \mu) (\xi - \mu)^T}A^T = 
A Q A^T
\end{align*}
\end{examp}

Gaussian random vectors are easy to handle in many cases because
independence is so easily verified; this is a case in which
uncorrelated implies independent.
\begin{prop}\label{GaussianIndependence}Let $\xi$ be a Gaussian random
  vector in $\reals^d$ then for any subset $I\subset \lbrace 1,
  \dotsc, d \rbrace$ let $\xi_I = (\xi_{i_1}, \dotsc, \xi_{i_r})$ be
  random vector in $\reals^r$ where $r$ is the cardinality of $I$.
  Given any two disjoint subsets $I,J$ we have $\xi_I$ and $\xi_J$ are
  independent if and only if $\covariance{\xi_I, \xi_J} = 0$.  
\end{prop}
\begin{proof}
Since independence always implies uncorrelated it suffices to assume
that $\xi_I$ and $\xi_J$ are uncorrelated and then show independence.

Concatenate $\xi_I$ and $\xi_J$ into a single random vector $\eta=(\xi_I, \xi_J)$.
Clearly $\eta$ is a Gaussian random vector since it is a projection of
a Gaussian random vector; furthermore since $\xi_I$ and $\xi_J$ are
uncorrelated
it has a block diagonal covariance matrix 
\begin{align*}
\covariance{\eta} = \begin{bmatrix}
\covariance{\xi_I} & 0 \\
0 & \covariance{\xi_J}
\end{bmatrix}
\end{align*}
By Theorem \ref{GaussianVectorCharacteristicFunction} we can calculate
the characteristic function.  
\begin{align*}
\varphi_\eta(u) &= 
e^{i\langle u, \mu \rangle - \frac{1}{2} \langle u, \covariance{\eta} u \rangle} 
=e^{i\langle u_I, \mu_I \rangle - \frac{1}{2} \langle u_I, \covariance{\xi_I} u_I \rangle} 
e^{i\langle u_J, \mu_J \rangle - \frac{1}{2} \langle u_J, \covariance{\xi_J} u_J \rangle} 
= \varphi_{\xi_I}(u) \varphi_{\xi_J}(u)
\end{align*}
which by Theorem \ref{IndependenceProductCharacteristicFunctions} shows that $\xi_I$ and $\xi_J$ are independent.
\end{proof}

The characterization of Gaussian random vectors using characteristic
functions allows us to see that almost sure limits of Gaussian random vectors are
Gaussian random vectors.  We will need this result when we construct
Brownian motion later on.
\begin{lem}\label{LimitOfGaussianRandomVectors}Let $\xi_1, \xi_2, \dots$ be a sequence of random
  vectors in $\reals^d$ with $\xi_n$ an $N(\mu_n, C_n)$ Gaussian
  random vector.  Suppose that $\xi$ is a random vector such
  that $\xi_n$ converges to $\xi$ almost surely.  If $\lim_{n \to
    \infty} \expectation{\xi_n} = \mu$ and $\lim_{n \to \infty}
  \covariance{\xi_n} = C$ then $\xi$ is a $N(\mu, C)$ Gaussian random vector.
\end{lem}
\begin{proof}
Since $\xi_n$ converges almost surely to $\xi$ then it converges in
distribution.  We know from Lemma
\ref{GaussianVectorCharacteristicFunction} and the Glivenko-Levy
Continuity Theorem (Theorem \ref{GlivenkoLevyContinuity}) we see
\begin{align*}
\varphi_\xi(u) &= \lim_{n \to \infty} \varphi_{\xi_n}(u) = \lim_{n \to
  \infty} e^{i\langle u, \mu_n\rangle - \frac{1}{2}\langle u, C_n
  y\rangle} = e^{i\langle u, \mu\rangle - \frac{1}{2}\langle u, C
  y\rangle} 
\end{align*}
where we have used continuity of $e^{ix}$.  Thus, using Lemma
\ref{GaussianVectorCharacteristicFunction}
again shows that $\xi$ is $N(\mu, C)$.
\end{proof}

TODO: Gaussian Random Variables in $\reals^n$ and the multidimensional
CLT.  Pretty sure this can be derived from the Cramer Wold device.
TODO: Show that a
given two independent Gaussian random variables their sum and
difference are independent Gaussian (that probably doesn't require
Gaussian random vectors).  Not sure we really need to call this out as a Lemma.

One last thing we need in the sequel are estimates on the tails of
normal random variables.  These results are not required yet nor do
they add anything significant to the conceptual picture so the
reader can safely skip over them and return to them when they are
referenced.

\begin{lem}\label{GaussianTailsElementary}Given an $N(0,1)$ random
  variable $\xi$ we have for all $\lambda > 0$, 
\begin{align*}
\frac{\lambda}{\sqrt{2\pi}(1+\lambda^2)} e^{-\lambda^2/2}&\leq \probability{\xi \geq
    \lambda} \leq \frac{1}{\sqrt{2\pi} \lambda} e^{-\lambda^2/2}
\end{align*}
\end{lem}
\begin{proof}
We start by showing the upper bound 
\begin{align*}
\probability{\xi \geq \lambda} &= \frac{1}{\sqrt{2\pi}} \int_\lambda^\infty
e^{\frac{-x^2}{2}} \, dx 
\leq  \frac{1}{\sqrt{2\pi}} \int_\lambda^\infty
\frac{x}{\lambda} e^{\frac{-x^2}{2}} \, dx 
= \frac{1}{\sqrt{2\pi}\lambda} e^{\frac{-\lambda^2}{2}}
\end{align*}
Interestingly, the lower bound follows from the upper bound.  Define 
\begin{align*}
f(\lambda) &= \lambda e^{-\lambda^2/2} - (1 + \lambda^2)
\int_\lambda^\infty e^{-x^2/2} \, dx
\end{align*}
and notice that $f(0) = -\int_0^\infty e^{-x^2/2} \, dx =
-\frac{\sqrt{2\pi}}{2} < 0$.  Furthermore if we use the upper bound
just proven
\begin{align*}
\lim_{\lambda \to \infty} f(\lambda)
&= \lim_{\lambda \to \infty} \lambda^2 \int_\lambda^\infty e^{-x^2/2}
\, dx \leq \lim_{\lambda \to \infty} \lambda e^{-\lambda^2/2} = 0
\end{align*}
and therefore $\lim_{\lambda \to \infty} f(\lambda) = 0$.  In addition
we have for $\lambda \geq 0$,
\begin{align*}
\frac{d}{d\lambda} f(\lambda) &= e^{-\lambda^2/2} - \lambda^2
e^{-\lambda^2/2} + (1+\lambda^2) e^{-\lambda^2/2} -2\lambda
\int_\lambda^\infty e^{-x^2/2} \, dx \\
&=2\lambda \left(\frac{1}{\lambda}e^{-\lambda^2/2} -
  \int_\lambda^\infty e^{-x^2/2} \, dx \right ) \geq 0
\end{align*}
where the last inequality follows from the upper bound just proven.
This shows that $f(\lambda) \geq 0$ for all $\lambda \geq 0$ and we
are done.
\end{proof}

\section{Laplace Transforms}
It turns out to be useful to specialize characteristic functions for
the case in which we have a measure that is supported on the positive
orthant $\reals^d_+$.  

\begin{defn}Let $\mu$ be a probability measure on $\reals^d_+$. Its
  \emph{Laplace Transform} is denoted $\tilde{\mu}$ and is the function on
  $\reals^d_+$ defined by 
\begin{align*}
\tilde{\mu}(u) &= \int e^{-\langle u,x \rangle} d \mu(x)
\end{align*}
\end{defn}

Next we observe that the behavior of the Laplace transform near zero
corresponds to the behavior of the measure near infinity.
\begin{lem}\label{LaplaceTailEstimate}Let $\mu$ be a probability
  measure on $\reals^d_+$ and let $\mathds{1}=(1, \dotsc, 1) \in
  \reals^d_+$.  Then for each $r > 0$ we have
\begin{align*}
\mu\lbrace \abs{x} \geq r\rbrace &\leq 2(1 - \tilde{\mu}(\mathds{1}/r))
\end{align*}
\end{lem}
\begin{proof}
In order to see how simple the estimate is, first assume that $d=1$.
Observe that because $e^{-ux}$ is a decreasing function of $x$ for $u
> 0$ we have $e^{-ux}  \leq e^{-1} < 1/2$ for all $x
\geq 1/u$ and $e^{-ux} \leq 1$ for all $x \geq 0$.  Therefore for a fixed $r > 0$,
\begin{align*}
\tilde{\mu}(r) &= \int e^{-rx} \, d\mu(x) = \int \characteristic{[0,1/r)}(x)  e^{-rx}
\, d\mu(x)  + \int \characteristic{[1/r,\infty)}(x)  e^{-rx} \,
d\mu(x)  \\
&\leq \mu[0,1/r) + \frac{1}{2} \mu[1/r, \infty) = 1 - \frac{1}{2} \mu[1/r, \infty) 
\end{align*}

To extend to the case of general $d$, we need a little bit more
information.  Note that minimum value of
$\langle \mathds{1}, x\rangle = \sum_{j=1}^d x_j$ on $\reals^d_+ \cap \lbrace \abs{x} \geq u \rbrace$ is
$u$ (it occurs at the points $(0, \dotsc, 0,u,0, \dotsc,0)$).  TODO:
Show this...
Therefore we know that for all fixed $r \in \reals_+$ we have $e^{-\langle r
\mathds{1}, x \rangle} \leq e^{-1} < 1/2$ for all $x \in \reals^d_+$
with $\abs{x} \geq 1/r$.  Now we can playback the same argument as the
case $d=1$:
\begin{align*}
\tilde{\mu}(r \cdot \mathds{1}) &= \int e^{-r\langle \mathds{1}, x \rangle}
\, d\mu(x) \\
&= \int \characteristic{\abs{x} < 1/r}(x)  e^{-r\langle \mathds{1}, x \rangle}
\, d\mu(x)  + \int \characteristic{\abs{x} \geq 1/r}(x)  e^{-r\langle \mathds{1}, x \rangle} \,
d\mu(x)  \\
&\leq \mu \lbrace \abs{x} < 1/r \rbrace + \frac{1}{2} \mu \lbrace
\abs{x} \geq 1/r \rbrace = 1 - \frac{1}{2} \mu \lbrace \abs{x} \geq
1/r \rbrace
\end{align*}
\end{proof}

\begin{lem}\label{LaplaceEquicontinuityAndTightness}Let $\lbrace
  \mu_\alpha \rbrace$
  be a family of probability measures on $\reals^d_+$, then the family
  $\lbrace \mu_\alpha \rbrace$ is tight if and only if the family
  $\lbrace \tilde{\mu}_\alpha \rbrace$ is equicontinuous at $0$.  If
  this is true then $\lbrace \tilde{\mu}_\alpha \rbrace$ is uniformly equicontinuous on all of $\reals^d_+$.
\end{lem}
\begin{proof}
First we assume that the family $\lbrace \mu_\alpha \rbrace$ is
equicontinuous and show tightness.  To do this, note that if $\epsilon
> 0$ is given,
 then by equicontinuity we can find
$\delta>0$ such that $1 - \tilde{\mu}_\alpha(u \mathds{1}) < \epsilon/2$ for all
$0 \leq u < \delta$ and all $\alpha$.  By Lemma
\ref{LaplaceTailEstimate} we get for every $r > 1/\delta$ 
\begin{align*}
\mu_\alpha\lbrace \abs{x} \geq r \rbrace &\leq 2 (1 - \tilde{\mu_\alpha}(\mathds{1}/r)) < \epsilon
\end{align*}
and therefore tightness is proven. 

Now assume that the family $\lbrace \tilde{\mu}_\alpha \rbrace$ is
tight.  For each $\alpha$ let $\xi_\alpha$ be a random vector with
distribution $\mu_\alpha$.  Using the elementary bound $\abs{e^{-x} - e^{-y}} \leq \abs{x
  -y} \wedge 1$ for $0 \leq x,y < \infty$ and Cauchy-Schwartz (Lemma
\ref{CauchySchwartz}) we see that for any
$0 < \epsilon < 2$,
\begin{align*}
\abs{\tilde{\mu}_\alpha(u) - \tilde{\mu}_\alpha(v)} &=
\abs{\expectation{ e^{-\langle u, \xi_\alpha \rangle} - e^{-\langle v,
      \xi_\alpha}}} \\ 
&\leq
\expectation{\abs{ e^{-\langle u, \xi_\alpha \rangle} - e^{-\langle v,
      \xi_\alpha \rangle}}} \\ 
&\leq \expectation{\abs{\langle u-v, \xi_\alpha \rangle} \wedge 1} \\ 
&= \expectation{\abs{\langle u-v, \xi_\alpha \rangle} \wedge 1 ;
  \langle u-v, \xi_\alpha \rangle < \epsilon/2}  + 
\expectation{\abs{\langle u-v, \xi_\alpha \rangle} \wedge 1 ;
  \langle u-v, \xi_\alpha \rangle \geq \epsilon/2} \\ 
&\leq \epsilon/2 + 
\probability{\langle u-v, \xi_\alpha \rangle \geq \epsilon/2} \\ 
&\leq \epsilon/2 + 
\probability{\abs{\xi_\alpha } \geq \frac{\epsilon}{2\abs{u-v}}} \\ 
\end{align*}
Thus by tightness for all $u,v \in \reals^d_+$ with $\abs{u-v}$
sufficiently small we have $\abs{\tilde{\mu}_\alpha(u) -
  \tilde{\mu}_\alpha(v)} < \epsilon$ uniformly in $\alpha$.  Thus we see that
the family
$\lbrace \tilde{\mu}_\alpha \rbrace$ is uniformly equicontinuous on
all of $\reals^d_+$ and in particular at $0$.
\end{proof}

The following result is analogous to the Glivenko-Levy Continuity
Theorem \ref{GlivenkoLevyContinuity} for characteristic functions.  As
with that result, here we point out that the assumption that $\mu_n$
converges to a probability measure is critical and we will return to the
question of how to remove the assumption (using the notion of tightness) later on.
\begin{thm}[Glivenko-Levy Continuity
  Theorem]\label{GlivenkoLevyContinuityLaplace}If $\mu, \mu_1, \mu_2, \dots$
  are probability measures on $(\reals^d_+, \mathcal{B}(\reals^d_+))$,
  then $\mu_n$ converge weakly to $\mu$ if and only if $\tilde{\mu}_n(u)$
  converge to $\tilde{\mu}(u)$ pointwise.  Moreover if this is true
  then the convergence is uniform on bounded sets.
\end{thm}
\begin{proof}
Since $\mu_n$ converge to $\mu$ weakly and $e^{-\langle u,x \rangle}$
is bounded and continuous we know that $\tilde{\mu}_n(u) \to
\tilde{\mu}(u)$ pointwise.  In fact, by Lemma
\ref{WeakConvergenceImpliesTight} we know that the family $\mu_n$
is tight and therefore by Lemma
\ref{LaplaceEquicontinuityAndTightness} it is uniformly equicontinuous
on $\reals^d_+$.  this convergence is uniform on
every bounded set.

Now we assume that $\tilde{\mu}_n(u)$converges to $\tilde{\mu}(u)$ for
every $u \in \reals^d_+$.  We now want to approximate general bounded
continuous functions by functions $e^{-\langle u,x \rangle}$ in order
derive weak convergence.  To do this, we will consider $[0,\infty]^d$
which is a compact Hausdorff space and therefore amenable to
application of the Stone Weierstrass Theorem
\ref{StoneWeierstrassApproximation}.  To use an approximation of
functions on $[0,\infty]^d$ derive an effective approximation on
$\reals^d_+$ is going to require that we are able to control behavior
of the measures $\mu_n$ at infinity and therefore we first show that
$\mu_n$ is a tight family.  Suppose $\epsilon > 0$ is given and use
the continuity of $\tilde{\mu}(u)$ and the fact that
$\tilde{\mu}(0)=1$ to find $r_0 > 0$ such that $1 - \tilde{\mu}(\mathds{1}/r_0) <
\epsilon/2$.  By pointwise convergence $\tilde{\mu}_n(\mathds{1}/r_0) \to \tilde{\mu}(\mathds{1}/r_0)$ we
can find an $N > 0$ such that $1 - \tilde{\mu_n}(\mathds{1}/r_0) < \epsilon$ for all $n
\geq N$ and therefore by Lemma \ref{LaplaceTailEstimate}, $\mu_n(\abs{x} \geq r)  \leq 1 - \mu_n(\mathds{1}/r_0) <
\epsilon$.  For each $n = 1, \dotsc, N-1$ by continuity of measure  we
can find $r_n > 0$ such that $\mu_n\lbrace \abs{x} \geq r_n \rbrace <
\epsilon$.  Therefore taking the maximum $r = r_0 \vee r_1 \vee \dotsb \vee
r_{N-1}$ we get $\mu_n\lbrace \abs{x} \geq r \rbrace <
\epsilon$ for all $n$ and we have shown $\mu_n$ is tight.

Suppose that $\epsilon > 0$ is given and
pick $r > 0$ such that $\mu_n(\reals^d_+ \setminus B(0, r)) <
\epsilon$ and $\mu(\reals^d_+ \setminus B(0, r)) < \epsilon$.

Having shown that $\mu_n$ is tight we return to the task of creating
an approximation.  Since $e^{-\langle u,x \rangle}$
has limits (either $0$ or $1$) as $x \to \infty$ we can extend each
such function to
a continuous function on $[0,\infty]^d$.  Note also that the family $e^{-\langle k, x
  \rangle}$ for $k \in \integers^d_+$ contains the constant functions
and separates points therefore we can apply the Stone Weierstrass
Theorem to conclude that any
continuous function on $[0,\infty]^d$ can be uniformly approximated by
a linear combination of $e^{-\langle k, x  \rangle}$. 

Given a bounded continuous function $f : \reals^d_+ \to \reals$ such
that $\abs{f(x)} \leq M$ we
apply a continuous cutoff $1 - d(x,B(0,r)) \vee 0$ to create a function $\hat{f} : \reals^d_+
\to \reals$ such that $\hat{f}(x) \leq M$ for all $x \in \reals^d_+$, $f(x) = \hat{f}(x)$ for $x \in B(0,r)$ and
$\hat{f}(x) = 0$ for $\abs{x} > r+1$.  Note that for every $n$ we have
\begin{align*}
\int \abs{f - \hat{f}} \, d\mu_n &= 
\int_{\abs{x} < r} \abs{f -  \hat{f}} \, d\mu_n + 
\int_{\abs{x} \geq r} \abs{f -  \hat{f}} \, d\mu_n < 2 M \epsilon
\end{align*}
and similarly for $\mu$.

The function $\hat{f}$ can be
extended by zero to define a continuous function on $[0,\infty]^d$ and
therefore we can find some finite linear combination $g = \sum_k c_k
e^{-\langle k, x \rangle}$ such that $\abs{\hat{f}(x) - g(x)}<
\epsilon$ for all $x \in [0, \infty]^d$ so \emph{a fortiori} for all $x \in \reals^d_+$.  Therefore
\begin{align*}
&\abs{\int f \, d\mu_n - \int f \, d \mu} \\
&\leq \int \abs{f - \hat{f}}
\, d\mu_n + \int \abs{\hat{f} - g}\, d\mu_n + \abs{\int  g\, d\mu_n -
  \int g \, d\mu} + \int \abs{\hat{f} - g}\, d\mu + \int \abs{f - \hat{f}}
\, d\mu \\
&\leq 2M \epsilon + \epsilon + \abs{\sum_k c_k ( \mu_n(k) - \mu(k))} +
\epsilon + 2M \epsilon
\end{align*}
Now take the limit as $n \to \infty$ and use the fact that $\mu_n \to
\mu$ pointwise (recall that the above sum
over $k$ is finite) and then let $\epsilon \to 0$.
\end{proof}

The Cramer-Wold device for Laplace transforms is a simple corollary.
\begin{cor}[Cramer Wold Device]\label{CramerWoldDeviceLaplace}Let $\xi, \xi_1, \xi_2, \dotsc$ be 
  random vectors in $\reals^d_+$.  If $\langle c, \xi_n \rangle
  \todist \langle c, \xi \rangle$ for all $c \in \reals^d_+$ then it
  follows that $\xi_n \todist \xi$.
\end{cor}
\begin{proof}
Since  $\langle c, \xi_n \rangle  \todist \langle c, \xi \rangle$ we
know that that $\expectation {e^{-\langle c, \xi_n \rangle } } \to
\expectation{e^{-\langle c, \xi \rangle}}$ for all $c \in \reals^d_+$
by definition of weak convergence
and therefore by Theorem \ref{GlivenkoLevyContinuityLaplace} we
conclude $\xi_n \todist \xi$.
\end{proof}