\chapter{Likelihood Theory}
TODO:
\begin{itemize}
\item[(i)] Definition of Likelihood function
\item[(ii)] Definition of Maximum Likelihood estimate
\item[(iii)] Fisher information: regularity conditions (FI and Le Cam), score function
  and information matrix; information matrix as Riemannian metric on
  manifold of parameters
\item[(iv)] Cramer-Rao Lower Bound
\item[(v)] Asymptotic distribution/Asymptotic Normality : Delta Method and Second Order Delta Method
\item[(vi)] Asymptotic consistency of MLEs 
\item[(vii)] Asymptotic efficiency of MLEs
\item[(viii)] Hypothesis testing with MLE: Likelihood Ratio Tests 
  Wilks Theorem(Schervish Thm 7.125, van der Vaart 16.9), Wald Tests and Score Tests
\item[(ix)] Problems with boundaries lack of regularity
\item[(x)] M-estimators
\item[(xi)]Observed information matrix...
\end{itemize}

As a quick motivation for where maximum likelihood estimation comes
from, consider the following measure of distance between two
probability distributions that was motivated by information theory.
\begin{defn}Suppose $\mu$ and $\nu$ such that $\mu \ll \nu$.  The the \emph{Kullback-Liebler divergence} or \emph{relative
    entropy} of $\mu$ and $\nu$ is defined as
\begin{align*}
\kldiv{\mu}{\nu} &= \sexpectation{\log \frac{d\mu}{d\nu}}{\mu}
\end{align*}
If $\mu$ is not absolutely continuous with respect to $\nu$ then by
convention $\kldiv{\mu}{\nu} = \infty$.
\end{defn}

\begin{examp}
Suppose $\mu$ and $\nu$ are probability measures that are both
absolutely continuous with respect to a third measure $\lambda$ and
furthermore $\mu \ll \nu$.  Then we may write $\mu = f \cdot \lambda$
and $\nu = g \cdot \lambda$ where we assume that $\lambda$- almost
surely $g=0$ implies $f=0$ (otherwise the event $A=\lbrace g=0; f>0 \rbrace$
satisfies $\nu(A)=0$ but $\mu(A)\neq 0$).  In this case we can make sense of the
ratio $\frac{f}{g}$ if we agree that $\frac{0}{0} = 0$ and then $\frac{d\mu}{d\nu} = \frac{f}{g}$.

In this case we get the formula 
\begin{align*}
\kldiv{\mu}{\nu} &= \int \log(\frac{f}{g}) f \, d\lambda
\end{align*}
that the user may have encountered before.
\end{examp}

\begin{examp}\label{RelativeEntropyFiniteSampleSpace}
Suppose $\mu$ and $\nu$ are probability measures on a finite state
space $\Omega = \lbrace a_1, \dotsc, a_n \rbrace$.  Let $\mu_j =
\mu(a_j)$ and $\nu_j = \nu(a_j)$ for $j=1, \dotsc, n$.  Both $\mu$ and
$\nu$ are absoutely continuous with respect to the counting measure on
$\Omega$.  Furthermore $\mu \ll \nu$ if and only if $\nu_j = 0$
implies $\mu_j=0$.  From the previous example  
\begin{align*}
\kldiv{\mu}{\nu} &= \sum_{j=1}^n \mu_j \log(\frac{\mu_j}{\nu_j}) 
\end{align*}
where again we have the convention that $\log(\frac{0}{0}) = 0$.
\end{examp}

\begin{examp}One interpretation of relative entropy is that is the
  number of bits of information that one gains updating ones that
  belief that a probability distribution is $\nu$ to a belief that a
  probability distribtuion is $\mu$.  The following simple example
  illustrates the point.  In what follows we interpret $\log$ to be
  the base 2 logarithm as opposed to the standard assumption that it
  represents the natural logarithm.  Suppose you believe that a coin is fair.  In
  this case you believe that the distribution is $\nu(H) = \nu(T) =
  1/2$.  If someone tells you that the coin is a trick coin that only
  lands with heads up then you change belief to $\mu(H) = 1$ and
  $\mu(T)=0$.  It is easy to see that $\mu \ll  \nu$ and using the formula for relative entropy in terms of
  densities in the previous example we compute
\begin{align*}
\kldiv{\mu}{\nu} &= \log(\frac{1}{1/2}) \cdot 1 + \log
(\frac{0}{1/2}) \cdot 0 = \log 2 = 1
\end{align*}
Thus one has gained $1$ bit of information; which is intuitively
correct because on updating one's view of the probability distribution
one has learned the outcome of a single binary trial.

It is also instructive to consider the example with the roles of $\mu$
and $\nu$ reversed.  In this case $\mu(T) = 0$ but $\nu(T) \neq 0$
hence $\nu$ is not absolutely continuous with respect $\mu$ and
therefore we have agreed that the relative entropy is infinite.  The convention
is corroborated by the heuristic calculation
\begin{align*}
\kldiv{\nu}{\mu} &= \log(\frac{1/2}{1}) \cdot \frac{1}{2} + \log
(\frac{1/2}{0}) \cdot \frac{1}{2} = \infty
\end{align*}
The intuition here is that in going from $\mu$ to $\nu$ we are
learning that something that was formerly thought to be impossible is
in fact possible and that the information gained from this is
infinitely large.  Along the lines of this example one will often hear
the relative entropy referred to as \emph{information gain} :
particularly in the machine learning literature.
\end{examp}

\begin{lem}[Gibbs Inequality]\label{GibbsInequality}For all
  probability distributions $\mu$ and $\nu$, $\kldiv{\mu}{\nu} \geq 0$
  with equality if and only if $\mu$ and $\nu$ agree except on a set
  of measure zero with respect to $\nu$.
\end{lem}
\begin{proof}
It suffices to handle the case in which $\nu \ll \mu$.  In this case we
can simply use the strict convexity of $x \log x$ and apply Jensen's
inequality and the definition of the Radon-Nikodym derivative to see
\begin{align*}
\kldiv{\mu}{\nu} &= \sexpectation{\log \frac{d\mu}{d\nu}}{\mu} = \sexpectation{\frac{d\mu}{d\nu}\log \frac{d\mu}{d\nu}}{\nu} \geq
\sexpectation{\frac{d\mu}{d\nu}}{\nu} \log \sexpectation{\frac{d\mu}{d\mu}}{\nu} = 
\sexpectation{1}{\mu} \log \sexpectation{1}{\mu} = 0
\end{align*}
By strict convexity of $x \log x$, we have equality if and only if $\frac{d\mu}{d\nu}$ is
almost surely (with respect to $\nu$) a constant.  This constant must be $1$ because $\mu$ and
$\nu$ are both probability measures.  
\end{proof}

\begin{examp}Continuing the previous example we specialize to case in
  which we consider a family of densities indexed by a set $\Theta$.
  Specifically for each $\theta \in \Theta$, we suppose we have a
  density $f(x \mid \theta)$ with respect to a base measure
  $\lambda$.  The problems of (parametric) statistical estimation generally start
  with such an assumption and and assume there is distinguished
  \emph{true} value $\theta_0$ from among the elements of the set
  $\Theta$.  Lemma \ref{GibbsInequality} suggests a potential path.  We know
  from the previous example that 
\begin{align*}
\kldiv{\theta_0}{\theta} &= \sexpectation{ \log( \frac{f(x \mid \theta_0)}{f(x
  \mid \theta)})}{\theta_0} = \sexpectation{ \log( f(x \mid \theta_0))
}{\theta_0} - \sexpectation{ \log( f(x \mid \theta))
}{\theta_0} \geq 0
\end{align*}
with equality if an only $f(x \mid \theta_0)$ and $f(x \mid \theta)$
give the same measure (which we generally assume to imply that
$\theta_0 = \theta$; a condition referred to as
\emph{identifiability}).  So this means that $\sexpectation{ \log( f(x \mid \theta))
}{\theta_0}$ has a unique maximum at the value $\theta_0$.  Now this
isn't of much use directly since it assume knowledge of the density
$f(x \mid \theta_0)$ in order to compute the expectations, but it
suggests that we should consider using an approximation of the measure
defined by the density such as one defined by sampling and consider
contexts in which we maximize the function $f(x \mid \theta)$
considered as a function of $\theta$.  This insight leads to the
method of maximum likelihood which we shall study in some detail in
the following chapter.  
\end{examp}

Now we apply this idea in the context of parametric estimation.  If we
suppose that we are given a parametric family of densities $f(x;
\theta)$ relative to some measure $\nu$.

TODO: To be continued...

\section{The Delta Method}

TODO: Move the discussion of tightness into the convergence chapter.

\begin{defn}Given a metric space $(S,d)$ and arbitrary index set $A$, a set of random elements
  $\xi_\alpha$ in $S$ with $\alpha \in A$ is said to be \emph{tight} if for
  every $\epsilon > 0$ there exists a compact set $K \subset S$ such
  that $\sup_\alpha \probability{\xi_\alpha \notin K} < \epsilon$.  In
  the case in which $\xi_\alpha$ are random vectors in some $\reals^n$
  it is also common to that a tight set of random vectors is
  \emph{bounded in probability}.
\end{defn}

Just as with convergence in distribution, note that tightness is
really a property of the law of the random elements $\xi_\alpha$.  
We will eventually see that tightness is a type of sequential
compactness; if one goes a bit farther than we intend to go, one can in
fact show that there is a metric on the space of measures (the
Levy-Prohorov metric which metrizes convergence in distribution) and
that tight sets are relatively compact sets of
measures in the corresponding metric space (are all compact sets tight???).

The first thing that we shall see about tightness is the fact that
sequence that converge in distribution are tight.

\begin{lem}\label{WeakConvergenceImpliesTight}Suppose $\xi_n \todist
  \xi$ with $\xi, \xi_1, \xi_2, \dots$ random vectors, then $\xi_n$ is a tight sequence.
\end{lem}
\begin{proof}
TODO: Can we use Portmanteau and clean up the argument by making the
continuous approximation unnecessary?  Answer is certainly yes but
it's not clear how much simpler it makes the argument.

Suppose we are given an $\epsilon > 0$.  First since $\xi$ is almost surely finite, continuity of measure shows
that $\lim_{M \to \infty} \probability{\abs{\xi} > M} = 0$ and
therefore we can find $M_1 > 0$ such $\probability{\abs{\xi} > M_1} <
\frac{\epsilon}{2}$.   Now pick an arbitrary $M_2 > M_1$ and let $f$ be a
bounded continuous function such that $\characteristic{\abs{x} > M_2}
\leq f \leq \characteristic{\abs{x} > M_1}$.  Then we have 
\begin{align*}
\probability{\abs{\xi_n} > M_2} &\leq \expectation{f(\xi_n)} 
\end{align*}
and 
\begin{align*}
\expectation{f(\xi)} \leq \probability{\abs{\xi} > M_1} \leq \frac{\epsilon}{2}
\end{align*}
but also we can find $N > 0$ such that $\abs{\expectation{f(\xi_n)} -
  \expectation{f(\xi)} } < \frac{\epsilon}{2}$ for all $n \geq
N$.  Putting the pieces together we have for all $n \geq N$, 
\begin{align*}
\probability{\abs{\xi_n} > M_2} &\leq \expectation{f(\xi_n)} \leq
\expectation{f(\xi)} + \abs{\expectation{f(\xi_n)} -
  \expectation{f(\xi)}} < \epsilon
\end{align*}
Now for each $0 \leq n \leq N$, we can find $M^\prime_n$ such that
$\probability{\abs{\xi_n} > M^\prime_n} < \epsilon$, so if we take
$M=\max(M_2, M^\prime_1, \dots, M^\prime_{n-1})$ then we get
$\sup_n\probability{\abs{\xi_n} > M} < \epsilon$ and tightness is shown.
\end{proof}
\begin{lem}\label{ScaledTightSequenceConvergeZeroProb}Suppose $r_n$ is a sequence of real numbers such that
  $\lim_{n \to \infty} \abs{r_n} = \infty$ and
  $\eta, \xi, \xi_1, \xi_2, \dots$ is a sequence of random vectors such that $r_n(\xi_n -
  \xi) \todist \eta$.  Then $\xi_n \toprob \xi$.
\end{lem}
\begin{proof}
The proof only relies on the fact that $r_n(\xi_n -  \xi)$ is a
tight sequence (Lemma \ref{WeakConvergenceImpliesTight}).  Suppose we
are given $\epsilon, \delta > 0$.  By
tightness, we can pick $M > 0$ such that 
\begin{align*}
\sup_n \probability{\abs{r_n(\xi_n - \xi)} > M} &= \sup_n \probability{\abs{\xi_n - \xi} > \frac{M}{\abs{r_n}}} < \delta
\end{align*}
Because $\lim_n \abs{r_n} = \infty$ we pick $N>0$ such that
$\frac{M}{\abs{r_n}} \leq \epsilon$ for $n \geq N$.  Then 
\begin{align*}
\probability{\abs{r_n(\xi_n - \xi)} > \epsilon} &\leq \sup_n \probability{\abs{\xi_n - \xi} > \frac{M}{\abs{r_n}}} < \delta
\end{align*}
for $n \geq N$ and we have show $\xi_n \toprob \xi$.
\end{proof}

In this result we have restricted ourselves to random vectors in
$\reals^n$ because it is an important special case (especially in
parametric statistics) and because it is a trivial matter to show that all random
vectors are tight.  Generalization to arbitrary metric spaces is
subtle because it is no longer the case that an arbitrary random
element is tight.  One can repair the argument above by adding the
assumption that the elements of the sequence are tight random elements
or one can explore what conditions on a metric space guarantee that
all random elements are tight.  Though we don't go into it at the
moment, it turns out separability and completeness (i.e. Polishness) 
are sufficient to guarantee tightness of arbitrary random
elements and there is also a more subtle necessary and sufficient
condition that has been identified (universal measurability see
Dudley's RAP).

Part of the importance of tightness is lies in its role as a
compactness property (that is to say the fact that it implies weak
convergence of a subsequence).  On the other hand, in some cases one
uses only the boundedness aspect.  This is particularly true in
asymptotic statistics.  TODO: Introduce the $O_P(r_n)$ and $o_p(r_n)$
notation.
\begin{lem}\label{AlgebraOfStochasticConvergence}Let $\xi_1,
  \xi_2, \dots$ and
  $\eta_1, \eta_2, \dots$ be sequences of random vectors.
\begin{itemize}
\item[(i)]If $\xi_n \toprob 0$ then $\xi_n$ is tight. ($o_p(1) = O_P(1)$).
\item[(ii)]If $\xi_n \toprob 0$ and $\eta_n \toprob 0$ then $\xi_n +
  \eta_n \toprob 0$.  ($o_P(1) + o_P(1) = o_P(1)$).
\item[(iii)]If $\xi_n$ is tight and $\eta_n \toprob 0$ then $\xi_n +
  \eta_n$ is tight.  ($O_P(1) + o_P(1) = O_P(1)$).
\item[(iv)]If $\xi_n$ is tight and $\eta_n \toprob 0$ then $\xi_n *
  \eta_n \toprob 0$ (this is true for many kinds of multiplication;
  scalar multiplication, dot product, matrix multiplication). ($O_P(1)
  o_P(1) = o_P(1)$.
\item[(v)]If $\eta_n$ is tight sequence of random variables and
  $\xi_n \eta_n \toprob 0$ then $\xi_n \toprob 0$. ($o_P(O_P(1)) = o_P(1)$).
\end{itemize}
\end{lem}
\begin{proof}
To prove (i) simply note that $\xi_n \toprob 0$ implies $\xi_n
\todist 0$ (Lemma
\ref{ConvergenceInProbabilityImpliesConvergenceInDistribution}) the
therefore we know $\xi_n$ is tight by Lemma
\ref{WeakConvergenceImpliesTight}.

The statement of (ii) is a corollary to the Continuous Mapping Theorem
(Corollary \ref{ConvergenceInProbabilityAndAlgebraicOperations}).

TODO: Finish...
\end{proof}

Here is a slightly more involved fact that we shall use in the sequel.
\begin{lem}\label{InvertMatrixInProbability}Let $\Psi_n$ be a sequence of
  random matrices such that $\Psi_n \toprob \Psi$ with $\Psi$ almost surely
  equal to a constant nonsingular matrix.  Suppose $\xi_n$ is a
  sequence of random vectors such that $\Psi_n \xi_n$ is tight, then
  $\xi_n$ is tight.
\end{lem}
\begin{proof}
Recall that because convergence in
probability only depends on the underlying topology induced by a
metric (Corollary \ref{ConvergenceInProbabilityIndependentOfMetric})
and that all norms on a finite dimensional vector space are
equivalent; this means that we are free to choose the operator norm when dealing with
the convergence of the matrices $\Psi_n$.  

We remind the reader of some basic facts about the operator norm.  
In any normed vector space of linear operators with the operator norm
we have Neumann series for inverting perturbations of the identity
operator.  Specifically for any $A$ with $\norm{A} < 1$, we have
\begin{align*}
(1 - A)^{-1} &= \sum_{n=0}^\infty A^n & & \text{converges absolutely} \\
\norm{(1 - A)^{-1}} &\leq \sum_{n=0}^\infty \norm{A^n} \leq 
\sum_{n=0}^\infty \norm{A}^n = (1 - \norm{A})^{-1} \\
(1-A) (1 - A)^{-1} &= \sum_{n=0}^\infty A^n - \sum_{n=1}^\infty A^n =
1 \\
(1 - A)^{-1}(1-A) &= \sum_{n=0}^\infty A^n - \sum_{n=1}^\infty A^n =
1 \\
\end{align*}
which shows that $(1-A)$ is invertible with inverse $(1-A)^{-1}$
defined by the Neumann series.  We now extend this argument
to show there is an  open neighborhood of any invertible operator in
the space of invertible operators.  Suppose $T$ is invertible and let
$\norm{T - A} < \frac{1}{\norm{T^{-1}}}$.  Then we can write $T - A =
T(1 - T^{-1}A)$ where $\norm{T^{-1}A} \leq \norm{T^{-1}}\norm{A} < 1$
so that $(1-T^{-1}A)$ is invertible.  This shows $T-A$ is product of
invertible operators hence is itself invertible.  Moreover we have the
norm bound
\begin{align*}
\norm{(T-A)^{-1}} &\leq \norm{T}\norm{(1 - T^{-1}A)^{-1}} \leq
\frac{\norm{T}}{1 -
\norm{T^{-1}A}} \leq \frac{\norm{T}}{1 -
\norm{T^{-1}}\norm{A}}
\end{align*}

With that little piece of operator theory out of the way we can return
statistics proper.  We have assumed $\Psi_n \toprob
\Psi$ with $\Psi$ an invertible a.s. constant matrix.  Pick $\delta > 0$ and $0
< \epsilon <
\frac{1}{2\norm{\Psi^{-1}}}$, then we know that
there exists an $N > 0$ such that $\probability{\norm{\Psi_n - \Psi} \leq \epsilon} \geq
1 - \frac{\delta}{2}$ for all $n > N$.  By the preceeding
discussion we know that whenever $\norm{\Psi_n - \Psi} \leq \epsilon$,
  $\Psi_n$ is invertible and $\norm{\Psi_n^{-1}} < 2 \norm{\Psi^{-1}}$.  By
  tightness of $\Psi_n \xi_n$ we can find $M > 0$ such that 
\begin{align*}
\sup_n \probability{\norm{\Psi_n \xi_n}>M} < \frac{\delta}{2}
\end{align*}
Therefore by applying the inverse of $\Psi_n$ and using
its operator norm bound we get
\begin{align*}
\sup_{n>N} \probability{\norm{\xi_n}>2M \norm{\Psi^{-1}}}< \delta
\end{align*} 
Because random vectors in $\reals^n$ are tight, we know that there is
an $M^\prime$ such that $\probability{\norm{\xi_n}>M^\prime} < \delta$ for all $0 < n \leq N$ and
therefore $\xi_n$ is tight.
\end{proof}

\begin{defn}Given an open set $U \subset \reals^m$ and function $\phi :
  U\to \reals^n$  we say that $\phi$ is \emph{Frechet differentiable} at a
  point $x\in U$ if there is a linear map $A : \reals^m \to
  \reals^n$ such that for every sequence $h_n \in \reals^m$ such that
$\lim_{n \to \infty} \abs{h_n} = 0$ we have 
\begin{align*}
\lim_{n \to \infty}
  \frac{\phi(x + h_n) - \phi(x) - A h_n}{\abs{h_n}} &= 0 
\end{align*}
The linear map $A$ is called the \emph{Frechet derivative} of $\phi$
at $x$ is usually written $D\phi(x)$.
\end{defn}

\begin{thm}[Delta Method]\label{DeltaMethod}Let $\phi : D \subset
  \reals^k \to \reals^m$ be Frechet differentiable at $\theta \in D$.  Let
  $\xi, \xi_1, \xi_2, \dots$ be random vectors with values in $D$ and $r_n$ be a sequence
  of real numbers such that $\lim_{n \to \infty} r_n = \infty$ and
  $r_n(\xi_n - \theta) \todist \xi$.  Then 
\begin{align*}
r_n(\phi(\xi_n) -
  \phi(\theta)) \todist D\phi(\theta) \xi
\end{align*}
and moreover 
\begin{align*}
\abs{r_n(\phi(\xi_n) - \phi(\xi)) - D\phi(\theta) r_n(\xi_n - \theta)} \toprob 0
\end{align*}
\end{thm}
\begin{proof}
By Lemma \ref{ScaledTightSequenceConvergeZeroProb} be know that $\xi_n
- \theta
\toprob 0$.  By differentiability of $\phi$ we know that  for every
sequence $h_n \to 0$,
\begin{align*}
\lim_n \frac{\phi(\theta + h_n) - \phi(\theta) - D\phi(\theta)h_n}{\abs{h_n}} = 0
\end{align*}

The first thing to show is that we can extend this fact to random
sequences.  We state this as a general fact.  Suppose $\psi(x)$ is a
function such that for every $h_n \to 0$ we have
$\frac{\psi(h_n)}{\abs{h_n}} \to 0$.  We claim that if we are given
random vectors $\eta_n$ such that $\eta_n \toprob 0$ then
$\frac{\psi(\eta_n)}{\abs{\eta_n}} \toprob 0$.  To see this define a
new function by 
\begin{align*}
f(x) &= \begin{cases}
\frac{\psi(x)}{\abs{x}} & \text{for $x \neq 0$} \\
0 & \text{for $x=0$}
\end{cases}
\end{align*}
and note that by assumption $f$ is continuous at $0$.  Now by the
Continuous Mapping Theorem (Theorem \ref{ContinuousMappingTheorem}) we know that $f(\eta_n)
\toprob f(0) = 0$.

Having shown the above fact, we can use $\xi_n - \theta \toprob 0$ to
conclude
\begin{align*}
\frac{\phi(\xi_n) - \phi(\theta) - D\phi(\theta) (\xi_n -
  \theta)}{\abs{\xi_n - \theta}} &\toprob 0
\end{align*}
and if we multiply top and bottom by $r_n$ and use linearity of the
Frechet derivative we get
\begin{align*}
\frac{r_n(\phi(\xi_n) - \phi(\theta)) - D\phi(\theta) r_n(\xi_n -
  \theta)}{\abs{r_n(\xi_n - \theta)}}  &\toprob 0
\end{align*}
Tightness of $r_n(\xi_n - \theta)$ allows us to conclude that 
\begin{align*}
r_n(\phi(\xi_n) - \phi(\theta)) - D\phi(\theta) r_n(\xi_n -  \theta)  \toprob 0
\end{align*}
which gives us the second conclusion of the Theorem.

To prove this last fact suppose $\xi_n, \eta_n$ are random vectors
such that $\frac{\xi_n}{\abs{\eta_n}} \toprob 0$ and $\eta_n$ is
tight.  Suppose we are given $\epsilon, \delta > 0$.  Use tightness to
pick an $M>0$ such that $\sup_n \probability{\abs{\eta_n} > M} <
\frac{\delta}{2}$ and use $\frac{\xi_n}{\abs{\eta_n}} \toprob 0$ to
pick an $N$ such that $\probability{\abs{\frac{\xi_n}{\eta_n}} >
  \frac{\epsilon}{M}} < \frac{\delta}{2}$ for all $n \geq N$.
Then
\begin{align*}
\probability{\abs{\xi_n}>\epsilon} &=
\probability{\abs{\xi_n}>\epsilon ; \abs{\eta_n} > M} +
\probability{\abs{\xi_n}>\epsilon ; \abs{\eta_n} \leq M} \\
&\leq \probability{\abs{\eta_n} > M} +
\probability{\frac{\abs{\xi_n}}{\abs{\eta_n}}>\frac{\epsilon}{M}} \\
&< \delta
\end{align*}
for all $n \geq N$ which shows $\xi_n \toprob 0$.  TODO: Is it better to
think of this as $O_P(1) o_P(1) = o_P(1)$; probably better to think of
this as $o_P(O_P(1)) = o_P(1)$?

To get the first conclusion we simply use the fact that matrix
multiplication is continuous and the Continuous Mapping Theorem
(Theorem \ref{ContinuousMappingTheorem}) to see
that $D\phi(\theta) r_n(\xi_n - \theta) \todist D\phi(\theta) \xi$ and Slutsky's
Lemma (Lemma \ref{Slutsky})) and the part of this Theorem just proven
to conclude $r_n(\phi(\xi_n) - \phi(\theta)) \todist D\phi(\theta) \xi$.
\end{proof}

\begin{examp}One of the most common problems in statistics is the
  comparison of binomial populations.  For example, to estimate
  treatment effectiveness one might want to compare the proportion of
  postive responses between a treated group and a control group.  One
  common way to estimate the difference in proportions between two
  independent populations is the \emph{risk ratio}
\begin{align*}
\hat{RR} &= \frac{\hat{p}_1}{\hat{p}_2}
\end{align*}
where $\hat{p}_i$ denotes the sample proportion.  Here we calculate
the asymptotic distribution of the risk ratio by using the Delta method.

The trick is to apply a logarithm to convert the division into
subtraction.  First we consider a single sample proportion $\hat{p}$.  Since
$\hat{p} = \frac{1}{n} \sum_n \xi_i$ for $\xi_i$ a Bernoulli random
variable with rate $p$, we can apply the Central Limit Theorem to
conclude that 
\begin{align*}
\sqrt{n} (\hat{p} - p) &\todist N(0, p(1-p))
\end{align*}
Assuming $p \neq 0$, the Delta Method (Theorem \ref{DeltaMethod}) yields
\begin{align*}
\sqrt{n} (\ln (\hat{p}) - \ln(p)) &\todist \frac{1}{p}N(0, p(1-p)) =
  N(0, \frac{1-p}{p})
\end{align*}
Therefore is we apply this reasoning to the risk ratio and use the
fact that a sum of independent normal random variables is normal, we
see that 
\begin{align*}
\sqrt{n} (\ln (\hat{RR}) - \ln(RR)) &\todist N(0, \frac{1-p_1}{p_1} + \frac{1-p_2}{p_2})
\end{align*}

This result can then be used to create asymptotic confidence intervals
for the estimation of risk ratio 
\begin{align*}
\ln(\hat{p}_1/\hat{p}_2) \pm z_{\alpha/2}\sqrt{\frac{1-\hat{p}_1}{n_1
  \hat{p}_1} + \frac{1-\hat{p}_2}{n_2 \hat{p}_2}}
\end{align*}


TODO: Discuss the implications of substituting the variance estimate
into this formula.
\end{examp}
TODO: Lay down the conceptual framework in which parametric statistics
is modeled.  Basic problem statement is this.  Assume that one has a
probability space $(\Omega, \mathcal{A}, P)$ and a family of random
elements $\xi_\theta$ in a measure space $(X, \mathcal{X}, \mu)$ with
$\theta \in \Theta$ an unknown parameter that determines the
distribution of $\xi_\theta$.   Assume we make observations of the
value of $\xi$ (or more properly observations of generally independent
random variables with the same distribution as $\xi$), we want to find an estimate of the value (or the
distribution) of $\theta$.

There is the subtlety around the notion of
having a random variable $\xi$ with \emph{conditional density}
$f(x\mid \theta)$.  The question is how rigorously one needs to think
about the parameter $\theta$.  In the simplest form, one can just
think of having a family of random variables $\xi_\theta$ for $\theta
\in \Theta$ and not concern oneself with measurability in $\theta$.
This seems to be sufficient when discussing frequentist methods for
example.  Note also that the notation $f(x\mid \theta)$ seems to hedge
on how we want to think of the functional dependence on $\theta$.
We'll see that understanding the dependence on $\theta$ is important but doesn't map
nicely to standard probabilisitc or measure theoretic notions and has
its own somewhat idiosyncratic notions of regularity.
In the Bayesian formulation it appears that one wants to
view $\theta$ as a random quantity as well and one assumes the
existence of a random element $\theta$ in $\Theta$ and a random
element $\xi$ in $X$ and take the conditional distribution
$P_\theta = \probability{\xi \in \cdot \mid \theta}$.  Then one assumes that the
conditional distributions are all absolutely continuous with respect
to $\mu$ and thereby get the conditional densities $f(x \mid
\theta)$ such that $P_\theta = f(x \mid \theta) \cdot \mu$.  It is not yet clear to me at what point one is forced to
take the latter approach.

Here is one account of the FI regularity conditions.

\begin{defn}
Suppose we are given a measure space $(X, \mathcal{X}, \mu)$ and a family of
probability measures $P_\theta$ with $\theta \in
\Theta \subset \reals^n$ for some $n > 0$.  Suppose that  such that there
exist densities $f(x \mid \theta)$  for each $P_\theta$ with respect so $\mu$.  The $f(x
\mid \theta)$ are said to satisfy the \emph{FI regularity constraints}
if the following are true:
\begin{itemize}
\item[(i)] $\Theta \subset \reals^n$ is convex and contains and open
  set.  There exists a set $B \in \mathcal{X}$ with $\mu(B^c) = 0$ such
  that $\frac{\partial}{\partial \theta_i} f(x \mid \theta)$ exists
  for every $i=1, \dots, n$, every $\theta \in \Theta$ and every $x
  \in B$.
\item[(ii)] For every $k = 1,\dots, n$, 
\begin{align*}
\frac{\partial}{\partial \theta_i} \int f(x \mid \theta)
  \, d \mu(x) = \int \frac{\partial}{\partial \theta_i} f(x \mid
  \theta) \, d \mu(x)
\end{align*}
\item[(iii)]The set $C = \lbrace x \in X \mid f(x \mid \theta) >
  0$ does not depend on $\theta$.
\end{itemize}
\end{defn}

\begin{defn}Let $\xi$ be a random element in the measure space $(X,
  \mathcal{X}, \mu)$ with conditional density
  $f(x \mid \theta)$ with respect to $\mu$.  Suppose that $f(x \mid
  \theta)$ satisfy the FI regularity constraints.  Then the random
  vector 
\begin{align*}
U(\xi \mid \theta) &= \left( \frac{\partial}{\partial \theta_1} \log f(\xi \mid \theta),
  \dots ,  \frac{\partial}{\partial \theta_n} \log f(\xi \mid \theta)\right)
\end{align*}
is called the \emph{score function}.
\end{defn}

The basic calculation with the score function is that if we assume
that $\xi$ is a random element with density $f(x \mid \theta)$ then
\begin{align*}
\sexpectation{\frac{\partial}{\partial \theta_i} \log f(\xi \mid
  \theta)}{\theta} &= \int \frac{\frac{\partial}{\partial \theta_i}  f(x \mid \theta)}{f(x \mid
  \theta)} f(x \mid \theta) \, d\mu(x) \\
&=\int \frac{\partial}{\partial \theta_i}  f(x \mid \theta) \, d\mu(x) \\
&=\frac{\partial}{\partial \theta_i}\int f(x \mid \theta) \, d\mu(x) =
\frac{\partial}{\partial \theta_i} 1 = 0
\end{align*}
and therefore $\sexpectation{U(\xi \mid \theta)}{\theta} = 0$ under
the FI regularity constraints.

If we differentiate both side of this latter equality 
\begin{align*}
0 &= \frac{\partial}{\partial \theta_j} \int \frac{\partial}{\partial
  \theta_i}\log f(x \mid  \theta) f(x \mid \theta) \, d\mu(x) \\
&=\int \frac{\partial^2}{\partial
  \theta_i\partial
  \theta_j}\log f(x \mid  \theta) f(x \mid \theta) + \frac{\partial}{\partial
  \theta_i}\log f(x \mid  \theta) \frac{\partial}{\partial
  \theta_j} f(x \mid  \theta)\, d\mu(x) \\
&=\int (\frac{\partial^2}{\partial
  \theta_i\partial
  \theta_j}\log f(x \mid  \theta) + \frac{\partial}{\partial
  \theta_i}\log f(x \mid  \theta) \frac{\partial}{\partial
  \theta_j} \log f(x \mid  \theta)) f(x \mid \theta)\, d\mu(x) \\
\end{align*}
which shows that when $\xi$ has density $f(x \mid \theta)$, we have
the identity
\begin{align*}
-\sexpectation{\frac{\partial^2}{\partial
  \theta_i\partial
  \theta_j}\log f(\xi \mid  \theta) }{\theta} = \sexpectation{\frac{\partial}{\partial
  \theta_i}\log f(\xi \mid  \theta) \frac{\partial}{\partial
  \theta_j} \log f(\xi \mid  \theta)}{\theta}
\end{align*}
This quantity is called the \emph{Fisher information matrix}.   TODO:
The Fisher information as a Riemannian metric on $\Theta$.

TODO: What kind of object is the score function (i.e. what domain and
range).  More specifically, how does one think of the $\theta$ dependence in the score
function?  In the Bayesian formulation everything is fine because
$\xi$ is an honest random element and we are just composing it with a
deterministic function.  In the formulation in which we don't think of
$\theta$ as being random, then are we thinking of $\xi$ as having
$\theta$-dependence when we plug it in?  The answer to this is YES. 

\begin{examp}Let $\xi$ be a parameteric Gaussian family with
  $\theta=(\mu, \sigma)$.  Then $f(x \mid \theta) =
  \frac{1}{\sqrt{2\pi\sigma^2}} e^{\frac{-(x-\mu)^2}{2\sigma^2}}$ and
  $U(\xi | \theta) = \frac{\xi - \mu}{\sigma^2}$.
\end{examp}

\begin{defn}Let $\xi$ be a random element with conditional density
  $f(x \mid \theta)$ with respect to a measure space $(X,
  \mathcal{X}, \mu)$.  For every $x \in X$, the function 
\begin{align*}
L(\theta) &= f(x \mid \theta)
\end{align*}
is called the \emph{likelihood function}.

Any random element $\hat{\theta}$ in $\Theta$ that satisfies
\begin{align*}
\max_{\theta \in \Theta} f(\xi \mid \theta) = f(\xi \mid \hat{\theta})
\end{align*}
is called a \emph{maximum likelihood estimator} of $\theta$.
\end{defn}

It is important to note that in most statistical applications the
random element $\xi$ whose likelihood we are investigating is a random
vector that corresponds to sampling from a population.  This is to say
that is some underlying distribution of interest that corresponds to
some random element $\xi$ and that we model repeated sampling as a
random element $\sample{\xi} = (\xi_1, \dotsc, \xi_n)$ in a product
space $\mathcal{X}^n$.  In all cases we shall be concerned about for
the moment, we assume that the samples are i.i.d. hence the joint
density of the sample is just the product of the density of $\xi$.  In
some cases it may be convenient to emphasize that the likelihood
function is of such a form; in those cases we may choose to write
$L_n(\theta)$ for the sample likelihood.

The fact that likelihood functions for independent samples are products
is leveraged constantly in what follows and is in large part
responsible for the nice asymptotic properties of maximum likelihood
estimators.  To release the power of this fact we simply convert the
product into a sum by taking log and create the log likelihood.  Note that because the log is
monotonic, one can perform maximum likelihood estimation equally well
by taking maxima of the log likelihood.  We shall usually write
$\ell(x \mid \theta)$ to denote a log likelihood and the case of
i.i.d. samples we shall use a subscript to emphasize the
dependence on sample size $\ell_n(\sample{\xi} \mid \theta) =
\sum_{i=1}^n\log f(\xi_i \mid \theta)$.  The maximum likelihood estimator
associated with i.i.d. samples of size $n$ is denoted:
\begin{align*}
\hat{\theta}_n &= \max_{\theta \in \Theta} \sum_{i=1}^n \log f(\xi_i \mid \theta)
\end{align*}
and it is the estimator that we shall spend some time studying.  The
motivation behind this mechanism is that we know from the Gibbs
Inequality (Lemma \ref{GibbsInequality}) that the true parameter
$\theta_0$ is characterized as the maximum of $\sexpectation{\log f(x
  \mid \theta)}{\theta_0}$.  Now we can view $\hat{\theta}_n$ as the
result of substituting the (random) empirical measure in the
expectation.  To the extent that the empirical measure converges we
may hope that the estimator converges as well.  Less abstractly, we
know from the Strong Law of Large Numbers that
$\frac{1}{n}\sum_{i=1}^n \log f(\xi_i \mid \theta) \toas
\sexpectation{ \log f(x \mid
  \theta)}{\theta_0}$ so thinking of this as convergence of functions
of $\theta$ we may hope that the convergence is strong enough so that
the maxima converge.

Note that the definition of the maximum likelihood estimator is using
the $\max$ and not the $\sup$; this means that in the case the
supremum is not actually attained on the set $\Theta$ (e.g. $\Theta$
is open and the supremum is attained on the boundary) then MLE may not
exist.  In some accounts of the theory, the maximum is taken over the
closure of the parameter domain (should we do this?)

\begin{examp}\label{MLENormalParameters}Consider the case parameter estimation in a normal
  distribution $\frac{1}{\sqrt{2\pi}\sigma} e^{-(x - \mu)^2 /
    2\sigma^2}$.  If we consider $\mu$ unknown and $\sigma$ known the
    the MLE for the mean is given by setting the derivative with
    respect to $\mu$ to be zero
\begin{align*}
\frac{\partial}{\partial \mu} \sum_{i=1}^n \log \frac{1}{\sqrt{2\pi}\sigma} e^{-(\xi_i - \mu)^2 /
    2\sigma^2} &= - \frac{1}{\sigma^2} \sum_{i=1}^n (\xi_i - \mu) = 0 
\end{align*}
which implies it is the sample mean $\hat{\mu}_n = \frac{1}{n}
\sum_{i=1}^n \xi_n$.

If we assume that $\mu$ is known and $\sigma$ is unknown the finding
the maximum by differentiation we get
\begin{align*}
\frac{\partial}{\partial \sigma} \sum_{i=1}^n \log \frac{1}{\sqrt{2\pi}\sigma} e^{-(\xi_i - \mu)^2 /
    2\sigma^2} &= \frac{1}{\sigma^3} \sum_{i=1}^n (\xi_i - \mu) - n \frac{1}{\sigma}= 0 
\end{align*}
and therefore the biased estimate of standard deviation $\hat{\sigma}_n = \frac{1}{n} \sum_{i=1}^n (\xi_n-\mu)^2$.
\end{examp}

TODO: Example of estimating the rate of a Bernoulli r.v.  Note the
boundary behavior.

TODO:  Example of $\xi$ as a random vector of independent observations
(factoring the likelihood function).

Note that we have allowed an MLE to be an arbitrary random element in
$\Theta$.  It makes intuitive sense however that the estimator should
depend on the value of $\xi$.  That is indeed the case in many cases
of interest and one of our goals shall be to understand the conditions
under which that dependence holds.

\begin{thm}If there is a sufficient statistic and the MLE exists, then
  the MLE is a function of the sufficient statistic.
\end{thm}
\begin{proof}
TODO: Apply the factorization theorem.
\end{proof}

TODO: Bring up the notion of \emph{identifiability}; clearly if the
likelihood function attains its maximum value for multiple values of
$\theta$ then it is subtle to describe what consistency means (which
is the correct value of $\theta$).

As we've seen in Example \ref{MLENormalParameters} we cannot expect
that maximum likelihood estimators will be consistent.  However it is
often the case that they will be asymptotically consistent.  
TODO: Define weakly and strongly asymptotically consistent.
The following theorem provides a set of sufficient conditions under
which a maximum likelihood estimator is strongly asymptotically consistent.
\begin{thm}[Asymptotic Consistency of MLE]\label{AsymptoticConsistencyMLE}Let $\xi, \xi_1, \xi_2, \dots$ be i.i.d. parametric
  family with distribution $f(x \mid \theta) \, d \mu$ with respect to measure
  space $(X, \mathcal{X}, \mu)$.  Assume that  $\theta_0$ is fixed and
  define 
\begin{align*}
Z(M, x) &= \inf_{\theta \in M} \log \frac{f(x \mid \theta_0)}{f(x \mid \theta)}
\end{align*}
Assume that for all $\theta \neq \theta_0$ there is an open neighborhood $U_\theta$
such that $\theta \in U_\theta$ and $\sexpectation{Z(U_\theta, \xi)}{\theta_0} >
0$.

If $\Theta$ is not compact, assume that there is a compact $K \subset
\Theta$ such that $\theta_0 \in K$ and $\sexpectation{Z(\Theta
  \setminus K, \xi)}{\theta_0} > 0$.  Then 
\begin{align*}
\lim_{n \to \infty} \hat{\theta}_n = \theta_0
\end{align*}
almost surely with respect to $P_{\theta_0}$.
\end{thm}

Before starting in on the proof make sure to understand the nature of
the hypotheses.  Given the observation $x$ we have $Z(U, x) < 0$ if
there is a $\theta \in U$ such that a $\theta$ this more likely than
$\theta_0$, whereas $Z(U, x) > 0$ tells us that $\theta_0$ is more
likely than any $\theta \in U$.  Thus the conditions $\sexpectation{Z(U_\theta, \xi)}{\theta_0} >
0$ are statements that on average there is no better explanation than
$\theta_0$.  One thing that is interesting about the result is that it
is only required that $\theta_0$ be the best average estimate locally
in $\Theta$ (admittedly the weakening to a local property is only
allowed over a compact set).

\begin{proof}
By Lemma \ref{ConvergenceAlmostSureByInfinitelyOften}, the Theorem is proven if we can show that
$\sprobability{ d(\hat{\theta}_n, \theta_0) \geq \epsilon \text{
  i.o.}}{\theta_0}= 0$
for every $\epsilon > 0$.  
So assume that we have fixed $\epsilon > 0$ and let $B(\theta_0,
\epsilon)$ be the $\epsilon$-ball around $\theta_0$.  Since $K
\setminus B(\theta_0, \epsilon)$ is compact and $U_\theta$ is a cover,
we can find an finite subcover $U_1, \dots, U_{m-1}$ of $K
\setminus B(\theta_0, \epsilon)$ such that each $U_j$ satisfies
$\sexpectation{Z(U_j, \xi)}{\theta_0} > 0$.  If we define $U_m
= \Theta\setminus K$ then we by hypothesis have a finite cover $U_1, \dots,
U_m$ of $\Theta
\setminus B(\theta_0, \epsilon)$ with each $U_j$ satisfying the same
property.

Now on each $U_j$ we can apply the Strong Law of Large Numbers to
conclude that for each $j$, $\frac{1}{n} \sum_{i=1}^n Z(U_j, \xi_i)
\toas \sexpectation{Z(U_j, \xi)}{\theta_0} > 0$ a.s. The key
point from this point on is to understand that if we assume that
$\hat{\theta}_n \in U_j$ infinitely often it would force the
expectation $\sexpectation{Z(U_j, \xi)}{\theta_0}$ to be nonpositive.  Precisely,
\begin{align*}
&\sprobability{\hat{\theta}_n \notin B(\theta_0, \epsilon) \text{
    i.o.}}{\theta_0} \\
&\leq \sprobability{\hat{\theta}_n
  \in \cup_{j=1}^m U_j \text{ i.o.}}{\theta_0} & & \text{since $B^c
  \subset \cup_{j=1}^m U_j $}\\
&= \sprobability{\cup_{j=1}^m \lbrace \hat{\theta}_n
  \in U_j \text{ i.o.} \rbrace }{\theta_0}  & & \text{by
    finiteness of $n$}  \\
&\leq \sum_{j=1}^m \sprobability{\hat{\theta}_n
  \in U_j \text{ i.o.}}{\theta_0} & & \text{by
  subadditivity} \\
&\leq \sum_{j=1}^m \sprobability{\inf_{\theta \in U_j} \sum_{i=1}^n
  \log \frac{f(\xi_i, \theta_0)}{f(\xi_i, \theta)} \leq 0 \text{
    i.o.}}{\theta_0} & & \text{because $\sum_{i=1}^n
  \log \frac{f(\xi_i, \theta_0)}{f(\xi_i, \hat{\theta}_n)} \leq 0$ } \\
&\leq \sum_{j=1}^m \sprobability{\sum_{i=1}^n \inf_{\theta \in U_j} 
  \log \frac{f(\xi_i, \theta_0)}{f(\xi_i, \theta)} \leq 0 \text{
    i.o.}}{\theta_0} \\
&= \sum_{j=1}^m \sprobability{\lim_{n \to \infty} \frac{1}{n}\sum_{i=1}^n \inf_{\theta \in U_j} 
  \log \frac{f(\xi_i, \theta_0)}{f(\xi_i, \theta)} \leq 0}{\theta_0}
\\
&= \sum_{j=1}^m \sprobability{\lim_{n \to    \infty} \frac{1}{n}
  \sum_{i=1}^n  Z(U_j, \xi_i) \leq 0 }{\theta_0}\\
&=0
\end{align*}
since as noted the last equality follows from fact that the Strong Law of Large Numbers tells us that almost
surely for all $1 \leq j \leq m$,
\begin{align*}
\lim_{n \to    \infty} \frac{1}{n} \sum_{i=1}^n  Z(U_j, \xi_i) &=
  \sprobability{Z(U_j, \xi)}{\theta_0} > 0
\end{align*}
\end{proof}

Note that the proof above has a gap in it from the outset.  The
functions $Z(M, x)$ for a fixed $M \subset \Theta$ are defined as an
infimum of an uncountable collection of random variables hence we do
not know that they are measurable.  On the other hand we clearly need
them to be in order to take expectations.  TODO:  How do we get around
these issues?  I suspect there are two paths to explore: 1) take a
countable dense subset and show that the infimum can be reduced to a
countable one or 2) abandon measurability and see if we can make due
with outer expectations (a la empirical process theory).  TODO: Check
with Charles Geyer's notes on MLE; I think he addresses this issue.

\begin{examp}
Consider the problem of estimating the parameter $\theta \in [0,
\infty)$ in the family $U(0, \theta)$.  Assume that $\theta_0$ is the
true parameter and we want to show consistency of the maximum
likelihood estimator.  The likelihood function in
this case is 
\begin{align*}
f(x \mid \theta) &= \begin{cases}
\frac{1}{\theta} & \text{if $0 \leq x \leq \theta$} \\
0 & \text{if $x < 0$ or $x > \theta$}
\end{cases}
\end{align*}
Note that you should be thinking of $f(x \mid \theta)$ as a function
of $\theta$ with $x$ fixed.  To apply Theorem
\ref{AsymptoticConsistencyMLE}
we need to show $\sexpectation{Z(U, \theta)}{\theta_0} > 0$ for
appropriately chosen $U \subset [0, \infty)$.  Since $\sprobability{x <
  0}{\theta_0} = \sprobability{x > \theta_0}{\theta_0} = 0$ for
purposes of computing the expectations we may assume that $0 \leq x
\leq \theta_0$.  With this in mind, for such an $x$, we have the likelihood ratio
\begin{align*}
\log \frac{f(x \mid \theta_0)}{f (x \mid \theta)} &= \begin{cases}
+\infty & \text{if $0 \leq \theta < x$} \\
\log \frac{\theta}{\theta_0} & \text{if $x \leq \theta$} 
\end{cases}
\end{align*}

So now we find our neighborhoods.  Pick $\theta > \theta_0$ and define
$U_\theta = (\frac{\theta + \theta_0}{2}, \infty)$ (any left hand
endpoint between $\theta_0$ and $\theta$ would suffice).  In this
case,
\begin{align*}
Z(U_\theta, x) &= \inf_{\psi > \frac{\theta + \theta_0}{2}} \frac{f(x
  \mid \theta_0)}{f(x \mid \theta)} = 
\inf_{\psi > \frac{\theta +    \theta_0}{2}} \log \frac{\psi}{\theta_0} =
\log \frac{\theta + \theta_0}{2 \theta_0} > 0
\end{align*}
therefore $\sexpectation{Z(U_\theta, x)}{\theta_0} > 0$.

If we pick $\theta < \theta_0$ then pick $U_\theta = (\theta/2,
\frac{\theta + \theta_0}{2})$ and note that 
\begin{align*}
Z(U_\theta, x) &= \begin{cases}
\log \frac{\theta}{2\theta_0} & \text{if $x \leq \frac{\theta}{2}$} \\
\log \frac{x}{\theta_0} & \text{if $\frac{\theta}{2} < x < \frac{\theta + \theta_0}{2}$} \\
+\infty & \text{if $\frac{\theta + \theta_0}{2} \leq x \leq \theta_0$}
\end{cases}
\end{align*} 
and therefore $\sexpectation{Z(U_\theta, x)}{\theta_0} = +\infty$.

Lastly we have to find a compact set $K$ such that
$\sexpectation{Z(\reals_+ \setminus K, x)}{\theta_0} > 0$.  Pick $a >
1$ and consider the interval $[\theta_0/a, a\theta_0]$.  Note that 
\begin{align*}
Z(\reals_+ \setminus [\theta_0/a, a\theta_0], x) &= \begin{cases}
\log \frac{x}{\theta_0} & \text{if $x < \frac{\theta_0}{a}$} \\
\log a & \text{if $\frac{\theta_0}{a} \leq x \leq \theta_0$} \\
\end{cases}
\end{align*}
so integrating,
\begin{align*}
\sexpectation{Z(\reals_+ \setminus [\theta_0/a, a\theta_0],
  x)}{\theta_0} &= 
\frac{1}{\theta_0} \int_0^{\frac{\theta_0}{a}} \log \frac{x}{\theta_0}
\, dx
+ \frac{\theta_0 - \frac{\theta_0}{a}}{\theta_0} \log a \\
&=(\frac{1}{a} \log \frac{1}{a} - \frac{\theta_0}{a}) + \frac{\theta_0 - \frac{\theta_0}{a}}{\theta_0} \log a \\
\end{align*}
Note that the first term goes to $0$ as $a$ goes to $\infty$ and the
second term goes to $\infty$ as $a$ goes to $\infty$ and therefore for
sufficiently large $a$ we have $\sexpectation{Z(\reals_+ \setminus [\theta_0/a, a\theta_0],
  x)}{\theta_0} > 0$.

Note also that as $a$ approaches $1$ the expectation approaches
$-\theta_0 \leq 0$.  In this specific sense if we allow
ourselves to consider regions of parameter space like $(\theta,
\theta_0+\epsilon)$ for $\epsilon > 0$ small, then under sampling we expect the there
is an estimate that is better (more likely) than the true parameter
value.  TODO: Think more carefully about this fact and how to
interpret it; should this disturb us?  Perhaps this shouldn't disturb
us because the thing that allows us to create these regions on which 
$\sexpectation{Z(U, x)}{\theta_0} < 0$ is precisely the fact that we
are allowing ourselves to include $\theta_0 \in U$; without allowing
that we can't create such a set.

The basic
phenomenon in this example can be summarized as:
\begin{itemize}
\item[(i)] Given a single observation $x$ then the
MLE is $x$ with likelihood $1/x$; any $\theta > x$ has strictly smaller
likelihood $1/\theta$ while any $\theta < x$ has likelihood $0$.
\item[(ii)] For any $\theta \geq \theta_0$ we know that $\theta_0$ is
  always a better estimator since we can only observe $x \leq
  \theta_0$ and for these observations $\theta_0$ is always better.
\item[(iii)] For any $\theta < \theta_0$ for any observations $x \leq
  \theta$ we know that $\theta$ is a better estimator than $\theta_0$
  by a finite factor, however for $\theta < x \leq \theta_0$ then
  $\theta_0$ is a infinitely better estimator than $\theta$.
\end{itemize}
\end{examp}

\begin{examp}
This example illustrates the difficulties that can arise in applying
the above results to conclude that an MLE is consistent when the
parameter space is not compact.  Consider a
normal family with parameter $\Theta = \lbrace (\mu, \sigma) \mid
\sigma > 0 \rbrace$ given by
$\frac{1}{\sqrt{2 \pi}\sigma} e^{-(x - \mu)^2/2\sigma}$.  We show that
for any compact $K \subset \reals^2$ we have $\sexpectation{Z(K^c, x)}{\theta_0}
= -\infty$ and therefore Theorem \ref{AsymptoticConsistencyMLE} does
not apply.  In fact we show that for any compact $K$ we have $Z(K^c,
x) = -\infty$.  This follows by noting that any compact $K$ is bounded
hence there exists a value of $\mu$ such that $\lbrace (\mu, \sigma)
\mid \sigma > 0\rbrace \subset K^c$.  Now we see that for such a
$\mu$, 
\begin{align*}
\lim_{\sigma \to 0^+} \frac{f(x\mid \mu_0, \sigma_0)}{f(x \mid \mu,
  \sigma)} &= \lim_{\sigma \to 0^+} \left( \log \sigma - \log \sigma_0
  - \frac{(x - \mu_0)^2}{2\sigma_0} + \frac{(x - \mu)^2}{2\sigma}
\right ) \neq -\infty
\end{align*}
TODO: Fix this argument; it is broken.  The limit is only negative
infinity when $x$ is large enough so that $\lbrace (x, \sigma) \mid
\sigma > 0 \rbrace \subset \Theta \setminus K$.  That should be enough
if we can show that the integral over the rest of the domain is not $+\infty$.

On the other hand, one can compute the MLE explicitly in this case and
verify that it is asymptotically consistent so we have shown that
conditions of the theorem are sufficient but not necessary.
\end{examp}

TODO: The following Theorem only requires upper semi-continuity.
\begin{thm}Let $\xi, \xi_1, \xi_2, \dots$ be i.i.d. parametric
  family with distribution $f(x \mid \theta) \, d \mu$ with respect to measure
  space $(X, \mathcal{X}, \mu)$.  Assume that  $\theta_0$ is fixed and
  define 
\begin{align*}
Z(M, x) &= \inf_{\theta \in M} \log \frac{f(x \mid \theta_0)}{f(x \mid \theta)}
\end{align*}
Assume that for all $\theta \neq \theta_0$ there is an open neighborhood $U_\theta$
such that $\theta \in U_\theta$ and $\sexpectation{Z(U_\theta, \xi)}{\theta_0} >
-\infty$.  Assume $f(x \mid \theta)$ is
  a continuous function of $\theta$ for almost all $x$ with respect to
  $P_{\theta_0}$

If $\Theta$ is not compact, assume that there is a compact $K \subset
\Theta$ such that $\theta_0 \in K$ and $\sexpectation{Z(\Theta
  \setminus K, \xi)}{\theta_0} > 0$.  Then 
\begin{align*}
\lim_{n \to \infty} \hat{\theta}_n = \theta_0
\end{align*}
almost surely with respect to $P_{\theta_0}$.
\end{thm}
\begin{proof}
We show that for all $\theta \neq \theta_0$ there exists a
neighborhood $\theta \in U_\theta$ such that
$\sexpectation{Z(U_\theta, \xi)}{\theta_0} > 0$ and then apply the
previous Theorem \ref{AsymptoticConsistencyMLE}.

Pick $\theta \neq \theta_0$ and assume that we have an open
neighborhood $U_\theta$ with $\theta \in U_\theta$ and $\sexpectation{Z(U_\theta, \xi)}{\theta_0} >
-\infty$.  If $\sexpectation{Z(U_\theta, \xi)}{\theta_0} > 0$ then
have found a suitable neighborhood so we may assume
$\sexpectation{Z(U_\theta, \xi)}{\theta_0} \leq 0$ as well (we really
just need to assume that the value if finite a bit later in the proof).  Now for each $n \in \naturals$ pick a closed ball $U^n_\theta =
B(\theta, r_n) \subset U_\theta$ such that $r_n \leq \frac{1}{n}$ and
$r_n$ are non-increasing.  Furthermore because $U^{n+1}_\theta \subset
U^n_\theta$ we have for fixed $x$, $Z(U^n_\theta, x)$ is increasing in
$n$.

Now assume that we have an $x$ such that $f(x \mid \theta)$ is
continuous.  This implies $\log \frac{f(x \mid \theta_0)}{f(x \mid
  \theta)}$ is continuous as well.  This continuity coupled with the
compactness of $U^n_\theta$ implies that there exists a $\theta_n(x)
\in U^n_\theta$ such that $Z(U^n_\theta, x) = \log \frac{f(x \mid \theta_0)}{f(x \mid
  \theta_n(x))}$.  Clearly we have $\cap_n U^n_\theta =
\lbrace \theta \rbrace$ and this implies $\lim_{n \to \infty}
\theta_n(x) = \theta$.  Again by continuity we get
\begin{align*}
\lim_{n \to \infty} Z(U^n_\theta, x) &= \lim_{n \to \infty}\log \frac{f(x \mid \theta_0)}{f(x \mid
  \theta_n(x))} = \log \frac{f(x \mid \theta_0)}{f(x \mid
  \theta)}
\end{align*}

Now because $U^n_\theta \subset U_\theta$ we have $Z(U^n_\theta, x)
\geq Z(U_\theta, x)$ and $\sexpectation{Z(U_\theta, \xi)}{\theta_0}$
is finite, we may apply Fatou's Lemma (Theorem
\ref{Fatou})
\begin{align*}
\liminf_{n \to \infty} \sexpectation{ Z(U^n_\theta, x)}{\theta_0} -
\sexpectation{Z(U_\theta, x)}{\theta_0} &= \liminf_{n \to \infty} \sexpectation{ Z(U^n_\theta, x) - Z(U_\theta,
x)}{\theta_0} \\
&\geq \sexpectation{ \lim_{n \to \infty}  \left (Z(U^n_\theta, x) - Z(U_\theta,
x) \right )}{\theta_0} \\
&=\sexpectation{ \log \frac{f(x \mid \theta_0)}{f(x \mid
  \theta)}}{\theta_0} -
\sexpectation{Z(U_\theta, x)}{\theta_0} 
\end{align*}
Cancelling the (finite) common term $\sexpectation{Z(U_\theta, x)}{\theta_0}$ we get
\begin{align*}
\liminf_{n \to \infty} \sexpectation{ Z(U^n_\theta, x)}{\theta_0} \geq \sexpectation{ \log \frac{f(x \mid \theta_0)}{f(x \mid
  \theta)}}{\theta_0}  > 0
\end{align*}
where the last inequality follows from the positivity of relative
entropy (Lemma \ref{GibbsInequality}).  Now by this inequality we can
find an $N >0$ such that $\sexpectation{ Z(U^n_\theta, x)}{\theta_0} >
0$ for all $n \geq N$, but in particular there is a single
neighborhood $U^N_\theta$ with this property.  
\end{proof}
The technical conditions above are sufficient to prove asymptotic
efficient of MLEs but it is certainly not necessary.

TODO: Example showing consistency without conditions.

TODO: Note a different condition that suffices (Martingale proof:
Schervish Lemma 7.83)

Maximum likelihood estimators are asymptotically normal under certain
circumstances.  If is unfortunate that any precise statement of those
circumstances is technical and verbose.  It is also unfortunate that
there is no definitive characterization of asymptotic normality as a
set of necessary and sufficient conditions.  Instead there are a
number of sufficient conditions available with different levels of
generality and sophistication.  TODO: This is equally true about
asymptotic consistency and asymptotic results in general; move this
comment to an appropriate place and generalize.

Before stating a rather classical version of such a result let's consider the
case of a scalar parameter in a somewhat heuristic fashion.  If we
assume that we have a consistent MLE such that $\hat{\theta}_n \toas
\theta_0$ and we want to prove that $\sqrt{n}(\hat{\theta}_n -
\theta_0 ) \todist N(0, \sigma^2)$ for an appropriate $\sigma$.  We assume that $f(x \mid
\theta)$ is twice continuously differentiable as a function of
$\theta$; under these conditions the
maximum of the likelihood implies a vanishing derivative
\begin{align*}
\frac{\partial}{\partial \theta} \ell_n (\xi \mid \hat{\theta}_n) &= 0
\end{align*} 
If we apply the mean value theorem to the function
$\frac{\partial}{\partial \theta} \ell_n (\xi \mid \theta)$ to conclude
that there is a value $\theta^*_n$ that lies between $\hat{\theta}_n$
and $\theta_0$ such that 
\begin{align*}
\frac{\frac{\partial}{\partial \theta} \ell_n (\xi \mid \hat{\theta}_n)
  - \frac{\partial}{\partial \theta} \ell_n (\xi \mid
  \theta_0)}{\hat{\theta}_n - \theta_0} &= \frac{\partial^2}{\partial \theta^2}\ell_n (\xi \mid \theta^*_n)
\end{align*}
or rearranging terms to set up ourselves up to take advantage of the
Central Limit Theorem (ignore the possiblity that the denominator vanishes):
\begin{align*}
\sqrt{n}(\hat{\theta}_n - \theta_0) &=
-\frac{\sqrt{n}
 \frac{\partial}{\partial \theta} \ell_n (\xi \mid
  \theta_0)}{\frac{\partial^2}{\partial \theta^2}\ell_n (\xi \mid \theta^*_n)} 
\end{align*}
Now consider the numerator $\mu = \sexpectation{\log f(\xi \mid
  \theta_0)}{\theta_0} = 0$ and variance $i(\theta_0) = \sexpectation{\log^2 f(\xi \mid
  \theta_0)}{\theta_0}$ and we can apply the Central Limit Theorem to see
\begin{align*}
\frac{1}{\sqrt{n}} \ell^\prime_n (\xi \mid \theta_0) &= \frac{1}{\sqrt{n}}
\sum_{i=1}^n \frac{\partial}{\partial \theta} \log f(\xi_i \mid
\theta_0) \todist N(0, i(\theta_0))
\end{align*}
This looks quite promising but there is a factor of
$\frac{1}{\sqrt{n}}$ that was added that will have to be addressed.

Now if we consider the denominator things don't look so good; however a small
modification seems amenable to analysis.  If we consider
$\frac{\partial^2}{\partial \theta^2}\ell_n (\xi \mid \theta_0)$, then
we see that the Weak Law Of Large Numbers tells us that
\begin{align*}
-\frac{1}{n}\frac{\partial^2}{\partial \theta^2}\ell_n (\xi \mid \theta_0) &=
-\frac{1}{n}\sum_{i=1}^n \frac{\partial^2}{\partial \theta^2}\ell_n (\xi_i \mid
\theta_0)  \toprob \sexpectation{-\frac{\partial^2}{\partial \theta^2}
  \log f (\xi \mid
\theta_0) }{\theta_0} =  i(\theta_0)
\end{align*}
Moreover, the factor of $\frac{1}{n}$ that we needed here to apply the
Law of Large Numbers
cancelled exactly with our use of $\frac{1}{\sqrt{n}}$ in the Central
Limit Theorem application so that our Taylor expansion can be written as
\begin{align*}
\sqrt{n}(\hat{\theta}_n - \theta_0) &=
-\frac{
 \frac{\partial}{\partial \theta} \ell_n (\xi \mid
  \theta_0)}{\sqrt{n}} \cdot 
\frac{n}{\frac{\partial^2}{\partial
    \theta^2}\ell_n (\xi \mid \theta_0)} \cdot
\frac {\frac{\partial^2}{\partial
    \theta^2}\ell_n (\xi \mid \theta_0)}
{\frac{\partial^2}{\partial \theta^2}\ell_n (\xi \mid \theta^*_n)} 
\end{align*}
and we are in position to use Slutsky's Lemma to extend the asymptotic
normality of the first factor to $\sqrt{n}(\hat{\theta}_n - \theta_0)$.
The rub is that we have
a term 
\begin{align*}
\frac{\frac{\partial^2}{\partial \theta^2}\ell_n (\xi \mid \theta_0) }{\frac{\partial^2}{\partial \theta^2}\ell_n (\xi \mid \theta_n^*) }
\end{align*}
to understand.  By consistency of the estimator we know that
$\theta_n^* \toas \theta_0$ we might hope that this term converges to
$1$ (at least in probability).  In fact additional smoothness
assumptions on $f$ are sufficient to guarantee that this is the case;
the expression of these smoothness constraints is what provides the
complexity to statements of asymptotic normality of MLEs.
When that is shown, then keeping track of the factors of $i(\theta_0)$
we see that Slutsky's Lemma will tell us that
\begin{align*}
\sqrt{n}(\hat{\theta}_n - \theta_0) \todist N(0, i(\theta_0)^{-1})
\end{align*}

In the following Theorem we capture all the varied assumptions that
are required to make an argument like the above rigorous; the result
is also stated for multivariate parameters.  The details of the proof
are organized a bit differently than the outline of the scalar case
given above (e.g. dealing with boundaries in parameter space) but the
main points of the proof remain the same:
\begin{itemize}
\item[1)] Taylor expand the likelihood function around ${\theta}_0$
\item[2)] Use the Central Limit Theorem to prove convergence of the
  first derivative term at $\theta_0$
\item[3)] Use the Weak Law of Large Numbers to prove convergence of
  the second derivative term at $\theta_0$
\item[4)] Use asymptotic consistency of $\hat{\theta}_n$ and bounds on the variation of the second derivative to
  conclude that the difference between the second derivatives at
  $\theta_0$ and $\hat{\theta}_n$ go to zero in probability.
\item[5)] Use Slutsky's Lemma to glue all the pieces together.
\end{itemize}


\begin{thm}Let $\xi, \xi_1, \xi_2, \dots$ be i.i.d. parametric
  family with distribution $f(x \mid \theta) \, d \mu$ with respect to measure
  space $(X, \mathcal{X}, \mu)$ with $\Theta \subset \reals^k$
  for some $k > 0$.  Assume
\begin{itemize}
\item[(i)]$\hat{\theta}_n \toprob \theta_0$ in $P_{\theta_0}$ for every
  $\theta_0 \in \Theta$.
\item[(ii)]$f(x \mid \theta)$ has continuous second partial
  derivatives with respect to $\theta$ and that differentiation can be
  passed under the integral sign
\item[(iii)]there exists $H_r(x, \theta)$ such that for each $\theta_0
  \in int(\Theta)$ and each $k,j$,
\begin{align*}
\sup_{\norm{\theta - \theta_0} \leq r}
\abs{\frac{\partial^2}{\partial\theta_k \partial\theta_j} \log f(x
  \mid \theta_0) - \frac{\partial^2}{\partial\theta_k \partial\theta_j} \log f(x
  \mid \theta)} \leq H_r(x, \theta_0)
\end{align*}
with $\lim_{r \to 0} \sexpectation{H_r(\xi, \theta_0)}{\theta_0} = 0$.
\item[(iv)]the Fisher information matrix $\mathcal{I}_\xi(\theta_0)$ is
  finite and nonsingular.
\end{itemize}
Then 
\begin{align*}
\sqrt{n} \left(\hat{\theta}_n - \theta_0\right ) \todist N(0, \mathcal{I}^{-1}_\xi(\theta_0))
\end{align*}
\end{thm}
\begin{proof}
We start with

Claim 1: $\frac{1}{\sqrt{n}} D_{\hat{\theta}_n} \ell_n(\sample{\xi} \mid \theta)
\toprob 0$

One might jump to the conclusion that $D_{\hat{\theta}_n} \ell_n(\sample{\xi}
\mid \theta) = 0$ everywhere because $\hat{\theta}_n$ is a maximum,
however there are some details about handling the issue of boundaries on
$\Theta$.  One does know that $D_{\hat{\theta}_n} \ell_n(
\sample{\xi} \mid \theta) = 0$ when $\hat{\theta}_n \in \interior(\Theta)$ but
there is the possibility that some $\hat{\theta}_n$ lies on the
boundary of $\Theta$ and the derivative might not vanish in this case.
To handle the boundary effects, first we know that $\theta_0 \in
\interior(\Theta)$  and therefore there is an open neighborhood
$\theta_0 \in U \subset \interior(\Theta)$.  By the vanishing of the
derivative at any maximum in the interior, we know
\begin{align*}
\frac{1}{\sqrt{n}} D_{\hat{\theta}_n} \ell_n(\sample{\xi} \mid \theta) &=
\frac{1}{\sqrt{n}} D_{\hat{\theta}_n} \ell_n(\sample{\xi} \mid \theta)
\characteristic{\hat{\theta}_n  \in U} + \frac{1}{\sqrt{n}} D_{\hat{\theta}_n} \ell_n(\sample{\xi} \mid \theta)
\characteristic{\hat{\theta}_n  \notin U} \\
&= \frac{1}{\sqrt{n}} D_{\hat{\theta}_n} \ell_n(\sample{\xi} \mid \theta)
\characteristic{\hat{\theta}_n  \notin U}
\end{align*}
Using the fact that $\hat{\theta}_n \toprob
\theta_0$ allows us to conclude that 
\begin{align*}
\lim_{n \to \infty} \sprobability{\hat{\theta}_n  \notin U}{\theta_0} &= 0
\end{align*}
so in particular,
\begin{align*}
\lim_{n \to \infty} \sprobability{ \frac{1}{\sqrt{n}} D_{\hat{\theta}_n} \ell_n(\sample{\xi}
  \mid \theta) \characteristic{\hat{\theta}_n  \notin U} = 0}{\theta_0} &=0
\end{align*}
Putting these two pieces of information together we see
\begin{align*}
 \frac{1}{\sqrt{n}} D_{\hat{\theta}_n} \ell_n(\sample{\xi} \mid \theta) &=
 \frac{1}{\sqrt{n}} D_{\hat{\theta}_n} \ell_n(\sample{\xi} \mid \theta)
\characteristic{\hat{\theta}_n  \notin U} \toprob 0
\end{align*}

Now we derive a quadratic approximation to the likelihood by using a
Taylor expansion (actually just the Mean Value Theorem) of
$D_\theta \ell_n(\sample{\xi} \mid \theta)$ around ${\theta}_0$.  Once
again there is the issue of boundaries but moreover  the domain
$\Theta$ is not convex so the Taylor series only applies cleanly when
$\hat{\theta}_n$ belongs to a ball around $\theta_0$.  To handle this,
pick an $R > 0$ such that we have $B(\theta_0; R) \subset
\interior(\Theta)$.  In this case, when $\norm{\hat{\theta}_n -
  \theta_0} < R$ then we know there exists a $\theta^*_n$ between $\theta_0$ and
$\hat{\theta}_n$ such that 
\begin{align*}
D_{\hat{\theta}_n} \ell_n(\sample{\xi} \mid \theta) - D_{\theta_0} \ell_n(\sample{\xi}
\mid \theta) &= D^2_{\theta^*_n} \ell_n(\sample{\xi} \mid \theta) \cdot
(\hat{\theta}_n - \theta_0)
\end{align*}
As it turns out what happens when $\norm{\hat{\theta}_n - \theta_0}
\geq R$ won't matter since it is an event that occurs with vanishingly
small probability as $n$ grows.  Accordingly, we define
\begin{align*}
\Delta_n &= \begin{cases}
 D^2_{\theta^*_n} \ell_n(\sample{\xi} \mid \theta) & \text{when $\norm{\hat{\theta}_n -
  \theta_0} < R$} \\
0 & \text{when $\norm{\hat{\theta}_n -
  \theta_0} \geq R$} \\
\end{cases}
\end{align*}
TODO: Do we need to justify measurability here...

Claim 2: $ \frac{1}{\sqrt{n}}(D_{\theta_0} \ell_n(\sample{\xi}
\mid \theta) + \Delta_n \cdot (\hat{\theta}_n - \theta_0)) \toprob 0$

Pick an $\epsilon > 0$.  From the definition of $\Delta_n$ we have
\begin{align*}
 \frac{1}{\sqrt{n}} (D_{\theta_0} \ell_n(\sample{\xi}
\mid \theta) + \Delta_n \cdot (\hat{\theta}_n - \theta_0))
&= \begin{cases}
 \frac{1}{\sqrt{n}} (D_{\hat{\theta}_n} \ell_n(\sample{\xi}
\mid \theta) & \text{when $\norm{\hat{\theta}_n -
  \theta_0} < R$} \\
 \frac{1}{\sqrt{n}} (D_{{\theta}_0} \ell_n(\sample{\xi}
\mid \theta) & \text{when $\norm{\hat{\theta}_n -
  \theta_0} \geq R$} \\
\end{cases}
\end{align*}
and therefore
\begin{align*}
&\lim_{n \to \infty} \sprobability{ \frac{1}{\sqrt{n}} (D_{\theta_0} \ell_n(\sample{\xi}
\mid \theta) + \Delta_n \cdot (\hat{\theta}_n - \theta_0)) >
\epsilon}{\theta_0} \\
&= 
\lim_{n \to \infty} \sprobability{  \frac{1}{\sqrt{n}} D_{\hat{\theta}_n} \ell_n(\sample{\xi}
\mid \theta) > \epsilon;\norm{\hat{\theta}_n -
  \theta_0} < R}{\theta_0} \\
&+ \lim_{n \to \infty}\sprobability{  \frac{1}{\sqrt{n}} D_{{\theta}_0} \ell_n(\sample{\xi}
\mid \theta) > \epsilon;\norm{\hat{\theta}_n -
  \theta_0} \geq R}{\theta_0} \\
&\leq 
\lim_{n \to \infty} \sprobability{ \frac{1}{\sqrt{n}} D_{\hat{\theta}_n} \ell_n(\sample{\xi}
\mid \theta) > \epsilon}{\theta_0} + \lim_{n \to \infty}\sprobability{\norm{\hat{\theta}_n -
  \theta_0} \geq R}{\theta_0} = 0
\end{align*}
where we have used Claim 1 and the weak consistency of the estimator
$\hat{\theta}_0$.

Claim 3: $\frac{1}{n}\Delta_n \toprob -\mathcal{I}_\xi(\theta_0)$

Write 
\begin{align*}
\frac{1}{n}\Delta_n &= \frac{1}{n}D^2_{\theta_0} \ell_n(\sample{\xi} \mid
\theta)  \characteristic{\norm{\hat{\theta}_n -
  \theta_0} < R} \\
&+
(D^2_{\theta^*_n} \ell_n(\sample{\xi} \mid \theta) - D^2_{\theta_0} \ell_n(\sample{\xi}
\mid \theta))  \characteristic{\norm{\hat{\theta}_n -
  \theta_0} < R} \\
\end{align*}
and we address the convergence of each of the summands.  First note
that by weak consistency of the estimator $\hat{\theta}_n$ we have
$\characteristic{\norm{\hat{\theta}_n -
  \theta_0} < R} \toprob 1$.
By the Weak Law of Large Numbers and the fact we can exchange
derivatives and expectations we have
\begin{align*}
\frac{1}{n}D^2_{\theta_0} \ell_n(\sample{\xi} \mid
\theta) &= \frac{1}{n}\sum_{i=1}^n D^2_{\theta_0} \log f(\xi_i \mid
\theta) \toprob \sexpectation{D^2_{\theta_0} \log f(\xi \mid
\theta)}{\theta_0} = -\mathcal{I}_\xi(\theta_0)
\end{align*}
and therefore by Corollary
\ref{ConvergenceInProbabilityAndAlgebraicOperations} to the Continuous
Mapping Theorem we can combine these facts to conclude
\begin{align*}
\frac{1}{n}D^2_{\theta_0} \ell_n(\sample{\xi} \mid
\theta) \characteristic{\norm{\hat{\theta}_n -
  \theta_0} < R} &\toprob -\mathcal{I}_\xi(\theta_0)
\end{align*}
We turn attention to the error term which we show is $o_P(1)$.
Let $\epsilon > 0$ be given.  Pick any $0 < r \leq R$ such that
$\sexpectation{H_r(\xi, \theta_0)}{\theta_0} < \frac{\epsilon}{2}$.
Again applying the Weak Law of Large Numbers 
\begin{align*}
\frac{1}{n} \sum_{i=1}^n H_r(\xi_i, \theta_0) \toprob \sexpectation{H_r(\xi, \theta_0)}{\theta_0} < \frac{\epsilon}{2}
\end{align*}
and therefore
\begin{align*}
\lim_{n \to \infty} \sprobability{\frac{1}{n} \sum_{i=1}^n H_r(\xi_i,
  \theta_0) < \epsilon}{\theta_0} &\leq \lim_{n \to \infty} \sprobability{\abs{\frac{1}{n} \sum_{i=1}^n H_r(\xi_i,
  \theta_0) - \sexpectation{H_r(\xi, \theta_0)}{\theta_0}} <
\frac{\epsilon}{2}}{\theta_0} = 0
\end{align*}
Now apply this fact to get a bound on each entry of the Hessian matrix
\begin{align*}
&\lim_{n \to \infty} \sprobability{\frac{1}{n}\abs{D^2_{\theta^*_n, j,k} \ell_n(\sample{\xi} \mid
\theta) - D^2_{\theta_0, j,k} \ell_n(\sample{\xi} \mid
\theta) } \characteristic{\norm{\theta^*_n - \theta_0} < R} <
\epsilon}{\theta_0} \\
&\lim_{n \to \infty} \sprobability{\frac{1}{n}\abs{D^2_{\theta^*_n, j,k} \ell_n(\sample{\xi} \mid
\theta) - D^2_{\theta_0, j,k} \ell_n(\sample{\xi} \mid
\theta) } \characteristic{\norm{\theta^*_n - \theta_0} < r} <
\epsilon}{\theta_0} \\
&+ \lim_{n \to \infty} \sprobability{\frac{1}{n}\abs{D^2_{\theta^*_n, j,k} \ell_n(\sample{\xi} \mid
\theta) - D^2_{\theta_0, j,k} \ell_n(\sample{\xi} \mid
\theta) } \characteristic{r \leq \norm{\theta^*_n - \theta_0} < R} <
\epsilon}{\theta_0} \\
&\leq
\lim_{n \to \infty} \sprobability{\frac{1}{n} \sum_{i=1}^n H_r(\xi_i,
  \theta_0) < \epsilon}{\theta_0} + 
\lim_{n \to \infty} \sprobability{ \characteristic{r \leq \norm{\theta^*_n -
      \theta_0} < R} }{\theta_0} \\
&= 0
\end{align*}
and therefore we have shown $\frac{1}{n} (D^2_{\theta^*_n, j,k} \ell_n(\sample{\xi} \mid
\theta) - D^2_{\theta_0, j,k} \ell_n(\sample{\xi} \mid
\theta) ) \characteristic{\norm{\theta^*_n - \theta_0} < R} \toprob 0$.

Claim 4: $ \frac{1}{\sqrt{n}} D_{\theta_0} \ell_n(\sample{\xi}
\mid \theta) \todist N(0, \mathcal{I}_\xi(\theta_0))$

First note that 
\begin{align*}
\frac{1}{n} D_{\theta_0} \ell_n(\sample{\xi}
\mid \theta) &= \frac{1}{n} \sum_{i=1}^n D_{\theta_0} \log f(\xi_i
\mid \theta) \toprob \sexpectation{D_{\theta_0} \log f(\xi \mid \theta)}{\theta_0}
\end{align*}
since we have an i.i.d. sum and we can apply the Weak Law of Large
Numbers.  Because we assume we can exchange expectations and
derivatives for any partial derivative 
\begin{align*}
\sexpectation{\frac{\partial}{\partial \theta_i} \log f(\xi_i \mid
  \theta)}{\theta_0} &= \int \frac{\partial}{\partial \theta_i} \log f(x \mid
  \theta) f(x \mid \theta_0) dx = \int \frac{\partial}{\partial \theta_i}  f(x \mid
  \theta_0) \, dx = \frac{\partial}{\partial \theta_i} \int f(x \mid
  \theta_0) \, dx = 0
\end{align*}
and thus we conclude $\frac{1}{n} D_{\theta_0} \ell_n(\sample{\xi}
\mid \theta) \toprob 0$.  We can also calculate the covariance matrix of the random variable
$D_{\theta_0} \log f(\xi \mid \theta)$ as $\mathcal{I}_\xi(\theta_0)$.

Now we simply apply the multivariate Central Limit Theorem and the
Claim is proven.

Claim 5: $ \frac{1}{\sqrt{n}} D_{\theta_0} \ell_n(\sample{\xi}
\mid \theta) -  \sqrt{n} \mathcal{I}_\xi(\theta_0) \cdot
(\hat{\theta}_n - \theta_0) \toprob 0$

We already know from Claim 2 that $ \frac{1}{\sqrt{n}} (D_{\theta_0} \ell_n(\sample{\xi}
\mid \theta) + \Delta_n \cdot
(\hat{\theta}_n - \theta_0)) \toprob 0$ so it suffices to show that $\frac{1}{\sqrt{n}}\Delta_n \cdot
(\hat{\theta}_n - \theta_0) +  \sqrt{n} \mathcal{I}_\xi(\theta_0) \cdot
(\hat{\theta}_n - \theta_0) \toprob 0$
as well.  

By Claim 4 and Lemma \ref{WeakConvergenceImpliesTight}, we know that 
$ \frac{1}{\sqrt{n}} D_{\theta_0} \ell_n(\sample{\xi} \mid \theta)$ is
tight.  Together with Claim 2 this tells us that $\frac{1}{\sqrt{n}}\Delta_n \cdot (\hat{\theta}_n -
\theta_0)$ is $o_P(1) + O_P(1)$ hence is tight as well (Lemma
\ref{AlgebraOfStochasticConvergence}).  
Claim 3 and the invertibility of $\mathcal{I}_\xi(\theta_0)$ allows us
to apply  Lemma \ref{InvertMatrixInProbability} to conclude that $\frac{1}{\sqrt{n}}\Delta_n \cdot (\hat{\theta}_n -
\theta_0)$ is tight.
Now by Claim 3 and Lemma
\ref{AlgebraOfStochasticConvergence} we
can conclude that $ (\frac{1}{n}\Delta_n + \mathcal{I}_\xi(\theta_0))
\cdot \sqrt{n} (\hat{\theta}_n - \theta_0) \toprob 0$ as required.


Now when we combine Claim 4 and Claim 5 with Slutsky's Lemma (Theorem
\ref{Slutsky}) we conclude that $\mathcal{I}_\xi(\theta_0)
\cdot \sqrt{n} (\hat{\theta}_n - \theta_0) \todist N(0,
\mathcal{I}_\xi(\theta_0)^{-1})$.  Because $\mathcal{I}_\xi(\theta_0)$ is invertible and matrix multiplication is
continuous, the Continuous Mapping Theorem allows us to conclude 
$\sqrt{n} 
(\hat{\theta}_n - \theta_0) \todist \mathcal{I}_\xi(\theta_0)^{-1}
N(0, \mathcal{I}_\xi(\theta_0)) = N(0, \mathcal{I}_\xi(\theta_0)^{-1})$.
and we are done.
\end{proof}

As a side effect of having shown that an MLE may be asymptotically
normal we computed its asymptotic variance.  Now it is intuitively
clear that given two estimators that are equal in every other way the
one with a smaller variance is to be preferred.  So a natural question
to ask is whether a variance of $\mathcal{I}_\xi(\theta)^{-1}$ is a good by some objective
standard.  It is in fact optimal.

\begin{thm}[Cramer-Rao Lower Bound]\label{CRLB}blah blah
\end{thm}

TODO: Binomial estimation
Ideas:  Frequentist vs. Bayesian.  Two sampling approaches: sample
fixed n vs. sequentially sample till n successes.  Same means but
different variances in frequentist approaches (failure of the
likelihood principle) but same in Bayesian.
The normal approximation and confidence intervals.  Discuss issues
with coverage.  Ratio of binomial (e.g. Koopman and the Bayesian approach).

TODO: Maybe a good idea to cover logistic regression as an application
of MLE.  Expressing regression as an MLE: requires a distribution
assumption on the residual and then regression becomes a location
scale family.  I don't see that the standard proofs of consistency and
normality work in these cases though (since the observations now are
independent but have differing distributions..)  I think this is an
accurate state of affairs; there are direct proofs of MLE asymptotic
properties for GLMs (and I suppose GAMs).  See also Hjort and Pollard,
``Asymptotics for minimisers of convex processes''  As for intuition
about why i.i.d. should not be necessary to prove asymptotic results
recall that the Weak Law of Large Numbers doesn't require i.i.d. but
only uniform integrability and that the Lindeberg C.L.T. applies
without full blown i.i.d.  It'll be an interesting exercise to see how
the asymptotic theory of logistic regression unfolds.

\section{Logistic Regression}
To motivate the logistic regression, assume that we have a binomial
random variable $y \sim B(n,p)$ and consider the maximum likelihood
estimate of the parameter $p$.  Introduce the log odds $\theta =
logit(p) = \ln(p/(1-p))$ rewrite the binomial
distribution in terms of $\theta$.
\begin{align}
\binom{n}{m}p^m(1-p)^{n-m} & = e^{\ln(\binom{n}{m})}
e^{\ln(p^m)}e^{\ln((1-p)^{n-m}} \\
& = e^{\ln(\binom{n}{m}) + m \ln(p/(1-p)) +
  n \ln(1-p)} \\
& = e^{\ln(\binom{n}{m}) + m \ln(p/(1-p)) -  n \ln(1 + p/(1-p))} \\
& = e ^ {\ln(\binom{n}{m}) + m \theta - n \ln(1+e^\theta)}
\end{align}
This allows us to write the loglikelihood function in terms of the
parameter $\theta$ as:
$$
l(\theta; y) = y\theta - n\ln(1+e^\theta) + \ln\binom{n}{y}
$$
and then it is easy to get the score and information functions
\begin{align}
s(\theta; y) & = \frac{\partial}{\partial \theta} l(\theta;y) = y -
\frac{n e^\theta}{1+e^\theta}= y - np \\
i(\theta; y) & = -\frac{\partial}{\partial \theta} s(\theta;y) = np(1-p)
\end{align}

\section{Bayesian Models}
Here are some simple examples of Bayesian updating for models in which
conjugate priors exist so that we have closed form solutions.

\begin{examp}\label{Estimating a Normal mean with know variance}
Suppose we have a normal population $N(\mu, \sigma^2)$ with $\sigma^2$
assumed known and $\mu$ assumed to be distributed $N(\mu_0,
\sigma_0^2)$ with $\mu_0$ and $\sigma_0^2$ known.  If we are given
independent observations $x_1, \dotsc, x_n$ then we have a likelihood
function 
\begin{align*}
p(\bold{x}\mid \mu) &= \prod_{i=1}^n \frac{1}{\sqrt{2 \pi} \sigma}
e^{-(x_i - \mu)^2/2\sigma^2} 
= \frac{1}{(2 \pi)^{n/2}  \sigma^n} e^{-\frac{1}{2\sigma^2} \sum_{i=1}^n (x_i - \mu)^2}
\end{align*}
and by Bayes' Theorem
\begin{align*}
p(\mu \mid \bold{x}) &\propto \frac{1}{\sqrt{2\pi}\sigma_0} e^{-(\mu -
  \mu_0)^2/2\sigma_0^2} \cdot \frac{1}{(2 \pi)^{n/2}  \sigma^n} e^{-\frac{1}{2\sigma^2} \sum_{i=1}^n (x_i - \mu)^2}
\end{align*}
\end{examp}

A generalization of the above to a linear regression scenario is
\begin{examp}\label{BayesianUnivariateLinearRegression}Here we have a
  model $y = X^t \beta + \epsilon$ with 
\begin{itemize}
\item[(i)]  $\epsilon$ is $N(0, \sigma^2)$ 
\item[(ii)]  $1/\sigma^2$ is $\Gamma(\alpha, \beta)$ with $\alpha$ and
  $\beta$ known.
\item[(iii)] $\beta$ is $N(\mu_0, \sigma^2 \Lambda_0^{-1})$ with
  $\mu_0$ and $\Lambda_0$ known
\end{itemize}
We suppose that we are given independent observations $X_1, \dotsc,
X_n$ which we assemble into an observation matrix $X$.  
TODO: Finish
\end{examp}