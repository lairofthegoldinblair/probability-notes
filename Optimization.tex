\chapter{Optimization in Banach Spaces}
As one will undoubtedly remember from one's first calculus class the
derivative is an extraordinarily useful tool for finding maxima and
minima of functions of a real variable.  Essentially all of that
theory carries over to the case of general Banach space domains.  One
of the goals in this subsection is to develop that basic theory.  

In a multivariate calculus course the reader almost certainly
encounterd problems of constrained optimization as well: learning the
tool of the Lagrange multiplier for solving such problems. This theory
also carries over to the Banach space setting and we develop it here.

What one has learned up to this point is the theory of equality
constrained optimization.  Though it tends not to be taught in the
introductory
calculus curriculum, in applications it is equally important to be able to
solve optimization with both equality and inequality constraints.
Having the tools we have developed it is no harder to develop such
theory in the general Banach space setting and we do so here.

We will discuss optimization problems in terms of minimization; as a
general rule there is no loss of generality in doing so as
maximization of a function $f$ may be performed by minimizing the
function $-f$.  

First we distinguish the different kinds of minima that we may
characterize.  The primary distinction is the dichotomy between local
and global minimization.  There are subtler distinctions to be made
between different type of local minima.  The definitions make sense
for arbitrary topological spaces.

\begin{defn}Let $X$ be a topological space and let $f : X \to \reals$
  be a function.  We say that $x^* \in X$ is a \emph{global minimizer}
  of $f$ if $f(x^*) \leq f(x)$ for all $x \in X$.  We say that $x^*
  \in X$ is a \emph{local minimizer} if there exists an open set $U
  \subset X$ such that $x^* \in U$ and $f(x^*) \leq f(x)$ for all $x
  \in U$.  We say that $x^*
  \in X$ is a \emph{strict local minimizer} if there exists an open set $U
  \subset X$ such that $x^* \in U$ and $f(x^*) < f(x)$ for all $x
  \in U$ with $x^* \neq x$.  We say that $x^* \in X$ is an
  \emph{isolated local minimizer} if there exists and open set $U
  \subset X$ such that $x^* \in U$ and $x^*$ is the only local
  minimizer in $U$.
\end{defn}

\begin{examp}Let 
\begin{align*}
f(x) &= \begin{cases}
x^4 \cos(1/x) + 2x^4 & \text{if $x \neq 0$} \\
0 & \text{if $x = 0$}
\end{cases}
\end{align*}
then $0$ is a strict local minimizer that is not isolated.  TODO: Show this
\end{examp}

Note that minimizers are not guaranteed to exist since functions may
be unbounded below.  More subtely may have a lower bound but may never
take the value of its greatest lower bound.  We already know a case in
which both of these problems are avoided: namely continuous images of
compact sets are compact in $\reals$ and this guarantees the exists of
a miminum (Theorem \ref{ContinuousImageOfCompact}).  As it turns out
this fact can be generalized a bit by relaxing the property of
continuity.

\begin{defn}Let $X$ be a topological space and let $f : X \to \reals$
  be a function, we say that $f$ is \emph{lower semicontinuous}
  (resp. \emph{upper semicontinuous}) if for every
  $\epsilon > 0$ there exists an open set $U$ containing $x$ such that
  $f(y) \geq f(x) - \epsilon$ (resp. $f(y) \leq f(x) + \epsilon$) for
  every $y \in U$. We say that $f$ is \emph{sequentially lower
    semicontinuous} (resp. \emph{sequentially upper semicontinuous})
  at $x$ if for every sequence $x_n$ such that $\lim_{n \to \infty}
  x_n = x$ we have $f(x) \leq \liminf_{n \to \infty} f(x_n)$ (resp. $f(x) \geq \limsup_{n \to \infty} f(x_n)$).
\end{defn}

It is simple to see that $f$ is lower semicontinuous
(resp. sequentially lower semicontinuous) if and only if $-f$ is upper
semicontinuous (resp. sequentially upper semicontinuous).

A function is lower (resp. upper) semicontinuous at $x$ if its values near $x$ are
either close to $f(x)$ or larger (resp. smaller) than $f(x)$.  In
general sequential semicontinuity is a weaker property than
semicontinuity (since sequences do not characterize convergence in
general topological spaces); however in metric spaces the two concepts
are equivalent.

\begin{prop}Let $X$ be a topological space and let $f$ be a lower
  (resp. upper) semicontinuous at $x$, then $f$ is sequentially
  lower (resp. upper) semicontinuous at $x$.  If $X$ is a metric space and
  $f$ is sequentially lower (resp. upper) semicontinuous at $x$ then $f$ is
  lower (resp. upper) semicontinuous at $x$.
\end{prop}
\begin{proof}If suffices to handle the cases of lower semicontinuity
  since the upper semicontinuity results follow by applying the lower
  semicontinuity case to $-f$.

If $f$ is lower semicontinuous and $x_n \to x$.  Let $\epsilon > 0$ be
given an find an open neighborhood $U$ of $x$ such that $f(y) \geq
f(x) - \epsilon$ for all $y \in U$.  Since $x_n \to x$ we know that
there exists $N > 0$  such that $x_n \in U$ for $n \geq N$ and thus
$\inf_{m \geq n} f(x_m) \geq f(x) - \epsilon$ for every $n \geq N$.
We take the limit at $n \to \infty$ to get $\liminf_{n \to \infty}
f(x_n) \geq f(x) - \epsilon$.  Since $\epsilon > 0$ was arbitrary we
conclude that $f$ is sequentially lower semicontinuous at $x$.

Now let $X$ be a metric space and suppose that $f$ is not
lower semicontinuous at $x$.  Then there exists an $\epsilon > 0$ such
that for every $n \in \naturals$ there exists $x_n$ with $d(x,x_n) <
1/n$ such that $f(x_n) < f(x) - \epsilon$.  Clearly $x_n \to x$ and
moreover $\liminf_{n \to \infty} f(x_n) \leq f(x) - \epsilon < f(x)$
which shows that $f$ is not sequentially lower semicontinuous at $x$.
\end{proof}

Compactness and sequential lower semicontinuity suffice to show that
a function has a global minimizer.
\begin{thm}\label{ExistenceGlobalMinimizerLowerSemicontinuous}Let $X$ be a topological space, let $f : X \to [-\infty,
  \infty]$ be a sequentially lower
  semicontinuous function and suppose that there exists $M \in \reals$
  such that $\lbrace x \in X \mid f(x) \leq M \rbrace$ is non-empty
  and compact then
  $f$ has a global minimizer.  Moreover if $f$ is lower
  semicontinuous, the set of global minimizers
  is compact.
\end{thm}
\begin{proof}
We first show the existence of a global minimizer.  Let $\alpha = \inf_{x \in X} f(x)$ and note that we know $\alpha \leq
M < \infty$.  If $\alpha = M$ then in fact $\lbrace x \in X \mid f(x)
\leq M \rbrace  = \lbrace x \in X \mid f(x) = M \rbrace $ and is assumed
non-empty and we are done.  Therefore we assume that $\alpha < M$.  Let $x_n$ be chosen so that
$\lim_{n \to \infty} f(x_n) = \alpha$ (if $\alpha > -\infty$ then
choose $x_n$ such that $f(x_n) < \alpha + 1/n$ otherwise choose $x_n$
so that $f(x_n) \leq -n$).  As $\alpha < M$ we know that there exists
$N>0$ such that $f(x_n) \leq M$ for all $n \geq N$ and therefore by
compactness there exists a convergent subsequence $x_{n_j}$.  Let $x =
\lim_{j \to \infty} x_{n_j}$ and note that by sequential lower semicontinuity
at $x$
\begin{align*}
\alpha \leq f(x) \leq \liminf_{j \to \infty} f(x_{n_j}) = \alpha
\end{align*}
which shows that $f(x) = \alpha$.

Let $G = \lbrace x \in X \mid f(x) = \alpha \rbrace$.  Now we know
that since $\alpha \leq M$  that $G \subset \lbrace x \in X \mid f(x)
\leq M \rbrace$ hence it suffices to show that the set of global
minimizers is closed (Corollary \ref{ClosedSubsetsCompact}).  Let $x$
be in $\overline{G}$ and let $\epsilon > 0$ be given.  Since $f$ is
lower semicontinuous we can find an open set $U$ containing $x$ such
that $f(y) \geq f(x) - \epsilon$.  However we know that $G \cap U \neq
\emptyset$ therefore $\alpha \geq f(x) - \epsilon$.  Since $\epsilon$
is arbitrary we conclude $\alpha \geq f(x)$ and thus $x \in G$.  
\end{proof}


Given that derivatives are determined by the behavior of functions on
arbitrarily small neighborhoods of a point it is clear that they have
little to say about when a point is a global minimizer.  On the other
hand derivatives are rather informative about local minimizers and we
turn our attention to this.

\section{Unconstrained Optimization}

The first thing to do is to note that there are \emph{necessary}
conditions for a point being a local minimizer that are described by
derivatives.  The first such is the vanishing of the first derivative.

\begin{thm}\label{VanishingFirstDerivativeAtLocalMinimum}Let $X$ be a
  Banach space, let $f : X \to \reals$ be a function and let $x^*$ be
  a local minimum.  If $f$ is $C^1$ on an open neighborhood of $x^*$
  then $Df(x^*) = 0$.
\end{thm}
\begin{proof}
The proof is by contradiction.  Suppose that $Df(x^*) \neq 0$.  Thus
there exists $y \in X$ such that $Df(x^*)y > 0$.  Let $f$ be $C^1$ on
an open neighborhood $U$ of $x^*$.  By continuity of $Df(x)$ on $U$ we
may also find a $\delta > 0$ such that $Df(x) y > 0$ for all $x \in
B(x^*, \delta) \subset U$.  By multiplying $y$ by an
appropriate positive constant we may assume that $\norm{y} < \delta$.  Now we can
apply the Mean Value Theorem \label{MeanValueTheoremBanachSpaces} to conclude that 
\begin{align*}
f(x^* - y) &= f(x^*) - \int_0^1 Df(x^*-ty) y \, dt < f(x^*)
\end{align*}
which shows that $x^*$ is not a local minimizer.

Here is an alternative proof that avoid appealing to the Mean Value Theorem.
Find an open ball $\delta > 0$ such that $f(x^*) \leq f(y)$
for all $y$ with $\norm{y - x} < \delta$.  Pick an arbitrary $y \in X$
then for all $0 < t < \delta/\norm{y}$ we have $f(x^* + t y) - f(x^*)
\geq 0$ which implies 
\begin{align*}
\lim_{t \to 0^+} \frac{f(x^* + ty) - f(x^*)}{t} \geq 0
\end{align*}
Arguing similarly for $- \delta/\norm{y}<t<0$ we get
\begin{align*}
\lim_{t \to 0^-} \frac{f(x^* + ty) - f(x^*)}{t} \geq 0
\end{align*}
On the other hand we have assumed that the  G\^{a}teaux derivative exists and thus
\begin{align*}
\lim_{t \to 0} \frac{f(x^* + ty) - f(x^*)}{t} &= \lim_{t \to 0^-} \frac{f(x^* + ty) - f(x^*)}{t} \leq 0 \\
&\leq \lim_{t \to 0^+} \frac{f(x^* + ty) - f(x^*)}{t} = \lim_{t \to 0} \frac{f(x^* + ty) - f(x^*)}{t}
\end{align*}
which shows that $df(x^*, y) = \lim_{t \to 0} \frac{f(x^* + ty) - f(x^*)}{t} = 0$. 

If we assume that $f$ is Frechet differentiable at $x^*$ then by Proposition \ref{FrechetDifferentiableImpliesGateauxDifferentiable} we know that
the Frechet derivative is equal to the directional derivative and thus $Df(x^*)=0$.

TODO: This latter proof seems to suggest that the vanishing doesn't require $C^1$ or even Frechet differentiability at $x^*$ just G\^{a}teaux differentiability at $x^*$.  Am I missing something?  I don't think so; I should probably rephrase the result in terms of  G\^{a}teaux derivatives.
\end{proof}

When $f$ has two derivatives then we can say even more.
\begin{thm}\label{PositiveSemidefiniteSecondDerivativeAtLocalMinimum}Let $X$ be a
  Banach space, let $f : X \to \reals$ be a function and let $x^*$ be
  a local minimum.  If $f$ is $C^2$ on an open neighborhood of $x^*$
  then $Df(x^*) = 0$ and $D^2f(x^*)$ is positive semidefinite
  (i.e. $D^2f(x^*) (v,v) \geq 0$ for all $v \in X$).
\end{thm}
\begin{proof}
Again we proceed by contradiction.  Suppose that $D^2f(x^*)(v,v) < 0$.
By continuity of $D^2f$ we may find a $\delta > 0$ such that $D^2f(x)(v,v) < 0$ for
all $x \in B(x^*,\delta) \subset U$.  If necessary multiply $v$ be a
small positive constant to guarantee that $\norm{v} < \delta$.  By Theorem \ref{VanishingFirstDerivativeAtLocalMinimum} we know that
$Df(x^*) = 0$ so Taylor's Theorem says
\begin{align*}
f(x^* + v) &= f(x^*) + \int_0^1 (1 - t) D^2f(x^* + tv) (v,v) \, dt < f(x^*)
\end{align*}
which is a contradiction.
\end{proof}

When $f$ has two derivatives there also exists sufficient conditions
that a point be a local minimizer.
\begin{thm}\label{LocalMinimumAtPositiveDefiniteSecondDerivative}Let $X$ be a
  Banach space, let $f : X \to \reals$ be a function and suppose $f$
  is $C^2$ on an open neighborhood $U$ of $x^*$.
  If $Df(x^*) = 0$ and $D^2f(x^*)$ is positive definite
  (i.e. there exists an $\alpha > 0$ such that $D^2f(x^*) (v,v) >
  \alpha \norm{v}^2$ for all $v \in X$ with $v \neq 0$) then
  $x^*$ is a strict local minimizer of $f$.
\end{thm}
\begin{proof}
Using continuity of $D^2f$ at $x^*$ we may find a $\delta > 0$ such
that $B(x^*,
\delta) \subset U$ and $\norm{D^2f(x^* + y) - D^2f(x^*)}
< \frac{\alpha}{2}$ for all $\norm{y} < \delta$.  Note in particular
that 
\begin{align*}
(D^2f(x^* + y) - D^2f(x^*) )(v,w) &> -\frac{\alpha \norm{v}
  \norm{w}}{2} \text{ for all $\norm{y} < \delta$ and $v,w \in X$}
\end{align*}  
By Taylor's Theorem, 
for all $y$ with $\norm{y} < \delta$
\begin{align*}
f(x^* + y) - f(x) &= 
\int_0^t (1-t) D^2f(x^* + ty)(y,y) \, dt \\
&=\frac{1}{2} D^2f(x^*)(y,y) + \int_0^t (1-t) (D^2f(x^* + ty)
  -D^2f(x^*)) (y,y) \, dt  \\
&\geq \frac{\alpha \norm{y}^2}{2} - \frac{\sup_{0 \leq t \leq 1}
  \norm{D^2f(x^* + ty) - D^2f(x^*)} \norm{y}^2}{2} \\
&\geq  \frac{\alpha
  \norm{y}^2}{4} > 0\\
\end{align*}
which shows that $x^*$ is a local minimizer.
\end{proof}

Note that in finite dimensions the condition of positive definiteness
is equivalent to the apparently weaker condition $D^2f(x^*) (v,v) > 0$
for all $v \neq 0$.

\section{Constrained Optimization}

We first consider an abstract version of constrained
optimization.  Let $X$ be a Banach space and consider a function $f :
X \to \reals$ then given a closed set $F \subset X$ we can consider
the problem of finding a minimizer of $f$ restricted to $F$.  Note
that the meaning of finding a constrained minimizer is captured by
using our existing definitions of minimizers on the space $F$ with the
relative topology.

In order
to apply derivatives to the problem of characterizing minimizers on
$F$ we need to restrict them to directions that don't leave $F$;
if we have a point $x \in F$ and $f$ is decreasing at $x$ in a direction that immediately takes one out of
$F$ then that alone won't means that $x$ isn't a minimizer when
restricted to $F$.  This leads us to a definition of direction tangent
to a closed set $F$.  Note that if a direction is tangent to a set at
a point then any positive multiple should be tangent (though if the
set has corners then negative multiples may fail to be tangents;
consider the behavior of $\abs{x}$ at the origin).  As a result of
this observation we should be seeking to characterize a cone of
tangent directions.  
\begin{defn}Let $X$ be a Banach space, let $F$ be a closed subset and
  let $x \in F$ then we say that $v \in X$ is a \emph{tangent vector to $F$
  at $x$} if there is a sequence $x_n$ such that $x_n \in F$ and
  $\lim_{n \to \infty} x_n = x$ and a sequence of positive real numbers
  $t_n$ such that $\lim_{n \to \infty} t_n = 0$ that together satisfy
\begin{align*}
\lim_{n \to \infty} \frac{x_n - x}{t_n} = v
\end{align*}
The set $T_F(x)$ of all tangent vectors to $F$ at $x$ is called the
\emph{tangent cone to $F$ at $x$}.
\end{defn}

We call out the fact that the tangent cone is in fact a cone.
\begin{prop}The tangent cone $T_F(x)$ is a cone (i.e. for every
  $\alpha \geq 0$ and $v \in T_F(x)$ we have $\alpha v \in T_F(x)$).
\end{prop}
\begin{proof}
It is trivial to see that $0 \in T_F(x)$ since we can just pick the
$x_n \equiv x$.  Let $v \in T_F(x)$, $\alpha > 0$ and pick sequences $x_n$ and $t_n$ such that $x_n \to x$, $t_n \to 0$ and
$\frac{x_n - x}{t_n} = v$.  Then let $\tilde{t}_n = t_n/\alpha$ and note that
$\tilde{t}_n \to 0$ and $\frac{x_n -x}{\tilde{t}_n} = \alpha v$.
\end{proof}

It is also useful to note the following equivalent form of the definition of a tangent vector.
\begin{prop}Let $X$ be a Banach space, let $F$ be a closed subset and
  let $x \in F$ then = $v \in X$ is a tangent vector to $F$
  at $x$ if and only if
\begin{align*}
\liminf_{t \to 0^+} \frac{d(F, x+tv)}{t} &= 0
\end{align*}
\end{prop}
\begin{proof}
Suppose $\liminf_{t \to 0^+} \frac{d(F, x+tv)}{t} = 0$.  Since $\frac{d(F, x+tv)}{t}$ is non-negative it is equivalent that $\inf_{0 < t < h} \frac{d(F, x+tv)}{t} = 0$ for every $h >0$.  Thus for every $n \in \naturals$ there exists $0 < t_n < 1/n$
and $x_n \in F$ such that 
\begin{align*}
\frac{d(x_n, x+t_n v)}{t_n} &= \norm{\frac{ x_n - x}{t_n}-v} < 1/n
\end{align*}
By construction, $\lim_{n \to \infty} t_n = 0$ and by the triangle inequality, 
\begin{align*}
\lim_{n \to \infty} \norm{x-x_n} &< \lim_{n \to \infty} (\norm{v}t_n + t_n/n) = 0
\end{align*}
thus $v$ is a tangent vector.

If $v$ is a tangent vector to $F$ at $x$ then pick $x_n \in F$ and $t_n$ such that $\lim_{n \to \infty} x_n = x$ and $\lim_{n \to \infty} (x-x_n)/t_n = v$.  Let $\epsilon>0$ be 
given and pick an $N>0$ such that $\norm{x - x_n -t_n v}/t_n < \epsilon$ for all $n \geq N$.  Then since $\lim_{n \to \infty} t_n = 0$ it follows that for any $h>0$ there exist
$0 < t_n < h$ such that $\norm{x - x_n -t_n v}/t_n < \epsilon$ which implies 
$\inf_{0 < t < h} \frac{d(F, x+tv)}{t} = 0$ for all $h > 0$.
\end{proof}

The first hint that we have the correct notion of tangent vector is
the following necessary condition for a local minimizer to exist.

A few facts about Landau notation.
\begin{defn}Let $X$, $Y$ and $Z$ be Banach spaces.  Let $x_n$ be a sequence in $X$ and let
  $y_n$ be a sequence in  $Y$ we say that $x_n = o(y_n)$
  as $n \to \infty$ if $\lim_{n \to \infty} \frac{x_n}{\norm{y_n}} =
  0$ (equivalently $\lim_{n \to \infty} \frac{\norm{x_n}}{\norm{y_n}} =
  0$).  We say that $x_n = O(y_n)$ if there exists $M > 0$ and $N \geq
  0$ such that
  $\frac{\norm{x_n}}{\norm{y_n}}  \leq M$ for all $n \geq N$.  
Given functions $f : X \to Y$, $g : X \to Z$ and $x_0 \in X$ we say that
  $f(x)$ is $o(g(x))$ as $x \to x_0$ if $\lim_{x \to x_0}
  \frac{f(x)}{\norm{g(x)}} = 0$ (equivalently $\lim_{x \to x_0}
  \frac{\norm{f(x)}}{\norm{g(x)}} = 0$) and we say that $f(x)$ is
  $O(g(x))$ if there exists $M > 0$ and $\delta > 0$ such that
  $\frac{\norm{f(x)}}{\norm{g(x)}} \leq M$ for all $\norm{x - x_0} < \delta$.
\end{defn}
Because the definitions above really only depend on the norms of the
sequences and functions in question, it is often useful to say that a
sequence $x_n \in X$ is $o(\norm{y_n})$ or a function $f(x)$ is
$o(\norm{g(x)})$.  It is also worth pointing out that Landau notation
is confusing for uninitiated in large part because of its abuse of the
equality sign.  

\begin{prop}\label{LandauNotationIdentities}The following are true:
\begin{itemize}
\item[(i)] $o(y_n) + o(y_n) = o(y_n)$.
\item[(ii)] Suppose $z_n = O(y_n)$ then if $x_n = o(z_n)$ it follows
  that $x_n = o(y_n)$.  In shorthand we say that $o(O(y_n)) = o(y_n)$.
\end{itemize}
\end{prop}
\begin{proof}
(i) follows from linearity: if $x_n = o(y_n)$ and $z_n = o(y_n)$ then
it follows that 
\begin{align*}
\lim_{n \to \infty} \frac{x_n + z_n}{\norm{y_n}} &= 
\lim_{n \to \infty} \frac{x_n}{\norm{y_n}}
+ \lim_{n \to \infty} \frac{z_n}{\norm{y_n}}
= 0
\end{align*}
To see (ii), we know that $\lim_{n \to \infty} \frac{\norm{x_n}}{\norm{z_n}} =
0$ and there exist $M,N \geq 0$ such that $\norm{z_n} \leq M
\norm{y_n}$ for all $n \geq N$, therefore 
\begin{align*}
0 &\leq \lim_{n \to \infty} \frac{\norm{x_n}}{\norm{y_n}} \leq M \lim_{n \to \infty}
\frac{\norm{x_n}}{\norm{z_n}} = 0
\end{align*}
\end{proof}


\begin{thm}\label{LocalConstrainedMinimizerFirstDerivative}Let $X$ be a Banach space, $F \subset X$ be closed and let
  $f : U  \to \reals$ be $C^1$ on an open set $U \supset F$.  Then if
  $x^*$ is a local minimizer of $f$ on $F$ we have $Df(x^*) v \geq 0$
  for all $v \in T_F(x^*)$.  
\end{thm}
\begin{proof}
Suppose that we have $x_n \in F$ with $x_n \to x$, $t_n>0$ with $t_n \to 0$
and $(x_n - x)/t_n \to v$.  By Taylor's Theorem and the fact that
$x^*$ is a local minimizer we know that we
can find a neighborhood $x^* \in V \subset U$ such that
\begin{align*}
f(y)  - f(x^*) = Df(x^*) (y - x) + o(\norm{y - x^*}) \geq 0
\end{align*}
and for $y \in V$.  From the fact that $x_n \to x$ we can find an $N \in \naturals$ such
that $x_n \in V$ for all $n \geq N$.  Since $(x_n - x)/t_n \to v$ we
know that $\norm{x_n -x^*}$ is $O(t_n)$ and therefore $o(\norm{x_n -
  x})=o(t_n)$ (Proposition \ref{LandauNotationIdentities}) and since $x_n - x -t_nv=o(t_n)$ we have
\begin{align*}
t_n Df(x^*) v + o(t_n) &= f(x_n) - f(x^*) \leq 0
\end{align*}
which implies that $Df(x^*) v \geq 0$ (divide by $t_n>0$ and let $n
\to \infty$).
\end{proof}

TODO: Define the normal cone(s)...
\begin{defn}Let $X$ be a Hilbert space, $F \subset X$ be closed and let
  $x \in F$ then we say that $v \in X$ is a \emph{regular normal
    vector to $F$ at $x$} if $\langle v, w \rangle \leq 0$ for all $w
  \in T_F(x)$.  The set of all regular normal vectors to $F$ at $x$ is
 called the \emph{regular normal cone} and is denoted
 $\widehat{N}_F(x)$.  We say that $v \in X$ is a \emph{limiting normal vector to $F$
    at $x$} or simply a \emph{normal vector to $F$
    at $x$} if there are sequences $x_n$, $v_n$ with $v_n$ a regular
  normal vector to $F$ at $x_n$ and $\lim_{n \to \infty} x_n = x$ and $\lim_{n \to \infty}
  v_n =v$.
\end{defn}

Facts:

The proximal normal cone is a convex cone but may not be closed
The limiting normal cone is a closed cone but may not be convex.
The limiting normal cone may be defined as the limit of proximal
normal vectors as well as by using regular normal vectors.

\begin{defn}Let $X$ be a Hilbert space, $F \subset X$ be closed and let
  $x \in F$ we say that $F$ is \emph{Clarke regular at $x$} if $F$ is locally closed at $x$ and
$\widehat{N}_F(x) = N_F(x)$.
\end{defn}

\begin{prop}\label{RegularNormalLittleO}$\widehat{N}_F(x)$ is a closed convex cone.  Moreover, $v$ is a regular normal to $F$ at $x$ if and only if
  $\langle v, y-x \rangle \leq o(y -x)$ for $y \in F$ and $x \neq y$.  That is to say
  for every sequence $x_n \in F$ with $x_n \neq x$ and $\lim_{n \to \infty} x_n = x$, we have 
\begin{align*}
\limsup_{n \to \infty} \frac{\langle v, x_n-x \rangle}{\norm{x_n - x}}
&\leq 0
\end{align*}
\end{prop}
\begin{proof}
The fact that $\widehat{N}_F(x)$ is a closed convex cone follows from
the fact that is is an intersection of closed halfspaces.

Suppose that there exists an sequence $x_n \in F$ with $x_n \neq x$,
$x_n \to x$ and 
\begin{align*}
\limsup_{n \to \infty} \frac{\langle v, x_n-x \rangle}{\norm{x_n - x}}
&> 0
\end{align*}
By passing to a subsequence we may assume that the $\limsup$ may be
replaced by a limit.  Let $w_n = \frac{x_n - x}{\norm{x_n - x}}$ and
  by compactness we may pass to a further
subsequence and assume that $w_n$ converges
  to a unit vector $w$ and $\langle v,w \rangle = \lim_{n \to \infty} \langle v,w_n \rangle > 0$. 
On the other hand, $w$ is seen to be a tangent vector to $F$ at $x$ because it is
the limit $\frac{x_n - x}{\norm{x_n - x}}$ and $\norm{x_n -x} \to 0$.

Now suppose that $\langle v, y-x \rangle \leq o(y -x)$ and let $w \in T_F(x)$.  Pick a defining
sequence $x_n \to x$ with $x_n \in F$ and $t_n \downarrow 0$ such that $\lim_{n \to \infty} \frac{x_n - x}{t_n} = w$.
\begin{align*}
\langle v,w \rangle &= \lim_{n \to \infty} \langle v, \frac{x_n - x}{t_n} \rangle \\
&\leq \limsup_{n \to \infty} \langle v, \frac{x_n - x}{\norm{x_n- x}} \rangle \lim_{n \to \infty} \frac{\norm{x_n - x}}{t_n} \\
&= \norm{w} \limsup_{n \to \infty} \langle v, \frac{x_n - x}{\norm{x_n- x}} \rangle \leq 0\\
\end{align*}
\end{proof}

\begin{thm}$v$ is a regular normal to $F$ at $x$ if and only if there exists a function $f$ differentiable at $x$ such that 
$f$ has a local minimum on $F$ at $x$ and $\nabla f (x) = v$.  In fact, we can choose $f$ be differentiable on all of $\reals^n$ 
with a global minimum at $x$.
\end{thm}
\begin{proof}
TODO:  
Let $v \in \widehat{N}_F(x)$ be given.  The first step to building $f$ is to define
\begin{align*}
\theta_0(r) = \sup \lbrace \langle v, y - x \rangle \mid y \in F \text{ and } \norm{y-x} \leq r \rbrace
\end{align*}
It is simple to see that $\theta_0$ is a non-decreasing function of $r$,  $\theta_0(0) = 0$ and
$\theta_0(r) \leq \norm{v} r$ so $\lim_{r \downarrow 0} \theta_0(r) = 0$. 
Moreover from Proposition \ref{RegularNormalLittleO} we get for any $\epsilon > 0$ there exists a $\delta > 0$ such that
$\langle v, y-x \rangle \leq \epsilon \norm{y-x}$ for $y \in F$ and $\norm{y-x} \leq \delta$.    Thus for $0 \leq r \leq \delta$, 
\begin{align*}
\theta_0(r)&= \sup \lbrace \langle v, y - x \rangle \mid y \in F \text{ and } \norm{y-x} \leq r \rbrace \\
&\leq \sup \lbrace \epsilon \norm{y-x} \mid y \in F \text{ and } \norm{y-x} \leq r \rbrace \\
&\leq \epsilon r \\
\end{align*}
and thus $\theta_0(r) = o(r)$.
Now let 
\begin{align*}
h_0(y) &= \langle v, y-x\rangle - \theta_0(\norm{y-x})
\end{align*}
and note since $\theta_0(r) = o(r)$,
\begin{align*}
\lim_{w \to 0} \frac{h_0(x+w) - h_0(x) - \langle v,w \rangle}{\norm{w}} =\lim_{w \to 0} \frac{-\theta_0(\norm{w})}{\norm{w}} =0\\
\end{align*}
which shows $h_0$ is differentiable at $x$ and moreover $\nabla h_0(x) = v$.
\end{proof}

\begin{defn}Let $X$ be a Hilbert space, $F \subset X$ be closed and let
  $x \in F$ then we say that $v \in X$ is a \emph{proximal normal
    vector to $F$ at $x$} if there exists an $M \geq 0$ such that
  $\langle v, y - x \rangle \leq M \norm{y - x}^2$ for all $y \in F$.
\end{defn}

Here is a useful way to think about the definition of a proximal normal vector.  First note that the set of proximal normal vectors is a cone so it suffices to think about unit proximal normal vectors.  By translating $F$ by $-x$ we can also assume that $x=0$.  Now $v \in X$ and $y \in F$ and let $\theta$ be the angle between $v$ and $y$ then the 
definition says that we can find a universal $M > 0$ such that
\begin{align*}
\langle v, y \rangle = \norm{y} \cos \theta \leq M \norm{y}^2
\end{align*}
which is to say
\begin{align*}
\cos \theta \leq M \norm{y}
\end{align*}
So intuitively if $y$ is infinitesimally close to $0$ in $F$ then $\pi/2 \leq \theta \leq 3\pi/2$.  

Here is another interpretation of a proximal normal vector.  A vector $v$ is a non-zero proximal vector to $F$ at $x$ if and only if there exists a $z \in X \setminus F$ such that $z -x$ points in the same direction as $v$ and $x$ is the closest point in $F$ to $z$.  
\begin{prop}\label{ProximalNormalVectorRealizedByBalls}Let $F$ be a closed set, $x \in F$ then $v$ is a proximal normal vector to $F$ at $x$ if and only if there exists $z \in X$ and $r > 0$ such that $v = r (z - x)$ and 
$\norm{z - x} = \min_{y \in F} \norm{z - y}$.
\end{prop}
\begin{proof}
$v$ is a proximal normal to $F$ at $x$ if and only if there exists $M > 0$ such that $\langle v, y - x \rangle \leq M \norm{y - x}^2$ for all $y \in F$ or equivalently $0 \leq \frac{1}{M} \langle v, x - y \rangle + \norm{y - x}^2$ for all $y \in F$.  Adding $\norm{v}^2/4M^2$ to both sides this is equivalent to 
\begin{align*}
(\norm{v}/2M)^2 &\leq (\norm{v}/2M)^2 + \frac{1}{M} \langle v, x - y \rangle + \norm{y - x}^2 = \norm{ x + v/2M - y}^2
\end{align*}
for all $y \in F$. Now define $z = x + v/2M$ and $r = 2M$.
\end{proof}
It is worth noting that it may not be possible to find $z$ in the direction of $v$ such that $x$ is the \emph{only} point in $F$ with $\norm{z-x} = \min_{y \in F} \norm{z-y}$ (is this true; can we get an example if so????)

TODO: Show that the any limiting normal vector is a limit of proximal normals.

\subsection{KKT}

 For computational purposes (and in particular for numerical
optimization problems) we are given a constraint set in some concrete
form rather than the abstract formulation we've used.  In practice it
is useful to formulate a constraint set using a combination of
equalities and inequalities.  For the moment we specialize to the case
of finite dimensions.  Let $X$ be a finite dimensional Banach space
(i.e. $\reals^d$) and suppose we are given finite sets $\mathcal{E}$
(the \emph{equality constraints})
and $\mathcal{I}$ (the \emph{inequality constraints}) and for each $i
\in \mathcal{E} \cup \mathcal{I}$ we have a $C^1$ 
function $c_i : X \to \reals$.  Let $f : X \to \reals$ be a $C^1$
function and we consider the constrained minimization problem for $f$
with constraint set 
\begin{align*}
F &= \lbrace x \in X \mid c_i(x) = 0 \text{ for all $i \in
    \mathcal{E}$ and } c_i(x) \geq 0 \text{ for all $i \in
    \mathcal{I}$} \rbrace
\end{align*}
It is clear from the continuity of the $c_i(x)$ that $F$ is closed and
therefore we have the first order necessary condition of Theorem
\ref{LocalConstrainedMinimizerFirstDerivative} for local minimizers of
$f$ restricted to $F$.  What we seek are
conditions in terms of $f$ and the $c_i$ that are implied by the
conditions in Theorem \ref{LocalConstrainedMinimizerFirstDerivative};
for that we need to understand how $T_F(x)$ might be expressed in
terms of $f$ and the $c_i$.  To that end, we first have the following
definitions.
\begin{defn}Given a Banach space $X$, disjoint sets $\mathcal{E}$ and
  $\mathcal{I}$, functions $c_i : X \to \reals$ for each $i \in
  \mathcal{E} \cup \mathcal{I}$ and the set
\begin{align*}
F &= \lbrace x \in X \mid c_i(x) = 0 \text{ for all $i \in
    \mathcal{E}$ and } c_i(x) \geq 0 \text{ for all $i \in
    \mathcal{I}$} \rbrace
\end{align*}
we say that a constraint $c_i$ is \emph{active} at $x \in F$ if either
$i \in \mathcal{E}$ or $i \in \mathcal{I}$ and $c_i(x) = 0$.  For each
$x \in F$ we let the \emph{active constraint set} be 
\begin{align*}
\mathcal{A}(x) &= \lbrace i \in \mathcal{E} \cup \mathcal{I} \mid i
                 \text{ is active at $x$} \rbrace
\end{align*}
Assume that the $c_i$ are continuously differentiable, then the set of \emph{linearized feasible directions at $x$} is defined to
be 
\begin{equation*}
\mathcal{F}(x) =
\left \{ v \in X \mid
\begin{aligned}
Dc_i(x) v = 0 & \text{ for all $i \in \mathcal{E}$} \\
Dc_i(x) v \geq 0 & \text{ for all $i \in \mathcal{A}(x) \cap \mathcal{I}$} \\
\end{aligned}
\right \}
\end{equation*}
\end{defn}

Note that it is trivial to see that $\mathcal{F}(x)$ is a cone.  The
first thing is to note that every tangent vector is a linearized
feasible direction.

\begin{prop}$T_F(x) \subset \mathcal{F}(x)$.
\end{prop}
\begin{proof}
Let $v \in T_F(x)$ and pick a feasible sequence $x_n \to x$ and
sequence of positive numbers $t_n
\to 0$ such that $x - x_n = t_nv + o(t_n)$.  Applying Taylor's Theorem
we can conclude that 
\begin{align*}
c_i(x_n) &= c_i(x) + Dc_i(x) (x_n - x) + o(\norm{x_n-x}) \\
&= c_i(x) + t_n Dc_i(x) v + o(t_n) 
\end{align*}
so if $i \in \mathcal{A}(x)$ we have $c_i(x_n) = t_n Dc_i(x) v +
o(t_n)$. Dividing by $t_n$ and taking the limit as $n \to
\infty$ we get $Dc_i(x) v = \lim_{n \to \infty}
\frac{c_i(x_n)}{t_n}$.  Thus it follows that  $i \in \mathcal{E}$
implies $Df(x) v = 0$ and $i \in \mathcal{A}(x) \cap \mathcal{I}$
implies $Df(x) v \geq 0$.
\end{proof}

In general it is not true that $T_F(x) = \mathcal{F}(x)$ yet the
result that we want to demonstrate requires that this equality holds.
A set of conditions that we place on the $c_i$ that guarantees such an
equality is called a \emph{constraint qualification}; more generally a
constraint qualification may a bit weaker than that and simply imply
that $T_F(x)$ and $\mathcal{F}(x)$ aren't too different.  There are a
variety of choices of constraint qualifications we state a
conceptually straightforward and useful one.
\begin{defn}A set of constraints $c_i$ satisfies the \emph{linearly
    independent constraint qualification (LICQ)} at $x$ if the set of
  derivatives $\lbrace Dc_i(x) \rbrace$ for $i \in \mathcal{A}(x)$ is
  linearly independent in $X^*$.
\end{defn}
The LICQ is a sufficient criterion for the equality of the tangent
cone and the linearized feasible set.

\begin{examp}Consider the set $F \subset \reals^2$ defined by the
  constraints $c_1(x,y) = 1 - x^2 - (y-1)^2 \geq 0$ and $c_2(x,y) =
  -y \geq 0$.

TODO: Show that $T_F(x)$ is a strict subset of $\mathcal{F}(x)$.
\end{examp}

\begin{prop}\label{LICQTangentRegularity}Let $F$ be defined by a set of constraints $c_i(x)$ which
  satisfy the LICQ at $x$ then $T_F(x) = \mathcal{F}(x)$.
\end{prop}
\begin{proof}
Let $v \in \mathcal{F}(x)$, we need to show that $v$ is a tangent
vector producing a sequence $x_n \in F$ and $t_n > 0$ such that $v = x_n + o(t_k)$. 
By assumption the set of derivatives $Dc_i(x)$ for $i \in
\mathcal{A}(x)$ is linearly independent hence is a basis for the linear 
span $V = \lbrace Dc_i(x) \rbrace_{i \in \mathcal{A}(x)}$.  Let $m$ be the cardinality of $\mathcal{A}(x)$ and $c : X \to \reals^m$ 
be defined by $c(y) = (c_{i_1}(y), \dotsc, c_{i_m}(y))$ where $\lbrace i_1, \dotsc, i_m \rbrace = \mathcal{A}(x)$.
Take the orthogonal complement $W$ of $V$ 
in $X^*$ and pick a basis $w_j$ for $W$ and let $w : X \to \reals^{d-m}$ be defined by $w(y) = (w_1(y), \dotsc, w_{d-m}(y))$.  Now define $R : X \times \reals \to \reals^d$ by
\begin{align*}
R(y,t) &= \begin{bmatrix}
c(y) - t Dc(x) v \\
w(y - x - tv)
\end{bmatrix}
\end{align*}
and note that $R(x,0) = 0$.  Moreover 
\begin{align*}
D_1R(x,0)(u, 0) &= \begin{bmatrix}
Dc(x) u \\
w(u)
\end{bmatrix}
\end{align*}
which is invertible by construction of $w$.  Now we can apply the Implicit Function Theorem \ref{ImplicitFunctionTheorem} to 
conclude that there exists an $\epsilon > 0$ and a differentiable function $f : (-\epsilon, \epsilon) \to X$ such that $f(0) = x$, $R(f(t), t) = 0$ 
for all $-\epsilon < t < \epsilon$ and moreover $f(t)$ is the unique solution to the equation $R(x,t) = 0$.  In addition note that
since we have assumed that $v \in \mathcal{F}(x)$ we have from $R(f(t), t) = 0$,
\begin{align*}
c_i(f(t)) &= t Dc_i(x) v = 0 \text{ for $i \in \mathcal{E}$} \\
c_i(f(t)) &= t Dc_i(x) v >= 0 \text{ for $t > 0$ and $i \in \mathcal{A}(x)\cap \mathcal{I}$} \\
\end{align*}
and moreover by continuity we have $c_i(f(t)) > 0$ for $i \in \mathcal{I} \setminus \mathcal{A}(x)$ (here we may need to shrink $\epsilon$ for this to be true).  Thus we have $f(t) \in F$.  

Now pick any sequence $0 < t_n < \epsilon$ with $\lim_{n \to \infty} t_n = 0$ and define $x_n = f(t_n)$; by continuity of $f$ and and the 
fact that $f(0) = x$ we have $x_n \to x$.  If we Taylor expand $R(y,t)$ around $(x,0)$ we get
\begin{align*}
0 &= f(x_n, t_n) = \begin{bmatrix}
Dc(x) (x_n - x - t_n v) \\
w(x_n - x - t_n v)
\end{bmatrix}
+ o(\norm{(x_n,t_n) - (x,0)})
\end{align*}
Since $x_n = f(t_n)$ and $f$ is differentiable it follows that $x_n - x = O(t_n)$ and therefore $o(\norm{(x_n,t_n) - (x,0)}) = o(t_n)$.  Thus by
considering the first component of the vector we have $Dc(x)  (x_n - x - t_n v)  = o(t_n)$ and since 
$Dc(x)$ is invertible we get that $x_n - x -t_n v = o(t_n)$ which shows that $v \in T_F(x)$.
\end{proof}

\begin{lem}[Farkas' Lemma]\label{FarkasLemma}Let $A$ be a $d \times m$ real matrix, $B$ be a $d \times p$ real matrix and let 
\begin{align*}
K &= \lbrace A y + B w \mid y \in \reals^m, y \geq 0 \text{ and } w \in \reals^p \rbrace
\end{align*}
then for any $x \in \reals^d$ either $x \in K$ or there exists $z \in \reals^d$ such that 
\begin{align*}
\langle z, x \rangle < 0 & A^T z \geq 0 & B^T z = 0
\end{align*}
but not both.
\end{lem}
\begin{proof}
The proof begins by showing that the alternatives are mutually exclusive.  Suppose that $x \in K$ and write 
$x = A y + B w$ for appropriate $y$ and $w$.  If $z$ and in the Lemma exists as well then we can calculate
\begin{align*}
\langle z, x \rangle &= y^T A^T z + w^T B^T z \geq 0
\end{align*}
which contradicts $\langle z,x \rangle < 0$.

Now we suppose that $x \notin K$ and we show how to construct $z$.  

\begin{clm} $K$ is closed.
\end{clm}
First note that 
\begin{align*}
K &= \lbrace A y + B w_+ - B w_-\mid y \in \reals^m, y \geq 0 \text{ and } w_\pm \in \reals^p \text{ and } w_\pm \geq 0 \rbrace
\end{align*}
so we can reduce ourselves to the case in which $B = 0$.  For any subset $I \subset \lbrace 1, \dotsc, m \rbrace$ we let $A_I$ be the matrix
with columns of index $I$.  For any $x \in K$ we can find a representation $x = A_I y_I$ in which $I$ has the smallest cardinality among such
representations.  Note that $y_I > 0$ and the columns of $A_I$ must be linearly independent (if not select $w_I$ such that $A_I w_I = 0$ and 
then pick a constant $c$ to be the coefficient of $y_I$ with the smallest magnitude; it follows that $x = A_I (y_I + c w_I)$ and $y_I + c w_I$ has coefficients equal to zero).
Now select a convergent sequence $x_k$ in with $x_k \in K$ and pick minimal representations $x_k = A_{I_k} y^k_{I_k}$.  By passing to a subsequence 
we may assume that $I_k=I$ is constant.  Since $A_I$ has full column rank  $A^T_I A_I$ is invertible and $y^k_I$ is uniquely determined to be
$y^k_I = (A_I^T A_I)^{-1} A^T_I x_k$.  Let $x = \lim_{k \to \infty} x_k$ and let 
\begin{align*}
y_I &= (A_I^T A_I)^{-1} A^T_I x = \lim_{k \to \infty} (A_I^T A_I)^{-1} A^T_I x_k =  \lim_{k \to \infty} y^k_I 
\end{align*}
It follows from $y^k_I > 0$ that $y_I \geq 0$ and by continuity of $A_I$, $A_I y_I = \lim_{k \to \infty} A_I y^k_I =  \lim_{k \to \infty} x_k = x$.  Thus $x \in K$.

Now given that $K$ is closed we can pick an $s_0 \in K$ such that $\norm{s_0 - x} = d(K, x) = \inf_{s \in K} \norm{s - x}> 0$.  
Since $K$ is a cone we know that $\lambda s_0 \in K$ for all $\lambda > 0$ and by choice of $s_0$ we know that 
$g(\lambda) = \norm{\lambda s_0 - x}^2$ has a minimum at $\lambda = 1$.  Now by Theorem \ref{VanishingFirstDerivativeAtLocalMinimum}
we know that $0 = \frac{dg}{d\lambda} \norm{\lambda s_0 - x}^2(1) = 2 \langle s_0,  s_0 \range  - 2 \langle s_0, x \rangle$ and therefore
we conclude that $\langle s_0, s_0 - x \rangle = 0$.  Let $s \in K$ be arbitrary and since $K$ is convex we know that $s_0 + \theta (s - s_0) \in K$ for
all $0 \leq \theta \leq 1$ and by the defining property of $s_0$ and algebra we get
\begin{align*}
0 &\leq \norm{s_0 + \theta (s - s_0) -x }^2 - \norm{s_0 -x }^2  \\
&=\langle s_0 - x + \theta (s - s_0), s_0 - x + \theta (s - s_0) \rangle - \langle s_0 -x, s_0 -x \rangle \\
&=2 \theta \langle s_0 - x, s - s_0 \rangle + \theta^2 \norm{s - s_0}^2
\end{align*}
Dividing by $\theta$,  letting $\theta \to 0$ and using $\langle s_0, s_0 - x \rangle = 0$ we get 
\begin{align*}
\langle s_0 - x, s \rangle &= \langle s_0 - x, s - s_0 \rangle = \lim_{\theta \to 0^+}  \langle s_0 - x, s - s_0 \rangle + \frac{1}{2}\theta \norm{s - s_0}^2 \geq 0
\end{align*}

Now define $z = s_0 - x$.  As observed $z \neq 0$ and therefore
\begin{align*}
\langle z, x \rangle &= \langle z, s_0 - z \rangle = \langle z, s_0 \rangle - \norm{z}^2 = \langle s_0 - x, s_0 \rangle - \norm{z}^2 = - \norm{z}^2 < 0
\end{align*}

Writing an arbitrary $s \in K$ as $s = A y + B w$ with $y \geq 0$ we know that $\langle z, A y + B w \rangle$.  In particular if we let $y = 0$ and
$w = -B^T z$ then we get 
\begin{align*}
-\norm{B^T z}^2 &= -\langle B^T z, B^T z \rangle= -\langle z, B B^T z \rangle \geq 0
\end{align*}
hence $B^T z = 0$.  Choosing $w=0$ and $y = e_j$ for $j=1, \dotsc, m$ we get
\begin{align*}
(A^Tz)_j &= \langle A^T z, e_j \rangle = \langle z , A e_j \rangle \geq 0
\end{align*}
which shows that $A^T z \geq 0$.
\end{proof}

Geometrically, Farkas' Lemma says that given the cone $K$ and a point $x$ either the point $x \in K$ or there exists a 
hyperplane (defined by as the set $\lbrace w \mid \langle z,w \rangle = 0 \rbrace$) that separates $x$ and $K$.
The reason why such a separating hyperplane condition is relevant to our discussion is made clear in the proof of the following
extremely important result. 

\begin{thm}[Karush-Kuhn-Tucker  Conditions]\label{KKTFiniteDimensions}If $x^*$ is a local minimizer of $f$ constrained to $F$ then if we define
the Lagrangian 
\begin{align*}
\mathcal{L}(x, \lambda) = f(x) - \lambda \sum_{i \in \mathcal{E} \cup \mathcal{I}} \lambda_i c_i(x)
\end{align*}
there exists $\lambda^*$ such that
\begin{itemize}
\item[(i)]$\nabla_x \mathcal{L}(x^*, \lambda^*) = 0$
\item[(ii)]$\lambda_i^* \geq 0$ for all $i \in \mathcal{I}$
\item[(iii)]$\lambda_i^* c_i(x^*) = 0$ for all $i \in \mathcal{E} \cup \mathcal{I}$
\end{itemize}
\end{thm}
\begin{proof}
By Theorem \ref{LocalConstrainedMinimizerFirstDerivative} we know that
$\langle \nabla_x f(x^*), v \rangle \geq 0$  for all $v \in T_F(x^*)$.  By the LICQ we know from
Proposition \ref{LICQTangentRegularity} that $T_F(x^*)=\mathcal{F}(x^*)$ and thus it follows that
$\langle \nabla_x f(x^*), v \rangle \geq 0$ for all $v \in \mathcal{F}(x^*)$.  

On the other hand, let $A$ be the matrix with columns $\nabla c_i(x^*)$ for $i \in \mathcal{A}(x^*) \cap \mathcal{I}$
and $B$ be the matrix with columns $\nabla c_i(x^*)$ for $i \in \mathcal{E}$.
By Farkas' Lemma \ref{FarkasLemma} applied to the cone $\lbrace A y + B w \mid y \geq 0 \rbrace$ either
there exist $\lambda^*_i$ for $i \in \mathcal{A}(x^*)$ such that $\nabla_xf(x^*) = \sum_{i \in \mathcal{A}(x^*)} \lambda^*_i \nabla c_i(x^*)$
and $\lambda^*_i \geq 0$ for $i \in \mathcal{A}(x^*) \cap \mathcal{I}$ or there exists $z$ such that 
$\langle z, \nabla_xf(x^*) \rangle < 0$, $A^T z \geq 0$ and $B^T z = 0$.  Note however that the latter two
conditions on $z$ are precisely the conditions that determine $z \in T_F(x^*)$.  Thus we can conclude that 
$\nabla_x f(x^*) = \sum_{i \in \mathcal{A}(x^*)} \lambda^*_i \nabla c_i(x^*)$ for $\lambda^*_i$ as above.  For $i \in \mathcal{I} \setminus \mathcal{A}(x^*)$
we define $\lambda^*_i = 0$.

Having defined $\lambda^*$ we proceed to show that it satisfies the properties required by the theorem.  To see (i) we simply
take the gradient $\nabla_x \mathcal{L} = \nabla_x f - \lambda \sum_{i \in \mathcal{E} \cup \mathcal{I}} \lambda_i \nabla_x c_i$ and use the definition
of $\lambda^*$.  Property (ii) follows immediately since we had $\lambda^*_i \geq 0$ for $i \in \mathcal{A}(x^*) \cap \mathcal{I}$ by
Farkas' Lemma and $\lambda^*_i = 0$ for $i \in \mathcal{I} \setminus \mathcal{A}(x^*)$ by definition.  To see (iii) we have
$\lambda^*_i c_i(x^*) = 0$ for $i \in \mathcal{I} \setminus \mathcal{A}(x^*)$ since $\lambda^*_i = 0$ and 
$\lambda^*_i c_i(x^*) = 0$ for $i \in \mathcal{A}(x^*)$ since for such indices $c_i(x^*) = 0$ by definition of $\mathcal{A}(x^*)$.  
\end{proof}

\begin{lem}\label{FarkasLemmaTaoVersion}For $i=1, \dotsc, n$ let $P_i : \reals^d \to \reals$ be an affine-linear function 
(i.e. $P(x) = a_0 + \sum_{i=1}^d a_j x_j$ for $a_0, \dotsc, a_d \in \reals$).  Then either there exists $x^* \in \reals^d$ such that
$P_i(x^*) \geq 0$ for all $i=1, \dotsc, n$ or there exist non-negative real numbers $q_1 \geq 0, \dotsc, q_n \geq 0$ such that $\sum_{i=1}^n q_i P_i = -1$.
\end{lem}
\begin{proof}
First, note that the result holds for a given set $P_1, \dotsc, P_n$ if and only if it holds for the set $q_1 P_1, \dotsc, q_n P_n$ with $q_1 > 0, \dotsc, q_n > 0$.  

The proof is by induction on $d$.  For the case $d=1$ by multiplying each $P_i$ by an appropriate positive constant we may assume that each
$P_i$ is either of the form $x - a_i$, $b_i - x$ or $c_i$ with $a_i,b_i,c_i \in \reals$.  If any of the $c_i < 0$ then define $q_i = -c_i^{-1}$ and the rest of the $q$ to be zero 
and we are done.  Thus we may assume that all $c_i \geq 0$.  In this case the $c_i$ provide no obstruction to finding an $x^*$ but neither to they present an obstruction
to finding $q_i$ such that $\sum_{i=1}^d q_i P_i =-1$ (just set the appropriate $q_i = 0$).  Thus we are reduced to the case in which each $P_i$ is either of the
form $x - a_i$ or $b_i - x$.  Suppose there is some $b_i < a_j$, then let $q_i=q_j=(a_j-b_i)^{-1}$ and set the rest of the $q$ to zero.   It follows that
\begin{align*}
\sum_{i=1}^n q_i P_i(x) &= (a_j-b_i)^{-1} (P_i(x) + P_j(x)) =  (a_j-b_i)^{-1} (b_i - x + x - a_j)  = -1
\end{align*}
On the other hand, any $x^*$ that lies between the maximum of the $a_i$ and the minimum of the $b_i$ is a solution
to $P_i(x^*) \geq 0$ for all $i=1, \dotsc, n$.

For the induction step for $d \geq 2$, writing $x=(y,t) \in \reals^d$ and arguing as in the case $d=1$ we can assume that
each $P_i(y,t)$ is of the form $t - a_i(y)$, $b_i(y) -t$ or $c_i(y)$.  Assuming $y^*$ is fixed, by the analysis in the case $d=1$ we know
that a $t^*$ such that $P_i(y^*, t^*) \geq 0$ exists for each $i=1, \dotsc, n$ if and only if $c_i(y^*) \geq 0$ and the maximum of the 
$a_i(y^*)$ is less than or equal to the minimum of the $b_i(y^*)$.  So if these conditions hold for any $y^* \in \reals^{d-1}$ we construct
$x^* = (y^*,t^*)$ and we are done.  Suppose on the other hand that for every $y \in \reals^{d-1}$ either $c_i(y) < 0$ or some $b_j(y) - a_k(y) < 0$, the applying
the induction hypothesis to the family $c_i$, $b_j - a_k$ we conclude that there exist non-negative constants $r_i$ and $r_{j,k}$ such that
\begin{align*}
-1 &= \sum_i r_i c_i(y) + \sum_{j,k} r_{j,k} (b_j(y) - a_k(y)) = \sum_i r_i P_i(x) + \sum_{j,k} r_{j,k} (P_j(x) + P_k(x)) = \sum_i r_i P_i(x) + \sum_{j} \sum_{k} r_{j,k} P_j(x) + \sum_{k} \sum_{j} r_{j,k} P_k(x)
\end{align*}
and if we define $q_i = r_i$, $q_j = \sum_{k} q_{j,k}$ and $q_k = \sum_{j} q_{j,k}$ we are done.
\end{proof}

TODO: At this point Tao shows how to prove a separation lemma for convex polytopes by casting the problem of finding a separating hyperplane as solving a set of affine linear inequalities.  
I don't yet understand this problem formulation; seems like it might be clearer using the vertex description of the polytope but need the unbounded case so I think I need to see the translation
in the hyperplane description.

Absent constraint qualifications the KKT conditions are not implied at a local minimum as the following example shows
\begin{examp}Consider the problem of minimizing $-x$ subject to $(x-1)^3 + y \geq 0$, $x \geq 0$ and $y \geq 0$.
TODO:
\end{examp}


TODO:  How does the above theory about KKT need to be modified to
handle the following example (namely the fact that the Shannon entropy isn't defined on an open
set of the constraint set).  Maybe there is a $C^1$ extension of Shannon to $\reals^d$?  This example motivates one to consider exponential families of probability distributions.
\begin{examp}\label{MaximumEntropyPrinciple}Consider a finite set $S=\lbrace 1, \dotsc, m \rbrace$ and a finite family
of functions $\phi_j : S \to \reals$ for $j=1, \dotsc, n$.  Suppose we are given real number $\hat{\mu}_1, \dotsc, \hat{\mu}_n$ and 
consider the optimization problem of finding the probability distribution $\pi = \lbrace p_1, \dotsc, p_m \rbrace$ on $S$ which satisfies
$\sexpectation{\phi_j}{\pi} = \hat{\mu}_j$ and minimizes the Shannon entropy $H(\pi) = -\sum_{j=1}^m p_j \log p_j$ (where as usual we assume $0 \cdot \log 0 = 0$.
The main point here is that there is a constant $c$ such that $p_j = c e^{\sum_{i=1}^n phi_i(j) \lambda_i^*}$ where $\lambda^*_j$ are Lagrange multipliers for the
constraints $\sexpectation{\phi_j}{\pi} = \hat{\mu}_j$  in the optimization problem (presumably the $\hat{\mu}_j$ figure into the normalization constant $c$!).
\end{examp}

\section{Algorithms for Unconstrained Optimization}

We have developed criteria for detecting minimizers (mostly local)
however we have not yet addressed the issue of how we might find one.
There are two basic paradigms to consider: line search and trust
region methods.  We first consider line search.

For motivation we give an interpretation of the Frechet derivative of
a real valued function on a Hilbert space $X$.  Since $Df(x)$ is a
bounded linear functional on $X$, we know by Reisz
representation that there is a unique element of $X$ representing the
functional.

\begin{defn}Let $X$ be a Hilbert space, $U \subset X$ be open and let
  $f : U \to \reals$ be differentiable at $x \in U$.  The
  \emph{gradient of $f$ at $x$} is the unique element $\nabla f(x)$ of
  $X$ such that $\langle \nabla f(x), v \rangle = Df(x) v$ for all $v
  \in X$.
\end{defn}

We now proceed to interpret the vector $-\nabla f(x)$ as the
direction of steepest decrease of the function $f$.  Suppose
that we are at a point $x$ for which $\nabla f(x) \neq 0$.  To see
this, let $v \in X$ be an arbitrary unit vector in $X$ and consider the function of a single
real variable $g_v(t) = f(x + tv)$.  The question we ask is what is the
direction $v$ along which $f$ is decreasing the fastest at $x$.  By the Chain Rule, the definition of the
gradient and Taylor's Theorem we can write
\begin{align*}
f(x + tv) = f(x) + t \langle \nabla f(x), v \rangle + o(t)
\end{align*}
which implies that $g_v^\prime(0) = \langle \nabla f(x), v \rangle$.  So
what we want is to find the unit vector $v$ which minimizes the value
of $g_v^\prime(0)$.  We can write $v = \alpha \nabla f(x)/\norm{\nabla
  f(x)} + w$ where $\langle \nabla f(x), w \rangle = 0$.  Note that on
the one hand $\langle \nabla f(x), v \rangle = \alpha/\norm{\nabla
  f(x)}$ and on the other hand from $\norm{v} = 1$ we see that $-1
\leq \alpha \leq 1$.  Therefore it is clear that the minimum of
$g_v^\prime(0)$ occurs for $\alpha = -1$ which implies that $v = -
\nabla f(x) /\norm {\nabla f(x)}$. It is colloquial to say that the direction of
the gradient is the \emph{direction of steepest descent of $f$}.  Note
that the computation above shows that $g_v^\prime(0) < 0$ precisely
when $\langle \nabla f(x), v \rangle < 0$ which motivates the
following defintion
\begin{defn}Let $X$ be a Hilbert space, $U \subset X$ be open and let
  $f : U \to \reals$ be differentiable at $x \in U$ we say that $v \in
  X$ is a \emph{descent direction for $f$ at $x$} if $\langle \nabla
  f(x) , v\rangle < 0$.
\end{defn}


TODO: Discuss gradient flow in the Hilbert space and observe how the
solutions of the differential equation have limit points equal to the
stationary points of $f$.

Armed with the idea that when we are in possesion of derivatives we can
find directions along which the values of a function decreases, we
seek find an iterative algorithm for minimization.  The obvious idea
is that if at a given point $x_k$ we can find a descent direction
(e.g. the gradient $\nabla f(x_k)$ then we should move in that
direction and thereby expect that the function decreases.  There are
three problems to address about such an algorithm.  The first issue is
that the descent direction is characterized by an infinitessimal
condition and therefore there is no guarantee that a finite step in
that direction will result in a decrease in the function value.  The
second issue is that if our step sizes in the descent direction are
too small asymptotically we may never reach the minimum.  The third
issue is that if we choose a variable descent direction, the descent
direction may get increasing close to being orthogonal to the gradient
in which case function values may not decrease enough to converge
(note this is a non-issue is we choose the steepest descent direction).  We
seek conditions on the choice of step sizes and descent directions
that give us convergence to a stationary point of $f$.



